\documentclass{ltxdockit}[2011/03/25]
\usepackage{btxdockit}
\usepackage{format-biblatex-zh-cn}
\usepackage{hyperref}
\usepackage{zref-xr}
\setmonofont[Scale=MatchLowercase, Ligatures=TeX]{DejaVu Sans Mono}
\setmainfont[Ligatures=TeX]{Linux Libertine O}
\setsansfont[Ligatures=TeX]{Linux Biolinum O}
%\usepackage[american]{babel}%增加回来,因为一些表述需要用到比如low"=level等
\usepackage[strict]{csquotes}
\usepackage{tabularx}
\usepackage{longtable}
\usepackage{booktabs}
\usepackage{shortvrb}
\usepackage{pifont}
\usepackage{microtype}
\usepackage{typearea}
\usepackage{mdframed}

\areaset[current]{370pt}{700pt}
\lstset{
    basicstyle=\ttfamily,
    keepspaces=true,
    upquote=true,
    frame=single,
    breaklines=true,
    postbreak=\raisebox{0ex}[0ex][0ex]{\ensuremath{\color{red}\hookrightarrow\space}}
}
\KOMAoptions{numbers=noenddot}
\addtokomafont{title}{\sffamily}
\addtokomafont{paragraph}{\spotcolor}
\addtokomafont{section}{\spotcolor}
\addtokomafont{subsection}{\spotcolor}
\addtokomafont{subsubsection}{\spotcolor}
\addtokomafont{descriptionlabel}{\spotcolor}
\setkomafont{caption}{\bfseries\sffamily\spotcolor}
\setkomafont{captionlabel}{\bfseries\sffamily\spotcolor}
\pretocmd{\cmd}{\sloppy}{}{}
\pretocmd{\bibfield}{\sloppy}{}{}
\pretocmd{\bibtype}{\sloppy}{}{}
\makeatletter
\patchcmd{\paragraph}
  {3.25ex \@plus1ex \@minus.2ex}{-3.25ex\@plus -1ex \@minus -.2ex}{}{}
\patchcmd{\paragraph}{-1em}{1.5ex \@plus .2ex}{}{}
\makeatother

\MakeAutoQuote{«}{»}
\MakeAutoQuote*{<}{>}
\MakeShortVerb{\|}

\newcommand*{\biber}{\sty{biber}\xspace}
\newcommand*{\biblatex}{\sty{biblatex}\xspace}
\newcommand*{\biblatexml}{\sty{biblatexml}\xspace}
\newcommand*{\biblatexhome}{http://sourceforge.net/projects/biblatex/}
\newcommand*{\biblatexctan}{http://ctan.org/pkg/biblatex/}

\titlepage{%
  title={The \biblatex Package},
  subtitle={Programmable Bibliographies and Citations},
  url={\biblatexhome},
  author={Philipp Lehman \\(with Philip Kime, Audrey Boruvka and Joseph Wright)},
  email={},
  revision={3.9},%3.7
  date={21/11/2017}%16/11/2016
  }

\hypersetup{%
  pdftitle={The \biblatex Package},
  pdfsubject={Programmable Bibliographies and Citations},
  pdfauthor={Philipp Lehman, Philip Kime},
  pdfkeywords={tex, latex, bibtex, bibliography, references, citation}}

% tables

\newcolumntype{H}{>{\sffamily\bfseries\spotcolor}l}
\newcolumntype{L}{>{\raggedright\let\\=\tabularnewline}p}
\newcolumntype{R}{>{\raggedleft\let\\=\tabularnewline}p}
\newcolumntype{C}{>{\centering\let\\=\tabularnewline}p}
\newcolumntype{V}{>{\raggedright\let\\=\tabularnewline\ttfamily}p}

\newcommand*{\sorttablesetup}{%
  \tablesetup
  \ttfamily
  \def\new{\makebox[1.25em][r]{\ensuremath\rightarrow}\,}%
  \def\alt{\par\makebox[1.25em][r]{\ensuremath\hookrightarrow}\,}%
  \def\note##1{\textrm{##1}}}

\newcommand{\tickmarkyes}{\Pisymbol{psy}{183}}
\newcommand{\tickmarkno}{\textendash}
\providecommand*{\textln}[1]{#1}
\providecommand*{\lnstyle}{}

% markup and misc

\setcounter{secnumdepth}{4}
\makeatletter

\newenvironment{nameparts}
  {\trivlist\item
   \tabular{@{}ll@{}}}
  {\endtabular\endtrivlist}

\newenvironment{namedelims}
  {\trivlist\item
   \tabularx{\textwidth}{@{}c@{=}l>{\raggedright\let\\=\tabularnewline}X@{}}}
  {\endtabularx\endtrivlist}

\newenvironment{namesample}
  {\def\delim##1##2{\@delim{##1}{\normalfont\tiny\bfseries##2}}%
   \def\@delim##1##2{{%
     \setbox\@tempboxa\hbox{##1}%
     \@tempdima=\wd\@tempboxa
     \wd\@tempboxa=\z@
     \box\@tempboxa
     \begingroup\spotcolor
     \setbox\@tempboxa\hb@xt@\@tempdima{\hss##2\hss}%
     \vrule\lower1.25ex\box\@tempboxa
     \endgroup}}%
   \ttfamily\trivlist
   \setlength\itemsep{0.5\baselineskip}}
  {\endtrivlist}

\makeatother

\newrobustcmd*{\Deprecated}{%
  \textcolor{spot}{\margnotefont Deprecated}}
\newrobustcmd*{\DeprecatedMark}{%
  \leavevmode\marginpar{\Deprecated}}
\newrobustcmd*{\BiberOnly}{%
  \textcolor{spot}{\margnotefont Biber only}}
\newrobustcmd*{\BiberOnlyMark}{%
  \leavevmode\marginpar{\BiberOnly}}
\newrobustcmd*{\BibTeXOnly}{%
  \textcolor{spot}{\margnotefont BibTeX only}}
\newrobustcmd*{\BibTeXOnlyMark}{%
  \leavevmode\marginpar{\BibTeXOnly}}
\newrobustcmd*{\LF}{%
  \textcolor{spot}{\margnotefont Label field}}
\newrobustcmd*{\LFMark}{%
  \leavevmode\marginpar{\LF}}
\newrobustcmd*{\CSdelim}{%
  \textcolor{spot}{\margnotefont Context Sensitive}}
\newrobustcmd*{\CSdelimMark}{%
  \leavevmode\marginpar{\CSdelim}}

% following snippet is based on code by Michael Ummels (TeX Stack Exchange)
% <http://tex.stackexchange.com/a/13073/8305>
\makeatletter
  \newcommand\fnurl@[1]{\footnote{\url@{#1}}}
  \DeclareRobustCommand{\fnurl}{\hyper@normalise\fnurl@}

\makeatother

\hyphenation{%
  star-red
  bib-lio-gra-phy
  white-space
}


%注意:ltxdockit相关的文件包括:btxdockit.sty,ltxdockit.sty,ltxdockit.cls
%ltxdockit.def,scratcl.cls,tocbasic.sty,ltxdockit.cgf等
%需要修改文档的排版样式等,可以从这些文件入手去查。

% 使用 ltxdockit.sty 定义的标记命令与效果
% command string: \cs{foo}		\foo
% command: \cmd{foo}		\foo
% environment: \env{foo}		foo
% length: \len{foo}		\foo
% count: \cnt{foo}		foo
% primary key: \prm{foo}		<foo>
% mandatory prm: \mprm{foo}	{<foo>}
% optional prm: \oprm{foo}	[<foo>]
% option: \opt{foo}		foo
% key-value option: \kvopt{foo}{bar}	foo=bar
% file name: \file{foo}		foo
% style name: \sty{foo}		foo
% binary file: \bin{foo}	foo
% acronym: acr{Foo}		\textsc{foo}	
% key-value: \keyval 	<key>=<value>

\begin{document}

\printtitlepage

\begin{trivlist}\item\relax
{\hfill\heiti\zihao{3}Biblatex 用户手册\footnote{Doc last revised at 2017-12-15}\hfill\hspace*{1pt}}\par
{\hfill\fangsong\zihao{-4}Trans. by:~ Zhenzhen Hu\footnote{Email: hzzmail@163.com} ~and~ %
    	Wenbo Sheng \footnote{Email: wbsheng88@foxmail.com}\hfill\hspace*{2.5cm}}\par
{\hfill\kaiti\zihao{-4}{2016-02-08}\hfill\hspace*{0.5cm}}\par
\end{trivlist}
\renewcommand{\contentsname}{目录}
\renewcommand{\listtablename}{表格}
\tableofcontents
\listoftables

% Introduction.tex


\section{引言}
\label{int}

这是关于 \biblatex 包的语法文档,使用范例文档参考文档\fnurl{\biblatexctan doc/examples}。
快速开始,请浏览\secref{int:abt, bib:typ, bib:fld, bib:use, use:opt, use:xbx, use:bib, use:cit, use:use} 节。

%译文文档说明: 译文文档直接在原biblatex说明文档基础上进行修改,biblatex说明文档除了使用了一些基本宏包外主要使用类和包文件包括:
%scrartcl.cls,ltxdockit.cls,ltxdockit.cfg,ltxdockit.def,btxdockit.sty,ltxdockit.sty。需要修改一些默认设置和命令可在这些文件中查找。
%还要注意,文档没有参考文献,使用texstudio自动编译时设置驱动是bibtex或bibtex8可一遍过?


\subsection{关于 \biblatex}
\label{int:abt}

%This package provides advanced bibliographic facilities for use with \latex. 
%The package is a complete reimplementation of the bibliographic facilities provided by \latex. 
%The \biblatex\ package works with the \enquote{backend} (program) \biber, which is used to process \bibtex\ format data files and them performs all sorting, label generation (and a great deal more). 
%Formatting of the bibliography is entirely controlled by \tex\ macros. 
%Good working knowledge in \latex should be sufficient to design new bibliography and citation styles. 

%This package also supports subdivided bibliographies, multiple bibliographies within one document, and separate lists of bibliographic information such as abbreviations of various fields.
%Bibliographies may be subdivided into parts and\slash or segmented by topics. 
%Just like the bibliography styles, all citation commands may be freely defined. 

%Features such as full Unicode support for bibliography data, customisable sorting, multiple bibliographies with different sorting, customisable labels and dynamic data modification are available. 
%Please refer to \secref{int:pre:bibercompat} for information on \biber/\biblatex version compatibility. 
%The package is completely localised and can interface with the \sty{babel} and \sty{polyglossia} packages. 
%Please refer to \tabref{bib:fld:tab1} for a list of languages currently supported by this package.
\biblatex 包提供了一套与 \LaTeX\ 配合使用的高级参考文献工具。
它重新实现了 \LaTeX 提供的参考文献功能。
该包使用后端程序 \biber 来处理\BibTeX\ 格式的数据文件,并完成排序、标签生成和更多功能。
参考文献的格式化完全由 \TeX\ 宏指令控制。
具备良好的 \LaTeX 知识就足以设计新的参考文献著录样式和标注样式。

\biblatex 也支持参考文献表细分、在一个文档内包含多个参考文献表、以及域缩写等参考文献信息表。
参考文献表可以根据主题进行分块或者分段。
与参考文献著录样式类似,所有的标注引用命令也可以自由定义。

提供的功能还包括:文献数据的Unicode支持、自定义排序、不同排序方式的多参考文献表、自定义标签和动态数据修改等。
\biber/\biblatex 的版本兼容性见 \secref{int:pre:bibercompat} 节。
该包可完全实现本地化,可与 \sty{babel} 和 \sty{polyglossia} 宏包配合使用。
该包支持的语言详见表~\tabref{bib:fld:tab1}。


\subsection{许可}

Copyright \textcopyright\ 2006--2012 Philipp Lehman, 2012--2013 Philip Kime, Audrey Boruvka, Joseph Wright. Permission is granted to copy, distribute and\slash or modify this software under  the terms of the \lppl, version 1.3.\fnurl{http://www.ctan.org/tex-archive/macros/latex/base/lppl.txt}

\subsection{反馈}
\label{int:feb}

%Please use the \biblatex project page on GitHub to report bugs and submit feature requests.\fnurl{http://github.com/plk/biblatex} Before making a feature request, please ensure that you have thoroughly studied this manual. If you do not want to report a bug or request a feature but are simply in need of assistance, you might want to consider posting your question on the \texttt{comp.text.tex} newsgroup or \tex-\latex Stack Exchange.\fnurl{http://tex.stackexchange.com/questions/tagged/biblatex}

请使用 Github 的 \biblatex 项目页报告bug和提交所需功能\fnurl{http://github.com/plk/biblatex}。
在提出功能需求,请确保你已经彻底研究过本手册。
如果你不想报告bug或者请求新功能,而只是需要帮助,
可以考虑在 \texttt{comp.text.tex} 新闻组或者 \TeX-\LaTeX\ Stack Exchange 提交问题。\fnurl{http://tex.stackexchange.com/questions/tagged/biblatex}

\subsection{致谢}

The language modules of this package are made possible thanks to the following contributors:
Augusto Ritter Stoffel, Mateus Araújo (Brazilian);
Sebastià Vila-Marta (Catalan);
Ivo Pletikosić (Croatian);
Michal Hoftich (Czech);
Jonas Nyrup (Danish);
Johannes Wilm (Danish\slash Norwegian);
Alexander van Loon, Pieter Belmans, Hendrik Maryns (Dutch);
Hannu Väisänen, Janne Kujanpää (Finnish);
Denis Bitouzé (French);
Apostolos Syropoulos, Prokopis (Greek);
Baldur Kristinsson (Icelandic);
Enrico Gregorio, Andrea Marchitelli (Italian);
Håkon Malmedal (Norwegian);
Anastasia Kandulina, Yuriy Chernyshov (Polish);
José Carlos Santos (Portuguese);
Oleg Domanov (Russian);
Tea Tušar and Bogdan Filipič (Slovene);
Ignacio Fernández Galván (Spanish);
Per Starbäck, Carl-Gustav Werner, Filip Åsblom (Swedish).

\subsection{前提与必备}
\label{int:pre}

本节介绍所需资源和兼容性问题。

\subsubsection{必须资源}
\label{int:pre:req}

%The resources listed in this section are strictly required for \biblatex to function. The package will not work if they are not available.

如下资源是必须的,否则 \biblatex 无法正常工作。

\begin{marglist}

\item[\eTeX]
%The \biblatex package requires \etex. \tex distributions have been shipping \etex binaries for quite some time, the popular distributions use them by default these days. The \biblatex package checks if it is running under \etex. Simply try compiling your documents as you usually do, the chances are that it just works. If you get an error message, try compiling the document with \bin{elatex} instead of \bin{latex} or \bin{pdfelatex} instead of \bin{pdflatex}, respectively.
\biblatex 宏包依赖于 \eTeX 。
很长时间以来,\TeX 发行版就带有 \eTeX ,并且近来主流的发行版都默认使用。
\biblatex 宏包会检查是否在 \eTeX 下运行。
只需要像平常一样编译你的文档即可,基本上是可以运行的。
如果你得到错误信息,尝试用 \bin{elatex} 或 \bin{pdfelatex} 分别代替 \bin{latex} 或 \bin{pdflatex} 来编译文档。

\item[\biber]
%\biber is the backend of \biblatex used to transfer data from source files to the \latex code. \biber comes with TeX Live and is also available from SourceForge.\fnurl{http://biblatex-biber.sourceforge.net/}. \biber uses the \texttt{btparse} C library for \bibtex format file parsing which aimed to be compatible with \bibtex's parsing rules but also aimed at correcting some of the common problems. For details, see the manual page for the Perl \texttt{Text::BibTeX} module\fnurl{http://search.cpan.org/~ambs/Text-BibTeX}.
\biber 是\biblatex 默认的后端程序。你只需要 \BibTeX 或者 \biber 中的一个后端程序。
\TeX Live 中带有 \biber ,也可以从 SourceForge 得到。\fnurl{http://biblatex-biber.sourceforge.net/}
\biber 使用 C 程序库 \texttt{btparse} 解析 \BibTeX 格式文件,
这既为了兼容 \BibTeX 的解析规则,也用于修正一些常见问题。
详见 Perl 的 \texttt{Text::BibTeX} 模块(module)的手册页。\fnurl{http://search.cpan.org/~ambs/Text-BibTeX}

\item[etoolbox] 
%This \latex package, which is loaded automatically, provides generic programming facilities required by \biblatex. It is available from \acr{CTAN}.\fnurl{http://ctan.org/pkg/etoolbox}
自动加载,提供\biblatex 所需的通用编程工具,可以从 \acr{CTAN} 下载。\fnurl{http://ctan.org/pkg/etoolbox}

\item[kvoptions] 
%This \latex package, which is also loaded automatically, is used for internal option handling. It is available with the \sty{oberdiek} package bundle from \acr{CTAN}.\fnurl{http://ctan.org/pkg/kvoptions}
自动加载,用于内部选项处理。可以和 \sty{oberdiek} 宏包集一起从 \acr{CTAN} 下载。\fnurl{http://ctan.org/pkg/kvoptions}

\item[logreq]
%This \latex package, which is also loaded automatically, provides a frontend for writing machine-readable messages to an auxiliary log file. It is available from \acr{CTAN}.\fnurl{http://ctan.org/pkg/logreq/}
自动加载,它提供的前端可用于将机器可读信息写入辅助 log 文件,
可以从 \acr{CTAN} 下载。\fnurl{http://ctan.org/pkg/logreq/}

\item[xstring]
%This \latex package, which is also loaded automatically, provides advanced string processing macros It is available from \acr{CTAN}.\fnurl{http://ctan.org/pkg/xstring/}
自动加载,提供了一些高级字符串处理宏。
可以从 \acr{CTAN} 下载。\fnurl{http://ctan.org/pkg/xstring/}

\end{marglist}

%Apart from the above resources, \biblatex also requires the standard \latex packages \sty{keyval} and \sty{ifthen} as well as the \sty{url} package. These package are included in all common \tex distributions and will be loaded automatically.

除了上述资源,\biblatex 还需要 \sty{keyval}、\sty{ifthen} 以及 \sty{url} 等标准 \LaTeX 宏包。
常见的 \TeX 发行版中都会带有这些宏包,而且本宏包会自动加载。

\subsubsection{推荐包}
\label{int:pre:rec}

%The packages listed in this section are not required for \biblatex to function, but they provide recommended additional functions or enhance existing features. The package loading order does not matter.

这一节所列出的宏包对于运行 \biblatex 不是必须的。
不过,它们可以提供一些值得推荐的额外功能,或者加强已有的特征。
宏包载入的顺序并不重要。

\begin{marglist}

\item[babel/polyglossia]
%The \sty{babel} and \sty{polyglossia} packages provides the core architecture for multilingual typesetting. If you are writing in a language other than American English, using one of these packages is strongly recommended. You should load \sty{babel} or \sty{polyglossia} before \biblatex and then \biblatex will detect \sty{babel} or \sty{polyglossia} automatically.

\sty{babel} 和 \sty{polyglossia} 宏包提供了多语种排版的核心架构。
如果你使用美式英语以外的语言写作,那么强烈推荐使用这两个宏包中的一个。
你应当在 \biblatex 之前载入 \sty{babel} 或 \sty{polyglossia},
这样 \biblatex 宏包可以自动检测。

\item[csquotes]
%If this package is available, \biblatex will use its language sensitive quotation facilities to enclose certain titles in quotation marks. If not, \biblatex uses quotes suitable for American English as a fallback. When writing in a language other than American English, loading \sty{csquotes} is strongly recommended.\fnurl{http://ctan.org/pkg/csquotes/}

如果使用该宏包,\biblatex 会使用它的引用语工具给相应标题加上语言相关的引号。
如果没有,那么 \biblatex 会使用作为后备的美式英语的引号。
当使用其它语言写作时,强烈推荐使用 \sty{csquotes} 宏包。\fnurl{http://ctan.org/pkg/csquotes/}

\item[xpatch]
%The \sty{xpatch} package extends the patching commands of \sty{etoolbox} to \biblatex bibliography macros, drivers and formatting directives.\fnurl{http://ctan.org/pkg/xpatch/}

\sty{xpatch} 宏包为 \biblatex 宏、驱动和格式指令扩展了 \sty{etoolbox} 的一些补丁命令。\fnurl{http://ctan.org/pkg/xpatch/}

\end{marglist}

\subsubsection{兼容的包}
\label{int:pre:cmp}

%The \biblatex package provides dedicated compatibility code for the classes and packages listed in this section.

\biblatex 宏包专门为本节所列出的文档类和宏包提供了兼容性代码。

\begin{marglist}

\item[hyperref]
%The \sty{hyperref} package transforms citations into hyperlinks. See the \opt{hyperref} and \opt{backref} package options in \secref{use:opt:pre:gen} for further details. When using the \sty{hyperref} package, it is preferable to load it after \biblatex.
\sty{hyperref} 宏包将引用转化为超链接。
详见 \secref{use:opt:pre:gen}  一节中的 \opt{hyperref} 和 \opt{backref} 宏包选项。
当使用 \sty{hyperref} 宏包时,最好在 \biblatex 之后载入。

\item[showkeys]
%The \sty{showkeys} package prints the internal keys of, among other things, citations in the text and items in the bibliography. The package loading order does not matter.
\sty{showkeys} 宏包会打印出文本引用和参考条目的内部键值。
宏包载入的顺序不重要。

\item[memoir]
%When using the \sty{memoir} class, the default bibliography headings are adapted such that they blend well with the default layout of this class. See \secref{use:cav:mem} for further usage hints.
使用 \sty{memoir} 文档类会调整默认的参考文献标题,从而与该文档类默认的页面布局相协调。
更多使用提示请参考 \secref{use:cav:mem} 一节。

\item[\acr{KOMA}-Script]
%When using any of the \sty{scrartcl}, \sty{scrbook}, or \sty{scrreprt} classes, the default bibliography headings are adapted such that they blend with the default layout of these classes. See \secref{use:cav:scr} for further usage hints.
使用 \sty{scrartcl}、\sty{scrbook} 或 \sty{scrreprt} 文档类中的任何一个都会调整默认的参考文献标题,
从而与这些文档类默认的页面布局相协调。
更多使用提示请参考 \secref{use:cav:scr} 一节。
\end{marglist}

\subsubsection{不兼容的包}
\label{int:pre:inc}

%The packages listed in this section are not compatible with \biblatex. Since it reimplements the bibliographic facilities of \latex from the ground up, \biblatex naturally conflicts with all packages modifying the same facilities. This is not specific to \biblatex. Some of the packages listed below are also incompatible with each other for the same reason.

本节列出了与 \biblatex 不兼容的宏包。
\biblatex 从根本上重新实现了 \LaTeX 的文献功能,因此很自然地与修改这些功能的所有宏包相冲突。
这并不是 \biblatex 独有的——在列出的宏包中,出于同样的原因,有些宏包相互之间也是不兼容的。

\begin{marglist}

\item[babelbib]
%The \sty{babelbib} package provides support for multilingual bibliographies. This is a standard feature of \biblatex. Use the \bibfield{langid} field and the package option \opt{autolang} for similar functionality. Note that \biblatex automatically adjusts to the main document language if \sty{babel} or \sty{polyglossia} is loaded. You only need the above mentioned features if you want to switch languages on a per"=entry basis within the bibliography. See \secref{bib:fld:spc, use:opt:pre:gen} for details. Also see \secref{use:lng}.
\sty{babelbib} 宏包为多语种文献提供了支持,这正是 \biblatex 的一个典型特点。
使用 \bibfield{langid} 域和宏包选项 \opt{autolang} 即可实现类似的功能。
请注意,当载入 \sty{bable} 或 \sty{polyglossia} 宏包时 \biblatex 会自动调整主文档的语言。
如果想要在文献中每个条目里切换语言,你只需要以上提到的特性。
具体细节请参考 \secref{bib:fld:spc, use:opt:pre:gen} 以及 \secref{use:lng} 几节。

\item[backref]
%The \sty{backref} package creates back references in the bibliography. See the package options \opt{hyperref} and \opt{backref} in \secref{use:opt:pre:gen} for comparable functionality.
\sty{backref} 宏包可以在参考文献中创建反向引用。
类似的功能请参考 \secref{use:opt:pre:gen} 一节中的宏包选项 \opt{hyperref} 和 \opt{backref}。

\item[bibtopic]
%The \sty{bibtopic} package provides support for bibliographies subdivided by topic, type, or other criteria. For bibliographies subdivided by topic, see the category feature in \secref{use:bib:cat} and the corresponding filters in \secref{use:bib:bib}. Alternatively, you may use the \bibfield{keywords} field in conjunction with the \opt{keyword} and \opt{notkeyword} filters for comparable functionality, see \secref{bib:fld:spc, use:bib:bib} for details. For bibliographies subdivided by type, use the \opt{type} and \opt{nottype} filters. Also see \secref{use:use:div} for examples.
\sty{bibtopic} 宏包支持根据主题、类型或者其它标准细分文献。
对于按照主题细分文献,可以参考 \secref{use:bib:cat} 一节的类型特征以及 \secref{use:bib:bib} 一节中相应的。
另外,你也可以使用 \bibfield{keywords} 域结合 \opt{keyword} 和 \opt{notkeyword} 过滤器来实现相应功能,细节请参考 \secref{bib:fld:spc, use:bib:bib}。
对于按照类型细分文献,可以使用 \opt{type} 和 \opt{nottype} 过滤器。
相关例子请参考 \secref{use:use:div}。

\item[bibunits]
%The \sty{bibunits} package provides support for multiple partial (\eg per chapter) bibliographies. See \sty{chapterbib}.
\sty{bibunits} 宏包支持多个部分(例如每一章内)的参考文献。请参考 \sty{chapterbib}。

\item[chapterbib]
%The \sty{chapterbib} package provides support for multiple partial bibliographies. Use the \env{refsection} environment and the \opt{section} filter for comparable functionality. Alternatively, you might also want to use the \env{refsegment} environment and the \opt{segment} filter. See \secref{use:bib:sec, use:bib:seg, use:bib:bib} for details. Also see \secref{use:use:mlt} for examples.
\sty{chapterbib} 宏包支持多个部分的参考文献。
使用 \env{refsection} 环境和 \opt{section} 过滤器可以实现相应效果。
此外,你也可能需要 \env{refsegment} 环境和 \opt{segment} 过滤器。
细节请参考 \secref{use:bib:sec, use:bib:seg, use:bib:bib}。
相关实例请参考 \secref{use:use:mlt}。

\item[cite]
%The \sty{cite} package automatically sorts numeric citations and can compress a list of consecutive numbers to a range. It also makes the punctuation used in citations configurable. For sorted and compressed numeric citations, see the \opt{sortcites} package option in \secref{use:opt:pre:gen} and the \texttt{numeric-comp} citation style in \secref{use:xbx:cbx}. For configurable punctuation, see \secref{use:fmt}.
\sty{cite} 可以自动对引用编号进行排序,并且将连续的数字缩写为一个区间。
它也可以配置引用中的标点符号。
关于引用编号的排序和缩写,请参考 \secref{use:opt:pre:gen} 一节中的 \opt{sortcites} 宏包选项和 \secref{use:xbx:cbx} 一节中的 \texttt{numeric-comp} 引用样式。
关于可配置的标点请参考 \secref{use:fmt}。

\item[citeref]
%Another package for creating back references in the bibliography. See \sty{backref}.
另一个可以创建反向引用的宏包。参考 \sty{backref} 条目。

\item[inlinebib]
%The \sty{inlinebib} package is designed for traditional citations given in footnotes. For comparable functionality, see the verbose citation styles in \secref{use:xbx:cbx}.
\sty{inlinebib} 宏包用于脚注文献这种传统引用样式。
相应的功能请参考 \secref{use:xbx:cbx} 中详细的引用样式说明。

\item[jurabib]
%Originally designed for citations in law studies and (mostly German) judicial documents, the \sty{jurabib} package also provides features aimed at users in the humanities. In terms of the features provided, there are some similarities between \sty{jurabib} and \biblatex but the approaches taken by both packages are quite different. Since both \sty{jurabib} and \biblatex are full"=featured packages, the list of similarities and differences is too long to be discussed here.
\sty{jurabib} 宏包原本用于法学和司法文件(主要是德文)中的引用,它也为人文学科中的使用者提供了一些特性。
在提供这些特征方面,\sty{jurabib} 和 \biblatex 有一些类似之处,但是采用的手段是截然不同的。
由于 \sty{jurabib} 和 \biblatex 都是那种功能齐备的宏包,鉴于篇幅这里不再赘述它们的异同之处。

\item[mcite]
%The \sty{mcite} package provides support for grouped citations, \ie multiple items can be cited as a single reference and listed as a single block in the bibliography. The citation groups are defined as the items are cited. This only works with unsorted bibliographies. The \sty{biblatex} package also supports grouped citations, which are called <entry sets> or <reference sets> in this manual. See \secref{use:use:set,use:bib:set,use:cit:mct} for details.
\sty{mcite} 提供了分组引用的支持,也就是说,不同条目可以指向同一处引用,并且在参考文献中作为同一条目列在一起。
引用组依照被引用的条目定义,不过这只在未排序的参考文献中有效。
\biblatex 宏包同样支持分组引用,在本手册中称之为 “条目集” 或 “参考文献集”。
细节请参考 \secref{use:use:set,use:bib:set,use:cit:mct}。

\item[mciteplus]
%A significantly enhanced reimplementation of the \sty{mcite} package which supports grouping in sorted bibliographies. See \sty{mcite}.
\sty{mcite} 宏包的一个加强版的重新实现,可以支持排序文献的分组。参考 \sty{mcite} 宏包条目。

\item[multibib]
%The \sty{multibib} package provides support for bibliographies subdivided by topic or other criteria. See \sty{bibtopic}.
\sty{multibib} 宏包支持依照主题或其它标准细分文献。参考 \sty{bibtopic} 宏包条目。

\item[natbib]
%The \sty{natbib} package supports numeric and author"=year citation schemes, incorporating sorting and compression code found in the \sty{cite} package. It also provides additional citation commands and several configuration options. See the \texttt{numeric} and \texttt{author-year} citation styles and their variants in \secref{use:xbx:cbx}, the \opt{sortcites} package option in \secref{use:opt:pre:gen}, the citation commands in \secref{use:cit}, and the facilities discussed in \secref{use:bib:hdg, use:bib:nts, use:fmt} for comparable functionality. Also see \secref{use:cit:nat}.
\sty{natbib} 宏包支持编号和作者---年份引用格式,以及\sty{cite} 宏包中的合并排序和压缩代码。
它同样提供了一些额外的引用命令和几种设置选项。
相应的功能请参考 \secref{use:xbx:cbx} 中的 \texttt{numeric} 和 \texttt{author-year} 引用样式及其变种,
\secref{use:opt:pre:gen}中的 \opt{sortcites} 宏包选项,\secref{use:cit} 中的引用命令,
以及 \secref{use:bib:hdg, use:bib:nts, use:fmt} 中讨论的工具。
也可以参考 \secref{use:cit:nat}。

\item[splitbib]
%The \sty{splitbib} package provides support for bibliographies subdivided by topic. See \sty{bibtopic}.
\sty{splitbib} 宏包支持按照主题细分文献。参考 \sty{bibtopic} 宏包条目。

\item[titlesec]
%The \sty{titlesec} package redefines user-level document division commands such as \cmd{chapter} or \cmd{section}. This approach is not compatible with internal command changes applied by the \sty{biblatex} \texttt{refsection} and \texttt{refsegment} option settings described in \secref{use:opt:pre:gen}.
\sty{titlesec} 宏包重新定义了一些用户水平的文档划分命令,例如 \cmd{chapter} 或 \cmd{section}。
这种方法与 \biblatex 的 \texttt{refsection} 和 \texttt{refsegment} 选项设置引起的内部命令改动不兼容,具体描述在 \secref{use:opt:pre:gen} 一节。

\item[ucs]
%The \sty{ucs} package provides support for \utf encoded input. Either use \sty{inputenc}'s standard \file{utf8} module or a Unicode enabled engine such as \xetex or \luatex instead.
%The \sty{ucs} package provides support for \utf encoded input. Either use \sty{inputenc}'s standard \file{utf8} module or a Unicode enabled engine such as \xetex or \luatex instead.
\sty{ucs} 宏包提供 \utf 编码输入的支持。
可以使用 \sty{inputenc} 宏包的标准 \file{utf8} 模块或者 \XeTeX 、\LuaTeX 等支持 Unicode 的编译引擎来实现这一功能。

\end{marglist}

\subsubsection{\biber/\biblatex 兼容性}
\label{int:pre:bibercompat}

%\biber\ versions are closely coupled with \biblatex\ versions. You
%need to have the right combination of the two. \biber\ will warn you
%during processing if it encounters information which comes from a
%\biblatex\ version which is incompatible. \tabref{tab:int:pre:bibercompat} shows a
%compatibility matrix for the recent versions.

\biber 的版本与 \biblatex 的版本有着紧密的联系。
你需要二者正确的组合。
如果发现来自于不兼容的 \biblatex 版本信息,\biber 会在处理过程中发出警告。
\tabref{tab:int:pre:bibercompat} 展示了最近一些版本的兼容性状况。

\begin{table}
	\tablesetup\centering
	\begin{tabular}{cc}
		\toprule
		\sffamily\bfseries\spotcolor Biber 版本
		& \sffamily\bfseries\spotcolor \biblatex\ 版本\\
		\midrule
		2.6 & 3.5, 3.6\\
		2.5 & 3.4\\
		2.4 & 3.3\\
		2.3 & 3.2\\
		2.2 & 3.1\\
		2.1 & 3.0\\
		2.0 & 3.0\\
		1.9 & 2.9\\
		1.8 & 2.8\\
		1.7 & 2.7\\
		1.6 & 2.6\\
		1.5 & 2.5\\
		1.4 & 2.4\\
		1.3 & 2.3\\
		1.2 & 2.1, 2.2\\
		1.1 & 2.1\\
		1.0 & 2.0\\
		0.9.9 & 1.7x\\
		0.9.8 & 1.7x\\
		0.9.7 & 1.7x\\
		0.9.6 & 1.7x\\
		0.9.5 & 1.6x\\
		0.9.4 & 1.5x\\
		0.9.3 & 1.5x\\
		0.9.2 & 1.4x\\
		0.9.1 & 1.4x\\
		0.9 & 1.4x\\
		\bottomrule
	\end{tabular}
	\caption{\biber/\biblatex\ 兼容性}
	\label{tab:int:pre:bibercompat}
\end{table}


% !TeX encoding = UTF-8
% DatabaseGuide.tex
%\section{Database Guide}
\section{数据库指南}
\label{bib}
%This section describes the default data model defined in the \file{blx-dm.def} file which is part of \path{biblatex}. The data model is defined using the macros documented in \secref{aut:ctm:dm}. It is possible to redefine the data model which both \biblatex and \biber use so that datasources can contain new entrytypes and fields (which of course will need style support). The data model specification also allows for constraints to be defined so that data sources can be validated against the data model (using \biber's \opt{--validate-datamodel} option). Users who want to customise the data model need to look at the \file{blx-dm.def} file and to read \secref{aut:ctm:dm}.
本节描述 \file{blx-dm.def} 中定义的默认数据模型。该文件是宏包的一部分。
该数据模型的定义由\secref{aut:ctm:dm} 节中的宏实现。
因此,可以重新定义 \biblatex 和 \biber 所用的数据模型,
使得数据源可以包括新的\gls{条目类型}和\gls{域}(当然这需要\gls{样式}文件支持)。
数据模型规范还允许定义约束,使得数据源可以根据数据模型进行校验
(使用 \biber 的 \path{--validate_datamodel} 选项)。
若需要定制数据模型,请参考 \file{blx-dm.def} 文件和 \secref{aut:ctm:dm} 节。

%\subsection{Entry Types}
\subsection{条目类型}
\label{bib:typ}

%This section gives an overview of the entry types supported by the default \biblatex data model along with the fields supported by each type.

本节介绍 \biblatex 默认数据模型支持的条目类型及每种条目类型支持的域。
%\subsubsection{Regular Types}
\subsubsection{常规类型}
\label{bib:typ:blx}

%The lists below indicate the fields supported by each entry type. Note that the mapping of fields to an entry type is ultimately at the discretion of the bibliography style. The lists below therefore serve two purposes. They indicate the fields supported by the standard styles which ship with this package and they also serve as a model for custom styles. Note that the <required> fields are not strictly required in all cases, see \secref{bib:use:key} for details. The fields marked as <optional> are optional in a technical sense. Bibliographical formatting rules usually require more than just the <required> fields.
下面的列表说明了每种条目类型支持的域。
注意,每种条目类型对域的使用是由参考文献样式决定的。
因此,下面的列表有两个目的,一是说明本宏包标准样式支持的域,二是作为定制样式的模板。
注意,所谓“必选”域并不是在所有情况下都严格必不可少的,详见 \secref{bib:use:key} 节。
而标记“可选”的域技术上是可选的,
不过通常来说,文献格式化规则往往不会仅限于“必选”域。

%The default data model defined a few constraints for the format of date fields, ISBNs and some special fields like \bibfield{gender} but the constraints are only used if validating against the data model with \biber's \opt{--validate-datamodel} option. Generic fields like \bibfield{abstract} and \bibfield{annotation} or \bibfield{label} and \bibfield{shorthand} are not included in the lists below because they are independent of the entry type. The special fields discussed in \secref{bib:fld:spc}, which are also independent of the entry type, are not included in the lists either. See the default data model specification in the file \file{blx-dm.def} which comes with \biblatex for a complete specification.
默认的数据模型为日期域、ISBN类的域和 \bibfield{gender} 等特殊域定义了一些约束。
但这些约束仅用于校验这些域是否合乎数据模型(通过 \biber 的 \path{--validate_datamodel} 选项)。
通用域如 \bibfield{abstract}、\bibfield{annotation}、\bibfield{label} 和 \bibfield{shorthand} 并不在下面的列表中,因为它们独立于条目类型;
\secref{bib:fld:spc} 节讨论的特殊域同样也独立于条目类型,因此也不在下面的列表中。
需要了解完整的数据模型规范,详见\biblatex 附带文件 \file{blx-dm.def}中的默认数据模型。

\begin{typelist}

\typeitem{article}

%An article in a journal, magazine, newspaper, or other periodical which forms a self"=contained unit with its own title. The title of the periodical is given in the \bibfield{journaltitle} field. If the issue has its own title in addition to the main title of the periodical, it goes in the \bibfield{issuetitle} field. Note that \bibfield{editor} and related fields refer to the journal while \bibfield{translator} and related fields refer to the article.

指从期刊、杂志、报纸或其他周期性刊物中的析出文章,具有自己的标题,是一个独立单元。
刊物名在 \bibfield{journaltitle} 域中给出。
如果除刊物名外,某期刊物也有具体的题名,那么该题名在 \bibfield{issuetitle} 域中给出。
注意,\bibfield{editor} 及相关域指的是期刊,而 \bibfield{translator} 及其相关域则涉及到文章。

\reqitem{author, title, journaltitle, year/date}
\optitem{translator, annotator, commentator, subtitle, titleaddon, editor, editora, editorb, editorc, journalsubtitle, issuetitle, issuesubtitle, language, origlanguage, series, volume, number, eid, issue, month, pages, version, note, issn, addendum, pubstate, doi, eprint, eprintclass, eprinttype, url, urldate}

\typeitem{book}

%A single"=volume book with one or more authors where the authors share credit for the work as a whole. This entry type also covers the function of the \bibtype{inbook} type of traditional \bibtex, see \secref{bib:use:inb} for details.

单卷本的书籍,有一位或多位作者,其中多位作者名构成一个整体名单作为该著作的责任者。
该条目类型也涵盖了传统 \BibTeX 的 \bibtype{inbook} 类型,详见 \secref{bib:use:inb} 节。

\reqitem{author, title, year/date}
\optitem{editor, editora, editorb, editorc, translator, annotator, commentator, introduction, foreword, afterword, subtitle, titleaddon, maintitle, mainsubtitle, maintitleaddon, language, origlanguage, volume, part, edition, volumes, series, number, note, publisher, location, isbn, chapter, pages, pagetotal, addendum, pubstate, doi, eprint, eprintclass, eprinttype, url, urldate}

\typeitem{mvbook}

%A multi"=volume \bibtype{book}. For backwards compatibility, multi"=volume books are also supported by the entry type \bibtype{book}. However, it is advisable to make use of the dedicated entry type \bibtype{mvbook}.

多卷本书籍。为了向后兼容,多卷书也可用 \bibtype{book} 条目类型。
然而建议最好使用该专用条目类型 \bibtype{mvbook}。

\reqitem{author, title, year/date}
\optitem{editor, editora, editorb, editorc, translator, annotator, commentator, introduction, foreword, afterword, subtitle, titleaddon, language, origlanguage, edition, volumes, series, number, note, publisher, location, isbn, pagetotal, addendum, pubstate, doi, eprint, eprintclass, eprinttype, url, urldate}

\typeitem{inbook}

%A part of a book which forms a self"=contained unit with its own title. Note that the profile of this entry type is different from standard \bibtex, see \secref{bib:use:inb}.

书的一部分。它是一个独立的单元,有自己的标题。
注意,该类型的定义不同于标准 \BibTeX 给出的定义,见 \secref{bib:use:inb} 节。

\reqitem{author, title, booktitle, year/date}
\optitem{bookauthor, editor, editora, editorb, editorc, translator, annotator, commentator, introduction, foreword, afterword, subtitle, titleaddon, maintitle, mainsubtitle, maintitleaddon, booksubtitle, booktitleaddon, language, origlanguage, volume, part, edition, volumes, series, number, note, publisher, location, isbn, chapter, pages, addendum, pubstate, doi, eprint, eprintclass, eprinttype, url, urldate}

\typeitem{bookinbook}

%This type is similar to \bibtype{inbook} but intended for works originally published as a stand-alone book. A typical example are books reprinted in the collected works of an author.

类似于 \bibtype{inbook},但用于原本已经出版的单行本。
典型的例子是在一位作者作品集中再版的书籍。

\typeitem{suppbook}

%Supplemental material in a \bibtype{book}. This type is closely related to the \bibtype{inbook} entry type. While \bibtype{inbook} is primarily intended for a part of a book with its own title (\eg a single essay in a collection of essays by the same author), this type is provided for elements such as prefaces, introductions, forewords, afterwords, etc. which often have a generic title only. Style guides may require such items to be formatted differently from other \bibtype{inbook} items. The standard styles will treat this entry type as an alias for \bibtype{inbook}.

\bibtype{book} (书)的补充材料,与 \bibtype{inbook} 条目类型密切相关。
\bibtype{inbook} 主要用于一本书中带有自身标题的部分,例如一本散文集中相同作者的单独一篇散文;
而本条目用于诸如序言、导论、前言、后记等部分,往往只有一个通用标题。
一些参考文献著录标准可能会要求该条目类型的著录格式不同于 \bibtype{inbook}。
而标准样式将其视为 \bibtype{inbook} 的别名。

\typeitem{booklet}

%A book"=like work without a formal publisher or sponsoring institution. Use the field \bibfield{howpublished} to supply publishing information in free format, if applicable. The field \bibfield{type} may be useful as well.

类似于书籍,但没有正式的出版或赞助机构。
如果允许,可以使用 \bibfield{howpublished} 域提供自由格式的出版信息。
或者也可以使用 \bibfield{type} 域。

\reqitem{author/editor, title, year/date}
\optitem{subtitle, titleaddon, language, howpublished, type, note, location, chapter, pages, pagetotal, addendum, pubstate, doi, eprint, eprintclass, eprinttype, url, urldate}

\typeitem{collection}

%A single"=volume collection with multiple, self"=contained contributions by distinct authors which have their own title. The work as a whole has no overall author but it will usually have an editor.

单卷本的文集,由一些具有不同作者和题名的独立稿件构成。
作品集作为一个整体没有总体意义上的作者,但通常有一位编者。

\reqitem{editor, title, year/date}
\optitem{editora, editorb, editorc, translator, annotator, commentator, introduction, foreword, afterword, subtitle, titleaddon, maintitle, mainsubtitle, maintitleaddon, language, origlanguage, volume, part, edition, volumes, series, number, note, publisher, location, isbn, chapter, pages, pagetotal, addendum, pubstate, doi, eprint, eprintclass, eprinttype, url, urldate}

\typeitem{mvcollection}

%A multi"=volume \bibtype{collection}. For backwards compatibility, multi"=volume collections are also supported by the entry type \bibtype{collection}. However, it is advisable to make use of the dedicated entry type \bibtype{mvcollection}.

多卷本文集。为了向后兼容,也可用条目类型 \bibtype{collection} 表示。
然而,建议最好还是使用专用条目类型 \bibtype{mvcollection}。

\reqitem{editor, title, year/date}
\optitem{editora, editorb, editorc, translator, annotator, commentator, introduction, foreword, afterword, subtitle, titleaddon, language, origlanguage, edition, volumes, series, number, note, publisher, location, isbn, pagetotal, addendum, pubstate, doi, eprint, eprintclass, eprinttype, url, urldate}

\typeitem{incollection}

%A contribution to a collection which forms a self"=contained unit with a distinct author and title. The \bibfield{author} refers to the \bibfield{title}, the \bibfield{editor} to the \bibfield{booktitle}, \ie the title of the collection.

文集中的一篇稿件,是一个独立的单元,具有自己的标题。
\bibfield{author} 指的是 \bibfield{title} 的作者,
而 \bibfield{editor} 指的是 \bibfield{booktitle}( 即文集标题)的编者。

\reqitem{author, title, booktitle, year/date}
\optitem{editor, editora, editorb, editorc, translator, annotator, commentator, introduction, foreword, afterword, subtitle, titleaddon, maintitle, mainsubtitle, maintitleaddon, booksubtitle, booktitleaddon, language, origlanguage, volume, part, edition, volumes, series, number, note, publisher, location, isbn, chapter, pages, addendum, pubstate, doi, eprint, eprintclass, eprinttype, url, urldate}

\typeitem{suppcollection}

%Supplemental material in a \bibtype{collection}. This type is similar to \bibtype{suppbook} but related to the \bibtype{collection} entry type. The standard styles will treat this entry type as an alias for \bibtype{incollection}.

\bibtype{collection} 中的补充材料。
类似于 \bibtype{suppbook} 之于 \bibtype{book}。
标准样式将其视为 \bibtype{incollection} 的别名。

\typeitem{manual}

%Technical or other documentation, not necessarily in printed form. The \bibfield{author} or \bibfield{editor} is omissible in terms of \secref{bib:use:key}.

技术或其它文档,不必是印刷形式的。
按照 \secref{bib:use:key} 一节,\bibfield{author} 或者 \bibfield{editor} 是可以省略的。

\reqitem{author/editor, title, year/date}
\optitem{subtitle, titleaddon, language, edition, type, series, number, version, note, organization, publisher, location, isbn, chapter, pages, pagetotal, addendum, pubstate, doi, eprint, eprintclass, eprinttype, url, urldate}

\typeitem{misc}

%A fallback type for entries which do not fit into any other category. Use the field \bibfield{howpublished} to supply publishing information in free format, if applicable. The field \bibfield{type} may be useful as well. \bibfield{author}, \bibfield{editor}, and \bibfield{year} are omissible in terms of \secref{bib:use:key}.

备选类型,用于无法归入任何其它类别的条目。
合适的话,可以使用 \bibfield{howpublished} 域,可以提供自由格式的出版信息。
或者也可以使用 \bibfield{type} 域。
按照 \secref{bib:use:key} 节,\bibfield{author}、\bibfield{editor} 和 \bibfield{year} 可以省略。

\reqitem{author/editor, title, year/date}
\optitem{subtitle, titleaddon, language, howpublished, type, version, note, organization, location, date, month, year, addendum, pubstate, doi, eprint, eprintclass, eprinttype, url, urldate}

\typeitem{online}

%An online resource. \bibfield{author}, \bibfield{editor}, and \bibfield{year} are omissible in terms of \secref{bib:use:key}. This entry type is intended for sources such as web sites which are intrinsically online resources. Note that all entry types support the \bibfield{url} field. For example, when adding an article from an online journal, it may be preferable to use the \bibtype{article} type and its \bibfield{url} field.

在线资源。
按照 \secref{bib:use:key} 节,\bibfield{author},\bibfield{editor} 和 \bibfield{year} 可以省略。
该条目类型用于网址等固有在线资源。
注意:所有条目类型都支持 \bibfield{url} 域。
比如,当增加一篇来自在线期刊的文章时,应优先使用 \bibtype{article} 条目和它的 \bibfield{url} 域。

\reqitem{author/editor, title, year/date, url}
\optitem{subtitle, titleaddon, language, version, note, organization, date, month, year, addendum, pubstate, urldate}

\typeitem{patent}

%A patent or patent request. The number or record token is given in the \bibfield{number} field. Use the \bibfield{type} field to specify the type and the \bibfield{location} field to indicate the scope of the patent, if different from the scope implied by the \bibfield{type}. Note that the \bibfield{location} field is treated as a key list with this entry type, see \secref{bib:fld:typ} for details.

专利或专利申请。专利号或登记号在 \bibfield{number} 域中给出。
\bibfield{type} 域用于描述类型,
如果专利保护范围与 \bibfield{type} 域暗指的范围不一致,则可使用\bibfield{location} 域对专利的保护范围(权利范围)进行描述。
注意,\bibfield{location} 在本条目类型中以键值列表的方式处理,详见 \secref{bib:fld:typ} 节。

\reqitem{author, title, number, year/date}
\optitem{holder, subtitle, titleaddon, type, version, location, note, date, month, year, addendum, pubstate, doi, eprint, eprintclass, eprinttype, url, urldate}

\typeitem{periodical}

%An complete issue of a periodical, such as a special issue of a journal. The title of the periodical is given in the \bibfield{title} field. If the issue has its own title in addition to the main title of the periodical, it goes in the \bibfield{issuetitle} field. The \bibfield{editor} is omissible in terms of \secref{bib:use:key}.

周期性刊物中完整的一期,比如某一刊物的某一期特刊。
标题在 \bibfield{title} 域中给出。
如果该期除期刊主标题外还有自己的标题,那么在 \bibfield{issuetitle} 域中给出该信息。
根据 \secref{bib:use:key} 节,\bibfield{editor} 域可以省略。

\reqitem{editor, title, year/date}
\optitem{editora, editorb, editorc, subtitle, issuetitle, issuesubtitle, language, series, volume, number, issue, date, month, year, note, issn, addendum, pubstate, doi, eprint, eprintclass, eprinttype, url, urldate}

\typeitem{suppperiodical}

%Supplemental material in a \bibtype{periodical}. This type is similar to \bibtype{suppbook} but related to the \bibtype{periodical} entry type. The role of this entry type may be more obvious if you bear in mind that the \bibtype{article} type could also be called \bibtype{inperiodical}. This type may be useful when referring to items such as regular columns, obituaries, letters to the editor, etc. which only have a generic title. Style guides may require such items to be formatted differently from articles in the strict sense of the word. The standard styles will treat this entry type as an alias for \bibtype{article}.

\bibtype{periodical} 的补充材料,
类似于 \bibtype{suppbook} 之于 \bibtype{book}。
如果你意识到 \bibtype{article} 类型其实就是 \bibtype{inperiodical},那么本条目类型的作用就显而易见了。
该类型应用于仅具有通用标题的信息项,例如固定专栏、讣告、致编者的信等。
一些参考文献著录标准会严格要求这些信息的格式不同于 \bibtype{article}。
而标准样式则视其为 \bibtype{article} 的别名。

\typeitem{proceedings}

%A single"=volume conference proceedings. This type is very similar to \bibtype{collection}. It supports an optional \bibfield{organization} field which holds the sponsoring institution. The \bibfield{editor} is omissible in terms of \secref{bib:use:key}.

单卷本的会议录(会议文集,汇编)。与 \bibtype{collection} 非常相似。
它支持可选的 \bibfield{organization} 域用于给出主办机构。
根据 \secref{bib:use:key} 节,\bibfield{editor} 域可以省略。

\reqitem{title, year/date}
\optitem{editor, subtitle, titleaddon, maintitle, mainsubtitle, maintitleaddon, eventtitle, eventtitleaddon, eventdate, venue, language, volume, part, volumes, series, number, note, organization, publisher, location, month, isbn, chapter, pages, pagetotal, addendum, pubstate, doi, eprint, eprintclass, eprinttype, url, urldate}

\typeitem{mvproceedings}

%A multi"=volume \bibtype{proceedings} entry. For backwards compatibility, multi"=volume proceedings are also supported by the entry type \bibtype{proceedings}. However, it is advisable to make use of the dedicated entry type \bibtype{mvproceedings}

多卷 \bibtype{proceedings} 条目,类似于 \bibtype{mvbook} 之于 \bibtype{book}。

\reqitem{title, year/date}
\optitem{editor, subtitle, titleaddon, eventtitle, eventtitleaddon, eventdate, venue, language, volumes, series, number, note, organization, publisher, location, month, isbn, pagetotal, addendum, pubstate, doi, eprint, eprintclass, eprinttype, url, urldate}

\typeitem{inproceedings}

%An article in a conference proceedings. This type is similar to \bibtype{incollection}. It supports an optional \bibfield{organization} field.

会议文集中的一篇文章,与 \bibtype{incollection} 类似。支持 \bibfield{organization} 可选域。

\reqitem{author, title, booktitle, year/date}
\optitem{editor, subtitle, titleaddon, maintitle, mainsubtitle, maintitleaddon, booksubtitle, booktitleaddon, eventtitle, eventtitleaddon, eventdate, venue, language, volume, part, volumes, series, number, note, organization, publisher, location, month, isbn, chapter, pages, addendum, pubstate, doi, eprint, eprintclass, eprinttype, url, urldate}

\typeitem{reference}

%A single"=volume work of reference such as an encyclopedia or a dictionary. This is a more specific variant of the generic \bibtype{collection} entry type. The standard styles will treat this entry type as an alias for \bibtype{collection}.

单卷本的参考书,诸如百科全书或词典等。
它是一般的 \bibtype{collection} 条目类型的一种特殊变体。
标准样式将其视为 \bibtype{collection} 的别名。

\typeitem{mvreference}

%A multi"=volume \bibtype{reference} entry. The standard styles will treat this entry type as an alias for \bibtype{mvcollection}. For backwards compatibility, multi"=volume references are also supported by the entry type \bibtype{reference}. However, it is advisable to make use of the dedicated entry type \bibtype{mvreference}.

多卷本的 \bibtype{reference} 条目。标准样式将其视为 \bibtype{mvcollection} 的别名。
出于向后兼容考虑,也可以使用 \bibtype{reference} 条目。
不过,还是建议使用专门的 \bibtype{mvreference} 条目类型。

\typeitem{inreference}

%An article in a work of reference. This is a more specific variant of the generic \bibtype{incollection} entry type. The standard styles will treat this entry type as an alias for \bibtype{incollection}.。

参考书中的一篇文章,
它是一般的\bibtype{incollection} 条目的一种特殊变体。
标准样式将其视为 \bibtype{incollection} 的别名。

\typeitem{report}

%A technical report, research report, or white paper published by a university or some other institution. Use the \bibfield{type} field to specify the type of report. The sponsoring institution goes in the \bibfield{institution} field.

由大学或其它机构发行的技术报告、研究报告以及白皮书等。
使用 \bibfield{type} 域来指定报告的类型。
主办机构由 \bibfield{institution} 域给出。

\reqitem{author, title, type, institution, year/date}
\optitem{subtitle, titleaddon, language, number, version, note, location, month, isrn, chapter, pages, pagetotal, addendum, pubstate, doi, eprint, eprintclass, eprinttype, url, urldate}

\typeitem{set}

%An entry set. This entry type is special, see \secref{use:use:set} for details.

条目集,是一种特殊的类型条目,详见 \secref{use:use:set} 节。

\typeitem{thesis}

%A thesis written for an educational institution to satisfy the requirements for a degree. Use the \bibfield{type} field to specify the type of thesis.

为满足教育机构学位要求而撰写的学位论文。
使用 \bibfield{type} 域指定学位论文类型。

\reqitem{author, title, type, institution, year/date}
\optitem{subtitle, titleaddon, language, note, location, month, isbn, chapter, pages, pagetotal, addendum, pubstate, doi, eprint, eprintclass, eprinttype, url, urldate}

\typeitem{unpublished}

%A work with an author and a title which has not been formally published, such as a manuscript or the script of a talk. Use the fields \bibfield{howpublished} and \bibfield{note} to supply additional information in free format, if applicable.

有作者和标题但是没有正式出版的作品,例如手稿或演讲稿等。
允许的话,可使用 \bibfield{howpublished} 域和 \bibfield{note} 域提供自由格式的附加信息。

\reqitem{author, title, year/date}
\optitem{subtitle, titleaddon, language, howpublished, note, location, isbn, date, month, year, addendum, pubstate, url, urldate}

\typeitem{xdata}

%This entry type is special. \bibtype{xdata} entries hold data which may be inherited by other entries using the \bibfield{xdata} field. Entries of this type only serve as data containers; they may not be cited or added to the bibliography. See \secref{use:use:xdat} for details.

特殊类型,\bibtype{xdata} 条目保存有可以被其它条目用 \bibfield{xdata} 域继承的数据。
这一类型的条目只是作为数据容器,不可被引用或加入到文献表中,详见\secref{use:use:xdat} 节。

\typeitem{custom[a--f]}

%Custom types for special bibliography styles. Not used by the standard styles.

用于特殊参考文献样式的自定义条目类型,标准样式中不使用。

\end{typelist}

%\subsubsection{Type Aliases}
\subsubsection{类型别名}
\label{bib:typ:als}

%The entry types listed in this section are provided for backwards compatibility with traditional \bibtex styles. These aliases are resolved by the backend as the data is processed. Bibliography styles will see the entry type the alias points to, not the alias name. All unknown entry types are generally exported as \bibtype{misc}.

本节中列出的条目类型用于向后兼容传统的 \BibTeX 样式。
这些别名由后端程序在数据处理时统一处理。参考文献样式仅能见到这些别名所指代的类型,而不是这些别名本身。
所有未知的条目类型一般输出为 \bibtype{misc} 条目。

\begin{typelist}

\typeitem{conference} %A \bibtex legacy alias for \bibtype{inproceedings}.
\BibTeX 遗留的 \bibtype{inproceedings} 的别名。

\typeitem{electronic} %An alias for \bibtype{online}
\bibtype{online}的别名。

\typeitem{mastersthesis} %Similar to \bibtype{thesis} except that the \bibfield{type} field is optional and defaults to the localised term <Master's thesis>. You may still use the \bibfield{type} field to override that.
类似于 \bibtype{thesis},差别在于 \bibfield{type} 域是可选的,默认是本地化关键字 <Master's thesis> 所代表的本地化字符串。
用户可以直接使用 \bibfield{type} 域进行重新定义。

\typeitem{phdthesis} %Similar to \bibtype{thesis} except that the \bibfield{type} field is optional and defaults to the localised term <PhD thesis>. You may still use the \bibfield{type} field to override that.
类似于 \bibtype{thesis} ,差别在于 \bibfield{type} 域是可选的,默认是本地化关键字 <PhD thesis> 所代表的本地化字符串。
用户可以直接使用 \bibfield{type} 域进行重新定义。

\typeitem{techreport} % Similar to \bibtype{report} except that the \bibfield{type} field is optional and defaults to the localised term <technical report>. You may still use the \bibfield{type} field to override that.
类似于\bibtype{report} ,差别在于 \bibfield{type} 域是可选的,默认是本地化关键字 <technical report> 所代表的本地化字符串。
用户可以直接使用 \bibfield{type} 域进行重新定义。

\typeitem{www}
% An alias for \bibtype{online}, provided for \sty{jurabib} compatibility.
\bibtype{online} 的别名,用于兼容 \sty{jurabib} 宏包。

\end{typelist}

%\subsubsection{Unsupported Types}
\subsubsection{不支持的条目类型}
\label{bib:typ:ctm}

%The types in this section are similar to the custom types \bibtype{custom[a--f]}, \ie the standard bibliography styles provide no support for these types. When using the standard styles, they will be treated as \bibtype{misc} entries.

本节中的条目类型类似于自定义类型 \bibtype{custom[a--f]},
即,标准样式不支持这些类型,若使用标准样式,将会以 \bibtype{misc} 条目类型进行处理。

\begin{typelist}

\typeitem{artwork}
%Works of the visual arts such as paintings, sculpture, and installations.
视觉艺术作品,例如绘画、雕塑和装饰艺术品。

\typeitem{audio}
%Audio recordings, typically on audio \acr{CD}, \acr{DVD}, audio cassette, or similar media. See also \bibtype{music}.
录音作品,典型的有音频 \acr{CD}、\acr{DVD}、录音磁带或类似媒介。
参考 \bibtype{music} 类型。

\typeitem{bibnote}
%This special entry type is not meant to be used in the \file{bib} file like other types. It is provided for third-party packages like \sty{notes2bib} which merge notes into the bibliography. The notes should go into the \bibfield{note} field. Be advised that the \bibtype{bibnote} type is not related to the \cmd{defbibnote} command in any way. \cmd{defbibnote} is for adding comments at the beginning or the end of the bibliography, whereas the \bibtype{bibnote} type is meant for packages which render endnotes as bibliography entries.
这一特殊条目类型并不像其它类型那样用于 \file{bib} 文件中。
它主要是为第三方宏包提供 \sty{notes2bib} 等,用于将注记并入文献中。
注记应该在 \bibfield{note} 域中。
请谨记,\bibtype{bibnote} 类型与 \cmd{defbibnote} 命令毫无关系。
\cmd{defbibnote} 命令用来在参考文献表的开始或末尾处添加评论,
而 \bibtype{bibnote} 类型是为那些将尾注作为参考条目处理的宏包准备的。

\typeitem{commentary}
%Commentaries which have a status different from regular books, such as legal commentaries.
法律身份不同于一般书籍的评注,如司法评论等。

\typeitem{image}
%Images, pictures, photographs, and similar media.
图像、图画、摄影和类似媒介。

\typeitem{jurisdiction}
%Court decisions, court recordings, and similar things.
法庭判决、法庭记录和类似物。

\typeitem{legislation}
%Laws, bills, legislative proposals, and similar things.
法律、法案、立法提案和类似物。

\typeitem{legal}
%Legal documents such as treaties.
协议等法律文书。

\typeitem{letter}
%Personal correspondence such as letters, emails, memoranda, etc.
私人函件,例如信件、电子邮件、备忘录等。

\typeitem{movie}
%Motion pictures. See also \bibtype{video}.
动画。参考 \bibtype{video} 类型。

\typeitem{music}
%Musical recordings. This is a more specific variant of \bibtype{audio}.
音乐唱片,\bibtype{audio} 的一种特殊变体。

\typeitem{performance}
%Musical and theatrical performances as well as other works of the performing arts. This type refers to the event as opposed to a recording, a score, or a printed play.
音乐或戏剧表演和其它一些表演艺术作品。
这一条目类型指的是表演的事件,而不是一种录制品,乐谱或付印的剧本。

\typeitem{review}
%Reviews of some other work. This is a more specific variant of the \bibtype{article} type. The standard styles will treat this entry type as an alias for \bibtype{article}.
一些其它工作的回顾总结。
这是 \bibtype{article} 类型的一种特殊变体。
标准样式将其视为 \bibtype{article} 的一个别称。

\typeitem{software}
%Computer software.
电脑软件。

\typeitem{standard}
%National and international standards issued by a standards body such as the International Organization for Standardization.
由一个标准组织(例如国际标准组织)发布的国家或国际标准。

\typeitem{video}
%Audiovisual recordings, typically on \acr{DVD}, \acr{VHS} cassette, or similar media. See also \bibtype{movie}.
音像作品,典型的包括 \acr{DVD}、\acr{VHS} 录像带或其它类似媒介。
参考 \bibtype{movie} 类型。

\end{typelist}

\subsection{条目中的域}%\subsection{Entry Fields}
\label{bib:fld}

%This section gives an overview of the fields supported by the \biblatex default data model. See \secref{bib:fld:typ} for an introduction to the data types used by the data model specification and \secref{bib:fld:dat, bib:fld:spc} for the actual field listings.

本节将概略介绍 \biblatex 默认数据模型支持的域。
数据模型规范使用的数据类型简介,见 \secref{bib:fld:typ} 节,
实际使用的域的一览表,见 \secref{bib:fld:dat, bib:fld:spc} 节。

%\subsubsection{Data Types}
\subsubsection{数据类型}
\label{bib:fld:typ}

%In datasources such as a \file{bib} file, all bibliographic data is specified in fields. Some of those fields, for example \bibfield{author} and \bibfield{editor}, may contain a list of items. This list structure is implemented by the \bibtex file format via the keyword <|and|>, which is used to separate the individual items in the list. The \biblatex package implements three distinct data types to handle bibliographic data: name lists, literal lists, and fields. There are also several list and field subtypes and a content type which can be used to semantically distinguish fields which are otherwise not distinguishable on the basis of only their datatype (see \secref{aut:ctm:dm}). This section gives an overview of the data types supported by this package. See \secref{bib:fld:dat, bib:fld:spc} for information about the mapping of the \bibtex file format fields to \biblatex's data types.
在 \file{bib} 文件等数据源中,所有的文献数据都在域中给出。其中一些域,例如 \bibfield{author} 和 \bibfield{editor},可以包含一个项目列表。在 \BibTeX 文件格式中,这种列表结构通过关键词 “|and|” 来分隔列表中的每一项。
\biblatex 宏包实现了三种不同的数据类型来处理文献数据:姓名列表(name list)、文本列表(literal list)和域类型(field)。
此外,还有一些列表和域类型的子类,以及一个内容类型(content type),
用于从语义上区分那些无法仅根据数据类型进行区分的域(见 \secref{aut:ctm:dm} 节)。
本节总结了本宏包所支持的数据类型。\BibTeX 文件格式中的域与 \biblatex 数据类型的对应信息,请参考 \secref{bib:fld:dat, bib:fld:spc} 节。

\begin{description}

%\item[Name lists] are parsed and split up into the individual items at the \texttt{and} delimiter. Each item in the list is then dissected into the name part components: by default the given name, the name prefix (von, van, of, da, de, della, \dots), the family name, and the name suffix (junior, senior, \dots). The valid name parts can be customised by changing the datamodel definition described in \secref{aut:bbx:drv}. Name lists may be truncated in the \file{bib} file with the keyword <\texttt{and others}>. Typical examples of name lists are \bibfield{author} and \bibfield{editor}.

\item[姓名列表(name list)] 根据分隔词 \texttt{and} 将其解析并划分成独立的项。
然后列表中的每一项进一步分解成四个姓名成分:\footnote{
	这是针对西方人名的划分。对于中文来说,姓名无需划分。当然中文名的拼音可以进行对应的划分。——译注}
名(given name,默认值)、姓名前缀(name prefix, 如 von、van、of、da、de、della 等)、姓(family name),以及姓名后缀(name suffix, 如 junior、senior 等)。
可以通过调整数据模型的定义来定制有效的姓名成分,见 \secref{aut:bbx:drv} 节。
在 \file{bib} 文件中,姓名列表可以用关键词“\texttt{and others}”来截短。
典型的姓名列表是 \bibfield{author} 和 \bibfield{editor}。
%(译者注:在bib文件中录入参考文献数据的时候,当某些机构名中含有空格的情况下,最好将整个机构用\{\}包含起来。)

%Name list fields automatically have an \cmd{ifuse*} test created as per the name lists in the default data model (see \secref{aut:aux:tst}). They are also automatically have a \opt{ifuse*} option created which controls labelling and sorting behaviour with the name (see \secref{use:opt:bib:hyb}). \biber supports a customisable set of name parts but currently this is defined to be the same set of parts as supported by traditional \bibtex:

默认的数据模型为每一个姓名列表域自动创建了相应的 \cmd{ifuse*} 测试命令(见 \secref{aut:aux:tst} 节)。
同时也自动创建了一个 \opt{ifuse*} 选项用以控制姓名的标记和排序行为(见 \secref{use:opt:bib:hyb} 节)。
\biber 支持定制姓名成分组合,不过目前定义的姓名成分组合与传统 \BibTeX 支持的相同:

\begin{itemize}
	\item 姓(family name,即<last>部分)%Family name (also known as <last> part)
	\item 名(given name,即<first>部分) %Given name (also known as <first> part)
	\item 前缀(name prefix,即<von>部分)%Name prefix (also known as <von> part)
	\item 后缀(name suffix,即<Jr>部分)%Name suffix (also known as <Jr> part)
\end{itemize}

%The supported list of name parts is defined as a constant list in the default data model using the \cmd{DeclareDatamodelConstant} command (see \ref{aut:ctm:dm}). However, it is not enough to simply add to this list in order to add support for another name part as name parts typically have to be hard coded into bibliography drivers and the backend processing. See the example file \file{93-nameparts.tex} for details on how to define and use custom name parts. Also see \cmd{DeclareUniquenameTemplate} in \secref{aut:cav:amb} for information on how to customise name disambiguation using custom name parts.

在默认数据模型中,支持的姓名成分列表由 \cmd{DeclareDatamodelConstant} 命令定义为一个固定列表(见 \ref{aut:ctm:dm} 节)。然而,由于姓名成分在参考文献驱动(driver)\footnote{参考文献驱动是biblatex中的特有概念,本质上是一个针对具体条目类型组织参考文献数据输出的宏,由\cmd{DeclareBibliographyDriver}命令定义——译注}和后端处理过程中通常需要硬编码,
因此,如果想支持额外的姓名成分,将其简单地添加到姓名成分列表中是不够的。
关于如何定义和使用定制姓名成分的细节,可以参考示例文件 \file{93-nameparts.tex}。
关于如何使用定制姓名成分来消除姓名歧义的信息,参见 \secref{aut:cav:amb} 节中的 \cmd{DeclareUniquenameTemplate} 命令。

%\item[Literal lists] are parsed and split up into the individual items at the \texttt{and} delimiter but not dissected further. Literal lists may be truncated in the \file{bib} file with the keyword <\texttt{and others}>. There are two subtypes:
\item[文本列表(literal list)] 由分隔词 \texttt{and} 划分成独立的项,但各项不再进一步细分。
在 \file{bib} 文件中,文本列表可以用关键词“\texttt{and others}”来截短。
其中又有两个子类型:

\begin{description}

%\item[Literal lists] in the strict sense are handled as described above. The individual items are simply printed as is. Typical examples of such literal lists are \bibfield{publisher} and \bibfield{location}.
\item[(狭义的)文本列表(literal lists in the strict sense)] 按照如上所述进行处理。
各独立的项目就简单如实打印。
典型的狭义文本列表是 \bibfield{publisher} 和 \bibfield{location}。

%\item[Key lists] are a variant of literal lists which may hold printable data or localization keys. For each item in the list, a test is performed to determine whether it is a known localization key (the localization keys defined by default are listed in \secref{aut:lng:key}). If so, the localized string is printed. If not, the item is printed as is. A typical example of a key list is \bibfield{language}.
\item[关键字列表(key list)]  是文本列表的变体,可以包含可打印的数据或本地化关键字。
对于列表中每一项,首先测试它是否为已知的本地化关键字
(本地化关键字的默认定义在 \secref{aut:lng:key} 节中)。
如果是,那么打印本地化字符串;否则这些项就按本身内容如实打印。
典型的关键字列表是 \bibfield{language}。

\end{description}
\end{description}

\begin{description}

%\item[Fields] are usually printed as a whole. There are several subtypes:
\item[域(field)] 通常以整体打印。有如下多种子类型:

\begin{description}

%\item[Literal fields] are printed as is. Typical examples of literal fields are \bibfield{title} and \bibfield{note}.
\item[文本域(literal field)]  会如实打印。
典型的文本域是 \bibfield{title} 和 \bibfield{note}。

%\item[Range fields] consist of one or more ranges where all dashes are normalized and replaced by the command \cmd{bibrangedash}. A range is something optionally followed by one or more dashes optionally followed by some non-dash (e.g. \texttt{5--7}). Any number of consecutive dashes will only yield a single range dash. A typical example of a range field is the \bibfield{pages} field. See also the \cmd{bibrangessep} command which can be used to customise the separator between multiple ranges. Range fields will be skipped and will generate a warning if they do not consist of one or more ranges. You can normalise messy range fields before they are parsed using \cmd{DeclareSourcemap} (see \secref{aut:ctm:map}).

\item[范围域(range field)] 包含了一个或更多范围,其中所有的短划线都规范化用 \cmd{bibrangedash} 命令取代。
一个范围指的是一个非短划线部分后紧跟一个或多个短划线再紧跟一个非短划线部分(比如 \texttt{5--7})。
任意数目的连续短划线都只产生一个表示范围的横线。
典型的范围域是 \bibfield{pages} 域。
也可以参考 \cmd{bibrangessep} 命令,它用于定制多重范围间的分隔符。
如果不包括范围,那么范围域将被忽略并生成警告信息。
如果范围域内容混乱,可以使用 \cmd{DeclareSourcemap} 命令在对其解析之前进行规范化(见 \secref{aut:ctm:map} 节)。

%\item[Integer fields] hold integers which may be converted to ordinals or strings as they are printed. A typical example is the \bibfield{extrayear} or \bibfield{volume} field. Such fields are sorted as integers. \biber makes a (quite serious) effort to map non-arabic representations (roman numerals for example) to integers for sorting purposes.

\item[整数域(integer field)] 包含的整数打印时会转化为序数或者字符串。
典型的例子是 \bibfield{extrayear} 和 \bibfield{volume} 域。
这些域会按照数字进行排序。
出于排序的目的,\biber 会尝试将非阿拉伯数字的表示(例如罗马数字)转成相应的整数。

%\item[Datepart fields] hold unformatted integers which may be converted to ordinals or strings as they are printed. A typical example is the \bibfield{month} field. For every field X of datatype \bibfield{date} in the datamodel, datepart fields are automatically created with the following names: \bibfield{$<$datetype$>$year}, \bibfield{$<$datetype$>$endyear}, \bibfield{$<$datetype$>$month}, \bibfield{$<$datetype$>$endmonth}, \bibfield{$<$datetype$>$day}, \bibfield{$<$datetype$>$endday}, \bibfield{$<$datetype$>$hour}, \bibfield{$<$datetype$>$endhour}, \bibfield{$<$datetype$>$minute}, \bibfield{$<$datetype$>$endminute}, \bibfield{$<$datetype$>$second}, \bibfield{$<$datetype$>$endsecond}, \bibfield{$<$datetype$>$timezone}, \bibfield{$<$datetype$>$endtimezone}.
\item[日期成分域(datepart field)] 保存未格式化的整数,打印时会转化为序数或者字符串。
典型的例子是 \bibfield{month} 域。
在数据模型中,对于每一个数据类型为 \bibfield{date} 的域X,
会自动创建带有如下名称的日期成分域:
\begin{flushleft}
\bibfield{\prm{datetype}year}, \bibfield{\prm{datetype}endyear}, \bibfield{\prm{datetype}month}, \bibfield{\prm{datetype}endmonth}, \bibfield{\prm{datetype}day}, \bibfield{\prm{datetype}endday}, \bibfield{\prm{datetype}hour}, \bibfield{\prm{datetype}endhour}, \bibfield{\prm{datetype}minute}, \bibfield{\prm{datetype}endminute}, \bibfield{\prm{datetype}second}, \bibfield{\prm{datetype}endsecond}, \bibfield{\prm{datetype}timezone}, \bibfield{\prm{datetype}endtimezone}
\end{flushleft}
%$<$datetype$>$ is the string preceding <date> for any datamodel field of \kvopt{datatype}{date}. For example, in the default datamodel, <event>, <orig>, <url> and the empty string <> for the date field \bibfield{date}.
其中,对于任何 \kvopt{datatype}{date} 数据模型域,\prm{datetype} 是在 <date> 之前的前缀字符串。
例如,在默认数据模型中,日期域 \bibfield{date} 可能的前缀字符串包括<event>, <orig>, <url> 和空字符串 <>。

%\item[Date fields] hold a date specification in \texttt{yyyy-mm-ddThh:nn[+|-][hh[:nn]|Z]} format or a date range in \texttt{yyyy-mm-ddThh:nn[+|-][hh[:nn]|Z]/yyyy-mm-ddThh:nn[+|-][hh[:nn]|Z]} format and other formats permitted by \acr{ISO8601-2} Clause 4, level 1, see \secref{bib:use:dat}. Date fields are special in that the date is parsed and split up into its datepart type components. The \bibfield{datepart} components (see above) are automatically defined and recognised when a field of datatype \bibfield{date} is defined in the datamodel. A typical example is the \bibfield{date} field.

\item[日期域(date field)] 保存形如 \texttt{yyyy-mm-ddThh:nn[+|-][hh[:nn]|Z]} 格式的日期,
或者 \texttt{yyyy-mm-ddThh:nn[+|-][hh[:nn]|Z]/yyyy-mm-ddThh:nn[+|-][hh[:nn]|Z]} 格式的日期范围,
和其它\acr{ISO8601-2} 条款4 level 1允许的格式,见 \secref{bib:use:dat} 节。
日期域的特殊之处在于,日期会被解析并分解成各个日期成分。
在数据模型中,当定义了一个数据类型为 \bibfield{date} 的域后,那么相应的 \bibfield{datepart} 成分(见上文)会自动定义并识别。典型的例子是 \bibfield{date} 域。

%\item[Verbatim fields] are processed in verbatim mode and may contain special characters. Typical examples of verbatim fields are \bibfield{file} and \bibfield{doi}.

\item[抄录域(verbatim field)] 在抄录模式下处理,可以包含特殊字符。
典型的抄录域是 \bibfield{file} 和 \bibfield{doi}。

%\item[URI fields] are processed in verbatim mode and may contain special characters. They are also URL-escaped if they don't look like they already are. The typical example of a uri field is \bibfield{url}.
\item[URI 域] 在抄录模式下处理,可以包含特殊字符,也可以进行 URL 转义。
典型的例子是 \bibfield{url} 域。

%\item[Separated value fields] A separated list of literal values. Examples are the \bibfield{keywords} and \bibfield{options} fields. The separator can be configured to be any Perl regular expression via the \opt{xsvsep} option which defaults to the usual \bibtex comma surrounded by optional whitespace.

\item[分隔值域(separated value field)]
被分隔的文本值列表。
例子是 \bibfield{keywords} 和 \bibfield{options} 域。
通过 \opt{xsvsep} 选项可以将分隔符配置成任何Perl正则表达式,
其默认值是通常 \BibTeX 中的(西文)逗号或者逗号加空格。

%\item[Pattern fields] A literal field which must match a particular pattern. An example is the \bibfield{gender} field from \secref{bib:fld:spc}.

\item[模式域(pattern field)] 是必须匹配某一特定模式的文本域。
例如 \secref{bib:fld:spc} 节的 \bibfield{gender} 域。

%\item[Key fields] May hold printable data or localisation keys. A test is performed to determine whether the value of the field is a known localisation key (the localisation keys defined by default are listed in \secref{aut:lng:key}). If so, the localised string is printed. If not, the value is printed as is. A typical example is the \bibfield{type} field.

\item[关键字域(key field)] 可以保存可打印数据或本地化关键字。
使用时,首先测试是否为已知的本地化关键字(本地化关键字的默认定义在 \secref{aut:lng:key} 一节中)。
如果是,就打印本地化字符串;否则,就按本身内容如实打印。
典型的例子是 \bibfield{type} 域。

%\item[Code fields] Holds \tex code.

\item[代码域(code field)] 保存 \TeX\ 代码。

\end{description}
\end{description}

%\subsubsection{Data Fields}
\subsubsection{数据域}
\label{bib:fld:dat}

%The fields listed in this section are the regular ones holding printable data in the default data model. The name on the left is the default data model name of the field as used by \biblatex and its backend. The \biblatex data type is given to the right of the name. See \secref{bib:fld:typ} for explanation of the various data types.

本节所列的域是在默认数据模型中保存可打印数据的常规域。
下面的列表中,左边的名称是域的默认数据模型名,在 \biblatex 和后端使用。
名称右侧则是相应的 \biblatex 数据类型。
不同数据类型的解释请参考 \secref{bib:fld:typ} 节。

%Some fields are marked as <label> fields which means that they are often used as abbreviation labels when printing bibliography lists in the sense of section \secref{use:bib:biblist}. \biblatex automatically creates supporting macros for such fields. See \secref{use:bib:biblist}.

一些域标记为“label”域,
这表示这些域通常用于缩写标签(abbreviation labels),当打印文献列表(bibliography lists)时(见\secref{use:bib:biblist} 节的内容)。
\biblatex 会自动创建支持这些域的宏,详见 \secref{use:bib:biblist}。

\begin{fieldlist}

%\fielditem{abstract}{literal}

%This field is intended for recording abstracts in a \file{bib} file, to be printed by a special bibliography style. It is not used by all standard bibliography styles.

\fielditem{abstract}{文本}

该域保存\file{bib} 文件中记录的摘要,某些特殊的文献著录样式会将其打印出来。
但所有的标准样式中都不使用。

%\fielditem{addendum}{literal}

%Miscellaneous bibliographic data to be printed at the end of the entry. This is similar to the \bibfield{note} field except that it is printed at the end of the bibliography entry.

\fielditem{addendum}{文本}

在条目末尾打印的杂项文献数据。
它与 \bibfield{note} 域类似,差别在于它是在文献条目末尾打印。

%\listitem{afterword}{name}

%The author(s) of an afterword to the work. If the author of the afterword is identical to the \bibfield{editor} and\slash or \bibfield{translator}, the standard styles will automatically concatenate these fields in the bibliography. See also \bibfield{introduction} and \bibfield{foreword}.

\listitem{afterword}{姓名}

后记的作者。如果后记作者与 \bibfield{editor} 或 \bibfield{translator} 相同,
那么在参考文献表中标准样式会自动把这些域关联起来。
参考 \bibfield{introduction} 域和 \bibfield{foreword} 域。

%\fielditem{annotation}{literal}

%This field may be useful when implementing a style for annotated bibliographies. It is not used by all standard bibliography styles. Note that this field is completely unrelated to \bibfield{annotator}. The \bibfield{annotator} is the author of annotations which are part of the work cited.

\fielditem{annotation}{文本}

该域在实现带注释的参考文献著录样式时很有用。
所有的标准样式都不使用。
请注意,该域与 \bibfield{annotator} 域毫无关系,后者是释文(被引用著作的一部分)的作者。

%\listitem{annotator}{name}

%The author(s) of annotations to the work. If the annotator is identical to the \bibfield{editor} and\slash or \bibfield{translator}, the standard styles will automatically concatenate these fields in the bibliography. See also \bibfield{commentator}.

\listitem{annotator}{姓名}

释文的作者。如果与 \bibfield{editor} 或 \bibfield{translator} 相同,
那么在参考文献表中标准样式会自动把这些域关联起来。
参考 \bibfield{commentator} 域。

%\listitem{author}{name}

%The author(s) of the \bibfield{title}.

\listitem{author}{姓名}

\bibfield{title} 的作者。

%\fielditem{authortype}{key}

%The type of author. This field will affect the string (if any) used to introduce the author. Not used by the standard bibliography styles.

\fielditem{authortype}{关键字}

作者的类型。
该域会影响介绍作者的字符串。
标准文献样式不使用。

%\listitem{bookauthor}{name}

%The author(s) of the \bibfield{booktitle}.

\listitem{bookauthor}{姓名}

\bibfield{booktitle} 的作者。

%\fielditem{bookpagination}{key}

%If the work is published as part of another one, this is the pagination scheme of the enclosing work, \ie \bibfield{bookpagination} relates to \bibfield{pagination} like \bibfield{booktitle} to \bibfield{title}. The value of this field will affect the formatting of the \bibfield{pages} and \bibfield{pagetotal} fields. The key should be given in the singular form. Possible keys are \texttt{page}, \texttt{column}, \texttt{line}, \texttt{verse}, \texttt{section}, and \texttt{paragraph}. See also \bibfield{pagination} as well as \secref{bib:use:pag}.
\fielditem{bookpagination}{关键字}

如果当前作品是另一个大作品的一部分,该域表示包含当前作品的大作品的分页格式。
即,\bibfield{bookpagination} 与 \bibfield{pagination} 的关系如同 \bibfield{booktitle} 之于 \bibfield{title}的关系。
该域的值会影响 \bibfield{pages} 和 \bibfield{pagetotal} 域的格式。
关键字应当是单数形式。可能的关键字包括 \texttt{page}、\texttt{column}、\texttt{line}、\texttt{verse}、\texttt{section} 和 \texttt{paragraph} 等。参考 \bibfield{pagination} 域以及 \secref{bib:use:pag} 节。

%\fielditem{booksubtitle}{literal}

%The subtitle related to the \bibfield{booktitle}. If the \bibfield{subtitle} field refers to a work which is part of a larger publication, a possible subtitle of the main work is given in this field. See also \bibfield{subtitle}.

\fielditem{booksubtitle}{文本}

\bibfield{booktitle} 的副标题。
如果说 \bibfield{subtitle} 域指的是一个大出版物中一小部分作品的副标题,
那么该域则给出了整个大作品的副标题。参考 \bibfield{subtitle}。

%\fielditem{booktitle}{literal}

%If the \bibfield{title} field indicates the title of a work which is part of a larger publication, the title of the main work is given in this field. See also \bibfield{title}.

\fielditem{booktitle}{文本}

如果 \bibfield{title} 域指的是一个大出版物中的一小部分作品的标题,那么该域则给出了整个大作品的标题。
参考 \bibfield{title}。

%\fielditem{booktitleaddon}{literal}

%An annex to the \bibfield{booktitle}, to be printed in a different font.

\fielditem{booktitleaddon}{文本}

\bibfield{booktitle} 的附语,会用不同的字体打印。

%\fielditem{chapter}{literal}

%A chapter or section or any other unit of a work.

\fielditem{chapter}{文本}

作品的章节或其它单元。

%\listitem{commentator}{name}

%The author(s) of a commentary to the work. Note that this field is intended for commented editions which have a commentator in addition to the author. If the work is a stand"=alone commentary, the commentator should be given in the \bibfield{author} field. If the commentator is identical to the \bibfield{editor} and\slash or \bibfield{translator}, the standard styles will automatically concatenate these fields in the bibliography. See also \bibfield{annotator}.

\listitem{commentator}{姓名}

作品评论的作者。
请注意,该域用于那种带评论的作品版本,即,在作者之外还有一位评论作者。
如果作品是独立的评论,那么评论作者应该在 \bibfield{author} 域中给出。
如果评论作者与 \bibfield{editor} 或 \bibfield{translator} 相同,
那么在参考文献中标准样式会自动将这些域关联起来。
参考 \bibfield{annotator} 域。

%\fielditem{date}{date}

%The publication date. See also \bibfield{month} and \bibfield{year} as well as \secref{bib:use:dat}.

\fielditem{date}{日期}

出版日期。参考 \bibfield{month} 和 \bibfield{year} 域以及 \secref{bib:use:dat} 节。

%\fielditem{doi}{verbatim}

%The Digital Object Identifier of the work.

\fielditem{doi}{抄录}

作品的数字对象标识符(Digital Object Identifier,  DOI)。

%\fielditem{edition}{integer or literal}

%The edition of a printed publication. This must be an integer, not an ordinal. Don't say |edition={First}| or |edition={1st}| but |edition={1}|. The bibliography style converts this to a language dependent ordinal. It is also possible to give the edition as a literal string, for example «Third, revised and expanded edition».

\fielditem{edition}{整数或文本}

出版物的版次。
这必须是整数而不是序数。
不要用 |edition={First}| 或 |edition={1st}|,而要用 |edition={1}|。
文献样式会将其转为跟语言相关的序数。
也可以用文本字符串表示版次,例如“Third, revised and expanded edition”。

%\listitem{editor}{name}

%The editor(s) of the \bibfield{title}, \bibfield{booktitle}, or \bibfield{maintitle}, depending on the entry type. Use the \bibfield{editortype} field to specify the role if it is different from <\texttt{editor}>. See \secref{bib:use:edr} for further hints.

\listitem{editor}{姓名}

\bibfield{title}、\bibfield{booktitle} 或者 \bibfield{maintitle} 的编者,这取决于条目类型。
如果不同于真正的“\texttt{editor}”角色,可使用 \bibfield{editortype} 域来确定具体的角色。
更多提示参考 \secref{bib:use:edr} 节。

%\listitem{editora}{name}

%A secondary editor performing a different editorial role, such as compiling, redacting, etc. Use the \bibfield{editoratype} field to specify the role. See \secref{bib:use:edr} for further hints.

\listitem{editora}{姓名}

次要编者,执行汇集、编校等不同编辑任务。
可使用 \bibfield{editoratype} 域来指定具体的角色。
更多提示参考 \secref{bib:use:edr} 节。

%\listitem{editorb}{name}

%Another secondary editor performing a different role. Use the \bibfield{editorbtype} field to specify the role. See \secref{bib:use:edr} for further hints.

\listitem{editorb}{姓名}

执行不同任务的另一类次要编者。
可使用 \bibfield{editorbtype} 域来指定具体的角色。
更多提示参考 \secref{bib:use:edr} 节。

%\listitem{editorc}{name}

%Another secondary editor performing a different role. Use the \bibfield{editorctype} field to specify the role. See \secref{bib:use:edr} for further hints.

\listitem{editorc}{姓名}

执行不同编辑任务的另一类次要编者。
可使用 \bibfield{editorctype} 域来指定具体的角色。
更多提示参考 \secref{bib:use:edr} 节。

%\fielditem{editortype}{key}

%The type of editorial role performed by the \bibfield{editor}. Roles supported by default are \texttt{editor}, \texttt{compiler}, \texttt{founder}, \texttt{continuator}, \texttt{redactor}, \texttt{reviser}, \texttt{collaborator}. The role <\texttt{editor}> is the default. In this case, the field is omissible. See \secref{bib:use:edr} for further hints.

\fielditem{editortype}{关键字}

\bibfield{editor} 执行的编者角色类型。
默认支持的角色包括 \texttt{editor}、\texttt{compiler}、\texttt{founder}、\texttt{continuator}, \texttt{redactor}、\texttt{reviser} 和 \texttt{collaborator}。
默认值是“\texttt{editor}”,此时该域可以省略。
更多提示参考 \secref{bib:use:edr} 节。

%\fielditem{editoratype}{key}

%Similar to \bibfield{editortype} but referring to the \bibfield{editora} field. See \secref{bib:use:edr} for further hints.

\fielditem{editoratype}{关键字}

类似于 \bibfield{editortype} 但对应的是 \bibfield{editora} 域。
更多提示参考 \secref{bib:use:edr} 节。

%\fielditem{editorbtype}{key}

%Similar to \bibfield{editortype} but referring to the \bibfield{editorb} field. See \secref{bib:use:edr} for further hints.

\fielditem{editorbtype}{关键字}

类似于 \bibfield{editortype} 但对应的是 \bibfield{editorb} 域。
更多提示参考 \secref{bib:use:edr} 节。

%\fielditem{editorctype}{key}

%Similar to \bibfield{editortype} but referring to the \bibfield{editorc} field. See \secref{bib:use:edr} for further hints.

\fielditem{editorctype}{关键字}

类似于 \bibfield{editortype} 但对应的是 \bibfield{editorc} 域。
更多提示参考 \secref{bib:use:edr} 节。

%\fielditem{eid}{literal}

%The electronic identifier of an \bibtype{article}.

\fielditem{eid}{文本}

\bibtype{article} 的电子标识符(electronic identifier)。

%\fielditem{entrysubtype}{literal}

%This field, which is not used by the standard styles, may be used to specify a subtype of an entry type. This may be useful for bibliography styles which support a finer"=grained set of entry types.

\fielditem{entrysubtype}{文本}

该域用于指定一个条目类型的子类型。
它不会在标准样式中使用,但可用于支持更细化的条目类型集的参考文献样式。

%\fielditem{eprint}{verbatim}

%The electronic identifier of an online publication. This is roughly comparable to a \acr{doi} but specific to a certain archive, repository, service, or system. See \secref{use:use:epr} for details. Also see \bibfield{eprinttype} and \bibfield{eprintclass}.

\fielditem{eprint}{抄录}

在线出版物的电子标识符。
它大致相当于 \acr{doi},但专门针对某个档案、资源库、服务或系统。
参考 \secref{use:use:epr} 节以及 \bibfield{eprinttype} 和 \bibfield{eprintclass} 域。

%\fielditem{eprintclass}{literal}

%Additional information related to the resource indicated by the \bibfield{eprinttype} field. This could be a section of an archive, a path indicating a service, a classification of some sort, etc. See \secref{use:use:epr} for details. Also see \bibfield{eprint} and \bibfield{eprinttype}.

\fielditem{eprintclass}{文本}

域 \bibfield{eprinttype} 域指明的资源相关的附加信息。
它可以是档案的一部分、标示服务的路径、排序的某个分类等等。
参考 \secref{use:use:epr} 节以及 \bibfield{eprint} 和 \bibfield{eprinttype} 域。

%\fielditem{eprinttype}{literal}

%The type of \bibfield{eprint} identifier, \eg the name of the archive, repository, service, or system the \bibfield{eprint} field refers to. See \secref{use:use:epr} for details. Also see \bibfield{eprint} and \bibfield{eprintclass}.

\fielditem{eprinttype}{文本}

\bibfield{eprint} 标识符的类型,例如 \bibfield{eprint} 域所指的档案、资源库、服务或系统的名称。
参考 \secref{use:use:epr} 节以及 \bibfield{eprint} 和 \bibfield{eprintclass} 域。

%\fielditem{eventdate}{date}

%The date of a conference, a symposium, or some other event in \bibtype{proceedings} and \bibtype{inproceedings} entries. This field may also be useful for the custom types listed in \secref{bib:typ:ctm}. See also \bibfield{eventtitle} and \bibfield{venue} as well as \secref{bib:use:dat}.

\fielditem{eventdate}{日期}

\bibtype{proceedings} 和 \bibtype{inproceedings} 条目中的会议、研讨会或其它活动的日期。
该域还可用于 \secref{bib:typ:ctm} 一节所列的定制类型。
参考 \bibfield{eventtitle} 和 \bibfield{venue} 域以及 \secref{bib:use:dat} 节。

%\fielditem{eventtitle}{literal}

%The title of a conference, a symposium, or some other event in \bibtype{proceedings} and \bibtype{inproceedings} entries. This field may also be useful for the custom types listed in \secref{bib:typ:ctm}. Note that this field holds the plain title of the event. Things like «Proceedings of the Fifth XYZ Conference» go into the \bibfield{titleaddon} or \bibfield{booktitleaddon} field, respectively. See also \bibfield{eventdate} and \bibfield{venue}.

\fielditem{eventtitle}{文本}

\bibtype{proceedings} 和 \bibtype{inproceedings} 条目中的会议、研讨会或其它活动的标题。
该域还可以用于在 \secref{bib:typ:ctm} 一节所列的定制类型。
请注意,该域保存事件的普通标题。而诸如“Proceedings of the Fifth XYZ Conference”之类的信息会归入 \bibfield{titleaddon} 或 \bibfield{booktitleaddon} 域。参考 \bibfield{eventdate} 和 \bibfield{venue} 域。

%\fielditem{eventtitleaddon}{literal}

%An annex to the \bibfield{eventtitle} field. Can be used for known event acronyms, for example.

\fielditem{eventtitleaddon}{文本}

\bibfield{eventtitle} 域的附语。
例如可以用于已知活动的首字母缩略词。

%\fielditem{file}{verbatim}

%A local link to a \acr{pdf} or other version of the work. Not used by the standard bibliography styles.

\fielditem{file}{抄录}

某个作品的 \acr{pdf} 或其它版本的本地链接。
标准样式中不使用。

%\listitem{foreword}{name}

%The author(s) of a foreword to the work. If the author of the foreword is identical to the \bibfield{editor} and\slash or \bibfield{translator}, the standard styles will automatically concatenate these fields in the bibliography. See also \bibfield{introduction} and \bibfield{afterword}.

\listitem{foreword}{姓名}

作品前言的作者。
如果前言的作者与 \bibfield{editor} 或 \bibfield{translator} 相同,
那么在参考文献表中标准样式会自动将其与这些域关联起来。
参考 \bibfield{introduction} 和 \bibfield{afterword} 域。

%\listitem{holder}{name}

%The holder(s) of a \bibtype{patent}, if different from the \bibfield{author}. Not that corporate holders need to be wrapped in an additional set of braces, see \secref{bib:use:inc} for details. This list may also be useful for the custom types listed in \secref{bib:typ:ctm}.

\listitem{holder}{名称}

\bibtype{patent} 的持有者(如果与 \bibfield{author} 不同的话)。
注意,持有者是一个集体(机构)时,需要将其放在额外的花括号内,参考 \secref{bib:use:inc} 节。
该域可以用于 \secref{bib:typ:ctm} 节所列的定制类型中。

%\fielditem{howpublished}{literal}

%A publication notice for unusual publications which do not fit into any of the common categories.

\fielditem{howpublished}{文本}

不能划归任何常见类型的非常规出版物的出版通告。

%\fielditem{indextitle}{literal}

%A title to use for indexing instead of the regular \bibfield{title} field. This field may be useful if you have an entry with a title like «An Introduction to \dots» and want that indexed as «Introduction to \dots, An». Style authors should note that \biblatex automatically copies the value of the \bibfield{title} field to \bibfield{indextitle} if the latter field is undefined.
\fielditem{indextitle}{文本}

替代常规的 \bibfield{title} 域在索引中使用的标题。
如果你有一个带有“An Introduction to~\dots”之类标题的条目,并且想索引为“Introduction to~\dots, An”,那么就可以使用该域。样式作者需要注意,如果 \bibfield{indextitle} 未定义,那么 \biblatex 会自动将 \bibfield{title} 域的值复制给 \bibfield{indextitle}。

%\listitem{institution}{literal}

%The name of a university or some other institution, depending on the entry type. Traditional \bibtex uses the field name \bibfield{school} for theses, which is supported as an alias. See also \secref{bib:fld:als, bib:use:and}.

\listitem{institution}{文本}

大学或其它研究机构的名称,这取决于条目类型。
而传统上,\BibTeX 使用 \bibfield{school} 域来表示这些信息。
本宏包也支持 \bibfield{school},但只作为本域的别名。
另见 \secref{bib:fld:als, bib:use:and} 节。

%\listitem{introduction}{name}

%The author(s) of an introduction to the work. If the author of the introduction is identical to the \bibfield{editor} and\slash or \bibfield{translator}, the standard styles will automatically concatenate these fields in the bibliography. See also \bibfield{foreword} and \bibfield{afterword}.

\listitem{introduction}{姓名}

作品导论的作者。
如果导论的作者与 \bibfield{editor} 或 \bibfield{translator} 相同,
那么在参考文献表中标准样式就会自动将这些域关联起来。
参考 \bibfield{foreword} 和 \bibfield{afterword} 域。

%\fielditem{isan}{literal}

%The International Standard Audiovisual Number of an audiovisual work. Not used by the standard bibliography styles.

\fielditem{isan}{文本}

音像作品的音像数码国际标准(International Standard Audiovisual Number,  ISAN)。
不会在标准文献样式中使用。

%\fielditem{isbn}{literal}

%The International Standard Book Number of a book.

\fielditem{isbn}{文本}

书籍的国际标准书号(International Standard Book Number, ISBN)。

%\fielditem{ismn}{literal}

%The International Standard Music Number for printed music such as musical scores. Not used by the standard bibliography styles.

\fielditem{ismn}{文本}

乐谱等发行音乐作品的国际标准音乐作品编码(International Standard Music Number,  ISMN)。

%\fielditem{isrn}{literal}

%The International Standard Technical Report Number of a technical report.

\fielditem{isrn}{文本}

技术报告的国际标准技术报告编码(International Standard Technical Report Number,  ISRN)。

%\fielditem{issn}{literal}

%The International Standard Serial Number of a periodical.

\fielditem{issn}{文本}

连续出版物的国际标准连续出版物号(International Standard Serial Number,  ISSN)。

%\fielditem{issue}{literal}

%The issue of a journal. This field is intended for journals whose individual issues are identified by a designation such as <Spring> or <Summer> rather than the month or a number. Since the placement of \bibfield{issue} is similar to \bibfield{month} and \bibfield{number}, this field may also be useful with double issues and other special cases. See also \bibfield{month}, \bibfield{number}, and \secref{bib:use:iss}.

\fielditem{issue}{文本}

期刊的期号。
该域适用于期号由“Spring”或“Summer”等名称而不是月份或数字确定的期刊。
由于 \bibfield{issue} 的位置与 \bibfield{month} 和 \bibfield{number} 类似,
该域也可用于合期或其它特殊场合\footnote{例如增刊、特刊等。——译注}。
参考 \bibfield{month} 和 \bibfield{number} 域以及 \secref{bib:use:iss} 节。

%\fielditem{issuesubtitle}{literal}

%The subtitle of a specific issue of a journal or other periodical.

\fielditem{issuesubtitle}{文本}

期刊或其它连续出版物中某一期的副标题。

%\fielditem{issuetitle}{literal}

%The title of a specific issue of a journal or other periodical.

\fielditem{issuetitle}{文本}

期刊或其它连续出版物中某一期的标题。

%\fielditem{iswc}{literal}

%The International Standard Work Code of a musical work. Not used by the standard bibliography styles.

\fielditem{iswc}{文本}

音乐作品的国际标准作品号(International Standard Work Code,  ISWC)。
标准文献样式中不使用。

%\fielditem{journalsubtitle}{literal}

%The subtitle of a journal, a newspaper, or some other periodical.

\fielditem{journalsubtitle}{文本}

期刊、报纸或其它连续出版物的副标题。

%\fielditem{journaltitle}{literal}

%The name of a journal, a newspaper, or some other periodical.

\fielditem{journaltitle}{文本}

期刊、报纸或其它连续出版物的标题。

%\fielditem{label}{literal}

%A designation to be used by the citation style as a substitute for the regular label if any data required to generate the regular label is missing. For example, when an author"=year citation style is generating a citation for an entry which is missing the author or the year, it may fall back to \bibfield{label}. See \secref{bib:use:key} for details. Note that, in contrast to \bibfield{shorthand}, \bibfield{label} is only used as a fallback. See also \bibfield{shorthand}.

\fielditem{label}{文本}

在标注样式中,如果生成常规标签的所需数据均缺失,那么该域的内容可用来代替常规标签。
例如,当|作者年制|标注样式要生成某个条目的标签,但该条目的作者或年份缺失,那么会使用后备的 \bibfield{label}。
详见 \secref{bib:use:key} 节。
请注意,与 \bibfield{shorthand} 域相反,\bibfield{label} 只是作为后备而使用。
另可参见 \bibfield{shorthand}。

%\listitem{language}{key}

%The language(s) of the work. Languages may be specified literally or as localisation keys. If localisation keys are used, the prefix \texttt{lang} is omissible. See also \bibfield{origlanguage} and compare \bibfield{langid} in \secref{bib:fld:spc}.

\listitem{language}{关键字}

作品的语言。
语言可以由文本内容或者本地化关键字指定。
如果使用本地化关键字,那么前缀 \opt{lang} 可省略。
参考 \bibfield{origlanguage} 域并比较 \secref{bib:fld:spc} 节中的 \bibfield{langid}。

%\fielditem{library}{literal}

%This field may be useful to record information such as a library name and a call number. This may be printed by a special bibliography style if desired. Not used by the standard bibliography styles.

\fielditem{library}{文本}

该域可用于记录图书馆名称或书架号码等信息。
某些特殊的参考文献样式可能需要将其打印出来。
但在标准样式中不使用。

%\listitem{location}{literal}

%The place(s) of publication, \ie the location of the \bibfield{publisher} or \bibfield{institution}, depending on the entry type. Traditional \bibtex uses the field name \bibfield{address}, which is supported as an alias. See also \secref{bib:fld:als, bib:use:and}. With \bibtype{patent} entries, this list indicates the scope of a patent. This list may also be useful for the custom types listed in \secref{bib:typ:ctm}.

\listitem{location}{文本}

出版地,即 \bibfield{publisher} 或 \bibfield{institution} (取决于条目类型)的所在地。
传统上 \BibTeX 使用 \bibfield{address} 域,本宏包也支持 \bibfield{address},但只作为本域的别名。
参考 \secref{bib:fld:als, bib:use:and} 几节。
在 \bibtype{patent} 条目里,该列表用于表示专利的权利范围。
该文本列表可用于 \secref{bib:typ:ctm} 中的定制类型。

%\fielditem{mainsubtitle}{literal}

%The subtitle related to the \bibfield{maintitle}. See also \bibfield{subtitle}.

\fielditem{mainsubtitle}{文本}

对应于 \bibfield{maintitle} 的副标题。参考 \bibfield{subtitle} 域。

%\fielditem{maintitle}{literal}

%The main title of a multi"=volume book, such as \emph{Collected Works}. If the \bibfield{title} or \bibfield{booktitle} field indicates the title of a single volume which is part of multi"=volume book, the title of the complete work is given in this field.

\fielditem{maintitle}{文本}

多卷本书籍(例如\emph{著作集})的主标题。
如果说 \bibfield{title} 或 \bibfield{booktitle} 域指的是多卷本中某一单卷的标题,
那么该域则给出了全集的标题。

%\fielditem{maintitleaddon}{literal}

%An annex to the \bibfield{maintitle}, to be printed in a different font.

\fielditem{maintitleaddon}{文本}

\bibfield{maintitle} 的附言,会用不同的字体打印。

%\fielditem{month}{datepart}

%The publication month. This must be an integer, not an ordinal or a string. Don't say |month={January}| but |month={1}|. The bibliography style converts this to a language dependent string or ordinal where required. This field is a literal field only when given explicitly in the data (for plain \bibtex compatibility for example). It is however better to use the \bibfield{date} field as this supports many more features. See also \bibfield{date} as well as \secref{bib:use:iss, bib:use:dat}.

\fielditem{month}{日期成分}

出版月份。必须是整数,而不能是序数或字符串。例如,使用 |month={1}| 而不是 |month={January}|。文献样式会在需要时将它转换为语言相关的字符串或序数。当显式给出数据时(例如为兼容原始的\bibtex ),该域转变为一个文本域。然而最好还是使用\bibfield{date},因为它能支持更多的功能。参考 \bibfield{date} 以及 \secref{bib:use:iss, bib:use:dat}。

%\fielditem{nameaddon}{literal}

%An addon to be printed immediately after the author name in the bibliography. Not used by the standard bibliography styles. This field may be useful to add an alias or pen name (or give the real name if the pseudonym is commonly used to refer to that author).

\fielditem{nameaddon}{文本}

参考文献中紧随作者名之后输出的插入语。
标准文献样式中不使用。
该域可用于添加别名或笔名(或者给出原名,如果常用化名来表示作者话)。

%\fielditem{note}{literal}

%Miscellaneous bibliographic data which does not fit into any other field. The \bibfield{note} field may be used to record bibliographic data in a free format. Publication facts such as «Reprint of the edition London 1831» are typical candidates for the \bibfield{note} field. See also \bibfield{addendum}.

\fielditem{note}{文本}

不可归类于其它域的杂项文献数据。
\bibfield{note} 域可以用于记录自由格式的文献数据。
\bibfield{note} 域包含一些典型信息,例如“Reprint of the edition London 1831” 这样出版信息。
另见 \bibfield{addendum} 域。

%\fielditem{number}{integer}

%The number of a journal or the volume\slash number of a book in a \bibfield{series}. See also \bibfield{issue} as well as \secref{bib:use:ser, bib:use:iss}. With \bibtype{patent} entries, this is the number or record token of a patent or patent request. It is expected to be an integer, not necessarily in arabic numerals since \biber will automatically from roman numerals or arabic letter to integers internally for sorting purposes.

\fielditem{number}{整数}

期刊的期数或者 \bibfield{series} 丛书中某本书的卷数\slash 期数。
另见 \bibfield{issue} 域以及 \secref{bib:use:ser, bib:use:iss} 节。
在 \bibtype{patent} 条目中,这是专利或专利申请号或登记号。
其值应该是一个整数,但不必是阿拉伯数字的形式,因为 \biber 为了排序会自动将罗马数字或者阿拉伯字符转成整数。

%\listitem{organization}{literal}

%The organization(s) that published a \bibtype{manual} or an \bibtype{online} resource, or sponsored a conference. See also \secref{bib:use:and}.

\listitem{organization}{文本}

出版 \bibtype{manual} 或 \bibtype{online} 资源,以及主办会议的组织。
另可参考 \secref{bib:use:and} 节。

%\fielditem{origdate}{date}

%If the work is a translation, a reprint, or something similar, the publication date of the original edition. Not used by the standard bibliography styles. See also \bibfield{date}.

\fielditem{origdate}{日期}

如果作品是译作、重印或其它类似情况,该域指的是初版日期。
在标准文献样式中不使用。另可参考 \bibfield{date} 域。

%\fielditem{origlanguage}{key} %v3.7

%If the work is a translation, the language of the original work. See also \bibfield{language}.

%\listitem{origlanguage}{key} %v3.9

%If the work is a translation, the language(s) of the original work. See also \bibfield{language}.

\listitem{origlanguage}{关键字}

如果作品是译作,该域指的是原作使用的语言。
另可参考 \bibfield{language} 域。

%\listitem{origlocation}{literal}

%If the work is a translation, a reprint, or something similar, the \bibfield{location} of the original edition. Not used by the standard bibliography styles. See also \bibfield{location} and \secref{bib:use:and}.

\listitem{origlocation}{文本}

如果作品是译作、重印或其它类似情况,该域指的是初版的 \bibfield{location}。
标准文献样式不使用。
另可参考 \bibfield{location} 域和 \secref{bib:use:and} 节。

%\listitem{origpublisher}{literal}

%If the work is a translation, a reprint, or something similar, the \bibfield{publisher} of the original edition. Not used by the standard bibliography styles. See also \bibfield{publisher} and \secref{bib:use:and}.

\listitem{origpublisher}{文本}

如果作品是译作、重印或其它类似情况,该域指的是初版的 \bibfield{publisher}。
在标准文献样式中不使用。参考 \bibfield{publisher} 域和 \secref{bib:use:and} 节。

%\fielditem{origtitle}{literal}

%If the work is a translation, the \bibfield{title} of the original work. Not used by the standard bibliography styles. See also \bibfield{title}.

\fielditem{origtitle}{文本}

如果作品是译作,该域指的是原作的 \bibfield{title}。
标准文献样式不使用。另可参考 \bibfield{title} 域。

%\fielditem{pages}{range}

%One or more page numbers or page ranges. If the work is published as part of another one, such as an article in a journal or a collection, this field holds the relevant page range in that other work. It may also be used to limit the reference to a specific part of a work (a chapter in a book, for example).

\fielditem{pages}{范围}

一个或多个页码数或页码范围。
如果当前作品是另一个大作品的一部分,例如期刊或文集析出中的文章,
该域指的是当前作品在相关大作品中的页码范围。
它也可以用于限定著作中某一特定部分(例如一本书中的一章)。

%\fielditem{pagetotal}{literal}

%The total number of pages of the work.

\fielditem{pagetotal}{文本}

作品的总页码数。

%\fielditem{pagination}{key}

%The pagination of the work. The value of this field will affect the formatting the \prm{postnote} argument to a citation command. The key should be given in the singular form. Possible keys are \texttt{page}, \texttt{column}, \texttt{line}, \texttt{verse}, \texttt{section}, and \texttt{paragraph}. See also \bibfield{bookpagination} as well as \secref{bib:use:pag, use:cav:pag}.

\fielditem{pagination}{关键字}

作品的页码标记格式。
该域的值会影响标注命令 \prm{postnote} 参数的格式。
关键字应当以单数的形式给出。
可能的关键字包括 \texttt{page}、\texttt{column}、\texttt{line}、\texttt{verse}、\texttt{section} 和 \texttt{paragraph}。
另可参考 \bibfield{bookpagination} 域以及 \secref{bib:use:pag, use:cav:pag} 节。

%\fielditem{part}{literal}

%The number of a partial volume. This field applies to books only, not to journals. It may be used when a logical volume consists of two or more physical ones. In this case the number of the logical volume goes in the \bibfield{volume} field and the number of the part of that volume in the \bibfield{part} field. See also \bibfield{volume}.

\fielditem{part}{文本}

某一分卷的编号。该域只用于书籍而不能用于期刊。
它可以用于一个逻辑卷包括两个或更多实际卷的情形。
如果这样的话,逻辑卷册的编号由 \bibfield{volume} 给出,而该逻辑卷的分卷编号由 \bibfield{part} 给出。
另可参考 \bibfield{volume} 域。

%\listitem{publisher}{literal}

%The name(s) of the publisher(s). See also \secref{bib:use:and}.

\listitem{publisher}{文本}

%The name(s) of the publisher(s). See also \secref{bib:use:and}.
出版者的名称。参考 \secref{bib:use:and} 节。

%\fielditem{pubstate}{key}

%The publication state of the work, \eg\ <in press>. See \secref{aut:lng:key:pst} for known publication states.

\fielditem{pubstate}{关键字}

作品的出版状态,例如“in press”。
已知的出版状态请参考 \secref{aut:lng:key:pst}  节。

%\fielditem{reprinttitle}{literal}

%The title of a reprint of the work. Not used by the standard styles.

\fielditem{reprinttitle}{文本}

作品重印版的标题。标准样式中不使用。

%\fielditem{series}{literal}

%The name of a publication series, such as «Studies in \dots», or the number of a journal series. Books in a publication series are usually numbered. The number or volume of a book in a series is given in the \bibfield{number} field. Note that the \bibtype{article} entry type makes use of the \bibfield{series} field as well, but handles it in a special way. See \secref{bib:use:ser} for details.

\fielditem{series}{文本}

丛书的名称,例如“Studies in \dots”,或者期刊系列的编号。
系列出版的丛书通常带有编号。
其编号或者卷号由 \bibfield{number} 域给出。
请注意,\bibtype{article} 条目类型也使用 \bibfield{series} 域,但是以一种特别的方式处理。
详见 \secref{bib:use:ser} 节。

%\listitem{shortauthor}{name\LFMark}

%The author(s) of the work, given in an abbreviated form. This field is mainly intended for abbreviated forms of corporate authors, see \secref{bib:use:inc} for details.

\listitem{shortauthor}{姓名\LFMark}

作者的缩写形式。
该域主要用于集体作者的缩写形式。
参考 \secref{bib:use:inc} 节。

%\listitem{shorteditor}{name\LFMark}

%The editor(s) of the work, given in an abbreviated form. This field is mainly intended for abbreviated forms of corporate editors, see \secref{bib:use:inc} for details.

\listitem{shorteditor}{姓名\LFMark}

编者的缩写形式。
该域主要用于集体编者的缩写形式。
参考 \secref{bib:use:inc} 节。

%\fielditem{shorthand}{literal\LFMark}

%A special designation to be used by the citation style instead of the usual label. If defined, it overrides the default label. See also \bibfield{label}.

\fielditem{shorthand}{文本\LFMark}

在标注样式中,用于替代常规标签的特殊标签。如果有定义,那么它会覆盖默认的标签。
另可参考 \bibfield{label} 域。

%\fielditem{shorthandintro}{literal}

%The verbose citation styles which comes with this package use a phrase like «henceforth cited as [shorthand]» to introduce shorthands on the first citation. If the \bibfield{shorthandintro} field is defined, it overrides the standard phrase. Note that the alternative phrase must include the shorthand.

\fielditem{shorthandintro}{文本}

本宏包附带的完整信息标注(verbose)样式会在首次引用时使用类似“henceforth cited as [shorthand]”这样的短语来引入[shorthand]。
如果 \bibfield{shorthandintro} 域有定义,它将覆盖标准短语。
请注意,使用的备选短语必须包含 shorthand。

%\fielditem{shortjournal}{literal\LFMark}

%A short version or an acronym of the \bibfield{journaltitle}. Not used by the standard bibliography styles.

\fielditem{shortjournal}{文本\LFMark}

\bibfield{journaltitle} 的缩写版本或其首字母缩略语。
标准文献样式中不使用。

%\fielditem{shortseries}{literal\LFMark}

%A short version or an acronym of the \bibfield{series} field. Not used by the standard bibliography styles.

\fielditem{shortseries}{文本\LFMark}

\bibfield{series} 的缩写版本或其首字母缩略语。
标准文献样式中不使用。

%\fielditem{shorttitle}{literal\LFMark}

%The title in an abridged form. This field is usually not included in the bibliography. It is intended for citations in author"=title format. If present, the author"=title citation styles use this field instead of \bibfield{title}.

\fielditem{shorttitle}{文本\LFMark}

缩略形式的标题。该域通常不会包括在参考文献表中。
它用于 |author-title| 格式的标注。
如果该域存在,|author-title| 标注样式使用该域来代替 \bibfield{title} 域。

%\fielditem{subtitle}{literal}

%The subtitle of the work.

\fielditem{subtitle}{文本}

作品的副标题。

%\fielditem{title}{literal}

%The title of the work.

\fielditem{title}{文本}

作品的标题。

%\fielditem{titleaddon}{literal}

%An annex to the \bibfield{title}, to be printed in a different font.

\fielditem{titleaddon}{文本}

\bibfield{title} 的附文,会用不同字体打印。

%\listitem{translator}{name}

%The translator(s) of the \bibfield{title} or \bibfield{booktitle}, depending on the entry type. If the translator is identical to the \bibfield{editor}, the standard styles will automatically concatenate these fields in the bibliography.

\listitem{translator}{姓名}

\bibfield{title} 或 \bibfield{booktitle} 的译者,具体取决于条目类型。
如果译者与 \bibfield{editor} 相同,在文献表中中标准样式会自动将这些域关联起来。

%\fielditem{type}{key}

%The type of a \bibfield{manual}, \bibfield{patent}, \bibfield{report}, or \bibfield{thesis}. This field may also be useful for the custom types listed in \secref{bib:typ:ctm}.

\fielditem{type}{关键字}

\bibfield{manual}、\bibfield{patent}、\bibfield{report} 或 \bibfield{thesis} 的类型。
该域可用于 \secref{bib:typ:ctm} 节的定制类型。

%\fielditem{url}{uri}

%The \acr{URL} of an online publication. If it is not URL-escaped (no <\%> chars) it will be URI-escaped according to RFC 3987, that is, even Unicode chars will be correctly escaped.

\fielditem{url}{uri}

在线出版物的 \acr{URL}。
如果它不是 URL-转义的(没有“\%”字符),
那么会根据 RFC 3987\footnote{参考 \url{https://tools.ietf.org/html/rfc3987} ——译注}
对其做 URI-转义,也就是说,即使 Unicode 字符也会正确转义。

%\fielditem{urldate}{date}

%The access date of the address specified in the \bibfield{url} field. See also \secref{bib:use:dat}.

\fielditem{urldate}{日期}

\bibfield{url} 域中网址的访问日期。见 \secref{bib:use:dat} 节。

%\fielditem{venue}{literal}

%The location of a conference, a symposium, or some other event in \bibtype{proceedings} and \bibtype{inproceedings} entries. This field may also be useful for the custom types listed in \secref{bib:typ:ctm}. Note that the \bibfield{location} list holds the place of publication. It therefore corresponds to the \bibfield{publisher} and \bibfield{institution} lists. The location of the event is given in the \bibfield{venue} field. See also \bibfield{eventdate} and \bibfield{eventtitle}.

\fielditem{venue}{文本}

\bibtype{proceedings} 和 \bibtype{inproceedings} 条目中的会议、研讨会或其它活动的地点。
该域可用于 \secref{bib:typ:ctm} 一节所列的定制类型。
请注意,\bibfield{location} 列表指的是出版地点,因此对应于 \bibfield{publisher} 和 \bibfield{institution} 列表。
而会议活动的会场地点则由 \bibfield{venue} 域给出。
另可参考 \bibfield{eventdate} 和 \bibfield{eventtitle} 域。

%\fielditem{version}{literal}

%The revision number of a piece of software, a manual, etc.

\fielditem{version}{文本}

软件、手册等作品的修订版本号。

%\fielditem{volume}{integer}

%The volume of a multi"=volume book or a periodical. It is expected to be an integer, not necessarily in arabic numerals since \biber will automatically from roman numerals or arabic letter to integers internally for sorting purposes. See also \bibfield{part}.

\fielditem{volume}{整数}

多卷本或连续出版物中的卷数。
其值应当是整数,但不必是阿拉伯数字的形式,因为 \biber 为了排序会将罗马数字和阿拉伯字符自动转成整数。
另可参考 \bibfield{part} 域。

%\fielditem{volumes}{integer}

%The total number of volumes of a multi"=volume work. Depending on the entry type, this field refers to \bibfield{title} or \bibfield{maintitle}. It is expected to be an integer, not necessarily in arabic numerals since \biber will automatically from roman numerals or arabic letter to integers internally for sorting purposes.

\fielditem{volumes}{整数}

多卷本著作的总卷数。
根据文献条目类型,该域对应于 \bibfield{title} 或 \bibfield{maintitle} 域。
其值应当是整数,但不必是阿拉伯数字的形式,因为 \biber 为了排序会将罗马数字和阿拉伯字符自动转成整数。

%\fielditem{year}{literal}%v3.7

%The year of publication. It is better to use the \bibfield{date} field as this is compatible with plain years too. See \secref{bib:use:dat}.

%\fielditem{year}{literal}%v3.9

%The year of publication. This field is a literal field only when given
%explicitly in the data (for plain \bibtex compatibility for example). It is
%however better to use the \bibfield{date} field as this is compatible with
%plain years too and supports many more features. See \secref{bib:use:dat}.

\fielditem{year}{文本}

出版年份。只有显式给出数据时(例如为兼容原始的\bibtex),该域才是文本类型的域。
不过最好使用 \bibfield{date} 域,因为它也能兼容显式年份(plain years)且支持更多功能\footnote{这里的 plain years 本质上是显式给出年份信息而不是由date解析给出的年份,结合\secref{bib:use:dat} 节的explicite year,译为显式年份。而plain \bibtex 的意义并不明确,可能是采用显式年份方式的\bibtex 这里暂不深入探究合适的译法——译注}。见 \secref{bib:use:dat} 节。



\end{fieldlist}

%\subsubsection{Special Fields}
\subsubsection{特殊域}
\label{bib:fld:spc}

%The fields listed in this section do not hold printable data but serve a different purpose. They apply to all entry types in the default data model.

本节中的域不保存可打印数据,而有其它用途。
在默认数据模型中,这些域可用于所有条目类型。

\begin{fieldlist}

%\fielditem{crossref}{entry key}

%This field holds an entry key for the cross"=referencing feature. Child entries with a \bibfield{crossref} field inherit data from the parent entry specified in the \bibfield{crossref} field. If the number of child entries referencing a specific parent entry hits a certain threshold, the parent entry is automatically added to the bibliography even if it has not been cited explicitly. The threshold is settable with the \opt{mincrossrefs} package option from \secref{use:opt:pre:gen}. Style authors should note that whether or not the \bibfield{crossref} fields of the child entries are defined on the \biblatex level depends on the availability of the parent entry. If the parent entry is available, the \bibfield{crossref} fields of the child entries will be defined. If not, the child entries still inherit the data from the parent entry but their \bibfield{crossref} fields will be undefined. Whether the parent entry is added to the bibliography implicitly because of the threshold or explicitly because it has been cited does not matter. See also the \bibfield{xref} field in this section as well as \secref{bib:cav:ref}.
\fielditem{crossref}{条目关键字}

该域保存的条目关键字可用于交叉引用。带有 \bibfield{crossref} 域的子条目可以从由 \bibfield{crossref} 域指定的父条目继承数据。如果引用某个父条目的子条目数量达到一个阈值,该父条目就会自动添加到参考文献表中,即使它没有在正文中被显式引用。
该阈值可以由 \secref{use:opt:pre:gen} 节中的 \opt{mincrossrefs} 宏包选项设置。样式作者请注意,在 \biblatex 层面上,子条目的 \bibfield{crossref} 域是否有定义取决于父条目是否可用。如果父条目可用,那么子条目的 \bibfield{crossref} 域将被定义。
反之,子条目仍然可以继承父条目的数据但是其 \bibfield{crossref} 域是未定义的。父条目是否被添加到文献中(由于阈值隐式地或者由于被引用而显式地被引入)对于该域的定义并不重要。另可参考本节的 \bibfield{xref} 域以及 \secref{bib:cav:ref} 节。

%\fielditem{entryset}{separated values}

%This field is specific to entry sets. See \secref{use:use:set} for details. This field is consumed by the backend processing and does not appear in the \path{.bbl}.

\fielditem{entryset}{分隔值}

该域是条目集专用的。详见 \secref{use:use:set} 节。
在后端程序处理过程中该域会被清除而不出现在 \path{.bbl} 中。

%\fielditem{execute}{code}

%A special field which holds arbitrary \tex code to be executed whenever the data of the respective entry is accessed. This may be useful to handle special cases. Conceptually, this field is comparable to the hooks \cmd{AtEveryBibitem}, \cmd{AtEveryLositem}, and \cmd{AtEveryCitekey} from \secref{aut:fmt:hok}, except that it is definable on a per"=entry basis in the \file{bib} file. Any code in this field is executed automatically immediately after these hooks.
\fielditem{execute}{代码}

保存任意 \TeX\ 代码的特殊域,这些代码会在获取各条目数据时被执行。这对处理特殊情况很有用。
概念上,该域可以类比于 \secref{aut:fmt:hok} 节中的钩子命令 \cmd{AtEveryBibitem}、\cmd{AtEveryLositem} 和 \cmd{AtEveryCitekey},但不同之处在于该域可以基于 \file{bib} 文件中的每一条目进行逐条定义。
该域中的任何代码都会在这些钩子命令后立即自动执行。

%\fielditem{gender}{Pattern matching one of: \opt{sf}, \opt{sm}, \opt{sn}, \opt{pf}, \opt{pm}, \opt{pn}, \opt{pp}}

%The gender of the author or the gender of the editor, if there is no author. The following identifiers are supported: \opt{sf} (feminine singular, a single female name), \opt{sm} (masculine singular, a single male name), \opt{sn} (neuter singular, a single neuter name), \opt{pf} (feminine plural, a list of female names), \opt{pm} (masculine plural, a list of male names), \opt{pn} (neuter plural, a list of neuter names), \opt{pp} (plural, a mixed gender list of names). This information is only required by special bibliography and citation styles and only in certain languages. For example, a citation style may replace recurrent author names with a term such as <idem>. If the Latin word is used, as is custom in English and French, there is no need to specify the gender. In German publications, however, such key terms are usually given in German and in this case they are gender"=sensitive.
\fielditem{gender}{匹配 \opt{sf}、\opt{sm}、\opt{sn}、\opt{pf}、\opt{pm}、\opt{pn}、\opt{pp} 其中之一的模式}

作者或编者(如果没有作者的话)的词性。
支持以下标识符:\opt{sf} (阴性单数,单个女性名), \opt{sm}(阳性单数,单个男性名),\opt{sn}(中性单数,单个中性名),\opt{pf}(阴性复数,多个女性名),\opt{pm}(阳性复数,多个男性名),\opt{pn}(中性复数,多个中性名),\opt{pp}(复数,不同词性名的组合)。这一信息只在特殊参考文献著录和标注样式,以及某些特定语言中是需要的。
例如,某一标注样式会用想“idem”这样的词汇来代替反复出现的作者姓名,如果按照英语或法语的习惯使用这一拉丁词汇,那么就没有必要指定词性。然而在德语出版物中,这样的关键词汇通常用德语给出,此时就会与词性相关。

%\begin{table}
%\tablesetup
%\begin{tabularx}{\textwidth}{@{}p{80pt}@{}p{170pt}@{}X@{}}
%\toprule
%\multicolumn{1}{@{}H}{Language} &
%\multicolumn{1}{@{}H}{Region/Dialect} &
%\multicolumn{1}{@{}H}{Identifiers} \\
%\cmidrule(r){1-1}\cmidrule(r){2-2}\cmidrule{3-3}
%Bulgarian    & Bulgaria       & \opt{bulgarian} \\
%Catalan      & Spain, France, Andorra, Italy & \opt{catalan} \\
%Croatian     & Croatia, Bosnia and Herzegovina, Serbia & \opt{croatian} \\
%Czech        & Czech Republic & \opt{czech} \\
%Danish       & Denmark        & \opt{danish} \\
%Dutch        & Netherlands    & \opt{dutch} \\
%English      & USA            & \opt{american}, \opt{USenglish}, \opt{english} \\
%			 & United Kingdom & \opt{british}, \opt{UKenglish} \\
%			 & Canada         & \opt{canadian} \\
%			 & Australia      & \opt{australian} \\
%			 & New Zealand    & \opt{newzealand} \\
%Estonian     & Estonia        & \opt{estonian} \\
%Finnish      & Finland        & \opt{finnish} \\
%French       & France, Canada & \opt{french} \\
%German       & Germany        & \opt{german} \\
%			 & Austria        & \opt{austrian} \\
%			 & Switzerland    & \opt{swissgerman} \\
%German (new) & Germany        & \opt{ngerman} \\
%			 & Austria        & \opt{naustrian} \\
%			 & Switzerland    & \opt{nswissgerman} \\
%Greek        & Greece         & \opt{greek} \\
%Italian      & Italy          & \opt{italian} \\
%Norwegian    & Norway         & \opt{norsk}, \opt{nynorsk} \\
%Polish       & Poland         & \opt{polish} \\
%Portuguese   & Brazil         & \opt{brazil} \\
%		   	 & Portugal       & \opt{portuguese}, \opt{portuges} \\
%Russian      & Russia         & \opt{russian} \\
%Slovak       & Slovakia       & \opt{slovak} \\
%Slovene      & Slovenia       & \opt{slovene}, \opt{slovenian} \\
%Spanish      & Spain          & \opt{spanish} \\
%Swedish      & Sweden         & \opt{swedish} \\
%Ukrainian    & Ukraine        & \opt{ukrainian} \\

%\bottomrule
%\end{tabularx}
%\caption{Supported Languages}
%\label{bib:fld:tab1}
%\end{table}

\begin{table}%[!t]
\tablesetup
\begin{tabularx}{\textwidth}{@{}p{80pt}@{}p{170pt}@{}X@{}}
\toprule
\multicolumn{1}{@{}H}{语言} &
\multicolumn{1}{@{}H}{地区/方言} &
\multicolumn{1}{@{}H}{标识符} \\
\cmidrule(r){1-1}\cmidrule(r){2-2}\cmidrule{3-3}
保加利亚语    & 保加利亚       & \opt{bulgarian} \\
加泰罗尼亚语  & 西班牙、法国、安道尔、意大利 & \opt{catalan} \\
克罗地亚语    & 克罗地亚、波黑、塞尔维亚 & \opt{croatian} \\
捷克语       & 捷克共和国 & \opt{czech} \\
丹麦语       & 丹麦        & \opt{danish} \\
荷兰语        & 荷兰    & \opt{dutch} \\
英语      	& 美国  & \opt{american}, \opt{USenglish}, \opt{english} \\
			& 英国 & \opt{british}, \opt{UKenglish} \\
			& 加拿大         & \opt{canadian} \\
			& 澳大利亚      & \opt{australian} \\
			& 新西兰    & \opt{newzealand} \\
爱沙尼亚语   & 爱沙尼亚        & \opt{estonian} \\
芬兰语      & 芬兰        & \opt{finnish} \\
法语        & 法国、加拿大 & \opt{french} \\
德语        & 德国        & \opt{german} \\
			& 奥地利        & \opt{austrian} \\
			& 瑞士    & \opt{swissgerman} \\
德语(新正字法) & 德国        & \opt{ngerman} \\
				& 奥地利        & \opt{naustrian} \\
				& 瑞士    & \opt{nswissgerman} \\
希腊语        & 希腊         & \opt{greek} \\
意大利语      & 意大利          & \opt{italian} \\
挪威语   	 & 挪威         & \opt{norwegian}, \opt{norsk}, \opt{nynorsk} \\
波兰语       & 波兰         & \opt{polish} \\
葡萄牙语  	 & 巴西         & \opt{brazil} \\
			& 葡萄牙       & \opt{portuguese}, \opt{portuges} \\
俄语     		 & 俄罗斯         & \opt{russian} \\
斯洛伐克语       & 斯洛伐克       & \opt{slovak} \\
斯洛文尼亚语      & 斯洛文尼亚       & \opt{slovene} \\
西班牙语      & 西班牙          & \opt{spanish} \\
瑞典语      	& 瑞典         & \opt{swedish} \\
乌克兰语    & 乌克兰        & \opt{ukrainian} \\
\bottomrule
\end{tabularx}
\caption{支持的语种}%Supported Languages
\label{bib:fld:tab1}
\end{table}

%\fielditem{langid}{identifier}

%The language id of the bibliography entry. The alias \bibfield{hyphenation} is provided for backwards compatibility. The identifier must be a language name known to the \sty{babel}/\sty{polyglossia} packages. This information may be used to switch hyphenation patterns and localise strings in the bibliography. Note that the language names are case sensitive. The languages currently supported by this package are given in \tabref{bib:fld:tab1}. Note that \sty{babel} treats the identifier \opt{english} as an alias for \opt{british} or \opt{american}, depending on the \sty{babel} version. The \biblatex package always treats it as an alias for \opt{american}. It is preferable to use the language identifiers \opt{american} and \opt{british} (\sty{babel}) or a language specific option to specify a language variant (\sty{polyglossia}, using the \bibfield{langidopts} field) to avoid any possible confusion. Compare \bibfield{language} in \secref{bib:fld:dat}.
\fielditem{langid}{标识符}

文献条目的语种标识。出于向后兼容性考虑,提供了别名 \bibfield{hyphenation}。标识符必须是 \sty{babel}/\sty{polyglossia} 宏包中的语言名称。该信息用于在文献表中切换断词模式和本地化字符串。请注意,语言名是大小写敏感的。目前本宏包支持的语言在\tabref{bib:fld:tab1} 中给出。需要注意的是,\sty{babel} 宏包将标识符 \opt{english} 当作 \opt{british} 或 \opt{american} 的别名,具体取决于 \sty{babel} 的版本。而 \biblatex 宏包总是将其当作 \opt{american} 的别名。
为了避免可能的混淆,最好使用语言标识符 \opt{american} 和 \opt{british}(\sty{babel})或者一个语言选项来指定一种变体语言
(\sty{polyglossia},使用 \bibfield{langidopts} 域)。可与 \secref{bib:fld:dat} 节中的 \bibfield{language} 域进行比较。

%\fielditem{langidopts}{literal}

%For \sty{polyglossia} users, allows per-entry language specific options. The literal value of this field is passed to \sty{polyglossia}'s language switching facility when using the package option \opt{autolang=langname}. For example, the fields:

\fielditem{langidopts}{文本}

对于 \sty{polyglossia} 的用户,该域使得每个条目可拥有自己的语言选项。
当使用本宏包的选项 \opt{autolang=langname} 时,该域的值将被传递到 \sty{polyglossia} 的语言切换工具中。
例如,使用如下域:

\begin{lstlisting}[style=bibtex]{}
langid         = {english},
langidopts     = {variant=british},
\end{lstlisting}

%would wrap the bibliography entry in:
会将文献条目置于如下代码块中

\begin{ltxexample}
\english[variant=british]
...
\endenglish
\end{ltxexample}
%

%\fielditem{ids}{separated list of entrykeys}

%Citation key aliases for the main citation key. An entry may be cited by any of its aliases and \biblatex will treat the citation as if it had used the primary citation key. This is to aid users who change their citation keys but have legacy documents which use older keys for the same entry. This field is consumed by the backend processing and does not appear in the \path{.bbl}.
\fielditem{ids}{条目关键字的分隔值列表}

主要引用关键字的别名。一个条目可以通过别名进行引用,\biblatex 会将其视为使用了原始的引用关键字。
借助该域,用户可以在改变引用关键字后,仍然可以使用原来使用旧的引用关键字的老文档。该域在后端程序处理过程中会被清楚,不出现在 \path{.bbl} 中。

%\fielditem{indexsorttitle}{literal}

%The title used when sorting the index. In contrast to \bibfield{indextitle}, this field is used for sorting only. The printed title in the index is the \bibfield{indextitle} or the \bibfield{title} field. This field may be useful if the title contains special characters or commands which interfere with the sorting of the index. Consider this example:
\fielditem{indexsorttitle}{文本}

排序索引时使用的标题。与 \bibfield{indextitle} 域不同,该域只用于排序。
而索引中打印出来的标题是 \bibfield{indextitle} 或\bibfield{title} 域。
当标题中含有与索引排序相冲突的特殊字符或命令时,该域会很有用。考虑如下例子:

\begin{lstlisting}[style=bibtex]{}
title          = {The \LaTeX\ Companion},
indextitle     = {\LaTeX\ Companion, The},
indexsorttitle = {LATEX Companion},
\end{lstlisting}
%
%Style authors should note that \biblatex automatically copies the value of either the \bibfield{indextitle} or the \bibfield{title} field to \bibfield{indexsorttitle} if the latter field is undefined.
文献样式作者请注意,当 \bibfield{indexsorttitle} 没有定义时,\biblatex 会自动将 \bibfield{indextitle} 或 \bibfield{title} 域的值复制给该域。


\fielditem{keywords}{分隔值} %\fielditem{keywords}{separated values}

%A separated list of keywords. These keywords are intended for the bibliography filters (see \secref{use:bib:bib, use:use:div}), they are usually not printed. Note that with the default separator (comma), spaces around the separator are ignored.
关键词的分隔值列表。这些关键词主要用于文献过滤器,通常不会打印出来(见\secref{use:bib:bib, use:use:div} 节)。请注意,使用默认的分隔符(西文逗号)时,分隔符左右的空格会被忽略。

\fielditem{options}{分隔的 \keyval 选项} %\fielditem{options}{separated \keyval options}

%A separated list of entry options in \keyval notation. This field is used to set options on a per"=entry basis. See \secref{use:opt:bib} for details. Note that citation and bibliography styles may define additional entry options.
\keyval 形式的条目选项分隔值列表。该域用于设置每一条目的选项,详见 \secref{use:opt:bib} 节。请注意,标注和著录样式会定义额外的条目选项。

\fielditem{presort}{字符串} %\fielditem{presort}{string}
%A special field used to modify the sorting order of the bibliography. This field is the first item the sorting routine considers when sorting the bibliography, hence it may be used to arrange the entries in groups. This may be useful when creating subdivided bibliographies with the bibliography filters. Please refer to \secref{use:srt} for further details. Also see \secref{aut:ctm:srt}. This field is consumed by the backend processing and does not appear in the \path{.bbl}.
用于修改文献排列顺序的特殊域。文献排序时,该域排序程序第一个考虑的项,因此可用于将文献条目分组。这在利用文献过滤器创建文献细分时很有用。更多细节请参考 \secref{use:srt} 以及 \secref{aut:ctm:srt} 节。该域在后端程序处理过程中被清楚,不出现在 \path{.bbl} 中。


\fielditem{related}{分隔值}%\fielditem{related}{separated values}
%Citation keys of other entries which have a relationship to this entry. The relationship is specified by the \bibfield{relatedtype} field. Please refer to \secref{use:rel} for further details.
与本条目关联的其它条目的引用关键字。其关联关系由 \bibfield{relatedtype} 域指定。更多细节请参考 \secref{use:rel} 节。


\fielditem{relatedoptions}{分隔值}%\fielditem{relatedoptions}{separated values}
%Per"=type options to set for a related entry. Note that this does not set the options on the related entry itself, only the \opt{dataonly} clone which is used as a datasource for the parent entry.
为关联条目设置类型相关的选项。请注意,这不会设置关联条目本身的选项,而只会影响作为数据源的父条目的临时副本。

%\fielditem{relatedtype}{identifier}

%An identifier which specified the type of relationship for the keys listed in the \bibfield{related} field. The identifier is a localised bibliography string printed before the data from the related entry list. It is also used to identify type-specific formatting directives and bibliography macros for the related entries. Please refer to \secref{use:rel} for further details.

\fielditem{relatedtype}{标识符}
标识符,为列在 \bibfield{related} 域中的关键字列表指定关联关系类型。
该标识符是本地化字符串,会在来自关联条目列表的数据之前打印。
该域也用于为关联条目指明类型相关的格式化指令和参考文献宏。详见 \secref{use:rel} 节。

%\fielditem{relatedstring}{literal}

%A field used to override the bibliography string specified by \bibfield{relatedtype}. Please refer to \secref{use:rel} for further details.

\fielditem{relatedstring}{文本}

用于覆盖 \bibfield{relatedtype} 指定的参考文献字符串。
更多细节请参考 \secref{use:rel} 节。

%\fielditem{sortkey}{literal}

%A field used to modify the sorting order of the bibliography. Think of this field as the master sort key. If present, \biblatex uses this field during sorting and ignores everything else, except for the \bibfield{presort} field. Please refer to \secref{use:srt} for further details. This field is consumed by the backend processing and does not appear in the \path{.bbl}.

\fielditem{sortkey}{文本}

用来修改文献排序的域。该域可以认为是最主要的排序键值。
当该域存在时,\biblatex 会在排序时使用它,并且忽略除 \bibfield{presort} 域之外的所有信息。
详见 \secref{use:srt} 节。该域在后端处理过程中会被清除,不出现在 \path{.bbl} 中。

%\listitem{sortname}{name}

%A name or a list of names used to modify the sorting order of the bibliography. If present, this list is used instead of \bibfield{author} or \bibfield{editor} when sorting the bibliography. Please refer to \secref{use:srt} for further details. This field is consumed by the backend processing and does not appear in the \path{.bbl}.

\listitem{sortname}{姓名}

用于修改文献排序的姓名或姓名列表。
如果该域存在,在文献排序时,它会取代  \bibfield{author} 或 \bibfield{editor} 域。
详见 \secref{use:srt} 节。该域在后端程序处理过程中会被清除,不出现在 \path{.bbl} 中。

%\fielditem{sortshorthand}{literal}

%Similar to \bibfield{sortkey} but used in the list of shorthands. If present, \biblatex uses this field instead of \bibfield{shorthand} when sorting the list of shorthands. This is useful if the \bibfield{shorthand} field holds shorthands with formatting commands such as \cmd{emph} or \cmd{textbf}. This field is consumed by the backend processing and does not appear in the \path{.bbl}.

\fielditem{sortshorthand}{文本}

与 \bibfield{sortkey} 类似但用于缩略语列表中。如果存在,在缩略语列表排序时,\biblatex 会用该域取代 \bibfield{shorthand} 域。当 \bibfield{shorthand} 域含有格式化命令(如 \cmd{emph} 或 \cmd{textbf})的缩略语时,该域是很有用的。
该域在后端程序过程中会被清除,不出现在 \path{.bbl} 中。

%\fielditem{sorttitle}{literal}

%A field used to modify the sorting order of the bibliography. If present, this field is used instead of the \bibfield{title} field when sorting the bibliography. The \bibfield{sorttitle} field may come in handy if you have an entry with a title like «An Introduction to\dots» and want that alphabetized under <I> rather than <A>. In this case, you could put «Introduction to\dots» in the \bibfield{sorttitle} field. Please refer to \secref{use:srt} for further details. This field is consumed by the backend processing and does not appear in the \path{.bbl}.
\fielditem{sorttitle}{文本}

用于修改文献排序的域。如果存在,在文献排序时,该域会取代 \bibfield{title} 域。如果一个条目带有“An Introduction to\dots”这样的标题,并且你想让它按字母“I”而不是“A”排序,那么 \bibfield{sorttitle} 域就会派上用场。
这时,你就可以在 \bibfield{sorttitle} 域中填上“Introduction to\dots”。详见 \secref{use:srt} 节。
该域在后端程序处理过程中被清除,不出现在 \path{.bbl} 中。

%\fielditem{sortyear}{integer}

%A field used to modify the sorting order of the bibliography. If present, this field is used instead of the \bibfield{year} field when sorting the bibliography. Please refer to \secref{use:srt} for further details. This field is consumed by the backend processing and does not appear in the \path{.bbl}.

\fielditem{sortyear}{整数}

用于修改文献排序的域。
如果存在,在文献排序时,该域会取代 \bibfield{year} 域。详见 \secref{use:srt} 节。
该域在后端程序处理过程中被清除,不出现在 \path{.bbl} 中。

%\fielditem{xdata}{separated list of entrykeys}

%This field inherits data from one or more \bibtype{xdata} entries. Conceptually, the \bibfield{xdata} field is related to \bibfield{crossref} and \bibfield{xref}: \bibfield{crossref} establishes a logical parent/child relation and inherits data; \bibfield{xref} establishes as logical parent/child relation without inheriting data; \bibfield{xdata} inherits data without establishing a relation. The value of the \bibfield{xdata} may be a single entry key or a separated list of keys. See \secref{use:use:xdat} for further details. This field is consumed by the backend processing and does not appear in the \path{.bbl}.
\fielditem{xdata}{条目关键字的分隔值列表}

该域从一个或更多 \bibtype{xdata} 条目中继承数据。从概念上讲,\bibfield{xdata} 域与 \bibfield{crossref} 和 \bibfield{xref} 域相关:\bibfield{crossref} 创建一个继承数据的父/子逻辑关系;
\bibfield{xref} 创建一个不继承数据的父/子逻辑关系;而 \bibfield{xdata} 则继承数据却不创建关系。
\bibfield{xdata} 的值可以是一个单个条目关键字或者条目关键字的分隔值列表。
详见 \secref{use:use:xdat} 节。该域在后端程序处理过程中被清除,不出现在 \path{.bbl} 中。

%\fielditem{xref}{entry key}

%This field is an alternative cross"=referencing mechanism. It differs from \bibfield{crossref} in that the child entry will not inherit any data from the parent entry specified in the \bibfield{xref} field. If the number of child entries referencing a specific parent entry hits a certain threshold, the parent entry is automatically added to the bibliography even if it has not been cited explicitly. The threshold is settable with the \opt{minxrefs} package option from \secref{use:opt:pre:gen}. Style authors should note that whether or not the \bibfield{xref} fields of the child entries are defined on the \biblatex level depends on the availability of the parent entry. If the parent entry is available, the \bibfield{xref} fields of the child entries will be defined. If not, their \bibfield{xref} fields will be undefined. Whether the parent entry is added to the bibliography implicitly because of the threshold or explicitly because it has been cited does not matter. See also the \bibfield{crossref} field in this section as well as \secref{bib:cav:ref}.

\fielditem{xref}{条目关键字}

该域可用于代替交叉引用机制。它与 \bibfield{crossref} 域的不同之处在于,子条目不会从其 \bibfield{xref} 域所列的父条目中继承数据。如果引用某个父条目的子条目数量达到一个阈值,该父条目就会自动添加到文献表中,即使它并没有显式地被引用。该阈值可以由 \secref{use:opt:pre:gen} 节的 \opt{mincrossrefs} 宏包选项设置。样式作者需要注意,在 \biblatex 层面上,子条目的 \bibfield{xref} 域是否有定义取决于父条目是否可用。如果父条目可用,那么子条目的 \bibfield{xref} 域将被定义。
反之,其 \bibfield{xref} 域是未定义的。父条目是否被添加到文献表中(由于阈值隐式地或者由于被引用而显式地被引入)对于域的定义并不重要。另可参考本节中的 \bibfield{crossref} 域以及 \secref{bib:cav:ref} 节。

\end{fieldlist}

%\subsubsection{Custom Fields}
\subsubsection{可定制域}
\label{bib:fld:ctm}

%The fields listed in this section are intended for special bibliography styles. They are not used by the standard bibliography styles.

本节中的域用于特殊的参考文献样式,标准样式不使用。

\begin{fieldlist}

\listitem{name{[a--c]}}{姓名}%\listitem{name{[a--c]}}{name}
%Custom lists for special bibliography styles. Not used by the standard bibliography styles.
特殊文献样式的定制列表。标准文献样式不使用。

%\fielditem{name{[a--c]}type}{key}

%Similar to \bibfield{authortype} and \bibfield{editortype} but referring to the fields \bibfield{name{[a--c]}}. Not used by the standard bibliography styles.

\fielditem{name{[a--c]}type}{关键字}

类似于 \bibfield{authortype} 和 \bibfield{editortype} 域,
不过对应的是 \bibfield{name{[a--c]}} 域。
标准文献样式不使用。

%\listitem{list{[a--f]}}{literal}

%Custom lists for special bibliography styles. Not used by the standard bibliography styles.

\listitem{list{[a--f]}}{文本}

特殊文献样式的定制列表。标准文献样式不使用。

%\fielditem{user{[a--f]}}{literal}

%Custom fields for special bibliography styles. Not used by the standard bibliography styles.

\fielditem{user{[a--f]}}{文本}

特殊文献样式的定制域。标准文献样式不使用。

%\fielditem{verb{[a--c]}}{literal}

%Similar to the custom fields above except that these are verbatim fields. Not used by the standard bibliography styles.

\fielditem{verb{[a--c]}}{文本}

类似于前述的定制域,不过这些是抄录域。标准文献样式不使用。

\end{fieldlist}

%\subsubsection{Field Aliases}
\subsubsection{域的别名}
\label{bib:fld:als}

%The aliases listed in this section are provided for backwards compatibility with traditional \bibtex and other applications based on traditional \bibtex styles. Note that these aliases are immediately resolved as the \file{bib} file is processed. All bibliography and citation styles must use the names of the fields they point to, not the alias. In \file{bib} files, you may use either the alias or the field name but not both at the same time.

本节列出的别名用于向后兼容传统 \BibTeX 以及其它基于传统 \BibTeX 样式的应用。
请注意,处理 \file{bib} 文件时即会解析这些别名。
因此所有的参考文献著录和标注样式必须使用它们所指域的名称,而不能这些别名。
但在 \file{bib} 文件中,既可以使用别名,也可以它们使用所指域的名称,但不能同时使用。


\begin{fieldlist}

\listitem{address}{文本}%\listitem{address}{literal}

%An alias for \bibfield{location}, provided for \bibtex compatibility. Traditional \bibtex uses the slightly misleading field name \bibfield{address} for the place of publication, \ie the location of the publisher, while \biblatex uses the generic field name \bibfield{location}. See \secref{bib:fld:dat,bib:use:and}.
\bibfield{location} 的别名,用于兼容 \BibTeX 。传统的 \BibTeX 使用这一稍微有些误导性的域 \bibfield{address} 来表示出版地点,即出版者的所在地,而 \biblatex 使用更一般的 \bibfield{location} 域。见 \secref{bib:fld:dat,bib:use:and} 节。

\fielditem{annote}{文本}%\fielditem{annote}{literal}

%An alias for \bibfield{annotation}, provided for \sty{jurabib} compatibility. See \secref{bib:fld:dat}.
\bibfield{annotation} 的别名,用于兼容 \sty{jurabib} 宏包。见 \secref{bib:fld:dat} 节。

\fielditem{archiveprefix}{文本}%\fielditem{archiveprefix}{literal}

%An alias for \bibfield{eprinttype}, provided for arXiv compatibility. See \secref{bib:fld:dat,use:use:epr}.
\bibfield{eprinttype} 的别名,用于兼容 arXiv 。见 \secref{bib:fld:dat,use:use:epr} 节。

%\fielditem{journal}{literal}

%An alias for \bibfield{journaltitle}, provided for \bibtex compatibility. See \secref{bib:fld:dat}.

\fielditem{journal}{文本}

\bibfield{journaltitle} 的别名,用于兼容 \BibTeX 。见 \secref{bib:fld:dat} 节。

\fielditem{key}{文本}%\fielditem{key}{literal}

%An alias for \bibfield{sortkey}, provided for \bibtex compatibility. See \secref{bib:fld:spc}.
\bibfield{sortkey} 的别名,用于兼容 \BibTeX 。见 \secref{bib:fld:spc} 节。

%\fielditem{pdf}{verbatim}

%An alias for \bibfield{file}, provided for JabRef compatibility. See \secref{bib:fld:dat}.

\fielditem{pdf}{抄录}

\bibfield{file} 的别名,用于兼容 JabRef 。见 \secref{bib:fld:dat} 节。

%\fielditem{primaryclass}{literal}

%An alias for \bibfield{eprintclass}, provided for arXiv compatibility. See \secref{bib:fld:dat,use:use:epr}.

\fielditem{primaryclass}{文本}

\bibfield{eprintclass} 的别名,用于兼容 arXiv 。见 \secref{bib:fld:dat,use:use:epr} 节。

%\listitem{school}{literal}

%An alias for \bibfield{institution}, provided for \bibtex compatibility. The \bibfield{institution} field is used by traditional \bibtex for technical reports whereas the \bibfield{school} field holds the institution associated with theses. The \biblatex package employs the generic field name \bibfield{institution} in both cases. See \secref{bib:fld:dat,bib:use:and}.

\listitem{school}{文本}

\bibfield{institution} 的别名,用于兼容 \BibTeX 。传统 \BibTeX 中,\bibfield{institution} 用于技术报告,而 \bibfield{school} 域保存与之相关的研究机构\footnote{原文中,所谓比较的两种情况并没有说清楚,暂不深入——译注}。
在这两种情况下,\biblatex 宏包都会使用一般的域 \bibfield{institution}。见 \secref{bib:fld:dat,bib:use:and}。

\end{fieldlist}

%\subsection{Usage Notes}
\subsection{使用注意事项}
\label{bib:use}

%The entry types and fields supported by this package should for the most part be intuitive to use for anyone familiar with \bibtex. However, apart from the additional types and fields provided by this package, some of the familiar ones are handled in a way which is in need of explanation.
%This package includes some compatibility code for \file{bib} files which were generated with a traditional \bibtex style in mind. Unfortunately, it is not possible to handle all legacy files automatically because \biblatex's data model is slightly different from traditional \bibtex. Therefore, such \file{bib} files will most likely require editing in order to work properly with this package. In sum, the following items are different from traditional \bibtex styles:
对于熟悉 \BibTeX 的用户来说,本宏包支持的绝大部分条目类型和域都是很直观的。然而,且不说本宏包额外新增的类型和域,一些熟悉的类型和域的处理方式也需要进一步解释一下。
宏包考虑到包含一些兼容性代码,用于处理那些由传统 \BibTeX 样式生成的 \file{bib} 文件。但不幸的是,对所有的老文件都进行自动处理是不可能的,因为 \biblatex 的数据模型与传统的 \BibTeX 有少许不同。因此,为了能在本宏包中正确使用,这样的 \file{bib} 文件也许需要稍作修改。大体上,下列事项是与传统的 \BibTeX 样式不同的:

\begin{itemize}
	\setlength{\itemsep}{0pt}
	%\item The entry type \bibtype{inbook}. See \secref{bib:typ:blx, bib:use:inb} for details.
	\item \bibtype{inbook} 条目类型。详见 \secref{bib:typ:blx, bib:use:inb} 节。

	%\item The fields \bibfield{institution}, \bibfield{organization}, and \bibfield{publisher} as well as the aliases \bibfield{address} and \bibfield{school}. See \secref{bib:fld:dat, bib:fld:als, bib:use:and} for details.
	\item \bibfield{institution}、\bibfield{organization}、\bibfield{publisher} 域以及相应的别名 \bibfield{address} 和 \bibfield{school}。详见 \secref{bib:fld:dat, bib:fld:als, bib:use:and} 节。

	%\item The handling of certain types of titles. See \secref{bib:use:ttl} for details.
	\item 一些标题类型的处理。详见 \secref{bib:use:ttl} 节。

	%\item The field \bibfield{series}. See \secref{bib:fld:dat, bib:use:ser} for details.
	\item \bibfield{series} 域。详见 \secref{bib:fld:dat, bib:use:ser} 节。

	%\item The fields \bibfield{year} and \bibfield{month}. See \secref{bib:fld:dat, bib:use:dat, bib:use:iss} for details.
	\item \bibfield{year} 和 \bibfield{month} 域。详见 \secref{bib:fld:dat, bib:use:dat, bib:use:iss} 节。

	%\item The field \bibfield{edition}. See \secref{bib:fld:dat} for details.
	\item \bibfield{edition} 域。详见 \secref{bib:fld:dat} 节。

	%\item The field \bibfield{key}. See \secref{bib:use:key} for details.
	\item \bibfield{key} 域。详见 \secref{bib:use:key} 节。
\end{itemize}

%Users of the \sty{jurabib} package should note that the \bibfield{shortauthor} field is treated as a name list by \sty{biblatex}, see \secref{bib:use:inc} for details.

\sty{jurabib} 宏包的用户请注意,\bibfield{shortauthor} 域被 \biblatex 视作姓名列表,详见 \secref{bib:use:inc} 节。

%\subsubsection{The Entry Type \bibtype{inbook}}
\subsubsection[\texttt{@inbook} 条目类型]{\bibtype{inbook} 条目类型}
\label{bib:use:inb}

%Use the \bibtype{inbook} entry type for a self"=contained part of a book with its own title only. It relates to \bibtype{book} just like \bibtype{incollection} relates to \bibtype{collection}. See \secref{bib:use:ttl} for examples. If you want to refer to a chapter or section of a book, simply use the \bibfield{book} type and add a \bibfield{chapter} and\slash or \bibfield{pages} field. Whether a bibliography should at all include references to chapters or sections is controversial because a chapter is not a bibliographic entity.

\bibtype{inbook} 条目类型用于书籍中有自己标题的独立部分。它与 \bibtype{book} 的关系正如同 \bibtype{incollection} 与 \bibtype{collection} 的关系。示例见 \secref{bib:use:ttl} 节。如果你想要指书中的某一章节,直接使用 \bibfield{book} 类型并添加 \bibfield{chapter} 或 \bibfield{pages} 域即可。参考文献表中究竟是否可以引用章节是有争议的,因为章并不是文献实体。

%\subsubsection{Missing and Omissible Data}
\subsubsection{缺失和可忽略数据}
\label{bib:use:key}

%The fields marked as <required> in \secref{bib:typ:blx} are not strictly required in all cases. The bibliography styles which ship with this package can get by with as little as a \bibfield{title} field for most entry types. A book published anonymously, a periodical without an explicit editor, or a software manual without an explicit author should pose no problem as far as the bibliography is concerned. Citation styles, however, may have different requirements. For example, an author"=year citation scheme obviously requires an \bibfield{author}\slash \bibfield{editor} and a \bibfield{year} field.

在 \secref{bib:typ:blx} 节中标记为“required”的域并不一定在所有情况下都是严格需要的。对于本宏包附带的参考文献样式,绝大部分条目类型,即便只包含 \bibfield{title} 域也能使用。就参考文献表而言,匿名出版的书籍、没有明确编者的周期性出版物、或者没有明确作者的软件手册都应当不会有问题。但是,标注样式也许会有不同的要求。例如,|author-year|标注样式就明确要求 \bibfield{author}\slash \bibfield{editor} 域和 \bibfield{year} 域。

%You may generally use the \bibfield{label} field to provide a substitute for any missing data required for citations. How the \bibfield{label} field is employed depends on the citation style. The author"=year citation styles which come with this package use the \bibfield{label} field as a fallback if either the \bibfield{author}\slash \bibfield{editor} or the \bibfield{year} is missing. The numeric styles, on the other hand, do not use it at all since the numeric scheme is independent of the available data. The author"=title styles ignore it as well, because the bare \bibfield{title} is usually sufficient to form a unique citation and a title is expected to be available in any case. The \bibfield{label} field may also be used to override the non"=numeric portion of the automatically generated \bibfield{labelalpha} field used by alphabetic citation styles. See \secref{aut:bbx:fld} for details.

一般来说,可以使用 \bibfield{label} 域代替标注所要求的任意缺失数据。\bibfield{label} 域的使用方式取决于标注样式。
如果 \bibfield{author}\slash \bibfield{editor} 域或 \bibfield{year} 域缺失,本宏包所带的 |author-year| 标注样式会将 \bibfield{label} 域作为备用信息。另一方面,顺序编码制样式根本不会用到这些,因为顺序编码格式与该数据无关。
此外,|author-title|样式也会忽略这些,因为单靠 \bibfield{title} 域已经足以生成惟一的标注,而标题几乎在所有情形中都是存在的。在顺序字母(alphabetic)标注样式中,\bibfield{label} 域也可以用于覆盖自动生成的 \bibfield{labelalpha} 域中的非数值部分。详见 \secref{aut:bbx:fld} 节。


%Note that traditional \bibtex styles support a \bibfield{key} field which is used for alphabetizing if both \bibfield{author} and \bibfield{editor} are missing. The \biblatex package treats \bibfield{key} as an alias for \bibfield{sortkey}. In addition to that, it offers very fine-grained sorting controls, see \secref{bib:fld:spc, use:srt} for details. The \sty{natbib} package employs the \bibfield{key} field as a fallback label for citations. Use the \bibfield{label} field instead.

请注意,当 \bibfield{author} 和 \bibfield{editor} 域都缺失时,传统的 \BibTeX 样式支持 \bibfield{key} 域用于依字母排序。
\biblatex 宏包将 \bibfield{key} 视为 \bibfield{sortkey} 的别名。
此外,\biblatex 还提供了非常细化的排序控制,详见 \secref{bib:fld:spc, use:srt} 节。
\sty{natbib} 宏包使用 \bibfield{key} 域作为备用的标注标签,而\biblatex 则使用 \bibfield{label} 域来代替。

%\subsubsection{Corporate Authors and Editors}
\subsubsection{集体作者和集体编者}
\label{bib:use:inc}

%Corporate authors and editors are given in the \bibfield{author} or \bibfield{editor} field, respectively. Note that they must be wrapped in an extra pair of curly braces to prevent data parsing from treating them as personal names which are to be dissected into their components. Use the \bibfield{shortauthor} field if you want to give an abbreviated form of the name or an acronym for use in citations.

集体作者和集体编者分别在 \bibfield{author} 和 \bibfield{editor} 域中给出。请注意,他们必须再用花括号括起来,以防被认为是个人姓名进而被分解成姓名成分。如果你想在标注时给出简称或首字母缩写的形式,请使用 \bibfield{shortauthor} 域。

\begin{lstlisting}[style=bibtex]{}
author       = {<<{National Aeronautics and Space Administration}>>},
shortauthor  = {NASA},
\end{lstlisting}
%
%The default citation styles will use the short name in all citations while the full name is printed in the bibliography. For corporate editors, use the corresponding fields \sty{editor} and \sty{shorteditor}. Since all of these fields are treated as name lists, it is possible to mix personal names and corporate names, provided that the names of all corporations and institutions are wrapped in braces.
默认的标注样式会在所有标注里使用短名称而在参考文献表中打印全名。对于集体编者,则使用 \bibfield{editor} 和 \bibfield{shorteditor} 域。由于这些域都被视作姓名列表,因此,只要把所有的集体作者和单位用花括号括起来,就可以将个人姓名与集体名称混合使用。

\begin{lstlisting}[style=bibtex]{}
editor       = {<<{National Aeronautics and Space Administration}>>
                and Doe, John},
shorteditor  = {NASA and Doe, John},
\end{lstlisting}
%
%Users switching from the \sty{jurabib} package to \sty{biblatex} should note that the \bibfield{shortauthor} field is treated as a name list.
从 \sty{jurabib} 宏包转到 \biblatex 宏包的用户需要注意,\bibfield{shortauthor} 域被视作姓名列表。

\subsubsection{文本列表}%\subsubsection{Literal Lists}
\label{bib:use:and}

%The fields \bibfield{institution}, \bibfield{organization}, \bibfield{publisher}, and \bibfield{location} are literal lists in terms of \secref{bib:fld}. This also applies to \bibfield{origlocation}, \bibfield{origpublisher} and to the field aliases \bibfield{address} and \bibfield{school}. All of these fields may contain a list of items separated by the keyword <|and|>. If they contain a literal <|and|>, it must be wrapped in braces.

按照 \secref{bib:fld} 节,\bibfield{institution}、\bibfield{organization}、\bibfield{publisher} 和 \bibfield{location} 等域是文本列表。\bibfield{origlocation}、\bibfield{origpublisher},以及作为别名的 \bibfield{address} 和 \bibfield{school} 域也是如此。所有的这些域都可以包含一个由关键词“|and|”分隔的项列表。
如果它们本身带有“|and|”文本,那么必须用花括号括起来。

\begin{lstlisting}[style=bibtex]{}
publisher    = {William Reid <<{and}>> Company},
institution  = {Office of Information Management <<{and}>> Communications},
organization = {American Society for Photogrammetry <<{and}>> Remote Sensing
                and
		American Congress on Surveying <<{and}>> Mapping},
\end{lstlisting}
%
%Note the difference between a literal <|{and}|> and the list separator <|and|> in the above examples. You may also wrap the entire name in braces:
请注意以上例子中作为文本和作为列表分隔符的“|and|”之间的区别。你也可以把整个名称用括号括起来:

\begin{lstlisting}[style=bibtex]{}
publisher    = {<<{William Reid and Company}>>},
institution  = {<<{Office of Information Management and Communications}>>},
organization = {<<{American Society for Photogrammetry and Remote Sensing}>>
                and
		<<{American Congress on Surveying and Mapping}>>},
\end{lstlisting}
%
%Legacy files which have not been updated for use with \biblatex will still work if these fields do not contain a literal <and>. However, note that you will miss out on the additional features of literal lists in this case, such as configurable formatting and automatic truncation.
对于一些老文件,即使没有针对 \biblatex 宏包做更新,即这些域中不含“and”文本,仍然可以使用。
然而需要注意,这种情况下,你会丢失那些属于文本列表的额外特性,例如配置格式和自动截短。

%\subsubsection{Titles}
\subsubsection{标题}
\label{bib:use:ttl}

%The following examples demonstrate how to handle different types of titles. Let's start with a five"=volume work which is referred to as a whole:
以下例子展示了如何处理不同类型的标题。首先是一个作为整体的五卷本作品:

\begin{lstlisting}[style=bibtex]{}
@MvBook{works,
  author     = {Shakespeare, William},
  title      = {Collected Works},
  volumes    = {5},
  ...
\end{lstlisting}
%
%The individual volumes of a multi"=volume work usually have a title of their own. Suppose the fourth volume of the \emph{Collected Works} includes Shakespeare's sonnets and we are referring to this volume only:
多卷本作品的每一卷通常有自己的标题。假设该\emph{多卷文选}的第四卷是莎士比亚的十四行诗,并且我们要单独引用该卷:

\begin{lstlisting}[style=bibtex]{}
@Book{works:4,
  author     = {Shakespeare, William},
  maintitle  = {Collected Works},
  title      = {Sonnets},
  volume     = {4},
  ...
\end{lstlisting}
%
%If the individual volumes do not have a title, we put the main title in the \bibfield{title} field and include a volume number:
如果单卷没有标题,我们在 \bibfield{title} 域中给出主标题,并标明卷数:

\begin{lstlisting}[style=bibtex]{}
@Book{works:4,
  author     = {Shakespeare, William},
  title      = {Collected Works},
  volume     = {4},
  ...
\end{lstlisting}
%
%In the next example, we are referring to a part of a volume, but this part is a self"=contained work with its own title. The respective volume also has a title and there is still the main title of the entire edition:
在下个例子里,我们引用一卷的某一部分,但是该部分是一个独立作品且有自己的标题。
相应的卷也有一个标题,并且整个作品有一个主标题:

\begin{lstlisting}[style=bibtex]{}
@InBook{lear,
  author     = {Shakespeare, William},
  bookauthor = {Shakespeare, William},
  maintitle  = {Collected Works},
  booktitle  = {Tragedies},
  title      = {King Lear},
  volume     = {1},
  pages      = {53-159},
  ...
\end{lstlisting}
%
%Suppose the first volume of the \emph{Collected Works} includes a reprinted essay by a well"=known scholar. This is not the usual introduction by the editor but a self"=contained work. The \emph{Collected Works} also have a separate editor:
假设\emph{多卷文选}的第一卷是由一位著名学者写的再版评论文章。这不是常见的由编者写的简介,而是一份独立的作品。
且\emph{多卷文选} 另有编者:

\begin{lstlisting}[style=bibtex]{}
@InBook{stage,
  author     = {Expert, Edward},
  title      = {Shakespeare and the Elizabethan Stage},
  bookauthor = {Shakespeare, William},
  editor     = {Bookmaker, Bernard},
  maintitle  = {Collected Works},
  booktitle  = {Tragedies},
  volume     = {1},
  pages      = {7-49},
  ...
\end{lstlisting}
%
%See \secref{bib:use:ser} for further examples.
更多例子请参考 \secref{bib:use:ser} 节。

%\subsubsection{Editorial Roles}
\subsubsection{编者角色}
\label{bib:use:edr}

%The type of editorial role performed by an editor in one of the \bibfield{editor} fields (\ie \bibfield{editor}, \bibfield{editora}, \bibfield{editorb}, \bibfield{editorc}) may be specified in the corresponding \bibfield{editor...type} field. The following roles are supported by default. The role <\texttt{editor}> is the default. In this case, the \bibfield{editortype} field is omissible.
编者域(包括 \bibfield{editor}、\bibfield{editora}、\bibfield{editorb}、\bibfield{editorc} 等)中编者角色类型可以由相应的\bibfield{editor...type} 域指定。biblatex 默认支持下述多种角色,其中“\texttt{editor}”是默认缺省角色,当采用缺省角色时,\bibfield{editortype} 域可省略。

\begin{marglist}
	\setlength{\itemsep}{0pt}
	\item[editor] %The main editor. This is the most generic editorial role and the default value.
	主要编者。这是最普通的编者角色,也是默认值。

	\item[compiler] %Similar to \texttt{editor} but used if the task of the editor is mainly compiling.
	类似于 \texttt{editor},但适用于编者主要进行编纂工作的情况。

	\item[founder] %The founding editor of a periodical or a comprehensive publication project such as a <Collected Works> edition or a long"=running legal commentary.
	诸如“多卷文选”或连续的法律评论等连续的或综合的出版项目的创始编者。

	\item[continuator] %An editor who continued the work of the founding editor (\texttt{founder}) but was subsequently replaced by the current editor (\texttt{editor}).
	继续创立者(\texttt{founder})工作的编者。创立者的工作由现任编辑(\texttt{editor})所接替。

	\item[redactor] %A secondary editor whose task is redacting the work.
	从事编修工作的次要编者。

	\item[reviser] %A secondary editor whose task is revising the work.
	从事校订工作的次要编者。

	\item[collaborator] %A secondary editor or a consultant to the editor.
	次要编者或者主编的顾问。

	\item[organizer] %Similar to \texttt{editor} but used if the task of the editor is mainly organizing.
    类似于\texttt{editor},但适用于当编者主要进行组织整理工作的情况。
\end{marglist}
%
%For example, if the task of the editor is compiling, you may indicate that in the corresponding \bibfield{editortype} field:
例如,如果编者的任务是编纂的话,你可以在相应的 \bibfield{editortype} 域中指明:

\begin{lstlisting}[style=bibtex]{}
@Collection{...,
  editor      = {Editor, Edward},
  editortype  = {compiler},
  ...
\end{lstlisting}
%
%There may also be secondary editors in addition to the main editor:
除主编之外可以有次要编者:

\begin{lstlisting}[style=bibtex]{}
@Book{...,
  author      = {...},
  editor      = {Editor, Edward},
  editora     = {Redactor, Randolph},
  editoratype = {redactor},
  editorb     = {Consultant, Conrad},
  editorbtype = {collaborator},
  ...
\end{lstlisting}
%
%Periodicals or long"=running publication projects may see several generations of editors. For example, there may be a founding editor in addition to the current editor:
期刊或长期连续的出版项目通常有不同阶段的编者。例如,除现任编者之外还可以有一位创始编者:

\begin{lstlisting}[style=bibtex]{}
@Book{...,
  author      = {...},
  editor      = {Editor, Edward},
  editora     = {Founder, Frederic},
  editoratype = {founder},
  ...
\end{lstlisting}
%
%Note that only the \bibfield{editor} is considered in citations and when sorting the bibliography. If an entry is typically cited by the founding editor (and sorted accordingly in the bibliography), the founder goes into the \bibfield{editor} field and the current editor moves to one of the \bibfield{editor...} fields:
请注意,在正文标注中以及在文献表排序时,只有 \bibfield{editor} 域会起作用。
如果一个条目要特地引用创始编者(并且据此在文献中排列),那么创始编者应在 \bibfield{editor} 域中给出,而现任编者则移动到 \bibfield{editor...} 域中:

\begin{lstlisting}[style=bibtex]{}
@Collection{...,
  editor      = {Founder, Frederic},
  editortype  = {founder},
  editora     = {Editor, Edward},
  ...
\end{lstlisting}
%
%You may add more roles by initializing and defining a new localization key whose name corresponds to the identifier in the \bibfield{editor...type} field. See \secref{use:lng,aut:lng:cmd} for details.
可以通过初始化和定义新的本地化关键字来增加更多的角色,关键字的名称对应于 \bibfield{editor...type} 域中的标识符。
详见 \secref{use:lng,aut:lng:cmd} 节。

\subsubsection{出版物和期刊系列}%\subsubsection{Publication and Journal Series}
\label{bib:use:ser}

%The \bibfield{series} field is used by traditional \bibtex styles both for the main title of a multi"=volume work and for a publication series, \ie a loosely related sequence of books by the same publisher which deal with the same general topic or belong to the same field of research. This may be ambiguous. This package introduces a \bibfield{maintitle} field for multi"=volume works and employs \bibfield{series} for publication series only. The volume or number of a book in the series goes in the \bibfield{number} field in this case:
在传统的 \BibTeX 样式中,\bibfield{series} 域既用于多卷本作品的主标题(main title),也用于出版物系列,例如同一出版者的针对大致相同的一个方向或者同一个研究领域的关系较松散的一系列书籍。这种用法是容易导致模糊的。因此,本宏包引入了 \bibfield{maintitle} 域来表示多卷本作品,而 \bibfield{series} 只用于出版物系列。系列中某一本书的卷号或序号由 \bibfield{number} 域给出:

\begin{lstlisting}[style=bibtex]{}
@Book{...,
  author        = {Expert, Edward},
  title         = {Shakespeare and the Elizabethan Age},
  series        = {Studies in English Literature and Drama},
  number        = {57},
  ...
\end{lstlisting}
%
%The \bibtype{article} entry type makes use of the \bibfield{series} field as well, but handles it in a special way. First, a test is performed to determine whether the value of the field is an integer. If so, it will be printed as an ordinal. If not, another test is performed to determine whether it is a localization key. If so, the localized string is printed. If not, the value is printed as is. Consider the following example of a journal published in numbered series:
\bibtype{article} 条目类型也使用 \bibfield{series} 域,但是使用方式比较特殊。首先,会执行一个测试来确定该域的值是否是整数。如果是的话,它会以序数的形式打印;反之,会执行另一个测试来确定它是否是本地化关键字。如果是的话,会打印本地化字符串;反之则按照本身内容如实打印。考虑下面这个以数字系列出版的期刊例子:

\begin{lstlisting}[style=bibtex]{}
@Article{...,
  journal         = {Journal Name},
  series          = {3},
  volume          = {15},
  number          = {7},
  year            = {1995},
  ...
\end{lstlisting}
%
%This entry will be printed as «\emph{Journal Name}. 3rd ser. 15.7 (1995)». Some journals use designations such as «old series» and «new series» instead of a number. Such designations may be given in the \bibfield{series} field as well, either as a literal string or as a localization key. Consider the following example which makes use of the localization key \texttt{newseries}:
该条目会打印成“\emph{Journal Name}. 3rd ser. 15.7 (1995)”。
一些期刊也会使用“旧系列”(«old series»)和“新系列”(«new series»)等标识来代替数字。
这样的标识也可以由 \bibfield{series} 域给出,或者是一个文本字符串,或者是一个本地化关键字。
考虑如下这个使用本地化关键字 \texttt{newseries} 的例子:

\begin{lstlisting}[style=bibtex]{}
@Article{...,
  journal         = {Journal Name},
  series          = {newseries},
  volume          = {9},
  year            = {1998},
  ...
\end{lstlisting}
%
%This entry will be printed as «\emph{Journal Name}. New ser. 9 (1998)». See \secref{aut:lng:key} for a list of localization keys defined by default.
该条目会打印成“\emph{Journal Name}. New ser. 9 (1998)”。默认定义的本地化关键字列表请参考 \secref{aut:lng:key} 节。

\subsubsection{日期和时间规范}%\subsubsection{Date and Time Specifications}
\label{bib:use:dat}

\begin{table}
\tablesetup
\begin{tabularx}{\textwidth}{@{}>{\ttfamily}llX@{}}
\toprule
\multicolumn{1}{@{}H}{Date Specification} &
\multicolumn{2}{H}{Formatted Date (Examples)} \\
\cmidrule(l){2-3}
&
\multicolumn{1}{H}{Short/12-hour Format} &
\multicolumn{1}{H}{Long/24-hour Format} \\
\cmidrule{1-1}\cmidrule(l){2-2}\cmidrule(l){3-3}
1850			& 1850				& 1850 \\
1997/			& 1997--			& 1997-- \\
/1997			& --1997			& --1997 \\
1997/..		& 1997--			& 1997-- \\
../1997		& --1997			& --1997 \\
1967-02			& 02/1967			& February 1967 \\
2009-01-31		& 31/01/2009			& 31st January 2009 \\
1988/1992		& 1988--1992			& 1988--1992 \\
2002-01/2002-02		& 01/2002--02/2002		& January 2002--February 2002 \\
1995-03-30/1995-04-05	& 30/03/1995--05/04/1995	& 30th March 1995--5th April 1995 \\
2004-04-05T14:34:00 & 05/04/2004 2:34 PM & 5th April 2004 14:34:00\\
\bottomrule
\end{tabularx}
\caption{Date Specifications}
\label{bib:use:tab1}
\end{table}

\begin{table}
	\tablesetup
	\begin{tabularx}{\textwidth}{@{}>{\ttfamily}llX@{}}
		\toprule
		\multicolumn{1}{@{}H}{日期规格} &
		\multicolumn{2}{H}{日期格式(例)} \\
		\cmidrule(l){2-3}
		&
		\multicolumn{1}{H}{短格式/12小时格式} &
		\multicolumn{1}{H}{长格式/24小时格式} \\
		\cmidrule{1-1}\cmidrule(l){2-2}\cmidrule(l){3-3}
		1850			& 1850				& 1850 \\
		1997/			& 1997--			& 1997-- \\
		/1997			& --1997			& --1997 \\
		
		1997/..			& 1997--			& 1997-- \\
		../1997			& --1997			& --1997 \\
		1967-02			& 02/1967			& February 1967 \\
		2009-01-31		& 31/01/2009		& 31st January 2009 \\
		1988/1992		& 1988--1992		& 1988--1992 \\
		2002-01/2002-02	& 01/2002--02/2002	& January 2002--February 2002 \\
		1995-03-30/1995-04-05	& 30/03/1995--05/04/1995	& 30th March 1995--5th April 1995 \\
		2004-04-05T14:34:00 & 05/04/2004 2:34 PM & 5th April 2004 14:34:00\\
		\bottomrule
	\end{tabularx}
	\caption{日期规范}
	\label{bib:use:tab1}
\end{table}

%Date fields such as the default data model dates \bibfield{date}, \bibfield{origdate}, \bibfield{eventdate}, and \bibfield{urldate} adhere to \acr{ISO8601-2} Extended Format specification level 1.
日期域,例如默认数据模型的日期域 \bibfield{date}、\bibfield{origdate}、\bibfield{eventdate} 和 \bibfield{urldate}等,遵循\acr{ISO8601-2}扩展格式规范 level 1。
%In addition to the \acr{ISO8601-2} empty date range markers, you may also specify an open ended/start date range by giving the range separator and omitting the end/start date (\eg \texttt{YYYY/}, \texttt{/YYYY}). See \tabref{bib:use:tab1} for some examples of valid date specifications and the formatted dates automatically generated by \biblatex. The formatted date is language specific and will be adapted automatically. If there is no \bibfield{date} field in an entry, \biblatex will also consider the fields \bibfield{year} and \bibfield{month} for backwards compatibility with traditional \bibtex but this is not encouraged as explicit \bibfield{year} and \bibfield{month} are not parsed for date meta-information markers or times and are used as-is.
除了 \acr{ISO8601-2} 空日期范围标记外,还通过给定范围分隔符并省略结束或开始日期的方式(例如 \texttt{YYYY/}、\texttt{/YYYY})来指定无末端或无开端的日期范围。
\tabref{bib:use:tab1} 列出了一些有效的日期规范以及由 \biblatex 自动生成的日期格式。
日期格式与语言有关,因此会自动调整。如果条目中没有 \bibfield{date} 域,\biblatex 还会考虑 \bibfield{year} 和 \bibfield{month} 域,不过这仅仅出于对传统 \BibTeX 的向后兼容性考虑,并不鼓励使用。因为显式的 \bibfield{year} 和 \bibfield{month} 域不能解析为日期的元信息标记,只能原样使用。

%Style authors should note that date fields like \bibfield{date} or \bibfield{origdate} are only available in the \file{bib} file. All dates are parsed and dissected into their components as the \file{bib} file is processed. The date and time components are made available to styles by way of the special fields discussed in \secref{aut:bbx:fld:dat}. See this section and \tabref{aut:bbx:fld:tab1} on page~\pageref{aut:bbx:fld:tab1} for further information.
样式作者需要注意,\bibfield{date} 或 \bibfield{origdate}  等日期域只在 \file{bib} 文件中有效。
随着 \file{bib} 文件的处理,所有的日期都被解析分解为各个日期成分。样式主要借助\secref{aut:bbx:fld:dat} 节讨论的特殊域来使用日期和时间成分。更多信息请参考该节和 \pageref{aut:bbx:fld:tab1} 页的\tabref{aut:bbx:fld:tab1}。

%\acr{ISO8601-2} Extended Format dates are astronomical dates in which year <0> exists. When outputting dates in BCE or BC era (see the \opt{dateera} option below), note that they will typically be one year earlier since BCE/BC era do not have a year 0 (year 0 is 1 BCE/BC). This conversion is automatic. See examples in \tabref{bib:use:tab2}.
\acr{ISO8601-2} 扩展格式日期是天文日期,其中第“0”年是存在的。
当输出公元前年代(BCE/BC era)的日期时(见下面的 \opt{dateera} 选项),
请注意它们通常要早一年,因为公元前年代没有第0年(第0年就是公元前1年)。
该转换是自动完成的,见\tabref{bib:use:tab2} 中的例子。

%Date field names \emph{must} end with the string <date>, as with the default date fields. Bear this in mind when adding new date fields to the datamodel (see \secref{aut:ctm:dm}). \biblatex will check all date fields after reading the date model and will exit with an error if it finds a date field which does not adhere to this naming convention.
如同默认日期域,日期域的名称\emph{必须}以字符串“date”结尾。当需要在数据模型中添加新的日期域时(见 \secref{aut:ctm:dm} 节)必须记住这一点。\biblatex 在读入日期模型后会检查所有的日期域,如果发现有日期域不遵循这一命名约定就会报错并退出。

%\acr{ISO8601-2} supports dates before common era (BCE/BC) by way of a negative date format and supports  <approximate> (circa) and uncertain dates. Such date formats set internal markers which can be tested for so that appropriate localised markers (such as \opt{circa} or \opt{beforecommonera}) can be inserted. Also supported are <unspecified> dates (\acr{ISO8601-2} 4.3) which are automatically expanded into appropriate data ranges accompanied by a field \bibfield{$<$datetype$>$dateunspecified} which details the granularity of the unspecified data.
\acr{ISO8601-2} 通过负日期格式支持公元前(before common era, BCE/BC)日期,此外还支持“近似”(circa)和不确定的日期。
这样的日期格式设置可以检测的内部标记,进而可以插入合适的本地化标记(例如 \opt{circa} 或 \opt{beforecommonera})。
另外,不确定日期(\acr{ISO8601-2} 4.3)会根据\bibfield{$<$datetype$>$dateunspecified} 域指定的未定数据间隔尺寸,自动展开成合适的日期范围。

%Styles may use this information to format such dates appropriately but the standard styles do not do this. See \tabref{bib:use:tab3} on page~\pageref{bib:use:tab3} for the allowed \acr{ISO8601-2} <unspecified> formats, their range expansions and \bibfield{$<$datetype$>$dateunspecified} values (see \secref{aut:bbx:fld:gen}).
参考文献样式可以使用该信息构造合适的日期格式,但标准样式不使用。
\pageref{bib:use:tab3} 页的\tabref{bib:use:tab3} 列出了允许的 \acr{ISO8601-2} 未定日期格式,及其范围展开和 \bibfield{\prm{datetype}dateunspecified} 域的值(\secref{aut:bbx:fld:gen} 节)。

%\begin{table}
%\tablesetup
%\begin{tabularx}{\textwidth}{@{}>{\ttfamily}llX@{}}
%\toprule
%\multicolumn{1}{@{}H}{Date Specification} &
%\multicolumn{1}{H}{Expanded Range} &
%\multicolumn{1}{H}{Meta-information} \\
%\cmidrule{1-1}\cmidrule(l){2-2}\cmidrule(l){3-3}
%199X       & 1990/1999             & yearindecade \\
%19XX       & 1900/1999             & yearincentury \\
%1999-XX    & 1999-01/1999-12       & monthinyear \\
%1999-01-XX & 1999-01-01/1999-01-31 & dayinmonth \\
%1999-XX-XX & 1999-01-01/1999-12-31 & dayinyear \\
%\bottomrule
%\end{tabularx}
%\caption{ISO8601-2 4.3 Unspecified Date Parsing}
%\label{bib:use:tab3}
%\end{table}
\begin{table}
	\tablesetup
	\begin{tabularx}{\textwidth}{@{}>{\ttfamily}llX@{}}
		\toprule
		\multicolumn{1}{@{}H}{日期规范} &
		\multicolumn{1}{H}{扩展范围} &
		\multicolumn{1}{H}{元信息} \\
		\cmidrule{1-1}\cmidrule(l){2-2}\cmidrule(l){3-3}
		199X       & 1990/1999             & yearindecade \\
		19XX       & 1900/1999             & yearincentury \\
		1999-XX    & 1999-01/1999-12       & monthinyear \\
		1999-01-XX & 1999-01-01/1999-01-31 & dayinmonth \\
		1999-XX-XX & 1999-01-01/1999-12-31 & dayinyear \\
		\bottomrule
	\end{tabularx}
	\caption{ISO8601-2 4.3 未定日期解析}%Unspecified Date Parsing
	\label{bib:use:tab3}
\end{table}

%\tabref{bib:use:tab2} shows formats which use appropriate tests and formatting. See the date meta-information tests in \secref{aut:aux:tst} and the localisation strings in \secref{aut:lng:key:dt}. See also the \file{96-dates.tex} example file for complete examples of the tests and localisation strings use.
\tabref{bib:use:tab2} 展示了使用适当测试和格式化的日期格式。
参考 \secref{aut:aux:tst} 节的日期元信息测试以及 \secref{aut:lng:key:dt} 节的本地化字符串。
关于测试和本地化字符串使用的完整例子请参考 \file{96-dates.tex} 示例文件。

%The output of <circa>, uncertainty and era information in standard styles (or custom styles not customising the internal \cmd{mkdaterange*} macros) is controlled by the package options \opt{datecirca}, \opt{dateuncertain}, \opt{dateera} and \opt{dateeraauto} (see \secref{use:opt:pre:gen}). See \tabref{bib:use:tab2} on page~\pageref{bib:use:tab2} for examples which assumes these options are all used.
在标准样式或没有定制内部宏 \cmd{mkdaterange*} 的定制样式中,
<circa>、不确定信息和纪元信息的输出由 \secref{use:opt:pre:gen} 节中的宏包选项 \opt{datecirca}、\opt{dateuncertain}、\opt{dateera} 和 \opt{dateeraauto} 控制。
\pageref{bib:use:tab2} 页中的\tabref{bib:use:tab2} 列出了使用全部这些选项的例子。

%\begin{table}
%	\tablesetup
%	\begin{tabularx}{\textwidth}{@{}>{\ttfamily}llX@{}}
%		\toprule
%		\multicolumn{1}{@{}H}{Date Specification} &
%		\multicolumn{2}{H}{Formatted Date (Examples)} \\
%		\cmidrule(l){2-3}
%		&
%		\multicolumn{1}{H}{Output Format} &
%		\multicolumn{1}{H}{Output Format Notes} \\
%		\cmidrule{1-1}\cmidrule(l){2-2}\cmidrule(l){3-3}
%		0000        & 1 BC            & \kvopt{dateera}{christian} prints \opt{beforechrist} localisation\\
%		-0876			  & 877 BCE			     & \kvopt{dateera}{secular} prints \opt{beforecommonera} localisation string\\
%		-0877/-0866 & 878 BC--867 BC & using \cmd{ifdateera} test and \opt{beforechrist} localisation string\\
%		0768 & 0768 CE & using \opt{dateeraauto} set to <1000>  and \opt{commonera} localisation string\\
%		-0343-02 & 344-02 BCE & \\
%		0343-02-03 & 343-02-03 CE & with \opt{dateeraauto=400} \\
%		0343-02-03 & 343-02-02 CE & with \opt{dateeraauto=400} and \opt{julian} \\
%		1723\textasciitilde & circa 1723 & using \cmd{ifdatecirca} test\\
%		1723? & 1723? & using \cmd{ifdateuncertain} test\\
%		1723?\textasciitilde & circa 1723? & using \cmd{ifdateuncertain} and \cmd{ifdatecirca} tests\\
%		2004-22 & 2004 & also, \bibfield{season} is set to the localisation string <summer>\\
%		2004-24 & 2004 & also, \bibfield{season} is set to the localisation string <winter>\\
%		\bottomrule
%	\end{tabularx}
%	\caption{Enhanced Date Specifications}
%	\label{bib:use:tab2}
%\end{table}

\begin{table}
	\tablesetup
	\begin{tabularx}{\textwidth}{@{}>{\ttfamily}llX@{}}
		\toprule
		\multicolumn{1}{@{}H}{日期规范} &
		\multicolumn{2}{H}{格式化日期(例)} \\
		\cmidrule(l){2-3}
		&
		\multicolumn{1}{H}{输出格式} &
		\multicolumn{1}{H}{输出格式注记} \\
		\cmidrule{1-1}\cmidrule(l){2-2}\cmidrule(l){3-3}
		0000        & 1 BC            & \kvopt{dateera}{christian} 打印本地化字符串 \opt{beforechrist} \\
		-0876			  & 877 BCE			     & \kvopt{dateera}{secular} 打印本地化字符串 \opt{beforecommonera} \\
		-0877/-0866 & 878 BC--867 BC & 使用 \cmd{ifdateera} 测试和本地化字符串 \opt{beforechrist}  \\
		0768 & 0768 CE & \opt{dateeraauto} 设置为 1000,并使用本地化字符串 \opt{commonera}\\
		-0343-02 & 344-02 BCE & \\
		0343-02-03 & 343-02-03 CE & 以及 \opt{dateeraauto=400} \\
		0343-02-03 & 343-02-02 CE & 以及 \opt{dateeraauto=400} 和 \opt{julian} \\
		1723\textasciitilde & circa 1723 & 使用 \cmd{ifdatecirca} 测试\\
		1723? & 1723? & 使用 \cmd{ifdateuncertain} 测试\\
		1723?\textasciitilde & circa 1723? & 使用 \cmd{ifdateuncertain} 和 \cmd{ifdatecirca} 测试\\
		2004-22 & 2004 & 另外,\bibfield{season} 设置为本地化字符串 <summer>\\
		2004-24 & 2004 & 另外,\bibfield{season} 设置为本地化字符串 <winter>\\
		\bottomrule
	\end{tabularx}
	\caption{增强的日期规范}
	\label{bib:use:tab2}
\end{table}

%\subsubsection{Months and Journal Issues}
\subsubsection{月份和期刊的期号}
\label{bib:use:iss}

%The \bibfield{month} field is an integer field. The bibliography style converts the month to a language"=dependent string as required. For backwards compatibility, you may also use the following three"=letter abbreviations in the \bibfield{month} field: \texttt{jan}, \texttt{feb}, \texttt{mar}, \texttt{apr}, \texttt{may}, \texttt{jun}, \texttt{jul}, \texttt{aug}, \texttt{sep}, \texttt{oct}, \texttt{nov}, \texttt{dec}.
\bibfield{month} 是整数域。文献样式按照要求将月份转化成不同语言的字符串。
出于向后兼容考虑,你也可以在 \bibfield{month} 域中使用以下的三字母缩写形式:
\texttt{jan}、\texttt{feb}、\texttt{mar}、\texttt{apr}、\texttt{may}、\texttt{jun}、
\texttt{jul}、\texttt{aug}、\texttt{sep}、\texttt{oct}、\texttt{nov}、\texttt{dec}。
%Note that these abbreviations are \bibtex strings which must be given without any braces or quotes. When using them, don't say |month={jan}| or |month="jan"| but |month=jan|. It is not possible to specify a month such as |month={8/9}|. Use the \bibfield{date} field for date ranges instead. Quarterly journals are typically identified by a designation such as <Spring> or <Summer> which should be given in the \bibfield{issue} field. The placement of the \bibfield{issue} field in \bibtype{article} entries is similar to and overrides the \bibfield{month} field.
请注意,这些缩写词是 \BibTeX 字符串,不能带有任何括号或引号。
即,不要用 |month={jan}| 或 |month="jan"|,而直接使用 |month=jan|。
不可以像 |month={8/9}| 这样指定月份,而可以使用 \bibfield{date} 域来表示日期范围。
季刊通常由“Spring”或“Summer”等标识指定,这些标识应在 \bibfield{issue} 域中给出。
在 \bibtype{article} 条目中,\bibfield{issue} 域的位置与 \bibfield{month} 域类似,并且会覆盖后者。

%\subsubsection{Pagination}
\subsubsection{标记页码}
\label{bib:use:pag}

%When specifying a page or page range, either in the \bibfield{pages} field of an entry or in the \prm{postnote} argument to a citation command, it is convenient to have \biblatex add prefixes like <p.> or <pp.> automatically and this is indeed what this package does by default. However, some works may use a different pagination scheme or may not be cited by page but rather by verse or line number. This is when the \bibfield{pagination} and \bibfield{bookpagination} fields come into play. As an example, consider the following entry:
当在条目的 \bibfield{pages} 域中或标注命令的 \prm{postnote} 选项中指明页码或页码范围时,
可以很方便地使用 \biblatex 自动添加“p.”或“pp.”等前缀,而这也确实是本宏包的默认方式。
然而,一些作品或许使用不同的页码标记格式,或者不是按页码而是按诗节或者行号引用。
此时 \bibfield{pagination} 和 \bibfield{bookpagination} 就可以起作用了。例如,考虑如下条目:

\begin{lstlisting}[style=bibtex]{}
@InBook{key,
  title          = {...},
  pagination     = {verse},
  booktitle      = {...},
  bookpagination = {page},
  pages          = {53--65},
  ...
\end{lstlisting}
%
%The \bibfield{bookpagination} field affects the formatting of the \bibfield{pages} and \bibfield{pagetotal} fields in the list of references. Since \texttt{page} is the default, this field is omissible in the above example. In this case, the page range will be formatted as <pp.~53--65>. Suppose that, when quoting from this work, it is customary to use verse numbers rather than page numbers in citations. This is reflected by the \bibfield{pagination} field, which affects the formatting of the \prm{postnote} argument to any citation command. With a citation like |\cite[17]{key}|, the postnote will be formatted as <v.~17>. Setting the \bibfield{pagination} field to \texttt{section} would yield <\S~17>. See \secref{use:cav:pag} for further usage instructions.
\bibfield{bookpagination} 域会影响文献列表中  \bibfield{pages} 和 \bibfield{pagetotal} 的格式。
由于 \texttt{page} 是默认的,因此在上面这个例子中该域可以省略。此时页码范围的格式是“pp.~53--65”。
假设引用该作品时习惯使用韵节号而不是页码数,这可以通过 \bibfield{pagination} 域反映出来,进而影响任何标注命令的 \prm{postnote} 参数的格式。引用命令如果是 |\cite[17]{key}| ,注记(postnote)的格式就会是“v.~17”。
若设置 \bibfield{pagination} 域为 \texttt{section},那么就会产生“\S~17”。
用法的进一步说明,请参考 \secref{use:cav:pag} 节。

%The \bibfield{pagination} and \bibfield{bookpagination} fields are key fields. This package will try to use their value as a localization key, provided that the key is defined. Always use the singular form of the key name in \file{bib} files, the plural is formed automatically. The keys \texttt{page}, \texttt{column}, \texttt{line}, \texttt{verse}, \texttt{section}, and \texttt{paragraph} are predefined, with \texttt{page} being the default.
\bibfield{pagination} 和 \bibfield{bookpagination} 都是关键字域。如果关键字是已定义的,本宏包会尝试使用这些域的值作为本地化关键字。在 \file{bib} 文件中要使用关键字名的单数形式,复数形式是自动形成的。预定义的关键字有 \texttt{page}、\texttt{column}、\texttt{line}、\texttt{verse}、\texttt{section} 和 \texttt{paragraph},其中 \texttt{page} 是默认值。
%The string <\texttt{none}> has a special meaning when used in a \bibfield{pagination} or \bibfield{bookpagination} field. It suppresses the prefix for the respective entry. If there are no predefined localization keys for the pagination scheme required by a certain entry, you can simply add them. See the commands \cmd{NewBibliographyString} and \cmd{DefineBibliographyStrings} in \secref{use:lng}. You need to define two localization strings for each additional pagination scheme: the singular form (whose localization key corresponds to the value of the \bibfield{pagination} field) and the plural form (whose localization key must be the singular plus the letter <\texttt{s}>). See the predefined keys in \secref{aut:lng:key} for examples.
在使用 \bibfield{pagination} 和 \bibfield{bookpagination} 时,字符串“\texttt{none}”有特殊意义,
它将取消相应条目页码标记的前缀。如果某一条目页码标记格式使用的本地化关键字未定义,你可以直接添加它们。
参考 \secref{use:lng} 节中的 \cmd{NewBibliographyString} 和 \cmd{DefineBibliographyStrings} 命令。
你需要定义两个本地化字符串来对应附加的页码标记格式:单数形式(本地化关键字对应于 \bibfield{pagination} 域的值)和复数形式(本地化关键字必须是单数形式加上字母“\texttt{s}”)。具体例子可以参考 \secref{aut:lng:key} 节的预定义关键字。

%\subsection{Hints and Caveats}
\subsection{提示与警告}
\label{bib:cav}

%This section provides some additional hints concerning the data interface of this package. It also addresses some common problems.
本节提供了一些关于本宏包的数据接口的额外提示,另外也讨论了一些常见问题。

\subsubsection{交叉引用}%\subsubsection{Cross-referencing}
\label{bib:cav:ref}

%\biber features a highly customizable cross-referencing mechanism with flexible data inheritance rules. Duplicating certain fields in the parent entry or adding empty fields to the child entry is no longer required. Entries are specified in a natural way:

\biber 的一大特色是高度可定义的交叉引用机制以及灵活的数据继承规则。因此不再需要从父条目复制一些域或者向子条目添加一些空白域,而可以用很自然的方式指定条目:

\begin{lstlisting}[style=bibtex]{}
@Book{book,
  author	= {Author},
  title		= {Booktitle},
  subtitle	= {Booksubtitle},
  publisher	= {Publisher},
  location	= {Location},
  date		= {1995},
}
@InBook{inbook,
  crossref	= {book},
  title		= {Title},
  pages		= {5--25},
}
\end{lstlisting}
%
%The \bibfield{title} field of the parent will be copied to the \bibfield{booktitle} field of the child, the \bibfield{subtitle} becomes the \bibfield{booksubtitle}. The \bibfield{author} of the parent becomes the \bibfield{bookauthor} of the child and, since the child does not provide an \bibfield{author} field, it is also duplicated as the \bibfield{author} of the child. After data inheritance, the child entry is similar to this:
父条目的 \bibfield{title} 和 \bibfield{subtitle} 会分别复制给子条目的 \bibfield{booktitle} 和 \bibfield{booksubtitle}。父条目的 \bibfield{author} 会成为子条目的 \bibfield{bookauthor},
并且由于子条目没有提供 \bibfield{author} 域,它也会复制给子条目的 \bibfield{author} 域。
继承数据之后子条目会大致如下:

\begin{lstlisting}[style=bibtex]{}
author	  	= {Author},
bookauthor	= {Author},
title		= {Title},
booktitle	= {Booktitle},
booksubtitle	= {Booksubtitle},
publisher	= {Publisher},
location	= {Location},
date		= {1995},
pages		= {5--25},
\end{lstlisting}
%
%See \apxref{apx:ref} for a list of mapping rules set up by default. Note that all of this is customizable. See \secref{aut:ctm:ref} on how to configure \biber's cross"=referencing mechanism. See also \secref{bib:fld:spc}.
默认的映射规则列表请参考\apxref{apx:ref}。请注意,所有这一切都是可以定制的。
关于如何配置 \biber 的交叉引用机制请参考 \secref{aut:ctm:ref} 以及 \secref{bib:fld:spc} 节。

\paragraph{\bibfield{xref} 域}%\paragraph{The \bibfield{xref} field}
\label{bib:cav:ref:ref}

%In addition to the \bibfield{crossref} field, \biblatex supports a simplified cross"=referencing mechanism based on the \bibfield{xref} field. This is useful if you want to establish a parent\slash child relation between two associated entries but prefer to keep them independent as far as the data is concerned. The \bibfield{xref} field differs from \bibfield{crossref} in that the child entry will not inherit any data from the parent. If the parent is referenced by a certain number of child entries, \biblatex will automatically add it to the bibliography. The threshold is controlled by the \opt{minxrefs} package option  from \secref{use:opt:pre:gen}.u See also \secref{bib:fld:spc}.

除了 \bibfield{crossref} 域之外,\biblatex 也支持一种基于 \bibfield{xref} 域的简化交叉引用机制。
如果你想在两个关联条目间创建父\slash 子关系,但又希望保持它们之间的数据独立性,那么该域会很有用。
\bibfield{xref} 域与 \bibfield{crossref} 的不同之处在于子条目不会从父条目继承任何数据。
如果一个父条目被一定数量的子条目引用,那么它将被 \biblatex 自动添加到参考文献表中。
关联子条目数量的阈值由 \secref{use:opt:pre:gen} 节的 \opt{minxrefs} 宏包选项所控制。
另可参考 \secref{bib:fld:spc} 节。

%\subsubsection{Sorting and Encoding Issues}
\subsubsection{排序和编码问题}
\label{bib:cav:enc}

%\biber handles Ascii, 8-bit encodings such as Latin\,1, and \utf. It features true Unicode support and is capable of reencoding the \file{bib} data on the fly in a robust way. For sorting, \biber uses a Perl implementation of the Unicode Collation Algorithm (\acr{UCA}), as outlined in Unicode Technical Standard \#10.\fnurl{http://unicode.org/reports/tr10/}
%Collation tailoring based on the Unicode Common Locale Data Repository (\acr{CLDR}) is also supported.\fnurl{http://cldr.unicode.org/}

\biber 能处理 Ascii编码,Latin\,1等8比特编码,以及\utf 。它支持真正的Unicode,并能以一种鲁棒的方式对\file{bib} 数据进行即时重新编码
\footnote{译注——perl的编码转换见\href{Encode}{Encode}}。
对于排序,\biber 使用perl实现的Unicode排序算法(Unicode Collation Algorithm,\acr{UCA}),
该算法见Unicode技术标准\#10\fnurl{http://unicode.org/reports/tr10/}。
另外也支持基于Unicode通用本地化数据库(Common Locale Data Repository,\acr{CLDR})的排序调整
\fnurl{http://cldr.unicode.org/}
\footnote{译注——排序的本地化调整方案见
\href{Unicode::Collation::locale}{Unicode::Collation::locale}}。


%Supporting Unicode implies much more than handling \utf input. Unicode is a complex standard covering more than its most well-known parts, the Unicode character encoding and transport encodings such as \utf. It also standardizes aspects such as string collation, which is required for language-sensitive sorting. For example, by using the Unicode Collation Algorithm, \biber can handle the character <ß> without any manual intervention. All you need to do to get localised sorting is specify the locale:

支持Unicode不仅意味着能处理 \utf 输入。Unicode是一个复杂的标准,不仅涵盖了它最著名的部分——Unicode字符编码和 \utf 等传输编码。它同样对字符串排序等方面做了标准化,用于语言相关的排序。例如,使用Unicode排序算法(\acr{UCA}),\biber 可以处理字符“ß”,而不需要任何人工干预。要做本地化排序,你只需要指定本地化设置:

\begin{ltxexample}
\usepackage[sortlocale=de]{biblatex}
\end{ltxexample}
%
%or if you are using German as the main document language via \sty{babel} or \sty{polyglossia}:
如果通过 \sty{babel} 或者 \sty{polyglossia} 等宏包将德语设置为主文档语言,设置方式为:

\begin{ltxexample}
\usepackage[sortlocale=auto]{biblatex}
\end{ltxexample}
%
%This will make \biblatex pass the \sty{babel}/\sty{polyglossia} main document language
%as the locale which \biber will map into a suitable default locale. \biber
%will not try to get locale information from its environment as this makes
%document processing dependent on something not in the document which is
%against \tex's spirit of reproducibility. This also makes sense since
%\sty{babel}/\sty{polyglossia} are in fact the relevant environment for a document.
这时,\biblatex 会将 \sty{babel}/\sty{polyglossia} 主文档语言作为本地化语言传递进来,\biber 会将其映射为合适的默认本地化语言。
\biber 不会尝试从操作环境中获取本地化信息,因为这会使得文本处理依赖于文档以外的东西,而这有悖于 \TeX 要求可移植性的精神。
由于 \sty{babel}/\sty{polyglossia}  实际上提供了文档的相关环境,这种处理方式也是合理的。

%Note that this will also work with 8-bit encodings such as Latin\,9, \ie you can
%take advantage of Unicode-based sorting even though you are not using \utf
%input. See \secref{bib:cav:enc:enc} on how to specify input and data
%encodings properly.
请注意,这对于  Latin\,9 等8比特编码也是有效的,也就是说,你可以利用基于Unicode的排序,即使你没有使用 \utf 输入。
关于如何正确指定输入和数据编码,请参考 \secref{bib:cav:enc:enc}。

%\paragraph{Specifying Encodings}
\paragraph{指定编码}
\label{bib:cav:enc:enc}

%When using a non-Ascii encoding in the \file{bib} file, it is important to understand what \biblatex can do for you and what may require manual intervention. The package takes care of the \latex side, \ie it ensures that the data imported from the \file{bbl} file is interpreted correctly, provided that the \opt{bibencoding} package option is set properly. All of this is handled automatically and no further steps, apart from setting the \opt{bibencoding} option in certain cases, are required. Here are a few typical usage scenarios along with the relevant lines from the document preamble:
当在 \file{bib} 中使用非Ascii 编码时,理解\biblatex 能做什么以及哪些还需要进行人工干预很重要。
本宏包能满足 \LaTeX 需要,只要 \opt{bibencoding} 宏包选项设置正确,就能确保从 \file{bbl} 文件导入的数据能被正确解析。
所有这一切都会自动处理,除\opt{bibencoding} 选项设置为某些特定值的情况外,不需要额外的步骤。
以下给出了一些典型的使用场景以及文件导言区中的相关行:

\begin{itemize}
\setlength{\itemsep}{0pt}

\item
%Ascii notation in both the \file{tex} and the \file{bib} file with \pdftex or traditional \tex:
\file{tex} 和 \file{bib} 文件都使用 Ascii 编码,使用 \pdfTeX 或传统的 \TeX 编译:

\begin{ltxexample}
\usepackage{biblatex}
\end{ltxexample}

\item
%Latin\,1 encoding (\acr{ISO}-8859-1) in the \file{tex} file, Ascii notation in the \file{bib} file with \pdftex or traditional \tex :
\file{tex} 使用 Latin\,1 编码(\acr{ISO}-8859-1),\file{bib} 文件使用Ascii编码,
用 \pdfTeX 或传统的 \TeX 编译:

\begin{ltxexample}
\usepackage[latin1]{inputenc}
\usepackage[bibencoding=ascii]{biblatex}
\end{ltxexample}

\item
%Latin\,9 encoding (\acr{ISO}-8859-15) in both the \file{tex} and the \file{bib} file with \pdftex or traditional:
\file{tex} 和 \file{bib} 文件中都使用 Latin\,9 编码(\acr{ISO}-8859-15),
用 \pdfTeX 或传统的 \TeX 编译:

\begin{ltxexample}
\usepackage[latin9]{inputenc}
\usepackage[bibencoding=auto]{biblatex}
\end{ltxexample}
%
%Since \kvopt{bibencoding}{auto} is the default setting, the option is omissible. The following setup will have the same effect:
由于 \kvopt{bibencoding}{auto} 是默认设置,因此该选项可以省略。如下设置具有相同效果:

\begin{ltxexample}
\usepackage[latin9]{inputenc}
\usepackage{biblatex}
\end{ltxexample}

\item
%\utf encoding in the \file{tex} file, Latin\,1 (\acr{ISO}-8859-1) in the \file{bib} file with \pdftex or traditional \tex:
\file{tex} 文件中使用 \utf 编码,\file{bib} 文件中使用 Latin\,1(\acr{ISO}-8859-1)编码,
用 \pdfTeX 或传统的 \TeX 编译:

\begin{ltxexample}
\usepackage[utf8]{inputenc}
\usepackage[bibencoding=latin1]{biblatex}
\end{ltxexample}

%The same scenario with \xetex or \luatex in native \utf mode:
在原生 \utf 模式下使用 \XeTeX 或 \LuaTeX 编译的相同场景:

\begin{ltxexample}
\usepackage[bibencoding=latin1]{biblatex}
\end{ltxexample}

\end{itemize}

%\biber can handle Ascii notation, 8-bit encodings such as Latin\,1, and \utf. It is also capable of reencoding the \file{bib} data on the fly (replacing the limited macro-level reencoding feature of \biblatex). This will happen automatically if required, provided that you specify the encoding of the \file{bib} files properly. In addition to the scenarios discussed above, \biber can also handle the following cases:

\biber 可以处理 Ascii 记法、Latin\,1等8比特编码,以及\utf。
它也能在运行中对\file{bib}数据进行实时重新编码(取代\biblatex 在宏层面有限的重新编码功能)。
如果你能正确指定 \file{bib} 文件的编码,这将会在需要时自动处理。
除了以上讨论的场景外,\biber 还能够处理以下情况:

\begin{itemize}

\item
%Transparent \utf workflow, \ie \utf encoding in both the \file{tex} and the \file{bib} file with \pdftex or traditional \tex:
直接的 \utf 工作流,即,在 \file{tex} 和 \file{bib} 文件中都使用 \utf 编码并使用 \pdfTeX 或传统的 \TeX 编译:

\begin{ltxexample}
\usepackage[utf8]{inputenc}
\usepackage[bibencoding=auto]{biblatex}
\end{ltxexample}
%
%Since \kvopt{bibencoding}{auto} is the default setting, the option is omissible:
由于 \kvopt{bibencoding}{auto} 是默认设置,因此该选项可以省略:

\begin{ltxexample}
\usepackage[utf8]{inputenc}
\usepackage{biblatex}
\end{ltxexample}

%The same scenario with \xetex or \luatex in native \utf mode:
在原生 \utf 模式下使用 \XeTeX 或 \LuaTeX 编译的相同场景:

\begin{ltxexample}
\usepackage{biblatex}
\end{ltxexample}

\item
%It is even possible to combine an 8-bit encoded \file{tex} file with \utf encoding in the \file{bib} file, provided that all characters in the \file{bib} file are also covered by the selected 8-bit encoding:
甚至可以在 \file{tex} 文件中使用8比特编码,而在 \file{bib} 文件中使用 \utf 编码,只要 \file{bib} 文件中的所有字符都能被所选择的8比特编码覆盖:

\begin{ltxexample}
\usepackage[latin1]{inputenc}
\usepackage[bibencoding=utf8]{biblatex}
\end{ltxexample}

\end{itemize}

%Some workarounds may be required when using traditional \tex or \pdftex with \utf encoding because \sty{inputenc}'s \file{utf8} module does not cover all of Unicode. Roughly speaking, it only covers the Western European Unicode range. When loading \sty{inputenc} with the \file{utf8} option, \biblatex will normally instruct \biber to reencode the \file{bib} data to \utf. This may lead to \sty{inputenc} errors if some of the characters in the \file{bib} file are outside the limited Unicode range supported by \sty{inputenc}.
当对 \utf 编码使用传统的 \TeX 或 \pdfTeX 时,可能需要一些变通处理,因为 \sty{inputenc} 的 \file{utf8} 模块并不能覆盖所有的Unicode。粗略地讲,它只覆盖了西欧字符的Unicode范围。当载入带有 \file{utf8} 选项的 \sty{inputenc} 宏包时,\biblatex 通常会指示 \biber 将 \file{bib} 数据重新编码为 \utf。
如果 \file{bib} 文件中的字符超出了 \sty{inputenc} 支持的Unicode范围,这可能会导致 \sty{inputenc} 报错。

\begin{itemize}

\item
%If you are affected by this problem, try setting the \opt{safeinputenc} option:
如果你受到这个问题的影响,尝试设置 \opt{safeinputenc} 选项:

\begin{ltxexample}
\usepackage[utf8]{inputenc}
\usepackage[safeinputenc]{biblatex}
\end{ltxexample}
%
%If this option is enabled, \biblatex will ignore \sty{inputenc}'s \opt{utf8} option and use Ascii. \biber will then try to convert the \file{bib} data to Ascii notation. For example, it will convert \k{S} to |\k{S}|.
如果该选项被启用,\biblatex 会忽略 \sty{inputenc} 的 \opt{utf8} 选项而使用Ascii。\biber 随后会尝试将 \file{bib} 数据转化为Ascii记法。例如,它将 \k{S} 转化为 |\k{S}|。
%This option is similar to setting \kvopt{texencoding}{ascii} but will only take effect in this specific scenario (\sty{inputenc}\slash \sty{inputenx} with \utf). This workaround takes advantage of the fact that both Unicode and the \utf transport encoding are backwards compatible with Ascii.
该选项类似于设置 \kvopt{texencoding}{ascii} 但是只影响这一特定场合(带有 \utf 的 \sty{inputenc}\slash \sty{inputenx} 宏包)。这一变通处理利用了一个事实:Unicode和\utf 传输编码都向后兼容Ascii。

\end{itemize}

%This solution may be acceptable as a workaround if the data in the \file{bib} file is mostly Ascii anyway, with only a few strings, such as some authors' names, causing problems.
如果 \file{bib} 文件中的数据主要是Ascii,仅含有很少部分会导致问题的字符串(例如一些作者的名字),那么这一变通方法式可以接受的。
%However, keep in mind that it will not magically make traditional \tex or \pdftex support Unicode. It may help if the occasional odd character is not supported by \sty{inputenc}, but may still be processed by \tex when using an accent command (\eg |\d{S}| instead of \d{S}). If you need full Unicode support, however, switch to \xetex or \luatex.
然而,需要记住的是,它不会奇迹般地让传统的 \TeX 或 \pdfTeX 支持Unicode。当使用重音命令(例如用 |\d{S}| 取代 \d{S})时,如果遇到零星一些\sty{inputenc} 不支持字符,但仍需要用\TeX 处理,这种方式会有所帮助。然而,如果你需要完全的Unicode支持,请使用 \XeTeX 或 \LuaTeX 。

%Typical errors when \sty{inputenc} cannot handle a certain UTF-8 character are:
\sty{inputenc} 不能处理某一特定 \utf 字符时典型的错误是:

\begin{verbatim}
! Package inputenc Error: Unicode char <char> (U+<codepoint>)
(inputenc)                not set up for use with LaTeX.
\end{verbatim}
%
%but also less obvious things like:
但也可能不那么明显,如:

\begin{verbatim}
! Argument of \UTFviii@three@octets has an extra }.
\end{verbatim}

\endinput
%% update


% !TeX encoding = UTF-8
% UserGuide.tex

%\section{User Guide}
\section{用户使用手册}
\label{use}

%This part of the manual documents the user interface of the \biblatex package. The user guide covers everything you need to know in order to use \biblatex with the default styles that come with this package. You should read the user guide first in any case. If you want to write your own citation and\slash or bibliography styles, continue with the author guide afterwards.

本部分介绍了 \biblatex 宏包的用户接口,
涵盖要使用 \biblatex 自带的标准样式所需的所有信息。
无论如何,用户都应首先阅读这一部分内容。
如果你想编写你自己的著录和标注(引用)样式,请继续阅读随后的作者指南。

\subsection{宏包选项}
%Package Options
\label{use:opt}

%All package options are given in \keyval notation. The value \texttt{true} is omissible with all boolean keys. For example, giving \opt{sortcites} without a value is equivalent to \kvopt{sortcites}{true}.

所有的宏包选项都以 \keyval 记法给出。
对于所有的布尔型键值,\texttt{true} 是可以省略的。
例如,给出不带值的 \opt{sortcites} 等价于 \kvopt{sortcites}{true}。

\subsubsection{载入时选项}
%Load-time Options
\label{use:opt:ldt}

%The following options must be used as \biblatex is loaded, \ie in the optional argument to \cmd{usepackage}.

以下的选项必须在 \biblatex 载入时使用,即作为 \cmd{usepackage} 的可选参数。

\begin{optionlist}

\optitem[biber]{backend}{\opt{bibtex}, \opt{bibtex8}, \opt{biber}}

%Specifies the database backend. The following backends are supported:

指定数据库后端程序。支持以下后端:

\begin{valuelist}
	
\item[biber] %	\biber, the default backend of \biblatex, supports Ascii, 8-bit encodings, \utf, on-the-fly reencoding, locale"=specific sorting, and many other features. Locale"=specific sorting, case"=sensitive sorting, and upper\slash lowercase precedence are controlled by the options \opt{sortlocale}, \opt{sortcase}, and \opt{sortupper}, respectively.
\biber{},\biblatex 的默认后端程序,支持Ascii、8比特编码、\utf{}、实时重新编码、本地化定制排序,及许多其它特性。
本地化定制排序,大小写敏感排序和大小写优先排序分别由选项 \opt{sortlocale}、\opt{sortcase} 和 \opt{sortupper} 控制。

\item[bibtex] %Legacy \bibtex. Traditional \bibtex supports Ascii encoding only. Sorting is always case"=insensitive.
遗留下来的 \BibTeX{}。传统的 \BibTeX 只支持Ascii编码,并且排序总是大小写不敏感的。
	
\item[bibtex8] %\bin{bibtex8}, the 8-bit implementation of \bibtex, supports Ascii and 8-bit encodings such as Latin~1.
\bin{bibtex8} 是 \BibTeX 的8比特实现,支持Ascii和Latin~1等8比特编码。

\end{valuelist}

%See \secref{use:bibtex} for details of using \bibtex as a backend.
使用 \BibTeX 作为后端的细节见 \secref{use:bibtex} 节。

\valitem[numeric]{style}{file}

%Loads the bibliography style \prm{file}\path{.bbx} and the citation style \prm{file}\path{.cbx}. See \secref{use:xbx} for an overview of the standard styles.

加载著录样式文件 \prm{file}\path{.bbx} 和标注(引用)样式文件 \prm{file}\path{.cbx}。
标准样式概览见 \secref{use:xbx} 节。

\valitem[numeric]{bibstyle}{file}

%Loads the bibliography style \prm{file}\path{.bbx}. See \secref{use:xbx:bbx} for an overview of the standard bibliography styles.

加载著录样式文件 \prm{file}\path{.bbx}。
标准著录样式概览见 \secref{use:xbx:bbx} 节。

\valitem[numeric]{citestyle}{file}

%Loads the citation style \prm{file}\path{.cbx}. See \secref{use:xbx:cbx} for an overview of the standard citation styles.

加载标注(引用)样式文件 \prm{file}\path{.cbx}。
标准标注样式概览见 \secref{use:xbx:cbx} 节。

\boolitem[false]{natbib}

%Loads compatibility module which provides aliases for the citation commands of the \sty{natbib} package. See \secref{use:cit:nat} for details.

加载兼容性模块,该模块提供 \sty{natbib} 宏包引用命令的同名替代命令。
详见 \secref{use:cit:nat} 节。

\boolitem[false]{mcite}

%Loads a citation module which provides \sty{mcite}\slash\sty{mciteplus}-like citation commands. See \secref{use:cit:mct} for details.

加载一个标注(引用)模块,该模块提供 \sty{mcite}\slash 类 \sty{mciteplus} 的标注(引用)命令。
详见 \secref{use:cit:mct} 节。

\end{optionlist}

\subsubsection{导言区选项}
%Preamble Options
\label{use:opt:pre}

\paragraph{一般选项}
%General
\label{use:opt:pre:gen}

%The following options may be used in the optional argument to \cmd{usepackage} as well as in the configuration file and the document preamble. The default value listed to the right is the package default. Note that bibliography and citation styles may modify the default setting at load time, see \secref{use:xbx} for details.

下列选项既可以作为 \cmd{usepackage} 的可选项,也可以在配置文件和导言区中使用。
右侧列出的是选项的宏包默认值。
请注意,著录和标注样式可以在载入时修改默认设置,详见 \secref{use:xbx} 节。

\begin{optionlist}

\optitem[nty]{sorting}{\opt{nty}, \opt{nyt}, \opt{nyvt}, \opt{anyt}, \opt{anyvt}, \opt{ynt}, \opt{ydnt}, \opt{none}, \opt{debug}, \prm{name}}

%The sorting order of the bibliography. Unless stated otherwise, the entries are sorted in ascending order. The following choices are available by default:
参考文献的排序方式。除非另加说明,文献条目总是按升序排列。
默认提供以下可选值:

\begin{valuelist}
\item[nty] %Sort by name, title, year.
按照姓名、标题、年份排序。
\item[nyt] %Sort by name, year, title.
按照姓名、年份、标题排序。
\item[nyvt] %Sort by name, year, volume, title.
按照姓名、年份、卷数、标题排序。
\item[anyt] %Sort by alphabetic label, name, year, title.
按照字母标签、姓名、年份、标题排序。
\item[anyvt] %Sort by alphabetic label, name, year, volume, title.
按照字母标签、姓名、年份、卷数、标题排序。
\item[ynt] %Sort by year, name, title.
按照年份、姓名、标题排序。
\item[ydnt] %Sort by year (descending), name, title.
按照年份(降序)、姓名、标题排序。
\item[none] %Do not sort at all. All entries are processed in citation order.
不进行排序。所有的条目按照引用顺序处理。
\item[debug] %Sort by entry key. This is intended for debugging only.
按照条目键值排列。该选项只用于程序调试。
\item[\prm{name}] %Use \prm{name}, as defined with \cmd{DeclareSortingScheme} (\secref{aut:ctm:srt})
使用由 \cmd{DeclareSortingScheme}(\secref{aut:ctm:srt})定义的 \prm{name}。
\end{valuelist}

%Using any of the <alphabetic> sorting templates only makes sense in conjunction with a bibliography style which prints the corresponding labels. Note that some bibliography styles initialize this package option to a value different from the package default (\opt{nty}). See \secref{use:xbx:bbx} for details. Please refer to \secref{use:srt} for an in"=depth explanation of the above sorting options as well as the fields considered in the sorting process. See also \secref{aut:ctm:srt} on how to adapt the predefined templates or define new ones.

只有与打印相应标签的参考文献样式配合使用,“字母顺序”排序格式才有意义。
请注意,一些参考文献样式会将宏包选项从宏包的默认值(\opt{nty})初始化为另外一个值。
详见 \secref{use:xbx:bbx} 节。
关于以上排序选项的深度解读以及排序过程中涉及的域,请参考 \secref{use:srt} 节。
关于如何调整预定义格式或者定义新格式也可参考 \secref{aut:ctm:srt}。

\boolitem[true]{sortcase}

%Whether or not to sort the bibliography and the list of shorthands case"=sensitively.

文献和缩略语列表的排序是否是大小写敏感的。

\boolitem[true]{sortupper}

%This option corresponds to \biber's \opt{--sortupper} command-line option. If enabled, the bibliography is sorted in <uppercase before lowercase> order. Disabling this option means <lowercase before uppercase> order.

对应于 \biber 的 |--sortupper| 命令行选项。
当该选项激活时,文献表会按照“大写在前小写在后”的顺序排列。
关闭该选项则使用“小写在前大写在后”的顺序。

\optitem{sortlocale}{\opt{auto}, \prm{locale}}

%This option sets the global sorting locale. Every sorting template inherits this locale if none is specified using the \prm{locale} option to \cmd{printbibliography}. Setting this to \opt{auto} requests that it be set to the \sty{babel}/\sty{polyglossia} main document language identifier, if these packages are used and \texttt{en\_US} otherwise. \biber will map \sty{babel}/\sty{polyglossia} language identifiers into sensible locale identifiers (see the \biber documentation). You can therefore specify either a normal locale identifier like \texttt{de\_DE\_phonebook}, \texttt{es\_ES} or one of the supported \sty{babel}/\sty{polyglossia} language identifiers if the mapping \biber makes of this is fine for you.
该选项设置全局的排序区域语言(本地语言)。
只要\cmd{printbibliography}时未对 \prm{locale} 选项做设置,那么各排序格式都会继承该区域语言设置。
当该选项值设置为 \opt{auto},如果使用 \sty{babel}/\sty{polyglossia} 宏包,那么\prm{locale}选项会设置为主文档语言的标识符,否则\prm{locale}将设置为 \texttt{en\_US}。
\biber 会将 \sty{babel}/\sty{polyglossia} 语言标识符映射为有意义的本地化标识符(参考 \biber 文档)。
因此,你可以指定标准的本地化标识符,例如 \texttt{de\_DE\_phonebook}、\texttt{es\_ES};
或者指定本宏包支持的 \sty{babel}/\sty{polyglossia} 语言标识符,如果你对\biber 的语言映射还满意的话。

\boolitem[true]{related}

%Whether or not to use information from related entries or not. See \secref{use:rel}.

是否使用来自关联条目的信息。参考 \secref{use:rel} 节。

\boolitem[false]{sortcites}

%Whether or not to sort citations if multiple entry keys are passed to a citation command. If this option is enabled, citations are sorted according to the current bibliography context sorting scheme (see \secref{use:bib:context}). This feature works with all citation styles.

当多个条目的引用关键字传给一个标注(引用)命令时,是否对标注进行排序。
如果该选项激活,就会根据当前的文献表的排序格式(见 \secref{use:bib:context} 节)对标注进行排序。
该功能对所有的标注样式都有效。

\boolitem[false]{sortsets}

是否根据当前参考文献文境排序格式对条目集成员排序。默认设置为false,条目集成员以数据源给出的顺序排列。
%Whether or not to sort set members according to the active reference context sorting scheme. By default this is false and set members appear in the order given in the data source.


\intitem[3]{maxnames}

%A threshold affecting all lists of names (\bibfield{author}, \bibfield{editor}, etc.). If a list exceeds this threshold, \ie if it holds more than \prm{integer} names, it is automatically truncated according to the setting of the \opt{minnames} option. \opt{maxnames} is the master option which sets both \opt{maxbibnames} and \opt{maxcitenames}.
影响所有名称列表(\bibfield{author}、\bibfield{editor} 等)的阈值。如果某个列表超过了该阈值,即,它包含的姓名数量超过 \prm{integer},那么就会根据 \opt{minnames} 选项的设置进行自动截断。\opt{maxnames} 是设置 \opt{maxbibnames} 和 \opt{maxcitenames} 两个选项的支配选项。

\intitem[1]{minnames}

%A limit affecting all lists of names (\bibfield{author}, \bibfield{editor}, etc.). If a list holds more than \prm{maxnames} names, it is automatically truncated to \prm{minnames} names. The \prm{minnames} value must be smaller than or equal to \prm{maxnames}. \opt{minnames} is the master option which sets both \opt{minbibnames} and \opt{mincitenames}.
影响所有名称列表(\bibfield{author}、\bibfield{editor} 等)的限制值。如果某个列表包含的姓名数量超过 \prm{integer},
那么就会自动截断至\opt{minnames}个姓名。\prm{minnames} 的值必须小于或等于 \prm{maxnames}。
\opt{minnames} 是设置 \opt{minbibnames} 和 \opt{mincitenames} 两个选项的支配选项。

\intitem[\prm{maxnames}]{maxbibnames}

%Similar to \opt{maxnames} but affects only the bibliography.

类似于 \opt{maxnames} 但只影响参考文献表。

\intitem[\prm{minnames}]{minbibnames}

%Similar to \opt{minnames} but affects only the bibliography.

类似于  \opt{minnames} 但只影响参考文献表。

\intitem[\prm{maxnames}]{maxcitenames}

%Similar to \opt{maxnames} but affects only the citations in the document body.

类似于 \opt{maxnames} 但只影响正文中的标注(引用)。

\intitem[\prm{minnames}]{mincitenames}

%Similar to \opt{minnames} but affects only the citations in the document body.

类似于 \opt{minnames} 但只影响正文中的标注(引用)。

\intitem[3]{maxitems}

%Similar to \opt{maxnames}, but affecting all literal lists (\bibfield{publisher}, \bibfield{location}, etc.).

类似于 \opt{maxnames} 但影响所有的文本列表(\bibfield{publisher}、\bibfield{location}等)。

\intitem[1]{minitems}

%Similar to \opt{minnames}, but affecting all literal lists (\bibfield{publisher}, \bibfield{location}, etc.).

类似于 \opt{minnames} 但影响所有的文本列表(\bibfield{publisher}、\bibfield{location}等)。

\optitem{autocite}{\opt{plain}, \opt{inline}, \opt{footnote}, \opt{superscript}, \opt{...}}

%This option controls the behavior of the \cmd{autocite} command discussed in \secref{use:cit:aut}. The \opt{plain} option makes \cmd{autocite} behave like \cmd{cite}, \opt{inline} makes it behave like \cmd{parencite}, \opt{footnote} makes it behave like \cmd{footcite}, and \opt{superscript} makes it behave like \cmd{supercite}. The options \opt{plain}, \opt{inline}, and \opt{footnote} are always available, the \opt{superscript} option is only provided by the numeric citation styles which come with this package. The citation style may also define additional options. The default setting of this option depends on the selected citation style, see \secref{use:xbx:cbx}.
该选项控制 \secref{use:cit:aut} 节中讨论的 \cmd{autocite} 命令的行为。\opt{plain} 选项使得 \cmd{autocite} 相当于 \cmd{cite};\opt{inline} 选项使得它相当于 \cmd{parencite};\opt{footnote} 选项使得它相当于 \cmd{footcite};\opt{superscript} 选项使得它相当于 \cmd{supercite}。
选项 \opt{plain}、\opt{inline} 和 \opt{footnote} 总是可用的,而  \opt{superscript} 只能用于本宏包所带的顺序编码制标注样式中。标注样式也可以定义其它选项值。该选项的默认设置取决于所选的标注样式,参考 \secref{use:xbx:cbx} 节。

\boolitem[true]{autopunct}

%This option controls whether the citation commands scan ahead for punctuation marks. See \secref{use:cit} and \cmd{DeclareAutoPunctuation} in \secref{aut:pct:cfg} for details.

该选项控制标注(引用)命令是否区分标点。详见 \secref{use:cit} 节和 \secref{aut:pct:cfg} 节中的命令 \cmd{DeclareAutoPunctuation}。

\optitem[autobib]{language}{\opt{autobib}, \opt{autocite}, \opt{auto}, \prm{language}}

%This option controls multilingual support. When set to \opt{autobib}, \opt{autocite} or \opt{auto}, \biblatex will try to get the main document language from the \sty{babel}/\sty{polyglossia} package (and fall back to English if \sty{babel}/\sty{polyglossia} is not available). It is also possible to select the package language manually. In this case, the language chosen will override the \bibfield{langid} of entries and you should still choose a language switching environment with the \opt{autolang} option to select how the switch to the manually chosen language is handled. Please refer to \tabref{bib:fld:tab1} for a list of supported languages and the corresponding identifiers. \opt{autobib} switches the language for each entry in the bibliography using the \bibfield{langid} field and the language environment specified by the \opt{autolang} option. \opt{autocite} switches the language for each citation using the \bibfield{langid} field and the language environment specified by the \opt{autolang} option. \opt{auto} is a shorthand to set both \opt{autobib} and \opt{autocite}.
该选项控制多语种支持功能。当其设置为 \opt{autobib}、\opt{autocite} 或 \opt{auto} 时,\biblatex 会尝试从 \sty{babel}/\sty{polyglossia} 宏包中获取文档的主语言(如果\sty{babel}/\sty{polyglossia} 宏包不可用则设置为后备的英语)。也可以手动选择文档的语言,手动选择的语言会覆盖条目的 \bibfield{langid} 域,并且仍需要使用 \opt{autolang} 选项选择语言切换环境,以确定如何处理手动选择语言的切换方式。关于所支持的语言和相应的标识符请参考 \tabref{bib:fld:tab1}。
\opt{autobib} 使用条目中的 \bibfield{langid} 域和 \opt{autolang} 选项确定的语言环境为文献表中每个条目切换语言。
\opt{autocite} 使用条目中的 \bibfield{langid} 域和 \opt{autolang} 选项确定的语言环境为每个标注(引用)切换语言。
而 \opt{auto} 是同时设置 \opt{autobib} 和 \opt{autocite} 的缩略语。

\boolitem[true]{clearlang}

%If this option is enabled, \biblatex will automatically clear the \bibfield{language} field of all entries whose language matches the \sty{babel}/\sty{polyglossia} language of the document (or the language specified explicitly with the \opt{language} option) in order to omit redundant language specifications. The language mappings required by this feature are provided by the \cmd{DeclareRedundantLanguages} command from \secref{aut:lng:cmd}.

如果激活该选项,\biblatex 会自动清除那些与文档的 \sty{babel}/\sty{polyglossia} 语言(或者由 \opt{language} 选项显式指定的语言)相匹配的所有条目的 \bibfield{language} 域,以便略去冗余的语言设定。
该功能所需的语言映射由 \secref{aut:lng:cmd} 节的 \cmd{DeclareRedundantLanguages} 命令提供。

\optitem[none]{autolang}{\opt{none}, \opt{hyphen}, \opt{other}, \opt{other*}, \opt{langname}}

%This option controls which \sty{babel} language environment\footnote{\sty{polyglossia} understands the \sty{babel} language environments too and so this option controls both the \sty{babel} and \sty{polyglossia} language environments.} is used if the \sty{babel}/\sty{polyglossia} package is loaded and a bibliography entry includes a \bibfield{langid} field (see \secref{bib:fld:spc}). Note that \biblatex automatically adjusts to the main document language if \sty{babel}/\sty{polyglossia} is loaded. In multilingual documents, it will also continually adjust to the current language as far as citations and the default language of the bibliography is concerned. The effect of language adjustment depends on the language environment selected by this option. The possible choices are:
如果载入了 \sty{babel}/\sty{polyglossia} 宏包并且文献条目包含 \bibfield{langid} 域
(见 \secref{bib:fld:spc} 节),那么该选项可以控制使用哪种 \sty{babel} 语言环境
\footnote{	\sty{polyglossia} 宏包也可以理解 \sty{babel} 的语言环境,	因此本选项可以同时控制 \sty{babel}/\sty{polyglossia} 语言环境。}。
请注意,当载入 \sty{babel}/\sty{polyglossia} 宏包时 \biblatex 会自动适应主文档的语言。
在多语言文档中,只要涉及到标注(引用)和文献表的默认语言,本宏包也会持续地调整以适应当前语言。
切换语言的效果取决于该选项选择的语言环境,可用的选择有:

\begin{valuelist}

\item[none]
%Disable this feature, \ie do not use any language environment at all.

关闭该功能,也就是不使用任何语言环境。

\item[hyphen]
%Enclose the entry in a \env{hyphenrules} environment. This will load hyphenation patterns for the language specified in the \env{hyphenation} field of the entry, if available.
将条目装入  \env{hyphenrules} 环境中。如果可用的话,会为条目的 \env{hyphenation} 域指定的语言导入断词模式。

\item[other]
%Enclose the entry in an \env{otherlanguage} environment. This will load hyphenation patterns for the specified language, enable all extra definitions which \sty{babel}/\sty{polyglossia} and \biblatex provide for the respective language, and translate key terms such as <editor> and <volume>. The extra definitions include localisations of the date format, of ordinals, and similar things.

将条目装入 \env{otherlanguage} 环境中。
这将导入特定语言的断词模式,激活 \sty{babel}/\sty{polyglossia} 和 \biblatex 为相应语言提供的所有额外定义,并翻译“editor”和“volume”等键项。这些额外定义包括日期格式、序数和其它类似内容的本地化。

\item[other*]
%Enclose the entry in an \env{otherlanguage*} environment. Please note that \sty{biblatex} treats \env{otherlanguage*} like \env{otherlanguage} but other packages may make a distinction in this case.
将条目装入 \env{otherlanguage*} 环境中。
请注意,此时 \biblatex 将 \env{otherlanguage*} 环境视为 \env{otherlanguage} 环境但其它宏包不会。

\item[langname]
%\sty{polyglossia} only. Enclose the entry in a \env{$<$languagename$>$} environment. The benefit of this option value for \sty{polyglossia} users is that it takes note of the \bibfield{langidopts} field so that you can add per-language options to an entry (like selecting a language variant). When using \sty{babel}, this option does the same as the \opt{other} option value.
只用于 \sty{polyglossia} 宏包。将条目装入 \env{\prm{languagename}} 环境中。
该选项值对 \sty{polyglossia} 用户的好处是它会留意 \bibfield{langidopts} 域,
这样可以为一个条目增加语言相关选项(类似于选择语言变种)。当使用 \sty{babel} 时,该选项值与 \opt{other} 选项值相同。

\end{valuelist}

\optitem[none]{block}{\opt{none}, \opt{space}, \opt{par}, \opt{nbpar}, \opt{ragged}}

%This option controls the extra spacing between blocks, \ie larger segments of a bibliography entry. The possible choices are:
该选项用于控制块之间的额外空白,即文献表中条目内大的分段(块(block)相比于单元(unit)更大)。可用的选项值有:

\begin{valuelist}

\item[none] %Do not add anything at all.
不添加任何东西。

\item[space] %Insert additional horizontal space between blocks. This is similar to the default behavior of the standard \latex document classes.
在块之间添加额外的水平间距,类似于标准 \LaTeX 文档类的默认行为。

\item[par] %Start a new paragraph for every block. This is similar to the \opt{openbib} option of the standard \latex document classes.
每一块都另起一段,类似于标准 \LaTeX 文档类的 \opt{openbib} 选项。

\item[nbpar] %Similar to the \opt{par} option, but disallows page breaks at block boundaries and within an entry.
类似于 \opt{par} 选项但是不允许在块的边界和条目内部分页。

\item[ragged] %Inserts a small negative penalty to encourage line breaks at block boundaries and sets the bibliography ragged right.
插入一个小的负的断行罚值以鼓励在块边界处断行并设置左对齐。

\end{valuelist}

%The \cmd{newblockpunct} command may also be redefined directly to achieve different results, see \secref{use:fmt:fmt}. Also see \secref{aut:pct:new} for additional information.

也可以直接重新定义 \cmd{newblockpunct} 命令实现不同的效果,见  \secref{use:fmt:fmt} 节。
更多信息见 \secref{aut:pct:new} 节。

\optitem[foot+end]{notetype}{\opt{foot+end}, \opt{footonly}, \opt{endonly}}

%This option controls the behavior of \cmd{mkbibfootnote}, \cmd{mkbibendnote}, and similar wrappers from \secref{aut:fmt:ich}. The possible choices are:

该选项控制 \secref{aut:fmt:ich} 节中 \cmd{mkbibfootnote}、\cmd{mkbibendnote} 和类似封套的行为。
可用的选项值有:

\begin{valuelist}
\item[foot+end] %Support both footnotes and endnotes, \ie \cmd{mkbibfootnote} will generate footnotes and \cmd{mkbibendnote} will generate endnotes.
同时支持脚注和尾注,即,\cmd{mkbibfootnote} 会生成脚注而 \cmd{mkbibendnote} 会生成尾注。
\item[footonly] %Force footnotes, \ie make \cmd{mkbibendnote} generate footnotes.
强制脚注,即,\cmd{mkbibendnote} 也生成脚注。
\item[endonly] %Force endnotes, \ie make \cmd{mkbibfootnote} generate endnotes.
强制尾注,即,\cmd{mkbibfootnote} 也生成尾注。
\end{valuelist}

\optitem[auto]{hyperref}{\opt{true}, \opt{false}, \opt{auto}}

%Whether or not to transform citations and back references into clickable hyperlinks. This feature requires the \sty{hyperref} package. It also requires support by the selected citation style. All standard styles which ship with this package support hyperlinks. \kvopt{hyperref}{auto} automatically detects if the \sty{hyperref} package has been loaded.
是否将标注(引用)和反向引用转化为可点击的超链接。这一功能需要 \sty{hyperref} 宏包。也需要标注样式的支持。
本宏包所带的所有标准样式都支持超链接。\kvopt{hyperref}{auto} 会自动检测 \sty{hyperref} 宏包是否已载入。

\boolitem[false]{backref}

%Whether or not to print back references in the bibliography. The back references are a list of page numbers indicating the pages on which the respective bibliography entry is cited. If there are \env{refsection} environments in the document, the back references are local to the reference sections. Strictly speaking, this option only controls whether the \biblatex package collects the data required to print such references. This feature still has to be supported by the selected bibliography style. All standard styles which ship with this package do so.
是否在文献表中打印出反向引用。反向引用是一组标明引用相应文献所在页码构成的列表。
如果在文档中有 \env{refsection} 环境,反向引用是局部的,针对相应的参考文献分节(refsection)。
严格地讲,该选项只控制 \biblatex 是否收集所需的数据。该功能也需要文献样式的支持。
本宏包所带的所有标准样式都支持该功能。

\optitem[three]{backrefstyle}{\opt{none}, \opt{three}, \opt{two}, \opt{two+}, \opt{three+}, \opt{all+}}

%This option controls how sequences of consecutive pages in the list of back references are formatted. The following styles are available:
该选项控制反向引用中的连续页码的格式。可用的样式有:

\begin{valuelist}

\item[none] %Disable this feature, \ie do not compress the page list.
不启用该特性,即,不压缩页码列表。

\item[three] %Compress any sequence of three or more consecutive pages to a range, \eg the list <1, 2, 11, 12, 13, 21, 22, 23, 24> is compressed to <1, 2, 11--13, 21--24>.
将任意连续三页或更多页缩写为页码范围,例如,“1, 2, 11, 12, 13, 21, 22, 23, 24” 会压缩成“1, 2, 11--13, 21--24”。

\item[two] %Compress any sequence of two or more consecutive pages to a range, \eg the above list is compressed to <1--2, 11--13, 21--24>.
将任意连续两页或更多页缩写成页码范围,例如上面的例子会变成“1--2, 11--13, 21--24”。

\item[two+] %Similar in concept to \opt{two} but a sequence of exactly two consecutive pages is printed using the starting page and the localization string \texttt{sequens}, \eg the above list is compressed to <1\,sq., 11--13, 21--24>.
概念类似于 \opt{two},但是连续两页的打印格式为开始页和本地化字符串 \texttt{sequens},例如上面的例子会变成“1\,sq., 11--13, 21--24”。

\item[three+] %Similar in concept to \opt{two+} but a sequence of exactly three consecutive pages is printed using the starting page and the localization string \texttt{sequentes}, \eg the above list is compressed to <1\,sq., 11\,sqq., 21--24>.
概念类似于 \opt{two+},但是连续三页的打印格式为开始页可本地化字符串 \texttt{sequentes},例如上面的例子会变成“1\,sq., 11\,sqq., 21--24”。

\item[all+] %Similar in concept to \opt{three+} but any sequence of consecutive pages is printed as an open-ended range, \eg the above list is compressed to <1\,sq., 11\,sqq., 21\,sqq.>.
概念类似于 \opt{three+},但是任意连续页码将打印成末端不封闭的形式,例如上面的例子会变成“1\,sq., 11\,sqq., 21\,sqq.”。

\end{valuelist}

%All styles support both Arabic and Roman numerals. In order to avoid potentially ambiguous lists, different sets of numerals will not be mixed when generating ranges, \eg the list <iii, iv, v, 6, 7, 8> is compressed to <iii--v, 6--8>.
所有这些样式都同时支持阿拉伯和罗马数字。为了避免可能的歧义,不同数字集在生成数字范围时不会混在一起,
例如,“iii, iv, v, 6, 7, 8”将缩写为“iii--v, 6--8”。

\optitem[setonly]{backrefsetstyle}{\opt{setonly}, \opt{memonly}, \opt{setormem}, \opt{setandmem}, \opt{memandset}, \opt{setplusmem}}

%This option controls how back references to \bibtype{set} entries and their members are handled. The following options are available:
该选项控制怎样处理指向 \bibtype{set} 条目及其成员的反向引用。可用选项有:

\begin{valuelist}

\item[setonly] %All back references are added to the \bibtype{set} entry. The \bibfield{pageref} lists of set members remain blank.
所有的反向引用都添加到 \bibtype{set} 条目中。
而其成员的 \bibfield{pageref} 列表为空。

\item[memonly] %References to set members are added to the respective member. References to the \bibtype{set} entry are added to all members. The \bibfield{pageref} list of the \bibtype{set} entry remains blank.
条目集成员的引用添加到各自成员中。
\bibtype{set} 条目的引用添加到所有成员中。
\bibtype{set} 条目的 \bibfield{pageref} 列表为空。

\item[setormem] %References to the \bibtype{set} entry are added to the \bibtype{set} entry. References to set members are added to the respective member.
\bibtype{set} 条目的引用添加到 \bibtype{set} 条目中。
其成员的引用添加到各自成员中。

\item[setandmem] %References to the \bibtype{set} entry are added to the \bibtype{set} entry. References to set members are added to the respective member and to the \bibtype{set} entry.
\bibtype{set} 条目的引用添加到  \bibtype{set} 条目中。
其成员的引用添加到各自成员和 \bibtype{set} 条目中。

\item[memandset] %References to the \bibtype{set} entry are added to the \bibtype{set} entry and to all members. References to set members are added to the respective member.
\bibtype{set} 条目的引用添加到  \bibtype{set} 条目和所有成员中。
其成员的引用添加到各自成员中。

\item[setplusmem] %References to the \bibtype{set} entry are added to the \bibtype{set} entry and to all members. References to set members are added to the respective member and to the \bibtype{set} entry.
\bibtype{set} 条目的引用添加到  \bibtype{set} 条目和所有成员中。
其成员的引用添加到各自成员和 \bibtype{set} 条目中。

\end{valuelist}

\optitem[false]{indexing}{\opt{true}, \opt{false}, \opt{cite}, \opt{bib}}

%This option controls indexing in citations and in the bibliography. More precisely, it affects the \cmd{ifciteindex} and \cmd{ifbibindex} commands from \secref{aut:aux:tst}. The option is settable on a global, a per-type, or on a per-entry basis. The possible choices are:
该选项控制文献表和标注(引用)中的索引。
更准确地说,它影响 \secref{aut:aux:tst} 节的 \cmd{ifciteindex} 和 \cmd{ifbibindex} 命令。
该选项可以全局、针对某一类型或针对某一条目进行设置。
可用的选择有:

\begin{valuelist}
\item[true] %Enable indexing globally.
全局激活索引。
\item[false] %Disable indexing globally.
全局不激活索引。
\item[cite] %Enable indexing in citations only.
只在标注中激活索引。
\item[bib] %Enable indexing in the bibliography only.
只在文献表中激活索引。
\end{valuelist}

%This feature requires support by the selected citation style. All standard styles which ship with this package support indexing of both citations and entries in the bibliography. Note that you still need to enable indexing globally with \cmd{makeindex} to get an index.
该特性需要标注样式的支持。本宏包所带的所有标准样式都支持标注和文献条目中的索引。
请注意,为了得到索引表,仍然需要用 \cmd{makeindex} 命令全局激活索引模式。

\boolitem[false]{loadfiles}

%This option controls whether external files requested by way of the \cmd{printfile} command are loaded. See also \secref{use:use:prf} and \cmd{printfile} in \secref{aut:bib:dat}. Note that this feature is disabled by default for performance reasons.
该选项控制是否载入 \cmd{printfile} 命令请求的外部文件。另参考 \secref{use:use:prf} 和 \secref{aut:bib:dat} 节中的 \cmd{printfile} 命令。请注意,出于性能考虑,该特性默认不激活。

\optitem[none]{refsection}{\opt{none}, \opt{part}, \opt{chapter}, \opt{section}, \opt{subsection}}

%This option automatically starts a new reference section at a document division such as a chapter or a section. This is equivalent to the \cmd{newrefsection} command, see \secref{use:bib:sec} for details. The following choice of document divisions is available:
该选项自动在文档划分时(例如一章或一节)开始一个新的参考文献分节。这等价于 \cmd{newrefsection} 命令,参考 \secref{use:bib:sec} 节。可用的文档划分如下:

\begin{valuelist}
\item[none] %Disable this feature.
不启用该特性。
\item[part] %Start a reference section at every \cmd{part} command.
在每个 \cmd{part} 命令处开始一个参考文献分节。
\item[chapter] %Start a reference section at every \cmd{chapter} command.
在每个 \cmd{chapter} 命令处开始一个参考文献分节。
\item[section] %Start a reference section at every \cmd{section} command.
在每个 \cmd{section} 命令处开始一个参考文献分节。
\item[subsection] %Start a reference section at every \cmd{subsection} command.
在每个 \cmd{subsection} 命令处开始一个参考文献分节。
\end{valuelist}
%
%The starred versions of these commands will not start a new reference section.
对应这些命令的带星号版本不会开始新的参考文献分节。

\optitem[none]{refsegment}{\opt{none}, \opt{part}, \opt{chapter}, \opt{section}, \opt{subsection}}

%Similar to the \opt{refsection} option but starts a new reference segment. This is equivalent to the \cmd{newrefsegment} command, see \secref{use:bib:seg} for details. When using both options, note that you can only apply this option to a lower"=level document division than the one \opt{refsection} is applied to and that nested reference segments will be local to the enclosing reference section.
类似于 \opt{refsection} 选项,但是开始一个新的参考文献片段。这等价于 \cmd{newrefsegment} 命令,详见 \secref{use:bib:seg} 节。当两个选项都使用时,请注意该选项只能应用到比 \opt{refsection} 选项应用的文档划分更低层划分中,同时,嵌套的参考文献片段相对于所附属的参考文献分节来讲是局部的。

\optitem[none]{citereset}{\opt{none}, \opt{part}, \opt{chapter}, \opt{section}, \opt{subsection}}

%This option automatically executes the \cmd{citereset} command from \secref{use:cit:msc} at a document division such as a chapter or a section. The following choice of document divisions is available:

该选项在文档划分处(例如一章或一节)自动执行 \secref{use:cit:msc} 节介绍的 \cmd{citereset} 命令。可用的文档分段有:

\begin{valuelist}
\item[none] %Disable this feature.
不启用该特性。
\item[part] %Perform a reset at every \cmd{part} command.
在每个 \cmd{part}  命令后执行重置。
\item[chapter] %Perform a reset at every \cmd{chapter} command.
在每个 \cmd{chapter}  命令后执行重置。
\item[section] %Perform a reset at every \cmd{section} command.
在每个 \cmd{section}  命令后执行重置。
\item[subsection] %Perform a reset at every \cmd{subsection} command.
在每个 \cmd{subsection}  命令后执行重置。
\end{valuelist}
%
%The starred versions of these commands will not trigger a reset.
这些命令对应的带星号版本不会引起重置。

\boolitem[true]{abbreviate}

%Whether or not to use long or abbreviated strings in citations and in the bibliography. This option affects the localisation modules. If this option is enabled, key terms such as <editor> are abbreviated. If not, they are written out.
是否在标注和文献表中使用长字符串或缩写字符串。该选项会影响本地化模块。如果启用该选项,“editor”等键值项会被缩写,反之则会全部写出。

\optitem[comp]{date}{\opt{year}, \opt{short}, \opt{long}, \opt{terse}, \opt{comp}, \opt{ymd}, \opt{edtf}}

%This option controls the basic format of printed date specifications. The following choices are available:

该选项控制日期规范的基本格式。有以下选择:

\begin{valuelist}
\item[year] %Use only years, for example:\par
只使用年份,例如:\par
2010\par
2010--2012\par
\item[short] %Use the short format with verbose ranges, for example:\par
使用详细日期范围的短格式,例如:\par
01/01/2010\par
21/01/2010--30/01/2010\par
01/21/2010--01/30/2010
\item[long] %Use the long format with verbose ranges, for example:\par
使用详细的日期范围的长格式,例如:\par
1st January 2010\par
21st January 2010--30th January 2010\par
January 21, 2010--January 30, 2010\par
\item[terse] %Use the short format with compact ranges, for example:\par
使用压缩日期范围的短格式,例如:\par
21--30/01/2010\par
01/21--01/30/2010
\item[comp] %Use the long format with compact ranges, for example:\par
使用压缩日期范围长格式,例如:\par
21st--30th January 2010\par
January 21--30, 2010\par
\item[iso] Use ISO8601 Extended Format (\texttt{yyyy-mm-dd}), for example:\par
使用ISO8601扩展格式(\texttt{yyyy-mm-dd}),例如:\par
2010-01-01\par
2010-01-21/2010-01-30
\item[ymd] %A year-month-day format which can be modified by other options unlike strict \acr{ISO8601-2}, for example:\par
不同于严格的\acr{ISO8601-2}格式,使用可以被其它选项修改的年-月-日格式,例如:\par
2010-1-1\par
2010-1-21/2010-1-30
\end{valuelist}
%
%Note that \opt{iso} format will enforce \kvopt{dateera}{astronomical}, \kvopt{datezeros}{true}, \kvopt{timezeros}{true}, \kvopt{seconds}{true}, \kvopt{$<$datetype$>$time}{24h} and \kvopt{julian}{false}. \opt{ymd} is an ETFT-like format but which can change the various options which the strict \opt{iso} option does not allow for.
注意,\opt{iso} 格式会强制开启 \kvopt{dateera}{astronomical}, \kvopt{datezeros}{true}, \kvopt{timezeros}{true}, \kvopt{seconds}{true}, \kvopt{\prm{datetype}time}{24h} 以及 \kvopt{julian}{false} 等键值。\opt{ymd} 是ETFT类格式,但可以由不同的选项做出改变,而这是\opt{iso}选项不允许的。

%As seen in the above examples, the actual date format is language specific. Note that the month name in all long formats is responsive to the \opt{abbreviate} package option. The leading zeros for months and days in all short formats may be controlled separately with the \opt{datezeros} package option. The leading zeros for hours, minutes and seconds in all short formats may be controlled separately with the \opt{timezeros} package option. If outputting times, the printing of seconds and timezones is controlled by the \opt{seconds} and \opt{timezones} options respectively.
从上述例子可以看出,实际的日期格式是与语言相关的。请注意,所有长格式中的月份名受 \opt{abbreviate} 宏包选项影响。所有短格式中月和日的首位零可以另外由 \opt{datezeros} 宏包选项控制。所有短格式中时分秒的首位零可以另外由 \opt{timezeros} 宏包选项控制。如果要输出时刻,那么秒和时区的打印分别由 \opt{seconds} 和 \opt{timezones} 选项控制。

%The options \opt{julian} and \opt{gregorianstart}  may be used to control when to output Julian Calendar dates.

\opt{julian} 和 \opt{gregorianstart} 选项可以用于控制何时输出儒略历日期。

\optitem[year]{labeldate}{\opt{year}, \opt{short}, \opt{long}, \opt{terse}, \opt{comp}, \opt{ymd}, \opt{edtf}}

%Similar to the \opt{date} option but controls the format of the date field selected with \cmd{DeclareLabeldate}.
类似于 \opt{date} 选项但是控制由 \cmd{DeclareLabeldate} 选择的日期域的格式。

\optitem[comp]{\prm{datetype}date}{\opt{year}, \opt{short}, \opt{long}, \opt{terse}, \opt{comp}, \opt{ymd}, \opt{edtf}}

%Similar to the \opt{date} option but controls the format of the \bibfield{$<$datetype$>$date} field in the datamodel.
类似于 \opt{date} 选项但是控制数据模型中 \bibfield{\prm{datetype}date} 域的格式。

\optitem{alldates}{\opt{year}, \opt{short}, \opt{long}, \opt{terse}, \opt{comp}, \opt{iso}}

%Sets the option for all dates in the datamodel to the same value. The date fields in the default data model are \bibfield{date}, \bibfield{origdate}, \bibfield{eventdate} and \bibfield{urldate}.
将数据模型中所有日期的选项设置为同一值。默认数据模型中的日期域为 \bibfield{date}、\bibfield{origdate}、\bibfield{eventdate} 和 \bibfield{urldate}。

\boolitem[false]{julian}

%This option controls whether dates before the date specified in the \opt{gregorianstart} option will be converted automatically to the Julian Calendar. Dates so changed will return <true> for the \cmd{ifdatejulian} and \cmd{if$<$datetype$>$datejulian} tests (see \secref{aut:aux:tst}). Please bear in mind that dates consisting of just a year like <1565> will never be converted to a Julian Calendar date because a date without a month and day has an ambiguous Julian Calendar representation\footnote{This is potentially true for dates missing times too but this is not relevant for bibliographic work.}. For example, in the case of <1565>, this is Julian year <1564> until after the Gregorian date <10th January 1565> when the Julian year becomes <1565>.
该选项控制是否将由 \opt{gregorianstart} 选项指定日期之前的日期自动转换为儒略历。改变的日期在 \cmd{ifdatejulian} 和 \cmd{if\prm{datetype}datejulian} 测试下会返回“true”(见 \secref{aut:aux:tst} 节)。请谨记,只包含年份的日期不会转换为儒略历,例如“1565”,这是因为没有月日的日期在儒略历表示下会引起混乱
\footnote{缺失时刻的日期也有可能出现这一问题,不过对于文献作品问题不大。}
例如,在“1565”的例子中,在公历(格里高利历)“1565年1月10日”之后的日期是儒略历“1565”年,而之前的日期是儒略历“1564”年。

\valitem{gregorianstart}{YYYY-MM-DD}

%This option controls the date before which dates are converted to the Julian Calendar. It is a strict format string, 4-digit year, 2-digit month and day, separated by a single dash character (any valid Unicode character with the <Dash> property). The default is '1582-10-15', the date of the instigation of the standard Gregorian Calendar. This option does not nothing unless \opt{julian} is set to <true>.
该选项控制在哪一日期之前的日期可以转换为儒略历。选项值有严格的字符串格式:4位的年份、2位的月份和日期,并且由短划线(或者具有“Dash”属性的任何有效Unicode字符)分隔。默认值是“1582-10-15”,即标准公历(格里高利历)的颁布日期。只有 \opt{julian} 设置为 “true”时本选项才起作用。

\boolitem[true]{datezeros}

%This option controls whether \texttt{short} and \texttt{terse} date components are printed with leading zeros unless overridden by specific formatting.
该选项控制当没有特定格式覆盖时,打印 \texttt{short} 和 \texttt{terse} 日期成分是否首位补零。

\boolitem[true]{timezeros}

%This option controls whether time components are printed with leading zeros unless overridden by specific formatting.
该选项控制没有覆盖特定格式时,打印时刻成分是否首位补零。

\boolitem[false]{timezones}

%This option controls whether timezones are printed when printing times.
该选项控制打印时刻时是否输出时区。

\boolitem[false]{seconds}

%This option controls whether seconds are printed when printing times.
该选项控制打印时刻时是否输出秒。

\boolitem[true]{dateabbrev}

%This option controls whether \texttt{long} and \texttt{comp} dates are printed with long or abbreviated month/season names. The option is similar to the generic \opt{abbreviate} option but specific to the date formatting.
该选项控制打印\texttt{long} 和 \texttt{comp} 日期格式时,使用完整的或是缩写的月份名。该选项类似于一般的 \opt{abbreviate} 选项但是只针对日期格式。

\boolitem[false]{datecirca}

%This option controls whether to output <circa> information about dates. If set to \opt{true}, dates will be preceded by the expansion of the \cmd{datecircaprint} macro (\secref{use:fmt:fmt}).
该选项控制是否输出日期的“circa”信息。如果设置为 \opt{true},那么日期将由\secref{use:fmt:fmt} 节的\cmd{datecircaprint} 宏的展开来引导。

\boolitem[false]{dateuncertain}

%This option controls whether to output uncertainty information about dates. If set to \opt{true}, dates will be followed by the expansion of the \cmd{dateuncertainprint} macro and end dates will be followed by the \cmd{enddateuncertainprint} macro (\secref{use:fmt:fmt}).
该选项控制是否输出日期的不确定信息。如果设置为 \opt{true},那么日期将由 \cmd{dateuncertainprint} 宏的展开来引导,
并由 \cmd{enddateuncertainprint} 宏结束,见 \secref{use:fmt:fmt} 节。

\optitem[astronomical]{dateera}{\opt{astronomical}, \opt{secular}, \opt{christian}}

%This option controls how date era information is printed. <astronomical> uses \cmd{dateeraprintpre} to print era information \emph{before} start/end dates. <secular> and <christian> uses \cmd{dateeraprint} to print era information \emph{after} the start/end/dates. By default <astronomical> results in a minus sign before BCE/BC dates and <secular>/<christian> results in the relevant localisation strings like <BCE> or <BC> after BCE/BC dates. See the relevant comments in \secref{use:fmt:fmt} and the localisation strings in \secref{aut:lng:key:dt}.
该选项控制如何打印日期纪元信息。选项值“astronomical”使用 \cmd{dateeraprintpre} 命令在起止日期\emph{之前}打印纪元信息。
而选项值“secular”和“christian”使用 \cmd{dateeraprint} 命令在起止日期\emph{之后}打印纪元信息。
缺省情况下,使用“astronomical”效果是在公元前日期之前使用负号,而使用“<secular>”或“<christian>”的效果是在公元前日期之后加上“BCE”或“BC”等相关的本地化字符串。见 \secref{use:fmt:fmt} 节的有关评论以及 \secref{aut:lng:key:dt} 节的本地化字符串。

\intitem[0]{dateeraauto}

%This option sets the astronomical year, below which era localisation strings are automatically added. This option does nothing without \opt{dateera} being set to <secular> or <christian>.
该选项设置回归年,在其之前自动添加纪元的本地化字符串。只有当 \opt{dateera} 设置为“secular”或者“christian”时本选项才起作用。

\optitem[24h]{time}{\opt{12h}, \opt{24h}, \opt{24hcomp}}

%This option controls the basic format of printed time specifications. The following choices are available:

该选项控制时刻规范的基本格式。
可用的选择有:

\begin{valuelist}
	\item[24h] %24-hour format, for example:\par
	24小时格式,例如:\par
	14:03:23\par
	14:3:23\par
	14:03:23+05:00\par
	14:03:23Z\par
	14:21:23--14:23:45\par
	14:23:23--14:23:45\par
	\item[24hcomp] %24-hour format with compressed ranges, for example:\par
	带有缩写范围的24小时格式,例如:\par
	14:21--23 (小时相同)\par %(hours are the same)\par
	14:23:23--45 (小时和分钟相同)\par %(hour and minute are the same)\par
	\item[12h] %12-hour format with (localised) AM/PM markers, for example:\par
	带有本地化上下午标识符的12小时格式,例如:\par
	2:34 PM\par
	2:34 PM--3:50 PM\par
\end{valuelist}
%
%As seen in the above examples, the actual time format is language specific. Note that the AM/PM string is responsive to the \opt{abbreviate} package option, if this makes a difference in the specific locale. The leading zeros in the 24-hour formats may be controlled separately with the \opt{timezeros} package option. The separator between time components (\cmd{bibtimesep} and \cmd{bibtzminsep})and between the time and any timezone (\cmd{bibtimezonesep}) are also language specific and customisable, see \secref{use:fmt:lng}. There are global package options which determine whether seconds and timezones are printed (\opt{seconds} and \opt{timezones}, respectively, see \secref{use:opt:pre:gen}). Timezones, if present, are either <Z> or a numeric positive or negative offset. No default styles print time information. Custom styles may print times by using the \cmd{print<datetype>time} commands, see \secref{aut:bib:dat}.
从以上例子可以看出,实际的时刻格式是与语言相关的。请注意,如果与指定的区域不同,那么上下午(AM/PM)字符串受 \opt{abbreviate} 宏包选项影响。24小时格式首位补零的话可以单独用 \opt{timezeros} 宏包选项控制。此外与语言相关并可以单独定制的还有时刻成分之间的分隔符\cmd{bibtimesep}、\cmd{bibtzminsep},以及时刻与时区之间的分隔符 \cmd{bibtimezonesep},见 \secref{use:fmt:lng} 节。还有一些全局的宏包选项可以控制是否打印秒和时区(分别是 \opt{seconds} 和 \opt{timezones},见 \secref{use:opt:pre:gen} 节)。如果有时区的话,要么是字符`Z',要么是表示正负偏移量的数值。标准样式不打印时刻信息。
定制样式可以使用 \cmd{print\prm{datetype}time} 命令打印时刻,见 \secref{aut:bib:dat} 节。

\optitem[24h]{labeltime}{\opt{12h}, \opt{24h}, \opt{24hcomp}}

%Similar to the \opt{time} option but controls the format of the time part fields obtained from the field selected with \cmd{DeclareLabeldate}.
类似于 \opt{time} 选项但是控制由 \cmd{DeclareLabeldate} 选择的域中得到的时刻部分的格式。

\optitem[24h]{\prm{datetype}time}{\opt{12h}, \opt{24h}, \opt{24hcomp}}

%Similar to the \opt{time} option but controls the format of the time part fields obtained from the \bibfield{$<$datetype$>$date} field in the datamodel.
类似于 \opt{time} 选项但是控制数据模型中 \bibfield{\prm{datetype}date} 域中得到的时刻部分格式。

\optitem{alltimes}{\opt{12h}, \opt{24h}, \opt{24hcomp}}

%Sets \opt{labeltime} and the \opt{$<$datetype$>$time} option for all times in the datamodel to the same value. The date fields supporting time parts in the default data model are \bibfield{date}, \bibfield{origdate}, \bibfield{eventdate} and \bibfield{urldate}.
为数据模型中所有的时刻设置相同的 \opt{labeltime} 和 \opt{\prm{datetype}time} 值。默认数据模型中支持时刻部分的日期域有 \bibfield{date}、\bibfield{origdate}、\bibfield{eventdate} 和 \bibfield{urldate}。

\boolitem[false]{dateusetime}

%Specifies whether to print any time component of a date field after the date component. The separator between the date and time components is \cmd{bibdatetimesep} from \secref{use:fmt:lng}. This option does nothing if a compact date format is being used (see \secref{use:opt:pre:gen}) as this would be very confusing.
确定在日期域的日期成分后是否打印时刻成分。日期和时刻成分的分隔符是  \cmd{bibdatetimesep},见 \secref{use:fmt:lng} 节。如果使用压缩的日期格式(见  \secref{use:opt:pre:gen} 节),那么该选项则不起作用,否则会引起混乱。

\boolitem[false]{labeldateusetime}

%Similar to the \opt{dateusetime} option but controls the whether to print time components for the field selected with \cmd{DeclareLabeldate}.
类似于 \opt{dateusetime} 选项,但是控制是否打印 \cmd{DeclareLabeldate} 选择的域中的时刻成分。

\boolitem[false]{\prm{datetype}dateusetime}

%Similar to the \opt{dateusetime} option but controls the whether to print time components for the \bibfield{\prm{datetype}date} field in the datamodel.
类似于 \opt{dateusetime} 选项,但是控制是否打印数据模型中 \bibfield{\prm{datetype}date} 域的时刻成分。

\boolitem[false]{alldatesusetime}

%Sets \opt{labeldateusetime} and the \opt{\prm{datetype}dateusetime} option for all \bibfield{\prm{datetype}date} fields in the datamoel.
为数据模型中所有的 \bibfield{\prm{datetype}date} 域设置 \opt{labeldateusetime} 和 \opt{\prm{datetype}dateusetime} 选项值。

\boolitem[false]{defernumbers}

%In contrast to standard \latex, the numeric labels generated by this package are normally assigned to the full list of references at the beginning of the document body. If this option is enabled, numeric labels (\ie the \bibfield{labelnumber} field discussed in \secref{aut:bbx:fld}) are assigned the first time an entry is printed in any bibliography. See \secref{use:cav:lab} for further explanation.  This option requires two \latex runs after the data has been exported to the \file{bbl} file by the backend (in addition to any other runs required by page breaks changing etc.). An important thing to note is that if you change the value of this option in your document (or the value of options which depend on this like some of the options to the \cmd{printbibliography} macro, see \secref{use:bib:bib}), then it is likely that you will need to delete your current \file{aux} file and re-run \latex to obtain the correct numbering. See \secref{aut:int}.
与标准 \LaTeX 不同,本宏包生成的顺序编码标签一般在文档正文开始处就分配给文献表的全体文献。
如果该选项被激活,各文献的顺序编码标签(也就是 \secref{aut:bbx:fld} 中讨论的 \bibfield{labelnumber} 域)只有该文献在任意文献表中被打印时才会做第一次分配。进一步解释见 \secref{use:cav:lab} 节。该选项需要在后端将数据导出到 \file{bbl} 文件后再运行两次 \LaTeX (除由分页变化等要求的编译外)。需注意的一个要点是,如果你在文档中改变了该选项值(或者那些依赖于本选项的选项值,例如与\cmd{printbibliography} 宏相关的选项,见\secref{use:bib:bib} 节),那么很可能需要删除当前的 \file{aux} 文件然后重新运行 \LaTeX 来获得正确的顺序编码,见 \secref{aut:int} 节。(\emph{需要注意:该选项只与顺序编码标签有关,而与文献表中文献的排序无关。-译者注})

\boolitem[false]{punctfont}

%This option enables an alternative mechanism for dealing with unit punctuation after a field printed in a different font (for example, a title printed in italics). See \cmd{setpunctfont} in \secref{aut:pct:new} for details.
该选项激活一个可选机制,用来处理不同字体打印的域(例如斜体的标题)之后的单位标点。详见 \secref{aut:pct:new} 节中的 \cmd{setpunctfont}。

\optitem[abs]{arxiv}{\opt{abs}, \opt{ps}, \opt{pdf}, \opt{format}}

%Path selector for arXiv links. If hyperlink support is enabled, this option controls which version of the document the arXiv \bibfield{eprint} links will point to. The following choices are available:
arXiv 链接的路径选择。如果启用超链接支持,该选项会控制 arXiv 的 \bibfield{eprint} 链接指向该文件的哪个版本。以下是可用的选择:

\begin{valuelist}
\item[abs] %Link to the abstract page.
链接到摘要页面。
\item[ps] %Link to the PostScript version.
链接到 PostScript 版本。
\item[pdf] %Link to the \pdf version.
链接到 \pdf 版本。
\item[format] %Link to the format selector page.
链接到格式选择页面。
\end{valuelist}

%See \secref{use:use:epr} for details on support for arXiv and electronic publishing information.
关于 arXiv 和电子出版信息的支持详见 \secref{use:use:epr} 节。

\optitem[auto]{texencoding}{\opt{auto}, \prm{encoding}}

%Specifies the encoding of the \file{tex} file. This option affects the data transferred from the backend to \biblatex. This corresponds to \biber's \opt{--output-encoding} option. The following choices are available:
指定 \file{tex} 文件的编码。该选项影响从后端传递给 \biblatex  的数据。该选项对应于 \biber 的 \opt{--output-encoding} 选项。可用的选择有:

\begin{valuelist}

\item[auto] %Try to auto-detect the input encoding. If the \sty{inputenc}\slash \sty{inputenx}\slash \sty{luainputenc} package is available, \biblatex will get the main encoding from that package. If not, it assumes \utf encoding if \xetex or \luatex has been detected, and Ascii otherwise.
尝试自动识别输入的编码。如果有 \sty{inputenc}\slash\sty{inputenx}\slash\sty{luainputenc} 等宏包,
\biblatex 会从宏包中获取主要编码。否则,当探测到 \XeTeX 或 \LuaTeX 引擎时设定为 \utf 编码,其余情况设为 Ascii。

\item[\prm{encoding}] %Specifies the \prm{encoding} explicitly. This is for odd cases in which auto-detection fails or you want to force a certain encoding for some reason.
显式指定编码为 \prm{encoding}。少数情况下自动检测失败,或你出于某种原因想强制使用某一编码,那么此时可以使用该选项。

\end{valuelist}
%
%Note that setting \kvopt{texencoding}{\prm{encoding}} will also affect the \opt{bibencoding} option if \kvopt{bibencoding}{auto}.
请注意如果\kvopt{bibencoding}{auto},那么设置 \kvopt{texencoding}{\prm{encoding}} 也会影响 \opt{bibencoding} 选项。

\optitem[auto]{bibencoding}{\opt{auto}, \prm{encoding}}

%Specifies the encoding of the \file{bib} files. This corresponds to \biber's \opt{--input-encoding} option. The following choices are available:
指定 \file{bib} 文件的编码。该选项对应于 \biber 程序的 |--output-encoding| 选项。可用选择有:

\begin{valuelist}

\item[auto] %Use this option if the workflow is transparent, \ie if the encoding of the \file{bib} file is identical to the encoding of the \file{tex} file.
当工作流是透明时,即,当 \file{bib} 文件的编码与 \file{tex} 文件的编码相同时使用该选项。

\item[\prm{encoding}] %If the encoding of the \file{bib} file is different from the one of the \file{tex} file, you need to specify it explicitly.
如果 \file{bib} 文件的编码与 \file{tex} 不同,你需要显式地指定编码。

\end{valuelist}

%By default, \biblatex assumes that the \file{tex} file and the \file{bib} file use the same encoding (\kvopt{bibencoding}{auto}).
默认情况下,\biblatex 假定 \file{tex} 和 \file{bib} 文件使用相同的编码(\kvopt{bibencoding}{auto})。

\boolitem[false]{safeinputenc}

%If this option is enabled, \biblatex will automatically force \kvopt{texencoding}{ascii} if the \sty{inputenc}\slash \sty{inputenx} package has been loaded and the input encoding is \utf, \ie it will ignore any macro-based \utf support and use Ascii only. \biber will then try to convert any non-Ascii data in the \file{bib} file to Ascii. For example, it will convert \texttt{\d{S}} to |\d{S}|. See \secref{bib:cav:enc:enc} for an explanation of why you may want to enable this option.
如果启用该选项,\biblatex 会在载入 \sty{inputenc}\slash \sty{inputenx} 宏包并且输入代码是 \utf 时自动加入 \kvopt{texencoding}{ascii},也就是说,它会忽略任何基于宏的 \utf 支持而只是用Ascii。然后 \biber 会尝试将 \file{bib} 文件中的任意非Ascii数据转化为 Ascii。例如,它会将 \texttt{\d{S}} 转化为 |\d{S}|。关于为什么需要启用该选项的原因,请参考 \secref{bib:cav:enc:enc} 节。

\boolitem[true]{bibwarn}

%By default, \biblatex will report warnings issued by the backend concerning the data in the \file{bib} file as \latex warnings. Use this option to suppress such warnings.
默认情况下,\biblatex 会报告后端产生的关于 \file{bib} 文件数据的警告,就像 \LaTeX 警告一样。使用该选项会取消此警告。

\intitem[2]{mincrossrefs}

%Sets the minimum number of cross references to \prm{integer} when requesting a backend run.\footnote{If an entry which is cross-referenced by other entries in the \file{bib} file hits this threshold, it is included in the bibliography even if it has not been cited explicitly. This is a standard feature of the \bibtex format and not specific to \biblatex. See the description of the \bibfield{crossref} field in \secref{bib:fld:spc} for further information.} This option also affects the handling of the \bibfield{xref} field. See the field description in \secref{bib:fld:spc} as well as \secref{bib:cav:ref} for details.
当需要运行后端时,设置交叉引用的最小数目为 \prm{integer}
\footnote{如果一个条目被 \file{bib} 文件中的其它条目引用的数目达到这个阈值,它就会载入到参考文献中,即使没有显式地被引用。	这是 \bibtex  格式的标准特性,并不是 \biblatex 特有的。更多信息见 \secref{bib:fld:spc} 节中关于 \bibfield{crossref} 域的描述。}。该选项也影响 \bibfield{xref} 域的处理。详见 \secref{bib:fld:spc} 节以及 \secref{bib:cav:ref} 节对该域的描述。

\intitem[2]{minxrefs}

%As \opt{mincrossrefs} but for \bibfield{xref} fields.
类似于 \opt{mincrossrefs} 但针对于 \bibfield{xref} 域。

\end{optionlist}

\paragraph{特定样式选项}%\paragraph{Style-specific}
\label{use:opt:pre:bbx}

%The following options are provided by the standard styles (as opposed to the core package). Technically, they are preamble options like those in \secref{use:opt:pre:gen}.
下列选项由标准样式(而不是宏包内核)提供。技术上讲,它们和 \secref{use:opt:pre:gen} 中的选项一样也是导言区选项。

\begin{optionlist}

\boolitem[true]{isbn}

%This option controls whether the fields \bibfield{isbn}\slash \bibfield{issn}\slash \bibfield{isrn} are printed.
该选项控制是否打印 \bibfield{isbn}\slash \bibfield{issn}\slash \bibfield{isrn} 等域。

\boolitem[true]{url}

%This option controls whether the \bibfield{url} field and the access date is printed. The option only affects entry types whose \bibfield{url} information is optional. The \bibfield{url} field of \bibtype{online} entries is always printed.
该选项控制是否打印 \bibfield{url} 域和访问日期。该选项只影响 \bibfield{url} 信息是可选的那些条目类型。而 \bibtype{online} 条目的 \bibfield{url} 域总是打印出来的。

\boolitem[true]{doi}

%This option controls whether the field \bibfield{doi} is printed.
该选项控制是否打印 \bibfield{doi} 域。

\boolitem[true]{eprint}

%This option controls whether \bibfield{eprint} information is printed.
该选项控制是否打印 \bibfield{eprint} 信息。

\end{optionlist}

\paragraph{内部选项}%\paragraph{Internal}
\label{use:opt:pre:int}

%The default settings of the following preamble options are controlled by bibliography and citation styles. Apart from the \opt{pagetracker} and \opt{$<$name$>$inits} options, which you may want to adapt, there is normally no need to set them explicitly.
下列导言区选项的默认设置由著录和标注样式控制。除了 \opt{pagetracker} 和 \opt{\prm{name}inits} 这两个你可能想调整的选项外,一般没必要显式设置。

\begin{optionlist}

\optitem[false]{pagetracker}{\opt{true}, \opt{false}, \opt{page}, \opt{spread}}

%This option controls the page tracker which is required by the \cmd{ifsamepage} and \cmd{iffirstonpage} tests from \secref{aut:aux:tst}. The possible choices are:
该选项控制 \secref{aut:aux:tst} 节的  \cmd{ifsamepage} 和 \cmd{iffirstonpage} 测试所需的页码跟踪器。可用选择有:

\begin{valuelist}
\item[true] %Enable the tracker in automatic mode. This is like \opt{spread} if \latex is in twoside mode, and like \opt{page} otherwise.
在自动模式下激活。该选项在 \LaTeX 处于双面模式时类似于 \opt{spread},否则类似于 \opt{page}。
\item[false] %Disable the tracker.
不激活跟踪器。
\item[page] %Enable the tracker in page mode. In this mode, tracking works on a per"=page basis.
在页面模式下激活。此时跟踪器基于每一页。
\item[spread] %Enable the tracker in spread mode. In this mode, tracking works on a per"=spread (double page) basis.
在跨页模式下激活。此时跟踪器基于每一页面(双页)。
\end{valuelist}

%Note that this tracker is disabled in all floats, see \secref{aut:cav:flt}.
注意所有的浮动体都禁用该跟踪器,见 \secref{aut:cav:flt} 节。

\optitem[false]{citecounter}{\opt{true}, \opt{false}, \opt{context}}

%This option controls the citation counter which is required by \cnt{citecounter} from \secref{aut:aux:tst}. The possible choices are:
该选项控制 \secref{aut:aux:tst} 节的 \cnt{citecounter} 所需的引用计数器。可用的选择有:

\begin{valuelist}
\item[true] %Enable the citation counter in global mode.
在全局模式下启用引用计数器。
\item[false] %Disable the citation counter.
禁用引用计数器。
\item[context] %Enable the citation counter in context"=sensitive mode. In this mode, citations in footnotes and in the body text are counted independently.
在环境敏感模式下启用引用计数器。此时,脚注和正文中的引用将独立计数。
\end{valuelist}

\optitem[false]{citetracker}{\opt{true}, \opt{false}, \opt{context}, \opt{strict}, \opt{constrict}}

%This option controls the citation tracker which is required by the \cmd{ifciteseen} and \cmd{ifentryseen} tests from \secref{aut:aux:tst}. The possible choices are:
该选项控制 \secref{aut:aux:tst} 节的 \cmd{ifciteseen} 和 \cmd{ifentryseen} 测试所需的引用跟踪器。可用选择有:

\begin{valuelist}
\item[true] %Enable the tracker in global mode.
在全局模式下启用跟踪器。
\item[false] %Disable the tracker.
禁用跟踪器。
\item[context] %Enable the tracker in context"=sensitive mode. In this mode, citations in footnotes and in the body text are tracked independently.
在环境敏感模式下启用跟踪器。此时,脚注和正文中的引用将独立跟踪。
\item[strict] %Enable the tracker in strict mode. In this mode, an item is only considered by the tracker if it appeared in a stand-alone citation, \ie if a single entry key was passed to the citation command.
在严格模式下启用跟踪器。此时,跟踪器只考虑独立的引用,即,引用命令只传递单个的条目键。
\item[constrict] %This mode combines the features of \opt{context} and \opt{strict}.
该模式是 \opt{context} 和 \opt{strict} 的结合。
\end{valuelist}

%Note that this tracker is disabled in all floats, see \secref{aut:cav:flt}.
注意所有的浮动体都禁用该跟踪器,见 \secref{aut:cav:flt} 节。

\optitem[false]{ibidtracker}{\opt{true}, \opt{false}, \opt{context}, \opt{strict}, \opt{constrict}}

%This option controls the <ibidem> tracker which is required by the \cmd{ifciteibid} test from \secref{aut:aux:tst}. The possible choices are:

该选项控制 \secref{aut:aux:tst} 节的 \cmd{ifciteibid} 测试所需的“如前所述”(ibidem)跟踪器。可用的选择有:

\begin{valuelist}
\item[true] %Enable the tracker in global mode.
在全局模式下启用跟踪器。
\item[false] %Disable the tracker.
禁用跟踪器。
\item[context] %Enable the tracker in context"=sensitive mode. In this mode, citations in footnotes and in the body text are tracked separately.
在环境敏感模式下启用跟踪器。此时,脚注和正文中的引用将独立跟踪。
\item[strict] %Enable the tracker in strict mode. In this mode, potentially ambiguous references are suppressed. A reference is considered ambiguous if either the current citation (the one including the <ibidem>) or the previous citation (the one the <ibidem> refers to) consists of a list of references.\footnote{For example, suppose the initial citation is «Jones, \emph{Title}; Williams, \emph{Title}» and the following one «ibidem». From a technical point of view, it is fairly clear that the <ibidem> refers to <Williams> because this is the last reference processed by the previous citation command. To a human reader, however, this may not be obvious because the <ibidem> may also refer to both titles. The strict mode avoids such ambiguous references.}
在严格模式下启用跟踪器。此时将不考虑那些模糊不清的参考文献。如果当前引用(包含“ibidem”)或者之前引用(“ibidem”的指向)包含多个文献时,就认为是模糊不清的。
\footnote{例如,假设一开始的引用是“Jones, \emph{Title}; Williams, \emph{Title}”,而接下来的是“ibidem”。
	从技术角度来看,“ibidem”指向“Williams”是相当清晰的,因为这是之前引用命令处理的最后文献。然而对于用户而言,“ibidem”也可以同时指向这两个标题,因此含义不清。	严格模式就避免这种含义不清的文献。}
\item[constrict] %This mode combines the features of \opt{context} and \opt{strict}. It also keeps track of footnote numbers and detects potentially ambiguous references in footnotes in a stricter way than the \opt{strict} option. In addition to the conditions imposed by the \opt{strict} option, a reference in a footnote will only be considered as unambiguous if the current citation and the previous citation are given in the same footnote or in immediately consecutive footnotes.
该模式是 \opt{context} 和 \opt{strict} 的结合。它也保持对脚注数量的跟踪,不过检测脚注中含义不清的文献时比 \opt{strict} 更加严格。除了 \opt{strict} 选项的条件外,只有当前引用和之前引用都在同一个或者连续的脚注中时,脚注中的文献才认为是含义清晰的。
\end{valuelist}

%Note that this tracker is disabled in all floats, see \secref{aut:cav:flt}.
注意所有的浮动体都禁用该跟踪器,见 \secref{aut:cav:flt} 节。

\optitem[false]{opcittracker}{\opt{true}, \opt{false}, \opt{context}, \opt{strict}, \opt{constrict}}

%This option controls the <opcit> tracker which is required by the \cmd{ifopcit} test from \secref{aut:aux:tst}. This feature is similar to the <ibidem> tracker, except that it tracks citations on a per-author/editor basis, \ie \cmd{ifopcit} will yield \texttt{true} if the cited item is the same as the last one by this author\slash editor. The possible choices are:
该选项控制 \secref{aut:aux:tst} 节的 \cmd{ifopcit} 测试所需的“opcit”跟踪器。该特性类似于“ibidem”跟踪器,不同之处在于它跟踪的是基于某一作者或编辑的引用,即,如果引用项目与之前项目的作者或编者相同,那么 \cmd{ifopcit} 的结果为 \texttt{true}。可用选择有:

\begin{valuelist}
\item[true] %Enable the tracker in global mode.
在全局模式下启用该跟踪器。
\item[false] %Disable the tracker.
禁用该跟踪器。
\item[context] %Enable the tracker in context"=sensitive mode. In this mode, citations in footnotes and in the body text are tracked separately.
在环境敏感模式下启用该跟踪器。此时,脚注和正文中的引用会独立跟踪。
\item[strict] %Enable the tracker in strict mode. In this mode, potentially ambiguous references are suppressed. See \kvopt{ibidtracker}{strict} for details.
在严格模式下启用该跟踪器。此时将不跟踪那些含义不清的引用。详见 \kvopt{ibidtracker}{strict} 选项。
\item[constrict] %This mode combines the features of \opt{context} and \opt{strict}. See the explanation of \kvopt{ibidtracker}{constrict} for details.
该模式是 \opt{context} 和 \opt{strict} 的结合。详见  \kvopt{ibidtracker}{constrict} 选项的解释。
\end{valuelist}

%Note that this tracker is disabled in all floats, see \secref{aut:cav:flt}.
注意所有的浮动体都禁用该跟踪器,见 \secref{aut:cav:flt} 节。

\optitem[false]{loccittracker}{\opt{true}, \opt{false}, \opt{context}, \opt{strict}, \opt{constrict}}

%This option controls the <loccit> tracker which is required by the \cmd{ifloccit} test from \secref{aut:aux:tst}. This feature is similar to the <opcit> tracker except that it also checks whether the \prm{postnote} arguments match, \ie \cmd{ifloccit} will yield \texttt{true} if the citation refers to the same page cited before. The possible choices are:
该选项控制 \secref{aut:aux:tst} 节的 \cmd{ifloccit} 测试所需的“loccit”跟踪器。该特性类似于“opcit”跟踪器,不同之处在于它也会检查 \prm{postnote} 选项是否匹配,即,如果引用项目与之前引用指向的页数相同,那么 \cmd{ifloccit} 的结果为 \texttt{true}。可用选择有:

\begin{valuelist}
\item[true] %Enable the tracker in global mode.
在全局模式下启用该跟踪器。
\item[false] %Disable the tracker.
禁用该跟踪器。
\item[context] %Enable the tracker in context"=sensitive mode. In this mode, citations in footnotes and in the body text are tracked separately.
在环境敏感模式下启用该跟踪器。此时,脚注和正文中的引用会独立跟踪。
\item[strict] %Enable the tracker in strict mode. In this mode, potentially ambiguous references are suppressed. See \kvopt{ibidtracker}{strict} for details. In addition to that, this mode also checks if the \prm{postnote} argument is numerical (based on \cmd{ifnumerals} from \secref{aut:aux:tst}).
在严格模式下启用该跟踪器。此时将不跟踪那些含义不清的引用。详见 \kvopt{ibidtracker}{strict} 选项。此外,该模式也会检查 \prm{postnote} 选项是否是数值型的(基于 \secref{aut:aux:tst} 节的 \cmd{ifnumerals} 命令)。
\item[constrict] %This mode combines the features of \opt{context} and \opt{strict}. See the explanation of \kvopt{ibidtracker}{constrict} for details. In addition to that, this mode also checks if the \prm{postnote} argument is numerical (based on \cmd{ifnumerals} from \secref{aut:aux:tst}).
该模式是 \opt{context} 和 \opt{strict} 的结合。详见  \kvopt{ibidtracker}{constrict} 选项的解释。此外,该模式也会检查 \prm{postnote} 选项是否是数值型的(基于 \secref{aut:aux:tst} 节的 \cmd{ifnumerals} 命令)。
\end{valuelist}

%Note that this tracker is disabled in all floats, see \secref{aut:cav:flt}.
注意所有的浮动体都禁用该跟踪器,见 \secref{aut:cav:flt} 节。

\optitem[false]{idemtracker}{\opt{true}, \opt{false}, \opt{context}, \opt{strict}, \opt{constrict}}

%This option controls the <idem> tracker which is required by the \cmd{ifciteidem} test from \secref{aut:aux:tst}. The possible choices are:
该选项控制 \secref{aut:aux:tst} 节的 \cmd{ifciteidem} 测试所需的“idem”跟踪器。
可用选择有:

\begin{valuelist}
\item[true] %Enable the tracker in global mode.
在全局模式下启用该跟踪器。
\item[false] %Disable the tracker.
禁用该跟踪器。
\item[context] %Enable the tracker in context"=sensitive mode. In this mode, citations in footnotes and in the body text are tracked separately.
在环境敏感模式下启用该跟踪器。此时,脚注和正文中的引用会独立跟踪。
\item[strict] %This is an alias for \texttt{true}, provided only for consistency with the other trackers. Since <idem> replacements do not get ambiguous in the same way as <ibidem> or <op.~cit.>, the \texttt{strict} tracking mode does not apply to them.
该选项是 \texttt{true} 的别名。提供该选项只是为了和其它跟踪器保持一致。由于“idem”不会像“ibidem”或“op.~cit.”那样引起歧义,因此不必使用 \texttt{strict} 跟踪模式。
\item[constrict] %This mode is similar to \opt{context} with one additional condition: a reference in a footnote will only be considered as unambiguous if the current citation and the previous citation are given in the same footnote or in immediately consecutive footnotes.
该模式类似于 \opt{context},只有一个额外条件:对于脚注中的引用,只有当前引用和之前引用位于同一个或相连的脚注中才会认为是含义明确的。
\end{valuelist}

%Note that this tracker is disabled in all floats, see \secref{aut:cav:flt}.
注意所有的浮动体都禁用该跟踪器,见 \secref{aut:cav:flt} 节。

\boolitem[true]{parentracker}

%This option controls the parenthesis tracker which keeps track of nested parentheses and brackets. This information is used by \cmd{parentext} and \cmd{brackettext} from \secref{use:cit:txt}, \cmd{mkbibparens} and \cmd{mkbibbrackets} from \secref{aut:fmt:ich} and \cmd{bibopenparen}, \cmd{bibcloseparen}, \cmd{bibopenbracket}, \cmd{bibclosebracket} (also \secref{aut:fmt:ich}).

该选项控制括号跟踪器,
用于对嵌套的圆括号和方括号的跟踪。
所得信息用于 \secref{use:cit:txt} 节的  \cmd{parentext} 和 \cmd{brackettext} 命令、
\secref{aut:fmt:ich} 节的 \cmd{mkbibparens} 和 \cmd{mkbibbrackets} 命令,
以及同样来自于 \secref{aut:fmt:ich} 节的 \cmd{bibopenparen}, \cmd{bibcloseparen}, \cmd{bibopenbracket}, \cmd{bibclosebracket} 等命令。

\intitem[3]{maxparens}

%The maximum permitted nesting level of parentheses and brackets. If parentheses and brackets are nested deeper than this value, \biblatex will issue errors.

圆括号和方括号嵌套的最大层级。
如果嵌套深度大于该值则会报错。

\boolitem[false]{\prm{namepart}inits}

%When enabled, all $<$namepart$>$ name parts will be rendered as initials. The option will affect the \cmd{if$<$namepart$>$inits} test from \secref{aut:aux:tst}. The valid name parts are defined in the data model by the \cmd{DeclareDatamodelConstant} command (\secref{aut:bbx:drv}).

启用该选项时所有的 \prm{namepart} 姓名部分都会用首字母表示。
该选项会影响 \secref{aut:aux:tst} 节的 \cmd{if\prm{namepart}inits} 测试。
在数据模型中有效的姓名部分由 \cmd{DeclareDatamodelConstant} 命令定义,见 \secref{aut:bbx:drv} 节。

\boolitem[false]{terseinits}

%This option controls the format of all initials generated by \biblatex. If enabled, initials are rendered using a terse format without dots and spaces. For example, the initials of Donald Ervin Knuth would be rendered as <D.~E.> by default, and as <DE> if this option is enabled. The option will affect the \cmd{ifterseinits} test from \secref{aut:aux:tst}. The option works by redefining some macros which control the format of initials. See \secref{use:cav:nam} for details.

该选项控制 \biblatex 生成的首字母格式。
当启用时,首字母缩写将采用没有点号和空格的紧凑形式。
例如 Donald Ervin Knuth 的首字母缩写默认是“D.~E.”,而在此选项启用时会是“DE”。
该选项会影响 \secref{aut:aux:tst} 中的 \cmd{ifterseinits} 测试。
该选项会重新定义一些控制首字母格式的宏。详见 \secref{use:cav:nam} 节。

\boolitem[false]{labelalpha}

%Whether or not to provide the special fields \bibfield{labelalpha} and \bibfield{extraalpha}, see \secref{aut:bbx:fld} for details.
%This option is also settable on a per-type basis. See also \opt{maxalphanames} and \opt{minalphanames}. Table \ref{use:opt:tab1} summarises the various \opt{extra*} disambiguation counters and what they track.

是否提供特殊的域 \bibfield{labelalpha} 和 \bibfield{extraalpha},详见 \secref{aut:bbx:fld} 节。
该选项可以基于每一条目设置。
另见 \opt{maxalphanames} 和 \opt{minalphanames} 选项。
表 \ref{use:opt:tab1} 总结了各种 \opt{extra*} 歧义消除的计数器以及它们所跟踪的选项。

\intitem[3]{maxalphanames}

%Similar to the \opt{maxnames} option but customizes the format of the \bibfield{labelalpha} field.

类似于 \opt{maxnames} 但用于定制 \bibfield{labelalpha} 域的格式。

\intitem[1]{minalphanames}

%Similar to the \opt{minnames} option but customizes the format of the \bibfield{labelalpha} field.
类似于 \opt{minnames} 但用于定制 \bibfield{labelalpha} 域的格式。

\boolitem[false]{labelnumber}

%Whether or not to provide the special field \bibfield{labelnumber}, see \secref{aut:bbx:fld} for details.
%This option is also settable on a per-type basis.

是否提供特殊域 \bibfield{labelnumber},详见  \secref{aut:bbx:fld} 节。
该选项可基于每一类型而设置。

\boolitem[false]{labeltitle}

%Whether or not to provide the special field \bibfield{extratitle}, see \secref{aut:bbx:fld} for details. Note that the special field \bibfield{labeltitle} is always provided and this option controls rather whether \bibfield{labeltitle} is used to generate \bibfield{extratitle} information. This option is also settable on a per-type basis. Table \ref{use:opt:tab1} summarises the various \opt{extra*} disambiguation counters and what they track.

是否提供特殊域\bibfield{extratitle},详见 \secref{aut:bbx:fld} 节。
请注意,特殊域 \bibfield{labeltitle} 总是提供的,
而该选项控制是否利用 \bibfield{labeltitle} 生成  \bibfield{extratitle} 信息。
该选项也可基于每一类型而设置。
表 \ref{use:opt:tab1} 总结了各种 \opt{extra*} 消除歧义的计数器以及所跟踪的选项。

\boolitem[false]{labeltitleyear}

%Whether or not to provide the special field \bibfield{extratitleyear}, see \secref{aut:bbx:fld} for details. Note that the special field \bibfield{labeltitle} is always provided and this option controls rather whether \bibfield{labeltitle} is used to generate \bibfield{extratitleyear} information. This option is also settable on a per-type basis. Table \ref{use:opt:tab1} summarises the various \opt{extra*} disambiguation counters and what they track.

是否提供特殊域  \bibfield{extratitleyear},详见 \secref{aut:bbx:fld} 节。
请注意,特殊域 \bibfield{labeltitle} 总是提供的,
而该选项控制是否利用 \bibfield{labeltitle} 生成  \bibfield{extratitleyear} 信息。
该选项也可基于每一类型而设置。
表 \ref{use:opt:tab1} 总结了各种 \opt{extra*} 消除歧义的计数器以及所跟踪的选项。

\boolitem[false]{labeldateparts}

%Whether or not to provide the special fields \bibfield{labelyear}, \bibfield{labelmonth}, \bibfield{labelday}, \bibfield{labelendyear}, \bibfield{labelendmonth}, \bibfield{labelendday}, \bibfield{labelhour}, \bibfield{labelendhour}, \bibfield{labelminute}, \bibfield{labelendminute}, \bibfield{labelsecond}, \bibfield{labelendsecond}, \bibfield{labelseason}, \bibfield{labelendseason}, \bibfield{labeltimezone}, \bibfield{labelendtimeone} and \bibfield{extrayear}, see \secref{aut:bbx:fld} for details.
%This option is also settable on a per-type basis. Table \ref{use:opt:tab1} summarises the various \opt{extra*} disambiguation counters and what they track.

是否提供特殊域 \bibfield{labelyear}, \bibfield{labelmonth}, \bibfield{labelday}, \bibfield{labelendyear}, \bibfield{labelendmonth}, \bibfield{labelendday}, \bibfield{labelhour}, \bibfield{labelendhour}, \bibfield{labelminute}, \bibfield{labelendminute}, \bibfield{labelsecond}, \bibfield{labelendsecond}, \bibfield{labelseason}, \bibfield{labelendseason}, \bibfield{labeltimezone}, \bibfield{labelendtimeone} 以及 \bibfield{extrayear},详见 \secref{aut:bbx:fld} 节。
该选项也可基于每一类型而设置。
表 \ref{use:opt:tab1} 总结了各种 \opt{extra*} 消除歧义的计数器以及所跟踪的选项。


\begin{table}
	\footnotesize
	\ttfamily
	\tablesetup
	\begin{tabularx}{\textwidth}{XXX}
		\toprule
		\multicolumn{1}{@{}H}{选项} &
		\multicolumn{1}{@{}H}{测试} &
		\multicolumn{1}{@{}H}{跟踪的域} \\
		\cmidrule(r){1-1}\cmidrule(r){2-2}\cmidrule(r){3-3}
		singletitle & \cmd{ifsingletitle} & labelname\\
		uniquetitle & \cmd{ifuniquetitle} & labeltitle\\
		uniquebaretitle & \cmd{ifuniquebaretitle} & labeltitle (当 labelname 为空时) \\
		uniquework  & \cmd{ifuniquework}  & labelname+labeltitle\\
		\bottomrule
	\end{tabularx}
	\caption{惟一性选项}%Work Uniqueness options
	\label{use:opt:wu}
\end{table}

\boolitem[false]{singletitle}

%Whether or not to provide the data required by the \cmd{ifsingletitle} test, see \secref{aut:aux:tst} for details. See \tabref{use:opt:wu} for details on what determines the data for this test.
%This option is also settable on a per-type basis.

是否提供 \cmd{ifsingletitle} 测试所需的数据,详见 \secref{aut:aux:tst} 节。
关于该测试中数据的控制因素详见 \tabref{use:opt:wu}。
该选项也可基于每一类型而设置。

\boolitem[false]{uniquetitle}

%Whether or not to provide the data required by the \cmd{ifuniquetitle} test, see \secref{aut:aux:tst} for details. See \tabref{use:opt:wu} for details on what determines the data for this test.
%This option is also settable on a per-type basis.

是否提供 \cmd{ifuniquetitle} 测试所需的数据,详见 \secref{aut:aux:tst} 节。
关于该测试中数据的控制因素详见 \tabref{use:opt:wu}。
该选项也可基于每一类型而设置。

\boolitem[false]{uniquebaretitle}

%Whether or not to provide the data required by the \cmd{ifuniquebaretitle} test, see \secref{aut:aux:tst} for details. See \tabref{use:opt:wu} for details on what determines the data for this test.
%This option is also settable on a per-type basis.

是否提供 \cmd{ifuniquebaretitle} 测试所需的数据,详见 \secref{aut:aux:tst} 节。
关于该测试中数据的控制因素详见 \tabref{use:opt:wu}。
该选项也可基于每一类型而设置。

\boolitem[false]{uniquework}

%Whether or not to provide the data required by the \cmd{ifuniquework} test, see \secref{aut:aux:tst} for details. See \tabref{use:opt:wu} for details on what determines the data for this test.
%This option is also settable on a per-type basis.

是否提供 \cmd{ifuniquework} 测试所需的数据,详见 \secref{aut:aux:tst} 节。
关于该测试中数据的控制因素详见 \tabref{use:opt:wu}。
该选项也可基于每一类型而设置。

\boolitem[false]{uniqueprimaryauthor}

%Whether or not to provide the data required by the \cmd{ifuniqueprimaryauthor} test, see \secref{aut:aux:tst} for details.

是否提供 \cmd{ifuniqueprimaryauthor} 测试所需的数据,详见 \secref{aut:aux:tst} 节。

\optitem[false]{uniquename}{\opt{true}, \opt{false}, \opt{init}, \opt{full}, \opt{allinit}, \opt{allfull},
\opt{mininit}, \opt{minfull}}

%Whether or not to update the \cnt{uniquename} counter, see \secref{aut:aux:tst} for details. This feature will disambiguate individual names in the \bibfield{labelname} list. This option is also settable on a per-type basis. The possible choices are:

是否更新 \cnt{uniquename} 计数器,详见 \secref{aut:aux:tst} 节。
该特性会消除 \bibfield{labelname} 列表中各个姓名的歧义。
该选项也可基于每一类型而设置。可用的选择有:

\begin{valuelist}
\item[true] %An alias for \opt{full}.
\opt{full} 的别称。
\item[false] %Disable this feature.
禁用该特性。
\item[init] %Disambiguate using initials only.
只使用首字母消除歧义。
\item[full] %Disambiguate using initials or full names, as required.
根据要求使用首字母或全名消除歧义。
\item[allinit] %Similar to \opt{init} but disambiguates all names in the \bibfield{labelname} list, beyond \opt{maxnames}\slash \opt{minnames}\slash \opt{uniquelist}.
类似于 \opt{init},但是会对 \bibfield{labelname} 列表中所有姓名消除歧义,
即便超出了 \opt{maxnames}\slash \opt{minnames}\slash \opt{uniquelist} 选项。
\item[allfull] %Similar to \opt{full} but disambiguates all names in the \bibfield{labelname} list, beyond \opt{maxnames}\slash \opt{minnames}\slash \opt{uniquelist}.
类似于 \opt{full},但是会对 \bibfield{labelname} 列表中所有姓名消除歧义,
即便超出了 \opt{maxnames}\slash \opt{minnames}\slash \opt{uniquelist} 选项。
\item[mininit] %A variant of \texttt{init} which only disambiguates names in lists with identical last names.
\texttt{init} 的变种,只对列表中有同一姓(last name)的姓名消除歧义。
\item[minfull] %A variant of \texttt{full} which only disambiguates names in lists with identical last names.
\texttt{full} 的变种,只对列表中有同一姓(last name)的姓名消除歧义。
\end{valuelist}
%
%Note that the \opt{uniquename} option will also affect \opt{uniquelist}, the \cmd{ifsingletitle} test, and the \bibfield{extrayear} field. See \secref{aut:cav:amb} for further details and practical examples.
请注意,\opt{uniquename} 选项也会影响 \opt{uniquelist} 选项、
\cmd{ifsingletitle} 测试,以及 \bibfield{extrayear} 域。
更多细节和实例见 \secref{aut:cav:amb} 节。

\optitem[false]{uniquelist}{\opt{true}, \opt{false}, \opt{minyear}}

%Whether or not to update the \cnt{uniquelist} counter, see \secref{aut:aux:tst} for details. This feature will disambiguate the \bibfield{labelname} list if it has become ambiguous after \opt{maxnames}\slash \opt{minnames} truncation. Essentially, it overrides \opt{maxnames}\slash \opt{minnames} on a per-field basis. This option is also settable on a per-type basis. The possible choices are:

是否更新 \cnt{uniquelist} 计数器,详见 \secref{aut:aux:tst} 节。
如果 \bibfield{labelname} 列表在 \opt{maxnames}\slash \opt{minnames} 截断后含义不清,
那么该特性会消除 \bibfield{labelname} 列表中的歧义。
本质上,该选项会覆盖基于每一域的 \opt{maxnames}\slash \opt{minnames} 设置。
该选项也可基于每一类型而设置。可用的选择有:

\begin{valuelist}
\item[true] %Disambiguate the \bibfield{labelname} list.
消除 \bibfield{labelname} 列表的歧义。
\item[false] %Disable this feature.
禁用该特性。
\item[minyear] %Disambiguate the \bibfield{labelname} list only if the truncated list is identical to another one with the same \bibfield{labelyear}. This mode of operation is useful for author-year styles and requires \kvopt{labeldateparts}{true}.
只有当被截断的列表与带有相同 \bibfield{labelyear} 的另一列表相同时才会消除 \bibfield{labelname} 列表的歧义。
该操作模式适用于作者---年份样式中,如果要求 \kvopt{labeldateparts}{true} 的场合。
\end{valuelist}
%
%Note that the \opt{uniquelist} option will also affect the \cmd{ifsingletitle} test and the \bibfield{extrayear} field. See \secref{aut:cav:amb} for further details and practical examples.
请注意,\opt{uniquelist} 选项也会影响 \cmd{ifsingletitle} 测试和 \bibfield{extrayear} 域。
更多细节和实例见 \secref{aut:cav:amb} 节。

\end{optionlist}

\begin{table}
\footnotesize
\ttfamily
\tablesetup
\begin{tabularx}{\textwidth}{XXXX}
\toprule
\multicolumn{1}{@{}H}{选项} &
\multicolumn{1}{@{}H}{激活域} & %Enabled field(s)
\multicolumn{1}{@{}H}{激活计数器} & %Enabled counter
\multicolumn{1}{@{}H}{计数器跟踪} \\ %Counter tracks
\cmidrule(r){1-1}\cmidrule(r){2-2}\cmidrule(r){3-3}\cmidrule{4-4}
labelalpha     & labelalpha       & extraalpha     &  label\\
labeldateparts & labelyear        & extradate      &  labelname+\\
		& labelmonth       &                &  labelyear\\
		& labelday         &                &  \\
		& labelendyear     &                &  \\
		& labelendmonth    &                &  \\
		& labelendday      &                &  \\
		& labelhour        &                &  \\
		& labelminute      &                &  \\
		& labelsecond      &                &  \\
		& labelendhour     &                &  \\
		& labelendminute   &                &  \\
		& labelendsecond   &                &  \\
		& labelseason      &                &  \\
		& labelendseason   &                &  \\
		& labeltimezone    &                &  \\
		& labelendtimezone &                &  \\
labeltitle     & \rmfamily{---}   & extratitle     &  labelname+labeltitle\\
labeltitleyear & \rmfamily{---}   & extratitleyear &  labeltitle+labelyear\\
\bottomrule
\end{tabularx}
\caption{歧义消除计数器}%Disambiguation counters
\label{use:opt:tab1}
\end{table}

\subsubsection{条目选项}%\subsubsection{Entry Options}
\label{use:opt:bib}

%Entry options are package options which determine how bibliography data entries are handled. They may be set at various scopes defined below.

条目选项是控制参考文献数据条目处理的包选项。
可以在以下不同尺度上设置。

\paragraph{导言区/类型/条目选项}%\paragraph{Preamble/Type/Entry Options}
\label{use:opt:bib:hyb}

%The following options are settable on a per"=type basis or on a per"=entry in the \bibfield{options} field. In addition to that, they may also be used in the optional argument to \cmd{usepackage} as well as in the configuration file and the document preamble. This is useful if you want to change the default behaviour globally.

下列选项可以基于类型或条目在 \bibfield{options} 域中设置。
此外还可以在 \cmd{usepackage} 的可选项以及配置文件和导言区中使用。
这可用于全局改变默认样式。

\begin{optionlist}

\boolitem[true]{useauthor}

%Whether the \bibfield{author} is used in labels and considered during sorting. This may be useful if an entry includes an \bibfield{author} field but is usually not cited by author for some reason. Setting \kvopt{useauthor}{false} does not mean that the \bibfield{author} is ignored completely. It means that the \bibfield{author} is not used in labels and ignored during sorting. The entry will then be alphabetized by \bibfield{editor} or \bibfield{title}. With the standard styles, the \bibfield{author} is printed after the title in this case. See also \secref{use:srt}.
%This option is also settable on a per-type and per-entry basis.

是否在标签和排序中使用 \bibfield{author} 域。
如果一个条目包含 \bibfield{author} 域但出于某种原因通常不以作者引用,该选项是很有用的。
设置 \kvopt{useauthor}{false} 并不意味着 \bibfield{author} 被完全忽略了,而只是在标签和排序中不使用。
该条目将按照 \bibfield{editor} 或 \bibfield{title} 域的字母顺序排列。
在标准样式中,此时 \bibfield{author} 会在标题后打印。参考 \secref{use:srt} 节。
该选项可以基于每一类型设置。

\boolitem[true]{useeditor}

%Whether the \bibfield{editor} replaces a missing \bibfield{author} in labels and during sorting. This may be useful if an entry includes an \bibfield{editor} field but is usually not cited by editor. Setting \kvopt{useeditor}{false} does not mean that the \bibfield{editor} is ignored completely. It means that the \bibfield{editor} does not replace a missing \bibfield{author} in labels and during sorting. The entry will then be alphabetized by \bibfield{title}. With the standard styles, the \bibfield{editor} is printed after the title in this case. See also \secref{use:srt}.
%This option is also settable on a per-type and per-entry basis.

在标签和排序中是否用 \bibfield{editor} 域来代替缺失的 \bibfield{author} 域。
如果一个条目包含 \bibfield{editor} 域但是通常不以作者引用,那么该选项是很有用的。
设置 \kvopt{useeditor}{false} 并不意味着 \bibfield{editor} 被完全忽略了,而只是在标签和排序中不用 \bibfield{editor} 来代替缺失的 \bibfield{author}。
该条目会以 \bibfield{title} 的字母顺序排列。
在标准样式中,此时 \bibfield{editor} 会在标题之后打印。参考 \secref{use:srt} 节。
该选项可以基于每一类型或条目而设置。

\boolitem[false]{usetranslator}

%Whether the \bibfield{translator} replaces a missing \bibfield{author}\slash \bibfield{editor} in labels and during sorting. Setting \kvopt{usetranslator}{true} does not mean that the \bibfield{translator} overrides the \bibfield{author}\slash \bibfield{editor}. It means that the \bibfield{translator} is considered as a fallback if the \bibfield{author}\slash \bibfield{editor} is missing or if \opt{useauthor} and \opt{useeditor} are set to \texttt{false}. In other words, in order to cite a book by translator rather than by author, you need to set the following options:
%This option is also settable on a per-type and per-entry basis.

在标签和排序中是否用 \bibfield{translator} 代替缺失的 \bibfield{author}\slash \bibfield{editor}。
设置 \kvopt{usetranslator}{true} 并不意味着 \bibfield{translator} 会覆盖 \bibfield{author}\slash \bibfield{editor},而只是当 \bibfield{author}\slash \bibfield{editor} 缺失或者  \opt{useauthor} 和 \opt{useeditor} 选项设置为 \texttt{false} 时作为后备。
也就是说,如果要按译者而不是作者引用一本书,你需要设置如下选项:

\begin{lstlisting}[style=bibtex]{}
@Book{...,
  options    = {useauthor=false,usetranslator=true},
  author     = {...},
  translator = {...},
  ...
\end{lstlisting}
%
%With the standard styles, the \bibfield{translator} is printed after the title by default. See also \secref{use:srt}.
该选项也可以基于每一类型或条目而设置。
在标准样式中,\bibfield{translator} 默认在标题之后打印,参考 \secref{use:srt} 节。

\boolitem[true]{use\prm{name}}

%As per \opt{useauthor}, \opt{useeditor} and \opt{usetranslator}, all name lists defined in the data model have an option controlling their behaviour in sorting and labelling automatically defined. Global, per-type and per-entry options called <use$<$name$>$>are automatically created.

按照 \opt{useauthor}、\opt{useeditor} 和 \opt{usetranslator},
数据模型中定义的所有的姓名列表都有一个选项用于控制自动定义的排序和标签行为。
此时会自动创建全局、基于类型和基于条目选项而调用的 \opt{use\prm{name}}。

\boolitem[false]{useprefix}

%Whether the default date model name part <prefix> (von, van, of, da, de, della, etc.) is considered when:

是否在以下几种情况下考虑默认数据模型中的姓名前缀部分(von、van、of,、da、de、della 等):

\begin{itemize}
\item %Printing the family name in citations
在引用中打印姓
\item %Sorting
排序
\item %Generation of certain types of labels
生成标签的某些类型
\item %Generating name uniqueness information
生成姓名惟一性信息。
\item %Formatting aspects of the bibliography
参考文献的格式方面
\end{itemize}
%
%For example, if this option is enabled, \biblatex precedes the family name with the prefix---Ludwig van Beethoven would be cited as «van Beethoven» and alphabetized as «Van Beethoven, Ludwig». If this option is disabled (the default), he is cited as «Beethoven» and alphabetized as «Beethoven, Ludwig van» instead.
%This option is also settable on a per-type scope. With \biblatexml datasources and the \bibtex extended name format supported by \biber, this is also settable on per-namelist and per-name scopes.
如果激活该选项,\biblatex 总是在姓氏前面加上该前缀。
例如 Ludwig van Beethoven 将引作“van Beethoven”并按照“Van Beethoven, Ludwig”排序,
而如果未激活该选项(默认情况),将引作“Beethoven”并按照“Beethoven, Ludwig van”排序。
该选项也可基于每一类型设置。
当使用 \biblatexml 数据源以及 \biber 支持的 \BibTeX 扩展名格式时,
还可以基于每一姓名列表或姓名而设置。

\optitem{indexing}{\opt{true}, \opt{false}, \opt{cite}, \opt{bib}}

%The \opt{indexing} option is also settable per-type or per-entry basis. See \secref{use:opt:pre:gen} for details.
\opt{indexing} 选项也可以基于每一类型或条目设置。详见 \secref{use:opt:pre:gen} 节。

\end{optionlist}

\paragraph{类型/条目选项}%\paragraph{Type/Entry Options}
\label{use:opt:bib:ded}

%The following options are settable on a per"=type basis or on a per"=entry in the \bibfield{options} field. They are not available globally.
下列选项只能基于类型或条目在 \bibfield{options} 域中设置,而不能全局设置。

\begin{optionlist}

\boolitem[false]{skipbib}

%If this option is enabled, the entry is excluded from the bibliography but it may still be cited.
%This option is also settable on a per-type basis.

如果激活该选项,该条目在参考文献中将被排除在外但仍然可以引用。
该选项也可以基于每一类型设置。

\boolitem[false]{skipbiblist}

%If this option is enabled, the entry is excluded from and bibliography lists. It is still included in the bibliography and it may also be cited by shorthand etc.
%This option is also settable on a per-type basis.

如果激活该选项,该选项将在文献列表中被排除在外,
但仍然包含在参考文献中并可以被shorthand引用。
该选项也可以基于每一类型设置。

\boolitem[false]{skiplab}

%If this option is enabled, \biblatex will not assign any labels to the entry. It is not required for normal operation. Use it with care. If enabled, \biblatex can not guarantee unique citations for the respective entry and citations styles which require labels may fail to create valid citations for the entry.
%This option is also settable on a per-type basis.

如果激活该选项,\biblatex 不会给该条目分配标签。
正常操作不需要该选项。要小心使用。
当激活时,\biblatex 不能保证相应条目有唯一的引用,
而且那些需要标签的引用样式可能不能为该条目创建有效的引用。
该选项也可以基于每一类型设置。

\boolitem[false]{dataonly}

%Setting this option is equivalent to \kvopt{uniquename}{false}, \kvopt{uniquelist}{false, }\opt{skipbib}, \opt{skipbiblist}, and \opt{skiplab}. It is not required for normal operation. Use it with care.
%This option is also settable on a per-type basis.

设置该选项等价于设置 \kvopt{uniquename}{false}、\kvopt{uniquelist}{false}、\opt{skipbib}、\opt{skipbiblist} 和 \opt{skiplab}。
正常操作不需要该选项。要小心使用。
该选项也可以基于每一类型设置。

\end{optionlist}

\paragraph{条目选项}%\paragraph{Entry Only Options}
\label{use:opt:bib:entry}

%The following options are settable only on a per"=entry in the \bibfield{options} field. They are not available globally or per"=type.
下列选项只能基于条目在 \bibfield{options} 域中而不能全局或基于类型设置。

\begin{optionlist}

\valitem{labelnamefield}{fieldname}

%Specifies the field to consider first when looking for a \bibfield{labelname} candidate. It is essentially prepended to the search list created by \cmd{DeclareLabelname} for just this entry.
指定搜索 \bibfield{labelname} 时首先考虑的域。
本质上,只在该条目中该域会放到 \cmd{DeclareLabelname} 创建的搜索列表之前。

\valitem{labeltitlefield}{fieldname}

%Specifies the field to consider first when looking for a \bibfield{labeltitle} candidate. It is essentially prepended to the search list created by \cmd{DeclareLabeltitle} for just this entry.
指定搜索 \bibfield{labeltitle} 时首先考虑的域。
本质上,只在该条目中该域会放到 \cmd{DeclareLabeltitle} 创建的搜索列表之前。

\end{optionlist}

\subsubsection{遗留选项}

%The following legacy option may be used globally in the optional argument to \cmd{documentclass} or locally in the optional argument to \cmd{usepackage}:
下面的遗留选项可以全局地在 \cmd{documentclass} 的可选项中使用,
也可以局部地在 \cmd{usepackage} 的可选项中使用:

\begin{optionlist}
	
\legitem{openbib}\DeprecatedMark  %This option is provided for backwards compatibility with the standard LaTeX document classes. \opt{openbib} is similar to \kvopt{block}{par}.
该选项用于向后兼容标准 \LaTeX 文档类。
\opt{openbib} 类似于 \kvopt{block}{par}。

\end{optionlist}

\subsection{全局定制}%\subsection{Global Customization}
\label{use:cfg}

%Apart from writing new citation and bibliography styles, there are numerous ways to customize the styles which ship with this package. Customization will usually take place in the preamble, but there is also a configuration file for permanent adaptions. The configuration file may also be used to initialize the package options to a value different from the package default.

除了编写新的引用和文献样式,本宏包中还有很多定制样式的方法。
定制通常在导言区中进行,但也可以在配置文件中进行以便长期使用。
配置文件也可以用于将宏包选项从默认值初始化为不同的值。

\subsubsection{配置文件}%\subsubsection{Configuration File}
\label{use:cfg:cfg}

%If available, this package will load the configuration file \path{biblatex.cfg}. This file is read at the end of the package, immediately after the citation and bibliography styles have been loaded.

当可用时,本宏包会导入配置文件 \path{biblatex.cfg}。
该文件会在宏包的末尾,紧跟在引用和文献样式导入之后立即被读取。

\subsubsection{设置宏包选项}%\subsubsection{Setting Package Options}
\label{use:cfg:opt}

%The load-time package options in \secref{use:opt:ldt} must be given in the optional argument to \cmd{usepackage}. The package options in \secref{use:opt:pre} may also be given in the preamble. The options are executed with the following command:

\secref{use:opt:ldt} 节中的实时载入宏包选项必须在 \cmd{usepackage} 的可选项中给出。
\secref{use:opt:pre} 节中的宏包选项同样可以在导言区中给出。
以下命令用于执行选项:

\begin{ltxsyntax}

\cmditem{ExecuteBibliographyOptions}[entrytype, \dots]{key=value, \dots}

%This command may also be used in the configuration file to modify the default setting of a package option. Certain options are also settable on a per-type basis. In this case, the optional \prm{entrytype} argument specifies the entry type. The \prm{entrytype} argument may be a comma"=separated list of values.

该命令也可以在配置文件中使用,以修改宏包选项的默认设置。
某些选项还可以基于每一条目而设置。
此时,可选的 \prm{entrytype} 选项用来确定条目类型。
\prm{entrytype} 选项可以是逗号分隔的值列表。

\end{ltxsyntax}

\subsection{标准样式}%Standard Styles
\label{use:xbx}

%This section provides a short description of all bibliography and citation styles which ship with the \biblatex package. If you want to write your own styles, see \secref{aut}.

本节简要地描述了本宏包所带的所有文献和引用样式。
如果你想自己写样式文件,请参考 \secref{aut} 节。

\subsubsection{标注样式}%\subsubsection{Citation Styles}
\label{use:xbx:cbx}

%The citation styles which come with this package implement several common citation schemes. All standard styles cater for the \bibfield{shorthand} field and support hyperlinks as well as indexing.

本宏包所带的引用样式实现了一些常见的引用格式。
所有的标准样式都支持 \bibfield{shorthand} 域,并且支持超链接和索引。


\begin{marglist}

\item[numeric]
%This style implements a numeric citation scheme similar to the standard bibliographic facilities of \latex. It should be employed in conjunction with a numeric bibliography style which prints the corresponding labels in the bibliography. It is intended for in-text citations. The style will set the following package options at load time: \kvopt{autocite}{inline}, \kvopt{labelnumber}{true}. This style also provides an additional preamble option called \opt{subentry} which affects the handling of entry sets. If this option is disabled, citations referring to a member of a set will point to the entire set. If it is enabled, the style supports citations like «[5c]» which point to a subentry in a set (the third one in this example). See the style example for details.
该样式实现与 \LaTeX 的标准文献工具类似的数值式引用格式。
它应当与某种能在参考文献中打印出相应标签的数值式文献样式一起使用,
并用于文内引用。
该样式在宏包载入时设置如下的宏包选项:\kvopt{autocite}{inline}、\kvopt{labelnumber}{true}。
该样式还额外提供了一个导言区选项 \opt{subentry},这会影响条目集的处理。
如果该选项被禁止,指向条目集中某一成员的引用会指向整个的条目集。
如果该选项被激活,该样式会支持类似于“[5c]”这样指向条目集中的子条目的引用(这个例子是第三个条目)。
详见样式例子。

\item[numeric-comp]
%A compact variant of the \texttt{numeric} style which prints a list of more than two consecutive numbers as a range. This style is similar to the \sty{cite} package and the \opt{sort\&compress} option of the \sty{natbib} package in numerical mode. For example, instead of «[8, 3, 1, 7, 2]» this style would print «[1--3, 7, 8]». It is intended for in-text citations. The style will set the following package options at load time: \kvopt{autocite}{inline}, \kvopt{sortcites}{true}, \kvopt{labelnumber}{true}. It also provides the \opt{subentry} option.
\texttt{numeric} 样式紧凑形式的变种,会将两个以上的连续数字打印成一个区间。
该样式类似于 \sty{cite} 宏包和数值模式中的 \sty{natbib} 宏包的 \opt{sort\&compress} 选项。
例如,“[8, 3, 1, 7, 2]”会变成“[1--3, 7, 8]”。
它用于文中引用。
该样式在宏包载入时设置如下的宏包选项:
\kvopt{autocite}{inline}、\kvopt{sortcites}{true}、\kvopt{labelnumber}{true}。
它也提供了 \opt{subentry} 选项。

\item[numeric-verb]
%A verbose variant of the \texttt{numeric} style. The difference affects the handling of a list of citations and is only apparent when multiple entry keys are passed to a single citation command. For example, instead of «[2, 5, 6]» this style would print «[2]; [5]; [6]». It is intended for in-text citations. The style will set the following package options at load time: \kvopt{autocite}{inline}, \kvopt{labelnumber}{true}. It also provides the \opt{subentry} option.
\texttt{numeric} 样式详细形式的变种。
不同之处在于对一组引用的处理,
并且只当不同的条目键值传递给单个引用命令时才会显示差异。
例如,“[2, 5, 6]”会变成“[2]; [5]; [6]”。
它用于文中引用。
该样式在宏包载入时设置如下的宏包选项:
\kvopt{autocite}{inline}、\kvopt{labelnumber}{true}。
它也提供了 \opt{subentry} 选项。

\item[alphabetic]
%This style implements an alphabetic citation scheme similar to the \path{alpha.bst} style of traditional \bibtex. The alphabetic labels resemble a compact author"=year style to some extent, but the way they are employed is similar to a numeric citation scheme. For example, instead of «Jones 1995» this style would use the label «[Jon95]». «Jones and Williams 1986» would be rendered as «[JW86]». This style should be employed in conjunction with an alphabetic bibliography style which prints the corresponding labels in the bibliography. It is intended for in-text citations. The style will set the following package options at load time: \kvopt{autocite}{inline}, \kvopt{labelalpha}{true}.

该样式实现的字母顺序引用格式类似于传统 \BibTeX 的 \path{alpha.bst} 样式。
字母标签某种程度上类似于紧凑的作者---年份样式,但是使用的方式类似于数字引用格式。
例如,“Jones 1995” 会是“[Jon95]”;而“Jones and Williams 1986”会缩写为“[JW86]”。
该样式应当与一种字母顺序文献样式一起使用,从而可以在参考文献中打印出相应的标签。
它用于文内引用。
该样式在载入时设置如下的宏包选项:\kvopt{autocite}{inline}、\kvopt{labelalpha}{true}。

\item[alphabetic-verb]
%A verbose variant of the \texttt{alphabetic} style. The difference affects the handling of a list of citations and is only apparent when multiple entry keys are passed to a single citation command. For example, instead of «[Doe92; Doe95; Jon98]» this style would print «[Doe92]; [Doe95]; [Jon98]». It is intended for in-text citations. The style will set the following package options at load time: \kvopt{autocite}{inline}, \kvopt{labelalpha}{true}.
\texttt{alphabetic} 样式详细格式的变种。
不同之处在于对一组引用的处理,
并且只当不同的条目键值传递给单个引用命令时才会显示差异。
例如“[Doe92; Doe95; Jon98]”会变成“[Doe92]; [Doe95]; [Jon98]”。
它用于文内引用。
该样式在载入时设置如下的宏包选项:\kvopt{autocite}{inline}、\kvopt{labelalpha}{true}。

\item[authoryear]
%This style implements an author"=year citation scheme. If the bibliography contains two or more works by the same author which were all published in the same year, a letter is appended to the year. For example, this style would print citations such as «Doe 1995a; Doe 1995b; Jones 1998». This style should be employed in conjunction with an author"=year bibliography style which prints the corresponding labels in the bibliography. It is primarily intended for in-text citations, but it could also be used with citations given in footnotes. The style will set the following package options at load time: \kvopt{autocite}{inline}, \kvopt{labeldate}{true}, \kvopt{uniquename}{full}, \kvopt{uniquelist}{true}.
该样式实现了作者---年份引用格式。
如果参考文献中包含多个由同一作者同一年份发表的作品,那么年份后会附加一个字母用以区分。
例如该样式会打印出“Doe 1995a; Doe 1995b; Jones 1998”这样的引用。
该样式应当与一种作者---年份文献样式一起使用,从而可以在参考文献中打印出相应的标签。
它起初用于文内引用,但也可以用在脚注中。
该样式在载入时设置如下的宏包选项:
\kvopt{autocite}{inline}、\kvopt{labeldate}{true}、\kvopt{uniquename}{full}、\kvopt{uniquelist}{true}。

\item[authoryear-comp]
%A compact variant of the \texttt{authoryear} style which prints the author only once if subsequent references passed to a single citation command share the same author. If they share the same year as well, the year is also printed only once. For example, instead of «Doe 1995b; Doe 1992; Jones 1998; Doe 1995a» this style would print «Doe 1992, 1995a,b; Jones 1998». It is primarily intended for in-text citations, but it could also be used with citations given in footnotes. The style will set the following package options at load time: \kvopt{autocite}{inline}, \kvopt{sortcites}{true}, \kvopt{labeldate}{true}, \kvopt{uniquename}{full}, \kvopt{uniquelist}{true}.
\texttt{authoryear} 样式紧凑格式的变种。
如果传递给单个引用命令的一列文献作者相同,那么该作者只会打印一次。
如果它们年份也相同,那么年份也只会打印一次。
例如,“Doe 1995b; Doe 1992; Jones 1998; Doe 1995a”在该样式下会变成“Doe 1992, 1995a,b; Jones 1998”。
它起初用于文中引用,但也可以用在脚注中。
该样式在载入时设置如下的宏包选项:
\kvopt{autocite}{inline}、\kvopt{sortcites}{true}、\kvopt{labeldate}{true}、\kvopt{uniquename}{full}、\kvopt{uniquelist}{true}。

\item[authoryear-ibid]
%A variant of the \texttt{authoryear} style which replaces repeated citations by the abbreviation \emph{ibidem} unless the citation is the first one on the current page or double-page spread, or the \emph{ibidem} would be ambiguous in the sense of the package option \kvopt{ibidtracker}{constrict}. The style will set the following package options at load time: \kvopt{autocite}{inline}, \kvopt{labeldate}{true}, \kvopt{uniquename}{full}, \kvopt{uniquelist}{true}, \kvopt{ibidtracker}{constrict}, \kvopt{pagetracker}{true}. This style also provides an additional preamble option called \opt{ibidpage}. See the style example for details.
\texttt{authoryear} 样式的变种,会用缩略语 \emph{ibidem} 替代重复的引用,
除非该引用在当前页或跨页是第一次出现,
或者 \emph{ibidem} 在宏包选项 \kvopt{ibidtracker}{constrict} 的意义下表意不清。
该样式在载入时设置如下的宏包选项:
\kvopt{autocite}{inline}、\kvopt{labeldate}{true}、\kvopt{uniquename}{full}、\kvopt{uniquelist}{true}、\kvopt{ibidtracker}{constrict}、\kvopt{pagetracker}{true}。
该样式还额外提供了一个导言区选项 \opt{ibidpage}。
详见样式例子。

\item[authoryear-icomp]
%A style combining \texttt{authoryear-comp} and \texttt{authoryear-ibid}. The style will set the following package options at load time: \kvopt{autocite}{inline}, \kvopt{labeldate}{true}, \kvopt{uniquename}{full}, \kvopt{uniquelist}{true}, \kvopt{ibidtracker}{constrict}, \kvopt{pagetracker}{true}, \kvopt{sortcites}{true}. This style also provides an additional preamble option called \opt{ibidpage}. See the style example for details.
一个结合了 \texttt{authoryear-comp} 和 \texttt{authoryear-ibid} 的样式。
该样式在载入时设置如下的宏包选项:
\kvopt{autocite}{inline}、\kvopt{labeldate}{true}、\kvopt{uniquename}{full}、\kvopt{uniquelist}{true}、\kvopt{ibidtracker}{constrict}、\kvopt{pagetracker}{true}、\kvopt{sortcites}{true}。
该样式还额外提供了一个导言区选项 \opt{ibidpage}。
详见样式例子。

\item[authortitle]
%This style implements a simple author"=title citation scheme. It will make use of the \bibfield{shorttitle} field, if available. It is intended for citations given in footnotes. The style will set the following package options at load time: \kvopt{autocite}{footnote}, \kvopt{uniquename}{full}, \kvopt{uniquelist}{true}.
该样式实现了一个简单的作者---标题引用格式。
如果可用的话,它会使用 \bibfield{shorttitle} 域。
它用于脚注中给出的引用。
该样式在载入时设置如下的宏包选项:
\kvopt{autocite}{footnote}、\kvopt{uniquename}{full}、\kvopt{uniquelist}{true}。

\item[authortitle-comp]
%A compact variant of the \texttt{authortitle} style which prints the author only once if subsequent references passed to a single citation command share the same author. For example, instead of «Doe, \emph{First title}; Doe, \emph{Second title}» this style would print «Doe, \emph{First title}, \emph{Second title}». It is intended for citations given in footnotes. The style will set the following package options at load time: \kvopt{autocite}{footnote}, \kvopt{sortcites}{true}, \kvopt{uniquename}{full}, \kvopt{uniquelist}{true}.
\texttt{authortitle} 样式紧凑格式的变种。
如果传递给单个引用命令的一列文献作者相同,那么该作者只会打印一次。
例如,“Doe, \emph{First title}; Doe, \emph{Second title}”在此样式下会变成“Doe, \emph{First title}, \emph{Second title}”。
它用于脚注中给出的引用。
该样式在载入时设置如下的宏包选项:
\kvopt{autocite}{footnote}、\kvopt{sortcites}{true}、\kvopt{uniquename}{full}、\kvopt{uniquelist}{true}。

\item[authortitle-ibid]
%A variant of the \texttt{authortitle} style which replaces repeated citations by the abbreviation \emph{ibidem} unless the citation is the first one on the current page or double-page spread, or the \emph{ibidem} would be ambiguous in the sense of the package option \kvopt{ibidtracker}{constrict}. It is intended for citations given in footnotes. The style will set the following package options at load time: \kvopt{autocite}{footnote}, \kvopt{uniquename}{full}, \kvopt{uniquelist}{true}, \kvopt{ibidtracker}{constrict}, \kvopt{pagetracker}{true}. This style also provides an additional preamble option called \opt{ibidpage}. See the style example for details.
\texttt{authortitle} 样式的变种,会用缩略语 \emph{ibidem} 替代重复的引用,
除非该引用在当前页或跨页是第一次出现,
或者 \emph{ibidem} 在宏包选项 \kvopt{ibidtracker}{constrict} 的意义下表意不清。
它用于脚注中给出的引用。
该样式在载入时设置如下的宏包选项:
\kvopt{autocite}{footnote}、\kvopt{uniquename}{full}、\kvopt{uniquelist}{true}、\kvopt{ibidtracker}{constrict}、\kvopt{pagetracker}{true}。
该样式还额外提供了一个导言区选项 \opt{ibidpage}。
详见样式例子。

\item[authortitle-icomp]
%A style combining the features of \texttt{authortitle-comp} and \texttt{authortitle-ibid}. The style will set the following package options at load time: \kvopt{autocite}{footnote}, \kvopt{uniquename}{full}, \kvopt{uniquelist}{true}, \kvopt{ibidtracker}{constrict}, \kvopt{pagetracker}{true}, \kvopt{sortcites}{true}. This style also provides an additional preamble option called \opt{ibidpage}. See the style example for details.
结合了 \texttt{authortitle-comp} 和 \texttt{authortitle-ibid} 特性的样式。
该样式在载入时设置如下的宏包选项:
\kvopt{autocite}{footnote}、\kvopt{uniquename}{full}、\kvopt{uniquelist}{true}、\kvopt{ibidtracker}{constrict}、\kvopt{pagetracker}{true}、\kvopt{sortcites}{true}。
该样式还额外提供了一个导言区选项 \opt{ibidpage}。
详见样式例子。

\item[authortitle-terse]
%A terse variant of the \texttt{authortitle} style which only prints the title if the bibliography contains more than one work by the respective author\slash editor. This style will make use of the \bibfield{shorttitle} field, if available. It is suitable for in-text citations as well as citations given in footnotes. The style will set the following package options at load time: \kvopt{autocite}{inline}, \kvopt{singletitle}{true}, \kvopt{uniquename}{full}, \kvopt{uniquelist}{true}.
\texttt{authortitle} 样式简明格式的变种,
如果文献中包含多个相应作者/编辑的作品,那么只会打印出标题。
如果可用的话,该样式会使用 \bibfield{shorttitle} 域。
它在文中引用和脚注中引用都适用。
该样式在载入时设置如下的宏包选项:\kvopt{autocite}{inline}、\kvopt{singletitle}{true}、\kvopt{uniquename}{full}、\kvopt{uniquelist}{true}。

\item[authortitle-tcomp]
%A style combining the features of \texttt{authortitle-comp} and \texttt{authortitle-terse}. This style will make use of the \bibfield{shorttitle} field, if available. It is suitable for in-text citations as well as citations given in footnotes. The style will set the following package options at load time: \kvopt{autocite}{inline}, \kvopt{sortcites}{true}, \kvopt{singletitle}{true}, \kvopt{uniquename}{full}, \kvopt{uniquelist}{true}.
结合了 \texttt{authortitle-comp} 和 \texttt{authortitle-terse} 特性的样式。
如果可用的话,该样式会使用 \bibfield{shorttitle} 域。
它在文内引用和脚注中引用都适用。
该样式在载入时设置如下的宏包选项:
\kvopt{autocite}{inline}、\kvopt{sortcites}{true}、\kvopt{singletitle}{true}、\kvopt{uniquename}{full}、\kvopt{uniquelist}{true}。

\item[authortitle-ticomp]
%A style combining the features of \texttt{authortitle-icomp} and \texttt{authortitle-terse}. In other words: a variant of the \texttt{authortitle-tcomp} style with an \emph{ibidem} feature. This style is suitable for in-text citations as well as citations given in footnotes. It will set the following package options at load time: \kvopt{autocite}{inline}, \kvopt{ibidtracker}{constrict}, \kvopt{pagetracker}{true}, \kvopt{sortcites}{true}, \kvopt{singletitle}{true}, \kvopt{uniquename}{full}, \kvopt{uniquelist}{true}. This style also provides an additional preamble option called \opt{ibidpage}. See the style example for details.
结合了 \texttt{authortitle-icomp} 和 \texttt{authortitle-terse} 特性的样式。
换句话说就是带有 \emph{ibidem} 特性的 \texttt{authortitle-tcomp} 样式变种。
它在文中引用和脚注中引用都适用。
该样式在载入时设置如下的宏包选项:
\kvopt{autocite}{inline}、\kvopt{ibidtracker}{constrict}、\kvopt{pagetracker}{true}、\kvopt{sortcites}{true}、\kvopt{singletitle}{true}、\kvopt{uniquename}{full}、\kvopt{uniquelist}{true}。
该样式还额外提供了一个导言区选项 \opt{ibidpage}。
详见样式例子。

\item[verbose]
%A verbose citation style which prints a full citation similar to a bibliography entry when an entry is cited for the first time, and a short citation afterwards. If available, the \bibfield{shorttitle} field is used in all short citations. If the \bibfield{shorthand} field is defined, the shorthand is introduced on the first citation and used as the short citation thereafter. This style may be used without a list of references and shorthands since all bibliographic data is provided on the first citation. It is intended for citations given in footnotes. The style will set the following package options at load time: \kvopt{autocite}{footnote}, \kvopt{citetracker}{context}. This style also provides an additional preamble option called \opt{citepages}. See the style example for details.
详细的引用样式,在第一次引用某条目时会打印出类似于参考文献那样的长引用格式,并且在之后打印出短格式。
如果可用的话,\bibfield{shorttitle} 域会用在所有的短格式中。
如果 \bibfield{shorthand} 域有定义,该shorthand会在第一次引用时被引入并在之后作为短格式被使用。
由于在第一次引用时提供了所有的文献数据,因此该样式的使用不需要参考文献和shorthand列表。
它用于脚注中给出的引用。
该样式在载入时设置如下的宏包选项:\kvopt{autocite}{footnote}、\kvopt{citetracker}{context}。
该样式还额外提供了一个导言区选项 \opt{citepages}。
详见样式例子。

\item[verbose-ibid]
%A variant of the \texttt{verbose} style which replaces repeated citations by the abbreviation \emph{ibidem} unless the citation is the first one on the current page or double-page spread, or the \emph{ibidem} would be ambiguous in the sense of \kvopt{ibidtracker}{strict}. This style is intended for citations given in footnotes. The style will set the following package options at load time: \kvopt{autocite}{footnote}, \kvopt{citetracker}{context}, \kvopt{ibidtracker}{constrict}, \kvopt{pagetracker}{true}. This style also provides additional preamble options called \opt{ibidpage} and \opt{citepages}. See the style example for details.
\texttt{verbose} 样式的变种,会用缩略语 \emph{ibidem} 替代重复的引用,
除非该引用在当前页或跨页是第一次出现,
或者 \emph{ibidem} 在宏包选项 \kvopt{ibidtracker}{strict} 的意义下表意不清。
它用于脚注中给出的引用。
该样式在载入时设置如下的宏包选项:
\kvopt{autocite}{footnote}、\kvopt{citetracker}{context}、\kvopt{ibidtracker}{constrict}、\kvopt{pagetracker}{true}。
该样式还额外提供了导言区选项 \opt{ibidpage} 和 \opt{citepages}。
详见样式例子。

\item[verbose-note]
%This style is similar to the \texttt{verbose} style in that it prints a full citation similar to a bibliography entry when an entry is cited for the first time, and a short citation afterwards. In contrast to the \texttt{verbose} style, the short citation is a pointer to the footnote with the full citation. If the bibliography contains more than one work by the respective author\slash editor, the pointer also includes the title. If available, the \bibfield{shorttitle} field is used in all short citations. If the \bibfield{shorthand} field is defined, it is handled as with the \texttt{verbose} style. This style may be used without a list of references and shorthands since all bibliographic data is provided on the first citation. It is exclusively intended for citations given in footnotes. The style will set the following package options at load time: \kvopt{autocite}{footnote}, \kvopt{citetracker}{context}, \kvopt{singletitle}{true}. This style also provides additional preamble options called \opt{pageref} and \opt{citepages}. See the style example for details.
该样式与 \texttt{verbose} 样式类似,
会在第一次引用某条目时打印出类似于参考文献那样的长格式,并且在之后打印出短格式。
与 \texttt{verbose} 样式不同的是,短格式会指向带有长格式的脚注。
如果文献包含了多个同一作者/编辑的作品,该短格式会带有标题。
如果可用的话,所有的短格式会使用 \bibfield{shorttitle} 域。
如果 \bibfield{shorthand} 域有定义,它会被 \texttt{verbose} 样式处理。
由于在第一次引用时提供了所有的文献数据,因此该样式的使用不需要参考文献和shorthand列表。
该样式仅仅用于脚注中给出的引用。
该样式在载入时设置如下的宏包选项:
\kvopt{autocite}{footnote}、\kvopt{citetracker}{context}、\kvopt{singletitle}{true}。
该样式还额外提供了导言区选项 \opt{pageref} 和 \opt{citepages}。
详见样式例子。

\item[verbose-inote]
%A variant of the \texttt{verbose"=note} style which replaces repeated citations by the abbreviation \emph{ibidem} unless the citation is the first one on the current page or double-page spread, or the \emph{ibidem} would be ambiguous in the sense of \kvopt{ibidtracker}{strict}. This style is exclusively intended for citations given in footnotes. It will set the following package options at load time: \kvopt{autocite}{footnote}, \kvopt{citetracker}{context}, \kvopt{ibidtracker}{constrict}, \kvopt{singletitle}{true}, \kvopt{pagetracker}{true}. This style also provides additional preamble options called \opt{ibidpage}, \opt{pageref}, and \opt{citepages}. See the style example for details.
\texttt{verbose"=note} 样式的变种,会用缩略语 \emph{ibidem} 替代重复的引用,
除非该引用在当前页或跨页是第一次出现,
或者 \emph{ibidem} 在宏包选项 \kvopt{ibidtracker}{strict} 的意义下表意不清。
该样式仅仅用于脚注中给出的引用。
该样式在载入时设置如下的宏包选项:
\kvopt{autocite}{footnote}、\kvopt{citetracker}{context}、\kvopt{ibidtracker}{constrict}、\kvopt{singletitle}{true}、\kvopt{pagetracker}{true}。
该样式还额外提供了导言区选项 \opt{ibidpage}、\opt{pageref} 和 \opt{citepages}。
详见样式例子。

\item[verbose-trad1]
%This style implements a traditional citation scheme. It is similar to the \texttt{verbose} style in that it prints a full citation similar to a bibliography entry when an item is cited for the first time, and a short citation afterwards. Apart from that, it uses the scholarly abbreviations \emph{ibidem}, \emph{idem}, \emph{op.~cit.}, and \emph{loc.~cit.} to replace recurrent authors, titles, and page numbers in repeated citations in a special way. If the \bibfield{shorthand} field is defined, the shorthand is introduced on the first citation and used as the short citation thereafter. This style may be used without a list of references and shorthands since all bibliographic data is provided on the first citation. It is intended for citations given in footnotes. The style will set the following package options at load time: \kvopt{autocite}{footnote}, \kvopt{citetracker}{context}, \kvopt{ibidtracker}{constrict}, \kvopt{idemtracker}{constrict}, \kvopt{opcittracker}{context}, \kvopt{loccittracker}{context}. This style also provides additional preamble options called \opt{ibidpage}, \opt{strict}, and \opt{citepages}. See the style example for details.
该样式实现了传统的引用格式。
与 \texttt{verbose} 样式类似,
它会在第一次引用某条目时打印出类似于参考文献那样的长格式,并且在之后打印出短格式。
此外,它在重复的引用中使用学术性缩略语 \emph{ibidem}、\emph{idem}、\emph{op.~cit.} 和 \emph{loc.~cit.} 来代替重复的作者、标题、页码数。
如果 \bibfield{shorthand} 域有定义,那么会在第一次引用时被引入并在之后作为短格式被使用。
由于在第一次引用时提供了所有的文献数据,因此该样式的使用不需要参考文献和shorthand列表。
它用于脚注中给出的引用。
该样式在载入时设置如下的宏包选项:
\kvopt{autocite}{footnote}、\kvopt{citetracker}{context}、\kvopt{ibidtracker}{constrict}、\kvopt{idemtracker}{constrict}、\kvopt{opcittracker}{context}、\kvopt{loccittracker}{context}。
该样式还额外提供了导言区选项 \opt{ibidpage}、\opt{strict} 和 \opt{citepages}。
详见样式例子。

\item[verbose-trad2]
%Another traditional citation scheme. It is also similar to the \texttt{verbose} style but uses scholarly abbreviations like \emph{ibidem} and \emph{idem} in repeated citations. In contrast to the \texttt{verbose-trad1} style, the logic of the \emph{op.~cit.} abbreviations is different in this style and \emph{loc.~cit.} is not used at all. It is in fact more similar to \texttt{verbose-ibid} and \texttt{verbose-inote} than to \texttt{verbose-trad1}. The style will set the following package options at load time: \kvopt{autocite}{footnote}, \kvopt{citetracker}{context}, \kvopt{ibidtracker}{constrict}, \kvopt{idemtracker}{constrict}. This style also provides additional preamble options called \opt{ibidpage}, \opt{strict}, and \opt{citepages}. See the style example for details.
另外一种传统引用格式。
它同样类似于 \texttt{verbose} 样式但是在重复的引用中使用 \emph{ibidem} 和 \emph{idem} 等学术性缩略语。
与 \texttt{verbose-trad1} 样式不同的是,
\emph{op.~cit.} 缩略语的逻辑有所不同,并且不使用 \emph{loc.~cit.}。
事实上它更类似于 \texttt{verbose-ibid} 和 \texttt{verbose-inote} 而不是 \texttt{verbose-trad1}。
该样式在载入时设置如下的宏包选项:
\kvopt{autocite}{footnote}、\kvopt{citetracker}{context}、\kvopt{ibidtracker}{constrict}、\kvopt{idemtracker}{constrict}。
该样式还额外提供了导言区选项 \opt{ibidpage}、\opt{strict} 和 \opt{citepages}。
详见样式例子。

\item[verbose-trad3]
%Yet another traditional citation scheme. It is similar to the \texttt{verbose-trad2} style but uses the scholarly abbreviations \emph{ibidem} and \emph{op.~cit.} in a slightly different way. The style will set the following package options at load time: \kvopt{autocite}{footnote}, \kvopt{citetracker}{context}, \kvopt{ibidtracker}{constrict}, \kvopt{loccittracker}{constrict}. This style also provides additional preamble options called \opt{strict} and \opt{citepages}. See the style example for details.
仍然是一种传统的引用格式。
它类似于 \texttt{verbose-trad2} 样式,
但是使用缩略语 \emph{ibidem} 和 \emph{op.~cit.} 的方式稍有不同。
该样式在载入时设置如下的宏包选项:
\kvopt{autocite}{footnote}、\kvopt{citetracker}{context}、\kvopt{ibidtracker}{constrict}、\kvopt{loccittracker}{constrict}。
该样式也额外提供了导言区选项 \opt{strict} 和 \opt{citepages}。
详见样式例子。

\item[reading]
%A citation style which goes with the bibliography style by the same name. It simply loads the \texttt{authortitle} style.
一个同名的文献样式所带的引用样式,会载入 \texttt{authortitle} 样式。

\end{marglist}

%The following citation styles are special purpose styles. They are not intended for the final version of a document:
下列样式具有特殊目的,不用于文档的最终版本。

\begin{marglist}

\item[draft]
%A draft style which uses the entry keys in citations. The style will set the following package options at load time: \kvopt{autocite}{plain}.
在引用中使用条目键的草稿样式。
该样式在载入时设置如下的宏包选项:\kvopt{autocite}{plain}。

\item[debug]
%This style prints the entry key rather than some kind of label. It is intended for debugging only and will set the following package options at load time: \kvopt{autocite}{plain}.
该样式会打印出条目键而不是标签。
它只用于调试,在载入时设置如下的宏包选项:\kvopt{autocite}{plain}。

\end{marglist}

\subsubsection{参考文献样式}%\subsubsection{Bibliography Styles}
\label{use:xbx:bbx}

%All bibliography styles which come with this package use the same basic format for the individual bibliography entries. They only differ in the kind of label printed in the bibliography and the overall formatting of the list of references. There is a matching bibliography style for every citation style. Note that some bibliography styles are not mentioned below because they simply load a more generic style. For example, the bibliography style \texttt{authortitle-comp} will load the \texttt{authortitle} style.

本宏包所带的所有文献样式对于每一文献条目都使用相同的基本格式。
不同之处仅仅在于参考文献中打印的标签种类和文献列表的总体格式。
每一个引用样式都有一个对应的文献样式。
请注意,一些文献样式仅仅载入了另外更一般的样式,因此这里没有提及。
例如,文献样式 \texttt{authortitle-comp} 会载入 \texttt{authortitle} 样式。

\begin{marglist}

\item[numeric]
%This style prints a numeric label similar to the standard bibliographic facilities of \latex. It is intended for use in conjunction with a numeric citation style. Note that the \bibfield{shorthand} field overrides the default label. The style will set the following package options at load time: \kvopt{labelnumber}{true}. This style also provides an additional preamble option called \opt{subentry} which affects the formatting of entry sets. If this option is enabled, all members of a set are marked with a letter which may be used in citations referring to a set member rather than the entire set. See the style example for details.
该样式打印出类似于 \LaTeX 标准文献功能的数值标签。
它应与数值引用样式结合使用。
请注意,\bibfield{shorthand} 域会覆盖默认标签。
该样式在载入时设置如下的宏包选项:\kvopt{labelnumber}{true}。
该样式还额外提供了一个导言区选项 \opt{subentry},这会影响条目集的处理。
如果该选项被激活,条目集中的所有成员都会用一个字母标记,这可用于集成员的引用而不是整个条目集。
详见样式例子。

\item[alphabetic]
%This style prints an alphabetic label similar to the \path{alpha.bst} style of traditional \bibtex. It is intended for use in conjunction with an alphabetic citation style. Note that the \bibfield{shorthand} field overrides the default label. The style will set the following package options at load time: \kvopt{labelalpha}{true}, \kvopt{sorting}{anyt}.
该样式打印的字母顺序标签类似于传统 \BibTeX 的 \path{alpha.bst} 样式。
它应与字母顺序引用样式结合使用。
请注意,\bibfield{shorthand} 域会覆盖默认标签。
该样式在载入时设置如下的宏包选项:\kvopt{labelalpha}{true}、\kvopt{sorting}{anyt}。

\item[authoryear]
%This style differs from the other styles in that the publication date is not printed towards the end of the entry but rather after the author\slash editor. It is intended for use in conjunction with an author"=year citation style. Recurring author and editor names are replaced by a dash unless the entry is the first one on the current page or double-page spread. This style provides an additional preamble option called \opt{dashed} which controls this feature. It also provided a preamble option called \opt{mergedate}. See the style example for details. The style will set the following package options at load time: \kvopt{labeldate}{true}, \kvopt{sorting}{nyt}, \kvopt{pagetracker}{true}, \kvopt{mergedate}{true}.
该样式不同于其它样式之处在于,发表日期不是在条目的末尾而是在作者/编辑之后。
它应与一个作者---年份引用样式结合使用。
重复的作者和编辑名会用短横线代替,除非该条目是当前页或跨页的第一个。
该样式额外提供了导言区选项 \opt{dashed} 来控制该特征。
此外还额外提供了导言区选项 \opt{mergedate}。
详见样式例子。
该样式在载入时设置如下的宏包选项:
\kvopt{labeldate}{true}、\kvopt{sorting}{nyt}、\kvopt{pagetracker}{true}、 \kvopt{mergedate}{true}。

\item[authortitle]
%This style does not print any label at all. It is intended for use in conjunction with an author"=title citation style. Recurring author and editor names are replaced by a dash unless the entry is the first one on the current page or double-page spread. This style also provides an additional preamble option called \opt{dashed} which controls this feature. See the style example for details. The style will set the following package options at load time: \kvopt{pagetracker}{true}.
该样式不会打印出任何标签。
它应与一个作者---年份引用样式结合使用。
重复的作者和编辑名会用短横线代替,除非该条目是当前页或跨页的第一个。
该样式额外提供了一个导言区选项 \opt{dashed} 来控制该特征。
详见样式例子。
该样式在载入时设置如下的宏包选项:\kvopt{pagetracker}{true}。

\item[verbose]
%This style is similar to the \texttt{authortitle} style. It also provides an additional preamble option called \opt{dashed}. See the style example for details. The style will set the following package options at load time: \kvopt{pagetracker}{true}.
该样式类似于 \texttt{authortitle} 样式。
该样式额外提供了一个导言区选项 \opt{dashed}。
详见样式例子。
该样式在载入时设置如下的宏包选项:\kvopt{pagetracker}{true}。

\item[reading]
%This special bibliography style is designed for personal reading lists, annotated bibliographies, and similar applications. It optionally includes the fields \bibfield{annotation}, \bibfield{abstract}, \bibfield{library}, and \bibfield{file} in the bibliography. If desired, it also adds various kinds of short headers to the bibliography. This style also provides the additional preamble options \opt{entryhead}, \opt{entrykey}, \opt{annotation}, \opt{abstract}, \opt{library}, and \opt{file} which control whether or not the corresponding items are printed in the bibliography. See the style example for details. See also \secref{use:use:prf}. The style will set the following package options at load time: \kvopt{loadfiles}{true}, \kvopt{entryhead}{true}, \kvopt{entrykey}{true}, \kvopt{annotation}{true}, \kvopt{abstract}{true}, \kvopt{library}{true}, \kvopt{file}{true}.
这一特殊的文献样式是为个人阅读列表、带有注释的文献和类似应用而设计的。
它选择性地在参考文献中包含 \bibfield{annotation}、\bibfield{abstract}、\bibfield{library} 和 \bibfield{file} 等域。
如果需要的话,它还会在参考文献中添加不同种类的短标题。
该样式还额外提供了导言区选项 \opt{entryhead}、\opt{entrykey}、\opt{annotation}、\opt{abstract}、\opt{library} 和 \opt{file} 来控制是否在参考文献中打印相应的项目。
详见样式例子。见 \secref{use:use:prf} 节。
该样式在载入时设置如下的宏包选项:
\kvopt{loadfiles}{true}、\kvopt{entryhead}{true}、\kvopt{entrykey}{true}、\kvopt{annotation}{true}、\kvopt{abstract}{true}、\kvopt{library}{true}、\kvopt{file}{true}。

\end{marglist}

%The following bibliography styles are special purpose styles. They are not intended for the final version of a document:
下列样式具有特殊目的,不用于文档的最终版本。

\begin{marglist}

\item[draft]
%This draft style includes the entry keys in the bibliography. The bibliography will be sorted by entry key. The style will set the following package options at load time: \kvopt{sorting}{debug}.
草稿样式会在参考文献中包含条目键。
文献会按照条目键排序。
该样式在载入时设置如下的宏包选项:\kvopt{sorting}{debug}。

\item[debug]
%This style prints all bibliographic data in tabular format. It is intended for debugging only and will set the following package options at load time: \kvopt{sorting}{debug}.
该样式会以表格形式打印出所有的文献数据。
它只用于调试,在载入时设置如下的宏包选项:\kvopt{sorting}{debug}。

\end{marglist}

\subsection{关联条目}%\subsection{Related Entries}
\label{use:rel}

%Almost all bibliography styles require authors to specify certain types of relationship between entries such as «Reprint of», «Reprinted in» etc. It is impossible to provide data fields to cover all of these relationships and so \biblatex provides a general mechanism for this using the entry fields \bibfield{related}, \bibfield{relatedtype} and \bibfield{relatedstring}. A related entry does not need to be cited and does not appear in the bibliography itself (unless of course it is also cited itself independently) as a clone is taken of the related entry to be used as a data source. The \bibfield{relatedtype} field should specify a localization string which will be printed before the information from the related entries is printed, for example «Orig. Pub. as». The \bibfield{relatedstring} field can be used to override the string determined via \bibfield{relatedtype}. Some examples:

几乎所有的文献样式都需要作者去确定条目之间的某些关系类型,例如“Reprint of”、“Reprinted in”等等。
当然不可能通过提供数据域来覆盖所有的关系,
为此,\biblatex 通过使用条目域 \bibfield{related}、\bibfield{relatedtype} 和 \bibfield{relatedstring} 提供了一种一般性的机制。
被关联的条目不需要被引用,本身也不会出现在参考文献中(当然,除非它自己另外单独被引用),
而是作为数据源被拷贝一份副本。
\bibfield{relatedtype} 域需要确定在相关联条目的信息前打印的本地化字符串,例如“Orig. Pub. as”。
\bibfield{relatedstring} 域可以用于覆盖那些通过 \bibfield{relatedtype} 确定的字符串。
一些例子如下:

\begin{lstlisting}[style=bibtex]{}
@Book{key1,
  ...
  related     = {key2},
  relatedtype = {reprintof},
  ...
}

@Book{key2,
  ...
}
\end{lstlisting}
%
%Here we specify that entry \texttt{key1} is a reprint of entry \texttt{key2}. In the bibliography driver for \texttt{Book} entries, when \cmd{usebibmacro\{related\}} is called for entry \texttt{key1}:
这里我们指定条目 \texttt{key1} 是条目 \texttt{key2} 的重印本。
在 \texttt{Book} 条目的文献驱动里,
当为条目 \texttt{key1} 而调用 \cmd{usebibmacro\{related\}} 时:

\begin{itemize}
\item % If the localisation string «\texttt{reprintof}» is defined, it is printed in the \texttt{relatedstring:reprintof} format. If this formatting directive is undefined, the string is printed in the \texttt{relatedstring:default} format.
如果本地化字符串 “\texttt{reprintof}” 有定义,
那么将以 \texttt{relatedstring:reprintof} 格式打印出来。
如果该格式指令没有定义,这些字符串将以 \texttt{relatedstring:default} 格式打印。
\item %If the \texttt{related:reprintof} macro is defined, it is used to format the information contained in entry \texttt{key2}, otherwise the \texttt{related:default} macro is used
如果宏 \texttt{related:reprintof} 有定义,
那么将用于确定条目 \texttt{key2} 包含的信息的格式,否则将使用宏 \texttt{related:default}。
\item %If the \texttt{related:reprintof} format is defined, it is used to format both the localization string and data. If this format is not defined, then the \texttt{related} format is used instead.
如果 \texttt{related:reprintof} 格式有定义,
那么将用于确定本地化字符串和数据的格式;
如果该格式没有定义,将使用 \texttt{related} 格式。
\end{itemize}
%
%It is also supported to have cascading and/or circular relations:
也支持串联或者循环关系:

\begin{lstlisting}[style=bibtex]{}
@Book{key1,
  ...
  related     = {key2},
  relatedtype = {reprintof},
  ...
}

@Book{key2,
  ...
  related     = {key3},
  relatedtype = {translationof},
  ...
}

@Book{key3,
  ...
  related     = {key2},
  relatedtype = {translatedas},
  ...
}
\end{lstlisting}
%
%Multiple relations to the same entry are also possible:
也可以实现同一条目的多重关系:
\begin{lstlisting}[style=bibtex]{}
@MVBook{key1,
  ...
  related     = {key2,key3},
  relatedtype = {multivolume},
  ...
}

@Book{key2,
  ...
}

@Book{key3,
  ...
}
\end{lstlisting}
%
%Note the the order of the keys in lists of multiple related entries is important. The data from multiple related entries is printed in the order of the keys listed in this field. See \secref{aut:ctm:rel} for a more details on the mechanisms behind this feature. You can turn this feature off using the package option \opt{related} from \secref{use:opt:pre:gen}.
请注意,多重关联条目列表中的顺序是很重要的。
多重关联条目的数据将按照该域中所列的顺序打印。
关于该特征背后的机制请参考 \secref{aut:ctm:rel} 节。
可以通过 \secref{use:opt:pre:gen} 中的宏包选项 \opt{related} 来关闭该特征。

%You can use the \bibfield{relatedoptions} to set options on the related entry data clone. This is useful if you need to override the \opt{dataonly} option which is set by default on all related entry clones. For example, if you will expose some of the names in the related clone in your document, you may want to have them disambiguated from names in other entries but normally this won't happen as related clones have the per"=entry \opt{dataonly} option set and this in turn sets \kvopt{uniquename}{false} and \kvopt{uniquelist}{false}. In such a case, you can set \bibfield{relatedoptions} to just \opt{skiplab, skipbib, skipbiblist}.

可以使用 \bibfield{relatedoptions} 域来设置关联条目数据克隆的选项。
如果你需要覆盖默认设置的关于所有关联条目克隆的 \opt{dataonly} 选项,那么该域是很有用的。
例如,如果你想在文档中展示一些相关联克隆体的名称,同时想要它们不与其它条目的名称相混淆,
但是正常情况下这不会发生的,因为相关联克隆体由基于每一条目的 \opt{dataonly} 选项设置,
这反过来又设置了 \kvopt{uniquename}{false} 和 \kvopt{uniquelist}{false}。
此时,你只需设置 \bibfield{relatedoptions} 为 \opt{skiplab, skipbib}。

\subsection{排序选项}
\label{use:srt}

%This package supports fully customisable sorting templates for the bibliography. The default global sorting template is selected with the \opt{sorting} package option from \secref{use:opt:pre:gen}. Apart from the regular data fields there are also some special fields which may be used to optimize the sorting of the bibliography. \Apxref{apx:srt:a1, apx:srt:a2} give an outline of the default alphabetic sorting templates supported by \biblatex. Chronological sorting templates are listed in \apxref{apx:srt:chr}. A few explanations concerning the default templates are in order.

本宏包支持多种文献排序格式。
排序格式由 \secref{use:opt:pre:gen} 节中的 \opt{sorting} 宏包选项确定。
除了常规的数据域之外,还有一些特殊域也可用于优化文献排序。
\Apxref{apx:srt:a1, apx:srt:a2} 大致概述了 \biblatex 支持的字母顺序排序格式。
而年代顺序排序格式则列在 \apxref{apx:srt:chr} 中。
以下依次是这些格式的一些解释。

%The first item considered in the sorting process is always the \bibfield{presort} field of the entry. If this field is undefined, \biblatex will use the default value <\texttt{mm}> as a presort string. The next item considered is the \bibfield{sortkey} field. If this field is defined, it serves as the master sort key. Apart from the \bibfield{presort} field, no further data is considered in this case. If the \bibfield{sortkey} field is undefined, sorting continues with the name. The package will try using the \bibfield{sortname}, \bibfield{author}, \bibfield{editor}, and \bibfield{translator} fields, in this order. Which fields are considered also depends on the setting of the \opt{use$<$name$>$} options. If all such options are disabled, the \bibfield{sortname} field is ignored as well. Note that all name fields are responsive to \opt{maxnames} and \opt{minnames}. If no name field is available, either because all of them are undefined or because all \opt{use$<$name$>$} options are disabled, \biblatex will fall back to the \bibfield{sorttitle} and \bibfield{title} fields as a last resort. The remaining items are, in various order: the \bibfield{sortyear} field, if defined, or the first four digits of the \bibfield{year} field otherwise; the \bibfield{sorttitle} field, if defined, or the \bibfield{title} field otherwise; the \bibfield{volume} field. Note that the sorting schemes shown in \apxref{apx:srt:a2} include an additional item: \bibfield{labelalpha} is the label used by <alphabetic> bibliography styles. Strictly speaking, the string used for sorting is \bibfield{labelalpha}~+ \bibfield{extraalpha}. The sorting schemes in \apxref{apx:srt:a2} are intended to be used in conjunction with alphabetic styles only.

在排序过程中首先要考虑的事项总是条目的 \bibfield{presort} 域。
如果该域没有定义,\biblatex 会使用缺省值“\texttt{mm}”作为预排序字符串。
其次考虑的是 \bibfield{sortkey} 域。
如果该域有定义,它将作为主要的排序关键字。
此时除了 \bibfield{presort} 域,将不考虑其它信息。
如果 \bibfield{sortkey} 域没有定义,排序将使用姓名信息。
本宏包将依次尝试使用 \bibfield{sortname}、\bibfield{author}、\bibfield{editor} 和 \bibfield{translator} 等域。
考虑哪些域也取决于 \opt{useauthor}、\opt{useeditor} 和 \opt{usetranslator} 选项的设置。
如果这三个选项都没有启用,那么 \bibfield{sortname} 也将被忽略。
请注意,所有的名称域都与 \opt{maxnames} 和 \opt{minnames} 有关。
如果没有名称域是合适的,或者由于它们没有定义、或者由于 \opt{use\prm{name}} 域都未启用,
那么 \biblatex 将采用  \bibfield{sorttitle} 和 \bibfield{title} 作为最后的备选。
余下考虑的诸项依次是:\bibfield{sortyear} 域(如果给出的话),否则考虑 \bibfield{year} 域的前四个数字;
\bibfield{sorttitle} 域(如果给出的话),否则考虑  \bibfield{title} 域;
\bibfield{volume} 域。
请注意,\apxref{apx:srt:a2} 展示的排序格式包括了额外一项:
\bibfield{labelalpha} 域是“alphabetic”文献样式所使用的标签。
严格地讲,用于排序的字符串是 \bibfield{labelalpha}~+ \bibfield{extraalpha}。
\apxref{apx:srt:a2} 中的排序格式只可以与字母顺序样式联合使用。

%The chronological sorting templates presented in \apxref{apx:srt:chr} also make use of the \bibfield{presort} and \bibfield{sortkey} fields, if defined. The next item considered is the \bibfield{sortyear} or the \bibfield{year} field, depending on availability. The \opt{ynt} scheme extracts the first four Arabic figures from the field. If both fields are undefined, the string \texttt{9999} is used as a fallback value. This means that all entries without a year will be moved to the end of the list. The \opt{ydnt} scheme is similar in concept but sorts the year in descending order. As with the \opt{ynt} scheme, the string \texttt{9999} is used as a fallback value. The remaining items are similar to the alphabetic sorting schemes discussed above. Note that the \opt{ydnt} sorting scheme will only sort the date in descending order. All other items are sorted in ascending order as usual.

\apxref{apx:srt:chr} 展示的年代排序格式同样使用域 \bibfield{presort} 和 \bibfield{sortkey} (如果有定义的话)。
其次考虑的是 \bibfield{sortyear} 或者 \bibfield{year} 域,这当然取决于是否可用。
\opt{ynt} 格式将从该域中提取前四个数字。
如果这两个域都没有定义,那么将使用后备值 \texttt{9999}。
这意味着没有年份的条目都会移动到列表末尾。
\opt{ydnt} 格式从概念上也是类似的,不过是用降序排列年份。
与 \opt{ynt} 格式一样,后备值是 \texttt{9999}。
余下考虑的项与上面讨论的字母排序格式类似。
请注意,\opt{ydnt} 排序格式只对日期按照降序排列。
其它项仍和平常一样按照升序排列。

%Using special fields such as \bibfield{sortkey}, \bibfield{sortname}, or \bibfield{sorttitle} is usually not required. The \biblatex package is quite capable of working out the desired sorting order by using the data found in the regular fields of an entry. You will only need them if you want to manually modify the sorting order of the bibliography or if any data required for sorting is missing. Please refer to the field descriptions in \secref{bib:fld:spc} for details on possible uses of the special fields.

通常来说不需要使用 \bibfield{sortkey}、\bibfield{sortname} 或 \bibfield{sorttitle} 等特殊域。
\biblatex 宏包通过使用条目常规域的数据就很容易得到所需的排列顺序。
只有当你想手动修改文献排序或者所需的数据缺失时,你才需要使用这些特殊域。
关于特殊域的可能用法请参考 \secref{bib:fld:spc} 节中的描述。


\subsection{数据注解}%\subsection{Data Annotations}
\label{use:annote}
%Ideally, there should be no formatting information in a bibliography data file, however, sometimes such questionable practice seems to the only way in which the desired results can be achieved. Data annotations are a way of addressing this by allowing users to attach semantic information (rather than typographical markup) to information in a bibliography data source so that the information can be used at markup time by a style. For example, if you wanted to highlight certain names in a work depending on whether they were a student author (indicated by a superscript asterisk in the references) or a corresponding author (indicated by bold face), then you might be tempted to try:
理想状态下,文献数据文件中不应当有格式信息。
然而,有时只有通过这种有争议的做法才能实现想要的结果。
数据注解(data annotations) 就是一种解决该问题的方法。
通过允许用户在文献数据源中添加某种语义信息(而不是排版标记),
使得文献样式可以在标记时使用该信息。
例如,如果想要按照如下规则高亮某些作品中的姓名:学生作者在文献中用上标星号表示,而通讯作者用粗体表示;
那么,可以尝试如下方法:

\begin{lstlisting}[style=bibtex]{}
@MISC{Article1,
  AUTHOR = {Last1\textsuperscript{*}, First1 and \textbf{Last2}, \textbf{First2} and Last3, First3}
}
\end{lstlisting}
%
%There are several problems with this. Firstly, it will break \bibtex's fragile name parsing routines and probably won't compile at all. Secondly, it is not only mixing up data with markup, it does so in a hard-coded way: this data can't easily be shared and used with other styles. While it is possible to achieve this formatting using \biblatex internals in a style or document, this is a complex and unreliable method which many users will not wish to use.
这一做法有一些问题。
首先,它会打断 \BibTeX 脆弱的姓名解析程序指令,可能根本不能编译。
其次,数据与标记的混合是硬编码的:其它样式不易共享和使用该数据。
当然,在样式或者文件中使用 \biblatex 内部指令可能实现该格式,
但是这一做法比较复杂而且不可靠,很多用户不愿意使用。

%In order to address these issues, \biblatex has a general data annotation facility which allows you to attach a comma"=separated list of annotations to data fields, items within data field lists (like names) and even parts of specific items such as parts of names (given name, family name etc.). There are macros provided to check for annotations which can be used in formatting directives.
为了处理这些问题,\biblatex 提供了一般性的数据注解功能,
使得可以向数据域、数据域列表中的项目(例如姓名),以及某些项目的一部分(例如姓、名等姓名部分)中附加逗号分隔列表作为注解。
此外还提供了一些宏来检查可以用于格式指令的注解。

%There are three «scopes» for data annotations, in order of increasing specificity:
数据注解有三种“尺度”,按照特性增加的顺序依次为:
\begin{itemize}
\item \opt{field}---%applied to top-level fields in a data source entry
用于数据源条目中的顶层域
\item \opt{item}---%applied to items within a list field in a data source entry
用于数据源条目中列表域中的项目
\item \opt{part}---%applied to parts within items within a list field in a data source entry
用于数据源条目中列表域中项目的一部分
\end{itemize}
%
%Data annotations are supported for \bibtex and \biblatexml data sources.
\BibTeX 和 \biblatexml 数据源都支持数据注解。

%Attaching annotations to data is easy in \biblatexml data sources as they are specified by simple XML attributes. Continuing with the example above, we would have:
在 \biblatexml 数据源中添加数据注解是很容易的,
因为可以通过简单的XML属性来指定。
继续上面的例子,我们有:

\begin{lstlisting}[language=xml]
<bltx:entries xmlns:bltx="http://biblatex-biber.sourceforge.net/biblatexml">
  <bltx:entry id="test" entrytype="misc">
    <bltx:names type="author">
      <bltx:name>
        <bltx:namepart type="given" initial="F">First1</bltx:namepart>
        <bltx:namepart type="family" initial="L" annotation="student">Last1</bltx:namepart>
      </bltx:name>
      <bltx:name annotation="corresponding">
        <bltx:namepart type="given" initial="F">First2</bltx:namepart>
        <bltx:namepart type="family" initial="L">Last2</bltx:namepart>
      </bltx:name>
      <bltx:name>
        <bltx:namepart type="given" initial="F">First3</bltx:namepart>
        <bltx:namepart type="family" initial="L">Last3</bltx:namepart>
      </bltx:name>
    </bltx:names>
  </bltx:entry>
</bltx:entries>
\end{lstlisting}
%
%Here the annotations are attached in an obvious way to the data items. With \bibtex data sources, the format for annotations is not quite as intuitive:
这里,向数据项目中添加注解的方式是很显然的。
而在 \BibTeX 数据源中,注解的格式就没有那么直观了:

\begin{lstlisting}[style=bibtex]{}
@MISC{ann1,
  AUTHOR    = {Last1, First1 and Last2, First2 and Last3, First3},
  AUTHOR+an = {1:family=student;2=corresponding},
}
\end{lstlisting}
%
%Here the field name suffix \texttt{+an} is a user-definable\footnote{See \biber's \opt{--annotation-marker} option.} suffix which marks a data field as an annotation of the unsuffixed field. The format of annotation fields in \bibtex data sources is is as follows:
这里域姓名后缀 \texttt{+an} 可以由用户定义\footnote{
	见 \biber 的 \opt{--annotation-marker} 选项。},
用于标记某个数据域为去掉后缀的域的注解。
\BibTeX 注解域的格式如下:

\begin{lstlisting}
<annotationspecs> ::= <annotationspec> [ ";" <annotationspec> ]
<annotationspec>  ::= [ <itemcount> [ ":" <part> ] ] "=" <annotations>
<annotations>     ::= <annotation> [ "," <annotation> ]
<annotation>      ::= (string)
\end{lstlisting}
%
%That is, one or more specifications separated by semi-colons. Each specification is an equals sign followed by a comma"=separated list of annotation keywords. To annotation a specific item in a list, put the number of the list item before the equals sign (lists start at 1). If you need to annotate a specific part of the list item, give its name after the list item number, preceded by a colon. Name part names are defined in the data model, see \secref{aut:bbx:drv}. Some examples:
也就是说,多个特性之间由分号分隔。
每一特性是一个等号后跟一个逗号分隔的注解关键字列表。
为了为列表中某一项作注解,需要将该列表项的编号放在等号前面(列表从1开始编号)。
如果需要为列表项的某一部分做注解,需要将该部分名放在编号之后,并且前接一个冒号。
姓名部分的名称在数据模型中有定义,见 \secref{aut:bbx:drv} 节。
以下是一些例子:

\begin{lstlisting}[style=bibtex]{}
AUTHOR      = {Last1, First1 and Last2, First2 and Last3, First3},
AUTHOR+an   = {3:given=annotation1, annotation2},
TITLE       = {A title},
TITLE+an    = {=a title annotation, another title annotation},
LANGUAGE    = {english and french},
LANGUAGE+an = {1=annotation3; 2=annotation4}
}
\end{lstlisting}
%
%To access the annotation information when formatting bibliography data, three macros are provided, corresponding to the three annotation scopes:
为了在文献格式中获取注解信息,
这里提供了三个宏,分别对应与相应的注解尺度:

\begin{ltxsyntax}

\cmditem{iffieldannotation}{annotation}{true}{false}

%Executes \prm{true} if the current data field has an annotation \prm{annotation} and false otherwise.

如果当前数据域有注解,那么执行 \prm{true},否则为 false。

\cmditem{ifitemannotation}{annotation}{true}{false}

%Executes \prm{true} if the current item in the current data field has an annotation \prm{annotation} and false otherwise.

如果当前数据域的当前项目有注解,那么执行 \prm{true},否则为 false。

\cmditem{ifpartannotation}{part}{annotation}{true}{false}

%Executes \prm{true} if the part named \prm{part} in current item in the current data field has an annotation \prm{annotation} and false otherwise.

如果当前数据域中当前项目中名为 \prm{part} 的部分有注解,那么执行 \prm{true},否则为 false。

\end{ltxsyntax}
%
%These macros are available in the same places as \cmd{currentfield}, \cmd{currentlist} and \cmd{currentname} (see \secref{aut:bib:fmt}), that is, inside formatting directives. They automatically determine the name of the current data field being processed and also the current \opt{listcount} value which determines the current item in list fields. Parts such as name parts need to be named explicitly. As an example of how to use the annotation information to solve the problem originally presented in this section, this could be used in the name formatting directives to put an asterisk after all family names annotated as «student»:
这些宏的使用场合与 \cmd{currentfield}, \cmd{currentlist} 和 \cmd{currentname} 等命令相同(见 \secref{aut:bib:fmt} 节),
即,在格式指令内部。
它们自动确定当前被处理的数据域的名称,以及能够确定列表域中当前项目的 \opt{listcount} 值。
姓名部分等项目部分需要显式地指明。
下面的例子可以用于姓名格式指令,说明如何使用注解信息来解决本节之前提出的问题:
在所有注解为“student”的姓之后加上星号:

\begin{lstlisting}[style=latex]{}
  \ifpartannotation{family}{student}
    {\textsuperscript{*}}
    {}%
\end{lstlisting}
%
%To put the given and family names of name list items annotated as «corresponding» in boldface:
将标记为 “corresponding”的姓名列表项中的姓和名加粗:

\begin{lstlisting}[style=latex]{}
\renewcommand*{\mkbibnamegiven}[1]{%
  \ifitemannotation{corresponding}
    {\textbf{#1}}
    {#1}}

\renewcommand*{\mkbibnamefamily}[1]{%
  \ifitemannotation{corresponding}
    {\textbf{#1}}
    {#1}}
\end{lstlisting}

\subsection{参考文献命令}%\subsection{Bibliography Commands}
\label{use:bib}

\subsubsection{数据源}%\subsubsection{Resources}
\label{use:bib:res}

\begin{ltxsyntax}

\cmditem{addbibresource}[options]{resource}

%Adds a \prm{resource}, such as a \file{.bib} file, to the default resource list. This command is only available in the preamble. It replaces the \cmd{bibliography} legacy command. Note that files must be specified with their full name, including the extension. Do not omit the \file{.bib} extension from the filename. Also note that the \prm{resource} is a single resource. Invoke \cmd{addbibresource} multiple times to add more resources, for example:

将 \prm{resource} 添加到默认资源列表中,例如 \file{.bib} 文件。
该命令只能在导言区中使用。
它取代了过时的 \cmd{bibliography} 命令。
请注意,文件名包括扩展名,所以不要省略文件名中的 \file{.bib} 扩展名。
另外要注意的是,\prm{resource} 只能是一个单独的数据源。
添加更多的资源需要多次调用 \cmd{addbibresource} 命令,例如:
\begin{ltxexample}
\addbibresource{bibfile1.bib}
\addbibresource{bibfile2.bib}
\addbibresource[location=remote]{http://www.citeulike.org/bibtex/group/9517}
\addbibresource[location=remote,label=lan]{ftp://192.168.1.57/~user/file.bib}
\end{ltxexample}
%
%Since the \prm{resource} string is read in a verbatim-like mode, it may contain arbitrary characters. The only restriction is that any curly braces must be balanced. The following \prm{options} are available:
由于 \prm{resource} 字符串的读取类似于抄录模式,因此它可以包含任意的字符。
唯一的限制是其中任何的花括号必须左右匹配。
可用的 \prm{options} 如下:

\begin{optionlist*}

\valitem{label}{identifier}

%Assigns a label to a resource. The \prm{identifier} may be used in place of the full resource name in the optional argument of \env{refsection} (see \secref{use:bib:sec}).

给该数据源分配一个标签。
\prm{identifier} 可以用于在 \env{refsection} 环境的可选参数中以取代该数据源的全名
(见 \secref{use:bib:sec} 节)。

\valitem[local]{location}{location}

%The location of the resource. The \prm{location} may be either \texttt{local} for local resources or \texttt{remote} for \acr{URL}s. Remote resources require \biber. The protocols \acr{HTTP} and \acr{FTP} are supported. The remote \acr{URL} must be a fully qualified path to a \file{bib} file or a \acr{URL} which returns a \file{bib} file.

数据源的地址。
\prm{location} 可以是 \texttt{local} 或者 \texttt{remote},
分别对应本地数据和在线 \acr{URL} 数据。
远程资源需要 \biber{} 程序。
支持 \acr{HTTP} 和 \acr{FTP} 协议。
远程的 \acr{URL} 必须是 \file{bib} 文件的合法路径全称或者是返回 \file{bib} 文件的 \acr{URL}。

\valitem[file]{type}{type}

%The type of resource. Currently, the only supported type is \texttt{file}.

资源的类型。目前唯一支持的类型是 \texttt{file}。

\valitem[bibtex]{datatype}{datatype}

%The data type (format) of the resource. The following formats are currently supported:

资源的数据类型(格式)。目前支持以下格式:

\begin{valuelist}[zoterordfxml]

\item[bibtex] %\bibtex format.
\BibTeX 格式。

\item[biblatexml] %Experimental XML format for \biblatex. See \secref{apx:biblatexml}.
针对 \biblatex 的实验性 XML 格式。见 \secref{apx:biblatexml}。

\end{valuelist}

\end{optionlist*}


\cmditem{addglobalbib}[options]{resource}

%This command differs from \cmd{addbibresource} in that the \prm{resource} is added to the global resource list. The difference between default resources and global resources is only relevant if there are reference sections in the document and the optional argument of \env{refsection} (\secref{use:bib:sec}) is used to specify alternative resources which replace the default resource list. Any global resources are added to all reference sections.

该命令不同于 \cmd{addbibresource} 之处在于将 \prm{resource} 添加到全局数据源列表中。
不过,只有当文档中有参考文献章节并且使用 \env{refsection} 环境的可选参数(见 \secref{use:bib:sec} 节)
作为确定代替默认资源列表的备选资源时,考虑默认数据源和全局数据源的不同才是有意义的。
任何全局资源将被添加到所有的参考文件章节中。

\cmditem{addsectionbib}[options]{resource}

%This command differs from \cmd{addbibresource} in that the resource \prm{options} are registered but the \prm{resource} not added to any resource list. This is only required for resources which 1) are given exclusively in the optional argument of \env{refsection} (\secref{use:bib:sec}) and 2) require options different from the default settings. In this case, \cmd{addsectionbib} is employed to qualify the \prm{resource} prior to using it by setting the appropriate \prm{options} in the preamble. The \opt{label} option may be useful to assign a short name to the resource.

该命令与 \cmd{addbibresource} 的不同之处在于,会记录数据源的 \prm{options} 但是 \prm{resource} 没有添加到任何数据源列表中。
有该需求的场合是 (1) 该数据源仅仅用于 \env{refsection} 环境的可选参数中(\secref{use:bib:sec} 节);
(2) 该数据源需要不同于默认设置的选项。
此时,\cmd{addsectionbib} 会在导言区中设置合适的 \prm{options},从而在其使用前声明 \prm{resource}。
\opt{label} 选项可以用于分配给该资源一个简短的名称。

\cmditem{bibliography}{bibfile, \dots}|\DeprecatedMark|

%The legacy command for adding bibliographic resources, supported for backwards compatibility. Like \cmd{addbibresource}, this command is only available in the preamble and adds resources to the default resource list. Its argument is a comma"=separated list of \file{bib} files. The \file{.bib} extension may be omitted from the filename. Invoking this command multiple times to add more files is permissible. This command is deprecated. Please consider using \cmd{addbibresource} instead.

添加文献资源的过时命令,仅处于向后兼容性而支持。
类似 \cmd{addbibresource},该命令只能在导言区中使用,并将资源添加到默认资源列表中。
它的选项是逗号分隔的 \file{bib} 文件列表。
文件名中的 \file{.bib} 扩展名可以省略。
也可以通过多次调用该命令来添加更多文件。
该命令已过时,请考虑使用 \cmd{addbibresource} 来取代。

\end{ltxsyntax}

\subsubsection{参考文献表}%\subsubsection{The Bibliography}
\label{use:bib:bib}

\begin{ltxsyntax}

\cmditem{printbibliography}[key=value, \dots]

%This command prints the bibliography. It takes one optional argument, which is a list of options given in \keyval notation. The following options are available:

该命令可以打印出参考文献。
它的可选参数是以 \keyval 形式给出的一列选项。
可用的选项如下:

\end{ltxsyntax}

\begin{optionlist*}

\valitem[bibliography/shorthands]{env}{name}

%The <high-level> layout of the bibliography and the list of shorthands is controlled by environments defined with \cmd{defbibenvironment}. This option selects an environment. The \prm{name} corresponds to the identifier used when defining the environment with \cmd{defbibenvironment}. By default, the \cmd{printbibliography} command uses the identifier \texttt{bibliography}; \cmd{printbiblist} uses \texttt{shorthands}. See also \secref{use:bib:biblist,use:bib:hdg}.

可以用 \cmd{defbibenvironment} 定义的环境来控制参考文献和shorthands列表的高层次布局。
该选项选择了一个环境。
\prm{name} 对应于用 \cmd{defbibenvironment} 定义环境时的标识符。
缺省状态下,\cmd{printbibliography} 命令使用标识符 \texttt{bibliography};
而 \cmd{printshorthands} 使用 \texttt{shorthands}。
另见 \secref{use:bib:los,use:bib:hdg} 节。


\valitem[bibliography/shorthands]{heading}{name}

%The bibliography and the list of shorthands typically have a chapter or section heading. This option selects the heading \prm{name}, as defined with \cmd{defbibheading}. By default, the \cmd{printbibliography} command uses the heading \texttt{bibliography}; \cmd{printbiblist} uses \texttt{shorthands}. See also \secref{use:bib:biblist,use:bib:hdg}.

参考文献和shorthand列表通常有一个章标题或者节标题。
该选项选择由 \cmd{defbibheading} 定义的标题名 \prm{name}。
缺省状态下,\cmd{printbibliography} 命令使用标题名 \texttt{bibliography};
而 \cmd{printshorthands} 使用 \texttt{shorthands}。
另见 \secref{use:bib:los,use:bib:hdg} 节。

\valitem{title}{text}

%This option overrides the default title provided by the heading selected with the \opt{heading} option, if supported by the heading definition. See \secref{use:bib:hdg} for details.

如果标题定义支持的话,该选项覆盖由 \opt{heading} 选项提供的缺省标题名。
详见 \secref{use:bib:hdg} 节。

\valitem{prenote}{name}

%The prenote is an arbitrary piece of text to be printed after the heading but before the list of references. This option selects the prenote \prm{name}, as defined with \cmd{defbibnote}. By default, no prenote is printed. The note is printed in the standard text font. It is not affected by \cmd{bibsetup} and \cmd{bibfont} but it may contain its own font declarations. See \secref{use:bib:nts} for details.

前注是打印在标题之后、文献列表之前的任意文本片段。
该选项选择由 \cmd{defbibnote} 所定义的前注 \prm{name}。
缺省状态下不打印任何前注。
该注记使用标准正文字体。
它不受 \cmd{bibsetup} 和 \cmd{bibfont} 的影响但可以包含自己的字体声明。
详见 \secref{use:bib:nts} 节。

\valitem{postnote}{name}

%The postnote is an arbitrary piece of text to be printed after the list of references. This option selects the postnote \prm{name}, as defined with \cmd{defbibnote}. By default, no postnote is printed. The note is printed in the standard text font. It is not affected by \cmd{bibsetup} and \cmd{bibfont} but it may contain its own font declarations. See \secref{use:bib:nts} for details.

后注是打印在参考文献列表之后的任意文本片段。
该选项选择由 \cmd{defbibnote} 所定义的后注 \prm{name}。
缺省状态下不打印任何后注。
该注记使用标准正文字体。
它不受 \cmd{bibsetup} 和 \cmd{bibfont} 的影响但可以包含自己的字体声明。
详见 \secref{use:bib:nts}。

\intitem[current section]{section}

%Print only entries cited in reference section \prm{integer}. The reference sections are numbered starting at~1. All citations given outside a \env{refsection} environment are assigned to section~0. See \secref{use:bib:sec} for details and \secref{use:use:mlt} for usage examples.

只打印在第 \prm{integer} 文节中引用的条目。
该参考文献节从~1 开始编号。
所有在 \env{refsection} 环境外给出的引用标记为第零节。
详见 \secref{use:bib:sec} 和 \secref{use:use:mlt} 节中的使用例子。

\intitem[0]{segment}

%Print only entries cited in reference segment \prm{integer}. The reference segments are numbered starting at~1. All citations given outside a \env{refsegment} environment are assigned to segment~0. See \secref{use:bib:seg} for details and \secref{use:use:mlt} for usage examples. Remember that segments within a section are numbered local to the section so the segment you request will be the nth segment in the requested (or currently active enclosing) section.

只打印在第 \prm{integer} 文献段中引用的条目。
参考文献段从~1 开始编号。
所有在 \env{refsection} 环境外给出的引用标记为第零段。
详见 \secref{use:bib:sec} 和 \secref{use:use:mlt} 节中的使用例子。
请注意,一节内部的片段是在该节中局部编号的,故而需要的片段是被查询(或者当前激活的)节的第 n 段。

\valitem{type}{entrytype}

%Print only entries whose entry type is \prm{entrytype}.

只打印类型为 \prm{entrytype} 的条目。

\valitem{nottype}{entrytype}

%Print only entries whose entry type is not \prm{entrytype}. This option may be used multiple times.

只打印类型不为 \prm{entrytype} 的条目。该选项可以使用多次。

\valitem{subtype}{subtype}

%Print only entries whose \bibfield{entrysubtype} is defined and \prm{subtype}.

只打印域 \bibfield{entrysubtype} 定义为 \prm{subtype} 的条目。

\valitem{notsubtype}{subtype}

%Print only entries whose \bibfield{entrysubtype} is undefined or not \prm{subtype}. This option may be used multiple times.

只打印域 \bibfield{entrysubtype} 没有定义或者不为 \prm{subtype} 的条目。该选项可以使用多次。

\valitem{keyword}{keyword}

%Print only entries whose \bibfield{keywords} field includes \prm{keyword}. This option may be used multiple times.

只打印域 \bibfield{keywords} 包括 \prm{keyword} 的条目。该选项可以使用多次。

\valitem{notkeyword}{keyword}

%Print only entries whose \bibfield{keywords} field does not include \prm{keyword}. This option may be used multiple times.

只打印域 \bibfield{keywords} 不包括 \prm{keyword} 的条目。该选项可以使用多次。

\valitem{category}{category}

%Print only entries assigned to category \prm{category}. This option may be used multiple times.

只打印属于 \prm{category} 类型的条目。该选项可以使用多次。

\valitem{notcategory}{category}

%Print only entries not assigned to category \prm{category}. This option may be used multiple times.

只打印不属于 \prm{category} 类型的条目。该选项可以使用多次。

\valitem{filter}{name}

%Filter the entries with filter \prm{name}, as defined with \cmd{defbibfilter}. See \secref{use:bib:flt} for details.

使用由 \cmd{defbibfilter} 定义的 filter \prm{name} 来过滤条目。详见 \secref{use:bib:flt} 节。

\valitem{check}{name}

%Filter the entries with check \prm{name}, as defined with \cmd{defbibcheck}. See \secref{use:bib:flt} for details.

使用由 \cmd{defbibcheck} 定义的 check \prm{name} 来过滤条目。详见  \secref{use:bib:flt} 节。

\valitem{resetnumbers}{true,false,number}

%This option applies to numerical citation\slash bibliography styles only and requires that the \opt{defernumbers} option from \secref{use:opt:pre:gen} be enabled globally. If enabled, it will reset the numerical labels assigned to the entries in the respective bibliography, \ie the numbering will restart at~1. You can also pass a number to this option, for example: \texttt{resetnumbers=10} to reset numbering to the specified number to aid numbering continuity across documents. Use this option with care as \biblatex can not guarantee unique labels globally if they are reset manually.

该选项只用于数值引用/参考文献样式,
并且要求 \secref{use:opt:pre:gen} 中的 \opt{defernumbers} 选项全局启用。
如果启用的话,它将重新设置分配给相应文献中条目的数值标签,即,编号会重新从 1 开始。
此外还可以传递数值给该选项以重置编号为给定的数值,例如 \texttt{resetnumbers=10},
这样可以改进整个文档中编号的连续性。
请小心使用本选项,因为在手动重新设置下,\biblatex 不能保证标签是全局唯一的。

\boolitem{omitnumbers}

%This option applies to numerical citation\slash bibliography styles only and requires that the \opt{defernumbers} option from \secref{use:opt:pre:gen} be enabled globally. If enabled, \biblatex will not assign a numerical label to the entries in the respective bibliography. This is useful when mixing a numerical subbibliography with one or more subbibliographies using a different scheme (\eg author-title or author-year).

该选项只用于数值引用/参考文献样式,
并且要求 \secref{use:opt:pre:gen} 中的 \opt{defernumbers} 选项全局启用。
如果启用的话,\biblatex 不会为相应文献中的条目分配数值标签。
当数值型子文献和其它不同格式(例如作者-标题或者作者-年份)的子文献相混合时,这是很有用的。

\end{optionlist*}

\begin{ltxsyntax}

\cmditem{bibbysection}[key=value, \dots]

%This command automatically loops over all reference sections. This is equivalent to giving one \cmd{printbibliography} command for every section but has the additional benefit of automatically skipping sections without references. Note that \cmd{bibbysection} starts looking for references in section \texttt{1}. It will ignore references given outside of \env{refsection} environments since they are assigned to section~0. See \secref{use:use:mlt} for usage examples. The options are a subset of those supported by \cmd{printbibliography}. Valid options are \opt{env}, \opt{heading}, \opt{prenote}, \opt{postnote}. The current bibliography context sorting scheme is used for all sections (see \secref{use:bib:context}).

该命令会自动遍历所有的参考文献节。
这等价于为每一节给出一个 \cmd{printbibliography} 命令,
不过会有额外好处:自动跳过不含参考文献的节。
请注意,\cmd{bibbysection} 一开始寻找第 \texttt{1} 节中的文献。
它会忽略 \env{refsection} 外给出的文献,因为它们被分配给第零节。
使用例子请参考 \secref{use:use:mlt} 节。
选项可以是由 \cmd{printbibliography} 支持的一个子集。
有效选项是 \opt{env}、\opt{heading}、\opt{prenote} 和 \opt{postnote}。
当前文献内容排序格式会应用在所有的节中(见 \secref{use:bib:context} 节)。

\cmditem{bibbysegment}[key=value, \dots]

%This command automatically loops over all reference segments. This is equivalent to giving one \cmd{printbibliography} command for every segment in the current \env{refsection} but has the additional benefit of automatically skipping segments without references. Note that \cmd{bibbysegment} starts looking for references in segment \texttt{1}. It will ignore references given outside of \env{refsegment} environments since they are assigned to segment~0. See \secref{use:use:mlt} for usage examples. The options are a subset of those supported by \cmd{printbibliography}. Valid options are \opt{env}, \opt{heading}, \opt{prenote}, \opt{postnote}. The current bibliography context sorting scheme is used for all segments (see \secref{use:bib:context}).

该命令会自动遍历所有的参考文献段。
这等价于为当前 \env{refsection} 的每一段给出一个 \cmd{printbibliography} 命令,不过会有额外好处:自动跳过不含参考文献的片段。
请注意,\cmd{bibbysection} 一开始寻找第 \texttt{1} 段中的文献。
它会忽略 \env{refsection} 外给出的文献,因为它们被分配给第 0 段。
使用例子请参考 \secref{use:use:mlt}。
选项可以是由 \cmd{printbibliography} 支持的一个子集。
有效选项是 \opt{env}、\opt{heading}、\opt{prenote} 和 \opt{postnote}。
当前文献内容排序格式会用于所有的段中(见 \secref{use:bib:context} 节)。

\cmditem{bibbycategory}[key=value, \dots]

%This command loops over all bibliography categories. This is equivalent to giving one \cmd{printbibliography} command for every category but has the additional benefit of automatically skipping empty categories. The categories are processed in the order in which they were declared. See \secref{use:use:mlt} for usage examples. The options are a subset of those supported by \cmd{printbibliography}. Valid options are \opt{env}, \opt{prenote}, \opt{postnote}, \opt{section}. Note that \opt{heading} is not available with this command. The name of the current category is automatically used as the heading name. This is equivalent to passing \texttt{heading=\prm{category}} to \cmd{printbibliography} and implies that there must be a matching heading definition for every category. The current bibliography context sorting scheme is used for all categories (see \secref{use:bib:context}).

该命令遍历所有的文献类型。
这等价于为每一类型给出一个 \cmd{printbibliography} 命令,
不过会有额外好处:自动跳过空类型。
类型按照声明的顺序处理。
示例见 \secref{use:use:mlt} 节。
选项可以是由 \cmd{printbibliography} 支持的一个子集。
有效选项是 \opt{env}、\opt{heading}、\opt{prenote} 和 \opt{postnote}。
请注意,\opt{heading} 对于该命令是无效的。
当前类型的名字会自动作为标题名。
这等价于传递 \texttt{heading=\prm{category}} 给 \cmd{printbibliography},
并且意味着对于每一类型都必须有一个匹配的标题定义。
当前文献内容排序格式会用于所有的类型中(见 \secref{use:bib:context} 节)。

\cmditem{printbibheading}[key=value, \dots]

%This command prints a bibliography heading defined with \cmd{defbibheading}. It takes one optional argument, which is a list of options given in \keyval notation. The options are a small subset of those supported by \cmd{printbibliography}. Valid options are \opt{heading} and \opt{title}. By default, this command uses the heading \texttt{bibliography}. See \secref{use:bib:hdg} for details. Also see \secref{use:use:mlt,use:use:div} for usage examples.

该命令打印出由 \cmd{defbibheading} 定义的参考文献标题。
它有一个可选项,是用 \keyval 记号给出的选项列表。
选项是 \cmd{printbibliography} 支持的一个小子集。
有效选项是 \opt{heading} 和 \opt{title}。
缺省情况下,该命令使用标题 \texttt{bibliography}。
详见 \secref{use:bib:hdg} 节。实例也可见 \secref{use:use:mlt,use:use:div} 节。

\end{ltxsyntax}
%
%To print a bibliography with a different sorting scheme than the global sorting scheme, use the bibliography context switching commands from \secref{use:bib:context}.
如果想要在参考文献中使用非全局排序格式的另外一种排序格式,
使用 \secref{use:bib:context} 节提供的文献内容切换命令。

%\subsubsection{缩略表 The List of Shorthands}
%\label{use:bib:los}
%
%\BibTeXOnlyMark This section applies only to \bibtex. When using \biber, the list of shorthands is just a special case of a bibliography list. See \secref{use:bib:biblist}.\footnote{本节需要重点关注一下,之前没有搞清楚}
%
%If any entry includes a \bibfield{shorthand} field, \sty{biblatex} automatically builds a list of shorthands which may be printed in addition to the regular bibliography. The following command prints the list of shorthands.
%
%\begin{ltxsyntax}
%
%\cmditem{printshorthands}[key=value, \dots]
%
%This command prints the list of shorthands. It takes one optional argument, which is a list of options given in \keyval notation. Valid options are all options supported by \cmd{printbibliography} (\secref{use:bib:bib}) except \opt{prefixnumbers}, \opt{resetnumbers}, and \opt{omitnumbers}. If there are any \env{refsection} environments in the document, the list of shorthands will be local to these environments; see \secref{use:bib:sec} for details. By default, this command uses the heading \texttt{shorthands}. See \secref{use:bib:hdg} for details.
%
%The \opt{sorting} option differs from \cmd{printbibliography} in that if omitted, the default is to sort by shorthand.
%
%\end{ltxsyntax}




\subsubsection{参考文献列表} %\subsubsection{Bibliography Lists}
\label{use:bib:biblist}

%\biblatex can, in addition to printing normal bibliographies, also print arbitrary lists of information derived from the bibliography data such as a list of shorthand abbreviations for particular entries or a list of abbreviations of journal titles.

\biblatex 除了可以打印常规参考文献之外,还能根据文献数据打印任意文献信息列表,
例如,与特定条目或者期刊标题缩写有关的速记缩写列表。

%A bibliography list differs from a normal bibliography in that the same bibliography driver is used to print all entries rather than a specific driver being used for each entry depending on the entry type.

文献列表与常规参考文献不同的是,
使用同一文献驱动打印所有条目,
而不是根据条目类型使用特定于条目的驱动。

\begin{ltxsyntax}

\cmditem{printbiblist}[key=value, \dots]{$<$biblistname$>$}

%This command prints a bibliography list. It takes an optional argument, which is a list of options given in \keyval notation. Valid options are all options supported by \cmd{printbibliography} (\secref{use:bib:bib}) except \opt{resetnumbers} and \opt{omitnumbers}. If there are any \env{refsection} environments in the document, the bibliography list will be local to these environments; see \secref{use:bib:sec} for details. By default, this command uses the heading \texttt{biblist}. See \secref{use:bib:hdg} for details.

该命令用于打印文献列表。
其可选项是 \keyval 形式的一列选项。
除了 \opt{resetnumbers} 和 \opt{omitnumbers},
\cmd{printbibliography} 命令(见 \secref{use:bib:bib} 节)支持的其它选项在这里都是有效的。
如果文档中有任何 \env{refsection} 环境,
那么文献列表只针对于这些环境,详见 \secref{use:bib:sec} 节。
默认情况下该命令使用标题 \texttt{biblist},详见 \secref{use:bib:hdg} 节。

%The \prm{biblistname} is a mandatory argument which names the bibliography list. This name is used to identify:
必选项 \prm{biblistname} 是文献列表的标题,用于确定如下项目:
\begin{itemize}
\item %The default bibliography driver used to print the list entries
用于打印列表条目的默认文献驱动。
\item %A default filter declared with \cmd{DeclareBiblistFilter} (see \secref{aut:ctm:bibfilt}) used to filter the entries returned from \biber
使用 \cmd{DeclareBiblistFilter} 声明的默认filter(见 \secref{aut:ctm:bibfilt} 节),
用于过滤 \biber 返回的条目。
\item %A default check declared with \cmd{defbibcheck} (see \secref{use:bib:flt}) used to post-process the list entries
使用 \cmd{defbibcheck} 命令声明的默认check(见 \secref{use:bib:flt} 节),
用于后置处理列表条目。
\item %The default bib environment to use
默认使用的bib 环境。
\item %The default sorting template to use
默认使用的排序格式名称。
\end{itemize}

%In terms of sorting the list, the default is to sort use the sorting scheme named after the bibliography list (if it exists) and only then to fall back to the current context sorting scheme is this is not defined (see \secref{use:bib:context}).

在列表的排序方面,默认使用与该文献列表同名的排序格式(如果存在的话)。
只有当未定义时才会切换到备选的当前内容排序格式(见 \secref{use:bib:context} 节)。

%The most common bibliography list is a list of shorthand abbreviations for certain entries and so this has a convenience alias \cmd{printshorthands[\dots]} for backwards compatibility which is defined as:

最常用的文献列表是关于某些条目的速记列表,
出于向后兼容性专门有一个别名 \cmd{printshorthands[\dots]},定义如下:

\begin{lstlisting}[style=latex]{}
\printbiblist[...]{shorthand}
\end{lstlisting}

%\biblatex provides automatic support for data source fields in the default data model marked as <Label fields> (See \secref{bib:fld:dat}). Such fields automatically have defined for them:

\biblatex 自动支持默认数据模型中标记为“Label fields”的数据域(见 \secref{bib:fld:dat} 节)。
这些域已经自动为其定义了如下项目:

\begin{itemize}
\item %A default bib environment (See \secref{use:bib:hdg})
默认的 bib 环境(见 \secref{use:bib:hdg} 节)。
\item %A bibliography list filter (See \secref{aut:ctm:bibfilt})
文献列表filter(见 \secref{aut:ctm:bibfilt} 节)
\item %Some supporting formats and lengths (See \secref{aut:fmt:ilc} and \secref{aut:fmt:ich})
一些支持的格式和长度(见 \secref{aut:fmt:ilc} 和 \secref{aut:fmt:ich} 节)。
\end{itemize}
%
%Therefore only a minimal setup is required to print bibliography lists with such fields. For example, to print a list of journal title abbreviations, you can minimally put this in your preamble:
因此,打印带有这些域的文献列表只需要很少的设置。
例如,想要打印出期刊标题缩写列表,
只需要将如下一小段代码放在导言区中:

\begin{ltxexample}
\DeclareBibliographyDriver{shortjournal}{%
  \printfield{journaltitle}}
\end{ltxexample}
%
%Then you can put this in your document where you want to print the list:
然后在正文中想要打印列表的地方使用如下代码:

\begin{ltxexample}
\printbiblist[title={Journal Shorthands}]{shortjournal}
\end{ltxexample}
%
%Since \bibfield{shortjournal} is defined in the default data model as a <Label field>, this example:
由于默认数据模型将 \bibfield{shortjournal} 定义为“标签域”,
因此在这个例子中:
\begin{itemize}
\item %Uses the automatically created <shortjournal> bib environment
使用自动创建的“shortjournal”bib环境。
\item %Uses the automatically created <shortjournal> bibliography list filter to return only entries with a \bibfield{shortjournal} field in the \file{.bbl}
使用自动创建的“shortjournal”文献列表filter,
返回 \file{.bbl} 文件中只带有 \bibfield{shortjournal} 域的条目。
\item %Uses the defined <shortjournal> bibliography driver to print the entries
使用定义的“shortjournal”文献驱动来打印条目。
\item %Uses the default <biblist> heading but overrides the title with <Journal Shorthands>
使用默认的“biblist”标题,但是这里用“Journal Shorthands”来代替。
\item %Uses the current bibliography context sorting scheme if no scheme exists with the name \bibfield{shortjournal}
如果没有名为 \bibfield{shortjournal} 的格式,那么使用当前文献内容排序格式。
\end{itemize}
%
%Often, you will want to sort on the label field of the list and since a sorting scheme is automatically picked up if it is named after the list, in this case you could simply do:
很多情况下想要根据列表中标签域进行排序。
由于根据列表名可以自动获取排序格式,因此此时可以简单地使用如下代码:

\begin{ltxexample}
\DeclareSortingTemplate{shortjournal}{
  \sort{
        \field{shortjournal}
  }
}
\end{ltxexample}

%Naturally all defaults can be overridden by options to \cmd{printbiblist} and definitions of the environments, filters etc. and in this way arbitrary types of bibliography lists can be printed containing a variety of information from the bibliography data.

自然地,\cmd{printbiblist} 命令的选项以及环境、filters等的定义可以覆盖所有的默认设置。
因此通过这种方法可以从文献数据中打印任意类型的文献列表,并且包含各式信息。
\end{ltxsyntax}

%Bibliography lists are often used to print lists of various kinds of shorthands and this can result in duplicate entries if more than one bibliography entry has the same shorthand. For example, several journal articles in the same journal would result in duplicate entries in a list of journal shorthands. You can use the fact that such lists automatically pick up a \cmd{bibcheck} with the same name as the list to define a check to remove duplicates. If you are defining a list to print all of the journal shorthands using the \bibfield{shortjournal} field, you could define a \cmd{bibcheck} like this:

文献列表通常用于打印各类shorthand列表。
如果多个条目有相同的shorthand就会导致重复的条目。
例如,如果有几篇论文在同一期刊上,那么期刊缩写列表中就会出现重复条目。
不过,这样的列表会自动获取与列表同名的 \cmd{bibcheck},进而定义相应的check来删除重复项目。
如果使用 \bibfield{shortjournal} 域来定义打印所有期刊缩写的列表,
那么需要定义如下的 \cmd{bibcheck}:

\begin{ltxexample}
\defbibcheck{shortjournal}{%
   \iffieldundef{shortjournal}
     {\skipentry}
     {\iffieldundef{journaltitle}
       {\skipentry}
       {\ifcsdef{\strfield{shortjournal}=\strfield{journaltitle}}
         {\skipentry}
         {\savefieldcs{journaltitle}{\strfield{shortjournal}=\strfield{journaltitle}}}}}}
\end{ltxexample}

\subsubsection{参考文献分节}% \subsubsection{Bibliography Sections}
\label{use:bib:sec}

%The \env{refsection} environment is used in the document body to mark a reference section. This environment is useful if you want separate, independent bibliographies and bibliography lists in each chapter, section, or any other part of a document. Within a reference section, all cited works are assigned labels which are local to the environment. Technically, reference sections are completely independent from document divisions such as \cmd{chapter} and \cmd{section} even though they will most likely be used per chapter or section. See the \opt{refsection} package option in \secref{use:opt:pre:gen} for a way to automate this. Also see \secref{use:use:mlt} for usage examples.

在文档中,\env{refsection} 环境用于标记参考文献分节。
该环境主要用于在文档的每一章、节或其它部分中实现各自独立的参考文献和shorthand列表。
在一个文献分节内部,所有引用文献分配的标签都局部在该环境中。
技术上,尽管文献分节通常在每一章或每一节中使用,
但它们与 \cmd{chapter} 和 \cmd{section} 等文档划分是完全独立的。
关于自动实现这一功能请参考 \secref{use:opt:pre:gen} 中的 \opt{refsection} 宏包选项。
使用例子也可以参见 \secref{use:use:mlt}。

\begin{ltxsyntax}

\envitem{refsection}[resource, \dots]

%The optional argument is a comma"=separated list of resources specific to the reference section. If the argument is omitted, the reference section will use the default resource list, as specified with \cmd{addbibresource} in the preamble. If the argument is provided, it replaces the default resource list. Global resources specified with \cmd{addglobalbib} are always considered. \env{refsection} environments may not be nested, but you may use \env{refsegment} environments within a \env{refsection} to subdivide it into segments. Use the \opt{section} option of \cmd{printbibliography} to select a section when printing the bibliography, and the corresponding option of \cmd{printbiblist} when printing bibliography lists. Bibliography sections are numbered starting at~\texttt{1}. The number of the current section is also written to the transcript file. All citations given outside a \env{refsection} environment are assigned to section~0. If \cmd{printbibliography} is used within a \env{refsection}, it will automatically select the current section. The \opt{section} option is not required in this case. This also applies to \cmd{printbiblist}.

可选项是特定于该参考文献分节的逗号分隔资源列表。
如果省略了该选项,参考文献节会使用缺省的数据源列表,由导言区的 \cmd{addbibresource} 指定。
如果提供了该选项,它会替代缺省的资源列表。
不过,由 \cmd{addglobalbib} 指定的全局文献资源总是包含在内的。
\env{refsection} 环境不可以相互嵌套,
但是可以在 \env{refsection} 环境内使用 \env{refsegment} 环境来进一步分段。
当打印参考文献时,使用 \cmd{printbibliography} 的 \opt{section} 选项来选择节;
同样地当打印文献列表时使用 \cmd{printbiblist} 对应的选项。
参考文献分节从~\texttt{1} 开始编号。当前节的编号也被写入副本文件中。
所有在 \env{refsection} 环境外给出的引用都归到第~0 节中。
如果在 \env{refsection} 内部使用 \cmd{printbibliography} 环境,它会自动选择当前节。
此时不需要 \opt{section} 选项。这也适用于 \cmd{printbiblist}。

\cmditem{newrefsection}[resource, \dots]

%This command is similar to the \env{refsection} environment except that it is a stand"=alone command rather than an environment. It automatically ends the previous reference section (if any) and immediately starts a new one. Note that the reference section started by the last \cmd{newrefsection} command in the document will extend to the very end of the document. Use \cmd{endrefsection} if you want to terminate it earlier.

该命令类似于 \env{refsection} 环境,不同之处在于它是单独命令而不是一个环境。
它会自动结束之前的文献分节(如果有的话)并立即开始新的一节。
请注意,文档中由最后一个 \cmd{newrefsection} 开始的文献节会延续到文档的最后。
如果你想提前终止的话可以使用 \cmd{endrefsection}。

\end{ltxsyntax}

\subsubsection{参考文献分段}%\subsubsection{Bibliography Segments}
\label{use:bib:seg}

%The \env{refsegment} environment is used in the document body to mark a reference segment. This environment is useful if you want one global bibliography which is subdivided by chapter, section, or any other part of the document. Technically, reference segments are completely independent from document divisions such as \cmd{chapter} and \cmd{section} even though they will typically be used per chapter or section. See the \opt{refsegment} package option in \secref{use:opt:pre:gen} for a way to automate this. Also see \secref{use:use:mlt} for usage examples.

在文档中,\env{refsegment} 环境用来标记参考文献片段。
该环境用于实现在文档的每一章、节或其它部分中将全局的参考文献分成片段。
技术上,尽管文献分段通常在每一章或每一节中使用,
但它们与 \cmd{chapter} 和 \cmd{section} 等文档划分是完全独立的。
关于自动实现这一功能请参考 \secref{use:opt:pre:gen} 中的 \opt{refsegment} 宏包选项。
使用例子也可以参见 \secref{use:use:mlt}。

\begin{ltxsyntax}

\envitem{refsegment}

%The difference between a \env{refsection} and a \env{refsegment} environment is that the former creates labels which are local to the environment whereas the latter provides a target for the \opt{segment} filter of \cmd{printbibliography} without affecting the labels. They will be unique across the entire document. \env{refsegment} environments may not be nested, but you may use them in conjunction with \env{refsection} to subdivide a reference section into segments. In this case, the segments are local to the enclosing \env{refsection} environment. Use the \env{segment} option of \cmd{printbibliography} to select a segment when printing the bibliography. Within a section, the reference segments are numbered starting at~\texttt{1} and the number of the current segment will be written to the transcript file. All citations given outside a \env{refsegment} environment are assigned to segment~0. In contrast to the \env{refsection} environment, the current segment is not selected automatically if \cmd{printbibliography} is used within a \env{refsegment} environment.

\env{regsection} 与 \env{refsegment} 环境的不同之处在于,
前者创建局部于该环境的标签而后者仅为 \cmd{printbibliography} 命令的 \opt{segment} filter 提供目标而不影响标签。
在整个文档中它们是唯一确定的。
\env{refsegment} 环境不可以嵌套,
但是你可以将其与 \env{refsection} 环境结合使用来将文献节细分为段。
此时,这些文献分段是局部于被包含的 \env{refsection} 环境的。
当打印参考文献时,使用 \cmd{printbibliography} 的 \opt{segment} 选项来选择文献分段。
在一节内,文献段从~\texttt{1} 开始编号,并且当前段的编号会被写入到一个副本文件中。
所有在 \env{refsegment} 环境之外的引用都归到第~0 段。
与 \env{refsection} 环境相反,
当 \cmd{printbibliography} 在一个 \env{refsegment} 环境内使用时,当前文献分段并不自动选定。

\csitem{newrefsegment}

%This command is similar to the \env{refsegment} environment except that it is a stand"=alone command rather than an environment. It automatically ends the previous reference segment (if any) and immediately starts a new one. Note that the reference segment started by the last \cmd{newrefsegment} command will extend to the end of the document. Use \cmd{endrefsegment} if you want to terminate it earlier.

该命令类似于 \env{refsegment} 环境,不同之处在于它是单独命令而不是一个环境。
它会自动结束之前的文献分段(如果有的话)并立即开始新的一段。
请注意,由最后一个 \cmd{newrefsegment} 开始的文献分段会延续到文档结束。
如果你想提前终止的话可以使用 \cmd{endrefsegment}。

\end{ltxsyntax}

\subsubsection{参考文献分类}%\subsubsection{Bibliography Categories}
\label{use:bib:cat}

%Bibliography categories allow you to split the bibliography into multiple parts dedicated to different topics or different types of references, for example primary and secondary sources. See \secref{use:use:div} for usage examples.

参考文献分类允许将参考文献针对不同主题或不同文献类型分成若干部分,例如分成主要文献和次要文献。
使用例子参见 \secref{use:use:div} 节。

\begin{ltxsyntax}

\cmditem{DeclareBibliographyCategory}{category}

%Declares a new \prm{category}, to be used in conjunction with \cmd{addtocategory} and the
\opt{category} and \opt{notcategory} filters of \cmd{printbibliography}. This command is used in the document preamble.

声明一个新的 \prm{category},
可以和 \cmd{addtocategory} 以及 \cmd{printbibliography} 的 \opt{category}、\opt{notcategory} filter结合使用。
该命令在导言区中使用。

\cmditem{addtocategory}{category}{key}

%Assigns a \prm{key} to a \prm{category}, to be used in conjunction with the \opt{category} and \opt{notcategory} filters of \cmd{printbibliography}. This command may be used in the preamble and in the document body. The \prm{key} may be a single entry key or a comma"=separated list of keys. The assignment is global.

将 \prm{key} 关键字分配给 \prm{category} 类,
可以和 \cmd{addtocategory} 以及 \cmd{printbibliography} 的 \opt{category}、\opt{notcategory} filter结合使用。
该命令可以在导言区和正文中使用。
\prm{key} 可以是一个单独条目关键字或者逗号分隔的键值列表。该分配是全局的。

\end{ltxsyntax}

\subsubsection{参考文献标题与环境}%\subsubsection{Bibliography Headings and Environments}
\label{use:bib:hdg}

\begin{ltxsyntax}

\cmditem{defbibenvironment}{name}{begin code}{end code}{item code}

%This command defines bibliography environments. The \prm{name} is an identifier passed to the \opt{env} option of \cmd{printbibliography} and \cmd{printbiblist} when selecting the environment. The \prm{begin code} is \latex code to be executed at the beginning of the environment; the \prm{end code} is executed at the end of the environment; the \prm{item code} is code to be executed at the beginning of each entry in the bibliography or a bibliography list. Here is an example of a definition based on the standard \latex \env{list} environment:

该命令定义参考文献环境。
其中 \prm{name} 是标识符,
当选择该环境时会传递给 \cmd{printbibliography} 和 \cmd{printshorthands} 的 \opt{env} 选项。
\prm{begin code} 是该环境开始时执行的 \LaTeX 代码;
而 \prm{end code} 在该环境结束时执行;
\prm{item code} 是在参考文献或者shorthand列表的每一条目开始时执行的代码。
如下是基于 \LaTeX 标准 \env{list} 环境定义的例子。

\begin{ltxexample}
\defbibenvironment{bibliography}
  {\list{}
     {\setlength{\leftmargin}{\bibhang}%
      \setlength{\itemindent}{-\leftmargin}%
      \setlength{\itemsep}{\bibitemsep}%
      \setlength{\parsep}{\bibparsep}}}
  {\endlist}
  {\item}
\end{ltxexample}
%
%As seen in the above example, usage of \cmd{defbibenvironment} is roughly similar to \cmd{newenvironment} except that there is an additional mandatory argument for the \prm{item code}.
如上述例子所示,\cmd{defbibenvironment} 的使用大体类似于 \cmd{newenvironment},
不同之处在于有一个额外的必选项 \prm{item code}。

\cmditem{defbibheading}{name}[title]{code}

%This command defines bibliography headings. The \prm{name} is an identifier to be passed to the \opt{heading} option of \cmd{printbibliography} or \cmd{printbibheading} and \cmd{printbiblist} when selecting the heading. The \prm{code} should be \latex code generating a fully"=fledged heading, including page headers and an entry in the table of contents, if desired. If \cmd{printbibliography} or \cmd{printbiblist} are invoked with a \opt{title} option, the title will be passed to the heading definition as |#1|. If not, the default title specified by the optional \prm{title} argument is passed as |#1| instead. The \prm{title} argument will typically be \cmd{bibname}, \cmd{refname}, or \cmd{biblistname} (see \secref{aut:lng:key:bhd}). This command is often needed after changes to document headers in the preamble. Here is an example of a simple heading definition:

该命令定义参考文献标题。
其中 \prm{name} 是标识符,
当选择该标题时会传递给 \cmd{printbibliography} 和 \cmd{printshorthands} 的 \opt{env} 选项。
\prm{code} 是能生成完整标题的 \LaTeX 代码,包括页眉和目录中的条目(如果必要的话)。
如果 \cmd{printbibliography} 或 \cmd{printshorthands} 带有 \opt{title} 选项,
那么 \opt{title} 将作为 |#1| 传递给标题定义;
否则由可选的 \prm{title} 确定的标题将作为 |#1| 传递给标题定义。
\prm{title} 选项通常是 \cmd{bibname}、\cmd{refname} 或者 \cmd{biblistname}
(见 \secref{aut:lng:key:bhd} 节)。
如果在导言区中改变文档标题时,那么之后通常需要该命令。
如下是一个简单标题定义的例子:

\begin{ltxexample}
\defbibheading{bibliography}[\bibname]{%
  \chapter*{#1}%
  \markboth{#1}{#1}}
\end{ltxexample}

\end{ltxsyntax}

%The following headings, which are intended for use with \cmd{printbibliography} and \cmd{printbibheading}, are predefined:

以下预定义的标题与 \cmd{printbibliography} 和 \cmd{printbibheading} 结合使用:

\begin{valuelist*}

\item[bibliography]
%This is the default heading used by \cmd{printbibliography} if the \opt{heading} option is not given. Its default definition depends on the document class. If the class provides a \cmd{chapter} command, the heading is similar to the bibliography heading of the standard \latex \texttt{book} class, \ie it uses \cmd{chapter*} to create an unnumbered chapter heading which is not included in the table of contents. If there is no \cmd{chapter} command, it is similar to the bibliography heading of the standard \latex \texttt{article} class, \ie it uses \cmd{section*} to create an unnumbered section heading which is not included in the table of contents. The string used in the heading also depends on the document class. With \texttt{book}-like classes the localisation string \texttt{bibliography} is used, with other classes it is \texttt{references} (see \secref{aut:lng:key}). See also \secref{use:cav:scr, use:cav:mem} for class-specific hints.
如果没有给出 \opt{heading} 选项,那么这是 \cmd{printbibliography} 使用的默认标题。
缺省定义取决于文档类。
如果文类提供 \cmd{chapter} 命令,那么该标题就类似于标准 \LaTeX 的 \texttt{book} 文类的参考文献标题,
即使用 \cmd{chapter*} 来创建不带编号的章,并且不包含在目录中。
如果没有 \cmd{chapter} 命令,那么它将类似于标准 \LaTeX 的 \texttt{article} 文类的参考文献标题,
即使用 \cmd{section*} 来创建不带编号的节,并且不包含在目录中。
标题中使用的字符串也取决于文档类。
\texttt{book} 文档类使用本地化字符串 \texttt{bibliography},
在其它文档类中则是 \texttt{references}(见 \secref{aut:lng:key} 节)。
关于文档类的提示也可以见 \secref{use:cav:scr, use:cav:mem} 节。

\item[subbibliography]
%Similar to \texttt{bibliography} but one sectioning level lower. This heading definition uses \cmd{section*} instead of \cmd{chapter*} with a \texttt{book}-like class and \cmd{subsection*} instead of \cmd{section*} otherwise.
类似于 \texttt{bibliography},但是标题格式低一级。
即,在 \texttt{book} 文档类中使用 \cmd{section*} 而不是 \cmd{chapter*},
其它情况使用 \cmd{subsection*} 而不是 \cmd{section*}。

\item[bibintoc]
%Similar to \texttt{bibliography} above but adds an entry to the table of contents.
类似于 \texttt{bibliography} 但是在目录中添加条目。

\item[subbibintoc]
%Similar to \texttt{subbibliography} above but adds an entry to the table of contents.
类似于 \texttt{subbibliography} 但是在目录中添加条目。

\item[bibnumbered]
%Similar to \texttt{bibliography} above but uses \cmd{chapter} or \cmd{section} to create a numbered heading which is also added to the table of contents.
类似于 \texttt{bibliography} 但是使用 \cmd{chapter} 或 \cmd{section} 来创建带编号的条目,
同时也添加到目录中。

\item[subbibnumbered]
%Similar to \texttt{subbibliography} above but uses \cmd{section} or \cmd{subsection} to create a numbered heading which is also added to the table of contents.
类似于 \texttt{bibliography} 但是使用 \cmd{section} 或 \cmd{subsection} 来创建带编号的条目,
同时也添加到目录中。

\item[none]
%A blank heading definition. Use this to suppress the heading.
空白的标题定义,用来取消标题。

\end{valuelist*}

%The following headings intended for use with \cmd{printbiblist} are predefined:
以下预定义的标题与 \cmd{printshorthands} 结合使用:

\begin{valuelist*}

\item[biblist]
%This is the default heading used by \cmd{printbiblist} if the \opt{heading} option is not given. It is similar to \texttt{bibliography} above except that it uses the localisation string \texttt{shorthands} instead of \texttt{bibliography} or \texttt{references} (see \secref{aut:lng:key}). See also \secref{use:cav:scr, use:cav:mem} for class-specific hints.
如果没有给出 \opt{heading} 选项,那么这是 \cmd{printbiblist} 使用的缺省标题。
类似于上面的 \texttt{bibliography},
不过是使用本地化字符串 \texttt{shorthands} 而不是 \texttt{bibliography} 或 \texttt{references}
(见 \secref{aut:lng:key} 节)。
关于文档类的提示另见 \secref{use:cav:scr, use:cav:mem} 节。

\item[biblistintoc]
%Similar to \texttt{biblist} above but adds an entry to the table of contents.
类似于 \texttt{shorthands} 但是在目录中添加条目。

\item[biblistnumbered]
%Similar to \texttt{biblist} above but uses \cmd{chapter} or \cmd{section} to create a numbered heading which is also added to the table of contents.
类似于上面的 \texttt{biblist} 但是使用 \cmd{chapter} 或 \cmd{section} 来创建带编号的标题,
同时也添加到目录中。

\end{valuelist*}

\subsubsection{参考文献注记}%\subsubsection{Bibliography Notes}
\label{use:bib:nts}

\begin{ltxsyntax}

\cmditem{defbibnote}{name}{text}

%Defines the bibliography note \prm{name}, to be used via the \opt{prenote} and \opt{postnote} options of \cmd{printbibliography} and \cmd{printbiblist}. The \prm{text} may be any arbitrary piece of text, possibly spanning several paragraphs and containing font declarations. Also see \secref{use:cav:act}.

定义名为 \prm{name} 的参考文献注记 ,
通过 \cmd{printbibliography} 和 \cmd{printbiblist} 的 \opt{prenote} 和 \opt{postnote} 选项使用。
\prm{text} 可以是任意文本片段,通常包含若干段落和字体声明。
另见 \secref{use:cav:act} 节。

\end{ltxsyntax}

\subsubsection{参考文献过滤和检查} %\subsubsection{Bibliography Filters and Checks}
\label{use:bib:flt}

\begin{ltxsyntax}

\cmditem{defbibfilter}{name}{expression}

%Defines the custom bibliography filter \prm{name}, to be used via the \opt{filter} option of \cmd{printbibliography}. The \prm{expression} is a complex test based on the logical operators \texttt{and}, \texttt{or}, \texttt{not}, the group separator \texttt{(...)}, and the following atomic tests:

定义一个可定制的文献过滤 \prm{name},
可以通过 \cmd{printbibliography} 的 \opt{filter} 选项使用。
\prm{expression} 是复合测试,
基于逻辑运算符 \texttt{and}、\texttt{or}、\texttt{not},组运算符 \texttt{(...)},以及以下的基本测试:

\end{ltxsyntax}

\begin{optionlist*}

\valitem{segment}{integer}

%Matches all entries cited in reference segment \prm{integer}.

匹配所有在参考文献分段 \prm{integer} 中引用的条目。

\valitem{type}{entrytype}

%Matches all entries whose entry type is \prm{entrytype}.

匹配所有类型为 \prm{entrytype} 的条目。

\valitem{subtype}{subtype}

%Matches all entries whose \bibfield{entrysubtype} is \prm{subtype}.

匹配所有 \bibfield{entrysubtype} 域为 \prm{subtype} 的条目。

\valitem{keyword}{keyword}

%Matches all entries whose \bibfield{keywords} field includes \prm{keyword}. If the \prm{keyword} contains spaces, it needs to be wrapped in braces.

匹配所有 \bibfield{keywords} 域包含 \prm{keyword} 的条目。
如果 \prm{keyword} 包含空格,那么需要用括号括起来。

\valitem{category}{category}

%Matches all entries assigned to \prm{category} with \cmd{addtocategory}.

匹配所有由 \cmd{addtocategory} 归入 \prm{category} 类的条目。

\end{optionlist*}

%Here is an example of a filter expression:
如下是一个 filter 表达式的例子:

\begin{ltxexample}[style=latex,keywords={and,or,not,type,keyword}]{}
\defbibfilter{example}{%
  ( type=book or type=inbook )
  and keyword=abc
  and not keyword={x y z}
}
\end{ltxexample}
%
%This filter will match all entries whose entry type is either \bibtype{book} or \bibtype{inbook} and whose \bibfield{keywords} field includes the keyword <\texttt{abc}> but not <\texttt{x y z}>. As seen in the above example, all elements are separated by whitespace (spaces, tabs, or line endings). There is no spacing around the equal sign. The logical operators are evaluated with the \cmd{ifboolexpr} command from the \sty{etoolbox} package. See the \sty{etoolbox} manual for details about the syntax. The syntax of the \cmd{ifthenelse} command from the \sty{ifthen} package, which has been employed in older versions of \biblatex, is still supported. This is the same test using \sty{ifthen}-like syntax:
该 filter 匹配的条目规则是,条目类型是 \bibtype{book} 或 \bibtype{inbook},\bibfield{keywords} 域包含关键词 “\texttt{abc}”但不包含“\texttt{x y z}”。
从以上例子可以看出,所有的元素由空白分开(空格、制表符或者换行)。
等号周围没有空白。逻辑运算使用 \sty{etoolbox} 宏包的 \cmd{ifboolexpr} 执行。
关于该语法详见 \sty{etoolbox} 手册。
\biblatex 旧版本中使用的 \sty{ifthen} 宏包的 \cmd{ifthenelse} 语法这里仍然支持。
如下是相同的测试,使用 \sty{ifthen} 样式的语法:

\begin{ltxexample}[style=ifthen,morekeywords={\\type,\\keyword}]{}
\defbibfilter{example}{%
  \( \type{book} \or \type{inbook} \)
  \and \keyword{abc}
  \and \not \keyword{x y z}
}
\end{ltxexample}
%
%Note that custom filters are local to the reference section in which they are used. Use the \texttt{section} filter of \cmd{printbibliography} to select a different section. This is not possible from within a custom filter.
请注意,定制的 filter 对于所在的参考文献分节是局部的。
使用 \cmd{printbibliography} 的 \texttt{section} filter 来选择不同的分节。
这在定制filter中是不可能的。

\begin{ltxsyntax}

\cmditem{defbibcheck}{name}{code}

%Defines the custom bibliography filter \prm{name}, to be used via the \opt{check} option of \cmd{printbibliography}. \cmd{defbibcheck} is similar in concept to \cmd{defbibfilter} but much more low-level. Rather than a high-level expression, the \prm{code} is \latex code, much like the code used in driver definitions, which may perform arbitrary tests to decide whether or not a given entry is to be printed. The bibliographic data of the respective entry is available when the \prm{code} is executed. Issuing the command \cmd{skipentry} in the \prm{code} will cause the current entry to be skipped. For example, the following filter will only output entries with an \bibfield{abstract} field:

定义了可定制的参考文献 check \prm{name},
可以通过 \cmd{printbibliography} 的 \opt{check} 选项使用。
\cmd{defbibcheck} 从概念上类似于 \cmd{defbibfilter} 不过更加低层。
与高层次表达式不同,\prm{code} 是 \LaTeX 代码,更像是驱动定义中使用的代码,
可以执行任意测试来决定是否打印某个给定的条目。
当执行 \prm{code} 时,相应条目的文献数据是可用的。
在 \prm{code} 中使用 \cmd{skipentry} 命令会跳过当前条目。
例如,下面的 check 只会输出带有 \bibfield{abstract} 域的条目:

\begin{ltxexample}
\defbibcheck{<<abstract>>}{%
  \iffieldundef{abstract}{<<\skipentry>>}{}}
...
\printbibliography[<<check=abstract>>]
\end{ltxexample}
%
%The following check will exclude all entries published before the year 2000:
下面的 check 会排除所有在2000年之前出版的条目:

\begin{ltxexample}
\defbibcheck{recent}{%
  \iffieldint{year}
    {\ifnumless{\thefield{year}}{2000}
       {\skipentry}
       {}}
    {\skipentry}}
\end{ltxexample}
%
%See the author guide, in particular \secref{aut:aux:tst,aut:aux:ife}, for further details.
更多细节请参见作者指南,特别是 \secref{aut:aux:tst,aut:aux:ife} 节。

\end{ltxsyntax}

\subsubsection{著录文境}%\subsubsection{Reference Contexts}
\label{use:bib:context}

%References in a bibliography are cited and printed in a <context>. The context determines the data which is actually used to cite or provide bibliographic data for an entry. A context consists of the following information (the <context> concept is designed for future extensibility):

参考文献列表中文献的引用和打印都处于某个\emph{著录文境}(context)内。
对于某一条目,著录文境决定了实际用于引用或者提供文献信息的数据。
一个著录文境包括以下信息\footnote{
设计“著录文境”这一概念的目的在于,使其在未来具有可扩展性。}:

\begin{itemize}
 \item %A sorting template
 排序格式
 \item %A template for constructing the sorting keys for names
 构建姓名排序关键字的格式
 \item %A string prefix for citation schemes which use alphabetic or numeric labels
 使用字母或数值标签的引用格式的前缀字符串

 \item A template for calculating name uniqueness information
 \item A template for constructing alphabetic labels for names
\end{itemize}
%
%The purpose of bibliography contexts is twofold. Firstly, they are used to set options which influence a printed bibliography and secondly to influence the data printed by citation commands.
%The former use is the most common when one needs to print more than one bibliography list with different, for example, sorting.
著录文境具有双重意义。
首先,会用于设置影响打印参考文献的选项;
其次,设置的选项还可以影响引用命令打印的数据。
前一应用场景是很常见的,例如,
打印多个带有不同排序格式的参考文献表。

\begin{ltxexample}
\usepackage[sorting=nyt]{biblatex}
\begin{document}
\cite{one}
\cite{two}
\printbibliography
\newrefcontext[sorting=ydnt]
\printbibliography
\end{ltxexample}
%
%Here we print two bibliographies, one with the default <nyt> sorting scheme and one with the <ydnt> sorting scheme.
这里我们打印两个参考文献表。
其中一个带有默认的“nyt”排序格式,另一个则使用“ydnt”排序格式。

%To demonstrate the second type of use of bibliography contexts, we have to understand that the actual data for an entry can vary depending on the context. This is most obvious in the case of the \opt{extra*} fields like \opt{extrayear} which are generated by the backend according to the order of entries \emph{after} sorting so that they come out in the expected <a, b, c> order. This clearly shows that the \emph{data} in an entry can be different between sorting schemes. If a document contains more than one bibliography list with different sorting schemes, it can happen then that the \file{.bbl} contains sorting lists with the same entry but containing different data (a different value for \bibfield{extrayear}, for example). The purpose of bibliography contexts is to encapsulate things inside a context so that \biblatex can use the correct entry data. An example is printing a bibliography list with a different sorting order to the global sorting order where the \opt{extra*} fields are different for the same entry between sorting lists:
为了说明著录文境的第二种类型应用,我们必须意识到这一点:
条目的实际数据可以基于不同的著录文境而变化。
在以下的情况中这一点尤其明显:
由后端生成的 \opt{extra*} 域(例如 \opt{extrayear})与条目在排序\emph{之后}的顺序有关,
这样出来的结果就是预期的“a, b, c”的顺序。
这就表明,条目的\emph{数据}在不同排序格式下可以不一样。
如果文档中包含多个带有不同排序格式的参考文献列表,
那么 \file{.bbl} 文件中可能出现多个排序列表,它们带有同一条目但是其数据不同(例如 \bibfield{extrayear} 的值可以不同)。
著录文境的目的就在于将这些事项封装在一个语境内部,
这样 \biblatex 就可以使用正确的条目数据。
以下的例子展示了使用与全局排序格式不同的另一格式打印参考文献列表,
使得同一条目的 \opt{extra*} 域在不同排序列表中是不同的:

\begin{ltxexample}
\usepackage[sorting=nyt,style=authoryear]{biblatex}
\DeclareSortingTemplate{yntd}{
  \sort{
    \field[strside=left,strwidth=4]{sortyear}
    \field[strside=left,strwidth=4]{year}
    \literal{9999}
  }
  \sort{
    \field{sortname}
    \field{author}
    \field{editor}
  }
  \sort[direction=descending]{
    \field{sorttitle}
    \field{title}
  }
}
\begin{document}
\cite{one}
\cite{two}
\printbibliography
\newrefcontext[sorting=yntd]
\cite{one}
\cite{two}
\printbibliography
\end{ltxexample}
%
%Here, the second use of the citations, along with the \cmd{printbibliography} command will use data from the context of the custom <yntd> sorting scheme which may well be different from the data associated with the default <nyt> scheme. That is, the citation labels (in an authoryear style which uses \opt{extrayear}) may be different \emph{for the exact same entries} between different bibliography contexts and so the citations themselves may look different.
这里,第二次使用引用命令和 \cmd{printbibliography} 命令时会使用在定制“yntd”排序格式的著录文境中的数据,
这与默认的“nyt”格式相关联的数据可能会不相同。
也就是说,\emph{对于同一条目},
不同著录文境中的引用标签(在使用 \opt{extrayear} 的 \opt{authoryear} 样式中)可以不同,
这样对其引用就可以不一致。

%Reference contexts can be declared with \cmd{DeclareRefcontext} and referred to by name, see below.

引用文境可以使用 \cmd{DeclareRefcontext} 命令进行声明,
然后通过文境的名称进行使用,见以下说明。

%By default, data for a citation is drawn from the reference context of the last bibliography in which it was printed. For example:
默认情况下,用于引用的数据来自于打印该条目的最后一个参考文献列表所在的引用文境。例如:

\begin{ltxexample}[style=latex]{}
\DeclareRefcontext{ap}{labelprefix=A}
\begin{document}

\cite{book, article, misc}

\printbibliography[type=book]

\newrefcontext{ap}
\printbibliography[type=article]

\newrefcontext[sorting=ydnt]
\printbibliography[type=misc]

\end{document}
\end{ltxexample}
%
%This example also shows the declaration and use of a named reference context. Assuming the entrykeys are indicative of their entrytypes, this is the default situation for the citations which corresponds to what users normally expect:
这个例子同时展示了引用文境的声明和使用。
在该例子中假设条目类型就是条目的键名,
文献引用就对应与用户通常预料到的默认场景。

\begin{itemize}
\item %The citation of entry \bibfield{book} would draw its data from the global reference context, because the last bibliography in which it was printed was the one in the global reference context.
条目 \bibfield{book} 的引用会从全局引用文境中提取数据,
因为打印该条目的最后的参考文献列表位于全局引用文境中。
\item %The citation of entry \bibfield{article} would draw its data from reference context with \opt{labelprefix=A} and would therefore have a <A> prefix when cited.
条目 \bibfield{article} 的引用会从带有 \opt{labelprefix=A} 的引用文境中提取数据,
因此引用时会带有前缀“A”。
\item %The citation of entry \bibfield{misc} would draw its data from the reference context with \opt{sorting=ydnt}
条目 \bibfield{misc} 的引用会从带有 \opt{sorting=ydnt} 的引用文境中提取数据。
\end{itemize}
%
%In cases where the user has entries which occur in multiple bibliographies in different forms or with potentially different labels (in a numeric scheme with different \bibfield{labelprefix} values for example), it may be necessary to tell \biblatex from which reference context you wish to draw the citation information. As shown above this can be done by explicitly putting citations inside reference contexts. This can be onerous in a large document and so there is specific functionality for assigning citations to reference contexts programatically, see the \cmd{assignrefcontext*} macros below.
有这样一种情况,条目在多个参考文献列表中并且有不同的形式或者可能带有不同的标签
(例如,带有不同 \bibfield{labelprefix} 值的数字格式)。
此时需要告诉 \biblatex 希望从哪个引用文境中提取引用信息。
如上所述,这可以通过显式地将引用置于引用文境中而实现。
但是在大文档中这种方式会很繁重,
因此提供了将引用以程序化的方式分配到引用文境的功能,
见下面的 \cmd{assignrefcontext*} 宏命令。

\begin{ltxsyntax}

\cmditem{DeclareRefcontext}{name}{key=value, \dots}

%Declares a named reference context with name \prm{name}. The \prm{key=value} options define the context attributes. All context attributes are optional and default to the global settings if absent. The valid options are:
声明一个名称为 \prm{name} 的引用文境。
\prm{key=value} 选项定义该文境的属性。
所有的文境属性都是可选的,缺省为全局设置。
有效的选项为:

\begin{optionlist*}

\valitem{sorting}{name}

%Specify a sorting template defined previously with \cmd{DeclareSortingtemplate}. This template is used to determine which data to retrieve and/or print for an entry in the commands inside the context.
指定由之前的 \cmd{DeclareSortingScheme} 命令定义的排序格式。对于在该文境内引用命令中的条目,该格式用于确定检索和打印的数据。

\valitem{sortingnamekeytemplatename}{name}

%Specify a sorting name key template defined previously with \cmd{DeclareSortingNamekeyTemplate}. This scheme is used to construct sorting keys for names inside the context.
指定由之前的 \cmd{DeclareSortingNamekeyScheme} 命令定义的排序姓名关键字格式。该格式用于为文境内的姓名构建排序关键字。

\valitem{uniquenametemplatename}{name}

Specify a uniquename template defined previously with \cmd{DeclareUniquenameTemplate} (see \secref{aut:cav:amb}). This template is used to calculate uniqueness information for names inside the context. The template name can also be specified (in increasing order of preference) per"=entry, per"=name list and per"=name. See \secref{apx:opt} for information on setting per"=option, per"=namelist and per"=name options.

\valitem{labelalphanametemplatename}{name}

Specify a template defined previously with \cmd{DeclareLabelalphaNameTemplate} (see \secref{aut:ctm:lab}). This template is used to construct name parts of alphabetic labels for names inside the context. The template name can also be specified (in increasing order of preference) per"=entry, per"=name list and per"=name. See \secref{apx:opt} for information on setting per"=option, per"=namelist and per"=name options.

\valitem{nametemplates}{name}

A convenience meta-option which sets \opt{sortingnamekeytemplate}, \opt{uniquenametemplate} and \opt{labelalphanametemplate} to the same template name. This option can also be specified (in increasing order of preference) per"=entry, per"=name list and per"=name. See \secref{apx:opt} for information on setting per"=option, per"=namelist and per"=name options.

\valitem{labelprefix}{string}

%This option applies to numerical citation\slash bibliography styles only and requires that the \opt{defernumbers} option from \secref{use:opt:pre:gen} be enabled globally. Setting this option will implicitly enable \opt{resetnumbers} for the any \cmd{printbibliography} in the scope of the context (unless overridden by a user-specified value for \opt{resetnumbers}). The option assigns the \prm{string} as a prefix to all entries in the reference context. For example, if the \prm{string} is \texttt{A}, the numerical labels printed will be \texttt{[A1]}, \texttt{[A2]}, \texttt{[A3]}, etc. This is useful for subdivided numerical bibliographies where each subbibliography uses a different prefix. The \prm{string} is available to styles in the \bibfield{labelprefix} field of all affected entries. See \secref{aut:bbx:fld:lab} for details.

该选项只用于数字型引用和文献样式,
需要全局开启 \secref{use:opt:pre:gen} 节中的 \opt{defernumbers} 选项。
设置改选项也会为该文境范围内的任意 \cmd{printbibliography} 启用 \opt{resetnumbers} 选项
(除非 \opt{resetnumbers} 被用户指定的值覆盖)。
该选项将 \prm{string} 作为前缀分配到该引用文境中的所有条目。
例如,如果 \prm{string} 是 \texttt{A},那么打印出来的数值标签就会形如 \texttt{[A1]}, \texttt{[A2]}, \texttt{[A3]} 等。
特别适用的场合是,将参考文献列表划分成带有不同前缀的子列表。
\prm{string} 可以用于所有有关条目中 \bibfield{labelprefix} 域中的样式。
详见 \secref{aut:bbx:fld:lab} 节。

\end{optionlist*}
%

\envitem{refcontext}[key=value, \dots]{name}

%Wraps a reference context environment. The possible \prm{key=value} optional arguments are as for \cmd{DeclareRefcontext} and override options given for the named reference context \prm{name}. \prm{name} can also be omitted as \verb+{}+ or by omitting even the empty braces\footnote{This slightly odd syntax possibility is a result of backwards compatibility with \biblatex $<$3.5}.

将引用文境封装在一个环境内。
可能的 \keyval 可选项和 \cmd{DeclareRefcontext} 中的相同,
并且覆盖名为 \prm{name} 的引用文境的选项。
\prm{name} 也可以省略成 \verb+{}+,甚至空的括号也可以省略\footnote{
这种有点怪异的句法是出于对 \biblatex $<$3.5 的向后兼容性。}

%The \opt{refcontext} environment cannot be nested and \biblatex will generate an error if you try to do so.
\env{refcontext} 环境不可以相互嵌套,如果这样的话 \biblatex 会报错。

\cmditem{newrefcontext}[key=value, \dots]{name}

%This command is similar to the \env{refcontext} environment except that it is a stand"=alone command rather than an environment. It automatically ends any previous reference context section begun with \cmd{newrefcontext} (if any) and immediately starts a new one. Note that the context section started by the last \cmd{newrefcontext} command in the document will extend to the very end of the document. Use \cmd{endrefcontext} if you want to terminate it earlier.
该命令类似于 \env{refcontext} 环境,不同之处在于这是单独的命令而不是环境。
它会自动结束任何之前以 \cmd{newrefcontext} 开始的引用文境片段(如果有的话),
并立即开启新的引用文境。
注意,文档中最后的 \cmd{newrefcontext} 命令开启的引用文境会一直持续到文档的最后。
如果想要提前终止,那么需要使用 \cmd{endrefcontext} 命令。

\end{ltxsyntax}
%
%At the beginning of the document, there is always a global context containing global settings for each of the reference context options. Here is an example summarising the reference contexts with various settings:

在文档的开始,总会有一个全局的文境,其中为每一引用文境选项进行了全局设置。
这里的例子总结了引用文境的不同设置:

\begin{ltxexample}[style=latex]{}
\usepackage[sorting=nty]{biblatex}

\DeclareRefcontext{testrc}{sorting=nyt}

% Global reference context:
%   sorting=nty
%   sortingnamekeytemplate=global
%   labelprefix=

\begin{document}

\begin{refcontext}{testrc}
% reference context:
%   sorting=nyt
%   sortingnamekeytemplate=global
%   labelprefix=
\end{refcontext}

\begin{refcontext}[labelprefix=A]{testrc}
% reference context:
%   sorting=nyt
%   sortingnamekeytemplate=global
%   labelprefix=A
\end{refcontext}

\begin{refcontext}[sorting=ydnt,labelprefix=A]
% reference context:
%   sorting=ydnt
%   sortingnamekeytemplate=global
%   labelprefix=A
\end{refcontext}

\newrefcontext}[labelprefix=B]
% reference context:
%   sorting=nty
%   sortingnamekeytemplate=global
%   labelprefix=B
\endrefcontext

\newrefcontext}[sorting=ynt,labelprefix=C]{testrc}
% reference context:
%   sorting=ynt
%   sortingnamekeytemplate=global
%   labelprefix=C
\endrefcontext
\end{ltxexample}

\begin{ltxsyntax}

\cmditem{assignrefcontextkeyws}[key=value, \dots]{keyword1,keyword2, ...}
\cmditem{assignrefcontextkeyws*}[key=value, \dots]{keyword1,keyword2, ...}
\cmditem{assignrefcontextcats}[key=value, \dots]{category1, category2, ...}
\cmditem{assignrefcontextcats*}[key=value, \dots]{category1, category2, ...}
\cmditem{assignrefcontextentries}[key=value, \dots]{entrykey1, entrykey2, ...}
\cmditem{assignrefcontextentries*}[key=value, \dots]{entrykey1, entrykey2, ...}
\cmditem{assignrefcontextentries}[key=value, \dots]{*}
\cmditem{assignrefcontextentries*}[key=value, \dots]{*}

\end{ltxsyntax}
%These commands automate putting citations into refcontexts when the default behaviour is not sufficient. The default behaviour is that the data for a citation is drawn from the refcontext of the last bibliography in which it was printed. For citations that are used in some way but not printed in a bibliography or bibliography list, they default to drawing their data from the global refcontext established at the beginning of the document. To override this behaviour, instead of manually wrapping citation commands in \env{refcontext} environments, which might be error-prone and tedious, you can register a comma"=separated list of \prm{keywords}, \prm{categories} or \prm{entrykeys} which, respectively, make the entries with any of the specified keywords, entries in any of the specified categories (see \secref{use:use:div}) or entries with any of the specified citation keys draw their data from a particular refcontext specified by the \prm{refcontext key/values} which are parsed as the per the corresponding \env{refcontext} options. Such refcontext auto-assignments are specific to the current refsection. You may specify the same citation key in any of these commands but be aware that assignment is done in the order \prm{keywords}, \prm{categories}, \prm{entrykeys} with the later specifications overriding the earlier. \cmd{assignrefcontextentries} accepts a single asterisk instead of a list of entrykeys which allows the assignment of all keys in a refsection to a refcontext with having to explicitly list them. An example:

当默认行为不充分时,这些命令会自动将引用置于引用文境中。
默认行为是指,从最后打印条目的参考文献列表所在的引用文境中提取引用数据。
对于没有在参考文献列表中打印但是以某种方式使用的引用,
默认会从文档一开始建立的全局引用文境中提取数据。
为了覆盖这一行为,可以手动将引用命令放置在 \env{refcontext} 环境内,
但是这样容易出错并且很繁琐。
除此之外,可以登记一个关于 \prm{keywords}、\prm{categories} 和 \prm{entrykeys} 的逗号分隔列表。
这样,任何带有指定关键字的条目、任何指定类别的条目(见 \secref{use:use:div} 节)、任何指定引用关键字的条目都会分别从由 \prm{refcontext key/values} 指定的特定引用文境中提取数据,
并按照对应的 \env{refcontext} 环境选项进行解析。
这样的引用文境自动分配方式特定于当前的参考文献分节。
你可以在任意的这些命令中指定相同的引用键,
但需要注意的是,分配方式按照 \prm{keywords}, \prm{categories}, \prm{entrykeys} 的顺序,
后面的规范会覆盖之前的规范。
\cmd{assignrefcontextentries} 命令可以接受单个的星号作为选项以代替一列条目键,
这样可以将某一参考文献分节中的所有条目键都分配给某个引用文境,而不必显式列出。
例如:

\begin{ltxexample}[style=latex]{}
\assignrefcontextentries[labelprefix=A]{key2}
\cite{key1}
\begin{refcontext}[labelprefix=B]
\cite{key2}
\end{refcontext}
\end{ltxexample}
%
%Here, the data for the citation of \bibfield{key2} will be drawn from refcontext \opt{labelprefix=A} and not \opt{labelprefix=B} (resulting in a label with prefix <A> and not <B>).
%The starred versions do not override a local refcontext and so with:
这里 \bibfield{key2} 的引用数据会从引用文境 \opt{labelprefix=A} 中提取,而不是 \opt{labelprefix=B}。
即,标签的前缀是“A”不是“B”。
带星号的版本不会覆盖局部的引用文境,也就是:

\begin{ltxexample}[style=latex]{}
\assignrefcontextentries*[labelprefix=A]{key2}
\cite{key1}
\begin{refcontext}[labelprefix=B]
\cite{key2}
\end{refcontext}
\end{ltxexample}
%
%the data for the citation of \bibfield{key2} will be drawn from refcontext \opt{labelprefix=B}. Note that these commands are rarely necessary unless you have multiple bibliographies in which the same citations occur and \biblatex\ cannot by default tell which bibliography list a citation should refer to. See the example file \file{94-labelprefix.tex} for more details.
\bibfield{key2} 的引用数据会从 \opt{labelprefix=B} 引用文境中提取。
注意,这些命令大部分情况下都不必使用,
除非多个参考文献表中有相同的文献引用,并且 \biblatex 按照默认设置不知道引用应该指向哪一个文献列表。
详见文件 \file{94-labelprefix.tex} 中的例子。

\subsubsection{动态条目集}%\subsubsection{Dynamic Entry Sets}
\label{use:bib:set}

%In addition to the \bibtype{set} entry type, \biblatex also supports dynamic entry sets defined on a per-document\slash per-refsection basis. The following command, which may be used in the document preamble or the document body, defines the set \prm{key}:

除了 \bibtype{set} 条目类型之外,\biblatex 也支持基于文档或参考文献分节定义的动态条目集。
下面的命令定义了 \prm{key} 集合,可以用在导言区或正文中:

\begin{ltxsyntax}

\cmditem{defbibentryset}{key}{key1,key2,key3, \dots}

%The \prm{key} is the entry key of the set, which is used like any other entry key when referring to the set. The \prm{key} must be unique and it must not conflict with any other entry key. The second argument is a comma"=separated list of the entry keys which make up the set. \cmd{defbibentryset} implies the equivalent of a \cmd{nocite} command, \ie all sets which are declared are also added to the bibliography. When declaring the same set more than once, only the first invocation of \cmd{defbibentryset} will define the set. Subsequent definitions of the same \prm{key} are ignored and work like \cmd{nocite}\prm{key}. Dynamic entry sets defined in the document body are local to the enclosing \env{refsection} environment, if any. Otherwise, they are assigned to reference section~0. Those defined in the preamble are assigned to reference section~0. See \secref{use:use:set} for further details.

\prm{key} 是集合的条目键,像其它条目键一样用于指向该集合。
\prm{key} 必须是唯一的,并且不能与其它条目键名冲突。
第二个选项是组成该集合的逗号分隔条目键列表。
\cmd{defbibentryset} 也蕴含了与 \cmd{nocite} 命令的等价性,
即所有声明的集合也都添加到参考文献表中。
当多次声明相同集合时,只有第一次调用的 \cmd{defbibentryset} 会定义该集合。
接下来的对于相同 \prm{key} 的定义将被忽略并如同 \cmd{nocite}\prm{key} 一样处理。
在正文中定义的动态条目集如果包含在 \env{refsection} 环境中,那么是局部的。
否则它们会归到第~0 文献分节中。
在导言区中定义的动态条目集也归到第~0 文献分节中。
详见 \secref{use:use:set} 节。

\end{ltxsyntax}

\subsection{标注(引用)命令}%\subsection{Citation Commands}
\label{use:cit}

%All citation commands generally take one mandatory and two optional arguments. The \prm{prenote} is text to be printed at the beginning of the citation. This is usually a notice such as <see> or <compare>. The \prm{postnote} is text to be printed at the very end of the citation. This is usually a page number. If only one of these arguments is given, it is taken as a postnote. If you want to specify a prenote but no postnote, you need to leave the second optional argument empty, as in |\cite[see][]{key}|. The \prm{key} argument to all citation commands is mandatory. This is the entry key or a comma"=separated list of keys corresponding to the entry keys in the \sty{bib} file. In sum, all basic citations commands listed further down have the following syntax:

大体上,所有的引用命令都有一个必选参数和两个可选参数。
前注 \prm{prenote} 是引用开始时打印的文本,通常是“see”或“compare”等提示。
而后注 \prm{postnote} 是引用结束时打印的文本,通常是页码数。
如果只给出一个可选参数,那么将视作后缀。
如果想给出前注但不要后注,那么需要将第二个可选项设置为空,例如 |\cite[see][]{key}|。
所有的引用命令中选项 \prm{key} 都是必须的,
其内容是 \file{bib} 文件中的条目键或者对应于条目键的逗号分隔列表。
总的来说,以下所有的基本引用命令都具有如下的句法结构:

\begin{ltxsyntax}

\cmditem*{command}[prenote][postnote]{keys}<punctuation>

%If the \opt{autopunct} package option from \secref{use:opt:pre:gen} is enabled, they will scan ahead for any \prm{punctuation} immediately following their last argument. This is useful to avoid spurious punctuation marks after citations. This feature is configured with \cmd{DeclareAutoPunctuation}, see \secref{aut:pct:cfg} for details.
如果启用了 \secref{use:opt:pre:gen} 节中的 \opt{autopunct} 宏包选项,
这些命令会首先扫描紧跟在最后选项的 \prm{punctuation} 标点符号。
这可用于避免引用之后出现错误的标点符号。
该特性由 \cmd{DeclareAutoPunctuation} 命令配置,详见 \secref{aut:pct:cfg} 节。

\end{ltxsyntax}

\subsubsection{标准命令} %\subsubsection{Standard Commands}
\label{use:cit:std}

%The following commands are defined by the citation style. Citation styles may provide any arbitrary number of specialized commands, but these are the standard commands typically provided by general-purpose styles.

下列命令由引用样式定义。
引用样式可以提供任意特殊命令,但是这些是由通用样式提供的标准命令。

\begin{ltxsyntax}


\cmditem{cite}[prenote][postnote]{key}
\cmditem{Cite}[prenote][postnote]{key}

%These are the bare citation commands. They print the citation without any additions such as parentheses. The numeric and alphabetic styles still wrap the label in square brackets since the reference may be ambiguous otherwise. \cmd{Cite} is similar to \cmd{cite} but capitalizes the name prefix of the first name in the citation if the \opt{useprefix} option is enabled, provided that there is a name prefix and the citation style prints any name at all.

基本引用命令。只打印出引用,而不带括号等任何附加物。
不过数值和字母样式仍然会将标签放在方括号里,因为否则的话参考文献可能含糊不清。
\cmd{Cite} 与 \cmd{cite} 类似,不同之处仅仅在于,
如果开启了 \opt{useprefix} 选项,并且引用样式会打印全名,
那么引用中名部分(first name)的前缀要大写。

\cmditem{parencite}[prenote][postnote]{key}
\cmditem{Parencite}[prenote][postnote]{key}

%These commands use a format similar to \cmd{cite} but enclose the entire citation in parentheses. The numeric and alphabetic styles use square brackets instead. \cmd{Parencite} is similar to \cmd{parencite} but capitalizes the name prefix of the first name in the citation if the \opt{useprefix} option is enabled, provided that there is a name prefix and the citation style prints any name at all.

这些命令的格式类似于 \cmd{cite},不过将引用全部放入圆括号内。
不过数值和字母样式仍然使用方括号。
\cmd{Parencite} 与 \cmd{parencite} 类似,
不过,如果有姓名前缀并且引用样式打印全名,并且同时开启了 \opt{useprefix} 选项,
那么 \cmd{Parencite} 会使得该姓名前缀首字母大写。

\cmditem{footcite}[prenote][postnote]{key}
\cmditem{footcitetext}[prenote][postnote]{key}

%These command use a format similar to \cmd{cite} but put the entire citation in a footnote and add a period at the end. In the footnote, they automatically capitalize the name prefix of the first name if the \opt{useprefix} option is enabled, provided that there is a name prefix and the citation style prints any name at all. \cmd{footcitetext} differs from \cmd{footcite} in that it uses \cmd{footnotetext} instead of \cmd{footnote}.

这些命令的格式类似于 \cmd{cite},不过将引用的全部放入脚注内并在末尾加上句号。
在脚注中,如果有姓名前缀并且引用样式打印全名,同时开启了 \opt{useprefix} 选项的话,那么该姓名前缀的首字母会自动大写。
\cmd{footcitetext} 与 \cmd{footcite} 的不同之处在于它使用 \cmd{footnotetext} 而不是 \cmd{footnote} 命令。

\end{ltxsyntax}

\subsubsection{样式相关命令}%\subsubsection{Style-specific Commands}
\label{use:cit:cbx}

%The following additional citation commands are only provided by some of the citation styles which ship with this package.
下列额外的引用命令只由本宏包所带的某些引用样式提供。

\begin{ltxsyntax}

\cmditem{textcite}[prenote][postnote]{key}
\cmditem{Textcite}[prenote][postnote]{key}

%These citation commands are provided by all styles that come with this package. They are intended for use in the flow of text, replacing the subject of a sentence. They print the authors or editors followed by a citation label which is enclosed in parentheses. Depending on the citation style, the label may be a number, the year of publication, an abridged version of the title, or something else. The numeric and alphabetic styles use square brackets instead of parentheses. In the verbose styles, the label is provided in a footnote. Trailing punctuation is moved between the author or editor names and the footnote mark. \cmd{Textcite} is similar to \cmd{textcite} but capitalizes the name prefix of the first name in the citation if the \opt{useprefix} option is enabled, provided that there is a name prefix.

这些引用命令在本宏包所带的所有样式中都有提供。
它们用于正文中代替句子成分。
它们打印出作者或编辑,后面接着在括号中括起来的引用标签。
取决于引用样式,标签可以是数字、出版年份、缩写版本的标题等等。
数值和字母样式会使用方括号来代替圆括号。
在详细样式中,标签在脚注中提供。
作者或编辑名与脚注标记的空格将被移除。
\cmd{Textcite} 与 \cmd{textcite} 类似,
不同之处在于如果有前缀名并且启用 \opt{useprefix} 选项的话,引用中的前缀名是大写的。

\cmditem{smartcite}[prenote][postnote]{key}
\cmditem{Smartcite}[prenote][postnote]{key}

%Like \cmd{parencite} in a footnote and like \cmd{footcite} in the body.
在脚注中与 \cmd{parencite} 相似而在正文中与 \cmd{footcite} 类似。

\cmditem{cite*}[prenote][postnote]{key}

%This command is provided by all author-year and author-title styles. It is similar to the regular \cmd{cite} command but merely prints the year or the title, respectively.
该命令由所有的\texttt{作者--年份}和\texttt{作者--标题}样式提供。
它类似于常规的 \cmd{cite} 命令,但是只打印出年份或者标题。

\cmditem{parencite*}[prenote][postnote]{key}

%This command is provided by all author-year and author-title styles. It is similar to the regular \cmd{parencite} command but merely prints the year or the title, respectively.
该命令由所有的\texttt{作者--年份}和\texttt{作者--标题}样式提供。
它类似于常规的 \cmd{parencite} 命令,但是只打印出年份或者标题。

\cmditem{supercite}{key}

%This command, which is only provided by the numeric styles, prints numeric citations as superscripts without brackets. It uses \cmd{supercitedelim} instead of \cmd{multicitedelim} as citation delimiter. Note that any \prm{prenote} and \prm{postnote} arguments are ignored. If they are given, \cmd{supercite} will discard them and issue a warning message.
该命令只在数值样式中提供,会以不带括号的上标形式打印出引用编号。
它使用  \cmd{supercitedelim} 而不是 \cmd{multicitedelim} 作为引用定界符。
请注意,任何的  \prm{prenote} 和 \prm{postnote} 选项都会被忽略。
如果给出的话也会弃掉它们并且显示警告信息。

\end{ltxsyntax}

\subsubsection{合格的引用列表}%\subsubsection{Qualified Citation Lists}
\label{use:cit:mlt}

%This package supports a class of special citation commands called <multicite> commands. The point of these commands is that their argument is a list of citations where each item forms a fully qualified citation with a pre- and\slash or postnote. This is particularly useful with parenthetical citations and citations given in footnotes. It is also possible to assign a pre- and\slash or postnote to the entire list. The multicite commands are built on top of backend commands like \cmd{parencite} and \cmd{footcite}. The citation style provides a multicite definition with \cmd{DeclareMultiCiteCommand} (see \secref{aut:cbx:cbx}). The following example illustrates the syntax of multicite commands:

本宏包支持一类特殊的称之为“多重引用”的引用命令。
这些命令的特点是,它们的选项是一列引用,其中每一项都具有完整的前注/后注形式。
这对于带括号或脚注中给出的引用特别有用。
也可以将一个前注或后注分配给整个列表。
这种多重引用命令构建在 \cmd{parencite} 和 \cmd{footcite} 等后端命令之上。
引用样式通过命令 \cmd{DeclareMultiCiteCommand}(见 \secref{aut:cbx:cbx} 节)提供了多重引用的定义。
下面的例子展示了多重引用命令的语法。

\begin{ltxexample}
\parencites[35]{key1}[88--120]{key2}[23]{key3}
\end{ltxexample}
%
%The format of the arguments is similar to that of the regular citation commands, except that only one citation command is given. If only one optional argument is given for an item in the list, it is taken as a postnote. If you want to specify a prenote but no postnote, you need to leave the second optional argument of the respective item empty:
选项的格式与常规引用命令类似,不过只用给出一个引用命令。
如果对于列表中的某一项只给出一个可选参数,那么将视为后注。
如果只想要前注而不要后注,那么需要将相应项的第二个可选参数设置为空:

\begin{ltxexample}
\parencites[35]{key1}[chapter 2 in][]{key2}[23]{key3}
\end{ltxexample}
%
%In addition to that, the entire citation list may also have a pre- and\slash or postnote. The syntax of these global notes differs from other optional arguments in that they are given in parentheses rather than the usual brackets:
此外,整个的引用列表也可以有一个共同的前注或后注。
与其他可选参数的语法不同,这些全局注记在圆括号而不是通常的方括号中:

\begin{ltxexample}
\parencites<<(>>and chapter 3<<)>>[35]{key1}[78]{key2}[23]{key3}
\parencites<<(>>Compare<<)()>>[35]{key1}[78]{key2}[23]{key3}
\parencites<<(>>See<<)(>>and the introduction<<)>>[35]{key1}[78]{key2}[23]{key3}
\end{ltxexample}
%
%Note that the multicite commands keep on scanning for arguments until they encounter a token that is not the start of an optional or mandatory argument. If a left brace or bracket follows a multicite command, you need to mask it by adding \cmd{relax} or a control space (a backslash followed by a space) after the last valid argument. This will cause the scanner to stop.
请注意,多重引用命令会一直扫描选项,直到遇到一个不是可选或必选参数的字符记号。
如果文本中接着多重引用命令的是一个左圆括号或方括号,
那么需要手动在最后一个有效参数后添加 \cmd{relax} 命令或者控制空格(跟在反斜线后的空格)作为标记。
这样才会停止该命令的扫描。

\begin{ltxexample}[style=latex,showspaces]{}
\parencites[35]{key1}[78]{key2}<<\relax>>[...]
\parencites[35]{key1}[78]{key2}<<\ >>{...}
\end{ltxexample}
%
%By default, this package provides the following multicite commands which correspond to regular commands from \secref{use:cit:std, use:cit:cbx}:
默认情况下,本宏包提供了如下一些多重引用命令,分别对应于 \secref{use:cit:std, use:cit:cbx} 节中常规命令:

\begin{ltxsyntax}

\cmditem{cites}(multiprenote)(multipostnote)[prenote][postnote]{key}|...|[prenote][postnote]{key}
\cmditem{Cites}(multiprenote)(multipostnote)[prenote][postnote]{key}|...|[prenote][postnote]{key}

%The multicite version of \cmd{cite} and \cmd{Cite}, respectively.
\cmd{cite} 和 \cmd{Cite} 的多重引用版本。

\cmditem{parencites}(multiprenote)(multipostnote)[prenote][postnote]{key}|...|[prenote][postnote]{key}
\cmditem{Parencites}(multiprenote)(multipostnote)[prenote][postnote]{key}|...|[prenote][postnote]{key}

%The multicite version of \cmd{parencite} and \cmd{Parencite}, respectively.
\cmd{parencite} 和 \cmd{Parencite} 的多重引用版本。

\cmditem{footcites}(multiprenote)(multipostnote)[prenote][postnote]{key}|...|[prenote][postnote]{key}
\cmditem{footcitetexts}(multiprenote)(multipostnote)[prenote][postnote]{key}|...|[prenote][postnote]{key}

%The multicite version of \cmd{footcite} and \cmd{footcitetext}, respectively.
\cmd{footcite} 和 \cmd{footcitetext} 的多重引用版本。

\cmditem{smartcites}(multiprenote)(multipostnote)[prenote][postnote]{key}|...|[prenote][postnote]{key}
\cmditem{Smartcites}(multiprenote)(multipostnote)[prenote][postnote]{key}|...|[prenote][postnote]{key}

%The multicite version of \cmd{smartcite} and \cmd{Smartcite}, respectively.
\cmd{smartcite} 和 \cmd{Smartcite} 的多重引用版本。

\cmditem{textcites}(multiprenote)(multipostnote)[prenote][postnote]{key}|...|[prenote][postnote]{key}
\cmditem{Textcites}(multiprenote)(multipostnote)[prenote][postnote]{key}|...|[prenote][postnote]{key}

%The multicite version of \cmd{textcite} and \cmd{Textcite}, respectively.
\cmd{textcite} 和 \cmd{Textcite} 的多重引用版本。

\cmditem{supercites}(multiprenote)(multipostnote)[prenote][postnote]{key}|...|[prenote][postnote]{key}

%The multicite version of \cmd{supercite}. This command is only provided by the numeric styles.
\cmd{supercite} 的多重引用版本。该命令只由数值样式提供。

\end{ltxsyntax}

\subsubsection{样式无关命令} %\subsubsection{Style-independent Commands}
\label{use:cit:aut}

%Sometimes it is desirable to give the citations in the source file in a format that is not tied to a specific citation style and can be modified globally in the preamble. The format of the citations is easily changed by loading a different citation style. However, when using commands such as \cmd{parencite} or \cmd{footcite}, the way the citations are integrated with the text is still effectively hard"=coded. The idea behind the \cmd{autocite} command is to provide higher"=level citation markup which makes global switching from inline citations to citations given in footnotes (or as superscripts) possible. The \cmd{autocite} command is built on top of backend commands like \cmd{parencite} and \cmd{footcite}. The citation style provides an \cmd{autocite} definition with \cmd{DeclareAutoCiteCommand} (see \secref{aut:cbx:cbx}). This definition may be activated with the \opt{autocite} package option from \secref{use:opt:pre:gen}. The citation style will usually initialize this package option to a value which is suitable for the style, see \secref{use:xbx:cbx} for details. Note that there are certain limits to high"=level citation markup. For example, inline author-year citation schemes often integrate citations so tightly with the text that it is virtually impossible to automatically convert them to footnotes. The \cmd{autocite} command is only applicable in cases in which you would normally use \cmd{parencite} or \cmd{footcite} (or \cmd{supercite}, with a numeric style). The citations should be given at the end of a sentence or a partial sentence, immediately preceding the terminal punctuation mark, and they should not be a part of the sentence in a grammatical sense (like \cmd{textcite}, for example).

有时我们需要源文件中的引用格式不依赖于某种特定的引用样式,而是可以在导言区中全局修改。
通过导入不同的引用样式我们可以很容易地改变引用格式。
但是,当使用 \cmd{parencite} 或 \cmd{footcite} 等命令时,引用和正文结合的方式仍然是硬编码的(hard-coded)。
\cmd{autocite} 命令的想法是提供高层次上的引用标记,使得可以全局地切换行内引用和脚注引用(或上标引用)。
\cmd{autocite} 命令构建在 \cmd{parencite} 和 \cmd{footcite} 等后端命令之上。
由引用样式通过 \cmd{DeclareAutoCiteCommand} (见 \secref{aut:cbx:cbx} 节)来定义。
该定义可以通过 \secref{use:opt:pre:gen} 节中的 \opt{autocite} 宏包选项启用。
引用样式通常会将该宏包选项初始化为适合该样式的值,详见 \secref{use:xbx:cbx}。
请注意,这种高层次引用标记有一些限制。
例如,行内的\texttt{作者-年份}引用格式通常和正文结合得很紧密,事实上不可能自动转化为脚注。
只有当正常地使用 \cmd{parencite} 或 \cmd{footcite} (或者数值样式中的 \cmd{supercite})时,\cmd{autocite} 命令才是可用的。
引用应当在句子或其部分的末尾、标点符号之前给出,并且在语法上不应当是句子的一部分(例如 \cmd{textcite})。

\begin{ltxsyntax}

\cmditem{autocite}[prenote][postnote]{key}
\cmditem{Autocite}[prenote][postnote]{key}

%In contrast to other citation commands, the \cmd{autocite} command does not only scan ahead for punctuation marks following its last argument to avoid double punctuation marks, it actually moves them around if required. For example, with \kvopt{autocite}{footnote}, a trailing punctuation mark will be moved such that the footnote mark is printed after the punctuation. \cmd{Autocite} is similar to \cmd{autocite} but capitalizes the name prefix of the first name in the citation if the \opt{useprefix} option is enabled, provided that there is a name prefix and the citation style prints any name at all.

与其它引用命令不同,\cmd{autocite} 命令不仅扫描跟在最后一个选项的标点符号来避免双重标点,还在必要时挪动标点。
例如,后面的标点在 \kvopt{autocite}{footnote} 选项下会被移动使得脚注标记打印在标点之后。
\cmd{Autocite} 类似于 \cmd{autocite},不同之处在于,
如果引用样式打印全名并且有姓名前缀,并且启用了 \opt{useprefix} 选项,那么引用中的姓名前缀会大写。

\cmditem*{autocite*}[prenote][postnote]{key}
\cmditem*{Autocite*}[prenote][postnote]{key}

%The starred variants of \cmd{autocite} do not behave differently from the regular ones. The asterisk is simply passed on to the backend command. For example, if \cmd{autocite} is configured to use \cmd{parencite}, then \cmd{autocite*} will execute \cmd{parencite*}.

\cmd{autocite} 带星号的变种没有什么不同。
星号只是传递给后端命令。
例如,如果 \cmd{autocite} 配置使用 \cmd{parencite},那么 \cmd{autocite*} 就会执行 \cmd{parencite*}。

\cmditem{autocites}(multiprenote)(multipostnote)[prenote][postnote]{key}|...|[prenote][postnote]{key}
\cmditem{Autocites}(multiprenote)(multipostnote)[prenote][postnote]{key}|...|[prenote][postnote]{key}

%This is the multicite version of \cmd{autocite}. It also detects and moves punctuation if required. Note that there is no starred variant. \cmd{Autocites} is similar to \cmd{autocites} but capitalizes the name prefix of the first name in the citation if the \opt{useprefix} option is enabled, provided that there is a name prefix and the citation style prints any name at all.

\cmd{autocite} 的多重引用版本。
它同样会在必要时检测和移动标点。
请注意它没有带星号的变种。
\cmd{Autocites} 类似于 \cmd{autocites},不同之处在于,
如果引用样式打印全名并且有姓名前缀,并且启用了 \opt{useprefix} 选项,那么引用中的姓名前缀会大写。

\end{ltxsyntax}

\subsubsection{文本命令}%\subsubsection{Text Commands}
\label{use:cit:txt}

%The following commands are provided by the core of \biblatex. They are intended for use in the flow of text. Note that all text commands are excluded from citation tracking.

下列命令由 \biblatex 内核提供,在文本流中使用。
请注意,这里所有的文本命令都与引用追踪无关。

\begin{ltxsyntax}

\cmditem{citeauthor}[prenote][postnote]{key}
\cmditem*{citeauthor*}[prenote][postnote]{key}
\cmditem{Citeauthor}[prenote][postnote]{key}
\cmditem*{Citeauthor*}[prenote][postnote]{key}

%These commands print the authors. Strictly speaking, it prints the \bibfield{labelname} list, which may be the
\bibfield{author}, the \bibfield{editor}, or the \bibfield{translator}. \cmd{Citeauthor} is similar to \cmd{citeauthor} but capitalizes the name prefix of the first name in the citation if the \opt{useprefix} option is enabled, provided that there is a name prefix. The starred variants effectively force maxcitenames to 1 for just this command on so only print the first name in the labelname list (potentially followed by the «et al» string if there are more names). This allows more natural textual flow when refering to a paper in the singular when otherwise \cmd{citeauthor} would generate a (naturally plural) list of names.

这些命令打印出作者。
严格地讲,打印出的是 \bibfield{labelnames} 列表,
这可以是 \bibfield{author}、\bibfield{editor} 或者 \bibfield{translator} 等域。
\cmd{Citeauthor} 类似于 \cmd{citeauthor},不同之处在于,
如果有姓名前缀并且启用  \opt{useprefix} 选项的话,引用中的名字前缀采用大写形式。
带星号的版本会强制设置 \texttt{maxcitenames} 为 1,
这样只会打印标签名称列表中第一个姓名(如果有更多名字的话会后接“et al”字符串)。
当涉及单一作者的文章时,这会使行文更加自然;
否则的话(多位作者时),使用 \cmd{citeauthor} 会产生名称列表(当然是复数的)。


\cmditem{citetitle}[prenote][postnote]{key}
\cmditem*{citetitle*}[prenote][postnote]{key}

%This command prints the title. It will use the abridged title in the \bibfield{shorttitle} field, if available. Otherwise it falls back to the full title found in the \bibfield{title} field. The starred variant always prints the full title.

该命令会打印出标题。
它会在可用时使用 \bibfield{shorttitle} 域中的短标题,否则的话会使用备用的 \bibfield{title} 域中的标题全称。
带星号的版本总会打印出标题全称。

\cmditem{citeyear}[prenote][postnote]{key}
\cmditem*{citeyear*}[prenote][postnote]{key}

%This command prints the year (\bibfield{year} field or year component of \bibfield{date}). The starred variant includes the \bibfield{extrayear} information, if any.

该命令会打印出年份(\bibfield{year} 域或者 \bibfield{date} 域中的年份成分)。
带星号的版本会包括可能有的 \bibfield{extrayear} 信息。

\cmditem{citedate}[prenote][postnote]{key}
\cmditem*{citedate*}[prenote][postnote]{key}

%This command prints the full date (\bibfield{date} or \bibfield{year}). The starred variant includes the \bibfield{extrayear} information, if any.

该命令会打印出日期(\bibfield{date} 或 \bibfield{year} 域)。
带星号的版本会包括可能有的 \bibfield{extrayear} 信息。

\cmditem{citeurl}[prenote][postnote]{key}

%This command prints the \bibfield{url} field.

该命令打印 \bibfield{url} 域。

\cmditem{parentext}{text}

%This command wraps the \prm{text} in context sensitive parentheses.

该命令将 \prm{text} 装入上下文相关的圆括号中。

\cmditem{brackettext}{text}

%This command wraps the \prm{text} in context sensitive brackets.

该命令将 \prm{text} 装入上下文相关的方括号中。

\end{ltxsyntax}

\subsubsection{特殊命令}%\subsubsection{Special Commands}
\label{use:cit:spc}

%The following special commands are also provided by the core of \biblatex.

以下特殊命令同样由 \biblatex 内核提供。

\begin{ltxsyntax}

\cmditem{nocite}{key}
\cmditem*{nocite}|\{*\}|

%This command is similar to the standard \latex \cmd{nocite} command. It
adds the \prm{key} to the bibliography without printing a citation. If the
\prm{key} is an asterisk, all entries available in the in-scope bibliography datasource(s) are added to the bibliography. Like all other citation commands, \cmd{nocite} commands in the document body are local to the enclosing \env{refsection} environment, if any. In contrast to standard \latex, \cmd{nocite} may also be used in the document preamble. In this case, the references are assigned to reference section~0.

该命令类似于 \LaTeX\ 中的 \cmd{nocite} 命令。
它将没有引用的条目 \prm{key} 添加到参考文献中。
如果 \prm{key} 是星号,\file{bib} 文件中的所有可用条目都会被添加到参考文献中。
与其它引用命令一样,正文中的 \cmd{nocite} 是在可能的 \env{refsection} 环境中的局部命令。
与标准的\LaTeX\ 不同的是,\cmd{nocite} 也可以在导言区中使用,此时参考文献将归到第零参考文献分节中。

\cmditem{fullcite}[prenote][postnote]{key}

%This command uses the bibliography driver for the respective entry type to create a full citation similar to the bibliography entry. It is thus related to the bibliography style rather than the citation style.

该命令针对相应的条目类型使用参考文献驱动,从而创建类似于文献条目的完整引用格式。
当然,这样的话就与参考文献样式相关而不是与引用样式相关。

\cmditem{footfullcite}[prenote][postnote]{key}

%Similar to \cmd{fullcite} but puts the entire citation in a footnote and adds a period at the end.

类似于 \cmd{fullcite},但是会将整个引用放在脚注中并在末尾添加句号。

\cmditem{volcite}[prenote]{volume}[page]{key}
\cmditem{Volcite}[prenote]{volume}[page]{key}

%These commands are similar to \cmd{cite} and \cmd{Cite} but intended for references to multi"=volume works which are cited by volume and page number. Instead of the \prm{postnote}, they take a mandatory \prm{volume} and an optional \prm{page} argument. Since they merely compose the postnote and pass it to the \cmd{cite} command provided by the citation style as a \prm{postnote} argument, these commands are style independent. The format of the volume portion is controlled by the field formatting directive \opt{volcitevolume}, the format of the page/text portion is controlled by the field formatting directive \opt{volcitepages} (\secref{aut:fmt:ich}). The delimiter printed between the volume portion and the page/text portion may be modified by redefining the macro \cmd{volcitedelim} (\secref{aut:fmt:fmt}).

这些命令类似于 \cmd{cite} 和 \cmd{Cite},不过用于多卷作品以卷数和页码样式的引用。
与 \prm{postnote} 不同,命令中有一个必选的 \prm{volume} 和一个可选的 \prm{page} 选项。
这些命令实际上是样式无关的,
因为 \prm{volume} 和 \prm{page} 选项构成后注并且将其以 \prm{postnote} 选项传递给由引用样式提供的 \cmd{cite} 命令。
卷数部分的格式由域格式指令 \opt{volcitevolume} 直接控制,
而页码或文本部分的格式由域格式指令 \opt{volcitepages} (\secref{aut:fmt:ich})直接控制。
卷数部分与页码部分之间的定界符可用通过重新定义 \cmd{volcitedelim} 宏(见 \secref{aut:fmt:fmt} 节)来修改。

\cmditem{volcites}(multiprenote)(multipostnote)[prenote]{volume}[page]{key}|\\...|[prenote]{volume}[page]{key}
\cmditem{Volcites}(multiprenote)(multipostnote)[prenote]{volume}[page]{key}|\\...|[prenote]{volume}[page]{key}

%The multicite version of \cmd{volcite} and \cmd{Volcite}, respectively.
\cmd{volcite} 和 \cmd{Volcite} 命令的多重引用版本。

\cmditem{pvolcite}[prenote]{volume}[page]{key}
\cmditem{Pvolcite}[prenote]{volume}[page]{key}

%Similar to \cmd{volcite} but based on \cmd{parencite}.
类似于 \cmd{volcite},不过基于 \cmd{parencite} 命令。

\cmditem{pvolcites}(multiprenote)(multipostnote)[prenote]{volume}[page]{key}|\\...|[prenote]{volume}[page]{key}
\cmditem{Pvolcites}(multiprenote)(multipostnote)[prenote]{volume}[page]{key}|\\...|[prenote]{volume}[page]{key}

%The multicite version of \cmd{pvolcite} and \cmd{Pvolcite}, respectively.
\cmd{pvolcite} 和 \cmd{Pvolcite} 相应的多重引用版本。

\cmditem{fvolcite}[prenote]{volume}[page]{key}
\cmditem{ftvolcite}[prenote]{volume}[page]{key}

%Similar to \cmd{volcite} but based on \cmd{footcite} and \cmd{footcitetext}, respectively.
类似于 \cmd{volcite} 但是分别基于 \cmd{footcite} 和 \cmd{footcitetext}。

\cmditem{fvolcites}(multiprenote)(multipostnote)[prenote]{volume}[page]{key}|\\...|[prenote]{volume}[page]{key}
\cmditem{Fvolcites}(multiprenote)(multipostnote)[prenote]{volume}[page]{key}|\\...|[prenote]{volume}[page]{key}

%The multicite version of \cmd{fvolcite} and \cmd{Fvolcite}, respectively.
\cmd{fvolcite} 和 \cmd{Fvolcite} 相应的多重引用版本。

\cmditem{svolcite}[prenote]{volume}[page]{key}
\cmditem{Svolcite}[prenote]{volume}[page]{key}

%Similar to \cmd{volcite} but based on \cmd{smartcite}.
类似于 \cmd{volcite} 但是基于 \cmd{smartcite}。

\cmditem{svolcites}(multiprenote)(multipostnote)[prenote]{volume}[page]{key}|\\...|[prenote]{volume}[page]{key}
\cmditem{Svolcites}(multiprenote)(multipostnote)[prenote]{volume}[page]{key}|\\...|[prenote]{volume}[page]{key}

%The multicite version of \cmd{svolcite} and \cmd{Svolcite}, respectively.
\cmd{svolcite} 和 \cmd{Svolcite} 相应的多重引用版本。

\cmditem{tvolcite}[prenote]{volume}[page]{key}
\cmditem{Tvolcite}[prenote]{volume}[page]{key}

%Similar to \cmd{volcite} but based on \cmd{textcite}.
类似于 \cmd{volcite} 但是基于 \cmd{textcite}。

\cmditem{tvolcites}(multiprenote)(multipostnote)[prenote]{volume}[page]{key}|\\...|[prenote]{volume}[page]{key}
\cmditem{Tvolcites}(multiprenote)(multipostnote)[prenote]{volume}[page]{key}|\\...|[prenote]{volume}[page]{key}

%The multicite version of \cmd{tvolcite} and \cmd{Tvolcite}, respectively.
\cmd{tvolcite} 和 \cmd{Tvolcite} 相应的多重引用版本。

\cmditem{avolcite}[prenote]{volume}[page]{key}
\cmditem{Avolcite}[prenote]{volume}[page]{key}

%Similar to \cmd{volcite} but based on \cmd{autocite}.
类似于 \cmd{volcite} 但是基于 \cmd{autocite}。

\cmditem{avolcites}(multiprenote)(multipostnote)[prenote]{volume}[page]{key}|\\...|[prenote]{volume}[page]{key}
\cmditem{Avolcites}(multiprenote)(multipostnote)[prenote]{volume}[page]{key}|\\...|[prenote]{volume}[page]{key}

%The multicite version of \cmd{avolcite} and \cmd{Avolcite}, respectively.
\cmd{avolcite} 和 \cmd{Avolcite} 相应的多重引用版本。

\cmditem{notecite}[prenote][postnote]{key}
\cmditem{Notecite}[prenote][postnote]{key}

%These commands print the \prm{prenote} and \prm{postnote} arguments but no citation. Instead, a \cmd{nocite} command is issued for every \prm{key}. This may be useful for authors who incorporate implicit citations in their writing, only giving information not mentioned before in the running text, but who still want to take advantage of the automatic \prm{postnote} formatting and the implicit \cmd{nocite} function. This is a generic, style"=independent citation command. Special citation styles may provide smarter facilities for the same purpose. The capitalized version forces capitalization (note that this is only applicable if the note starts with a command which is sensitive to \biblatex's punctuation tracker).

这些命令打印出 \prm{prenote} 和 \prm{postnote} 选项但是没有引用部分。
与之不同的是,\cmd{nocite} 命令用于每个 \prm{key}。
主要的应用场合是,
用户想要隐式地包含引用条目,但只给出没有在之前行文中提到的信息,
同时仍想利用自动的 \prm{postnote} 格式以及隐式的 \cmd{nocite} 功能。
这是一般的、样式无关的引用命令。
一些特殊的引用样式可以针对相同目的提供更加智能的工具。
其中,\cmd{Notecite} 命令会强制大写
(请注意,只有当注记以 \biblatex 标点追踪器相关的命令开始时,该命令才是可用的)。

\cmditem{pnotecite}[prenote][postnote]{key}
\cmditem{Pnotecite}[prenote][postnote]{key}

%Similar to \cmd{notecite} but the notes are printed in parentheses.
类似于 \cmd{notecite} 但是注记打印在括号中。

\cmditem{fnotecite}[prenote][postnote]{key}

%Similar to \cmd{notecite} but the notes are printed in a footnote.
类似于 \cmd{notecite} 但是注记打印在脚注中。

\end{ltxsyntax}

\subsubsection{底层命令}%\subsubsection{Low-level Commands}
\label{use:cit:low}

%The following commands are also provided by the core of \biblatex. They grant access to all lists and fields at a lower level.
以下命令同样由 \biblatex 内核提供,向所有列表和域提供底层接口。

\begin{ltxsyntax}

\cmditem{citename}[prenote][postnote]{key}[format]{name list}

%The \prm{format} is a formatting directive defined with \cmd{DeclareNameFormat}. Formatting directives are discussed in \secref{aut:bib:fmt}. If this optional argument is omitted, this command falls back to the format \texttt{citename}. The last argument is the name of a \prm{name list}, in the sense explained in \secref{bib:fld}.

\prm{format} 是由 \cmd{DeclareNameFormat} 定义的格式指令。
格式指令的讨论在 \secref{aut:bib:fmt} 节。
如果忽略可选参数,该命令调用备选的 \texttt{citename} 格式。
最后一个选项是 \prm{name list} 的名称(其意义在 \secref{bib:fld} 节中解释)。

\cmditem{citelist}[prenote][postnote]{key}[format]{literal list}

%The \prm{format} is a formatting directive defined with \cmd{DeclareListFormat}. Formatting directives are discussed in \secref{aut:bib:fmt}. If this optional argument is omitted, this command falls back to the format \texttt{citelist}. The last argument is the name of a \prm{literal list}, in the sense explained in \secref{bib:fld}.

\prm{format} 是由 \cmd{DeclareListFormat} 定义的格式指令。
格式指令的讨论在 \secref{aut:bib:fmt} 节。
如果忽略可选参数,该命令调用备选的 \texttt{citelist} 格式。
最后一个选项是 \prm{literal list} 的名称(其意义在 \secref{bib:fld} 节中解释)。

\cmditem{citefield}[prenote][postnote]{key}[format]{field}

%The \prm{format} is a formatting directive defined with \cmd{DeclareFieldFormat}. Formatting directives are discussed in \secref{aut:bib:fmt}. If this optional argument is omitted, this command falls back to the format \texttt{citefield}. The last argument is the name of a \prm{field}, in the sense explained in \secref{bib:fld}.

\prm{format} 是由 \cmd{DeclareFieldFormat} 定义的格式指令。
格式指令的讨论在 \secref{aut:bib:fmt} 一节。
如果忽略可选参数,该命令调用备选的 \texttt{citefield} 格式。
最后一个选项是 \prm{field} 的名称(其意义在 \secref{bib:fld} 节中解释)。

\end{ltxsyntax}

\subsubsection{其它命令}%\subsubsection{Miscellaneous Commands}
\label{use:cit:msc}

%The commands in this section are little helpers related to citations.

本节中的命令是一些与引用相关的小工具。

\begin{ltxsyntax}

\csitem{citereset}

%This command resets the citation style. This may be useful if the style replaces repeated citations with abbreviations like \emph{ibidem}, \emph{idem}, \emph{op. cit.}, etc. and you want to force a full citation at the beginning of a new chapter, section, or some other location. The command executes a style specific initialization hook defined with the \cmd{InitializeCitationStyle} command from \secref{aut:cbx:cbx}. It also resets the internal citation trackers of this package. The reset will affect the \cmd{ifciteseen}, \cmd{ifentryseen}, \cmd{ifciteibid}, and \cmd{ifciteidem} tests discussed in \secref{aut:aux:tst}. When used inside a \env{refsection} environment, the reset of the citation tracker is local to the current \env{refsection} environment. Also see the \opt{citereset} package option in \secref{use:opt:pre:gen}.

该命令重新设置引用样式。
如果样式使用 \emph{ibidem}、\emph{idem}、\emph{op. cit.} 等缩略语代替重复引用,
而你想在新的章节开始或别的地方强制改为引用全称,那么该命令很有用。
该命令执行由 \secref{aut:cbx:cbx} 节的 \cmd{InitializeCitationStyle} 命令所定义的样式相关初始化钩子(hook)。
它同样重新设置了本宏包的内部引用追踪器。
该重置会影响 \secref{aut:aux:tst} 节讨论的 \cmd{ifciteseen}、\cmd{ifentryseen}、\cmd{ifciteibid} 和 \cmd{ifciteidem} 等测试。
当在 \env{refsection} 环境内部使用时,引用追踪器的重置仅局部在当前 \env{refsection} 环境中。
另见 \secref{use:opt:pre:gen} 节的 \opt{citereset} 宏包选项。

\csitem{citereset*}

%Similar to \cmd{citereset} but only executes the style's initialization hook, without resetting the internal citation trackers.

类似于 \cmd{citereset} 但只执行样式初始化钩子,而不重置内部引用追踪器。

\csitem{mancite}

%Use this command to mark manually inserted citations if you mix automatically generated and manual citations. This is particularly useful if the citation style replaces repeated citations by an abbreviation like \emph{ibidem} which may get ambiguous or misleading otherwise. Always use \cmd{mancite} in the same context as the manual citation, \eg if the citation is given in a footnote, include \cmd{mancite} in the footnote. The \cmd{mancite} command executes a style specific reset hook defined with the \cmd{OnManualCitation} command from \secref{aut:cbx:cbx}. It also resets the internal <ibidem> and <idem> trackers of this package. The reset will affect the \cmd{ifciteibid} and \cmd{ifciteidem} tests discussed in \secref{aut:aux:tst}.

如果你想混合自动生成引用和手工引用,可以使用该命令来标记手工插入的引用。
如果引用样式使用 \emph{ibidem} 等可能含糊不清或引起歧义的缩略语代替重复引用,那么该命令特别有用。
应当总是在相同的场合使用 \cmd{mancite} 作为手工引用,
例如,如果引用在脚注中给出,那么在脚注中加入 \cmd{mancite}。
\cmd{mancite} 命令会执行 \secref{aut:cbx:cbx} 节的 \cmd{OnManualCitation} 命令定义的样式相关重置钩子。
它也会重置本宏包的内部“ibidem”和“idem”追踪器。
该重置会影响 \secref{aut:aux:tst} 节讨论的 \cmd{ifciteibid} 和 \cmd{ifciteidem} 等测试。

\csitem{pno}

%This command forces a single page prefix in the \prm{postnote} argument to a citation command. See \secref{use:cav:pag} for further details and usage instructions. Note that this command is only available locally in citations and the bibliography.

该命令在引用命令的 \prm{postnote} 选项中强制开始单页前缀。
更多细节和使用说明见 \secref{use:cav:pag} 节。
请注意,该命令只能用于引用和参考文献中局部。

\csitem{ppno}

%Similar to \cmd{pno} but forces a range prefix. See \secref{use:cav:pag} for further details and usage instructions. Note that this command is only available locally in citations and the bibliography.

类似于 \cmd{pno} 但是强制为区间前缀。
更多细节和使用说明见 \secref{use:cav:pag} 节。
请注意,该命令只能用于引用和参考文献中局部。

\csitem{nopp}

%Similar to \cmd{pno} but suppresses all prefixes. See \secref{use:cav:pag} for further details and usage instructions. Note that this command is only available locally in citations and the bibliography.

类似于 \cmd{pno} 但是抑制所有前缀。
更多细节和使用说明见 \secref{use:cav:pag} 节。
请注意,该命令只能用于引用和参考文献中局部。

\csitem{psq}

%In the \prm{postnote} argument to a citation command, this command indicates a range of two pages where only the starting page is given. See \secref{use:cav:pag} for further details and usage instructions. The suffix printed is the localisation string \texttt{sequens}, see \secref{aut:lng:key}. The spacing inserted between the suffix and the page number may be modified by redefining the macro \cmd{sqspace}. The default is an unbreakable interword space. Note that this command is only available locally in citations and the bibliography.

在引用命令的 \prm{postnote} 选项中,该命令在只给出开始页之处表示两页的范围。
更多细节和使用说明见 \secref{use:cav:pag} 节。
打印出的后缀是本地化字符串 \texttt{sequens}(见 \secref{aut:lng:key} 节)。
在后缀和页码之间的空格可以通过重定义 \cmd{sqspace} 修改。
默认值是一个不可打断的词间空格。
请注意,该命令只能用于引用和参考文献中局部。

\csitem{psqq}

%Similar to \cmd{psq} but indicates an open-ended page range. See \secref{use:cav:pag} for further details and usage instructions. The suffix printed is the localisation string \texttt{sequentes}, see \secref{aut:lng:key}. This command is only available locally in citations and the bibliography.

类似于 \cmd{psq} 但表示一个不包括结束页的页码范围。
更多细节和使用说明见 \secref{use:cav:pag} 节。
打印出的后缀是本地化字符串 \texttt{sequens}(见 \secref{aut:lng:key} 节)。
请注意,该命令只能用于引用和参考文献中局部。

\cmditem{RN}{integer}

%This command prints an integer as an uppercase Roman numeral. The formatting applied to the numeral may be modified by redefining the macro \cmd{RNfont}.

该命令将一个整数打印为大写罗马数字形式。
可以通过重定义 \cmd{RNfont} 宏来修改用于该数值的格式。

\cmditem{Rn}{integer}

%Similar to \cmd{RN} but prints a lowercase Roman numeral. The formatting applied to the numeral may be modified by redefining the macro \cmd{Rnfont}.

类似于 \cmd{RN} 但是打印出小写罗马数字。
可以通过重定义 \cmd{Rnfont} 宏来修改用于该数值的格式。

\end{ltxsyntax}

\subsubsection{\sty{natbib} 兼容命令} %\subsubsection{\sty{natbib} Compatibility Commands}
\label{use:cit:nat}

%The \opt{natbib} package option loads a \sty{natbib} compatibility module. The module defines aliases for the citation commands provided by the \sty{natbib} package. This includes aliases for the core citation commands \cmd{citet} and \cmd{citep} as well as the variants \cmd{citealt} and \cmd{citealp}. The starred variants of these commands, which print the full author list, are also supported. The \cmd{cite} command, which is handled in a particular way by \sty{natbib}, is not treated in a special way. The text commands (\cmd{citeauthor}, \cmd{citeyear}, etc.) are also supported, as are all commands which capitalize the name prefix (\cmd{Citet}, \cmd{Citep}, \cmd{Citeauthor}, etc.). Aliasing with \cmd{defcitealias}, \cmd{citetalias}, and \cmd{citepalias} is possible as well. Note that the compatibility commands will not emulate the citation format of the \sty{natbib} package. They merely alias \sty{natbib}'s commands to functionally equivalent facilities of the \biblatex package. The citation format depends on the main citation style. However, the compatibility style will adapt \cmd{nameyeardelim} to match the default style of the \sty{natbib} package.

本宏包的 \opt{natbib} 选项会导入 \sty{natbib} 兼容性模块。
该模块定义了由 \sty{natbib} 宏包提供的引用命令的别名。
其中包括核心引用命令 \cmd{citet} 和 \cmd{citep} 以及 \cmd{citealt} 和 \cmd{citealp} 等变形命令。
另外也支持这些命令带星号的变种,从而可以打印出完整的作者列表。
\cmd{cite} 命令在 \sty{natbib} 宏包中经过了特殊处理,但在这里不会专门处理。
同时,也支持文本命令(\cmd{citeauthor}、\cmd{citeyear}等)
以及所有首字母大写的命令(\cmd{Citet}、\cmd{Citep}、\cmd{Citeauthor} 等)。
另外,\cmd{defcitealias}、\cmd{citetalias} 和 \cmd{citepalias} 也是可以使用的。
请注意,这些兼容性命令不会模仿 \sty{natbib} 宏包的引用格式,
而仅仅是 \sty{biblatex} 宏包中功能上等价的 \sty{natbib} 命令别称。
引用格式取决于主引用样式。
不过,兼容性样式会调整 \cmd{nameyeardelim} 命令以匹配 \sty{natbib} 宏包的默认样式。

\subsubsection[\sty{mcite} 引用命令]{\sty{mcite} 引用命令}
%\subsubsection[\sty{mcite}-like Citation Commands]{\sty{mcite}-like Citation Commands}
\label{use:cit:mct}

%The \opt{mcite} package option loads a special citation module which provides \sty{mcite}\slash \sty{mciteplus}-like citation commands. Strictly speaking, what the module provides are wrappers for the commands of the main citation style. For example, the following command:

宏包的 \opt{mcite} 选项导入了一个特殊的引用模块,
提供了类似于 \sty{mcite}\slash\sty{mciteplus} 的引用命令。
严格地讲,该模块提供的是主引用样式命令的封装。
例如,以下命令:

\begin{ltxexample}
\mcite{key1,setA,*keyA1,*keyA2,*keyA3,key2,setB,*keyB1,*keyB2,*keyB3}
\end{ltxexample}
%
%is essentially equivalent to this:
本质上等价于

\begin{ltxexample}
\defbibentryset{setA}{keyA1,keyA2,keyA3}%
\defbibentryset{setB}{keyB1,keyB2,keyB3}%
\cite{key1,setA,key2,setB}
\end{ltxexample}
%
%The \cmd{mcite} command will work with any style since the \cmd{cite} backend command is controlled by the main citation style as usual. The \texttt{mcite} module provides wrappers for the standard commands in \secref{use:cit:std,use:cit:cbx}. See \tabref{use:cit:mct:tab2} for an overview. Pre and postnotes as well as starred variants of all commands are also supported. The parameters will be passed to the backend command. For example:
由于 \cmd{cite} 后端命令和通常一样由主引用样式控制,因此 \cmd{mcite} 命令可以和任何样式兼容。
\texttt{mcite} 模块提供了 \secref{use:cit:std,use:cit:cbx} 中标准命令的封装,
见\tabref{use:cit:mct:tab2}。
它也支持前注和后注以及所有命令带星号的版本,
其参数会传递到后端命令中。例如:

\begin{ltxexample}
\mcite*[pre][post]{setA,*keyA1,*keyA2,*keyA3}
\end{ltxexample}
%
%will execute:
会执行:

\begin{ltxexample}
\defbibentryset{setA}{keyA1,keyA2,keyA3}%
\cite*[pre][post]{setA}
\end{ltxexample}
%
%Note that the \texttt{mcite} module is not a compatibility module. It provides commands which are very similar but not identical in syntax and function to \sty{mcite}'s commands. When migrating from \sty{mcite}\slash\sty{mciteplus} to \biblatex, legacy files must be updated. With \sty{mcite}, the first member of the citation group is also the identifier of the group as a whole. Borrowing an example from the \sty{mcite} manual, this group:
请注意,\texttt{mcite} 模块不是完全兼容的。
它提供的命令与 \sty{mcite} 宏包的命令在语法和功能上十分相似但并不完全等价。
当从 \sty{mcite}\slash\sty{mciteplus} 迁移到 \biblatex 时必须更新旧文件。
在 \sty{mcite} 中,引用组的第一个成员也是该组整个的标识符。
举个 \sty{mcite} 手册中的例子,以下的组:

\begin{table}
\tablesetup
\begin{tabular}{@{}V{0.5\textwidth}@{}V{0.5\textwidth}@{}}
\toprule
%\multicolumn{1}{@{}H}{Standard Command} &
%\multicolumn{1}{@{}H}{\sty{mcite}-like Command} \\
\multicolumn{1}{@{}H}{标准命令} &
\multicolumn{1}{@{}H}{\sty{mcite} 命令} \\
\cmidrule(r){1-1}\cmidrule{2-2}
|\cite|		& |\mcite| \\
|\Cite|		& |\Mcite| \\
|\parencite|	& |\mparencite| \\
|\Parencite|	& |\Mparencite| \\
|\footcite|	& |\mfootcite| \\
|\footcitetext|	& |\mfootcitetext| \\
|\textcite|	& |\mtextcite| \\
|\Textcite|	& |\Mtextcite| \\
|\supercite|	& |\msupercite| \\
\bottomrule
\end{tabular}
%\caption{\sty{mcite}-like commands}
\caption{类 \sty{mcite} 命令}
\label{use:cit:mct:tab1}
\end{table}

\begin{ltxexample}
\cite{<<glashow>>,*salam,*weinberg}
\end{ltxexample}
%
%consists of three entries and the entry key of the first one also serves as identifier of the entire group. In contrast to that, a \biblatex entry set is an entity in its own right. Therefore, it requires a unique entry key which is assigned to the set as it is defined:
包括三个条目,且第一个条目键也同样是整个组的标识符。
与此相反,\biblatex 条目集是一个实体。因此,它需要分配给该集的唯一的条目键值:

\begin{ltxexample}
\mcite{<<set1>>,*glashow,*salam,*weinberg}
\end{ltxexample}
%
%Once defined, an entry set is handled like any regular entry in a \file{bib} file. When using one of the \texttt{numeric} styles which ship with \texttt{biblatex} and activating its \opt{subentry} option, it is even possible to refer to set members. See \tabref{use:cit:mct:tab2} for some examples. Restating the original definition of the set is redundant, but permissible. In contrast to \sty{mciteplus}, however, restating a part of the original definition is invalid. Use the entry key of the set instead.
一旦定义之后,条目集的处理与其它 \file{bib} 文件中的常规条目相同。
当在 \sty{biblatex} 中使用 \texttt{numeric} 样式并且启动了 \opt{subentry} 选项时,甚至可以指向集成员。
参考\tabref{use:cit:mct:tab2} 中的一些例子。
也可以重启条目集的原始定义,但这没有必要。
不过与 \sty{mciteplus} 不同的是,重启原始定义的一部分是无效的,需要使用集合的条目键。

\begin{table}
\tablesetup
\begin{tabular}{@{}V{0.5\textwidth}@{}V{0.1\textwidth}@{}p{0.4\textwidth}@{}}
\toprule
%\multicolumn{1}{@{}H}{Input} &
%\multicolumn{1}{@{}H}{Output} &
%\multicolumn{1}{@{}H}{Comment} \\
\multicolumn{1}{@{}H}{输入} &
\multicolumn{1}{@{}H}{输出} &
\multicolumn{1}{@{}H}{评注} \\
\cmidrule(r){1-1}\cmidrule(r){2-2}\cmidrule{3-3}
%|\mcite{set1,*glashow,*salam,*weinberg}|& [1]	& Defining and citing the set \\
%|\mcite{set1}|				& [1]	& Subsequent citation of the set \\
%|\cite{set1}|				& [1]	& Regular |\cite| works as usual \\
%|\mcite{set1,*glashow,*salam,*weinberg}|& [1]	& Redundant, but permissible \\
%|\mcite{glashow}|			& [1a]	& Citing a set member \\
%|\cite{weinberg}|			& [1c]	& Regular |\cite| works as well \\
|\mcite{set1,*glashow,*salam,*weinberg}|& [1]	& 定义并引用该集合 \\
|\mcite{set1}|				& [1]	& 该集合随后的引用 \\
|\cite{set1}|				& [1]	& 常规的 |\cite| \\
|\mcite{set1,*glashow,*salam,*weinberg}|& [1]	& 冗余,但是允许的 \\
|\mcite{glashow}|			& [1a]	& 引用集成员 \\
|\cite{weinberg}|			& [1c]	& 又一次常规的 |\cite| \\
\bottomrule
\end{tabular}
%\caption[\sty{mcite}-like syntax]
%{\sty{mcite}-like syntax (sample output with \kvopt{style}{numeric} and \opt{subentry} option)}
\caption[类\sty{mcite} 语法]{\sty{mcite} 语法(使用 \kvopt{style}{numeric} 和 \opt{subentry} 选项时的样例输出)}
\label{use:cit:mct:tab2}
\end{table}

\subsection{本地化命令}%\subsection{Localization Commands}
\label{use:lng}

%The \biblatex package provides translations for key terms such as <edition> or <volume> as well as definitions for language specific features such as the date format and ordinals. These definitions, which are loaded automatically, may be modified or extended in the document preamble or the configuration file with the commands introduced in this section.

\biblatex 宏包提供了“edition”和“volume”等关键词的翻译,
并定义日期格式和序数等语言相关的特征。
这些定义是自动导入的,可以通过本节所介绍的命令在导言区或配置文件中修改或扩展。

\begin{ltxsyntax}

\cmditem{DefineBibliographyStrings}{language}{definitions}

%This command is used to define localisation strings. The \prm{language} must be a language name known to the \sty{babel}/\sty{polyglossia} packages, \ie one of the identifiers listed in \tabref{bib:fld:tab1} on page \pageref{bib:fld:tab1}. The \prm{definitions} are \keyval pairs which assign an expression to an identifier:

该命令用于定义本地化字符串。
\prm{language} 选项必须是 \sty{babel}/\sty{polyglossia} 可知的语言名,
即 \pageref{bib:fld:tab1} 页的\tabref{bib:fld:tab1} 所列出的标识符。
\prm{definitions} 是 \keyval 对,将表达式分配给标识符:

\begin{ltxexample}
\DefineBibliographyStrings{american}{%
  bibliography  = {Bibliography},
  shorthands    = {Abbreviations},
  editor        = {editor},
  editors       = {editors},
}
\end{ltxexample}
%
%A complete list of all keys supported by default is given is \secref{aut:lng:key}. Note that all expressions should be capitalized as they usually are when used in the middle of a sentence. The \biblatex package will automatically capitalize the first word when required at the beginning of a sentence. Expressions intended for use in headings should be capitalized in a way that is suitable for titling. In contrast to \cmd{DeclareBibliographyStrings}, \cmd{DefineBibliographyStrings} overrides both the full and the abbreviated version of the string. See \secref{aut:lng:cmd} for further details.
默认支持的所有关键字列表在 \secref{aut:lng:key} 节中给出。
请注意,所有的表达式在句子中间时也要像本来一样首字母大写。
\biblatex 宏包会在必要时自动在句首处将首字母大写。
用于标题的表达式的大写方式要与标题匹配。
与 \cmd{DeclareBibliographyStrings} 相反,\cmd{DefineBibliographyStrings} 覆盖了字符串及其缩写两个版本。
详见 \secref{aut:lng:cmd} 节。

\cmditem{DefineBibliographyExtras}{language}{code}

%This command is used to adapt language specific features such as the date format and ordinals. The \prm{language} must be a language name known to the \sty{babel}/\sty{polyglossia} packages. The \prm{code}, which may be arbitrary \latex code, will usually consist of redefinitions of the formatting commands from \secref{use:fmt:lng}.

该命令用于调整日期格式和序数等语言相关的特征。
\prm{language} 必须是 \sty{babel}/\sty{polyglossia} 可知的语言名。
\prm{code} 可以是任意代码,通常包括 \secref{use:fmt:lng} 中格式命令的重定义。

\cmditem{UndefineBibliographyExtras}{language}{code}

%This command is used to restore the original definition of any commands modified with \cmd{DefineBibliographyExtras}. If a redefined command is included in \secref{use:fmt:lng}, there is no need to restore its previous definition since these commands are adapted by all language modules anyway.

该命令用于存储被 \cmd{DefineBibliographyExtras} 修改的命令的原始定义。
如果重定义的命令在 \secref{use:fmt:lng} 中,
那么没有必要存储之前的定义,因为这些命令总归会被所有的语言模块调整的。

\cmditem{DefineHyphenationExceptions}{language}{text}

%This is a \latex frontend to \tex's \cmd{hyphenation} command which defines hyphenation exceptions.
The \prm{language} must be a language name known to the \sty{babel}/\sty{polyglossia} packages. The \prm{text} is a whitespace"=separated list of words. Hyphenation points are marked with a dash:

这是 \TeX{} 的 \cmd{hyphenation} 命令的 \LaTeX 前端,用于定义了断词例外。
\prm{language} 必须是 \sty{babel}/\sty{polyglossia} 可知的语言名。
\prm{text} 是空格分开的单词列表,断词点用短横线标记:

\begin{ltxexample}
\DefineHyphenationExceptions{american}{%
  hy-phen-ation ex-cep-tion
}
\end{ltxexample}

\cmditem{NewBibliographyString}{key}

%This command declares new localisation strings, \ie it initializes a new \prm{key} to be used in the
\prm{definitions} of \cmd{DefineBibliographyStrings}. The \prm{key} argument may also be a comma"=separated list of key names. The keys listed in \secref{aut:lng:key} are defined by default.
该命令声明了新的本地化字符串,即初始化新的 \prm{key},用在命令 \cmd{DefineBibliographyStrings} 的 \prm{definitions} 中。选项 \prm{key} 也可以是逗号分隔的键值名列表。默认的键名在 \secref{aut:lng:key} 节中列出。
\end{ltxsyntax}

\subsection{Entry Querying Commands}
\label{use:eq}
The commands in this section are user-facing equivalents of the identically-named commands in section \secref{aut:aux:tst}. They can be used to test for the presence and attributes of specific bibliography entries. See section \secref{aut:aux:tst} for usage.

\begin{ltxsyntax}
\cmditem{ifentryseen}{entrykey}{true}{false}
\cmditem{ifentryinbib}{entrykey}{true}{false}
\cmditem{ifentrycategory}{entrykey}{category}{true}{false}
\cmditem{ifentrykeyword}{entrykey}{keyword}{true}{false}
\end{ltxsyntax}


\subsection{格式命令}%\subsection{Formatting Commands}
\label{use:fmt}

%The commands and facilities presented in this section may be used to adapt the format of citations and the bibliography.

本节介绍的命令和工具可以用于调整引用和参考文献的格式。

\subsubsection{一般命令和钩子} %\subsubsection{Generic Commands and Hooks}
\label{use:fmt:fmt}

%The commands in this section may be redefined with \cmd{renewcommand} in the document preamble. Those marked as <Context Sensitive> in the margin can also be customised with \cmd{DeclareDelimFormat} and are printed with \cmd{printdelim} (\secref{use:fmt:csd}). Note that all commands starting with \cmd{mk\dots} take one argument. All of these commands are defined in \path{biblatex.def}.

本小节的命令可以在导言区用 \cmd{renewcommand} 重定义。
页边标记为“Context Sensitive”的命令还可以用 \cmd{DeclareDelimFormat} 进行定制,
并使用 \secref{use:fmt:csd} 节的 \cmd{printdelim} 打印。
请注意,所有以 \cmd{mk\dots} 开头的命令都带有一个选项。
所有这些命令的定义在 \path{biblatex.def} 中。

\begin{ltxsyntax}

\csitem{bibsetup}
%Arbitrary code to be executed at the beginning of the bibliography, intended for commands which affect the layout of the bibliography.
在参考文献开始执行的任意代码,用于影响参考文献的页面布局的命令。

\csitem{bibfont}
%Arbitrary code setting the font used in the bibliography. This is very similar to \cmd{bibsetup} but intended for switching fonts.
在参考文献中用于设置字体的任意代码。
类似于 \cmd{bibsetup} 但用于切换字体。

\csitem{citesetup}
%Arbitrary code to be executed at the beginning of each citation command.
在引用命令开始执行的任意代码。

\csitem{newblockpunct}
%The separator inserted between <blocks> in the sense explained in \secref{aut:pct:new}. The default definition is controlled by the package option \opt{block} (see \secref{use:opt:pre:gen}).
插入在 <blocks>(其意义在 \secref{aut:pct:new} 节中解释)之间的分隔符。
缺省定义由宏包选项 \opt{block}(见 \secref{use:opt:pre:gen})所控制。

\csitem{newunitpunct}
%The separator inserted between <units> in the sense explained in \secref{aut:pct:new}. This will usually be a period or a comma plus an interword space. The default definition is a period and a space.
插入在 <units>(其意义在\secref{aut:pct:new} 节中解释)之间的分隔符。
这通常是句号或者逗号加上一个词间距。
缺省定义是句号加一个空格。

\csitem{finentrypunct}
%The punctuation printed at the very end of every bibliography entry, usually a period. The default definition is a period.
在每条文献条目最后打印的标点,通常是句号。
缺省定义是句号。

\csitem{entrysetpunct}
%The punctuation printed between bibliography subentries of an entry set. The default definition is a semicolon and a space.
一个条目集中文献子条目之间的标点。
默认定义是分号和一个空格。

\csitem{bibnamedelima}
%This delimiter controls the spacing between the elements which make up a name part. It is inserted automatically after the first name element if the element is less than three characters long and before the last element. The default definition is an interword space penalized by the value of the \cnt{highnamepenalty} counter (\secref{use:fmt:len}). Please refer to \secref{use:cav:nam} for further details.
该定界符控制组成姓名成分之间的空白。
它将自动插入在 first name 之后(如果该成分少于三个字符长度)和最后的姓名成分之前。
缺省值是一个词间距加上由计数器 \cnt{highnamepenalty}(\secref{use:fmt:len})的值控制的惩罚项。
详见 \secref{use:cav:nam} 节。

\csitem{bibnamedelimb}
%This delimiter is inserted between the elements which make up a name part where \cmd{bibnamedelima} does not apply. The default definition is an interword space penalized by the value of the \cnt{lownamepenalty} counter (\secref{use:fmt:len}). Please refer to \secref{use:cav:nam} for further details.
该定界符插入在 \cmd{bibnamedelima} 不可用的姓名成分之间。
缺省值是一个词间距加上由计数器 \cnt{lownamepenalty}(\secref{use:fmt:len})的值控制的惩罚项。
详见 \secref{use:cav:nam} 节。

\csitem{bibnamedelimc}
%This delimiter controls the spacing between name parts. It is inserted between the name prefix and the last name if \kvopt{useprefix}{true}. The default definition is an interword space penalized by the value of the \cnt{highnamepenalty} counter (\secref{use:fmt:len}). Please refer to \secref{use:cav:nam} for further details.
该定界符控制姓名成分之间的空白。
如果 \kvopt{useprefix}{true},那么它插入在名前缀(name prefix)和姓(last name)之间。
缺省值是一个词间距加上由计数器 \cnt{highnamepenalty}(\secref{use:fmt:len})的值控制的惩罚项。
详见 \secref{use:cav:nam} 节。

\csitem{bibnamedelimd}
%This delimiter is inserted between all name parts where \cmd{bibnamedelimc} does not apply. The default definition is an interword space penalized by the value of the \cnt{lownamepenalty} counter (\secref{use:fmt:len}). Please refer to \secref{use:cav:nam} for further details.
该定界符插入在所有的 \cmd{bibnamedelimc} 不可用的姓名成分之间。
缺省值是一个词间距加上由计数器 \cnt{lownamepenalty}(\secref{use:fmt:len})的值控制的惩罚项。
详见 \secref{use:cav:nam} 节。

\csitem{bibnamedelimi}
%This delimiter replaces \cmd{bibnamedelima/b} after initials. Note that this only applies to initials given as such in the \file{bib} file, not to the initials automatically generated by \biblatex which use their own set of delimiters.
该定界符在首字符缩写中取代 \cmd{bibnamedelima/b}。
请注意该情况只用于既定的首字母缩写(例如在 \file{bib} 文件中给出的),
而不是由 \biblatex 自动生成的首字母缩写,后者会使用自己的定界符集。

\csitem{bibinitperiod}
%The punctuation inserted after initials unless \cmd{bibinithyphendelim} applies. The default definition is a period (\cmd{adddot}). Please refer to \secref{use:cav:nam} for further details.
当没有使用 \cmd{bibinithyphendelim} 时插入在首字母缩写之后的标点。
缺省值是句点(\cmd{adddot})。
详见 \secref{use:cav:nam} 节。

\csitem{bibinitdelim}
%The spacing inserted between multiple initials unless \cmd{bibinithyphendelim} applies. The default definition is an unbreakable interword space. Please refer to \secref{use:cav:nam} for further details.
当没有使用 \cmd{bibinithyphendelim} 时多重首字母缩写之间的空白。
缺省值是一个不可打断的词间距。
详见 \secref{use:cav:nam} 节。

\csitem{bibinithyphendelim}
%The punctuation inserted between the initials of hyphenated name parts, replacing \cmd{bibinitperiod} and \cmd{bibinitdelim}. The default definition is a period followed by an unbreakable hyphen. Please refer to \secref{use:cav:nam} for further details.
带连字符的姓名成分首字母缩写之间插入的标点,用以替代 \cmd{bibinitperiod} 和 \cmd{bibinitdelim}。
缺省定义是一个句点后接一个不可打断的连字符。
详见 \secref{use:cav:nam} 节。

\csitem{bibindexnamedelima}
%Replaces \cmd{bibnamedelima} in the index.
在索引中代替 \cmd{bibnamedelima}。

\csitem{bibindexnamedelimb}
%Replaces \cmd{bibnamedelimb} in the index.
在索引中代替 \cmd{bibnamedelimb}。

\csitem{bibindexnamedelimc}
%Replaces \cmd{bibnamedelimc} in the index.
在索引中代替 \cmd{bibnamedelimc}。

\csitem{bibindexnamedelimd}
%Replaces \cmd{bibnamedelimd} in the index.
在索引中代替 \cmd{bibnamedelimd}。

\csitem{bibindexnamedelimi}
%Replaces \cmd{bibnamedelimi} in the index.
在索引中代替 \cmd{bibnamedelimi}。

\csitem{bibindexinitperiod}
%Replaces \cmd{bibinitperiod} in the index.
在索引中代替 \cmd{bibinitperiod}。

\csitem{bibindexinitdelim}
%Replaces \cmd{bibinitdelim} in the index.
在索引中代替 \cmd{bibinitdelim}。

\csitem{bibindexinithyphendelim}
%Replaces \cmd{bibinithyphendelim} in the index.
在索引中代替 \cmd{bibinithyphendelim}。

\csitem{revsdnamepunct}
%The punctuation to be printed between the first and last name parts when a name is reversed. Here is an example showing a name with the default comma as \cmd{revsdnamedelim}:
当名字反写(姓在前)时姓和名之间的标点。
如下是一个使用缺省的逗号作为 \cmd{revsdnamedelim} 的例子:

\begin{ltxexample}
Jones<<,>> Edward
\end{ltxexample}

%This command should be used with \cmd{bibnamedelimd} as a reversed-name separator in formatting directives for name lists. Please refer to \secref{use:cav:nam} for further details.
对于姓名列表,该命令应当与格式指令中的姓名反写分隔符 \cmd{bibnamedelimd} 一起使用。
详见 \secref{use:cav:nam} 节。

\csitem{bibnamedash}
%The dash to be used as a replacement for recurrent authors or editors in the bibliography. The default is an <em> or an <en> dash, depending on the indentation of the list of references.
用于替代文献中连续重复的作者或编辑的破折号。
缺省值是一个 <em> 或 <en> 长度横线,取决于文献列表的缩进。

\csitem{labelnamepunct}
%The separator printed after the name used for alphabetizing in the bibliography (\bibfield{author} or \bibfield{editor}, if the \bibfield{author} field is undefined). With the default styles, this separator replaces \cmd{newunitpunct} at this location. The default definition is \cmd{newunitpunct}, \ie it is not handled differently from regular unit punctuation.
在参考文献按字母排序时用在名字之后的分隔符(\bibfield{author},在其未定义时是 \bibfield{editor})。
对于缺省样式,该分隔符在此位置代替 \cmd{newunitpunct}。
缺省定义是 \cmd{newunitpunct},即对其处理与常规单元标点没有不同之处。

\csitem{subtitlepunct}
%The separator printed between the fields \bibfield{title} and \bibfield{subtitle}, \bibfield{booktitle} and \bibfield{booksubtitle}, as well as \bibfield{maintitle} and \bibfield{mainsubtitle}. With the default styles, this separator replaces \cmd{newunitpunct} at this location. The default definition is \cmd{newunitpunct}, \ie it is not handled differently from regular unit punctuation.
域 \bibfield{title} 和 \bibfield{subtitle}、\bibfield{booktitle} 和 \bibfield{booksubtitle},以及 \bibfield{maintitle} 和 \bibfield{mainsubtitle} 之间的分隔符。
对于缺省样式,该分隔符在此位置代替 \cmd{newunitpunct}。
缺省定义是 \cmd{newunitpunct},即对其处理与常规单元标点没有不同之处。

\csitem{intitlepunct}
%The separator between the word «in» and the following title in entry types such as \bibtype{article}, \bibtype{inbook}, \bibtype{incollection}, etc. The default definition is a colon plus an interword space (\eg «Article, in: \emph{Journal}» or «Title, in: \emph{Book}»). Note that this is the separator string, not only the punctuation mark. If you don't want a colon after «in», \cmd{intitlepunct} should still insert a space.
在 \bibtype{article}, \bibtype{inbook}, \bibtype{incollection} 等条目类型中单词“in”与之后的标题之间的分隔符。
缺省值是一个冒号加上一个词间距(例如,“Article, in: \emph{Journal}” 或者“Title, in: \emph{Book}”)。
请注意,这是分隔字符串而不仅是标点。
如果你不想在“in”后加冒号,\cmd{intitlepunct} 仍然应当插入一个空格。

\csitem{bibpagespunct}
%The separator printed before the \bibfield{pages} field. The default is a comma plus an interword space.
域 \bibfield{pages} 之前的分隔符。
缺省值是逗号加上一个词间距。

\csitem{bibpagerefpunct}
%The separator printed before the \bibfield{pageref} field. The default is an interword space.
域 \bibfield{pageref} 之前的分隔符。
缺省值是一个词间距。

\csitem{multinamedelim}\CSdelimMark
%The delimiter printed between multiple items in a name list like \bibfield{author} or \bibfield{editor} if there are more than two names in the list. The default is a comma plus an interword space. See \cmd{finalnamedelim} for an example.\footnote{Note that \cmd{multinamedelim} is not used at all if there are only two names in the list. In this case, the default styles use the \cmd{finalnamedelim}.}
在 \bibfield{author} 或 \bibfield{editor} 等姓名列表中各项之间的定界符(如果有两个以上姓名)。
缺省值是一个逗号加上一个词间距。
例子请参考 \cmd{finalnamedelim}。\footnote{%
	请注意,如果列表中只有两个姓名,那么不会使用 \cmd{multinamedelim},此时缺省样式使用 \cmd{finalnamedelim}。
}

\csitem{finalnamedelim}\CSdelimMark
%The delimiter printed instead of \cmd{multinamedelim} before the final name in a name list. The default is the localised term <and>, separated by interword spaces. Here is an example:
在姓名列表的最后一个姓名之前用以代替 \cmd{multinamedelim} 的定界符。
缺省值是用词间距分隔的本地化项“and”。这里是一个例子:

\begin{ltxexample}
Michel Goossens<<,>> Frank Mittelbach <<and>> Alexander Samarin
Edward Jones <<and>> Joe Williams
\end{ltxexample}
%
%The comma in the first example is the \cmd{multinamedelim} whereas the string <and> in both examples is the \cmd{finalnamedelim}. See also \cmd{finalandcomma} in \secref{use:fmt:lng}.
第一个例子中的逗号是 \cmd{multinamedelim},而这两个例子中的字符串“and”是 \cmd{finalnamedelim}。
另见 \secref{use:fmt:lng} 节中的 \cmd{finalandcomma}。

\csitem{revsdnamedelim}\CSdelimMark
%An extra delimiter printed after the first name in a name list if the first name is reversed. The default is an empty string, \ie no extra delimiter will be printed. Here is an example showing a name list with a comma as \cmd{revsdnamedelim}:
当姓名反序时打印在名(first name)后面的额外定界符。
缺省值是空字符串,即没有额外的定界符。
这里是一个设置 \cmd{revsdnamedelim} 为逗号的姓名列表例子:

\begin{ltxexample}
Jones, Edward<<, and>> Joe Williams
\end{ltxexample}
%
%In this example, the comma after <Edward> is the \cmd{revsdnamedelim} whereas the string <and> is the \cmd{finalnamedelim}, printed in addition to the former.
在本例中,“Edward”之后的逗号是 \cmd{revsdnamedelim} 而字符串“and”是前者之后的 \cmd{finalnamedelim}。

\csitem{andothersdelim}\CSdelimMark
%The delimiter printed before the localisation string <\texttt{andothers}> if a name list like \bibfield{author} or \bibfield{editor} is truncated. The default is an interword space.
当 \bibfield{author} 或 \bibfield{editor} 等姓名列表被截断时本地化字符串“\texttt{andothers}” 之前的定界符。
缺省值是一个词间距。

\csitem{multilistdelim}\CSdelimMark
%The delimiter printed between multiple items in a literal list like \bibfield{publisher} or \bibfield{location} if there are more than two items in the list. The default is a comma plus an interword space. See \cmd{multinamedelim} for further explanation.
在 \bibfield{publisher} 或 \bibfield{location} 等文本列表中诸项之间的定界符(如果列表中有两个以上项)。
缺省值是一个逗号加上一个词间距。
进一步解释参见 \cmd{multinamedelim}。

\csitem{finallistdelim}\CSdelimMark
%The delimiter printed instead of \cmd{multilistdelim} before the final item in a literal list. The default is the localised term <and>, separated by interword spaces. See \cmd{finalnamedelim} for further explanation.
在文本列表的最后一项之前代替 \cmd{multilistdelim} 的定界符。
缺省值是用词间距分隔的本地化字符串“and”。
进一步解释参见 \cmd{finalnamedelim}。

\csitem{andmoredelim}\CSdelimMark
%The delimiter printed before the localisation string <\texttt{andmore}> if a literal list like \bibfield{publisher} or \bibfield{location} is truncated. The default is an interword space.
当 \bibfield{publisher} 或 \bibfield{location} 等文本列表被截断时打印在本地化字符串“\texttt{andmore}”之前的定界符。
缺省值是一个词间距。

\csitem{multicitedelim}
%The delimiter printed between citations if multiple entry keys are passed to a single citation command. The default is a semicolon plus an interword space.
当多个条目键传递给单个引用命令时打印在引用之间的定界符。
缺省值是一个分号加上一个词间距。

\csitem{supercitedelim}
%Similar to \cmd{multicitedelim}, but used by the \cmd{supercite} command only. The default is a comma.
类似于 \cmd{multicitedelim} 但只用在 \cmd{supercite} 中。缺省值是一个逗号。

\csitem{compcitedelim}
%Similar to \cmd{multicitedelim}, but used by certain citation styles when <compressing> multiple citations. The default definition is a comma plus an interword space.
类似于 \cmd{multicitedelim} 但只用于某些“压缩”的多重引用样式。
缺省值是一个逗号加上一个词间距。

\csitem{datecircadelim}\CSdelimMark
%When formatting dates with the global option \opt{datecirca} enabled, the delimiter printed after any localised <circa> term. Defaults to interword space.
开启全局选项 \opt{datecirca} 时,在日期格式中本地化“circa”项之后的定界符。
缺省值是一个词间距。

\csitem{dateeradelim}\CSdelimMark
%When formatting dates with the global option \opt{dateera} set, the delimiter printed before the localisation era term. Defaults to interword space.
设置全局选项 \opt{dateera} 时,在日期格式中本地化纪年项之前的定界符。
缺省值是一个词间距。

\csitem{dateuncertainprint}
%Prints date uncertainty information when the global option \opt{dateuncertain} is enabled and the \cmd{ifdateuncertain} test is true. By default, prints the language specific \cmd{bibdateuncertain} string (\secref{use:fmt:lng}).
开启全局选项 \opt{dateuncertain} 并且 \cmd{ifdateuncertain} 测试为真时打印的日期不确定信息。
缺省打印语言相关的 \cmd{bibdateuncertain} 字符串(\secref{use:fmt:lng} 节)。

\csitem{enddateuncertainprint}
%Prints date uncertainty information when the global option \opt{dateuncertain} is enabled and the \cmd{ifenddateuncertain} test is true. By default, prints the language specific \cmd{bibdateuncertain} string (\secref{use:fmt:lng}).
开启全局选项 \opt{dateuncertain} 并且 \cmd{ifenddateuncertain} 测试为真时打印的日期不确定信息。
缺省打印语言相关的 \cmd{bibdateuncertain} 字符串(\secref{use:fmt:lng} 节)。

\csitem{datecircaprint}
%Prints date circa information when the global option \opt{datecirca} is enabled and the \cmd{ifdatecirca} test is true. By default, prints the <circa> localised term (\secref{aut:lng:key:dt}) and the \opt{datecircadelim} delimiter.
开启全局选项 \opt{datecirca} 并且 \cmd{ifdatecirca} 测试为真时打印日期约数信息。
缺省打印本地化的“circa”项(\secref{aut:lng:key:dt} 节)和 \opt{datecircadelim} 定界符。

\csitem{enddatecircaprint}
%Prints date circa information when the global option \opt{datecirca} is enabled and the \cmd{ifenddatecirca} test is true. By default, prints the <circa> localised term (\secref{aut:lng:key:dt}) and the \opt{datecircadelim} delimiter.
开启全局选项 \opt{datecirca} 并且 \cmd{ifenddatecirca} 测试为真时打印日期约数信息。
缺省打印本地化的“circa”项(\secref{aut:lng:key:dt} 节)和 \opt{datecircadelim} 定界符。

\csitem{datecircaprintiso}
Prints \acr{ISO8601-2} format date circa information when the global option \opt{datecirca} is enabled and the \cmd{ifdatecirca} test is true. Prints \cmd{textasciitilde}.
开启全局选项 \opt{datecirca} 并且 \cmd{ifdatecirca} 测试为真时打印EDTF格式的日期约数信息。
缺省打印 \cmd{textasciitilde}。

\csitem{enddatecircaprintiso}
Prints \acr{ISO8601-2} format date circa information when the global option \opt{datecirca} is enabled and the \cmd{ifenddatecirca} test is true. Prints \cmd{textasciitilde}.
开启全局选项 \opt{datecirca} 并且 \cmd{ifenddatecirca} 测试为真时打印EDTF格式的日期约数信息。
缺省打印 \cmd{textasciitilde}。

\csitem{dateeraprint}{yearfield}
%Prints date era information when the global option \opt{dateera} is set to <secular> or <christian>. By default, prints the \opt{dateeradelim} delimiter and the appropriate localised era term (\secref{aut:lng:key:dt}). If the \opt{dateeraauto} option is set, then the passed \prm{yearfield} (which is the name of a year field such as <year>, <origyear>, <endeventyear> etc.) is tested to see if its value is earlier than the \opt{dateeraauto} threshold and if so, then the BCE/CE localisation will be output too. The default setting for \opt{dateeraauto} is 0 and so only BCE/BC localisation strings are candidates for output. Detects whether the start or end year era information is to be printed by looking at the \prm{yearfield} name passed to it.
当全局选项 \opt{dateera} 设置为 \opt{secular} 或 \opt{christian} 时打印的日期纪年信息。
缺省情况下打印 \opt{dateeradelim} 定界符和合适的本地化纪年词语(\secref{aut:lng:key:dt} 节)。
如果设置了 \opt{dateeraauto},那么将测试传递过来的 \prm{yearfield}
(\opt{year}、\opt{origyear}、\opt{endeventyear} 等年份域的名称)
以确定对应的值是否早于 \opt{dateeraauto} 阈值。
如果更早的话还会输出本地化的 BCE/CE 词语。
\opt{dateeraauto} 的缺省设置为0,这样只会输出 BCE/BC 本地化字符串。
通过探测传递来的 \prm{yearfield} 名称确定是否打印开始和结束年份的纪年信息。

\csitem{dateeraprintpre}
%Prints date era information when the global option \opt{dateera} is set to <astronomical>. By default, prints \opt{bibdataeraprefix}. Detects whether the start or end year era information is to be printed by looking at the \prm{yearfield} name passed to it.
当全局选项 \opt{dateera} 设置为 \opt{astronomical} 时打印的日期约数信息。
缺省值是 \opt{bibdataeraprefix}。
通过探测传递来的 \prm{yearfield} 名称确定是否打印开始和结束年份的纪年信息。

\csitem{textcitedelim}
%Similar to \cmd{multicitedelim}, but used by \cmd{textcite} and related commands (\secref{use:cit:cbx}). The default is a comma plus an interword space. The standard styles modify this provisional definition to ensure that the delimiter before the final citation is the localised term <and>, separated by interword spaces. See also \cmd{finalandcomma} and \cmd{finalandsemicolon} in \secref{use:fmt:lng}.
类似于 \cmd{multicitedelim} 但用于 \cmd{textcite} 和相关命令(\secref{use:cit:cbx} 节)。
缺省值是一个逗号加上一个词间距。
标准样式会修改这一临时定义以确保最后一个引用之前的定界符是词间距分隔的本地化字符串“and”。
另见 \secref{use:fmt:lng} 节的 \cmd{finalandcomma} 和 \cmd{finalandsemicolon}。

\csitem{nametitledelim}\CSdelimMark
%The delimiter printed between the author\slash editor and the title by author-title and some verbose citation styles. The default definition inside bibliographies is \cmd{labelnamepunct} and is a comma plus an interword space otherwise.
作者---标题和其它一些详细引用样式中作者/编辑和标题之间的定界符。
缺省定义是一个逗号加上一个词间距。

\csitem{nameyeardelim}\CSdelimMark
%The delimiter printed between the author\slash editor and the year by author-year citation styles. The default definition is an interword space.
作者---年份引用样式中作者/编辑和年份之间的定界符。缺省定义是一个词间距。

\csitem{namelabeldelim}\CSdelimMark
%The delimiter printed between the name\slash title and the label by alphabetic and numeric citation styles. The default definition is an interword space.
字母样式和数值样式中姓名\slash 标题和标签之间的定界符。
缺省定义是一个词间距。

\csitem{nonameyeardelim}\CSdelimMark
%The delimiter printed between the substitute for the labelname when it does not exist (usually the label or title in standard styles) and the year in author-year citation styles. This is only used when there is no labelname since when the labelname exists, \cmd{nameyeardelim} is used. The default definition is an interword space.
在作者---年份引用样式中,当某一标签名不存在(标准样式中通常是标签或者标题)时其替代者与年份之间的定界符。
仅当没有标签名时使用。这是因为标签名存在时会使用 \cmd{nameyeardelim}。缺省值是一个词间距。
\csitem{authortypedelim}\CSdelimMark
The delimiter printed between the author and the \texttt{authortype}.

\csitem{editortypedelim}\CSdelimMark
The delimiter printed between the editor and the \texttt{editor} or \texttt{editortype} string.

\csitem{translatortypedelim}\CSdelimMark
The delimiter printed between the translator and the \texttt{translator} string.

\csitem{labelalphaothers}
%A string to be appended to the non"=numeric portion of the \bibfield{labelalpha} field (\ie the field holding the citation label used by alphabetic citation styles) if the number of authors\slash editors exceeds the \opt{maxalphanames} threshold or the \bibfield{author}\slash \bibfield{editor} list was truncated in the \file{bib} file with the keyword <\texttt{and others}>. This will typically be a single character such as a plus sign or an asterisk. The default is a plus sign. This command may also be redefined to an empty string to disable this feature. In any case, it must be redefined in the preamble.
当作者/编辑数目超过了 \opt{maxalphanames} 阈值或者 \bibfield{author}\slash \bibfield{editor} 列表在 \file{bib} 文件中由关键词“\texttt{and others}”截断时,
接在 \bibfield{labelalpha} 域的非数值部分之后的字符串(即,该域包含了由字母引用样式使用的引用标签)。
这通常是加号或者星号等单字符。缺省值是加号。
该命令可以通过重定义为空字符串来关闭该特性。
任何情况下必须在导言区中重定义。

\csitem{sortalphaothers}
%Similar to \cmd{labelalphaothers} but used in the sorting process. Setting it to a different value is advisable if the latter contains formatting commands, for example:
类似于 \cmd{labelalphaothers} 但使用在排序过程中。
如果 \cmd{labelalphaothers} 中包含了格式命令,建议将本命令设置为另外一个值,例如:

\begin{ltxexample}
\renewcommand*{\labelalphaothers}{\textbf{+}}
\renewcommand*{\sortalphaothers}{+}
\end{ltxexample}
%
%If \cmd{sortalphaothers} is not redefined, it defaults to \cmd{labelalphaothers}.
如果 \cmd{sortalphaothers} 没有定义,其缺省值为 \cmd{labelalphaothers}。

\csitem{prenotedelim}
%The delimiter printed after the \prm{prenote} argument of a citation command. See \secref{use:cit} for details. The default is an interword space.
引用命令的 \prm{prenote} 选项之后打印的定界符。
详见 \secref{use:cit} 节。缺省值是一个词间距。

\csitem{postnotedelim}
%The delimiter printed before the \prm{postnote} argument of a citation command. See \secref{use:cit} for details. The default is a comma plus an interword space.
引用命令的 \prm{postnote} 选项之前打印的定界符。
详见 \secref{use:cit} 节。缺省值是一个词间距。

\csitem{extpostnotedelim}
%The delimiter printed between the citation and the parenthetical \prm{postnote} argument of a citation command when the postnote occurs outside of the citation parentheses. In the standard styles, this occurs when the citation uses the shorthand field of the entry. See \secref{use:cit} for details. The default is an interword space.
在引用命令中,当后注出现在引用括号外时,引用和带括号的 \prm{postnote} 选项之间的定界符。
在标准样式中,如果引用使用条目的shorthand域就会出现这一现象。
详见 \secref{use:cit} 节。
缺省值是一个词间距。

\cmditem{mkbibname\prm{namepart}}{text}
%This command, which takes one argument, is used to format the name part <namepart> of name list fields. The default datamodel defines the name parts <family>, <given>, <prefix> and <suffix> and therefore the following macros are automatically defined:
该命令带有一个选项,用于姓名列表域的姓名成分 \prm{namepart} 的格式。
默认数据模型定义了姓名成分 <family>, <given>, <prefix> 和 <suffix>,
因此自动定义了以下宏命令:

\begin{ltxexample}
\mkbibnamefamily
\mkbibnamegiven
\mkbibnameprefix
\mkbibnamesuffix
\end{ltxexample}
%
%For backwards compatibility with the legacy \bibtex name parts, the following are also defined, will generate warnings and will set the correct macro:
出于对传统 \BibTeX 姓名成分的向后兼容性也定义以下宏命令。
这些宏会生成警告并设置正确的宏:

\begin{ltxexample}
\mkbibnamelast
\mkbibnamefirst
\mkbibnameaffix
\end{ltxexample}

\csitem{relatedpunct}
%The separator between the \bibfield{relatedtype} bibliography localisation string and the data from the first related entry. Here is an example with \cmd{relatedpunct} set to a dash:
在 \bibfield{relatedtype} 文献的本地化字符串和第一个相应条目数据之间的分隔符。
如下是将 \cmd{relatedpunct} 设置为短横线的例子:

\begin{ltxexample}
A. Smith. Title. 2000, (Orig. pub. as<<->>Origtitle)
\end{ltxexample}

\csitem{relateddelim}
%The generic separator between the data of multiple related entries. The default definition is an optional dot plus linebreak. Here is an example where volumes A-E are related entries of the 5 volume main work:
多重相关条目数据之间的分隔符。
缺省定义是一个可选的点号加上一个断行。
如下是一个五卷作品的例子,其中卷A-E是相关条目:

\begin{ltxexample}
Donald E. Knuth. Computers & Typesetting. 5 vols. Reading, Mass.: Addison-
Wesley, 1984-1986.
Vol. A: The TEXbook. 1984.
Vol. B: TEX: The Program. 1986.
Vol. C: The METAFONTbook. By. 1986.
Vol. D: METAFONT: The Program. 1986.
Vol. E: Computer Modern Typefaces. 1986.
\end{ltxexample}

\csitem{relateddelim\prm{relatedtype}}
%The separator between the data of multiple related entries inside related entries of type <relatedtype>. There is no default, if such a type-specific delimiter does not exist, \cmd{relateddelim} is used.
\prm{relatedtype} 类型的相关联条目中,多重关联条目数据之间的分隔符。
没有缺省值。
如果这样的类型相关的定界符不存在,那么将使用 \cmd{relateddelim}。

\end{ltxsyntax}

\subsubsection{Context-sensitive 定界符}% \subsubsection{Context-sensitive Delimiters}
\label{use:fmt:csd}
%The delimiters described in \secref{use:fmt:fmt} are globally defined. That is, no matter where you use them, they print the same thing. This is not necessarily desirable for delimiters which you might want to print different things in different contexts. Here <context> means things like <inside a text citation> or <inside a bibliography item>. For this reason, \biblatex\ provides a more sophisticated delimiter specification and user interface alongside the standard one based on normal macros defined with \cmd{newcommand}.
\secref{use:fmt:fmt} 节介绍的定界符是全局定义的。
也就是说,无论在哪里使用,结果都是相同的。
然而,如果想要在不同语境(context)中打印不同的效果,这些定界符就不可取了。
这里“context”可以是“引用文本内部”或者“参考文献项”等。
基于此,\biblatex 还提供了更为智能的定界符规范和用户接口(通过 \cmd{newcommand} 定义的常规宏)。

\begin{ltxsyntax}
\cmditem{DeclareDelimFormat}[context, \dots]{name, \dots}{code}
\cmditem*{DeclareDelimFormat}*[context, \dots]{name, \dots}{code}

%Declares the delimiter macros in the comma"=separated list \prm{names} with the replacement test \prm{code}. If the optional comma"=separated list of \prm{contexts} is given, declare the \prm{names} only for those contexts. \prm{names} defined without any \prm{contexts} behave just like the global delimiter definitions which \cmd{newcommand} gives---just a plain macro with a replacement which can be used as \cmd{name}. However, you can also call delimiter macros defined in this way by using \cmd{printdelim}, which is context-aware. The starred version clears all \prm{context} specific declarations for all \prm{names} first.

在逗号分隔列表 \prm{names} 中使用替代测试 \prm{code} 声明定界符宏。
如果给出了可选的逗号分隔列表 \prm{context},那么只为相应的 context 声明 \prm{names}。
如果没有可选的 \prm{context},那么被定义的 \prm{names} 与由 \cmd{newcommand} 给出的全局定界符的定义是一致的——
仅仅是一个简单的宏,用于替代 \cmd{name}。
不过,通过使用能探测 context 的 \cmd{printdelim} 命令,仍然可以调用以这种方式定义的定界符宏。
带星号的命令会首先清楚所有 \prm{names} 中的 \prm{context} 声明。

\cmditem{DeclareDelimAlias}{alias}{delim}
\cmditem*{DeclareDelimAlias}*[context, \dots]{alias}{delim}

Declares \prm{alias} to be an alias for the delimiter \prm{delim}. The assigment is valid for all existing contexts of \prm{alias}.
The starred version assigns the alias for the given \opt{contexts} only---if the optional argument is empty, assigment is for the global/empty context.


\cmditem{printdelim}[context]{name}

%Prints a delimiter with name \prm{name}, locally establishing a optional \prm{context} first. Without the optional \prm{context}, \cmd{printdelim} uses the currently active delimiter context.

打印名为 \prm{name} 的定界符,
首先建立可选的局部 \prm{context}。
如果没有可选的 \prm{context},那么 \cmd{printdelim} 会使用当前活动的定界符 context。

%Delimiter contexts are simply a string, the value of the internal macro \cmd{blx@delimcontext} which can be set manually by the command \cmd{delimcontext}

定界符文境是一个简单的字符串,内部宏 \cmd{blx@delimcontext} 的值。
后者可以通过 \cmd{delimcontext} 命令手动设置。

\cmditem{delimcontext}{context}

%Set the delimiter context to \prm{context}. This setting is not global so that delimiter contexts can be nested using the usual \latex group method.

设置定界符文境为 \prm{context}。
该设置不是全局的,因此定界符 context 可以使用通常的 \LaTeX 组方法进行嵌套。

\cmditem{DeclareDelimcontextAlias}{alias}{name}

%The context-sensitive delimiter system creates delimiter contexts based on
%the name of citation commands (<parencite>, <textcite> etc.) passed to
%\cmd{DeclareCiteCommand}. In certain cases where there are nested
%definitions of citation commands where \cmd{DeclareCiteCommand} calls
%itself (see the definition of \cmd{textcite} in \sty{authoryear-icomp}
%for example). The delimiter context is then usually incorrect and the
%delimiter specifications do not work. For example, the definition of
%\cmd{textcite} in fact defines and uses \cmd{cbx@textcite} and so the
%context is automatically set to \opt{cbx@textcite} when printing the
%citation. Delimiter definitions expecting to see the context \opt{textcite}
%therefore do not work. Therefore this command is provided for style authors
%which aliases the context \prm{alias} to the context \prm{name}. For
%example:

基于传递给 \cmd{DeclareCiteCommand} 的引用命令(“parencite”、“textcite”等)的名称,
定界符系统会创建定界符 context。
在某些情况下引用命令存在嵌套定义,此时 \cmd{DeclareCiteCommand} 会调用自己
(相关示例见 \sty{authoryear-icomp} 中 \cmd{textcite} 的定义)。
此时的 context 一般不准确,从而无法使用定界符规范。
例如,\cmd{textcite} 的定义实际上定义并使用了 \cmd{cbx@textcite},
因此当打印引用时 context 会自动设置为 \opt{cbx@textcite}。
因此期望 context 为 \opt{textcite} 的定界符定义就无法运行。
为此,该命令可以将 \prm{alias} 作为 \prm{name} 的别称,例如:

\begin{ltxexample}[style=latex]{}
\DeclareDelimcontextAlias{cbx@textcite}{textcite}
\end{ltxexample}
%
%This (which is a default setting), makes sure that when inside the
%\cmd{cbx@textcite} citation command, the context is in fact \opt{textcite}
%as expected.
这会确保 context 在 \cmd{cbx@textcite} 引用命令内部时就是预期的 \opt{textcite}。

\end{ltxsyntax}
%
%\biblatex\ has several default contexts which are established automatically in various places:
\biblatex 在不同位置自动创建了若干个默认的 context:

\begin{description}
	\item[none] %At begin document
	文档开始处。
	\item[bib] %Inside a bibliography begun with \cmd{printbibliography} or inside a \cmd{usedriver}
	以 \cmd{printbibliography} 开始的参考文献内部或者 \cmd{usedriver} 内部。
	\item[biblist] %Inside a bibliography list begun with \cmd{printbiblist}
	以 \cmd{printbiblist} 开始的参考文献列表内部
	\item[<citecommand>] %Inside a citation command \cmd{citecommand} defined with \cmd{DeclareCiteCommand}
	使用 \cmd{DeclareCiteCommand} 定义的 \cmd{citecommand} 引用命令内部
\end{description}

%For example, the defaults for \cmd{nametitledelim} are:
例如,\cmd{nametitledelim} 的默认设置为:

\begin{ltxexample}[style=latex]{}
\DeclareDelimFormat{nametitledelim}{\addcomma\space}
\DeclareDelimFormat[textcite]{nametitledelim}{\addspace}
\end{ltxexample}
%
%This means that \cmd{nametitledelim} is defined globally as <\cmd{addcomma}\cmd{space}> as per the standard delimiter interface. However, in addition, the delimiter can be printed using \cmd{printdelim} which would print the same as \cmd{nametitledelim} apart from inside a \cmd{textcite}, in which it would print \cmd{addspace} which is more suitable for running text. If desired, a context can be forced with the optional argument to \cmd{printdelim}, so
这意味着对于每个标准定界符接口,\cmd{nametitledelim} 全局定义为 \cmd{addcomma}\cmd{space}。
此外,总体上使用 \cmd{printdelim} 可以打印该定界符为 \cmd{nametitledelim},
不过在 \cmd{textcite} 内部会打印 \cmd{addspace},这对于行文更适合。
如果需要的话,可以强制添加 context 作为\cmd{printdelim} 的可选项,例如:

\begin{ltxexample}[style=latex]{}
\printdelim[textcite]{nametitledelim}
\end{ltxexample}
%
%Would print \cmd{addspace} regardless of the surrounding context of the \cmd{printdelim}. Contexts are just arbitrary strings and so you can establish them at any time, using \cmd{delimcontext}. If \cmd{printdelim} finds no special value for the delimiter \prm{name} in the current context, it simply prints \cmd{name}. This means that style authors can use \cmd{printdelim} and users expecting to be able to use \cmd{renewcommand} to redefine delimiters can do so with one caveat---such a definition won't change any context-specific delimiters which are defined:
无论 \cmd{printdelim} 周围的内容怎样,总会打印出 \cmd{addspace}。
由于context 是任意字符串,因此可以在任何时刻使用 \cmd{delimcontext} 构建。
如果 \cmd{printdelim} 在当前 context 没有找到定界符 \prm{name} 的特定值,那么就直接打印出 \cmd{name}。
这意味着样式作者可以使用 \cmd{printdelim}。
同时希望使用 \cmd{renewcommand} 重定义定界符的用户也可以这样做。
不过前提是,该定义不能改变任何 context 相关的定界符,如下所示:

\begin{ltxexample}[style=latex]{}
\DeclareDelimFormat{delima}{X}
\DeclareDelimFormat[textcite]{delima}{Y}
\renewcommand*{\delima}{Z}
\end{ltxexample}
%
%Here, \cmd{delima} always prints <Z>. \verb+\printdelim{delima}+ in any context other than <textcite> also prints \cmd{delima} and hence <Z> but in a <textcite> context prints <Y>. See the \file{04-delimiters.tex} example file that comes with \biblatex\ for more information.
这里,\cmd{delima} 总会打印“Z”。
在任何“textcite”之外的 context 中,\verb+\printdelim{delima}+ 也会打印 \cmd{delima} 也就是“Z”,
而在“textcite”中则打印“Y”。
更多信息参考 \biblatex 附带的示例文件 \file{04-delimiters.tex}。

\subsubsection{语言相关命令}%\subsubsection{Language-specific Commands}
\label{use:fmt:lng}

%The commands in this section are language specific. When redefining them, you need to wrap the new definition in a \cmd{DeclareBibliographyExtras} command (in an \file{.lbx} file) or a \cmd{DefineBibliographyExtras} command (user documents), see \secref{use:lng} for details. Note that all commands starting with \cmd{mk\dots} take one or more arguments.

本节中的命令是与语言相关的。
因此重定义时需要将新的定义包裹在 \cmd{DeclareBibliographyExtras} 命令(在 \file{.lbx} 文件中)
或者 \cmd{DefineBibliographyExtras} 命令中(用户文件中),详见 \secref{use:lng} 节。
请注意所有以 \cmd{mk\dots} 开头的命令都带有一个或更多选项。

\begin{ltxsyntax}

\csitem{bibrangedash}

%The language specific dash to be used for ranges of numbers. Defaults to \cmd{textendash}.
用于数字范围的横线。
默认值为 \cmd{textendash}。

\csitem{bibrangessep}

%The language specific separator to be used between multiple ranges. Defaults to a comma followed by a space.
用于多重范围之间的分隔符。
默认为一个逗号加一个空格。

\csitem{bibdatesep}

%The language specific separator used between date components in terse date formats. Defaults to \cmd{hyphen}.
在短日期格式用于日期成分之间的分隔符。
默认为 \cmd{hyphen}。

\csitem{bibdaterangesep}

%The language specific separator to be used for date ranges. Defaults to \cmd{textendash} for all date formats apart from \opt{ymd} which defaults to a \cmd{slash}. The date format option \opt{edtf} is hard-coded to \cmd{slash} since this is a standards compliant format.
用于日期范围的分隔符。
对于 \opt{ymd} 格式默认为 \cmd{slash},
对于其它日期格式默认为 \cmd{textendash}。
日期格式选项 \opt{edtf} 则硬编码为 \cmd{slash},因为这是标准兼容格式。

\csitem{mkbibdatelong}

%Takes the names of three field as arguments which correspond to three date components (in the order year\slash month\slash day) and uses their values to print the date in the language specific long date format.

以对应于三个日期成分(以年/月/日的顺序)的三个域名作为选项,
并在长日期格式中使用相应值来打印日期。

\csitem{mkbibdateshort}

%Similar to \cmd{mkbibdatelong} but using the language specific short date format.
类似于 \cmd{mkbibdatelong} 但是使用短日期格式。

\csitem{mkbibtimezone}

%Modifies a timezone string passed in as the only argument. By default this changes <Z> to the value of \cmd{bibtimezone}.
修改传入的时区字符串作为唯一选项。
默认情况会将“Z”改为 \cmd{bibtimezone} 的值。

\csitem{bibdateuncertain}

%The language specific marker to be used after uncertain dates when the global option \opt{dateuncertain} is enabled. Defaults to a space followed by a question mark.
当启用全局选项 \opt{dateuncertain} 时用于不确定日期之后的标识符。
默认为空格加一个问号。

\csitem{bibdateeraprefix}

%The language specific marker which is printed as a prefix to beginning BCE/BC dates in a date range when the option \opt{dateera} is set to <astronomical>. Defaults to \cmd{textminus}, if defined and \cmd{textendash} otherwise.
当设置 \opt{dateera} 选项为 \opt{astronomical} 时,在日期范围中公元前日期的前缀标识符。
命令 \cmd{textminus} 如果有定义则为默认值,否则 \cmd{textendash} 为默认值。

\csitem{bibdateeraendprefix}

%The language specific marker which is printed as a prefix to end BCE/BC dates in a date range when the option \opt{dateera} is set to <astronomical>. Defaults to a thin space followed by \cmd{bibdateeraprefix} when \cmd{bibdaterangesep} is set to a dash and to \cmd{bibdateeraprefix} otherwise.  This is a separate macro so that you may add extra space before a negative date marker which, for example follows a dash date range marker as this can look a little odd.
当设置 \opt{dateera} 选项为 \opt{astronomical} 时,在日期范围中公元前日期结束的标识符。
当 \cmd{bibdaterangesep} 选项设置为短横线时默认值为窄间距(thin space)加上 \cmd{bibdateeraprefix},
否则默认值为 \cmd{bibdateeraprefix}。

\csitem{bibtimesep}

%The language specific marker which separates time components. Defaults to a colon.
分隔时间成分的标识符,默认为冒号。

\csitem{bibtimezonesep}

%The language specific marker which separates an optional time zone component from a time. Empty by default.
分隔时间与可选的时区信息的标识符。默认为无。

\csitem{bibtzminsep}

%The language specific marker which separates hour and minute component of offset timezones. Defaults to a \cmd{bibtimesep}.
偏移时区的小时和分钟信息直接的分隔符。
默认值为 \cmd{bibtimesep}。

\csitem{bibdatetimesep}

The language specific separator printed between date and time components when printing time components along with date components (see the \opt{$<$datetype$>$dateusetime} option in \secref{use:opt:pre:gen}). Defaults to a space for non-\acr{ISO8601-2} output formats, and 'T' for \acr{ISO8601-2} output format.

当打印时间成分与日期成分时二者之间的分隔符
(见 \secref{use:opt:pre:gen} 节的 \opt{\prm{datetype}dateusetime} 选项)。
对于非EDTF输出格式,默认值为空格;
对于EDTF输出格式,默认值为“T”。

\csitem{finalandcomma}

%Prints the comma to be inserted before the final <and> in a list, if applicable in the respective language. Here is an example:
当在对应语言中可用时在列表的最后一个“and” 之前打印逗号。这里是一个例子:

\begin{ltxexample}
Michel Goossens, Frank Mittelbach<<,>> and Alexander Samarin
\end{ltxexample}
%
%\cmd{finalandcomma} is the comma before the word <and>. See also \cmd{multinamedelim}, \cmd{finalnamedelim}, \cmd{textcitedelim}, and \cmd{revsdnamedelim} in \secref{use:fmt:fmt}.
\cmd{finalandcomma} 是单词“and”之前的逗号。
另见 \secref{use:fmt:fmt} 节的 \cmd{multinamedelim}、\cmd{finalnamedelim}、\cmd{textcitedelim} 和 \cmd{revsdnamedelim}。

\csitem{finalandsemicolon}

%Prints the semicolon to be inserted before the final <and> in a list of lists, if applicable in the respective language. Here is an example:
当在对应语言中可用时在列表的最后一个“and”之前打印分号。例如:

\begin{ltxexample}
Goossens, Mittelbach, and Samarin; Bertram and Wenworth<<;>> and Knuth
\end{ltxexample}
%
%\cmd{finalandsemicolon} is the semicolon before the word <and>. See also \cmd{textcitedelim} in \secref{use:fmt:fmt}.
\cmd{finalandsemicolon} 是单词“and”之前的分号。
另见 \secref{use:fmt:fmt} 节的 \cmd{textcitedelim}。

\cmditem{mkbibordinal}{integer}

%This command, which takes an integer as its argument, prints an ordinal number.
该命令将一个整数作为选项,打印出一个序数。

\cmditem{mkbibmascord}{integer}

%Similar to \cmd{mkbibordinal}, but prints a masculine ordinal, if applicable in the respective language.
类似于 \cmd{mkbibordinal},但在对应语言中可用时打印出阳性序数。

\cmditem{mkbibfemord}{integer}

%Similar to \cmd{mkbibordinal}, but prints a feminine ordinal, if applicable in the respective language.
类似于 \cmd{mkbibordinal},但在对应语言中可用时打印出阴性序数。

\cmditem{mkbibneutord}{integer}

%Similar to \cmd{mkbibordinal}, but prints a neuter ordinal, if applicable in the respective language.
类似于 \cmd{mkbibordinal},但在对应语言中可用时打印出中性序数。

\cmditem{mkbibordedition}{integer}

%Similar to \cmd{mkbibordinal}, but intended for use with the term <edition>.
类似于 \cmd{mkbibordinal},但与单词“edition”一起使用。

\cmditem{mkbibordseries}{integer}

%Similar to \cmd{mkbibordinal}, but intended for use with the term <series>.
类似于 \cmd{mkbibordinal},但与单词“series”一起使用。

\end{ltxsyntax}

\subsubsection{长度和计数器}%\subsubsection{Lengths and Counters}
\label{use:fmt:len}

%The length registers and counters in this section may be changed in the document preamble with \cmd{setlength} and \cmd{setcounter}, respectively.

本节中的长度寄存器和计数器可以在导言区中分别用 \cmd{setlength} 和 \cmd{setcounter} 来修改。

\begin{ltxsyntax}

\lenitem{bibhang}

%The hanging indentation of the bibliography, if applicable. This length is initialized to \cmd{parindent} at load-time.

参考文献的悬挂缩进(当使用时)。
该长度在导入时初始化为 \cmd{parindent}。

\lenitem{biblabelsep}

%The horizontal space between entries and their corresponding labels in the bibliography. This only applies to bibliography styles which print labels, such as the \texttt{numeric} and \texttt{alphabetic} styles. This length is initialized to twice the value of \cmd{labelsep} at load-time.

参考文献中条目和相应标签之间的水平距离。
这只应用于 \texttt{numeric} 和 \texttt{alphabetic} 等打印标签的参考文献样式。
该长度在导入时初始化为 \cmd{labelsep} 值的两倍。

\lenitem{bibitemsep}

%The vertical space between the individual entries in the bibliography. This length is initialized to \cmd{itemsep} at load-time. Note that \len{bibitemsep}, \len{bibnamesep}, and \len{bibinitsep} obey the rules for \cmd{addvspace}, that is, when vertical space introduced by any of these commands immediately follows on from space introduced by another of them, the resulting total space is equal to the largest of them.

参考文献中每一条目之间的垂直间距。
该长度在导入时初始化为 \cmd{itemsep}。
请注意 \len{bibitemsep}、\len{bibnamesep} 和 \len{bibinitsep} 服从 \cmd{addvspace} 的规则,
也就是,当这些命令中任何一个直接在另外一个之后引入垂直间距时,所得到的总间距是其中的最大值。

\lenitem{bibnamesep}

%Vertical space to be inserted between two entries in the bibliography whenever an entry starts with a name which is different from the initial name of the previous entry. The default value is zero. Setting this length to a positive value greater than \len{bibitemsep} will group the bibliography by author\slash editor name. Note that \len{bibitemsep}, \len{bibnamesep}, and \len{bibinitsep} obey the rules for \cmd{addvspace}, that is, when vertical space introduced by any of these commands immediately follows on from space introduced by another of them, the resulting total space is equal to the largest of them.

参考文献中插入在两条姓名不同的条目之间的垂直间距。缺省值是零。
将该值设为大于 \len{bibitemsep} 会使得参考文献按照作者/编辑名分组。
请注意 \len{bibitemsep}、\len{bibnamesep} 和 \len{bibinitsep} 服从 \cmd{addvspace} 的规则,
也就是,当这些命令中任何一个直接在另外一个之后引入垂直间距时,所得到的总间距是其中的最大值。

\lenitem{bibinitsep}

%Vertical space to be inserted between two entries in the bibliography whenever an entry starts with a letter which is different from the initial letter of the previous entry. The default value is zero. Setting this length to a positive value greater than \len{bibitemsep} will group the bibliography alphabetically. Note that \len{bibitemsep}, \len{bibnamesep}, and \len{bibinitsep} obey the rules for \cmd{addvspace}, that is, when vertical space introduced by any of these commands immediately follows on from space introduced by another of them, the resulting total space is equal to the largest of them.

参考文献中插入在两条首字母不同的条目之间的垂直间距。缺省值是零。
将该值设置为大于 \len{bibitemsep} 会使得参考文献按字母分组。
请注意 \len{bibitemsep}、\len{bibnamesep} 和 \len{bibinitsep} 服从 \cmd{addvspace} 的规则,
也就是,当这些命令中任何一个直接在另外一个之后引入垂直间距时,所得到的总间距是其中的最大值。

\lenitem{bibparsep}

%The vertical space between paragraphs within an entry in the bibliography. The default value is zero.

参考文献中条目内部的段间距。缺省值为零。

\cntitem{abbrvpenalty}

%This counter, which is used by the localisation modules, holds the penalty used in short or abbreviated localisation strings. For example, a linebreak in expressions such as «et al.» or «ed. by» is unfortunate, but should still be possible to prevent overfull boxes. This counter is initialized to \cmd{hyphenpenalty} at load-time. The idea is making \tex treat the whole expression as if it were a single, hyphenatable word as far as line"=breaking is concerned. If you dislike such linebreaks, use a higher value. If you do not mind them at all, set this counter to zero. If you want to suppress them unconditionally, set it to <infinite> (10\,000 or higher).\footnote{The default values assigned to \cnt{abbrvpenalty}, \cnt{lownamepenalty}, and \cnt{highnamepenalty} are deliberately very low to prevent overfull boxes. This implies that you will hardly notice any effect on line-breaking if the text is set justified. If you set these counters to 10\,000 to suppress the respective breakpoints, you will notice their effect but you may also be confronted with overfull boxes. Keep in mind that line-breaking in the bibliography is often more difficult than in the body text and that you can not resort to rephrasing a sentence. In some cases it may be preferable to set the entire bibliography \cmd{raggedright} to prevent suboptimal linebreaks. In this case, even the fairly low default penalties will make a visible difference.}

该计数器在本地化模块中使用,用于设定本地化字符串中缩写和短语中使用的惩罚值。
例如,“et al”或“ed. by” 等短语中的断行是不美观的,但为了防止盒子溢出仍然应当可以使用。
该计数器在导入时初始化为 \cmd{hyphenpenalty}。
断行考虑的原则是,使得 \TeX 将整个语句看做是单个可以用连字符断行的单词。
如果你不喜欢相应的断行,可以设置为更高的值。
如果你不介意这些效果,可以设置为零。
如果你想无条件地取消这种效果,
可以设置为“无穷”(10\,000 或更高)。\footnote{%
这里很慎重地将 \cnt{abbrvpenalty}、\cnt{lownamepenalty} 和 \cnt{highnamepenalty} 的缺省值设定得非常低以防止盒子溢出。
这意味着,如果文本设置合理,那么你几乎不会注意到断行的影响。
如果你将这些值设置为 10\,000 以取消相应断点,那么你就会注意到它们的影响,不过同时你也许要面对盒子溢出现象。
需要注意的是,参考文献中的断行往往比正文中更困难,而且你不能通过换一种表达方式来解决。
在某些情况下,在整个参考文献中设置 \cmd{raggedright} 来阻止非最佳的断行往往更好。
此时,即使是相对低的缺省惩罚项也会造成不同效果。}

\cntitem{highnamepenalty}

%This counter holds a penalty affecting line"=breaking in names. Please refer to \secref{use:cav:nam,use:fmt:fmt} for explanation. The counter is initialized to \cmd{hyphenpenalty} at load-time. Use a higher value if you dislike the respective linebreaks. If you do not mind them at all, set this counter to zero. If you prefer the traditional \bibtex behavior (no linebreaks at \cnt{highnamepenalty} breakpoints), set it to <infinite> (10\,000 or higher).

该计数器设定了影响姓名中断行的惩罚值。其解释见 \secref{use:cav:nam,use:fmt:fmt} 节。
该计数器在导入时初始化为 \cmd{hyphenpenalty}。
如果你不喜欢相应的断行,可以设置为更高的值。
如果你不介意这些效果,可以设置为零。
如果你更喜欢传统的 \BibTeX 样式(在 \cnt{highnamepenalty} 处没有断行),可以设置为“无穷”(10\,000或更高)。

\cntitem{lownamepenalty}

%Similar to \cnt{highnamepenalty}. Please refer to \secref{use:cav:nam,use:fmt:fmt} for explanation. The counter is initialized to half the \cmd{hyphenpenalty} at load-time. Use a higher value if you dislike the respective linebreaks. If you do not mind them at all, set this counter to zero.

类似于 \cnt{highnamepenalty}。其解释见 \secref{use:cav:nam,use:fmt:fmt} 节。
该计数器在导入时初始化为 \cmd{hyphenpenalty} 的一半。
如果你不喜欢相应的断行,可以设置为更高的值。
如果你不介意这些效果,可以设置为零。

\end{ltxsyntax}

\subsubsection{通用命令}%\subsubsection{All-purpose Commands}
\label{use:fmt:aux}

%The commands in this section are all-purpose text commands which are generally available, not only in citations and the bibliography.
本节中的命令是通用文本命令,除了引用和参考文献之外在一般情况下都可以使用。

\begin{ltxsyntax}

\csitem{bibellipsis}

%An ellipsis symbol with brackets: <[\dots\unkern]>.
带有方括号的省略号:“[\dots\unkern]”。

\csitem{noligature}

%Disables ligatures at this position and adds some space. Use this command to break up standard ligatures like <fi> and <fl>. It is similar to the \verb+"|+ shorthand provided by some language modules of the \sty{babel}/\sty{polyglossia} packages.

在该位置取消连字并增加一些空格。
使用该命令来打断标准中“fi”、“fl”等连字。
类似于 \sty{babel}/\sty{polyglossia} 宏包中一些语言模块提供的 \verb+"|+ 缩写。

\csitem{hyphenate}

%A conditional hyphen. In contrast to the standard \cmd{-} command, this one allows hyphenation in the rest of the word. It is similar to the \verb|"-| shorthand provided by some language modules of the \sty{babel}/\sty{polyglossia} packages.

条件连字号。与标准的 \cmd{-} 命令不同,该命令允许在单词的剩余部分使用连字号。
类似于 \sty{babel}/\sty{polyglossia} 宏包中一些语言模块提供的 \verb+"-+ 缩写。

\csitem{hyphen}

%An explicit, breakable hyphen intended for compound words. In contrast to a literal <\texttt{-}>, this command allows hyphenation in the rest of the word. It is similar to the \verb|"=| shorthand provided by some language modules of the \sty{babel}/\sty{polyglossia} packages.

用于复合词的显式可断连字号。
与文本“\texttt{-}”不同,该命令允许在单词的剩余部分使用连字号。
类似于 \sty{babel}/\sty{polyglossia} 宏包中一些语言模块提供的 \verb+"=+ 缩写。

\csitem{nbhyphen}

%An explicit, non-breakable hyphen intended for compound words. In contrast to a literal <\texttt{-}>, this command does not permit line breaks at the hyphen but still allows hyphenation in the rest of the word. It is similar to the \verb|"~| shorthand provided by some language modules of the \sty{babel}/\sty{polyglossia} packages.

用于复合词的显式不可断连字号。
与文本“\texttt{-}”不同,该命令不允许在连字号处断行但仍然允许在单词的剩余部分使用连字号。
类似于 \sty{babel}/\sty{polyglossia} 宏包中一些语言模块提供的 \verb+"~+ 缩写。

\csitem{nohyphenation}

%A generic switch which suppresses hyphenation locally. Its scope should normally be confined to a group.

局部抑制连字号的一般切换命令。
正常情况下其作用范围应被限定在一个组内。

\cmditem{textnohyphenation}{text}

%Similar to \cmd{nohyphenation} but restricted to the \prm{text} argument.

类似于 \cmd{nohyphenation} 但是限制在 \prm{text} 选项中。

\cmditem{mknumalph}{integer}

%Takes an integer in the range 1--702 as its argument and converts it to a string as follows: 1=a, \textellipsis, 26=z, 27=aa, \textellipsis, 702=zz. This is intended for use in formatting directives for the \bibfield{extrayear} and \bibfield{extraalpha} fields.

将 1--702 间的整数作为选项并将其转化为如下的字符串: 1=a, \textellipsis, 26=z, 27=aa, \textellipsis, 702=zz。
这用于 \bibfield{extrayear} 和 \bibfield{extraalpha} 等域的格式设置。

\cmditem{mkbibacro}{text}

%Generic command which typesets an acronym using the small caps variant of the current font, if available, and as-is otherwise. The acronym should be given in uppercase letters.

在可用时使用当前字体的小型大写变体排版首字母缩写的一般性命令,否则依原样排版。
首字母缩写应当以大写字母形式给出。

\cmditem{autocap}{character}

%Automatically converts the \prm{character} to its uppercase form if \biblatex's punctuation tracker would capitalize a localisation string at the current location. This command is robust. It is useful for conditional capitalization of certain strings in an entry. Note that the \prm{character} argument is a single character given in lowercase. For example:

如果 \biblatex 的标点追踪能够在当前位置将本地化字符串大写,
该命令自动将 \prm{character} 转化为大写形式。
该命令是鲁棒的。在条目的某些字符串需要给定条件下的大写时该命令是很有用的。
请注意,\prm{character} 选项是以小写形式给出的单字符。例如:

\begin{ltxexample}
\autocap{s}pecial issue
\end{ltxexample}
%
%will yield <Special issue> or <special issue>, as appropriate. If the string to be capitalized starts with an inflected character given in Ascii notation, include the accent command in the \prm{character} argument as follows:
将产生合适的“Special issue”或者“special issue”。
如果被大写的字符串以Ascii记号给出的变体字符开始,包括如下 \prm{character} 选项中的重音命令:

\begin{ltxexample}
\autocap{\'e}dition sp\'eciale
\end{ltxexample}
%
%This will yield <Édition spéciale> or <édition spéciale>. If the string to be capitalized starts with a command which prints a character, such as \cmd{ae} or \cmd{oe}, simply put the command in the \prm{character} argument:
这会生成“Édition spéciale”或者“édition spéciale”。
如果大写的字符串以能打印出字符的命令开始,例如 \cmd{ae} 或 \cmd{oe},只要该命令放入 \prm{character} 选项即可:

\begin{ltxexample}
\autocap{\oe}uvres
\end{ltxexample}
%
%This will yield <Œuvres> or <œuvres>.
这会生成“Œuvres”或“œuvres”。

\end{ltxsyntax}

\subsection[语言注记]{特定语言注记}%\subsection[Language notes]{Language-specific Notes}
\label{use:loc}

%The facilities discussed in this section are specific to certain localisation modules.
本节讨论的功能特定于某些本地化模块。
\subsubsection{Bulgarian}
\label{use:loc:bul}

Like the Greek localisation module, the Bulgarian module also requires \utf support. It will not work with any other encoding.

\subsubsection{美式英文}%\subsubsection{American}
\label{use:loc:us}

%The American localisation module uses \cmd{uspunctuation} from \secref{aut:pct:cfg} to enable <American-style> punctuation. If this feature is enabled, all trailing commas and periods after \cmd{mkbibquote} will be moved inside the quotes. If you want to disable this feature, use \cmd{stdpunctuation} as follows:

美式英文本地化模块使用 \secref{aut:pct:cfg} 节的 \cmd{uspunctuation} 来激活“美式”标点。
如果启用该特性,所有在 \cmd{mkbibquote} 之后的逗号和句号会前移到引号内。
如果想要关闭该特性,使用如下的 \cmd{stdpunctuation}:

\begin{ltxexample}
\DefineBibliographyExtras{american}{%
  \stdpunctuation
}
\end{ltxexample}
%
%By default, the <American punctuation> feature is enabled by the \texttt{american} localisation module only. The above code is only required if you want American localisation without American punctuation. Since standard punctuation is the package default, it would be redundant with any other language.
缺省情况下,“美式标点”特性只由 \texttt{american} 本地化模块启用。
以上代码只在你想要不带美式标点的美式英文本地化时需要。
由于标准的标点是宏包缺省的,对于其它语言这会是多余的。

%It is highly advisable to always specify \texttt{american}, \texttt{british}, \texttt{australian}, etc. rather than \texttt{english} when loading the \sty{babel}/\sty{polyglossia} packages to avoid any possible confusion. Older versions of the \sty{babel} package used to treat \opt{english} as an alias for \opt{british}; more recent ones treat it as an alias for \opt{american}. The \biblatex package essentially treats \texttt{english} as an alias for \opt{american}, except for the above feature which is only enabled if \texttt{american} is requested explicitly.

为了避免可能的混淆,在导入 \sty{babel}/\sty{polyglossia} 宏包时,
强烈建议总是指明 \texttt{american}、\texttt{british}、\texttt{australian} 等而不是 \texttt{english}。
老版本的 \sty{babel} 宏包过去将 \opt{english} 作为 \opt{british} 的别名;
而更近的版本中将其作为 \opt{american} 的别名。
除了以上只在需要显式指明 \texttt{american} 时才启用的特性,
\biblatex 宏包本质上将 \texttt{english} 作为 \opt{american} 的别名。

\subsubsection{西班牙文}%\subsubsection{Spanish}
\label{use:loc:esp}

%Handling the word <and> is more difficult in Spanish than in the other languages supported by this package because it may be <y> or <e>, depending on the initial sound of the following word. Therefore, the Spanish localisation module does not use the localisation string <\texttt{and}> but a special internal <smart and> command. The behavior of this command is controlled by the \cnt{smartand} counter.

在本宏包中,处理西班牙文的单词“and”比其它语言更困难,
因为可以是“y”或“e”,这取决于下个单词的第一个音节。
因此,西班牙文的本地化模块不使用本地化字符串“\texttt{and}”,
而是使用特殊的内部“智能and”命令。
该命令的行为由 \cnt{smartand} 计数器控制。

\begin{ltxsyntax}

\cntitem{smartand}

%This counter controls the behavior of the internal <smart and> command. When set to 1, it prints <y> or <e>, depending on the context. When set to 2, it always prints <y>. When set to 3, it always prints <e>. When set to 0, the <smart and> feature is disabled. This counter is initialized to 1 at load-time and may be changed in the preamble. Note that setting this counter to a positive value implies that the Spanish localisation module ignores \cmd{finalnamedelim} and \cmd{finallistdelim}.

该计数器控制内部“智能and”命令的行为。
当设置为1时,取决于语境会打印“y”或“e”。
当设置为2时,总是打印“y”。
当设置为3时,总是打印“e”。
当设置为0时则禁用“智能and”特性。
该计数器在导入时初始化为1,可以在导言区中修改。
请注意,将该计数器设置为一个正整数则说明西班牙文的本地化模块会忽略 \cmd{finalnamedelim} 和 \cmd{finallistdelim}。

\csitem{forceE}

%Use this command in \file{bib} files if \biblatex gets the <and> before a certain name wrong. As its name suggests, it will enforce <e>. This command must be used in a special way to be correct \bibtex datafile format. Here is an example:

如果 \biblatex 在某个错误的名字前得到“and”,那么可以在 \file{bib} 文件中使用该命令。
如同该命令的名字,该命令会强制输出“e”。
为了获得正确的 \BibTeX 数据文件格式,必须以特定方式使用该命令。例如:

\begin{lstlisting}[style=bibtex]{}
author = {Edward Jones and Eoin Maguire},
author = {Edward Jones and <<{\forceE{E}}>>oin Maguire},
\end{lstlisting}
%
%Note that the initial letter of the respective name component is given as an argument to \cmd{forceE} and that the entire construct is wrapped in an additional pair of curly braces.
请注意,相应姓名成分的首字母作为 \cmd{forceE} 的选项,
然后整个地放在额外一对花括号中。

\csitem{forceY}

%Similar to \cmd{forceE} but enforces <y>.
类似于 \cmd{forceE} 但是强制输出“y”。

\end{ltxsyntax}

\subsubsection{希腊文}%\subsubsection{Greek}
\label{use:loc:grk}

%The Greek localisation module requires \utf support. It will not work with any other encoding. Generally speaking, the \biblatex package is compatible with the \sty{inputenc} package and with \xelatex. The \sty{ucs} package will not work. Since \sty{inputenc}'s standard \file{utf8} module is missing glyph mappings for Greek, this leaves Greek users with \xelatex. Note that you may need to load additional packages which set up Greek fonts. As a rule of thumb, a setup which works for regular Greek documents should also work with \biblatex. However, there is one fundamental limitation. As of this writing, \biblatex has no support for switching scripts. Greek titles in the bibliography should work fine, but English and other titles in the bibliography may be rendered in Greek letters. If you need multi-script bibliographies, using \xelatex is the only sensible choice.

希腊文本地化模块需要 \utf 支持,与其它编码不兼容。
一般来说,\biblatex 宏包与 \sty{inputenc} 宏包以及 \XeLaTeX 都兼容。
而 \sty{ucs} 宏包不可用。
由于 \sty{inputenc} 的标准 \opt{utf8} 模块缺失一部分希腊语字形映射,
因此希腊文用户可以选择 \XeLaTeX 。
请注意,用户仍需要载入额外宏包来设置希腊文字体。
根据经验,常规希腊文的文档设置一般也应当可以使用 \biblatex 。
然而有一个根本性限制:\biblatex 不支持切换语言。
参考文献中可以出现希腊文标题,但英文和其它标题可能会渲染为希腊字母。
如果需要多语言的参考文献,使用 \XeLaTeX 是明智的选择。

\subsubsection{俄文}%\subsubsection{Russian}
\label{use:loc:rus}

%Like the Greek localisation module, the Russian module also requires \utf support. It will not work with any other encoding.

与希腊文模块类似,俄文模块同样需要 \utf 支持。
与其它编码不兼容。

\subsection{使用注记}%\subsection{Usage Notes}
\label{use:use}

%The following sections give a basic overview of the \biblatex package and discuss some typical usage scenarios.

以下几节讨论了 \biblatex 宏包的基本概述以及一些典型的使用场合。

\subsubsection{概述}%\subsubsection{Overview}
\label{use:use:int}

%Using the \biblatex package is slightly different from using traditional \bibtex styles and related packages. Before we get to specific usage scenarios, we will therefore have a look at the structure of a typical document first:

使用 \biblatex 宏包与传统的 \BibTeX 样式和相关宏包稍有不同。
因此,在讨论具体的使用场景之前,我们首先要看一下典型的文件结构:

\begin{ltxexample}
\documentclass{...}
\usepackage[...]{biblatex}
<<\addbibresource>>{<<bibfile.bib>>}
\begin{document}
<<\cite>>{...}
...
<<\printbibliography>>
\end{document}
\end{ltxexample}
%
%With traditional \bibtex, the \cmd{bibliography} command serves two purposes. It marks the location of the bibliography and it also specifies the \file{bib} file(s). The file extension is omitted. With \biblatex, resources are specified in the preamble with  \cmd{addbibresource} using the full name with \file{.bib} suffix. The bibliography is printed using the \cmd{printbibliography} command which may be used multiple times (see \secref{use:bib} for details). The document body may contain any number of citation commands (\secref{use:cit}). Processing this example file requires that a certain procedure be followed. Suppose our example file is called \path{example.tex} and our bibliographic data is in \path{bibfile.bib}. The procedure, then, is as follows:
在传统的 \BibTeX 下,\cmd{bibliography} 命令提供了两个目的:
标记文献的位置并且确定 \file{bib} 文件。
文件扩展名是省略的。
而在 \biblatex 下则在导言区通过 \cmd{addbibresource} 使用文件名全称(带有 \file{.bib} 后缀)来确定文献资源。
文献的打印则使用 \cmd{printbibliography} 命令,而且该命令可以使用多次(详见 \secref{use:bib} 节)。
文档正文可以包含任意多个引用命令(\secref{use:cit} 节)。
处理示例文件需要以下若干步骤。
假设我们的示例文件叫 \path{example.tex},参考文献数据在 \path{bibfile.bib} 中,那么过程如下:

\begin{enumerate}

\item %Run \bin{latex} on \path{example.tex}. If the file contains any citations, \biblatex will request the respective data from \biber by writing commands to the auxiliary file \path{example.bcf}.
对 \file{example.tex} 运行 \bin{latex} 命令。
如果该文件包含引用,\biblatex 会将有关命令写入辅助文件 \file{example.bcf},进而从 \biber 调用相关数据。
\item %Run \bin{biber} on \path{example.bcf}. \biber will retrieve the data from \path{bibfile.bib} and write it to the auxiliary file \path{example.bbl} in a format which can be processed by \biblatex.
对 \file{example.bcf} 运行 \bin{biber} 命令。
\biber 会从 \file{bibfile.bib} 中检索数据,并将其写入辅助文件 \file{example.bib} 中,写入的格式可以被 \biblatex 处理。
\item %Run \bin{latex} on \path{example.tex}. \biblatex will read the data from \path{example.bbl} and print all citations as well as the bibliography.
对 \file{example.tex} 运行 \bin{latex} 命令。
\biblatex 会从 \file{example.bbl} 中读取数据并打印所有的引用及参考文献。

\end{enumerate}

\subsubsection{辅助文件}%\subsubsection{Auxiliary Files}
\label{use:use:aux}

%The \biblatex package uses one auxiliary \file{bcf} file only. Even if there are citation commands in a file included via \cmd{include}, you only need to run \biber on the main \file{bcf} file. All information \biber needs is in the \file{bcf} file, including information about all refsections if using multiple \env{refsection} environments (see \secref{use:use:mlt}).

\biblatex 宏包只使用一个 \file{bcf} 辅助文件。
即便文件中通过 \cmd{include} 包含引用命令,你也只需在主 \file{bcf} 文件上运行 \biber 。
\biber 需要的全部信息都在 \file{bcf} 文件中,包括当使用多重 \env{refsection} 环境
(见 \secref{use:use:mlt} 节)时关于所有参考文献分节的信息。

\subsubsection{多重文献}%\subsubsection{Multiple Bibliographies}
\label{use:use:mlt}

%In a collection of articles by different authors, such as a conference proceedings volume for example, it is very common to have one bibliography for each article rather than a global one for the entire book. In the example below, each article would be presented as a separate \cmd{chapter} with its own bibliography.

在由多位作者所写文章的合集中,例如会议文集的一卷,非常常见的做法是对每篇文章而不是对整本书分别做文献索引。
在以下的例子中,每篇文章是不同的一章 \cmd{chapter},并带有自己的文献索引。

\begin{ltxexample}
\documentclass{...}
\usepackage{biblatex}
\addbibresource{...}
\begin{document}
\chapter{...}
<<\begin{refsection}>>
...
<<\printbibliography[heading=subbibliography]>>
<<\end{refsection}>>
\chapter{...}
<<\begin{refsection}>>
...
<<\printbibliography[heading=subbibliography]>>
<<\end{refsection}>>
\end{document}
\end{ltxexample}
%
%If \cmd{printbibliography} is used inside a \env{refsection} environment, it automatically restricts the scope of the list of references to the enclosing \env{refsection} environment. For a cumulative bibliography which is subdivided by chapter but printed at the end of the book, use the \opt{section} option of \cmd{printbibliography} to select a reference section, as shown in the next example.
如果 \cmd{printbibliography} 在 \env{refsection} 环境内部使用,
它会自动将文献列表范围限制在 \env{refsection} 环境内。
对于在一本书末尾列出但是按照每一章划分的累积参考文献,
使用 \cmd{printbibliography} 的 \opt{section} 选项来选择参考文献分节,如下面的例子所示。

\begin{ltxexample}
\documentclass{...}
\usepackage{biblatex}
<<\defbibheading>>{<<subbibliography>>}{%
  \section*{References for Chapter \ref{<<refsection:\therefsection>>}}}
\addbibresource{...}
\begin{document}
\chapter{...}
<<\begin{refsection}>>
...
<<\end{refsection}>>
\chapter{...}
<<\begin{refsection}>>
...
<<\end{refsection}>>
\printbibheading
<<\printbibliography>>[<<section=1>>,<<heading=subbibliography>>]
<<\printbibliography>>[<<section=2>>,<<heading=subbibliography>>]
\end{document}
\end{ltxexample}
%
%Note the definition of the bibliography heading in the above example. This is the definition taking care of the subheadings in the bibliography. The main heading is generated with a plain \cmd{chapter} command in this case. The \biblatex package automatically sets a label at the beginning of every \env{refsection} environment, using the standard \cmd{label} command. The identifier used is the string \texttt{refsection:} followed by the number of the respective \env{refsection} environment. The number of the current section is accessible via the \cnt{refsection} counter. When using the \opt{section} option of \cmd{printbibliography}, this counter is also set locally. This means that you may use the counter in heading definitions to print subheadings like «References for Chapter 3», as shown above. You could also use the title of the respective chapter as a subheading by loading the \sty{nameref} package and using \cmd{nameref} instead of \cmd{ref}:
请注意上面例子中文献标题的定义。
该定义考虑到了参考文献中的子标题。
主标题由普通的 \cmd{chapter} 生成。
\biblatex 宏包会自动在每个相应的 \env{refsection} 环境开始处用标准 \cmd{label} 命令分别设置标签。
其标识符是字符串 \texttt{refsection:} 接上 \env{refsection} 环境的序数。
当前节的序号可以通过 \cnt{refsection} 计数器获得。
当使用 \cmd{printbibliography} 的 \opt{section} 选项时,该计数器也被设置为局部的。
这意味着你可以在标题定义中使用该计数器来打印类似于上面例子中“References for Chapter 3”这样的子标题。
你也可以通过载入 \sty{nameref} 宏包和使用 \cmd{nameref} 代替 \cmd{ref} 来使用相应的章名作为子标题:

\begin{ltxexample}
\usepackage{<<nameref>>}
\defbibheading{subbibliography}{%
  \section*{<<\nameref{refsection:\therefsection}>>}}
\end{ltxexample}
%
%Since giving one \cmd{printbibliography} command for each part of a subdivided bibliography is tedious, \biblatex provides a shorthand. The \cmd{bibbysection} command automatically loops over all reference sections. This is equivalent to giving one \cmd{printbibliography} command for every section but has the additional benefit of automatically skipping sections without references. In the example above, the bibliography would then be generated as follows:
为参考文献的每一个子部分都给出 \cmd{printbibliography} 是很繁琐的,所以 \biblatex 提供了一个缩写语。
\cmd{bibbysection} 命令会自动遍历所有的参考文献分节。
这等价于为每节给出一个 \cmd{printbibliography} 命令,此外还会自动跳过没有文献的节。
在上面的例子中,参考文献可以按如下方式生成:

\begin{ltxexample}
\printbibheading
<<\bibbysection[heading=subbibliography]>>
\end{ltxexample}
%
%When using a format with one cumulative bibliography subdivided by chapter (or any other document division) it may be more appropriate to use \env{refsegment} rather than \env{refsection} environments. The difference is that the \env{refsection} environment generates labels local to the environment while \env{refsegment} does not affect the generation of labels, hence they will be unique across the entire document. The next example could also be given in \secref{use:use:div} because, visually, it creates one global bibliography subdivided into multiple segments.
当使用按章(或其它文档单元)划分的累积参考文献格式时,使用 \env{refsegment} 比 \env{refsection} 环境更合适一些。
不同之处在于 \env{refsection} 环境生成的标签是环境局部的,
而 \env{refsegment} 环境不影响标签生成,因此在整个文档中是唯一的。
下面的例子也可以在 \secref{use:use:div} 节中给出,因为它看起来创建了一个划分为多重片段的全局参考文献。

\begin{ltxexample}
\documentclass{...}
\usepackage{biblatex}
<<\defbibheading>>{<<subbibliography>>}{%
  \section*{References for Chapter \ref{<<refsegment:\therefsection\therefsegment>>}}}
\addbibresource{...}
\begin{document}
\chapter{...}
<<\begin{refsegment}>>
...
<<\end{refsegment}>>
\chapter{...}
<<\begin{refsegment}>>
...
<<\end{refsegment}>>
\printbibheading
<<\printbibliography>>[<<segment=1>>,<<heading=subbibliography>>]
<<\printbibliography>>[<<segment=2>>,<<heading=subbibliography>>]
\end{document}
\end{ltxexample}
%
%The use of \env{refsegment} is similar to \env{refsection} and there is also a corresponding \opt{segment} option for \cmd{printbibliography}. The \biblatex package automatically sets a label at the beginning of every \env{refsegment} environment using the string \texttt{refsegment:} followed by the number of the respective \env{refsegment} environment as an identifier. There is a matching \cnt{refsegment} counter which may be used in heading definitions, as shown above. As with reference sections, there is also a shorthand command which automatically loops over all reference segments:
\env{refsegment} 的使用类似于 \env{refsection},也有对应于 \cmd{printbibliography} 的 \opt{segment} 选项。
\biblatex 宏包自动在每个 \env{refsegment} 环境开始用字符串 \texttt{refsegment:}
后接相应 \env{regsegment} 环境的序号来设置标签作为标识符。
如前所述,有一个匹配的 \cnt{refsegment} 计数器可以用在标题定义中。
对于文献节,也有缩写名可以自动遍历所有的文献片段:

\begin{ltxexample}
\printbibheading
<<\bibbysegment[heading=subbibliography]>>
\end{ltxexample}
%
%This is equivalent to giving one \cmd{printbibliography} command for every segment in the current \env{refsection}.
这等价于为当前 \env{refsection} 的每个片段分别给出 \cmd{printbibliography} 命令。

\subsubsection{文献表划分}%\subsubsection{Subdivided Bibliographies}
\label{use:use:div}

%It is very common to subdivide a bibliography by certain criteria. For example, you may want to list printed and online resources separately or divide a bibliography into primary and secondary sources. The former case is straightforward because you can use the entry type as a criterion for the \opt{type} and \opt{nottype} filters of \cmd{printbibliography}. The next example also demonstrates how to generate matching subheadings for the two parts of the bibliography.

依照某一标准进行文献划分是非常普遍的。
例如,你也许需要分别列出印刷和网络资源,或者将参考文献分为主要和次要类型。
前一种情况比较简单,
因为可以使用条目类型作为 \cmd{printbibliography} 的 \opt{type} 和 \opt{nottype} filter 的标准。
下面的例子也演示了如何为参考文献的两部分生成匹配的子标题。

\begin{ltxexample}
\documentclass{...}
\usepackage{biblatex}
\addbibresource{...}
\begin{document}
...
\printbibheading
\printbibliography[<<nottype=online>>,heading=subbibliography,
                   <<title={Printed Sources}>>]
\printbibliography[<<type=online>>,heading=subbibliography,
                   <<title={Online Sources}>>]

\end{document}
\end{ltxexample}
%
%You may also use more than two subdivisions:
也可以使用两个以上的划分:

\begin{ltxexample}
\printbibliography[<<type=article>>,...]
\printbibliography[<<type=book>>,...]
\printbibliography[<<nottype=article>>,<<nottype=book>>,...]
\end{ltxexample}
%
%It is even possible to give a chain of different types of filters:
甚至可以给出一组的不同类型的 filter:

\begin{ltxexample}
\printbibliography[<<section=2>>,<<type=book>>,<<keyword=abc>>,<<notkeyword=xyz>>]
\end{ltxexample}
%
%This would print all works cited in reference section~2 whose entry type is \bibtype{book} and whose \bibfield{keywords} field includes the keyword <abc> but not <xyz>. When using bibliography filters in conjunction with a numeric style, see \secref{use:cav:lab}. If you need complex filters with conditional expressions, use the \opt{filter} option in conjunction with a custom filter defined with \cmd{defbibfilter}. See \secref{use:bib:flt} for details on custom filters.
这会打印出所有在第二参考分节中条目类型为 \bibtype{book}
并且 \bibfield{keywords} 域包括关键词“abc”但是不包括“xyz”的作品。
关于结合数值样式使用文献 filter 见 \secref{use:cav:lab} 节。
如果你需要带有条件表达式的复杂 filter,
可以使用 \opt{filter} 选项结合由 \cmd{defbibfilter} 定义的定制filter。
关于定制filter详见 \secref{use:bib:flt} 节。

\begin{ltxexample}
\documentclass{...}
\usepackage{biblatex}
\addbibresource{...}
\begin{document}
...
\printbibheading
\printbibliography[<<keyword=primary>>,heading=subbibliography,%
                   <<title={Primary Sources}>>]
\printbibliography[<<keyword=secondary>>,heading=subbibliography,%
                   <<title={Secondary Sources}>>]
\end{document}
\end{ltxexample}
%
%Dividing a bibliography into primary and secondary sources is possible with a \bibfield{keyword} filter, as shown in the above example. In this case, with only two subdivisions, it would be sufficient to use one keyword as filter criterion:
如上例所示,将参考文献分为主要和次要部分可以通过 \bibfield{keyword} filter 实现。
在该情况下(只分成两部分),使用一个关键词作为 filter 标准就足够了:

\begin{ltxexample}
\printbibliography[<<keyword=primary>>,...]
\printbibliography[<<notkeyword=primary>>,...]
\end{ltxexample}
%
%Since \biblatex has no way of knowing if an item in the bibliography is considered to be primary or secondary literature, we need to supply the bibliography filter with the required data by adding a \bibfield{keywords} field to each entry in the \file{bib} file. These keywords may then be used as targets for the \opt{keyword} and \opt{notkeyword} filters, as shown above. It may be a good idea to add such keywords right away while building a \file{bib} file.
由于 \biblatex 无法知道文献中的某一条是否被认为是主要或者次要文献,
我们需要在 \file{bib} 文件中为每一条目增加 \bibfield{keywords} 域来提供文献 filter 所需的数据。
如上例所示,这些关键词可以用于 \opt{keyword} 和 \opt{notkeyword} filter 的目标。
在建立 \file{bib} 文件时就添加这样的关键词是一个不错的办法。

\begin{lstlisting}[style=bibtex]{}
@Book{key,
  <<keywords>>      = {<<primary>>,some,other,keywords},
  ...
\end{lstlisting}
%
%An alternative way of subdividing the list of references are bibliography categories. They differ from the keywords"=based approach shown in the example above in that they work on the document level and do not require any changes to the \file{bib} file.
另外一种划分文献列表的方法是使用参考文献类别。
这与上述例子中使用的基于关键词的方法的不同之处在于,
它们在文档水平处理而并不需要修改 \file{bib} 文件。

\begin{ltxexample}
\documentclass{...}
\usepackage{biblatex}
<<\DeclareBibliographyCategory>>{<<primary>>}
<<\DeclareBibliographyCategory>>{<<secondary>>}
<<\addtocategory>>{<<primary>>}{key1,key3,key6}
<<\addtocategory>>{<<secondary>>}{key2,key4,key5}
\addbibresource{...}
\begin{document}
...
\printbibheading
\printbibliography[<<category=primary>>,heading=subbibliography,%
                   <<title={Primary Sources}>>]
\printbibliography[<<category=secondary>>,heading=subbibliography,%
                   <<title={Secondary Sources}>>]
\end{document}
\end{ltxexample}
%
%In this case it would also be sufficient to use one category only:
在这个例子中,只使用一个类别也是可以的:

\begin{ltxexample}
\printbibliography[<<category=primary>>,...]
\printbibliography[<<notcategory=primary>>,...]
\end{ltxexample}
%
%It is still a good idea to declare all categories used in the bibliography explicitly because there is a \cmd{bibbycategory} command which automatically loops over all categories. This is equivalent to giving one \cmd{printbibliography} command for every category, in the order in which they were declared.
不过,显式地声明参考文献中使用的所有类别仍然是个不错的主意,
因为有一个 \cmd{bibbycategory} 命令能自动遍历所有的类别。
这等价于为每一类别按照所声明的顺序依次给出 \cmd{printbibliography} 命令。

\begin{ltxexample}
\documentclass{...}
\usepackage{biblatex}
<<\DeclareBibliographyCategory>>{<<primary>>}
<<\DeclareBibliographyCategory>>{<<secondary>>}
\addtocategory{primary}{key1,key3,key6}
\addtocategory{secondary}{key2,key4,key5}
<<\defbibheading>>{<<primary>>}{\section*{Primary Sources}}
<<\defbibheading>>{<<secondary>>}{\section*{Secondary Sources}}
\addbibresource{...}
\begin{document}
...
\printbibheading
<<\bibbycategory>>
\end{document}
\end{ltxexample}
%
%The handling of the headings is different from \cmd{bibbysection} and \cmd{bibbysegment} in this case. \cmd{bibbycategory} uses the name of the current category as a heading name. This is equivalent to passing \texttt{heading=\prm{category}} to \cmd{printbibliography} and implies that you need to provide a matching heading for every category.
在这个例子中,标题的处理与 \cmd{bibbysection} 和 \cmd{bibbysegment} 是不同的。
\cmd{bibbycategory} 使用当前类别的名字作为标题名。
这等价于将 \texttt{heading=\prm{category}} 传递给 \cmd{printbibliography},
从而意味着你需要为每一类别提供相匹配的标题。

\subsubsection{条目集}%\subsubsection{Entry Sets}
\label{use:use:set}

%An entry set is a group of entries which are cited as a single reference and listed as a single item in the bibliography. The individual entries in the set are separated by \cmd{entrysetpunct} (\secref{aut:fmt:fmt}). The \biblatex package supports two types of entry sets. Static entry sets are defined in the \file{bib} file like any other entry. Dynamic entry sets are defined with \cmd{defbibentryset} (\secref{use:bib:set}) on a per-document\slash per-refsection basis in the document preamble or the document body. This section deals with the definition of entry sets; style authors should also see \secref{aut:cav:set} for further information.

条目集是用单个引用并在参考文献中作为一项列出的一组条目。
条目集中每一项用 \cmd{entrysetpunct} 分隔(\secref{aut:fmt:fmt} 节)。
\biblatex 宏包支持两种类型的条目集。
静态条目集在 \file{bib} 文件中定义,这与其它条目类似。
而动态条目集在文档导言区或者正文中用 \cmd{defbibentryset} (\secref{use:bib:set})定义,
并且基于文档或参考文献分节。
本节讨论条目集的定义问题;样式作者对于更多信息也可以参考 \secref{aut:cav:set} 节。

\paragraph{静态条目集}%\paragraph{Static entry sets}

%Static entry sets are defined in the \file{bib} file like any other entry. Defining an entry set is as simple as adding an entry of type \bibtype{set}. The entry has an \bibfield{entryset} field defining the members of the set as a separated list of entry keys:

静态条目集如同其它条目一样在 \file{bib} 文件中定义。
定义这样的条目集只需添加一个类型为 \bibtype{set} 的条目。
该条目有一个 \bibfield{entryset} 域,其中使用条目键值的逗号分隔列表定义了条目集的元素:

\begin{lstlisting}[style=bibtex]{}
<<@Set>>{<<set1>>,
  <<entryset>> = {<<key1,key2,key3>>},
}
\end{lstlisting}
%
%Entries may be part of a set in one document\slash refsection and stand-alone references in another one, depending on the presence of the \bibtype{set} entry. If the \bibtype{set} entry is cited, the set members are grouped automatically. If not, they will work like any regular entry.
条目可以是文档或参考文件分节中一个集合的一部分,或者是另外一个条目集中的孤立文献,这取决于 \bibtype{set} 条目。
如果 \bibtype{set} 条目被引用,其中的成员自动分成一组。否则它们就像其它的常规条目一样。

%\paragraph[Dynamic entry sets]{Dynamic entry sets}
\paragraph[动态条目集]{动态条目集}

%Dynamic entry sets are set up and work much like static ones. The main difference is that they are defined in the document preamble or on the fly in the document body using the \cmd{defbibentryset} command from \secref{use:bib:set}:

动态条目集的设置和运行和静态条目集很相似。
主要的区别是,它们是在导言区或者实时地在文档中使用 \secref{use:bib:set} 节的 \cmd{defbibentryset} 命令来定义的:

\begin{lstlisting}[style=bibtex]{}
\defbibentryset{set1}{key1,key2,key3}
\end{lstlisting}
%
%Dynamic entry sets in the document body are local to the enclosing \env{refsection} environment, if any. Otherwise, they are assigned to reference section~0. Those defined in the preamble are assigned to reference section~0.
正文中的动态条目集在其所在的 \env{refsection} 环境中是局部的(如果有的话)。
否则它们被分配给第零文献分节。
定义在导言区的动态条目集也被分在第零文献节。

\subsubsection[数据容器]{数据容器}%\subsubsection[Data Containers]{Data Containers}
\label{use:use:xdat}

%The \bibtype{xdata} entry type serves as a data container holding one or more fields. These fields may be inherited by other entries using the \bibfield{xdata} field. \bibtype{xdata} entries may not be cited or added to the bibliography, they only serve as a data source for other entries. This data inheritance mechanism is useful for fixed field combinations such as \bibfield{publisher}\slash \bibfield{location} and for other frequently used data:

作为数据容器,\bibtype{xdata} 条目类型可以包含一个或更多域。
这些域可以被其它条目使用 \bibfield{xdata} 来继承。
\bibtype{xdata} 条目可以不被引用或者打印在参考文献中,它们只为其它条目提供数据源。
这种数据继承机制常用于 \bibfield{publisher}\slash \bibfield{location} 这样的固定域组合或者其它常用数据:

\begin{lstlisting}[style=bibtex]{}
<<@XData>>{<<hup>>,
  publisher  = {Harvard University Press},
  location   = {Cambridge, Mass.},
}
@Book{...,
  author     = {...},
  title	     = {...},
  date	     = {...},
  <<xdata>>      = {<<hup>>},
}
\end{lstlisting}
%
%Using a separated list of keys in its \bibfield{xdata} field, an entry may inherit data from several \bibtype{xdata} entries. Cascading \bibtype{xdata} entries are supported as well, \ie an \bibtype{xdata} entry may reference one or more other \bibtype{xdata} entries:
一个条目通过在 \bibfield{xdata} 域中使用分隔键列表,可以继承若干个 \bibtype{xdata} 条目的数据。
\bibtype{xdata} 条目的串联也是支持的,
即,一个 \bibtype{xdata} 条目可以涉及到一个或更多其它 \bibtype{xdata} 条目:

\begin{lstlisting}[style=bibtex]{}
@XData{macmillan:name,
  publisher  = {Macmillan},
}
@XData{macmillan:place,
  location   = {New York and London},
}
@XData{macmillan,
  xdata      = {macmillan:name,macmillan:place},
}
@Book{...,
  author     = {...},
  title	     = {...},
  date	     = {...},
  xdata	     = {macmillan},
}
\end{lstlisting}
%
%See also \secref{bib:typ:blx,bib:fld:spc}.
另见 \secref{bib:typ:blx,bib:fld:spc} 节。

\subsubsection{电子出版信息}%\subsubsection{Electronic Publishing Information}
\label{use:use:epr}

%The \biblatex package provides three fields for electronic publishing information: \bibfield{eprint}, \bibfield{eprinttype}, and \bibfield{eprintclass}. The \bibfield{eprint} field is a verbatim field similar to \bibfield{doi} which holds the identifier of the item. The \bibfield{eprinttype} field holds the resource name, \ie the name of the site or electronic archive. The optional \bibfield{eprintclass} field is intended for additional information specific to the resource indicated by the \bibfield{eprinttype} field. This could be a section, a path, classification information, etc. If the \bibfield{eprinttype} field is available, the standard styles will use it as a literal label. In the following example, they would print «Resource: identifier» rather than the generic «eprint: identifier»:

\biblatex 宏包为电子出版信息提供了三种域:\bibfield{eprint}、\bibfield{eprinttype} 和 \bibfield{eprintclass}。
\bibfield{eprint} 域类似于 \bibfield{doi},是一种保持项目标识符的抄录模式域。
\bibfield{eprinttype} 域保存资源名称,即网址或电子档案的名称。
可选的 \bibfield{eprintclass} 域用于标明特定于 \bibfield{eprinttype} 域所指资源的额外信息。
这可以是章节、路径、分类信息等。
如果 \bibfield{eprinttype} 可用,标准样式会将其当做文本标签使用。
在以下例子中,它们会打印“Resource: identifier”而不是一般的“eprint: identifier”:

\begin{lstlisting}[style=bibtex]{}
<<eprint>>     = {<<identifier>>},
<<eprinttype>> = {<<Resource>>},
\end{lstlisting}
%
%The standard styles feature dedicated support for a few online archives. For arXiv references, put the identifier in the \bibfield{eprint} field and the string \texttt{arxiv} in the \bibfield{eprinttype} field:
标准样式对一些在线资源提供了专门支持。
对于 arXiv 文献,将标识符放在 \bibfield{eprint} 域中,将字符串 \texttt{arxiv} 放在 \bibfield{eprinttype} 域中:

\begin{lstlisting}[style=bibtex]{}
<<eprint>>     = {<<math/0307200v3>>},
<<eprinttype>> = {<<arxiv>>},
\end{lstlisting}
%
%For papers which use the new identifier scheme (April 2007 and later) add the primary classification in the \bibfield{eprintclass} field:
对于使用新标识格式的文章(2007年四月之后),将主分类放在 \bibfield{eprintclass} 域中:

\begin{lstlisting}[style=bibtex]{}
eprint      = {1008.2849v1},
eprinttype  = {arxiv},
<<eprintclass>> = {<<cs.DS>>},
\end{lstlisting}
%
%There are two aliases which ease the integration of arXiv entries. \bibfield{archiveprefix} is treated as an alias for \bibfield{eprinttype}; \bibfield{primaryclass} is an alias for \bibfield{eprintclass}. If hyperlinks are enabled, the eprint identifier will be transformed into a link to \nolinkurl{arxiv.org}. See the package option \opt{arxiv} in \secref{use:opt:pre:gen} for further details.
为了方便 arXiv 条目的整合专门设置了两个别称。
\bibfield{archiveprefix} 是 \bibfield{eprinttype} 的别称;
而 \bibfield{primaryclass} 是 \bibfield{eprintclass} 的别称。
如果启用超链接,\bibfield{eprint} 标识符将转换为指向 \nolinkurl{arxiv.org} 的链接。
更多信息可参见 \secref{use:opt:pre:gen} 节中的宏包选项 \opt{arxiv}。

%For \acr{JSTOR} references, put the stable \acr{JSTOR} number in the \bibfield{eprint} field and the string \texttt{jstor} in the \bibfield{eprinttype} field:
对于 \acr{JSTOR} 资源,将稳定的 \acr{JSTOR} 号放在 \bibfield{eprint} 域中,将字符串 \texttt{jstor} 放在 \bibfield{eprinttype} 域中:

\begin{lstlisting}[style=bibtex]{}
<<eprint>>     = {<<number>>},
<<eprinttype>> = {<<jstor>>},
\end{lstlisting}
%
%When using \acr{JSTOR}'s export feature to export citations in \bibtex format, \acr{JSTOR} uses the \bibfield{url} field by default (where the \prm{number} is a unique and stable identifier):
当使用 \acr{JSTOR} 的导出功能来导出 \BibTeX 格式引用时,
\acr{JSTOR} 缺省使用 \bibfield{url} 域(当 \prm{number} 是唯一稳定标识符时):

\begin{lstlisting}[style=bibtex]{}
url = {http://www.jstor.org/stable/<<number>>},
\end{lstlisting}
%
%While this will work as expected, full \acr{URL}s tend to clutter the bibliography. With the \bibfield{eprint} fields, the standard styles will use the more readable «\acr{JSTOR}: \prm{number}» format which also supports hyperlinks. The \prm{number} becomes a clickable link if \sty{hyperref} support is enabled.
尽管这样可以运行,但整个的 \acr{URL} 会使参考文献变得杂乱无章。
而使用 \bibfield{eprint} 域,标准样式会使用更加可读的“\acr{JSTOR}: \prm{number}” 格式而且同样支持超链接。
当启用 \sty{hyperref} 支持时,\prm{number} 会变成可以点击的链接。

%For PubMed references, put the stable PubMed identifier in the \bibfield{eprint} field and the string \texttt{pubmed} in the \bibfield{eprinttype} field. This means that:
对于 PubMed 资源,将稳定的 PubMed 标识符放在 \bibfield{eprint} 域中,将字符串 \texttt{pubmed} 放在 \bibfield{eprinttype} 域中。
也就是

\begin{lstlisting}[style=bibtex]{}
url = {http://www.ncbi.nlm.nih.gov/pubmed/<<pmid>>},
\end{lstlisting}
%
%becomes:
会变成:

\begin{lstlisting}[style=bibtex]{}
<<eprint>>     = {<<pmid>>},
<<eprinttype>> = {<<pubmed>>},
\end{lstlisting}
%
%and the standard styles will print «\acr{PMID}: \prm{pmid}» instead of the lengthy \acr{URL}. If hyperref support is enabled, the \prm{pmid} will be a clickable link to PubMed.
并且标准样式会打印出“\acr{PMID}: \prm{pmid}” 来取代冗长的 \acr{URL}。
如果启用 \sty{hyperref} 支持,\prm{pmid} 会变成指向 PubMed 的可点击的链接。

%For handles (\acr{HDL}s), put the handle in the \bibfield{eprint} field and the string \texttt{hdl} in the \bibfield{eprinttype} field:
对于句柄系统\footnote{%
参考 \url{http://www.handle.net/}——译注}
(\acr{HDL}),将句柄放在 \bibfield{eprint} 域中,
将字符串 \texttt{hdl} 放在 \bibfield{eprinttype} 域中:

\begin{lstlisting}[style=bibtex]{}
<<eprint>>     = {<<handle>>},
<<eprinttype>> = {<<hdl>>},
\end{lstlisting}
%
%For Google Books references, put Google's identifier in the \bibfield{eprint} field and the string \texttt{googlebooks} in the \bibfield{eprinttype} field. This means that, for example:
对于 Google Books 资源,将Google标识符放在 \bibfield{eprint} 域中,
将字符串 \texttt{googlebooks} 放在 \bibfield{eprinttype} 域中。
如下例。

\begin{lstlisting}[style=bibtex]{}
url = {http://books.google.com/books?id=<<XXu4AkRVBBoC>>},
\end{lstlisting}
%
%would become:
会变成:

\begin{lstlisting}[style=bibtex]{}
<<eprint>>     = {<<XXu4AkRVBBoC>>},
<<eprinttype>> = {<<googlebooks>>},
\end{lstlisting}
%
%and the standard styles would print «Google Books: |XXu4AkRVBBoC|» instead of the full \acr{URL}. If hyperref support is enabled, the identifier will be a clickable link to Google Books.\footnote{Note that the Google Books \acr{id} seems to be a bit of an <internal> value. As of this writing, there does not seem to be any way to search for an \acr{id} on Google Books. You may prefer to use the \bibfield{url} in this case.}
并且标准样式会打印出“Google Books: |XXu4AkRVBBoC|”代替整个 \acr{URL}。
如果启用了 \sty{hyperref} 支持,该标识符会变成指向 Google Books 的可点击的链接。\footnote{ %
	请注意,Google Books \acr{id} 似乎是一个“内部”值。
	从这份手册开始,似乎没有办法在Google Books 上搜索 \acr{id}。
	此时也许最好使用 \bibfield{url} 域。%
}

%Note that \bibfield{eprint} is a verbatim field. Always give the identifier in its unmodified form. For example, there is no need to replace |_| with |\_|. Also see \secref{aut:cav:epr} on how to add dedicated support for other eprint resources.

请注意 \bibfield{eprint} 是一个抄录模式域,故而总是以未修改的形式给出标识符。
例如没有必要将 |_| 改成 |\_|。
对于如何为其它电子出版资源增加细致的支持,也可以参考 \secref{aut:cav:epr} 节。

\subsubsection{外部摘要和注释}%\subsubsection{External Abstracts and Annotations}
\label{use:use:prf}

%Styles which print the fields \bibfield{abstract} and\slash or \bibfield{annotation} may support an alternative way of adding abstracts or annotations to the bibliography. Instead of including the text in the \file{bib} file, it may also be stored in an external \latex file. For example, instead of saying

打印 \bibfield{abstract} 和/或  \bibfield{annotation} 域的样式可以支持另一种将摘要或注释添加到参考文献的方法。
与将文本包含在 \file{bib} 文件中不同,它也可以保存在一个外部的 \LaTeX 文件中。
例如,除了在 \file{bib} 文件中写入如下内容之外,

\begin{ltxexample}[style=bibtex]
@Article{key1,
  ...
  abstract	  = {This is an abstract of entry `key1'.}
}
\end{ltxexample}
%
%in the \file{bib} file, you may create a file named \path{bibabstract-key1.tex} and put the abstract in this file:
你也可以创建一个名为 \path{bibabstract-key1.tex} 的文件并将摘要放在该文件中:

\begin{ltxexample}
This is an abstract of entry `key1'.
\endinput
\end{ltxexample}
%
%The name of the external file must be the entry key prefixed with \texttt{bibabstract-} or \texttt{bibannotation-}, respectively. You can change these prefixes by redefining \cmd{bibabstractprefix} and \cmd{bibannotationprefix}. Note that this feature needs to be enabled explicitly by setting the package option \opt{loadfiles} from \secref{use:opt:pre:gen}. The option is disabled by default for performance reasons. Also note that any \bibfield{abstract} and \bibfield{annotation} fields in the \file{bib} file take precedence over the external files. Using external files is strongly recommended if you have long abstracts or a lot of annotations since this may increase memory requirements significantly. It is also more convenient to edit the text in a dedicated \latex file. Style authors should see \secref{aut:cav:prf} for further information.
外部文件名必须是条目键分别加上前缀\texttt{bibabstract-} 或 \texttt{bibannotation-}。
你可以通过重定义 \cmd{bibabstractprefix} 和 \cmd{bibannotationprefix} 来改变这些前缀。
请注意,该特性需要通过显式地设置 \secref{use:opt:pre:gen} 中的宏包选项 \opt{loadfiles} 来启用。
缺省情况下出于性能原因该选项是关闭的。
同样要注意的是,\file{bib} 文件中的任何 \bibfield{abstract} 和 \bibfield{annotation} 域都优先于外部文件。
如果你的摘要或注释很长(这会显著增加内存需求),那么强烈推荐使用外部文件。
此外,在专门的 \LaTeX 文件中编辑文本也是很方便的。
样式作者就更多信息应该参考 \secref{aut:cav:prf} 节。

%\subsection{Hints and Caveats}
\subsection{提示和注意事项}
\label{use:cav}

%This section provides additional usage hints and addresses some common problems and potential misconceptions.
本节提供了其它一些使用提示,并介绍了一些常见问题和可能的错误概念。

%\subsubsection{Usage with \acr{KOMA}-Script Classes}
\subsubsection{与 \acr{KOMA}-Script 文档类共用}
\label{use:cav:scr}

%When using \biblatex in conjunction with one of the \sty{scrbook}, \sty{scrreprt}, or \sty{scrartcl} classes, the headings \texttt{bibliography} and \texttt{biblist} from \secref{use:bib:hdg} are responsive to the bibliography"=related options of these classes.\footnote{This applies to the traditional syntax of these options (\opt{bibtotoc} and \opt{bibtotocnumbered}) as well as to the \keyval syntax introduced in \acr{KOMA}-Script 3.x, \ie to \kvopt{bibliography}{nottotoc}, \kvopt{bibliography}{totoc}, and \kvopt{bibliography}{totocnumbered}. The global \kvopt{toc}{bibliography} and \kvopt{toc}{bibliographynumbered} options as well as their aliases are detected as well. In any case, the options must be set globally in the optional argument to \cmd{documentclass}.} You can override the default headings by using the \opt{heading} option of \cmd{printbibliography}, \cmd{printbibheading} and \cmd{printbiblist}. See \secref{use:bib:bib, use:bib:biblist, use:bib:hdg} for details. All default headings are adapted at load-time such that they blend with the behavior of these classes. If one of the above classes is detected, \biblatex will also provide the following additional tests which may be useful in custom heading definitions:

当在 \sty{scrbook}、\sty{scrreprt} 或 \sty{scrartcl} 文档类中使用 \biblatex 时,
\secref{use:bib:hdg} 节中的标题 \texttt{bibliography} 和 \texttt{shorthands} 会与这些文档类的文献相关选项有关。\footnote{%
	这既可以应用到传统的选项语法(\opt{bibtotoc} 和 \opt{bibtotocnumbered}),
	也可以应用到 \acr{KOMA}-Script 3.x 引入的 \keyval 语法,
	即 \kvopt{bibliography}{nottotoc}、\kvopt{bibliography}{totoc} 和 \kvopt{bibliography}{totocnumbered}。
	全局的 \kvopt{toc}{bibliography} 和 \kvopt{toc}{bibliographynumbered} 以及它们的别称也会检测到。
	在任何一种情况下,选项必须在 \cmd{documentclass} 的可选项中全局地设置。
}
可以通过使用 \cmd{printbibliography}、\cmd{printbibheading} 和 \cmd{printshorthands} 的 \opt{heading} 选项来覆盖缺省标题。
详见 \secref{use:bib:bib, use:bib:los, use:bib:hdg} 节。
所有的缺省标题都在导入时调整以使得与这些文档类相称。
如果 \biblatex 探测到这些文档类中的某一个,它还会提供以下额外的测试,这对定制标题定义很有用:

\begin{ltxsyntax}

\cmditem{ifkomabibtotoc}{true}{false}

%Expands to \prm{true} if the class would add the bibliography to the table of contents, and to \prm{false} otherwise.

如果该文档类将参考文献加入目录中则展开为 \prm{true},否则为 \prm{false}。

\cmditem{ifkomabibtotocnumbered}{true}{false}

%Expands to \prm{true} if the class would add the bibliography to the table of contents as a numbered section, and to \prm{false} otherwise. If this test yields \prm{true}, \cmd{ifkomabibtotoc} will always yield \prm{true} as well, but not vice versa.

如果该文档类将参考文献加入目录作为带编号的一节,则展开为 \prm{true},否则为 \prm{false}。
如果该测试结果为 \prm{true},那么 \cmd{ifkomabibtotoc} 也总是为 \prm{true},但反过来不一定。

\end{ltxsyntax}

\subsubsection{与Memoir文档类共用}%\subsubsection{Usage with the Memoir Class}
\label{use:cav:mem}

%When using \biblatex with the \sty{memoir} class, most class facilities for adapting the bibliography have no effect. Use the corresponding facilities of this package instead (\secref{use:bib:bib, use:bib:hdg, use:bib:nts}). Instead of redefining \sty{memoir}'s \cmd{bibsection}, use the \opt{heading} option of \cmd{printbibliography} and \cmd{defbibheading} (\secref{use:bib:bib, use:bib:hdg}). Instead of \cmd{prebibhook} and \cmd{postbibhook}, use the \opt{prenote} and \opt{postnote} options of \cmd{printbibliography} and \cmd{defbibnote} (\secref{use:bib:bib, use:bib:nts}). All default headings are adapted at load-time such that they blend well with the default layout of this class. The default headings \texttt{bibliography} and \texttt{biblist} (\secref{use:bib:hdg}) are also responsive to \sty{memoir}'s \cmd{bibintoc} and \cmd{nobibintoc} switches. The length register \len{bibitemsep} is used by \biblatex in a way similar to \sty{memoir} (\secref{use:fmt:len}). This section also introduces some additional length registers which correspond to \sty{memoir}'s \cmd{biblistextra}. Lastly, \cmd{setbiblabel} does not map to a single facility of the \biblatex package since the style of all labels in the bibliography is controlled by the bibliography style. See \secref{aut:bbx:env} in the author section of this manual for details. If the \sty{memoir} class is detected, \biblatex will also provide the following additional test which may be useful in custom heading definitions:

当在 \sty{memoir} 文档类中使用 \biblatex 时,大部分调整参考文献的文档类工具都没有效果。
相反,请使用本宏包的相应工具(\secref{use:bib:bib, use:bib:hdg, use:bib:nts} 节)。
使用 \cmd{printbibliography} 和 \cmd{defbibheading}(\secref{use:bib:bib, use:bib:hdg} 节)的 \opt{heading} 选项,
而不要重定义 \sty{memoir} 的 \cmd{bibsection}。
使用 \cmd{printbibliography} 和 \cmd{defbibnote}(\secref{use:bib:bib, use:bib:nts} 节)的 \opt{prenote} 和 \opt{postnote} 来取代 \cmd{prebibhook} 和 \cmd{postbibhook}。
所有缺省标题都在导入时调整以与该文档类的缺省布局相称。
缺省的标题 \texttt{bibliography} 和 \texttt{shorthands} (\secref{use:bib:hdg} 节)也与 \sty{memoir} 的 \cmd{bibintoc} 和 \cmd{nobibintoc} 开关相对应。
长度计数器 \len{bibitemsep} 在 \biblatex 中的使用方法与在 \sty{memoir} 类似(\secref{use:fmt:len} 节)。
本节还解释额外一些对应于 \sty{memoir} 中 \cmd{biblistextra} 的长度计数器。
最后,\cmd{setbiblabel} 并不能对应于 \biblatex 宏包的某一单独工具,因为参考文献中所有标签的样式由参考文献样式控制。
详见本手册的 \secref{aut:bbx:env}。
如果 \biblatex 探测到 \sty{memoir} 文档类的使用,它还会提供以下额外的测试,这对定制标题定义很有用:

\begin{ltxsyntax}

\cmditem{ifmemoirbibintoc}{true}{false}

%Expands to \prm{true} or \prm{false}, depending on \sty{memoir}'s \cmd{bibintoc} and \cmd{nobibintoc} switches. This is a \latex frontend to \sty{memoir}'s \cmd{ifnobibintoc} test. Note that the logic of the test is reversed.

取决于 \sty{memoir} 的 \cmd{bibintoc} 和 \cmd{nobibintoc} 开关,可以展开为 \prm{true} 或 \prm{false}。
这是对应于 \sty{memoir} 的 \cmd{ifnobibintoc} 测试的 \LaTeX 前端。
请注意该测试的逻辑可以反过来。

\end{ltxsyntax}

\subsubsection{引用中的页码数}%\subsubsection{Page Numbers in Citations}
\label{use:cav:pag}

%If the \prm{postnote} argument to a citation command is a page number or page range, \biblatex will automatically prefix it with <p.> or <pp.> by default. This works reliably in typical cases, but sometimes manual intervention may be required. In this case, it is important to understand how this argument is handled in detail. First, \biblatex checks if the postnote is an Arabic or Roman numeral (case insensitive). If this test succeeds, the postnote is considered as a single page or other number which will be prefixed with <p.> or some other string which depends on the \sty{pagination} field (see \secref{bib:use:pag}). If it fails, a second test is performed to find out if the postnote is a range or a list of Arabic or Roman numerals. If this test succeeds, the postnote will be prefixed with <pp.> or some other string in the plural form. If it fails as well, the postnote is printed as is. Note that both tests expand the \prm{postnote}. All commands used in this argument must therefore be robust or prefixed with \cmd{protect}. Here are a few examples of \prm{postnote} arguments which will be correctly recognized as a single number, a range of numbers, or a list of numbers, respectively:

如果一个引用命令的 \prm{postnote} 选项是页码数或页码范围,
那么 \biblatex 会自动给其增加缺省前缀“p.”或“pp.”。
在通常情况下这很可靠,但有时也需要手动调整。
这时理解该选项怎样处理的细节就很重要。
首先 \biblatex 检查后注是否是阿拉伯或罗马数字(大小写不敏感)。
如果该测试成功,那么该后注会被认为是一个单独页码或者其它数字,
这时会被加上前缀“p.”或其它取决于 \bibfield{pagination} 域(见 \secref{bib:use:pag} 节)的字符串。
如果测试没有成功,那么会执行第二项测试来检测该后注是否是一个区间或者一列阿拉伯或罗马数字。
如果该测试成功,那么该后注会被加上前缀“pp.”或其它复数形式的字符串。
如果该测试也没有成功,该后注会依原样打印。
请注意这两项测试都会展开 \prm{postnote}。
因此所有在该选项中使用的命令都必须是鲁棒的或者用 \cmd{protect} 加以保护。
这里分别是一些 \prm{postnote} 选项会被正确识别为单独数字、数字范围或者一列数字的一些例子:

\begin{ltxexample}
\cite[25]{key}
\cite[vii]{key}
\cite[XIV]{key}
\cite[34--38]{key}
\cite[iv--x]{key}
\cite[185/86]{key}
\cite[XI \& XV]{key}
\cite[3, 5, 7]{key}
\cite[vii--x; 5, 7]{key}
\end{ltxexample}
%
%In some other cases, however, the tests may get it wrong and you need to resort to the auxiliary commands \cmd{pno}, \cmd{ppno}, and \cmd{nopp} from \secref{use:cit:msc}. For example, suppose a work is cited by a special pagination scheme consisting of numbers and letters. In this scheme, the string <|27a|> would mean <page~27, part~a>. Since this string does not look like a number or a range to \biblatex, you need to force the prefix for a single number manually:
然而在其它一些情况该测试会失败,
此时需要考虑 \secref{use:cit:msc} 节的一些辅助命令 \cmd{pno}、\cmd{ppno} 和 \cmd{nopp}。
例如,假设一部作品由一种包含数字和字母的特殊页码标记格式所引用。
在这种格式中,字符串“|27a|”的意思是“page~27, part~a”。
因为对于 \biblatex 来说该字符串并不像数字或者数字范围,因此你需要手动强制加上单独页码的前缀:

\begin{ltxexample}
\cite[\pno~27a]{key}
\end{ltxexample}
%
%There is also a \cmd{ppno} command which forces a range prefix as well as a \cmd{nopp} command which suppresses all prefixes:
同样地,\cmd{ppno} 命令会强制为范围前缀,而 \cmd{nopp} 命令会取消所有的前缀:

\begin{ltxexample}
\cite[\ppno~27a--28c]{key}
\cite[\nopp 25]{key}
\end{ltxexample}
%
%These commands may be used anywhere in the \prm{postnote} argument. They may also be used multiple times. For example, when citing by volume and page number, you may want to suppress the prefix at the beginning of the postnote and add it in the middle of the string:
这些命令可以用于 \prm{postnote} 选项的任何地方,也可以被多次使用。
例如,当以卷数和页码数引用时,你或许希望在后注的开始取消前缀,而在字符串的中间加上:

\begin{ltxexample}
\cite[VII, \pno~5]{key}
\cite[VII, \pno~3, \ppno~40--45]{key}
\cite[see][\ppno~37--46, in particular \pno~40]{key}
\end{ltxexample}
%
%There are also two auxiliary command for suffixes like <the following page(s)>. Instead of inserting such suffixes literally (which would require \cmd{ppno} to force a prefix):
还有两个用于形如“the following page(s)”的后缀的辅助命令。
使用辅助命令 \cmd{psq} 和 \cmd{psqq} 来代替用文本插入这样的后缀(这要求 \cmd{ppno} 来强加一个前缀):

\begin{ltxexample}
\cite[\ppno~27~sq.]{key}
\cite[\ppno~55~sqq.]{key}
\end{ltxexample}
%
%use the auxiliary commands \cmd{psq} and \cmd{psqq}. Note that there is no space between the number and the command. This space will be inserted automatically and may be modified by redefining the macro \cmd{sqspace}.
请注意数字和命令之前没有空格。该空格会自动插入并可以通过重定义 \cmd{sqspace} 宏来修改。

\begin{ltxexample}
\cite[27\psq]{key}
\cite[55\psqq]{key}
\end{ltxexample}
%
%Since the postnote is printed without any prefix if it includes any character which is not an Arabic or Roman numeral, you may also type the prefix manually:
由于当后注包括任何非阿拉伯或罗马数字时将会以不带任何前缀的方式打印,
也可以手动输入前缀:

\begin{ltxexample}
\cite[p.~5]{key}
\end{ltxexample}
%
%It is possible to suppress the prefix on a per"=entry basis by setting the \bibfield{pagination} field of an entry to <\texttt{none}>, see \secref{bib:use:pag} for details. If you do not want any prefixes at all or prefer to type them manually, you can also disable the entire mechanism in the document preamble or the configuration file as follows:
可以通过设置条目的 \bibfield{pagination} 域为 “\texttt{none}”来基于每一条目取消前缀,详见 \secref{bib:use:pag} 节。
如果你不需要任何前缀或者更想要手动输入,也可以在导言区或者配置文件中整个地关闭该机制,如下所示:

\begin{ltxexample}
\DeclareFieldFormat{postnote}{#1}
\end{ltxexample}
%
%The \prm{postnote} argument is handled as a field and the formatting of this field is controlled by a field formatting directive which may be freely redefined. The above definition will simply print the postnote as is. See \secref{aut:cbx:fld, aut:bib:fmt} in the author guide for further details.
\prm{postnote} 选项会像条目域一样处理,该域的格式由域格式指令来控制,而该指令可以自由地重定义。
以上的定义会简单地将后注依原样打印。
更多细节可以参见作者指南部分的 \secref{aut:cbx:fld, aut:bib:fmt} 节。

%\subsubsection{Name Parts and Name Spacing}
\subsubsection{姓名组成部分及其间距}
\label{use:cav:nam}

%The \biblatex package gives users and style authors very fine"=grained control of name spacing and the line"=breaking behavior of names. The commands discussed in the following are documented in \secref{use:fmt:fmt,aut:fmt:fmt}. This section is meant to give an overview of how they are put together. A note on terminology: a name \emph{part} is a basic part of the name, for example the given or the family name. Each part of a name may be a single name or it may be composed of multiple names. For example, the name part \enquote*{given name} may be composed of a given and a middle name. The latter are referred to as name \emph{elements} in this section. Let's consider a simple name first: \enquote{John Edward Doe}. This name is composed of the following parts:

\biblatex 宏包可以使用户和样式作者对姓名空格和换行进行精细的控制。
下面讨论的命令在本文档的 \secref{use:fmt:fmt,aut:fmt:fmt} 节。
本节的目的在于大致介绍如何将这些命令结合起来。
关于术语的注记:名\emph{部分}是姓名中的基本成分,例如first name 或last name。
姓名的每一部分可以是单个的名或者包含多个名。
例如,名部分 \enquote*{first name} 可以包含 first name 和 middle name。
后者可以认为是本节所说的名\emph{元素}。
我们首先考虑一个简单的名字 \enquote{John Edward Doe},它包含了如下部分:

\begin{nameparts}
Given	& John Edward \\
Prefix	& --- \\
Family	& Doe \\
Suffix	& --- \\
\end{nameparts}
%
%The spacing, punctuation and line"=breaking behavior of names is controlled by six macros:
姓名中的空格、标点和断行行为由六个宏所控制:

\begin{namedelims}
a & \cmd{bibnamedelima} &
%Inserted by the backend after the first element of every name part if that element is less than three characters long and before the last element of every name part.
由后端程序插入在每一名部分的第一个元素后(如果该元素少于三个字符长度),在每一名部分的最后元素之前。\\
b & \cmd{bibnamedelimb} &
%Inserted by the backend between all elements of a name part where \cmd{bibnamedelima} does not apply.
由后端程序插入在名部分的元素之间且 \cmd{bibnamedelima} 没有使用之处。\\
c & \cmd{bibnamedelimc} &
%Inserted by a formatting directive between the name prefix and the family name if \kvopt{useprefix}{true}. If \kvopt{useprefix}{false}, \cmd{bibnamedelimd} is used instead.
当 \kvopt{useprefix}{true} 时,由格式指令插入在 name prefix 和 last name 之间。
如果 \kvopt{useprefix}{false},将使用 \cmd{bibnamedelimd}。\\
d & \cmd{bibnamedelimd} &
%Inserted by a formatting directive between name parts where \cmd{bibnamedelimc} does not apply.
由格式指令插入在名部分之间且 \cmd{bibnamedelimc} 没有使用之处。\\
i & \cmd{bibnamedelimi} &
%Replaces \cmd{bibnamedelima/b} after initials
在首字符缩写之后代替 \cmd{bibnamedelima/b} 的命令。\\
p & \cmd{revsdnamepunct} &
%Inserted by a formatting directive after the family name when the name parts are reversed.
当名部分顺序反过来时,由格式指令插入在 last name 之后。
\end{namedelims}
%
%This is how the delimiters are employed:
以下展示了如何使用这些分隔符:

\begin{namesample}
\item John\delim{~}{a}Edward\delim{ }{d}Doe
\item Doe\delim{,}{p}\delim{ }{d}John\delim{~}{a}Edward
\end{namesample}
%
%Initials in the \file{bib} file get a special delimiter:
\file{bib} 文件中的首字符缩写会有一个特别的分隔符:

\begin{namesample}
\item J.\delim{~}{i}Edward\delim{ }{d}Doe
\end{namesample}
%
%Let's consider a more complex name: \enquote{Charles-Jean Étienne Gustave Nicolas de La Vallée Poussin}. This name is composed of the following parts:
考虑一个更复杂的名字:\enquote{Charles-Jean Étienne Gustave Nicolas de La Vallée Poussin}。
它包含了如下几部分:

\begin{nameparts}
Given	& Charles-Jean Étienne Gustave Nicolas \\
Prefix	& de \\
Family	& La Vallée Poussin \\
Suffix	& --- \\
\end{nameparts}
%
%The delimiters:
分隔符为:

\begin{namesample}
\item Charles-Jean\delim{~}{b}Étienne\delim{~}{b}Gustave\delim{~}{a}Nicolas\delim{ }{d}%
      de\delim{ }{c}%
      La\delim{~}{a}Vallée\delim{~}{a}Poussin
\end{namesample}
%
%Note that \cmd{bibnamedelima/b/i} are inserted by the backend. The backend processes the name parts and takes care of the delimiters between the elements that make up a name part, processing each part individually. In contrast to that, the delimiters between the parts of the complete name (\cmd{bibnamedelimc/d}) are added by name formatting directives at a later point in the processing chain. The spacing and punctuation of initials is also handled by the backend and may be customized by redefining the following three macros:
请注意 \cmd{bibnamedelima/b/i} 由后端程序插入。
后端程序处理名部分并考虑组成名部分的元素之间的分隔符,从而分别处理每一部分。
与此相反,全名的名部分之间的分隔符(\cmd{bibnamedelimc/d})由名称格式指令在处理过程的之后时间点添加。
首字符缩写的空格和标点同样由后端程序处理,并且可以通过重定义以下三个宏来定制:

\begin{namedelims}
a & \cmd{bibinitperiod} &
%Inserted by the backend after initials.
由后端程序插入在首字母缩写之后。\\
b & \cmd{bibinitdelim} &
%Inserted by the backend between multiple initials.
由后端程序插入在多个首字母缩写之间。\\
c & \cmd{bibinithyphendelim} &
%Inserted by the backend between the initials of hyphenated name parts, replacing \cmd{bibinitperiod} and \cmd{bibinitdelim}.
由后端程序插入在带有连字符的名部分中首字母缩写之间,用以代替 \cmd{bibinitperiod} 和 \cmd{bibinitdelim}。\\
\end{namedelims}
%
%This is how they are employed:
以下是使用方式:

\begin{namesample}
\item J\delim{.}{a}\delim{~}{b}E\delim{.}{a} Doe
\item K\delim{.-}{c}H\delim{.}{a} Mustermann
\end{namesample}

\subsubsection{文献 filter 和引用标签}%\subsubsection{Bibliography Filters and Citation Labels}
\label{use:cav:lab}

%The citation labels generated by this package are assigned to the full list of references before it is split up by any bibliography filters. They are guaranteed to be unique across the entire document (or a \env{refsection} environment), no matter how many bibliography filters you are using. When using a numeric citation scheme, however, this will most likely lead to discontinuous numbering in split bibliographies. Use the \opt{defernumbers} package option to avoid this problem. If this option is enabled, numeric labels are assigned the first time an entry is printed in any bibliography.

本宏包生成的引用标签在被文献filter分开之前就被分配给整个文献列表。
因此能够确保在整个文档(或者一个 \env{refsection} 环境中)是唯一的,无论使用多少文献filter。
然而,当使用数值型引用格式时,这很可能会导致在各个分片参考文献中的编号不是连续的。
使用 \opt{defernumber} 宏包选项可以避免这一问题。
如果启用该选项,数值标签会在任一文献条目中第一次打印时才被分配。

\subsubsection{参考文献标题中的活动字符}%\subsubsection{Active Characters in Bibliography Headings}
\label{use:cav:act}

%Packages using active characters, such as \sty{babel}, \sty{polyglossia}, \sty{csquotes}, or \sty{underscore}, usually do not make them active until the beginning of the document body to avoid interference with other packages. A typical example of such an active character is the Ascii quote |"|, which is used by various language modules of the \sty{babel}/\sty{polyglossia} packages. If shorthands such as |"<| and |"a| are used in the argument to \cmd{defbibheading} and the headings are defined in the document preamble, the non-active form of the characters is saved in the heading definition. When the heading is typeset, they do not function as a command but are simply printed literally. The most straightforward solution consists in moving \cmd{defbibheading} after |\begin{document}|. Alternatively, you may use \sty{babel}'s \cmd{shorthandon} and \cmd{shorthandoff} commands to temporarily make the shorthands active in the preamble. The above also applies to bibliography notes and the \cmd{defbibnote} command.

\sty{babel}、\sty{polyglossia}、\sty{csquotes} 和 \sty{underscore} 等使用活动字符的宏包通常直到正文开始才将这些字符激活,这样可以避免与其它宏包的冲突。
一个典型的活动字符例子是 Ascii 引号 |"|,
它用于 \sty{babel}/\sty{polyglossia} 宏包的不同语言模块。
如果在 \cmd{defbibheading} 的选项中使用 |"<| 和 |"a| 等速记方式,并且标题定义在导言区中,
那么标题定义中保存的是字符的非激活形式。
因此当标题打印出来时,它们不会像命令一样起作用而仅仅依照原文打印。
最直接的解决方法是将 \cmd{defbibheading} 放在  |\begin{document}| 之后。
此外,你也可以使用 \sty{babel} 的 \cmd{shorthandon} 和 \cmd{shorthandoff} 命令来临时在导言区中激活这些简记方式。
这同样应用于文献注记和 \cmd{defbibnote} 命令。

\subsubsection{参考文献分节和分段中的编组}%\subsubsection{Grouping in Reference Sections and Segments}
\label{use:cav:grp}

%All \latex environments enclosed in \cmd{begin} and \cmd{end} form a group. This may have undesirable side effects if the environment contains anything that does not expect to be used within a group. This issue is not specific to \env{refsection} and \env{refsegment} environments, but it obviously applies to them as well. Since these environments will usually enclose much larger portions of the document than a typical \env{itemize} or similar environment, they are simply more likely to trigger problems related to grouping. If you observe any malfunctions after adding \env{refsection} environments to a document (for example, if anything seems to be <trapped> inside the environment), try the following syntax instead:

所有在 \cmd{begin} 和 \cmd{end} 中的 \LaTeX 环境形成了一个分组。
如果该环境包含一些没有在组内使用的东西,那么可能会引起一些不良反应。
这个问题并不仅仅针对 \env{refsection} 和 \env{refsegment} 环境,但显然也包括它们。
由于这些环境通常比典型的 \env{itemize} 或其它环境包含更多文本,因此它们自然更有可能引起涉及到分组的问题。
如果你在添加 \env{refsection} 环境之后观察到任何不正常的现象(例如,如果环境内有任何“受限”情况),
请尝试使用以下语句来代替:

\begin{ltxexample}
\chapter{...}
<<\refsection>>
...
<<\endrefsection>>
\end{ltxexample}
%
%This will not from a group, but otherwise works as usual. As far as \biblatex is concerned, it does not matter which syntax you use. The alternative syntax is also supported by the \env{refsegment} environment. Note that the commands \cmd{newrefsection} and \cmd{newrefsegment} do not form a group. See \secref{use:bib:sec, use:bib:seg} for details.
这不会形成一个分组,但是像正常一样工作。
就 \biblatex 而言,它并不影响你使用哪种语句。
\env{refsegment} 环境也支持这种语句。
请注意,命令 \cmd{newrefsection} 和 \cmd{newrefsegment} 不会形成分组。
详见 \secref{use:bib:sec, use:bib:seg} 节。

\subsection{使用备选的 \BibTeX 后端}%\subsection{Using the fallback \bibtex\ backend}
\label{use:bibtex}

%To utilise all of the features described here, \biblatex must be used with the
%\biber program as a backend. Indeed, the documentation in general assumes this. However, for a \emph{limited} subset of use cases it is possible to use the
%long-established \bibtex program (either the 7-bit \texttt{bibtex} or
%8-bit \texttt{bibtex8}) as the supporting backend. This works in much the
%same way as for \biber with the only proviso being that \bibtex is much more
%limited as a backend.

\biblatex 必须使用 \biber 程序作为后端才能使用这里描述的所有特性。实际上,本文档正是默认这一假定。不过,如果只是使用\emph{受限制}的一部分功能,也可以使用历史悠久的 \BibTeX 程序(7-bit 的  \bin{bibtex} 或者 8-bit 的 \bin{bibtex8})作为后端程序。


%Using \bibtex as the backend requires that the option \opt{backend=bibtex}
%or \opt{backend=bibtex8} is given at load time. The \biblatex package will
%then write appropriate data for use by \bibtex into the auxiliary file(s)
%and a special data file (automatically included in those to be read by
%\bibtex).  The \bibtex(8) program should then be run on each auxiliary file:
%\biblatex will list all of the required files in the log.

使用 \BibTeX 作为后端程序需要在载入时开启选项 \opt{backend=bibtex} 或 \opt{backend=bibtex8}.
\biblatex 宏包随后会将合适的数据写入辅助文件以及特殊的数据文件中以供 \BibTeX 使用
(自动包含那些被 \BibTeX 读取的文件)。
然后 \bin{bibtex(8)} 程序应当运行在每一辅助文件上:
\biblatex 会在日志文件中列出所有所需的文件。

%Key limitations of the \bibtex backend are:
\BibTeX 后端的主要局限有:

\begin{itemize}

\item %Sorting is global and is limited to Ascii ordering
排序是全局的,而且只限于按照 Ascii 顺序。


\item %No re-encoding is possible and thus database entries must be in
%LICR form to work reliably
不可以重编码,因此数据库中的条目必须按照 \LaTeX 内部字符表示\footnote{\LaTeX{} Internal Character Representation, LICR——译注}的形式,才可以确保程序可靠。

\item %The data model is fixed
数据模型是固定的。

\item %Cross-referencing is more limited and entry sets must be written into the \path{.bib} file
交叉引用有限制,条目集必须写入 \path{.bib} 文件。
\item %Fixed memory capacity (using the \verb|--wolfgang| option with
%\verb|bibtex8| is strongly recommended to minimize the likelihood of
%an issue here)
内存容量有限。在 \verb|bibtex8| 中,强烈建议使用 \verb|--wolfgang| 选项以尽量减少这一问题的可能性。

\end{itemize}

\endinput


%注意一些命令比如\cmd,\file,<| |>,而$符号直接用\$表示即可。

%\section{Author Guide}
\section{样式作者指南}
\label{aut}

%This part of the manual documents the author interface of the \biblatex package. The author guide covers everything you need to know in order to write new citation and bibliography styles or localisation modules. You should read the user guide first before continuing with this part of the manual.

本节是样式作者指南,主要介绍\biblatex 包的接口。该指南囊括了设计参考文献著录和标注样式或者本地化模型所需了解的所有知识。在阅读本节内容前最好先阅读上一节的用户手册。

\subsection{概述}%Overview
\label{aut:int}
在讨论\biblatex 提供的命令和工具之前,我们首先介绍一些基本概念。\biblatex 包以一种特殊方式使用辅助文件。最值得注意的是当使用\bibtex 后端程序时,\file{bbl} 文件的使用方式存在差别,即只有一个\file{bst} 文件可用来实现结构化的数据接口,该文件并非用来输出可打印数据。

使用\latex 的标准参考文献工具,一个文档通常包含任意数量的文献引用命令,以及常放在文档末尾的\cmd{bibliographystyle} 和\cmd{bibliography} 命令。文献引用命令在文档中的位置是任意的,而\cmd{bibliographystyle} 和\cmd{bibliography} 命令则标记了打印参考文献表的位置,比如:
%Before we get to the commands and facilities provided by \biblatex, we will have a look at some of its fundamental concepts. The \biblatex package uses auxiliary files in a special way. Most notably, the \file{bbl} file is used differently and when using \bibtex as the backend, there is only one \file{bst} file which implements a structured data interface rather than exporting printable data. With \latex's standard bibliographic facilities, a document includes any number of citation commands in the document body plus \cmd{bibliographystyle} and \cmd{bibliography}, usually towards the end of the document. The location of the former is arbitrary, the latter marks the spot where the list of references is to be printed:

\begin{ltxexample}
\documentclass{...}
\begin{document}
\cite{...}
...
\bibliographystyle{...}
\bibliography{...}
\end{document}
\end{ltxexample}
%

%Processing this files requires that a certain procedure be followed. This procedure is as follows:
处理这些文件遵循一定的流程,其过程如下:

\begin{enumerate}

\item 运行\bin{latex}: 第一次运行\bin{latex}, 在\file{aux} 文件中写入\cmd{bibstyle} 和 \cmd{bibdata} 命令,以及所有标注的\cmd{citation} 命令。这时,各引文标注\footnote{译者:这里的references译为引文标注,指引用命令在正文中产生的标注,这个标注由标签label构成。} 是未定义的,因为 \latex 等待\bibtex 提供需要的数据,当然参考文献表也未生成。
    %Run \bin{latex}: On the first run, \cmd{bibstyle} and \cmd{bibdata} commands are written to the \file{aux} file, along with \cmd{citation} commands for all citations. At this point, the references are undefined because \latex is waiting for \bibtex to supply the required data. There is also no bibliography yet.


\item 运行\bin{bibtex}: \bibtex 在\file{bbl} 文件中写入一个\env{thebibliography} 环境,用以为\file{aux} 文件中\cmd{citation} 命令提供所需的所有\gls{条目},这些条目的数据来自\file{bib} 文件。
    %Run \bin{bibtex}: \bibtex writes a \env{thebibliography} environment to the \file{bbl} file, supplying all entries from the \file{bib} file which were requested by the \cmd{citation} commands in the \file{aux} file.


\item 运行\bin{latex}: 第二次运行\bin{latex},\env{thebibliography} 环境中的\cmd{bibitem} 命令为各参考文献条目在\file{aux} 文件中写入\cmd{bibcite} 命令。这些\cmd{bibcite} 命令定义的标签将用于\cmd{cite} 命令。然而,各引文标注仍然未定义,因为这些标签在最后一次运行\bin{latex} 前仍是未知的。
    %Run \bin{latex}: Starting with the second run, the \cmd{bibitem} commands in the \env{thebibliography} environment write one \cmd{bibcite} command for each bibliography entry to the \file{aux} file. These \cmd{bibcite} commands define the citation labels used by \cmd{cite}. However, the references are still undefined because the labels are not available until the end of this run.


\item 运行\bin{latex}: 第三次运行,随着导言区最后读入了\file{aux} 文件,引文标注的标签定义完成。至此所有的标注可以正确打印。
    %Run \bin{latex}: Starting with the third run, the citation labels are defined as the \file{aux} file is read in at the end of the preamble. All citations can now be printed.

\end{enumerate}

注意到所有的参考文献数据都以最终格式(指最后打印出的格式)写入\file{bbl} 文件。该文件的读取和处理如同任何文档中的可打印章节。例如,考虑在一个\file{bib} 文件中有如下条目:
%Note that all bibliographic data is written to the \file{bbl} file in the final format. The \file{bbl} file is read in and processed like any printable section of the document. For example, consider the following entry in a \file{bib} file:

\begin{lstlisting}[style=bibtex]{}
@Book{companion,
  author    = {Michel Goossens and Frank Mittelbach and Alexander Samarin},
  title     = {The LaTeX Companion},
  publisher = {Addison-Wesley},
  address   = {Reading, Mass.},
  year      = {1994},
}
\end{lstlisting}
%
根据\path{plain.bst} 样式,\bibtex 在\file{bbl} 文件中输出该条目如下:
%With the \path{plain.bst} style, \bibtex exports this entry to the \file{bbl} file as follows:

\begin{ltxexample}
\bibitem{companion}
Michel Goossens, Frank Mittelbach, and Alexander Samarin.
\newblock {\em The LaTeX Companion}.
\newblock Addison-Wesley, Reading, Mass., 1994.
\end{ltxexample}
%
默认情况下,\latex 生成顺序编码制标注标签,因此\cmd{bibitem} 命令在\file{aux} 文件中写入的行如下所示:
%By default, \latex generates numeric citation labels, hence \cmd{bibitem} writes lines such as the following to the \file{aux} file:

\begin{ltxexample}
\bibcite{companion}{1}
\end{ltxexample}
%
要实现一个不同的标注标签样式,意味着需要通过\file{aux} 文件传递更多的数据。比如,当使用\sty{natbib} 包时,\file{aux} 文件包含的标注(或引用)信息行,如下所示:
%Implementing a different citation style implies that more data has to be transferred via the \file{aux} file. With the \sty{natbib} package, for example, the \file{aux} file contains lines like this one:

\begin{ltxexample}
\bibcite{companion}{{1}{1994}{{Goossens et~al.}}{{Goossens, Mittelbach,
and Samarin}}}
\end{ltxexample}
%

\biblatex 包支持任何格式的标注标签,因此标注命令需要访问完整的参考文献数据。观察同样需要在标注中提供完整参考文献数据的\sty{jurabib} 包的输出,我们会更理解这一需求对上述处理过程意味着什么。
%The \biblatex package supports citations in any arbitrary format, hence citation commands need access to all bibliographic data. What this would mean within the scope of the procedure outlined above becomes obvious when looking at the output of the \sty{jurabib} package which also makes all bibliographic data available in citations:

\begin{ltxexample}
\bibcite{companion}{{Goossens\jbbfsasep Mittelbach\jbbstasep Samarin}%
  {}{{0}{}{book}{1994}{}{}{}{}{Reading, Mass.\bpubaddr{}Addison-Wesley%
  \bibbdsep{} 1994}}{{The LaTeX Companion}{}{}{2}{}{}{}{}{}}{\bibnf
  {Goossens}{Michel}{M.}{}{}\Bibbfsasep\bibnf{Mittelbach}{Frank}{F.}%
  {}{}\Bibbstasep\bibnf{Samarin}{Alexander}{A.}{}{}}{\bibtfont{The
  LaTeX Companion}.\ \apyformat{Reading, Mass.\bpubaddr{}
  Addison-Wesley\bibbdsep{} 1994}}}
\end{ltxexample}
%

在这种情况下,整个\env{thebibliography} 环境的内容是通过\file{aux} 文件进行有效传递的。数据首先从\file{bbl} 文件中读取出来,写入到\file{aux} 中,然后再从\file{aux} 读出保存到内存中。参考文献表本身的生成也需要先读入\file{bbl} 文件。这也使得\biblatex 包被迫通过\file{aux} 文件回收所有数据。这意味着上述过程处理过度且多余,因为不管怎样数据都需要保存到内存中。
%In this case, the contents of the entire \env{thebibliography} environment are effectively transferred via the \file{aux} file. The data is read from the \file{bbl} file, written to the \file{aux} file, read back from the \file{aux} file and then kept in memory. The bibliography itself is still generated as the \file{bbl} file is read in. The \biblatex package would also be forced to cycle all data through the \file{aux} file. This implies processing overhead and is also redundant because the data has to be kept in memory anyway.

这种传统的处理过程都基于一个假设,即条目的完整数据只是参考文献表需要而所有的标注都使用短标签。这对于有内存限制的情况是非常高效的,但也意味着它很难扩展。这就是\biblatex 采取另一种方式的原因。采用新的方式,首先文档结构略有变化。取消在文档内使用\cmd{bibliography} 命令,数据库文件由导言区的\cmd{addbibresource} 命令指定,完全忽略\cmd{bibliographystyle} 命令(所有的功能都将由包选项控制),参考文献表则使用\cmd{printbibliography} 命令打印:
%The traditional procedure is based on the assumption that the full bibliographic data of an entry is only required in the bibliography and that all citations use short labels. This makes it very effective in terms of memory requirements, but it also implies that it does not scale well. That is why \biblatex takes a different approach. First of all, the document structure is slightly different. Instead of using \cmd{bibliography} in the document body, database files are specified in the preamble with \cmd{addbibresource}, \cmd{bibliographystyle} is omitted entirely (all features are controlled by package options), and the bibliography is printed using \cmd{printbibliography}:

\begin{ltxexample}
\documentclass{...}
\usepackage[...]{biblatex}
\addbibresource{...}
\begin{document}
\cite{...}
...
\printbibliography
\end{document}
\end{ltxexample}
%

为简化整个过程,\biblatex 基本上以应用\file{aux} 文件的方式应用\file{bbl} 文件,并舍弃了\cmd{bibcite} 命令。于是,我们得到如下流程:
%In order to streamline the whole procedure, \biblatex essentially employs the \file{bbl} file like an \file{aux} file, rendering \cmd{bibcite} obsolete. We then get the following procedure:

\begin{enumerate}

\item 运行\bin{latex}:第一步类似于上述的传统方式:将\cmd{bibstyle} 和 \cmd{bibdata} 以及所有引用的\cmd{citation} 命令写入到\file{aux} 文件中(以\bibtex 为后端程序)或者写到\file{bcf} 文件中(以\biber 为后端程序)。然后等待后端程序提供需要的数据。当以\bibtex 为后端程序时,\biblatex 使用一个特殊的\file{bst} 文件,该文件用于实现\bibtex 后端程序的数据接口,因此\cmd{bibstyle} 命令必须是|\bibstyle{biblatex}|。
    %Run \bin{latex}: The first step is similar to the traditional procedure described above: \cmd{bibstyle} and \cmd{bibdata} commands are written to the \file{aux} file (\bibtex backend) or \file{bcf} file (\biber backend), along with \cmd{citation} commands for all citations. We then wait for the backend to supply the required data. With \bibtex as a backend, since \biblatex uses a special \file{bst} file which implements its data interface on the \bibtex end, the \cmd{bibstyle} command is always |\bibstyle{biblatex}|.


\item 运行\bin{biber} 或 \bin{bibtex}:后端程序为辅助文件中所有\cmd{citation} 命令提供所需的条目,这些条目的数据来自\file{bib} 文件。然而,它并不在\file{bbl} 文件中写出一个可打印的参考文献表,而是一个结构化表达的参考文献数据。类似于\file{aux} 文件,读入该\file{bbl} 文件时不打印任何东西,仅是将数据存入内存中。
    %Run \bin{biber} or \bin{bibtex}: The backend supplies those entries from the \file{bib} file which were requested by the \cmd{citation} commands in the auxiliary file. However, it does not write a printable bibliography to the \file{bbl} file, but rather a structured representation of the bibliographic data. Just like an \file{aux} file, this \file{bbl} file does not print anything when read in. It merely puts data in memory.


\item 运行\bin{latex}: 第二次运行,\file{bbl} 文件在文档正文开始的时候处理,类似于\file{aux} 文件。从这开始,所有参考文献数据都已在内存中,所以所有的引用标注都可以正确打印。\footnote{如果\opt{defernumbers} 包选项启用, \biblatex 以类似于传统过程的一种算法来生成顺序制标签。这种情况下,这些数字将在参考文献表打印时指定,且需从后端辅助文件中回收。因此需要额外再运行一次\latex 以在标注中应用。 } 引用命令不仅可以访问预定义的标签,还可以访问完整的参考文献数据。参考文献表由内存中的相同数据生成,可以根据需要进行筛选和划分。
    %Run \bin{latex}: Starting with the second run, the \file{bbl} file is processed right at the beginning of the document body, just like an \file{aux} file. From this point on, all bibliographic data is available in memory so that all citations can be printed right away.\footnote{If the \opt{defernumbers} package option is enabled \biblatex uses an algorithm similar to the traditional procedure to generate numeric labels. In this case, the numbers are assigned as the bibliography is printed and then cycled through the backend auxiliary file. It will take an additional \latex run for them to be picked up in citations.} The citation commands have access to the complete bibliographic data, not only to a predefined label. The bibliography is generated from memory using the same data and may be filtered or split as required.
\end{enumerate}

我们再次观察上面给出的条目示例:
%Let's consider the sample entry given above once more:

\begin{lstlisting}[style=bibtex]{}
@Book{companion,
  author    = {Michel Goossens and Frank Mittelbach and Alexander Samarin},
  title     = {The LaTeX Companion},
  publisher = {Addison-Wesley},
  address   = {Reading, Mass.},
  year      = {1994},
}
\end{lstlisting}
%
使用\biblatex 及\biber 后端程序,这一条目实际上以如下格式输出:
%With \biblatex and the \biber backend, this entry is essentially exported in the following format:

\begin{ltxexample}
\entry{companion}{book}{}
  \labelname{author}{3}{}{%
    {{uniquename=0,hash=...}{Goossens}{G.}{Michel}{M.}{}{}{}{}}%
    {{uniquename=0,hash=...}{Mittelbach}{M.}{Frank}{F.}{}{}{}{}}%
    {{uniquename=0,hash=...}{Samarin}{S.}{Alexander}{A.}{}{}{}{}}%
  }
  \name{author}{3}{}{%
    {{uniquename=0,hash=...}{Goossens}{G.}{Michel}{M.}{}{}{}{}}%
    {{uniquename=0,hash=...}{Mittelbach}{M.}{Frank}{F.}{}{}{}{}}%
    {{uniquename=0,hash=...}{Samarin}{S.}{Alexander}{A.}{}{}{}{}}%
  }
  \list{publisher}{1}{%
    {Addison-Wesley}%
  }
  \list{location}{1}{%
    {Reading, Mass.}%
  }
  \field{title}{The LaTeX Companion}
  \field{year}{1994}
\endentry
\end{ltxexample}
%

由该示例可见,某种程度上,结构化的数据构成了\file{bbl} 文件内容\footnote{译者:这里应该是bbl文件而不是原文的bib文件}。此时关于参考文献条目最终格式的任何决定都未作出。而参考文献表和引用标注的格式化最终由 \latex 宏控制,这些宏则定义在参考文献著录和标注样式文件中。
%As seen in this example, the data is presented in a structured format that resembles the structure of a \file{bib} file to some extent. At this point, no decision concerning the final format of the bibliography entry has been made. The formatting of the bibliography and all citations is controlled by \latex macros, which are defined in bibliography and citation style files.

\subsection{参考文献著录样式}%Bibliography Styles
\label{aut:bbx}

\gls{参考文献著录样式} 是用于控制打印参考文献表条目的宏的集合,定义在扩展名为\file{bbx} 的文件中。\biblatex 包在其末尾加载所选择的参考文献著录样式文件。需要注意: 多个标准样式共享的一些常用宏定义在\path{biblatex.def} 文件中。该文件同样在宏包末尾加载,但先于所选的著录样式文件。
%A bibliography style is a set of macros which print the entries in the bibliography. Such styles are defined in files with the suffix \file{bbx}. The \biblatex package loads the selected bibliography style file at the end of the package. Note that a small repertory of frequently used macros shared by several of the standard bibliography styles is included in \path{biblatex.def}. This file is loaded at the end of the package as well, prior to the selected bibliography style.

\subsubsection{参考文献著录样式文件}% Bibliography Style Files
\label{aut:bbx:bbx}

在我们讨论参考文献著录样式的各部分内容之前,先观察一个典型\file{bbx} 文件的总体结构:
%Before we go over the individual components of a bibliography style, consider this example of the overall structure of a typical \file{bbx} file:

\begin{ltxexample}
\ProvidesFile{example.bbx}[2006/03/15 v1.0 biblatex bibliography style]

\defbibenvironment{bibliography}
  {...}
  {...}
  {...}
\defbibenvironment{shorthand}
  {...}
  {...}
  {...}
\InitializeBibliographyStyle{...}
\DeclareBibliographyDriver{article}{...}
\DeclareBibliographyDriver{book}{...}
\DeclareBibliographyDriver{inbook}{...}
...
\DeclareBibliographyDriver{shorthand}{...}
\endinput
\end{ltxexample}
%
参考文献著录样式文件的主要结构包含如下命令:
%The main structure of a bibliography style file consists of the following commands:

\begin{ltxsyntax}

\cmditem{RequireBibliographyStyle}{style}

该命令是可选的,用于引入一些建立在更一般的参考文献样式上的特殊样式。该命令加载了样式文件\path{style.bbx}。
%This command is optional and intended for specialized bibliography styles built on top of a more generic style. It loads the bibliography style \path{style.bbx}.

\cmditem{InitializeBibliographyStyle}{code}

该命令在参考文献表开始之前插入任意给定的\prm{code},但在参考文献表所形成的组内。该命令是可选的。它可以用于共享不同参考文献驱动需要的一些相同定义,但不能用于参考文献组外。记住,文档中可以有多个参考文献表,如果在某个文献表中对参考文献驱动进行了一些全局设置,最好在下一个参考文献表开始前进行重设。
%Specifies arbitrary \prm{code} to be inserted at the beginning of the bibliography, but inside the group formed by the bibliography. This command is optional. It may be useful for definitions which are shared by several bibliography drivers but not used outside the bibliography. Keep in mind that there may be several bibliographies in a document. If the bibliography drivers make any global assignments, they should be reset at the beginning of the next bibliography.

\cmditem{DeclareBibliographyDriver}{entrytype}{code}

定义一个参考文献驱动。一个驱动 <driver> 是一个宏,用于控制某一具体的参考文献条目(当打印参考文献表时)或者某一具体命名了的参考文献表(当打印多个参考文献表时)。\prm{entrytype} 与\file{bib} 文件中使用的条目类型对应,以小写字母给出(见\secref{bib:typ})。\prm{entrytype} 变量可以是一个星号。这种情况下,该驱动将作为未定义具体驱动的条目类型的驱动。\prm{code} 是用于打印各自\prm{entrytype} 的参考文献条目的任意代码。该命令是必须的。每个参考文献著录样式都应提供所用到的每类条目的驱动。
%Defines a bibliography driver. A <driver> is a macro which handles a specific entry type (when printing bibliography lists) or a specific named bibliography list (when printing bibliography lists). The \prm{entrytype} corresponds to the entry type used in \file{bib} files, specified in lowercase letters (see \secref{bib:typ}). The \prm{entrytype} argument may also be an asterisk. In this case, the driver serves as a fallback which is used if no specific driver for the entry type has been defined. The \prm{code} is arbitrary code which typesets all bibliography entries of the respective \prm{entrytype}. This command is mandatory. Every bibliography style should provide a driver for each entry type.

\cmditem{DeclareBibliographyAlias}{alias}{entrytype}

如果一个参考文献驱动用于处理多个条目类型,该命令可以用来定义以\prm{entrytype} 命名的驱动的别名。\prm{alias} 选项可以是一个星号,这种情况下,该驱动将作为未定义具体驱动的条目类型的驱动。
%If a bibliography driver covers more than one entry type, this command may be used to define an alias where \prm{entrytype} is the name of a defined driver. This command is optional. The \prm{alias} argument may also be an asterisk. In this case, the \prm{entrytype} driver serves as a fallback which is used if no specific driver for an entry has been defined.

\cmditem{DeclareBibliographyOption}[datatype]{key}[value]{code}

该命令以\keyval 格式定义额外的导言区选项。\prm{key} 是选项键。\prm{code} 是当使用该选项时执行的任意\tex 代码。键值作为|#1|传递给\prm{code}。可选的\prm{value} 是当该选项仅有键名而无键值给出时的默认键值。这对于布尔选项非常有用。\prm{datatype} 是选项的数据类型(datatype),如果缺省,那么默认为 <boolean>(布尔类型)。一个定义示例如下:

%This command defines additional preamble options in \keyval format. The \prm{key} is the option key. The \prm{code} is arbitrary \tex code to be executed whenever the option is used. The value passed to the option is passed on to the \prm{code} as |#1|. The optional \prm{value} is a default value to be used if the bare key is given without any value. This is useful for boolean switches.
%The \prm{datatype} is a the datatype for the option. If omitted, it defaults to <boolean>. For example, with a definition like the following:

\begin{ltxexample}
\DeclareBibliographyOption[boolean]{somekey}[true]{...}
\end{ltxexample}
%
给出<\texttt{somekey}>而没有键值等价于<\kvopt{somekey}{true}>。有效的\prm{datatype} 在默认的\biber 数据模型中定义,包括:
%giving <\texttt{somekey}> without a value is equivalent to <\kvopt{somekey}{true}>. Valid \prm{datatype} values are defined in the default \biber Datamodel as:

\begin{ltxexample}
\DeclareDatamodelConstant[type=list]{optiondatatypes}{boolean,integer,string,xml}
\end{ltxexample}

\cmditem{DeclareEntryOption}[datatype]{key}[value]{code}

类似于\cmd{DeclareBibliographyOption},但用于定义具体条目类型的\bibfield{options} 域(见\secref{bib:fld:spc} 节)中的可设选项。\prm{code} 在\biblatex 为标注命令和参考文献驱动准备数据时执行。
%Similar to \cmd{DeclareBibliographyOption} but defines options which are settable on a per"=entry basis in the \bibfield{options} field from \secref{bib:fld:spc}. The \prm{code} is executed whenever \biblatex prepares the data of the entry for use by a citation command or a bibliography driver.

\end{ltxsyntax}

\subsubsection{参考文献表环境}%Bibliography Environments
\label{aut:bbx:env}

除了定义参考文献驱动,参考文献著录样式也要定义参考文献表环境用于控制参考文献表的输出。这些环境由\cmd{defbibenvironment} 命令定义。默认情况下,\cmd{printbibliography} 使用\opt{bibliography} 环境。下面是一个适用于不打印标签的参考文献表的环境定义:
%Apart from defining bibliography drivers, the bibliography style is also responsible for the environments which control the layout of the bibliography and bibliography lists. These environments are defined with \cmd{defbibenvironment}. By default, \cmd{printbibliography} uses the environment \opt{bibliography}. Here is a definition suitable for a bibliography style which does not print any labels in the bibliography:

\begin{ltxexample}
\defbibenvironment{bibliography}
  {\list
     {}
     {\setlength{\leftmargin}{\bibhang}%
      \setlength{\itemindent}{-\leftmargin}%
      \setlength{\itemsep}{\bibitemsep}%
      \setlength{\parsep}{\bibparsep}}}
  {\endlist}
  {\item}
\end{ltxexample}
%
该定义使用\biblatex 提供的\cmd{bibhang}\gls{尺寸},应用了一个带悬挂缩进的\env{list} 环境。它允许使用\cmd{bibitemsep} 和 \cmd{bibparsep} 来进行一定程度的布局调整,\biblatex 提供的这两个尺寸的目的也在于此(见 \secref{aut:fmt:len})。作者年制(\texttt{authoryear})和作者题名制(\texttt{authortitle})的参考文献样式使用类似于该例的定义。
%This definition employs a \env{list} environment with hanging indentation, using the \cmd{bibhang} length register provided by \biblatex. It allows for a certain degree of configurability by using \cmd{bibitemsep} and \cmd{bibparsep}, two length registers provided by \biblatex for this very purpose (see \secref{aut:fmt:len}). The \texttt{authoryear} and \texttt{authortitle} bibliography styles use a definition similar to this example.

\begin{ltxexample}
\defbibenvironment{bibliography}
  {\list
     {\printfield[labelnumberwidth]{labelnumber}}
     {\setlength{\labelwidth}{\labelnumberwidth}%
      \setlength{\leftmargin}{\labelwidth}%
      \setlength{\labelsep}{\biblabelsep}%
      \addtolength{\leftmargin}{\labelsep}%
      \setlength{\itemsep}{\bibitemsep}%
      \setlength{\parsep}{\bibparsep}}%
      \renewcommand*{\makelabel}[1]{\hss##1}}
  {\endlist}
  {\item}
\end{ltxexample}
%
一些参考文献样式在参考文献表中打印标签。比如,设计一个顺序编码的参考文献样式需要在参考文献表中每个条目前面打印顺序编码数字,这样参考文献表看起来就像一个顺序列表。在第一个示例中,\cmd{list} 命令的第一个参数是空的。在上面这个示例中,我们需要在其中插入数字,这些数字由\biblatex 的\bibfield{labelnumber} 域提供。我们也应用\biblatex 提供的几个尺寸和工具,详见 \secref{aut:fmt:ich, aut:fmt:ilc}。顺序编码制(\texttt{numeric})参考文献样式使用如上的定义。顺序字母制(\texttt{alphabetic})的样式也是类似的,只是用\texttt{labelalpha} 和\texttt{labelalphawidth} 代替了\textsf{\texttt{labelnumber}} 和\texttt{labelnumberwidth}。
%Some bibliography styles print labels in the bibliography. For example, a bibliography style designed for a numeric citation scheme will print the number of every entry such that the bibliography looks like a numbered list. In the first example, the first argument to \cmd{list} was empty. In this example, we need it to insert the number, which is provided by \biblatex in the \bibfield{labelnumber} field. We also employ several length registers and other facilities provided by \biblatex, see \secref{aut:fmt:ich, aut:fmt:ilc} for details. The \texttt{numeric} bibliography style uses the definition given above. The \texttt{alphabetic} style is similar, except that \textsf{\texttt{labelnumber}} is replaced by \texttt{labelalpha} and \texttt{labelnumberwidth} by \texttt{labelalphawidth}.

参考文献列表以类似方式处理。\cmd{printbiblist} 命令默认使用以bibliography list命名的环境。
%(当使用\bibtex 时,\cmd{printshorthands} 总是使用\texttt{shorthand} 环境)。
一个典型示例如下,其中的尺寸和工具定义详见第\secref{aut:fmt:ich, aut:fmt:ilc} 节。
%Bibliography lists are handled in a similar way. \cmd{printbiblist} uses the environment named after the bibliography list by default. A typical example is given below. See \secref{aut:fmt:ich, aut:fmt:ilc} for details on the length registers and facilities used in this example.

\begin{ltxexample}
\defbibenvironment{shorthand}
  {\list
     {\printfield[shorthandwidth]{shorthand}}
     {\setlength{\labelwidth}{\shorthandwidth}%
      \setlength{\leftmargin}{\labelwidth}%
      \setlength{\labelsep}{\biblabelsep}%
      \addtolength{\leftmargin}{\labelsep}%
      \setlength{\itemsep}{\bibitemsep}%
      \setlength{\parsep}{\bibparsep}%
      \renewcommand*{\makelabel}[1]{##1\hss}}}
  {\endlist}
  {\item}
\end{ltxexample}

\subsubsection{参考文献驱动} %Bibliography Drivers
\label{aut:bbx:drv}

在讨论\biblatex 包的数据接口命令前,了解一下参考文献驱动的结构是有益的。注意,虽然下面给出的示例是大为简化的,但仍具有说明价值。为可读性考虑,我们忽略了一些可能是\bibtype{book} 条目的域,并且简化处理没有忽略的域。主要是为了说明驱动的结构。关于\bibtex 文件格式域与\biblatex 宏包数据类型的映射信息,见\secref{bib:fld}。
%Before we go over the commands which form the data interface of the \biblatex package, it may be instructive to have a look at the structure of a bibliography driver. Note that the example given below is greatly simplified, but still functional. For the sake of readability, we omit several fields which may be part of a \bibtype{book} entry and also simplify the handling of those which are considered. The main point is to give you an idea of how a driver is structured. For information about the mapping of the \bibtex file format fields to \biblatex's data types, see \secref{bib:fld}.

\begin{ltxexample}
\DeclareBibliographyDriver{book}{%
  \printnames{author}%
  \newunit\newblock
  \printfield{title}%
  \newunit\newblock
  \printlist{publisher}%
  \newunit
  \printlist{location}%
  \newunit
  \printfield{year}%
  \finentry}
\end{ltxexample}
%
标准的参考文献样式应用两个参考文献宏\texttt{begentry} 和\texttt{finentry}:
%The standard bibliography styles employ two bibliography macros \texttt{begentry} and \texttt{finentry}:

\begin{ltxexample}
\DeclareBibliographyDriver{<<entrytype>>}{%
  \usebibmacro{begentry}
  ...
  \usebibmacro{finentry}}
\end{ltxexample}
%
默认定义为:
%with the default definitions
\begin{ltxexample}
\newbibmacro*{begentry}{}
\newbibmacro*{finentry}{\finentry}
\end{ltxexample}
%
为方便在驱动开始或结束时使用钩子,推荐使用这两个宏。
%Use of these macros is recommended for easy hooks into the beginning and end of the driver.

上述给出的\texttt{book} 条目类型的驱动中存在有一些缺省: 即\cmd{printnames}, \cmd{printlist}, 和 \cmd{printfield} 命令所使用的格式化指令。为了说明一个格式化指令是什么,这里给出上述驱动示例中所使用的指令。域格式是很直接的,域的值直接作为参数传递给能实现期望格式的指令。下面的指令简单地将输入参数用一个\cmd{emph} 命令包裹:
%Returning to the driver for the \texttt{book} entrytype above, there is still one piece missing: the formatting directives used by \cmd{printnames}, \cmd{printlist}, and \cmd{printfield}. To give you an idea of what a formatting directive looks like, here are some fictional ones used by our sample driver. Field formats are straightforward, the value of the field is passed to the formatting directive as an argument which may be formatted as desired. The following directive will simply wrap its argument in an \cmd{emph} command:

\begin{ltxexample}
\DeclareFieldFormat{title}{\emph{#1}}
\end{ltxexample}
%
列表格式则稍微要复杂一些。在将列表划分为独立的项后,\biblatex 将对列表中的每一项执行格式化指令。各项作为参数传递给格式化指令。列表中各项间的分隔符由相应的指令控制,因此我们在插入分隔符前必须要检查当前位置是位于列表中还是位于列表末尾。
%List formats are slightly more complex. After splitting up the list into individual items, \biblatex will execute the formatting directive once for every item in the list. The item is passed to the directive as an argument. The separator to be inserted between the individual items in the list is also handled by the corresponding directive, hence we have to check whether we are in the middle of the list or at the end when inserting it.

\begin{ltxexample}
\DeclareListFormat{location}{%
  #1%
  \ifthenelse{\value{listcount}<\value{liststop}}
    {\addcomma\space}
    {}}
\end{ltxexample}
%
%姓名(name)的格式化命令类似于这种抄录列表,但列表中的单个项是姓名,因此需要自动的解析为姓名的不同组成部分。列表格式化命令对列表中每个姓名都执行一次,信吗的各个部分以分开的参数传递给该命令。比如,|#1|是姓(last name)和|#3|是名(first name)。下面给出一个简化的格式化命令示例:
%上述各格式化命令调换了第一个作者的姓名前后顺序«Last, First»),而其余姓名则是常规顺序(«First Last»)。注意:必须要保证提供的姓名部分是姓(last name),因此我们必须要检查实际数据中姓名的哪些部分是存在的。如果姓名的一些部分不存在,则相关的变量就为空。如同抄录列表的命令,在各独立项之间插入的分隔符也由格式化命令控制,因为我们也要检查是否处于列表中还是在其末尾,这也是第二个\cmd{ifthenelse} 命令做的事情。

姓名(name)的格式化指令类似于抄录列表。
%Formatting directives for names are similar to those for literal lists.

依赖于数据模型常量<nameparts>的姓名有如下默认定义:
%Names depend on the datamodel constant <nameparts> which has the default definition:

\begin{ltxexample}
\DeclareDatamodelConstant[type=list]{nameparts}
                                    {prefix,family,suffix,given}
\end{ltxexample}
%
可以对其进行设置,比如添加更多姓名成分来处理类似父系姓的问题(见文件\file{93-nameparts.tex})。自然的,数据源需要一个扩展的姓名格式。\biblatexml (\secref{apx:biblatexml})可以处理该问题,其中有一个扩展的姓名格式,当使用\biber 后端时(见\biber 文档),可以处理自定义的姓名成分。
%This can be customised to add more name parts to deal with things like patronymics (see the example file \file{93-nameparts.tex}). Naturally this needs an extended name format for data sources. \biblatexml (\secref{apx:biblatexml}) handles this natively and there is an extended name format which can handle custom nameparts available when using \biber (see \biber documentation).

在姓名格式中,姓名成分常量声明将为数据模型中定义的每个姓名成分提供两个宏:
%Inside name formats, the nameparts constant declaration makes available two macros for each name part defined in the datamodel:

\begin{ltxexample}
\namepart<namepart>   \% The full <namepart>
\namepart<namepart>i  \% The initials of the <namepart>
\namepart<namepart>un \% Numeric value indicating uniqueness level for <namepart>
\end{ltxexample}
%
%\cmd{namepart<namepart>un} only exists if the package option \opt{uniquename} is not set to <false> and can take the following values.
\cmd{namepart<namepart>un}仅当\opt{uniquename}包选项未设置成<false>时存在,可以设置为如下值:

\begin{argumentlist}{00}
\item[0] 不用于消除姓名歧义的<namepart>(因为在\cmd{DeclareUniquenameTemplate}中设置了\opt{disambiguation=none},见\secref{aut:cav:amb}))。这种情况下,样式需要确定对应这一<namepart>需打印的内容。
%<namepart> was not used in disambiguating the name (because \opt{disambiguation=none} was set in \cmd{DeclareUniquenameTemplate}, see \secref{aut:cav:amb}). In this case the style should decide what to print for this <namepart>
\item[1] 仅需要为<namepart>打印首字母,用以根据\opt{uniquename}选项设置来确保唯一性。
%Initials only should be printed for <namepart> to ensure uniqueness according to the \opt{uniquename} package option setting
\item[2] 需要打印完整的<namepart>,用以根据\opt{uniquename}选项设置来确保唯一性。
%The full <namepart> should be printed to ensure uniqueness according to the \opt{uniquename} package option setting
\end{argumentlist}

%Note these per-namepart uniqueness macros are essentially an override of the \opt{uniquename} value (see \secref{aut:aux:tst}) for the name as a whole. Styles can choose to use either the less granular \opt{uniquename} value or the more detailed per-namepart values. Usually the general \opt{uniquename} value is enough for ordinary Western names but the more granular information per-namepart is provided to allow sophisticated name uniqueness processing for more complex name schemata.
注意,这些按姓名成分处理唯一性的宏实质上是对用于整个姓名的\opt{uniquename}值的覆盖(见\secref{aut:aux:tst})。样式可以选择使用更粗粒度的\opt{uniquename} 值或者更详细的依照姓名成分的值。通常情况下,\opt{uniquename}对于一般的西方姓名已经足够,但也提供了更细粒度的信息用于在复杂姓名格式中进行姓名唯一性的精细处理。
%The name formatting directive is executed once for each name in the name list. Here is a simplified example---the standard name formats are more intricate:
姓名的格式化指令对姓名列表中的每一个姓名进行处理,下面是一个简单示例,标准的格式化处理要更复杂:

\begin{ltxexample}
\DeclareNameFormat{author}{%
  \ifthenelse{\value{listcount}=1}
    {\namepartfamily%
     \ifdefvoid{\namepartgiven}{}{\addcomma\space\namepartgiven}}
    {\ifdefvoid{\namepartgiven}{}{\namepartgiven\space}%
     \namepartfamily}%
  \ifthenelse{\value{listcount}<\value{liststop}}
    {\addcomma\space}
    {}}
\end{ltxexample}
%
%The above directive reverses the name of the first author («Family, Given»)/(«Last, First») and prints the remaining names in their regular sequence («Given Family»)/(«First Last»). Note that the only component which is guaranteed to be available is the last/family name, hence we have to check which parts of the name are actually present. If a certain name part is not available, the corresponding macro will be empty. As with directives for literal lists, the separator to be inserted between the individual items in the name list is also handled by the formatting directive, hence we have to check whether we are in the middle of the list or at the end when inserting it. This is what the second \cmd{ifthenelse} test does. See also \secref{aut:bib:fmt}.
上述格式化指令调换了第一个作者的姓名前后顺序(«Last, First»),而其余姓名则是常规顺序(«First Last»)。
注意: 当仅有一个姓名成分时,必须确保要它是姓(last name),因此我们需要检查实际数据中姓名的哪些成分存在。
如果姓名的一些成分不存在,则相应的宏就为空。如同抄录列表的指令,在各独立项之间插入的分隔符也由格式化指令控制,
因为我们也要检查当前位置是处于列表中还是在其末尾,这也是第二个\cmd{ifthenelse} 命令做的事情。
另见\secref{aut:bib:fmt}节.



%A similar output that also respects the \cmd{multinamedelim} and \cmd{finalnamedelim} commands as well as the <prefix> and <suffix> name parts can be achieved with

一个涉及到\cmd{multinamedelim}和\cmd{finalnamedelim}命令,
及前缀(<prefix>) 和后缀(<suffix>)姓名成分的类似输出可以通过如下方式实现:

\begin{ltxexample}
\DeclareNameAlias{author}{family-given/given-family}
\end{ltxexample}

\subsubsection{特殊域}%Special Fields
\label{aut:bbx:fld}
下面的列表和域用于\biblatex 给参考文献驱动和标注命令传递数据。它们由宏包自动定义,并不在\file{bib} 文件中使用。但从参考文献著录和标注样式的角度看,它们与\file{bib} 文件中的域并没有什么不同。

%The following lists and fields are used by \biblatex to pass data to bibliography drivers and citation commands. They are not used in \file{bib} files but defined automatically by the package. From the perspective of a bibliography or citation style, they are not different from the fields in a \file{bib} file.

\paragraph{一般域} %Generic Fields
\label{aut:bbx:fld:gen}

\begin{fieldlist}

\fielditem{$<$datetype$>$dateunspecified}{string}

如果$<$datetype$>$date具有一个\acr{ISO8601-2} 4.3 <unspecified>,该域将被设置为\opt{yearindecade}, \opt{yearincentury}, \opt{monthinyear}, \opt{dayinmonth} 或\opt{dayinyear} 之一,这些字符串定义了unspecified 信息的粒度(即表示日期的不确定度,yearincentury表示一个世纪范围内不确定,yearindecade表示10年范围内不确定)。这些字符串可用于日期范围的判断,日期范围自动根据<unspecified>日期创建,一个样式可以选择一种特殊方式来格式化日期。参见\secref{bib:use:dat}。例如:一个条目的日期为:
%If $<$datetype$>$date held an \acr{ISO8601-2} 4.3 <unspecified>, this field will be set to one of \opt{yearindecade}, \opt{yearincentury}, \opt{monthinyear}, \opt{dayinmonth} or \opt{dayinyear} which specifies the granularity of the unspecified information. These strings can be tested for and along with the date ranges which are automatically created for such <unspecified> dates, a style may choose to format the date in a special way. See \secref{bib:use:dat}. For example, an entry with dates such as:

\begin{lstlisting}[style=bibtex]{}
@book{key,
  date     = {19uu},
  origdate = {199u}
}
\end{lstlisting}
%
将在\file{.bbl} 产生如下信息:
%would result in the same information in the \file{.bbl} as:
\begin{lstlisting}[style=bibtex]{}
@book{key,
  date     = {1900/1999},
  origdate = {1990/1999}
}
\end{lstlisting}
%
但也会额外地将\bibfield{dateunspecified} 域设置为<yearincentury>,将\bibfield{origdateunspecified} 设置为<yearindecade>。这一信息可以用来给\bibfield{date} 提供可能的信息<20th century>,给\bibfield{origdate} 提供<The 1990s>,这一信息无法单独从日期范围推算。当<unspecified>元信息给出时,这一自动生成的范围即已知,因此使用该范围值进行特殊的格式化相对容易。但标准样式不做此处理,\file{96-dates.tex} 给出了一些示例。
%but would additionally have the field \bibfield{dateunspecified} set to <yearincentury> and \bibfield{origdateunspecified} set to <yearindecade>. This information could be used to render the \bibfield{date} as perhaps <20th century> and \bibfield{origdate} as <The 1990s>, information which cannot be derived from the date ranges alone. Since such auto-generated ranges have a know values, given the <unspecified> meta-information, it is relatively easy to use the range values to format special cases. While the standard styles not do this, examples are given in the file \file{96-dates.tex}.

\fielditem{entrykey}{string}

\file{bib} 文件中某一项的条目关键词(entry key)。这是一个字符串,用于\biblatex 及其后端程序确定\file{bib} 文件中的某一条目。
%The entry key of an item in the \file{bib} file. This is the string used by \biblatex and the backend to identify an entry in the \file{bib} file.

\fielditem{childentrykey}{string}\DeprecatedMark

%This field is no longer necessary or recommended. For backwards
%compatibility, it is merely a copy of the \bibfield{entrykey} field in any
%set children.
该域不再必需的,也不推荐使用。为后向兼容考虑,它仅仅是集中子条目的\bibfield{entrykey}域的副本。


%老版本的信息:
%当引用一个条目集中的子条目时,\biblatex 给标注命令提供\bibtype{set} 父条目集的数据。这意味着\bibfield{entrykey} 将保存父条目集的关键词。而子条目的关键词在\bibfield{childentrykey} 域中提供。该域仅在引用条目集的某一子条目时使用。
%When citing a subentry of an entry set, \biblatex provides the data of the parent \bibtype{set} entry to citation commands. This implies that the \bibfield{entrykey} field holds the entry key of the parent. The entry key of the child entry being cited is provided in the \bibfield{childentrykey} field. This field is only available when citing a subentry of an entry set.

\fielditem{labelnamesource}{literal}

保存给\bibfield{labelname} 提供信息的域的域名,由\cmd{DeclareLabelname} 确定。
%Holds the name of the field used to populate \bibfield{labelname},
%determined by \cmd{DeclareLabelname}.

\fielditem{labeltitlesource}{literal}

保存给\bibfield{labeltitle} 提供信息的域的域名,由\cmd{DeclareLabeltitle} 确定。
%Holds the name of the field used to populate \bibfield{labeltitle},
%determined by \cmd{DeclareLabeltitle}.

\fielditem{labeldatesource}{literal}

保存如下内容之一:
%Holds one of:

\begin{itemize}
\item 由\cmd{DeclareLabeldate} 选择的日期域域名中<date>前的前缀内容。
%The prefix coming before <date> of the date field name chosen by
%\cmd{DeclareLabeldate}
\item 一个域的域名。
%The name of a field
\item 一个抄录或本地化字符串。\footnote{译者: literal 译为抄录}
%A literal or localisation string
\end{itemize}
%

一般情况下保存由\cmd{DeclareLabeldate} 选择的日期域域名中<date>前的前缀内容。例如,如果labeldate域是\bibfield{eventdate},那么\bibfield{labeldatesource} 就是<event>。如果\cmd{DeclareLabeldate} 命令选择了\bibfield{date} 域,\bibfield{labeldatesource} 将会定义为一个空字符串作为<date>的前缀,因为\bibfield{date} 域名中<date>前内容为空。这就是说\bibfield{labeldatesource} 的内容可以用于构建\cmd{DeclareLabeldate} 选择的域的指针。因为\cmd{DeclareLabeldate} 也可以选择抄录字符串作为备选,\bibfield{labeldatesource} 可以指向一个域或者不进行定义。记住:\cmd{DeclareLabeldate} 命令可以用于选择非日期域作为备选,所以\bibfield{labeldatesource} 也可能包含一个域名。所以,总结起来,规则如下:
%Normally holds the prefix coming before <date> of the date field name chosen by \cmd{DeclareLabeldate}. For example, if the labeldate field is \bibfield{eventdate}, then \bibfield{labeldatesource} will be <event>. In case \cmd{DeclareLabeldate} selects the \bibfield{date} field, then \bibfield{labeldatesource} will be defined but will be an empty string as the prefix coming before <date> in the date label name is empty. This is so that the contents of \bibfield{labeldatesource} can be used in constructing references to the field which \cmd{DeclareLabeldate} chooses. Since \cmd{DeclareLabeldate} can also select literal strings for fallbacks, \bibfield{labeldatesource} may not refer to a field or may be undefined. Bear in mind that \cmd{DeclareLabeldate} can also be used to select non-date fields as a fallback and so \bibfield{labeldatesource} might contain a field name. So, in summary, the rules are

\begin{ltxexample}
\iffieldundef{labeldatesource}
  {}% labeldate package option is not set
  {\iffieldundef{\thefield{labeldatesource}year}
    % \DeclareLabeldate resolved to either a literal/localisation
    % string or a non-date field since
    % if a date is defined by a date field, there is
    % at least a year
    {\iffieldundef{\thefield{labeldatesource}}
       {}% \DeclareLabeldate resolved to a literal/localisation string
       {}% \DeclareLabeldate resolved to a non-date field
    }
    {} % \DeclareLabeldate resolved a date field name prefix like "" or "orig"
  }
\end{ltxexample}

\fielditem{entrytype}{string}

	条目类型(\bibtype{book}, \bibtype{inbook} 等),以小写字母给出。
%The entry type (\bibtype{book}, \bibtype{inbook}, etc.), given in lowercase letters.

\fielditem{childentrytype}{string}\DeprecatedMark
%This field is no longer necessary or recommended. For backwards
%compatibility, it is merely a copy of the \bibfield{entrytype} field in any
%set children.
该域不再是必需的,也不推荐使用。为后向兼容考虑,它仅仅是集中子条目的\bibfield{entrytype}域的副本。


%老版本信息:
%当引用一个条目集的子条目时,\biblatex 为标注命令提供父集条目的数据。这意味着\bibfield{entrytype} 保存父条目的类型。子条目的类型则由\bibfield{childentrytype} 域提供。该域仅在引用一个条目集的子条目时使用。
%When citing a subentry of an entry set, \biblatex provides the data of the parent \bibtype{set} entry to citation commands. This implies that the \bibfield{entrytype} field holds the entry type of the parent. The entry type of the child entry being cited is provided in the \bibfield{childentrytype} field. This field is only available when citing a subentry of an entry set.

\fielditem{entrysetcount}{integer}

	
%This field holds an integer indicating the position of a set member in the entry set (starting at \texttt{1}). This field is only available in the subentries of an entry set.
该域保存的整数用于指明一个条目集中某个集成员的位置(起始值是\texttt{1})。该域仅对一个条目集的子条目有用。

\fielditem{hash}{string}

%This field is special in that it is only available locally in name formatting directives. It holds a hash string which uniquely identifies individual names in a name list. This information is available for all names in all name lists. See also \bibfield{namehash} and \bibfield{fullhash}.
该域非常特殊,仅在姓名格式化命令中使用。它保存一个hash字符串,用于唯一地确定姓名列表中的各个姓名。姓名列表中的所有姓名都具有该信息。另可参见\bibfield{namehash} 和\bibfield{fullhash}。


\fielditem{namehash}{string}

%A hash string which uniquely identifies the \bibfield{labelname} list. This is useful for recurrence checks. For example, a citation style which replaces recurrent authors or editors with a string like <idem> could save the \bibfield{namehash} field with \cmd{savefield} and use it in a comparison with \cmd{iffieldequals} later (see \secref{aut:aux:dat, aut:aux:tst}). The \bibfield{namehash} is derived from the truncated \bibfield{labelname} list, \ie it is responsive to \opt{maxnames} and \opt{minnames}. See also \bibfield{hash} and \bibfield{fullhash}.
一个hash字符串用于唯一确定\bibfield{labelname} 列表。用于在线检查。比如,一个将再次出现的作者和编者用一个类似<idem>的字符串代替的标注样式,
可以用\cmd{savefield} 命令保存\bibfield{namehash} 域,
并用于后面的\cmd{iffieldequals} 比较中(见\secref{aut:aux:dat, aut:aux:tst})。
\bibfield{namehash} 域由截短的\bibfield{labelname} 列表确定,
即与\opt{maxnames} 和\opt{minnames} 选项相关。
另可参见\bibfield{hash} 和\bibfield{fullhash}。


\fielditem{bibnamehash}{string}

%As \bibfield{namehash} but responsive to \opt{maxbibnames} and
%\opt{minbibnames}. This is not used in standard styles but may be used to
%make tests in bibliography lists (such as those used to determine whether
%dashes should replace repeated authors) behave differently.

类似于 \bibfield{namehash} ,但与\opt{maxbibnames} 和
\opt{minbibnames}相关。不用域标准样式中,但可以用在文献表中做测试
(比如用于判断是否用破折号代替重复作者的情况)


\fielditem{$<$namelist$>$namehash}{string}


%As \bibfield{namehash} for the name list called <namelist>.
类似于\bibfield{namehash},但用于 <namelist>姓名列表。

\fielditem{$<$namelist$>$bibnamehash}{string}

As \bibfield{bibnamehash} for the name list called <namelist>.

\fielditem{fullhash}{string}

%A hash string which uniquely identifies the \bibfield{labelname} list. This fields differs from \bibfield{namehash} in two details: 1) The \bibfield{shortauthor} and \bibfield{shorteditor} lists are ignored when generating the hash. 2) The hash always refers to the full list, ignoring \opt{maxnames} and \opt{minnames}. See also \bibfield{hash} and \bibfield{namehash}.
一个hash字符串用于唯一确定\bibfield{labelname} 列表。该域与\bibfield{namehash} 有两点不同:1.产生hash时忽略\bibfield{shortauthor} 和\bibfield{shorteditor} 列表。2.该hash指的是完整的列表,忽略\opt{maxnames} 和\opt{minnames} 选项。另可参见\bibfield{hash} 和\bibfield{namehash}。


\fielditem{$<$namelist$>$fullhash}{string}

%As \bibfield{fullhash} for the name list called <namelist>.
类似于\bibfield{fullhash},但用于 <namelist>姓名列表。


\listitem{pageref}{literal}

%If the \opt{backref} package option is enabled, this list holds the page numbers of the pages on which the respective bibliography entry is cited. If there are \env{refsection} environments in the document, the back references are local to the reference sections.
如果\opt{backref} 包选项启用,该域保存条目被引用所在各页的页码。如果文档中有\env{refsection} 环境,反向引用仅包含当前参考文献节(refsection)内的页码信息。


\fielditem{sortinit}{literal}

%This field holds the initial character of the information used during sorting.
该域保存排序所使用信息的首字符。%使用\bibtex 时,该域也用来代替\bibfield{sortinithash} 域。


\fielditem{sortinithash}{string}

%This field holds a hash of the (locale-specific) Unicode Collation Algorithm primary weight of the first extended grapheme cluster (essentially the first character) of the string used during sorting. This is useful when subdividing the bibliography alphabetically and is used internally by \cmd{bibinitsep} (see \secref{use:fmt:len}).
%使用\biber 时,
该域保存排序字符串的第一个扩展字素集(基本上是第一个字符)的Unicode排序规则算法主权重的hash值。可用于按照字母表顺序划分参考文献列表,该域在\cmd{bibinitsep} 命令内部使用(见\secref{use:fmt:len})。


\fielditem{clonesourcekey}{string}

%This field holds the entry key of the entry from which an entry was cloned. Clones are created for
%entries which are mentioned in \bibfield{related} fields as part of related entry processing, for example.
该域保存复制条目的来源条目的关键词。例如,关联条目处理中\bibfield{related} 域所涉及的条目往往需要进行复制。

\fielditem{urlraw}{verbatim}

%This is the unencoded, raw version of any \bibfield{url}. This is intended for use when the display version and clickable link version of a URL are different. This can be the case when the URL contains special or Unicode characters. In the case where no such characters occur, may be identical to the \bibfield{url}.
\bibfield{url}域的未编码,原始版本。用于当显示的网址与点击时的链接网址不一致的情况,比如网址带有一些特殊的或unicode字符的情况。倘若没有这些特殊字符,它与\bibfield{url}域是一致的。


\end{fieldlist}

\paragraph{标注(引用)标签中使用的域}%Fields for Use in Citation Labels
\label{aut:bbx:fld:lab}

\begin{fieldlist}

\fielditem{labelalpha}{literal}

	%当使用\bibtex 为后端程序时,
一个与传统\bibtex 的\path{alpha.bst} 样式产生标签类似的标签。这一标签默认由\bibfield{labelname} 列表抽取的首字母加上出版年的最后两个数字构成。\bibfield{label} 域可用来重设它的非数字部分(non-numeric portion)。如果定义了\bibfield{label} 域,\biblatex 将使用它的值加上出版年的后两个数字生成\bibfield{labelalpha}。\bibfield{shorthand} 域也可用于重设整个标签。如果定义了该域,\bibfield{labelalpha} 就是\bibfield{shorthand} 域,而不是一个自动生成的标签。用户可以定义一个模板用于构建字母顺序标签(见\secref{aut:ctm:lab}),而默认的模板与上面bibtex标签的格式相同。一个完整的字母顺序(<alphabetic>)标签由\bibfield{labelalpha} 加\bibfield{extraalpha} 域构成。注意: 使用\bibfield{labelalpha} 和\bibfield{extraalpha} 域需要打开\opt{labelalpha} 包选项(\secref{use:opt:pre:int})。另可参见\secref{use:fmt:fmt} 节的\bibfield{extraalpha} 和\cmd{labelalphaothers}。
%A label similar to the labels generated by the \path{alpha.bst} style of traditional \bibtex. This default label consists of initials drawn from the \bibfield{labelname} list plus the last two digits of the publication year. The \bibfield{label} field may be used to override its non"=numeric portion. If the \bibfield{label} field is defined, \biblatex will use its value and append the last two digits of the publication year when generating \bibfield{labelalpha}. The \bibfield{shorthand} field may be used to override the entire label. If defined, \bibfield{labelalpha} is the \bibfield{shorthand} rather than an automatically generated label. Users can specify a template used to construct the alphabetic label (see \secref{aut:ctm:lab}) and the default template mirrors the format mentioned for bibtex above. A complete <alphabetic> label consists of the fields \bibfield{labelalpha} plus \bibfield{extraalpha}. Note that the \bibfield{labelalpha} and \bibfield{extraalpha} fields need to be requested with the package option \opt{labelalpha} (\secref{use:opt:pre:int}). See also \bibfield{extraalpha} as well as \cmd{labelalphaothers} in \secref{use:fmt:fmt}.

\fielditem{extraalpha}{integer}

	当参考文献中包含同一作者同年出版的多个引文时,<alphabetic>标注样式的标签常需要一个额外的字母加以区分。这种情况下\bibfield{extraalpha} 域保存一个整数,该整数可用命令\cmd{mknumalph} 转换成字母或以其他方式格式化。该域类似于在作者年(author-year)样式中\bibfield{extrayear} 的作用。完整的 <alphabetic>的标签由\bibfield{labelalpha} 加 \bibfield{extraalpha} 构成。注意: 使用\bibfield{labelalpha} 和 \bibfield{extraalpha} 域需要启用包选项 \opt{labelalpha}(详见\secref{use:opt:pre:int})。另可参见\secref{use:fmt:fmt} 节的\bibfield{labelalpha} 和 \ cmd{labelalphaothers}。表\ref{use:opt:tab1} 总结了用于消除歧义的不同\opt{extra*}\gls{计数器} 及其记录的信息。
%The <alphabetic> citation scheme usually requires a letter to be appended to the label if the bibliography contains two or more works by the same author which were all published in the same year. In this case, the \bibfield{extraalpha} field holds an integer which may be converted to a letter with \cmd{mknumalph} or formatted in some other way. This field is similar to the role of \bibfield{extrayear} in the author"=year scheme. A complete <alphabetic> label consists of the fields \bibfield{labelalpha} plus \bibfield{extraalpha}. Note that the \bibfield{labelalpha} and \bibfield{extraalpha} fields need to be requested with the package option \opt{labelalpha}, see \secref{use:opt:pre:int} for details. See also \bibfield{labelalpha} as well as \cmd{labelalphaothers} in \secref{use:fmt:fmt}. Table \ref{use:opt:tab1} summarises the various \opt{extra*} disambiguation counters and what they track.

\listitem{labelname}{name}

%The name to be printed in citations. This list is a copy of either the \bibfield{shortauthor}, the \bibfield{author}, the \bibfield{shorteditor}, the \bibfield{editor}, or the \bibfield{translator} list, which are normally checked for in this order. If no authors and editors are available, this list is undefined. Note that this list is also responsive to the \opt{use$<$name$>$}, options, see \secref{use:opt:bib}. Citation styles should use this list when printing the name in a citation. This list is provided for convenience only and does not carry any additional meaning.
%This field may be customized. See \secref{aut:ctm:fld} for details.
标注标签中打印的姓名。该列表可以是\bibfield{shortauthor}, \bibfield{author},  \bibfield{shorteditor}, \bibfield{editor}, 或\bibfield{translator} 域的复制值,正常情况以该顺序检测。如果没有作者(authors)和编者(authors),该列表是未定义的。注意: 该列表也与\opt{use$<$name$>$} 相关,见\secref{use:opt:bib}。标注样式打印引用标签中的姓名时使用这一列表。提供该列表仅为方便起见,没有附加的意义。该域可以定制,详见\secref{aut:ctm:fld}。


\fielditem{extraname}{integer}

%Holds a count of the number of bibliography entries within a refsection which derive from the same \bibfield{labelname} list. This counter takes account of \opt{uniquename} settings (see \secref{use:opt:pre:int}). While not used by any standard styles, this field is useful in styles which wish to number bibliography entries on a per-\bibfield{labelname} basis. This field will only exist if there is a \bibfield{labelname}. The \bibfield{extraname} counter is related to, but conceptually different from \cmd{ifsingletitle} (see \secref{use:opt:pre:int} and \secref{aut:aux:tst}).
保存一个参考文献节内具有相同\bibfield{labelname}姓名列表的文献条目的数量。标准样式不使用。
该域对于希望在每个\bibfield{labelname}层次统计文献条目数很有用。该域仅在\bibfield{labelname}域存在时才会存在,
\bibfield{extraname}计数器逻辑上与 \cmd{ifsingletitle}类似,但概念不同(见\secref{use:opt:pre:int} 和 \secref{aut:aux:tst})。

\fielditem{labelnumber}{literal}

%The number of the bibliography entry, as required by numeric citation schemes. If the \bibfield{shorthand} field is defined, \biblatex does not assign a number to the respective entry. In this case \bibfield{labelnumber} is the shorthand rather than a number. Numeric styles must use the value of this field instead of a counter. Note that this field needs to be requested with the package option \opt{labelnumber}, see \secref{use:opt:pre:int} for details. Also see the package option \opt{defernumbers} in \secref{use:opt:pre:gen}.
参考文献条目的序号,用于顺序编码类的样式。如果定义了\bibfield{shorthand} 域,\biblatex 不再给各条目赋予一个数值。这种情况下,\bibfield{labelnumber} 就是shorthand而不是一个数字。顺序编码类的样式必须使用该域的值而不是一个计数器值。注意: 使用该域需要启用包选项\opt{labelnumber},详见\secref{use:opt:pre:int}。另可参见\secref{use:opt:pre:gen} 节的\opt{defernumbers} 选项。


\fielditem{labelprefix}{literal}

%If the \opt{labelprefix} option of \cmd{newrefcontext} has been set in order to prefix all entries in a subbibliography with a fixed string, this string is available in the \bibfield{labelprefix} field of all affected entries. If no prefix has been set, the \bibfield{labelprefix} field of the respective entry is undefined. See the \opt{labelprefix} option of \cmd{newrefcontext} in \secref{use:bib:context} for details. If the \bibfield{shorthand} field is defined, \biblatex does not assign the prefix to the \bibfield{labelprefix} field of the respective entry. In this case, the \bibfield{labelprefix} field is undefined.
如果要在一个subbibliography文献表的所有条目前都添加一个固定的字符串,设置了\cmd{newrefcontext} 命令的\opt{labelprefix} 选项,那么所有受影响的\bibfield{labelprefix} 域将提供该字符串。如果未设置前缀,相应条目的\bibfield{labelprefix} 域是未定义的。详见\secref{use:bib:context} 节\cmd{newrefcontext} 命令的\opt{labelprefix} 选项。如果定义了\bibfield{shorthand} 域,\biblatex 不会给相应条目的\bibfield{labelprefix} 域设置前缀。这种情况下\bibfield{labelprefix} 是未定义的。


\fielditem{labeltitle}{literal}
%The printable title of a work. In some circumstances, a style might need to choose a title from a list of a possible title fields. For example, citation styles printing short titles may want to print the \bibfield{shorttitle} field if it exists but otherwise print the \bibfield{title} field. The list of fields to be considered when constructing \bibfield{labeltitle} may be customized. See \secref{aut:ctm:fld} for details. Note that the \bibfield{extratitle} field needs to be requested with the package option \opt{labeltitle}, see \secref{use:opt:pre:int} for details. See also \bibfield{extratitle}. Note also that the \bibfield{extratitleyear} field needs to be requested with the package option \opt{labeltitleyear}. See also \bibfield{extratitleyear}.
一篇文献可打印题名(标题)。在一些环境中,一个样式可能需要在一些可能的标题域中选择一个标题。例如,标注样式打印短标题可能需要打印\bibfield{shorttitle} 域(如果它存在的话),否则将打印\bibfield{title} 域。构建\bibfield{labeltitle} 时考虑的域的列表可以自定义。详见 \secref{aut:ctm:fld}。注意: 使用\bibfield{extratitle} 域要求启用\opt{labeltitle} 包选项,详见\secref{use:opt:pre:int}。另可参见\bibfield{extratitle}。也要注意使用\bibfield{extratitleyear} 域也需要启用\opt{labeltitleyear} 包选项,另可参见\bibfield{extratitleyear}。


\fielditem{extratitle}{integer}

%It is sometimes useful, for example in author"=title citation schemes, to be able to disambiguate works with the same title. For works by the same \bibfield{labelname} with the same \bibfield{labeltitle}, the \bibfield{extratitle} field holds an integer which may be converted to a letter with \cmd{mknumalph} or formatted in some other way (or it can be merely used as a flag to say that some other field such as a date should be used in conjunction with the \bibfield{labeltitle} field). This field is undefined if there is only one work with the same \bibfield{labeltitle} by the same \bibfield{labelname} in the bibliography. Note that the \bibfield{extratitle} field needs to be requested with the package option \opt{labeltitle}, see \secref{use:opt:pre:int} for details. See also \bibfield{labeltitle}. Table \ref{use:opt:tab1} summarises the various \opt{extra*} disambiguation counters and what they track.
该命令有时很有用,比如在author-title标注样式中,用于区分标题相同的文献。当有文献具有相同的\bibfield{labelname} 和\bibfield{labeltitle} 时,\bibfield{extratitle} 域保存一个整数,可以利用\cmd{mknumalph} 转换为一个字母或者以其它方式格式化(或者可以仅仅作为一个标志,用于表示将一些其它域比如日期与\bibfield{labeltitle} 域合并)。当文献表中具有相同\bibfield{labeltitle} 和\bibfield{labelname} 的文献只有一篇时,该域不定义。注意:使用\bibfield{extratitle} 域需要启用\opt{labeltitle} 包选项,详见\secref{use:opt:pre:int}。另可参见\bibfield{labeltitle}。\ref{use:opt:tab1} 总结了各种\opt{extra*} 计数器及其作用。


\fielditem{extratitleyear}{integer}

%It is sometimes useful, for example in author"=title citation schemes, to be able to disambiguate works with the same title in the same year but with no author. For works with the same \bibfield{labeltitle} and with the same \bibfield{labelyear}, the \bibfield{extratitleyear} field holds an integer which may be converted to a letter with \cmd{mknumalph} or formatted in some other way (or it can be merely used as a flag to say that some other field such as a publisher should be used in conjunction with the \bibfield{labelyear} field). This field is undefined if there is only one work with the same \bibfield{labeltitle} and \bibfield{labelyear} in the bibliography. Note that the \bibfield{extratitleyear} field needs to be requested with the package option \opt{labeltitleyear}, see \secref{use:opt:pre:int} for details. See also \bibfield{labeltitleyear}. Table \ref{use:opt:tab1} summarises the various \opt{extra*} disambiguation counters and what they track.
该命令有时很有用,比如在author-title标注样式中,用于区分标题相同年份相同但没有责任者的文献。当有文献具有相同的\bibfield{labeltitle} 和\bibfield{labelyear},\bibfield{extratitleyear} 域保存一个整数,可以利用\cmd{mknumalph} 转换为一个字母或者以其它方式格式化(或者可以仅仅作为一个标志,用于表示将一些其它域比如出版者与\bibfield{labelyear} 域合并)。当文献表中具有相同\bibfield{labeltitle} 和\bibfield{labelyear} 的文献只有一篇时,该域不定义。注意: 使用\bibfield{extratitle} 域需要启用\opt{labeltitleyear} 包选项,详见\secref{use:opt:pre:int}。另可参见\bibfield{labeltitleyear}。\ref{use:opt:tab1} 总结了各种\opt{extra*} 计数器及其作用。


\fielditem{labelyear}{literal}

%The year of the date field selected by \cmd{DeclareLabeldate} (\secref{aut:ctm:fld}) or the \bibfield{year} field, for use in author-year labels. A complete author-year label consists of the fields \bibfield{labelyear} plus \bibfield{extrayear}. Note that the \bibfield{labelyear} and \bibfield{extrayear} fields need to be requested with the package option \opt{labeldateparts}, see \secref{use:opt:pre:int} for details. See also \bibfield{extrayear}.
\cmd{DeclareLabeldate}(\secref{aut:ctm:fld})命令选择的日期域的年,或者\bibfield{year} 域,用于作者年制的标签。一个完整的作者年制的标签由\bibfield{labelyear} 加\bibfield{extrayear} 域构成。注意使用\bibfield{labelyear} 和\bibfield{extrayear} 域需要启用\opt{labeldateparts} 包选项,详见\secref{use:opt:pre:int}。另可参见\bibfield{extrayear}。


\fielditem{labelendyear}{literal}

%The end year of the date field selected by \cmd{DeclareLabeldate} (\secref{aut:ctm:fld}) if the selected date is a range.
\cmd{DeclareLabeldate} (\secref{aut:ctm:fld})命令选择的日期域的终止年,如果选择的日期是一个范围。


\fielditem{labelmonth}{datepart}

%The month of the date field selected by \cmd{DeclareLabeldate} (\secref{aut:ctm:fld}), or the \bibfield{month} field for use in author-year labels. Note that the \bibfield{labelmonth} field needs to be requested with the package option \opt{labeldateparts}, see \secref{use:opt:pre:int} for details.
\cmd{DeclareLabeldate}(\secref{aut:ctm:fld})命令选择的日期域的月,或者\bibfield{month} 域,用于作者年制的标签。注意使用\bibfield{labelmonth} 域需要启用 \opt{labeldateparts} 包选项,详见\secref{use:opt:pre:int}。


\fielditem{labelendmonth}{datepart}

%The end month of the date field selected by \cmd{DeclareLabeldate} (\secref{aut:ctm:fld}) if the selected date is a range.
\cmd{DeclareLabeldate} (\secref{aut:ctm:fld})命令选择的日期域的终止月,如果选择的日期是一个范围。


\fielditem{labelday}{datepart}

%The month of the date field selected by \cmd{DeclareLabeldate} (\secref{aut:ctm:fld}) for use in author-year labels. Note that the \bibfield{labelday} field needs to be requested with the package option \opt{labeldateparts}, see \secref{use:opt:pre:int} for details.
\cmd{DeclareLabeldate}(\secref{aut:ctm:fld})命令选择的日期域的日,或者\bibfield{month} 域,用于作者年制的标签。注意使用 \bibfield{labelday} 域要求启用 \opt{labeldateparts} 包选项,详见\secref{use:opt:pre:int}。


\fielditem{labelendday}{datepart}

%The end day of the date field selected by \cmd{DeclareLabeldate} (\secref{aut:ctm:fld}) if the selected date is a range.
\cmd{DeclareLabeldate} (\secref{aut:ctm:fld})命令选择的日期域的终止日,如果选择的日期是一个范围。


\fielditem{extradate}{integer}

%The author"=year citation scheme usually requires a letter to be appended to the year if the bibliography contains two or more works by the same author which were all published in the same year. In this case, the \bibfield{extradate} field holds an integer which may be converted to a letter with \cmd{mknumalph} or formatted in some other way. This field is undefined if there is only one work by the author in the bibliography or if all works by the author have different publication years. A complete author-year label consists of the fields \bibfield{labelyear} plus \bibfield{extradate}. Note that the \bibfield{labelyear} and \bibfield{extradate} fields need to be requested with the package option \opt{labeldateparts}, see \secref{use:opt:pre:int} for details. See also \bibfield{labelyear}. Table \ref{use:opt:tab1} summarises the various \opt{extra*} disambiguation counters and what they track.
当参考文献表中包含两个或更多的具有相同作者的文献且出版年份也相同时,author-year标注样式常需要在年后面附加一个字母以示区分。这种情况下,\bibfield{extradate} 域保存一个整数可以利用\cmd{mknumalph} 转换为一个字母或者以其它方式格式化。当文献表中某作者的文献只有一篇或者所有该作者的文献的出版年不同时,该域不定义。完整的作者年标签由\bibfield{labelyear} 加\bibfield{extradate} 域构成。注意使用\bibfield{labelyear} 和\bibfield{extradate} 域需要启用\opt{labeldateparts} 包选项,详见\secref{use:opt:pre:int}。另可参见\bibfield{labelyear}。\ref{use:opt:tab1} 总结了各种\opt{extra*} 计数器及其作用。


\fielditem{extradatescope}{literal}

%This field contains the name of the most specific field which determined the value of \bibfield{extradate}. It is not used by the standard styles but may be useful in controlling the placement of the \bibfield{extradate} field value.
该域包含确定\bibfield{extradate}值的专用域的域名,标准样式不使用,但可能用于控制\bibfield{extradate}域值的配置。


\end{fieldlist}

\paragraph{Date的成分域}%Date Component Fields
\label{aut:bbx:fld:dat}

注意,可以在数据模型中定义新的日期域,这些新定义的日期域的使用方式与本节将介绍的默认数据模型类似。
%Note that it is possible to define new date fields in the datamodel which behave exactly like the default data model date fields described in this section.

\file{bib} 文件中的日期域与样式接口提供的日期域如何关联详见表\ref{aut:bbx:fld:tab1}。对样式中像\bibfield{origdate} 这样的域做判断时,使用如下代码:
%See \tabref{aut:bbx:fld:tab1} for an overview of how the date fields in \file{bib} files are related to the date fields provided by the style interface. When testing for a field like \bibfield{origdate} in a style, use code like:

\begin{ltxcode}
<<\iffieldundef>>{orig<<year>>}{...}{...}
\end{ltxcode}
%
它将告诉你相应的日期是否已定义。下面的判断:
%This will tell you if the corresponding date is defined at all. This test:

\begin{ltxcode}
<<\iffieldundef>>{orig<<endyear>>}{...}{...}
\end{ltxcode}
%
将告诉你相应的日期和一个(完全确定的)范围是否已定义。下面的判断
%will tell you if the corresponding date is defined and a (fully specified) range. This test:

\begin{ltxcode}
<<\iffieldequalstr>>{orig<<endyear>>}{}{...}{...}
\end{ltxcode}
%
将告诉你相应的日期和一个无期限的(open-ended)范围已经定义。 无期限(Open-ended,无终止日期的)范围由一个空的\texttt{endyear} 成分表示(而不是一个未定义的\texttt{endyear} 成分)。更多示例详见\secref{bib:use:dat} 节和\pageref{bib:use:tab1} 页的表\ref{bib:use:tab1}。
%will tell you if the corresponding date is defined and an open"=ended range. Open"=ended ranges are indicated by an empty \texttt{endyear} component (as opposed to an undefined \texttt{endyear} component). See \secref{bib:use:dat} and \tabref{bib:use:tab1} on page~\pageref{bib:use:tab1} for further examples.

\begingroup
\tablesetup
\begin{longtable}[l]{%
	@{}V{0.15\textwidth}%
	@{}V{0.4\textwidth}%
	@{}V{0.3\textwidth}%
	@{}V{0.2\textwidth}@{}}
\toprule
\multicolumn{2}{@{}H}{\file{bib} File} &
\multicolumn{2}{H}{Data Interface} \\
\cmidrule(r){1-2}\cmidrule(l){3-4}
\multicolumn{1}{@{}H}{Field} &
\multicolumn{1}{H}{Value (Example)} &
\multicolumn{1}{H}{Field} &
\multicolumn{1}{H}{Value (Example)} \\
\cmidrule{1-1}\cmidrule(lr){2-2}\cmidrule(l){3-3}\cmidrule(l){4-4}
date		& 1988			& day		& undefined \\
		&			& month		& undefined \\
		&			& year		& 1988 \\
		&			& season  & undefined \\
		&			& endday	& undefined \\
		&			& endmonth	& undefined \\
		&			& endyear	& undefined \\
		&			& endseason  & undefined \\
		&			& hour	& undefined \\
		&			& minute	& undefined \\
		&			& second	& undefined \\
		&			& timezone & undefined \\
		&			& endhour	& undefined \\
		&			& endminute	& undefined \\
		&			& endsecond	& undefined \\
		&			& endtimezone & undefined \\
date		& 1997/			& day		& undefined \\
		&			& month		& undefined \\
		&			& year		& 1997 \\
		&			& season  & undefined \\
		&			& endday	& undefined \\
		&			& endmonth	& undefined \\
		&			& endyear	& empty \\
		&			& endseason  & undefined \\
		&			& hour	& undefined \\
		&			& minute	& undefined \\
		&			& second	& undefined \\
		&			& timezone & undefined \\
		&			& endhour	& undefined \\
		&			& endminute	& undefined \\
		&			& endsecond	& undefined \\
		&			& endtimezone & undefined \\
urldate		& 2009-01-31		& urlday	& 31 \\
		&			& urlmonth	& 01 \\
		&			& urlyear	& 2009 \\
		&			& urlseason  & undefined \\
		&			& urlendday	& undefined \\
		&			& urlendmonth	& undefined \\
		&			& urlendyear	& undefined \\
		&			& urlendseason  & undefined \\
		&			& urlhour	& undefined \\
		&			& urlminute	& undefined \\
		&			& urlsecond	& undefined \\
		&			& urltimezone & undefined \\
		&			& urlendhour	& undefined \\
		&			& urlendminute	& undefined \\
		&			& urlendsecond	& undefined \\
		&			& urlendtimezone & undefined \\
urldate		& 2009-01-31T15:34:04Z		& urlday	& 31 \\
		&			& urlmonth	& 01 \\
		&			& urlyear	& 2009 \\
		&			& urlseason  & undefined \\
		&			& urlendday	& undefined \\
		&			& urlendmonth	& undefined \\
		&			& urlendyear	& undefined \\
		&			& urlendseason  & undefined \\
		&			& urlhour	& 15 \\
		&			& urlminute	& 34 \\
		&			& urlsecond	& 04 \\
		&			& urltimezone & Z \\
		&			& urlendhour	& undefined \\
		&			& urlendminute	& undefined \\
		&			& urlendsecond	& undefined \\
		&			& urlendtimezone & undefined \\
urldate		& 2009-01-31T15:34:04+05:00		& urlday	& 31 \\
		&			& urlmonth	& 01 \\
		&			& urlyear	& 2009 \\
		&			& urlseason  & undefined \\
		&			& urlendday	& undefined \\
		&			& urlendmonth	& undefined \\
		&			& urlendyear	& undefined \\
		&			& urlendseason  & undefined \\
		&			& urlhour	& 15 \\
		&			& urlminute	& 34 \\
		&			& urlsecond	& 04 \\
		&			& urltimezone & +0500 \\
		&			& urlendhour	& undefined \\
		&			& urlendminute	& undefined \\
		&			& urlendsecond	& undefined \\
		&			& urlendtimezone & undefined \\
urldate		& \parbox[t]{0.4\textwidth}{2009-01-31T15:34:04/\\2009-01-31T16:04:34}& urlday	& 31 \\
		&			& urlmonth	& 1 \\
		&			& urlyear	& 2009 \\
		&			& urlseason  & undefined \\
		&			& urlendday	& 31 \\
		&			& urlendmonth	& 1 \\
		&			& urlendyear	& 2009 \\
		&			& urlendseason  & undefined \\
		&			& urlhour	& 15 \\
		&			& urlminute	& 34 \\
		&			& urlsecond	& 4 \\
		&			& urltimezone & floating \\
		&			& urlendhour	& 16 \\
		&			& urlendminute	& 4 \\
		&			& urlendsecond	& 34 \\
		&			& urlendtimezone & floating \\
origdate	& 2002-21/2002-23	& origday	& undefined \\
		&			& origmonth	& 01 \\
		&			& origyear	& 2002 \\
		&			& origseason  & spring \\
		&			& origendday	& undefined \\
		&			& origendmonth	& 02 \\
		&			& origendyear	& 2002 \\
		&			& origendseason  & autumn \\
		&			& orighour	& undefined \\
		&			& origminute	& undefined \\
		&			& origsecond	& undefined \\
		&			& origtimezone & undefined \\
		&			& origendhour	& undefined \\
		&			& origendminute	& undefined \\
		&			& origendsecond	& undefined \\
		&			& origendtimezone & undefined \\
eventdate	& 1995-01-31/1995-02-05	& eventday	& 31 \\
		&			& eventmonth	& 01 \\
		&			& eventyear	& 1995 \\
		&			& eventseason  & undefined \\
		&			& eventendday	& 05 \\
		&			& eventendmonth	& 02 \\
		&			& eventendyear	& 1995 \\
		&			& eventendseason  & undefined \\
		&			& eventhour	& undefined \\
		&			& eventminute	& undefined \\
		&			& eventsecond	& undefined \\
		&			& eventtimezone & undefined \\
		&			& eventendhour	& undefined \\
		&			& eventendminute	& undefined \\
		&			& eventendsecond	& undefined \\
		&			& eventendtimezone & undefined \\
\bottomrule
%\end{tabularx}
\caption{日期接口(译者注:biblatex3.7版提供的四个可解析日期接口,分别是date,origdate,eventdate,urldate,在多数场合已经够用)}%Date Interface
\label{aut:bbx:fld:tab1}
\end{longtable}
\endgroup

\begin{fieldlist}

\fielditem{hour}{datepart}

该域保存\bibfield{date} 域的小时(hour)成分,当日期是一个范围时,它保存开始日期的小时成分。
%This field holds the hour component of the \bibfield{date} field. If the date is a range, it holds the hour component of the start date.

\fielditem{minute}{datepart}

该域保存\bibfield{date} 域的分钟成分,当日期是一个范围时,它保存开始日期的分钟成分。
%This field holds the minute component of the \bibfield{date} field. If the date is a range, it holds the minute component of the start date.

\fielditem{second}{datepart}

该域保存\bibfield{date} 域的秒钟成分,当日期是一个范围时,它保存开始日期的秒钟成分。
%This field holds the second component of the \bibfield{date} field. If the date is a range, it holds the second component of the start date.

\fielditem{timezone}{datepart}

该域保存\bibfield{date} 域的时区成分,当日期是一个范围时,它保存开始日期的时区成分。
%This field holds the timezone component of the \bibfield{date} field. If the date is a range, it holds the timezone component of the start date.

\fielditem{day}{datepart}

该域保存\bibfield{date} 域的日成分,当日期是一个范围时,它保存开始日期的日成分。
%This field holds the day component of the \bibfield{date} field. If the date is a range, it holds the day component of the start date.

\fielditem{month}{datepart}

该域保存数据源文件中的\bibfield{month} 域或者\bibfield{date} 域的月成分,当日期是一个范围时,它保存开始日期的月成分。
%This field is the \bibfield{month} as given in the database file or it holds the month component of the \bibfield{date} field. If the date is a range, it holds the month component of the start date.

\fielditem{year}{datepart}

该域保存数据源文件中的\bibfield{year} 域或者\bibfield{date} 域的年成分,当日期是一个范围时,它保存开始日期的年成分。
%This field is the \bibfield{year} as given in the database file or it holds the year component of the \bibfield{date} field. If the date is a range, it holds the year component of the start date.

\fielditem{season}{datepart}

该域保存由\acr{ISO8601-2} 4.7(见\secref{bib:use:dat})规定的\bibfield{date} 域的季节成分,它包含一个季节本地化字符串。当日期是一个范围时,它保存开始日期的季节成分。
%This field holds the season component of the \bibfield{date} field as specified by \acr{ISO8601-2} 4.7 (\secref{bib:use:dat}). It contains a season localisation string (\secref{aut:lng:key:dt}). If the date is a range, it holds the season component of the start date.

\fielditem{endhour}{datepart}

如果\bibfield{date} 域中给出的日期是一个范围,该域保存结束日期的小时成分。
%If the date specification in the \bibfield{date} field is a range, this field holds the hour component of the end date.

\fielditem{endminute}{datepart}

如果\bibfield{date} 域中给出的日期是一个范围,该域保存结束日期的分钟成分。
%If the date specification in the \bibfield{date} field is a range, this field holds the minute component of the end date.

\fielditem{endsecond}{datepart}

如果\bibfield{date} 域中给出的日期是一个范围,该域保存结束日期的秒钟成分。
%If the date specification in the \bibfield{date} field is a range, this field holds the second component of the end date.

\fielditem{endtimezone}{datepart}

如果\bibfield{date} 域中给出的日期是一个范围,该域保存结束日期的时区成分。
%If the date specification in the \bibfield{date} field is a range, this field holds the timezone component of the end date.

\fielditem{endday}{datepart}

如果\bibfield{date} 域中给出的日期是一个范围,该域保存结束日期的日成分。
%If the date specification in the \bibfield{date} field is a range, this field holds the day component of the end date.

\fielditem{endmonth}{datepart}

如果\bibfield{date} 域中给出的日期是一个范围,该域保存结束日期的月成分。
%If the date specification in the \bibfield{date} field is a range, this field holds the month component of the end date.

\fielditem{endyear}{datepart}

如果\bibfield{date} 域中给出的日期是一个范围,该域保存结束日期的年成分。空的(但已定义)的\bibfield{endyear} 成分表示无期限的日期范围。
%If the date specification in the \bibfield{date} field is a range, this field holds the year component of the end date. A blank (but defined) \bibfield{endyear} component indicates an open ended \bibfield{date} range.

\fielditem{endseason}{datepart}

如果\bibfield{date} 域中给出的日期是一个范围,该域保存\acr{EDTF} 5.2.5 (\secref{bib:use:dat})规定的结束日期的季节成分。它包含一个季节本地化字符串(见\secref{aut:lng:key:dt}),空的(但已定义)的\bibfield{endseason} 成分表示无期限的日期范围。
%If the date specification in the \bibfield{date} field is a range, this field holds the season component of the end date as specified by \acr{EDTF} 5.2.5 (\secref{bib:use:dat}). It contains a season localisation string (\secref{aut:lng:key:dt}). A blank (but defined) \bibfield{endseason} component indicates an open ended \bibfield{date} range.

\fielditem{orighour}{datepart}

该域保存\bibfield{origdate} 域的小时(hour)成分,当日期是一个范围时,它保存开始日期的小时成分。
%This field holds the hour component of the \bibfield{origdate} field. If the date is a range, it holds the hour component of the start date.

\fielditem{origminute}{datepart}

该域保存\bibfield{origdate} 域的分钟成分,当日期是一个范围时,它保存开始日期的分钟成分。
%This field holds the minute component of the \bibfield{origdate} field. If the date is a range, it holds the minute component of the start date.

\fielditem{origsecond}{datepart}

该域保存\bibfield{origdate} 域的秒钟成分,当日期是一个范围时,它保存开始日期的秒钟成分。
%This field holds the second component of the \bibfield{origdate} field. If the date is a range, it holds the second component of the start date.

\fielditem{origtimezone}{datepart}

该域保存\bibfield{origdate} 域的时区成分,当日期是一个范围时,它保存开始日期的时区成分。
%This field holds the timezone component of the \bibfield{origdate} field. If the date is a range, it holds the timezone component of the start date.

\fielditem{origday}{datepart}

该域保存\bibfield{origdate} 域的日成分,当日期是一个范围时,它保存开始日期的日成分。
%This field holds the day component of the \bibfield{origdate} field. If the date is a range, it holds the day component of the start date.

\fielditem{origmonth}{datepart}

该域保存\bibfield{origdate} 域的月成分,当日期是一个范围时,它保存开始日期的月成分。
%This field holds the month component of the \bibfield{origdate} field. If the date is a range, it holds the month component of the start date.

\fielditem{origyear}{datepart}

该域保存\bibfield{origdate} 域的年成分,当日期是一个范围时,它保存开始日期的年成分。
%This field holds the year component of the \bibfield{origdate} field. If the date is a range, it holds the year component of the start date.

\fielditem{origseason}{datepart}

该域保存由\acr{ISO8601-2} 4.7(见\secref{bib:use:dat})规定的\bibfield{origdate} 域的季节成分,它包含一个季节本地化字符串。当日期是一个范围时,它保存开始日期的季节成分。
%This field holds the season component of the \bibfield{origdate} field as specified by \acr{ISO8601-2} 4.7 (\secref{bib:use:dat}). It contains a season localisation string (\secref{aut:lng:key:dt}). If the date is a range, it holds the season component of the start date.

\fielditem{origendhour}{datepart}

如果\bibfield{origdate} 域中给出的日期是一个范围,该域保存结束日期的小时成分。
%If the date specification in the \bibfield{origdate} field is a range, this field holds the hour component of the end date.

\fielditem{origendminute}{datepart}

如果\bibfield{origdate} 域中给出的日期是一个范围,该域保存结束日期的分钟成分。
%If the date specification in the \bibfield{origdate} field is a range, this field holds the minute component of the end date.

\fielditem{origendsecond}{datepart}

如果\bibfield{origdate} 域中给出的日期是一个范围,该域保存结束日期的秒钟成分。
%If the date specification in the \bibfield{origdate} field is a range, this field holds the second component of the end date.

\fielditem{origendtimezone}{datepart}

如果\bibfield{origdate} 域中给出的日期是一个范围,该域保存结束日期的时区成分。
%If the date specification in the \bibfield{origdate} field is a range, this field holds the timezone component of the end date.

\fielditem{origendday}{datepart}

如果\bibfield{origdate} 域中给出的日期是一个范围,该域保存结束日期的日成分。
%If the date specification in the \bibfield{origdate} field is a range, this field holds the day component of the end date.

\fielditem{origendmonth}{datepart}

如果\bibfield{origdate} 域中给出的日期是一个范围,该域保存结束日期的月成分。
%If the date specification in the \bibfield{origdate} field is a range, this field holds the month component of the end date.

\fielditem{origendyear}{datepart}

如果\bibfield{origdate} 域中给出的日期是一个范围,该域保存结束日期的年成分。空的(但已定义)的\bibfield{origendyear} 成分表示无期限的日期范围。
%If the date specification in the \bibfield{origdate} field is a range, this field holds the year component of the end date. A blank (but defined) \bibfield{origendyear} component indicates an open ended \bibfield{origdate} range.

\fielditem{origendseason}{datepart}

如果\bibfield{origdate} 域中给出的日期是一个范围,该域保存\acr{EDTF} 5.2.5 (\secref{bib:use:dat})规定的结束日期的季节成分。它包含一个季节本地化字符串(见\secref{aut:lng:key:dt}),空的(但已定义)的\bibfield{origendseason} 成分表示无期限的\bibfield{origdate} 范围。
%If the date specification in the \bibfield{origdate} field is a range, this field holds the season component of the end date as specified by \acr{EDTF} 5.2.5 (\secref{bib:use:dat}). It contains a season localisation string (\secref{aut:lng:key:dt}). A blank (but defined) \bibfield{origendseason} component indicates an open ended \bibfield{origdate} range.

\fielditem{eventhour}{datepart}

该域保存\bibfield{eventdate} 域的小时(hour)成分,当日期是一个范围时,它保存开始日期的小时成分。
%This field holds the hour component of the \bibfield{eventdate} field. If the date is a range, it holds the hour component of the start date.

\fielditem{eventminute}{datepart}

该域保存\bibfield{eventdate} 域的分钟成分,当日期是一个范围时,它保存开始日期的分钟成分。
%This field holds the minute component of the \bibfield{eventdate} field. If the date is a range, it holds the minute component of the start date.

\fielditem{eventsecond}{datepart}

该域保存\bibfield{eventdate} 域的秒钟成分,当日期是一个范围时,它保存开始日期的秒钟成分。
%This field holds the second component of the \bibfield{eventdate} field. If the date is a range, it holds the second component of the start date.

\fielditem{eventtimezone}{datepart}

该域保存\bibfield{eventdate} 域的时区成分,当日期是一个范围时,它保存开始日期的时区成分。
%This field holds the timezone component of the \bibfield{eventdate} field. If the date is a range, it holds the timezone component of the start date.

\fielditem{eventday}{datepart}

该域保存\bibfield{eventdate} 域的日成分,当日期是一个范围时,它保存开始日期的日成分。
%This field holds the day component of the \bibfield{eventdate} field. If the date is a range, it holds the day component of the start date.

\fielditem{eventmonth}{datepart}

该域保存\bibfield{eventdate} 域的月成分,当日期是一个范围时,它保存开始日期的月成分。
%This field holds the month component of the \bibfield{eventdate} field. If the date is a range, it holds the month component of the start date.

\fielditem{eventyear}{datepart}

该域保存\bibfield{eventdate} 域的年成分,当日期是一个范围时,它保存开始日期的年成分
%This field holds the year component of the \bibfield{eventdate} field. If the date is a range, it holds the year component of the start date.

\fielditem{eventseason}{datepart}

该域保存由\acr{ISO8601-2} 4.7(见\secref{bib:use:dat})规定的\bibfield{eventdate} 域的季节成分,它包含一个季节本地化字符串。当日期是一个范围时,它保存开始日期的季节成分。
%This field holds the season component of the \bibfield{eventdate} field as specified by \acr{ISO8601-2} 4.7 (\secref{bib:use:dat}). It contains a season localisation string (\secref{aut:lng:key:dt}). If the date is a range, it holds the season component of the start date.

\fielditem{eventendhour}{datepart}

如果\bibfield{eventdate} 域中给出的日期是一个范围,该域保存结束日期的小时成分。
%If the date specification in the \bibfield{eventdate} field is a range, this field holds the hour component of the end date.

\fielditem{eventendminute}{datepart}

如果\bibfield{eventdate} 域中给出的日期是一个范围,该域保存结束日期的分钟成分。
%If the date specification in the \bibfield{eventdate} field is a range, this field holds the minute component of the end date.

\fielditem{eventendsecond}{datepart}

如果\bibfield{eventdate} 域中给出的日期是一个范围,该域保存结束日期的秒钟成分。
%If the date specification in the \bibfield{eventdate} field is a range, this field holds the second component of the end date.

\fielditem{eventendtimezone}{datepart}

如果\bibfield{eventdate} 域中给出的日期是一个范围,该域保存结束日期的时区成分。
%If the date specification in the \bibfield{eventdate} field is a range, this field holds the timezone component of the end date.

\fielditem{eventendday}{datepart}

如果\bibfield{eventdate} 域中给出的日期是一个范围,该域保存结束日期的日成分。
%If the date specification in the \bibfield{eventdate} field is a range, this field holds the day component of the end date.

\fielditem{eventendmonth}{datepart}

如果\bibfield{eventdate} 域中给出的日期是一个范围,该域保存结束日期的月成分。
%If the date specification in the \bibfield{eventdate} field is a range, this field holds the month component of the end date.

\fielditem{eventendyear}{datepart}

如果\bibfield{eventdate} 域中给出的日期是一个范围,该域保存结束日期的年成分。空的(但已定义)的\bibfield{eventendyear} 成分表示无期限的日期范围。
%If the date specification in the \bibfield{eventdate} field is a range, this field holds the year component of the end date. A blank (but defined) \bibfield{eventendyear} component indicates an open ended \bibfield{eventdate} range.

\fielditem{eventendseason}{datepart}

如果\bibfield{eventdate} 域中给出的日期是一个范围,该域保存\acr{EDTF} 5.2.5 (\secref{bib:use:dat})规定的结束日期的季节成分。它包含一个季节本地化字符串(见\secref{aut:lng:key:dt}),空的(但已定义)的\bibfield{eventendseason} 成分表示无期限的\bibfield{eventdate} 范围。
%If the date specification in the \bibfield{eventdate} field is a range, this field holds the season component of the end date as specified by \acr{EDTF} 5.2.5 (\secref{bib:use:dat}). It contains a season localisation string (\secref{aut:lng:key:dt}). A blank (but defined) \bibfield{eventendseason} component indicates an open ended \bibfield{eventdate} range.

\fielditem{urlhour}{datepart}

该域保存\bibfield{urldate} 域的小时(hour)成分,当日期是一个范围时,它保存开始日期的小时成分。
%This field holds the hour component of the \bibfield{urldate} field. If the date is a range, it holds the hour component of the start date.

\fielditem{urlminute}{datepart}

该域保存\bibfield{urldate} 域的分钟成分,当日期是一个范围时,它保存开始日期的分钟成分。
%This field holds the minute component of the \bibfield{urldate} field. If the date is a range, it holds the minute component of the start date.

\fielditem{urlsecond}{datepart}

该域保存\bibfield{urldate} 域的秒钟成分,当日期是一个范围时,它保存开始日期的秒钟成分。
%This field holds the second component of the \bibfield{urldate} field. If the date is a range, it holds the second component of the start date.

\fielditem{timezone}{urldatepart}

该域保存\bibfield{urldate} 域的时区成分,当日期是一个范围时,它保存开始日期的时区成分。
%This field holds the timezone component of the \bibfield{urldate} field. If the date is a range, it holds the timezone component of the start date.

\fielditem{urlday}{datepart}

该域保存\bibfield{urldate} 域的日成分。
%This field holds the day component of the \bibfield{urldate} field.

\fielditem{urlmonth}{datepart}

该域保存\bibfield{urldate} 域的月成分。
%This field holds the month component of the \bibfield{urldate} field.

\fielditem{urlyear}{datepart}

该域保存\bibfield{urldate} 域的年成分。
%This field holds the year component of the \bibfield{urldate} field.

\fielditem{urlseason}{datepart}

该域保存由\acr{ISO8601-2} 4.7(见\secref{bib:use:dat})规定的 \bibfield{urldate} 域的季节成分,它包含一个季节本地化字符串。当日期是一个范围时,它保存开始日期的季节成分。
%This field holds the season component of the \bibfield{urldate} field as specified by \acr{ISO8601-2} 4.7 (\secref{bib:use:dat}). It contains a season localisation string (\secref{aut:lng:key:dt}). If the date is a range, it holds the season component of the start date.

\fielditem{urlendhour}{datepart}

如果\bibfield{urldate} 域中给出的日期是一个范围,该域保存结束日期的小时成分
%If the date specification in the \bibfield{urldate} field is a range, this field holds the hour component of the end date.

\fielditem{urlendminute}{datepart}

如果\bibfield{urldate} 域中给出的日期是一个范围,该域保存结束日期的分钟成分
%If the date specification in the \bibfield{urldate} field is a range, this field holds the minute component of the end date.

\fielditem{urlendsecond}{datepart}

如果\bibfield{urldate} 域中给出的日期是一个范围,该域保存结束日期的秒钟成分
%If the date specification in the \bibfield{urldate} field is a range, this field holds the second component of the end date.

\fielditem{urlendtimezone}{datepart}

如果\bibfield{urldate} 域中给出的日期是一个范围,该域保存结束日期的时区成分
%If the date specification in the \bibfield{urldate} field is a range, this field holds the timezone component of the end date.

\fielditem{urlendday}{datepart}

如果\bibfield{urldate} 域中给出的日期是一个范围,该域保存结束日期的日成分
%If the date specification in the \bibfield{urldate} field is a range, this field holds the day component of the end date.

\fielditem{urlendmonth}{datepart}

如果\bibfield{urldate} 域中给出的日期是一个范围,该域保存结束日期的月成分
%If the date specification in the \bibfield{urldate} field is a range, this field holds the month component of the end date.

\fielditem{urlendyear}{datepart}

如果\bibfield{urldate} 域中给出的日期是一个范围,该域保存结束日期的年成分。空的(但已定义)的\bibfield{urlendyear} 成分表示无期限的日期范围。
%If the date specification in the \bibfield{urldate} field is a range, this field holds the year component of the end date. A blank (but defined) \bibfield{urlendyear} component indicates an open ended \bibfield{urldate} range.

\fielditem{urlendseason}{datepart}

如果\bibfield{urldate} 域中给出的日期是一个范围,该域保存\acr{EDTF} 5.2.5 (\secref{bib:use:dat})规定的结束日期的季节成分。它包含一个季节本地化字符串(见\secref{aut:lng:key:dt}),空的(但已定义)的\bibfield{urlendseason} 成分表示无期限的\bibfield{eventdate} 范围。
%If the date specification in the \bibfield{urldate} field is a range, this field holds the season component of the end date as specified by \acr{EDTF} 5.2.5 (\secref{bib:use:dat}). It contains a season localisation string (\secref{aut:lng:key:dt}). A blank (but defined) \bibfield{urlendseason} component indicates an open ended \bibfield{urldate} range.

\end{fieldlist}

\subsection{标注样式}%Citation Styles
\label{aut:cbx}
\gls{参考文献标注样式} 是诸如\cmd{cite} 等用于打印不同类型标注的命令集。这些样式定义在后缀为\file{cbx} 的文件中。\biblatex 在包末尾加载它们。注意:一些标准标注样式的常用共享宏集放在\path{biblatex.def} 文件中。这一文件也在包末尾加载,先于选择的标注样式。它也包含有来自\secref{use:cit:txt} 节的命令的定义。
%A citation style is a set of commands such as \cmd{cite} which print different types of citations. Such styles are defined in files with the suffix \file{cbx}. The \biblatex package loads the selected citation style file at the end of the package. Note that a small repertory of frequently used macros shared by several of the standard citation styles is also included in \path{biblatex.def}. This file is loaded at the end of the package as well, prior to the selected citation style. It also contains the definitions of the commands from \secref{use:cit:txt}.

\subsubsection{标注样式文件}% Citation Style Files
\label{aut:cbx:cbx}
在讨论标注样式文件提供的各个命令前,观察如下一个典型\file{cbx} 文件的总体结构:
%Before we go over the individual commands available in citation style files, consider this example of the overall structure of a typical \file{cbx} file:

\begin{ltxexample}
\ProvidesFile{example.cbx}[2006/03/15 v1.0 biblatex citation style]

\DeclareCiteCommand{\cite}{...}{...}{...}{...}
\DeclareCiteCommand{\parencite}[\mkbibparens]{...}{...}{...}{...}
\DeclareCiteCommand{\footcite}[\mkbibfootnote]{...}{...}{...}{...}
\DeclareCiteCommand{\textcite}{...}{...}{...}{...}
\endinput
\end{ltxexample}

\begin{ltxsyntax}

\cmditem{RequireCitationStyle}{style}

这个命令是可选的,用于加载在一些更一般样式基础上构建特殊的标注样式。它加载标注样式\path{style.cbx}。
%This command is optional and intended for specialized citation styles built on top of a more generic style. It loads the citation style \path{style.cbx}.

\cmditem{InitializeCitationStyle}{code}

指定初始化或重设标注样式需要的任意\prm{code}。这个钩子将在包加载的时候执行一次,在使用\secref{use:cit:msc} 节的\cmd{citereset} 命令时则每次都执行。\cmd{citereset} 命令也将重设本宏包的内部标注追踪器。它会影响\secref{aut:aux:tst} 节中列出的\cmd{ifciteseen}, \cmd{ifentryseen}, \cmd{ifciteibid} 和\cmd{ifciteidem} 等判断。当使用\env{refsection} 环境时,标注追踪器重设的是当前的\env{refsection} 局部环境。
%Specifies arbitrary \prm{code} required to initialize or reset the citation style. This hook will be executed once at package load-time and every time the \cmd{citereset} command from \secref{use:cit:msc} is used. The \cmd{citereset} command also resets the internal citation trackers of this package. The reset will affect the \cmd{ifciteseen}, \cmd{ifentryseen}, \cmd{ifciteibid}, and \cmd{ifciteidem} tests discussed in \secref{aut:aux:tst}. When used in a \env{refsection} environment, the reset of the citation tracker is local to the current \env{refsection} environment.

\cmditem{OnManualCitation}{code}
指定重设部分标注样式需要的任意\prm{code}。这一钩子将在使用\secref{use:cit:msc} 中的\cmd{mancite} 命令时调用。它有时特别有用,可以代替像 <ibidem>或<op. cit.>等缩写表示的重复标注,因为当自动生成和人工产生的标注混合使用的时候这些缩写可能会有歧义。\cmd{mancite} 命令也会重设宏包的内部<ibidem>和<idem>追踪器,进而影响\secref{aut:aux:tst} 节讨论的\cmd{ifciteibid} 和\cmd{ifciteidem} 判断。
%Specifies arbitrary \prm{code} required for a partial reset of the citation style. This hook will be executed every time the \cmd{mancite} command from \secref{use:cit:msc} is used. It is particularly useful in citation styles which replace repeated citations by abbreviations like <ibidem> or <op. cit.> which may get ambiguous if automatically generated and manual citations are mixed. The \cmd{mancite} command also resets the internal <ibidem> and <idem> trackers of this package. The reset will affect the \cmd{ifciteibid} and \cmd{ifciteidem} tests discussed in \secref{aut:aux:tst}.

\cmditem{DeclareCiteCommand}{command}[wrapper]{precode}{loopcode}{sepcode}{postcode} \cmditem*{DeclareCiteCommand*}{command}[wrapper]{precode}{loopcode}{sepcode}{postcode}

%This is the core command used to define all citation commands. It takes one optional and five mandatory arguments. The \prm{command} is the command to be defined, for example \cmd{cite}. If the optional \prm{wrapper} argument is given, the entire citation will be passed to the \prm{wrapper} as an argument, \ie the wrapper command must take one mandatory argument.\footnote{Typical examples of wrapper commands are \cmd{mkbibparens} and \cmd{mkbibfootnote}.} The \prm{precode} is arbitrary code to be executed at the beginning of the citation. It will typically handle the \prm{prenote} argument which is available in the \bibfield{prenote} field. It may also be used to initialize macros required by the \prm{loopcode}. The \prm{loopcode} is arbitrary code to be executed for each entry key passed to the \prm{command}. This is the core code which prints the citation labels or any other data. The \prm{sepcode} is arbitrary code to be executed after each iteration of the \prm{loopcode}. It will only be executed if a list of entry keys is passed to the \prm{command}. The \prm{sepcode} will usually insert some kind of separator, such as a comma or a semicolon. The \prm{postcode} is arbitrary code to be executed at the end of the citation. The \prm{postcode} will typically handle the \prm{postnote} argument which is available in the \bibfield{postnote} field.\footnote{The bibliographic data available to the \prm{loopcode} is the data of the entry currently being processed. In addition to that, the data of the first entry is available to the \prm{precode} and the data of the last one is available to the \prm{postcode}. <First> and <last> refer to the order in which the citations are printed. If the \opt{sortcites} package option is active, this is the order of the list after sorting. Note that no bibliographic data is available to the \prm{sepcode}.} The starred variant of \cmd{DeclareCiteCommand} defines a starred \prm{command}. For example, |\DeclareCiteCommand*{cite}| would define |\cite*|.\footnote{Note that the regular variant of \cmd{DeclareCiteCommand} defines a starred version of the \prm{command} implicitly, unless the starred version has been defined before. This is intended as a fallback. The implicit definition is an alias for the regular variant.}

这是用于定义所有标注(引用)命令的核心命令。它有1个可选参数和5个必选参数。\prm{command} 是要定义的命令,比如\cmd{cite}。如果给出可选的\prm{wrapper} 参数,整个标注将会作为一个参数传递给\prm{wrapper},即包围(wrapper)命令必须要取得一个必选参数。\footnote{典型的包围命令是\cmd{mkbibparens} 和\cmd{mkbibfootnote}。}
\prm{precode} 是在标注开始时执行的任意代码。典型地,它将处理由\bibfield{prenote} 域提供的\prm{prenote} 参数。它可以可用来对\prm{loopcode} 所需的宏进行初始化。\prm{loopcode} 是每个条目关键词传递给\prm{command} 命令时执行的任意代码。它是打印标注标签或其它任意数据的核心代码。\prm{sepcode} 是每次执行\prm{loopcode} 完成后执行的代码。它仅在条目关键词列表传递给\prm{command} 时起作用。\prm{sepcode} 常用于插入一些分隔符,比如逗号或分号等。
\prm{postcode} 是在标注结束时执行的代码。典型地,它将处理由\bibfield{postnote} 域提供的\prm{postnote} 参数。\footnote{能给\prm{loopcode} 提供的参考文献数据是正在处理的条目数据。此外,第一个(<First>)条目的数据可以用于\prm{precode},最后一个(<last>)条目的数据可以用于\prm{postcode}。<First> and <last> 指的是标注的打印顺序。如果\opt{sortcites} 包选项启用,这是经过排序处理后的顺序。注意: 任何参考文献数据无法用于\prm{sepcode}。}
带星号的\cmd{DeclareCiteCommand} 命令定义了一个带星号的\prm{command}。例如|\DeclareCiteCommand*{cite}|命令将定义|\cite*|。
\footnote{注意:无星号的\cmd{DeclareCiteCommand} 命令也将定义隐式的定义一个带星号的标注命令,除非该标注命令前面已经定义。这只是用于提供备选。这种隐式方式定义的命令将等同于不带星号的命令。}



\cmditem{DeclareMultiCiteCommand}{command}[wrapper]{cite}{delimiter}

该命令定义<multicite>类命令(见\secref{use:cit:mlt})。\prm{command} 是要定义的multicite命令,比如\cmd{cites}。它自动在由\cmd{DeclareCiteCommand} 定义的后端命令基础上构建鲁棒的命令,其中\prm{cite} 参数用于指定使用的后端命令名。注意后端命令的包围命令(封套)(即传递给\cmd{DeclareCiteCommand} 命令的\prm{wrapper} 参数)自动忽略。使用可选的\prm{wrapper} 参数作为替换。\prm{delimiter} 是列表中单个标注之间的分隔字符串。下面给出的示例是典型的\cmd{multicitedelim} 命令,取自\path{biblatex.def} 中的真实定义:
%This command defines <multicite> commands (\secref{use:cit:mlt}). The \prm{command} is the multicite command to be defined, for example \cmd{cites}. It is automatically made robust. Multicite commands are built on top of backend commands defined with \cmd{DeclareCiteCommand} and the \prm{cite} argument specifies the name of the backend command to be used. Note that the wrapper of the backend command (\ie the \prm{wrapper} argument passed to \cmd{DeclareCiteCommand}) is ignored. Use the optional \prm{wrapper} argument to specify an alternative wrapper. The \prm{delimiter} is the string to be printed as a separator between the individual citations in the list. This will typically be \cmd{multicitedelim}. The following examples are real definitions taken from \path{biblatex.def}:

\begin{ltxexample}
\DeclareMultiCiteCommand{\cites}%
	{\cite}{\multicitedelim}
\DeclareMultiCiteCommand{\parencites}[\mkbibparens]%
	{\parencite}{\multicitedelim}
\DeclareMultiCiteCommand{\footcites}[\mkbibfootnote]%
	{\footcite}{\multicitedelim}
\end{ltxexample}

\cmditem{DeclareAutoCiteCommand}{name}[position]{cite}{multicite}

%This command provides definitions for the \cmd{autocite} and \cmd{autocites} commands from \secref{use:cit:aut}. The definitions are enabled with the \opt{autocite} package option from \secref{use:opt:pre:gen}. The \prm{name} is an identifier which serves as the value passed to the package option. The autocite commands are built on top of backend commands like \cmd{parencite} and \cmd{parencites}. The arguments \prm{cite} and \prm{multicite} specify the backend commands to use. The \prm{cite} argument refers to \cmd{autocite} and \prm{multicite} refers to \cmd{autocites}. The \prm{position} argument controls the handling of any punctuation marks after the citation. Possible values are \texttt{l}, \texttt{r}, \texttt{f}. \texttt{r} means that the punctuation is placed to the right of the citation, \ie it will not be moved around. \texttt{l} means that any punctuation after the citation is moved to the left of the citation. \texttt{f} is like \texttt{r} in a footnote and like \texttt{l} otherwise. This argument is optional and defaults to \texttt{r}. See also \cmd{DeclareAutoPunctuation} in \secref{aut:pct:cfg} and the \opt{autopunct} package option in \secref{use:opt:pre:gen}. The following examples are real definitions taken from \path{biblatex.def}:
该命令为\cmd{autocite} 和\cmd{autocites} 类命令提供定义(见\secref{use:cit:aut})。要使定义生效需要启用 \secref{use:opt:pre:gen} 节的\opt{autocite} 包选项。\prm{name} 是一个标识向包选项传递一个值。autocite类命令是在\cmd{parencite} 和\cmd{parencites} 等后端命令基础上构建的。\prm{cite} 和\prm{multicite} 参数指定了使用的后端命令。\prm{cite} 参数用于\cmd{autocite},而\prm{multicite} 用于\cmd{autocites}。
\prm{position} 参数控制标注后的任何标点符号的处理。可能的值是\texttt{l}, \texttt{r}, \texttt{f}。\texttt{r} 表示标点置于标注的右侧,即它不会移动。 \texttt{l} 表示将标点移动到标注的左侧。\texttt{f} 在脚注中的作用类似于\texttt{r},在其它情况下则类似于\texttt{l}。该参数是可选的,默认是\texttt{r}。另可参见\secref{aut:pct:cfg} 节的\cmd{DeclareAutoPunctuation} 命令和\secref{use:opt:pre:gen} 节的\opt{autopunct} 包选项。下面的示例取自\path{biblatex.def} 中的真实定义:



\begin{ltxexample}
\DeclareAutoCiteCommand{plain}{\cite}{\cites}
\DeclareAutoCiteCommand{inline}{\parencite}{\parencites}
\DeclareAutoCiteCommand{footnote}[l]{\footcite}{\footcites}
\DeclareAutoCiteCommand{footnote}[f]{\smartcite}{\smartcites}
\end{ltxexample}
%
文档导言区提供的定义可以利用如下方式随后采用(见\secref{use:cfg:opt}):
%A definition provided in the document preamble can be subsequently adopted with the following: (see \secref{use:cfg:opt}).

\begin{ltxexample}
\ExecuteBibliographyOptions{autocite=<<name>>}
\end{ltxexample}

\cmditem{DeclareCitePunctuationPosition}{command}{position}

%Set up the cite command \prm{command} to move punctuation marks after the citation like \cmd{autocite}. The \prm{position} argument can take the values \opt{r}, \opt{l}, \opt{f}, \opt{c}, \opt{o} and \opt{d}.
%If an unknown \prm{position} identifier is used, it defaults to \opt{o}.

设置命令\prm{command}实现类似\cmd{autocite}的标点移动。
位置(\prm{position})参数的值包括 \opt{r}, \opt{l}, \opt{f}, \opt{c}, \opt{o} and \opt{d}。
当给出的位置标识不在这些值中,则默认设置为\opt{o}。

\begin{valuelist}
\item[r] %The punctuation mark is not moved and remains to the right of the citation.
移动标点符号,保留在标注的右侧。

\item[l] %The punctuation mark is moved to the left of the citation and thus appears before it.
移动标点符号到标注的左侧,即在标注之前显示。

\item[f] %Like \opt{r} in footnotes and like \opt{l} otherwise.
在脚注中同\opt{r},否则同\opt{l}。

\item[c] %Pass the punctuation on to the internal implementation of the citation commands. It will then be executed within the \prm{wrapper} command if given.
传递标点进入标注命令内部,将在标注命令内部的\prm{wrapper}命令中执行。


\item[o] %Retain the default set-up of \opt{c} for citation defined commands without \prm{wrapper} command and \opt{l} for citation commands defined with a \prm{wrapper} command.
当标注命令中不存在\prm{wrapper}时,与\opt{c}默认设置相同。
当标注命令中存在\prm{wrapper}命令,则与\opt{l}默认设置相同。


\item[d] %Drop the explicit punctuation mark. It will only be available as the field \bibfield{postpunct}.
丢弃显式的标点符号,标点符号仅由域\bibfield{postpunct}提供。

\end{valuelist}
%
This command can not be used for \cmd{autocite}, to configure \cmd{autocite} use the optional \prm{position} argument for \cmd{DeclareAutoCiteCommand}.
\end{ltxsyntax}

\subsubsection{特殊域}%Special Fields
\label{aut:cbx:fld}

下面的域用于向标注命令传递数据。它们不\file{bib} 文件中使用而由宏包自动定义。从标注样式的角度看,它们与\file{bib} 中的域并无区别。另可参见\secref{aut:bbx:fld}。
%The following fields are used by \biblatex to pass data to citation commands. They are not used in \file{bib} files but defined automatically by the package. From the perspective of a citation style, they are not different from the fields in a \file{bib} file. See also \secref{aut:bbx:fld}.

\begin{fieldlist}

\fielditem{prenote}{literal}

作为\prm{prenote} 参数向标注命令传递。该域仅用于标注而不能用在参考文献表中。如果\prm{prenote} 参数缺省或为空,该域不定义。
%The \prm{prenote} argument passed to a citation command. This field is specific to citations and not available in the bibliography. If the \prm{prenote} argument is missing or empty, this field is undefined.

\fielditem{postnote}{literal}

作为\prm{postnote} 参数向标注命令传递。该域仅用于标注而不能用在参考文献表中。如果\prm{postnote} 参数缺省或为空,该域不定义。
%The \prm{postnote} argument passed to a citation command. This field is specific to citations and not available in the bibliography. If the \prm{postnote} argument is missing or empty, this field is undefined.

\fielditem{multiprenote}{literal}

作为\prm{multiprenote} 参数向multicite类标注命令传递。该域仅用于标注而不能用在参考文献表中。如果\prm{multiprenote} 参数缺省或为空,该域不定义。
%The \prm{multiprenote} argument passed to a multicite command. This field is specific to citations and not available in the bibliography. If the \prm{multiprenote} argument is missing or empty, this field is undefined.

\fielditem{multipostnote}{literal}

作为\prm{multipostnote} 参数向ulticite类标注命令传递。该域仅用于标注而不能用在参考文献表中。如果\prm{multipostnote} 参数缺省或为空,该域不定义。
%The \prm{multipostnote} argument passed to a multicite command. This field is specific to citations and not available in the bibliography. If the \prm{multipostnote} argument is missing or empty, this field is undefined.

\fielditem{postpunct}{punctuation command}

作为拖尾的标点参数隐式地向标注命令传递。该域仅用于标注而不能用在参考文献表中。如果一个标注命令后面跟着的字符不在\cmd{DeclareAutoPunctuation} (\secref{aut:pct:cfg})命令的定义中,该域不定义。
%The trailing punctuation argument implicitly passed to a citation command. This field is specific to citations and not available in the bibliography. If the character following a given citation command is not specified in \cmd{DeclareAutoPunctuation} (\secref{aut:pct:cfg}), this field is undefined.

\end{fieldlist}

\subsection{数据接口}%Data Interface
\label{aut:bib}

数据接口是用于格式化和打印全部参考文献数据的工具。这些工具在著录和标注样式中均可使用。
%The data interface are the facilities used to format and print all bibliographic data. These facilities are available in both bibliography and citation styles.

\subsubsection{数据命令}%Data Commands
\label{aut:bib:dat}
本节介绍\biblatex 包的主要数据接口。这些命令处理了绝大部分工作,即实际上由它们来对列表和域提供的数据进行打印。
%This section introduces the main data interface of the \biblatex package. These are the commands doing most of the work, \ie they actually print the data provided in lists and fields.

\begin{ltxsyntax}

\cmditem{DeprecateField}{field}{message}
\cmditem{DeprecateList}{list}{message}
\cmditem{DeprecateName}{name}{message}

用于在打印\prm{field}, \prm{list}, \prm{name} 时给出表示不允许的警告信息\prm{message}。它能为那些需要在样式中修改域名的样式作者提供帮助。注意: 不允许的项只能是未在当前工作的数据模型中定义的项,\prm{field}, \prm{list} 或\prm{name} 不能出现在\cmd{DeclareDatamodelFields} 命令的参数中。
%When an attempt is made to print \prm{field}, \prm{list}, \prm{name}, a
%deprecation warning is issued with the additional \prm{message}.  This aids
%style authors who are changing field names in their style. Note that the
%deprecated item must no longer be defined in the datamodel for this work;
%\prm{field}, \prm{list} or \prm{name} cannot be listed anywhere as an
%argument to \cmd{DeclareDatamodelFields}.

\cmditem{DeprecateFieldWithReplacement}{field}{replacement}
\cmditem{DeprecateListWithReplacement}{list}{replacement}
\cmditem{DeprecateNameWithReplacement}{name}{replacement}

类似于\cmd{DeprecateField}, \cmd{DeprecateList} and \cmd{DeprecateName}.
这些命令不经能给出一个不允许的警告,还尝试定义用于代替不允许域输出的替代域。
\cmd{replacement}必须与不允许的\prm{field}, \prm{list} or \prm{name}域保持类型一致。
当打印不允许的域时需要应用\prm{replacement}的格式,那么需要由\cmd{DeclareFieldAlias}
给出请求(见\secref{aut:bib:fmt})。
注意:不允许的域必须是未在数据模型中定义的,即\prm{field}, \prm{list} or \prm{name}不能
出现在\cmd{DeclareDatamodelFields}的任意参数中。
%Similar to \cmd{DeprecateField}, \cmd{DeprecateList} and \cmd{DeprecateName}.
%The commands do not only issue a deprecation warning,
%they try to define a replacement for the deprecated field
%that is printed in its stead.
%The \cmd{replacement} must be of the same type as the deprecated
%\prm{field}, \prm{list} or \prm{name}.
%If the formatting of \prm{replacement} should be applied when printing
%the deprecated field, that needs to be requested with \cmd{DeclareFieldAlias}
%(see \secref{aut:bib:fmt}).
%Note that the deprecated item must no longer be defined in the datamodel
%for this work; \prm{field}, \prm{list} or \prm{name} cannot be listed
%anywhere as an argument to \cmd{DeclareDatamodelFields}.
\cmditem{printfield}[format]{field}

%This command prints a \prm{field} using the formatting directive \prm{format}, as defined with \cmd{DeclareFieldFormat}. If a type"=specific \prm{format} has been declared, the type"=specific formatting directive takes precedence over the generic one. If the \prm{field} is undefined, nothing is printed. If the \prm{format} is omitted, \cmd{printfield} tries using the name of the field as a format name. For example, if the \bibfield{title} field is to be printed and the \prm{format} is not specified, it will try to use the field format \texttt{title}.\footnote{In other words, \texttt{\textbackslash printfield\{title\}} is equivalent to \texttt{\textbackslash printfield[title]\{title\}}.} In this case, any type"=specific formatting directive will also take precedence over the generic one. If all of these formats are undefined, it falls back to \texttt{default} as a last resort. Note that \cmd{printfield} provides the name of the field currently being processed in \cmd{currentfield} for use in field formatting directives.
该命令使用由\cmd{DeclareFieldFormat} 定义的格式化指令\prm{format} 打印\prm{field}。如果声明了具体条目类型(type-specific)的格式指令,那么它将优先于设置的通用格式化指令。如果\prm{field} 未定义则不作打印。如果\prm{format} 缺省,\cmd{printfield} 将尝试使用以域名为指令名的格式化指令。例如:要打印\bibfield{title} 域,且\prm{format} 未指定,它将尝试使用域格式化指令\texttt{title}。\footnote{换句话说,\texttt{\textbackslash printfield\{title\}} 等价于\texttt{\textbackslash printfield[title]\{title\}}。}
这种情况下,任何具体类型的格式化指令将优先于通用指令。如果所有的这些格式都未定义,它将回退到\texttt{default} 作为最后的选择。注意: \cmd{printfield} 为格式化指令提供当前正在\cmd{currentfield} 中处理的域名。



\cmditem{printlist}[format][start\ensuremath\rangle--\ensuremath\langle stop]{literal list}

该命令对所有在\prm{literal list} 中的项进行循环处理,从项数\prm{start} 开始,到项数\prm{stop} 结束,包括\prm{start} 和\prm{stop}(所有的列表中项以1开始计数)。每一项都用由\cmd{DeclareListFormat} 定义的格式化指令\prm{format} 打印。如果声明了具体条目类型的格式指令,其将优先于设置的通用格式指令。如果\prm{literal list} 未定义则不作打印。如果\prm{format} 缺省,\cmd{printlist} 将尝试使用以列表名作为化指令名的格式化指令。这种情况下,任何具体类型的格式化指令将优先于通用指令采用。如果所有的这些格式都未定义,它将回退到\texttt{default} 作为最后的选择。\prm{start} 参数默认是1,\prm{stop} 默认是列表中项的总数。如果项的总数大于\prm{maxitems},\prm{stop} 默认为\prm{minitems}(见\secref{use:opt:pre:gen})。更多细节参见\cmd{printnames}。注意: \cmd{printlist} 为格式化指令提供当前正在\cmd{currentlist} 中处理的域名。
%This command loops over all items in a \prm{literal list}, starting at item number \prm{start} and stopping at item number \prm{stop}, including \prm{start} and \prm{stop} (all lists are numbered starting at~1). Each item is printed using the formatting directive \prm{format}, as defined with \cmd{DeclareListFormat}. If a type"=specific \prm{format} has been declared, the type"=specific formatting directive takes precedence over the generic one. If the \prm{literal list} is undefined, nothing is printed. If the \prm{format} is omitted, \cmd{printlist} tries using the name of the list as a format name. In this case, any type"=specific formatting directive will also take precedence over the generic one. If all of these formats are undefined, it falls back to \texttt{default} as a last resort. The \prm{start} argument defaults to 1; \prm{stop} defaults to the total number of items in the list. If the total number is greater than \prm{maxitems}, \prm{stop} defaults to \prm{minitems} (see \secref{use:opt:pre:gen}). See \cmd{printnames} for further details. Note that \cmd{printlist} provides the name of the literal list currently being processed in \cmd{currentlist} for use in list formatting directives.

\cmditem{printnames}[format][start\ensuremath\rangle--\ensuremath\langle stop]{name list}
该命令对所有在\prm{name list} 中的项进行循环处理,从项数\prm{start} 开始,到项数\prm{stop} 结束,包括\prm{start} 和\prm{stop}(所有的列表中项以1开始计数)。每一项都用由\cmd{DeclareNameFormat} 定义的格式化指令\prm{format} 打印。如果声明了具体条目类型的格式指令,其将优先于设置的通用格式指令。如果\prm{name list} 未定义则不作打印。如果\prm{format} 缺省, \cmd{printnames} 将尝试使用以列表名为指令名的格式化指令。这种情况下,任何具体类型的格式化指令将优先于通用指令采用。如果所有的这些格式都未定义,它将回退到\texttt{default} 作为最后的选择。\prm{start} 参数默认是1,\prm{stop} 默认是列表中项的总数。如果项的总数大于\prm{maxnames},\prm{stop} 默认为\prm{minnames}(见\secref{use:opt:pre:gen})。如果你要自己制定一个范围而又要使用默认的列表格式,第一个可选参数必须给出但要留空:
%This command loops over all items in a \prm{name list}, starting at item number \prm{start} and stopping at item number \prm{stop}, including \prm{start} and \prm{stop} (all lists are numbered starting at~1). Each item is printed using the formatting directive \prm{format}, as defined with \cmd{DeclareNameFormat}. If a type"=specific \prm{format} has been declared, the type"=specific formatting directive takes precedence over the generic one. If the \prm{name list} is undefined, nothing is printed. If the \prm{format} is omitted, \cmd{printnames} tries using the name of the list as a format name. In this case, any type"=specific formatting directive will also take precedence over the generic one. If all of these formats are undefined, it falls back to \texttt{default} as a last resort. The \prm{start} argument defaults to 1; \prm{stop} defaults to the total number of items in the list. If the total number is greater than \prm{maxnames}, \prm{stop} defaults to \prm{minnames} (see \secref{use:opt:pre:gen}). If you want to select a range but use the default list format, the first optional argument must still be given, but is left empty:

\begin{ltxexample}
\printnames[][1-3]{...}
\end{ltxexample}

\prm{start} 和\prm{stop} 之一可以缺省,因此下面的参数都是有效的:
%One of \prm{start} and \prm{stop} may be omitted, hence the following arguments are all valid:

\begin{ltxexample}
\printnames[...][-1]{...}
\printnames[...][2-]{...}
\printnames[...][1-3]{...}
\end{ltxexample}

如果你要重设\prm{maxnames} 和\prm{minnames} 并打印整个列表,你可以在第二个可选参数中以如下方式设置\cnt{listtotal} 计数器。
%If you want to override \prm{maxnames} and \prm{minnames} and force printing of the entire list, you may refer to the \cnt{listtotal} counter in the second optional argument:

\begin{ltxexample}
\printnames[...][-\value{listtotal}]{...}
\end{ltxexample}

%Whenever \cmd{printnames} and \cmd{printlist} process a list, information concerning the current state is accessible by way of four counters: the \cnt{listtotal} counter holds the total number of items in the current list, \cnt{listcount} holds the number of the item currently being processed, \cnt{liststart} is the \prm{start} argument passed to \cmd{printnames} or \cmd{printlist}, \cnt{liststop} is the \prm{stop} argument. These counters are intended for use in list formatting directives. \cnt{listtotal} may also be used in the second optional argument to \cmd{printnames} and \cmd{printlist}. Note that these counters are local to list formatting directives and do not hold meaningful values when used anywhere else. For every list, there is also a counter by the same name which holds the total number of items in the corresponding list. For example, the \cnt{author} counter holds the total number of items in the \bibfield{author} list. These counters are similar to \cnt{listtotal} except that they may also be used independently of list formatting directives. There are also \cnt{maxnames} and \cnt{minnames} as well as \cnt{maxitems} and \cnt{minitems} counters which hold the values of the corresponding package options. See \secref{aut:fmt:ilc} for a complete list of such internal counters. Note that \cmd{printnames} provides the name of the name list currently being processed in \cmd{currentname} for use in name formatting directives.

当\cmd{printnames} 和\cmd{printlist} 处理一个列表时,当前状态的信息可以通过4个计数器获知: \cnt{listtotal} 计数器保存当前列表中项的总数,\cnt{listcount} 保存当前正在处理的项的序号,\cnt{liststart} 是传递给\cmd{printnames} 或\cmd{printlist} 命令的 \prm{start} 参数,\cnt{liststop} 则是 \prm{stop} 参数。这些计数器用于列表的格式化指令。\cnt{listtotal} 也可以在\cmd{printnames} 和\cmd{printlist} 命令第二个可选参数中使用。
注意,这些计数器仅在列表格式化指令中有意义在其它任何地方都无效。对于每类列表,都有一个具有相同名字的计数器保存该类列表的项的总数。例如,\cnt{author} 计数器保存\bibfield{author} 列表中的项的总数。这些计数器类似于\cnt{listtotal},但可用于列表格式化指令之外。还有\cnt{maxnames} ,\cnt{minnames}, \cnt{maxitems} 和\cnt{minitems} 计数器,用于保存相应的包选项的值。这些内部计数器的完整列表详见\secref{aut:fmt:ilc}。注意: \cmd{printnames} 为格式化指令提供当前正在\cmd{currentname} 中处理的域名。



\cmditem{printtext}[format]{text}

该命令用于打印\prm{text},可以是可打印的文本或者产生可打印文本的任意代码。它清除插入\prm{text} 之前的标点缓存并且通知\biblatex 打印文本已经插入。这保证了所有之前和之后的\cmd{newblock} 和\cmd{newunit} 命令能产生预期的作用。\cmd{printfield}、\cmd{printnames} 、\cmd{bibstring} 及其相关命令都这般自动处理(见\secref{aut:str})。如果一个参考文献样式需要插入抄录文本(包括来自\secref{aut:pct:pct, aut:pct:spc} 的命令),需要使用该命令来确保block 和unit标点在\secref{aut:pct:new} 节中所述功能正常运转。可选参数\prm{format} 指定一个域格式指令用于格式化\prm{text}。当需要把若干个域打印为某一格式的集合块,这就会很有用,比如把集合块用括号或引号包围起来。如果声明了具体条目类型的格式化指令,其将优先于设置的通用格式化指令。如果\prm{format} 缺省,那么\prm{text} 如实输出(原样打印)。更多实用细节见第\secref{aut:cav:pct} 节。
%This command prints \prm{text}, which may be printable text or arbitrary code generating printable text. It clears the punctuation buffer before inserting \prm{text} and informs \biblatex that printable text has been inserted. This ensures that all preceding and following \cmd{newblock} and \cmd{newunit} commands have the desired effect. \cmd{printfield} and \cmd{printnames} as well as \cmd{bibstring} and its companion commands (see \secref{aut:str}) do that automatically. Using this command is required if a bibliography styles inserts literal text (including the commands from \secref{aut:pct:pct, aut:pct:spc}) to ensure that block and unit punctuation works as advertised in \secref{aut:pct:new}. The optional \prm{format} argument specifies a field formatting directive to be used to format \prm{text}. This may also be useful when several fields are to be printed as one chunk, for example, by enclosing the entire chunk in parentheses or quotation marks. If a type"=specific \prm{format} has been declared, the type"=specific formatting directive takes precedence over the generic one. If the \prm{format} is omitted, the \prm{text} is printed as is. See also \secref{aut:cav:pct} for some practical hints.

\cmditem{printfile}[format]{file}

该命令类似于\cmd{printtext},差别在于第二个参数是一个文件名而不是抄录文本。\prm{file} 参数必须是一个能在\tex 搜索路径找到的有效的\latex 文件。\cmd{printfile} 将使用\cmd{input} 来加载该\prm{file}。如果指定文件不存在,\cmd{printfile} 不做任何操作。可选的\prm{format} 参数指定了一个域格式化指令应用于该\prm{file}。如果声明了type"=specific的格式指令,其将优先于设置的通用格式指令。如果\prm{format} 缺省,那么\prm{file} 如实输出(原样打印)。注意该功能需要显式的启用\secref{use:opt:pre:gen} 节的包选项\opt{loadfiles}。默认情况下,\cmd{printfile} 不加载任何文件。
%This command is similar to \cmd{printtext} except that the second argument is a file name rather than literal text. The \prm{file} argument must be the name of a valid \latex file found in \tex's search path. \cmd{printfile} will use \cmd{input} to load this \prm{file}. If there is no such file, \cmd{printfile} does nothing. The optional \prm{format} argument specifies a field formatting directive to be applied to the \prm{file}. If a type"=specific \prm{format} has been declared, the type"=specific formatting directive takes precedence over the generic one. If the \prm{format} is omitted, the \prm{file} is printed as is. Note that this feature needs to be enabled explicitly by setting the package option \opt{loadfiles} from \secref{use:opt:pre:gen}. By default, \cmd{printfile} will not input any files.

\csitem{printdate}

该命令打印条目定义在\bibfield{date} 或\bibfield{month}\slash\bibfield{year} 域中的日期。日期格式由\secref{use:opt:pre:gen} 节中的\opt{date} 包选项控制。另外也可以通过调整域格式\texttt{date} (见\secref{aut:fmt:ich})来进一步格式化(比如设置字体等)。注意: 该命令与标点追踪器自动交互,不必使用\cmd{printtext} 命令将其包围起来。
%This command prints the date of the entry, as specified in the fields \bibfield{date} or \bibfield{month}\slash \bibfield{year}. The date format is controlled by the package option \opt{date} from \secref{use:opt:pre:gen}. Additional formatting (fonts etc.) may be applied by adjusting the field format \texttt{date} (\secref{aut:fmt:ich}). Note that this command interfaces with the punctuation tracker. There is no need to wrap it in a \cmd{printtext} command.

\csitem{printdateextra}

类似于\cmd{printdate},但指定的日期域是\bibfield{extradate} 域。用于设计作者年制的参考文献样式。
%Similar to \cmd{printdate} but incorporates the \bibfield{extradate} field in the date specification. This is useful for bibliography styles designed for author-year citations.

\csitem{printlabeldate}

类似于\cmd{printdate},但打印的是日期域由\cmd{DeclareLabeldate} 决定。日期格式由\secref{use:opt:pre:gen} 节中的\opt{labeldate} 包选项控制。另外也可以通过调整域格式\texttt{labeldate} (见\secref{aut:fmt:ich})来进一步格式化。
%Similar to \cmd{printdate} but prints the date field determined by \cmd{DeclareLabeldate}. The date format is controlled by the package option \opt{labeldate} from \secref{use:opt:pre:gen}. Additional formatting may be applied by adjusting the field format \texttt{labeldate} (\secref{aut:fmt:ich}).

\csitem{printlabeldateextra}

类似于\cmd{printlabeldate},但指定的日期域是\bibfield{extradate} 域,用于设计作者年制的参考文献样式。
%Similar to \cmd{printlabeldate} but incorporates the \bibfield{extradate} field in the date specification. This is useful for bibliography styles designed for author-year citations.

\csitem{print$<$datetype$>$date}

类似于\cmd{printdate},但打印的是日期域是条目的\bibfield{$<$datetype$>$date} 域。日期格式由\secref{use:opt:pre:gen} 节中的\opt{$<$datetype$>$date} 包选项控制。另外也可以通过调整域格式\texttt{$<$datetype$>$date} (见\secref{aut:fmt:ich})来进一步格式化。$<$datetype$>$在默认数据模型中有:<> (用于\bibfield{date} 域), <orig>, <event> 和<url>。
%As \cmd{printdate} but prints the \bibfield{$<$datetype$>$date} of the entry. The date format is controlled by the package option \opt{$<$datetype$>$date} from \secref{use:opt:pre:gen}. Additional formatting may be applied by adjusting the field format \texttt{$<$datetype$>$date} (\secref{aut:fmt:ich}). The $<$datetype$>$s in the default data model are <> (for the main \bibfield{date} field), <orig>, <event> and <url>.

\csitem{printtime}

该命令打印条目定义在\bibfield{date} 域(见\secref{bib:use:dat})中的时间范围,时间格式由\secref{use:opt:pre:gen} 节中的\opt{time} 包选项控制。另外也可以通过调整域格式\texttt{time} (见\secref{aut:fmt:ich})来进一步格式化(比如设置字体等)。时间格式化相关内容还包括\opt{timezeros} 选项,\cmd{bibtimesep} 和\cmd{bibtimezonesep} 宏(\secref{use:fmt:lng})。注意: 该命令与标点追踪器自动交互,不必使用\cmd{printtext} 命令将其包围起来。注意该命令打印的是独立于日期成分(元素)的时间范围。当\opt{$<$datepart$>$dateusetime} 选项启用时,也可以与日期范围一起打印,而不各自分开打印。
%This command prints the time range of the entry, as specified in the \bibfield{date} field (see \secref{bib:use:dat}). The time format is controlled by the package option \opt{time} from \secref{use:opt:pre:gen}. Additional formatting (fonts etc.) may be applied by adjusting the field format \texttt{time} (\secref{aut:fmt:ich}). Relevant to time formatting are the \opt{timezeros} option and the \cmd{bibtimesep} and \cmd{bibtimezonesep} macros (\secref{use:fmt:lng}). Note that this command interfaces with the punctuation tracker. There is no need to wrap it in a \cmd{printtext} command. Note that this command prints a stand-alone time range apart from the date elements. With the \opt{$<$datepart$>$dateusetime} option, you can have the printed along with a date when printing a date range instead of printing the time range completely separately, which is what this command allows for.

\csitem{print$<$datetype$>$time}

类似于\cmd{printtime},但打印的是条目的\bibfield{$<$datetype$>$time} 域。时间格式由\secref{use:opt:pre:gen} 节中的\opt{$<$datetype$>$time} 包选项控制。另外也可以通过调整域格式\texttt{$<$datetype$>$time}(见\secref{aut:fmt:ich})来进一步格式化。$<$datetype$>$在默认数据模型中有:<> (用于\bibfield{date} 域), <orig>, <event> 和<url>。
%As \cmd{printtime} but prints the \bibfield{$<$datetype$>$time} of the entry. The time format is controlled by the package option \opt{$<$datetype$>$time} from \secref{use:opt:pre:gen}. Additional formatting may be applied by adjusting the field format \texttt{$<$datetype$>$time} (\secref{aut:fmt:ich}). The $<$datetype$>$s in the default data model are <> (for the main \bibfield{date} field), <orig>, <event> and <url>.

\cmditem{indexfield}[format]{field}

该命令类似于\cmd{printfield},差别在于不是打印\prm{field} 而是将其添加到索引中,其格式化指令\prm{format} 由\cmd{DeclareIndexFieldFormat} 命令定义。如果声明了具体条目类型的格式指令,其将优先于设置的通用格式指令。如果\prm{field} 域未定义,该命令不做任何操作。如果\prm{format} 缺省,那么\cmd{indexfield} 将采用与域名相同的格式名。这种情况下任何具体条目类型的格式指令都将优先于通用的格式指令。若所有的这些格式都未定义,那么将采用\texttt{default} 格式作为最后的选择。
%This command is similar to \cmd{printfield} except that the \prm{field} is not printed but added to the index using the formatting directive \prm{format}, as defined with \cmd{DeclareIndexFieldFormat}. If a type"=specific \prm{format} has been declared, it takes precedence over the generic one. If the \prm{field} is undefined, this command does nothing. If the \prm{format} is omitted, \cmd{indexfield} tries using the name of the field as a format name. In this case, any type"=specific formatting directive will also take precedence over the generic one. If all of these formats are undefined, it falls back to \texttt{default} as a last resort.

\cmditem{indexlist}[format][start\ensuremath\rangle--\ensuremath\langle stop]{literal list}

该命令类似于\cmd{printlist},差别在于不是打印列表的项而是将其添加到索引中,其格式化指令\prm{format} 由\cmd{DeclareIndexListFormat} 命令定义。如果声明了具体条目类型的格式指令,其将优先于设置的通用格式指令。如果\prm{literal list} 未定义,该命令不做任何操作。如果\prm{format} 缺省,那么\cmd{indexlist} 将采用与列表名相同的格式名。这种情况下任何具体条目类型的格式指令都将优先于通用的格式指令。若所有的这些格式都未定义,那么将采用\texttt{default} 格式作为最后的选择。
%This command is similar to \cmd{printlist} except that the items in the list are not printed but added to the index using the formatting directive \prm{format}, as defined with \cmd{DeclareIndexListFormat}. If a type"=specific \prm{format} has been declared, the type"=specific formatting directive takes precedence over the generic one. If the \prm{literal list} is undefined, this command does nothing. If the \prm{format} is omitted, \cmd{indexlist} tries using the name of the list as a format name. In this case, any type"=specific formatting directive will also take precedence over the generic one. If all of these formats are undefined, it falls back to \texttt{default} as a last resort.

\cmditem{indexnames}[format][start\ensuremath\rangle--\ensuremath\langle stop]{name list}

该命令类似于\cmd{printnames},差别在于不是打印姓名列表的项而是将其添加到索引中,其格式化指令\prm{format} 由\cmd{DeclareIndexNameFormat} 命令定义。如果声明了具体条目类型的格式指令,其将优先于设置的通用格式指令。如果\prm{name list} 未定义,该命令不做任何操作。如果\prm{format} 缺省,那么\cmd{indexnames} 将采用与列表名相同的格式名。这种情况下任何具体条目类型的格式指令都将优先于通用的格式指令。若所有的这些格式都未定义,那么将采用\texttt{default} 格式作为最后的选择。
%This command is similar to \cmd{printnames} except that the items in the list are not printed but added to the index using the formatting directive \prm{format}, as defined with \cmd{DeclareIndexNameFormat}. If a type"=specific \prm{format} has been declared, the type"=specific formatting directive takes precedence over the generic one. If the \prm{name list} is undefined, this command does nothing. If the \prm{format} is omitted, \cmd{indexnames} tries using the name of the list as a format name. In this case, any type"=specific formatting directive will also take precedence over the generic one. If all of these formats are undefined, it falls back to \texttt{default} as a last resort.

\cmditem{entrydata}{key}{code}
\cmditem*{entrydata*}{key}{code}

类似\cmd{printfield} 的数据命令,正常情况下应用当前正在处理的条目数据。可以使用\cmd{entrydata} 在局部环境中转换应用数据。\prm{key} 是要局部使用的条目的关键词。\prm{code} 是在当前局部环境执行的任意代码。这一代码将在一个编组中执行。示例见\secref{aut:cav:mif} 节。注意: 该命令自动转换语言,如果\opt{autolang} 包选项启用的话。带星号的命令\cmd{entrydata*} 将复制封装条目(the enclosing entry)的所有域,并使用域、计数器和其它以字符串<\texttt{saved}>为前缀命名的资源。这在比较两个数据集时很有用。例如,在\prm{code} 的参数中,\bibfield{author} 域保存了条目\prm{key} 的作者,而封装条目的作者保存在\bibfield{savedauthor} 域中。\cnt{author} 计数器保存了\prm{key} 条目的\bibfield{author} 域的姓名数量,而封装条目的作者数量由\bibfield{savedauthor} 计数器保存。
%Data commands like \cmd{printfield} normally use the data of the entry currently being processed. You may use \cmd{entrydata} to switch contexts locally. The \prm{key} is the entry key of the entry to use locally. The \prm{code} is arbitrary code to be executed in this context. This code will be executed in a group. See \secref{aut:cav:mif} for an example. Note that this command will automatically switch languages if the \opt{autolang} package option is enabled. The starred version \cmd{entrydata*} will clone all fields of the enclosing entry, using field, counter, and other resource names prefixed with the string <\texttt{saved}>. This is useful when comparing two data sets. For example, inside the \prm{code} argument, the \bibfield{author} field holds the author of entry \prm{key} and the author of the enclosing entry is available as \bibfield{savedauthor}. The \cnt{author} counter holds the number of names in the \bibfield{author} field of \prm{key}; the \bibfield{savedauthor} counter refers to the author count of the enclosing entry.

\cmditem{entryset}{precode}{postcode}

该命令用在处理\bibtype{set} 条目集的参考文献驱动中。它将对由\bibfield{entryset} 域指出的集的所有成员进行循环处理,对集的各个成员执行相应的驱动。这相当于对每个集成员执行\secref{aut:aux:msc} 节的\cmd{usedriver} 命令。\prm{precode} 是在集的每项处理之前执行的任意代码。\prm{postcode} 是在集的每项处理之后执行的任意代码。两个参数语法上必须的,但可以留空。用法示例见\secref{aut:cav:set} 节。
%This command is intended for use in bibliography drivers handling \bibtype{set} entries. It will loop over all members of the set, as indicated by the \bibfield{entryset} field, and execute the appropriate driver for the respective set member. This is similar to executing the \cmd{usedriver} command from \secref{aut:aux:msc} for each set member. The \prm{precode} is arbitrary code to be executed prior to processing each item in the set. The \prm{postcode} is arbitrary code to be executed immediately after processing each item. Both arguments are mandatory in terms of the syntax but may be left empty. See \secref{aut:cav:set} for usage examples.

\cmditem{DeclareFieldInputHandler}{field}{code}

该命令用于定义从\file{.bbl} 读取数据所采用的域的数据输入处理器。在\prm{code} 内,宏\cmd{NewValue} 包含了域的值。比如,要忽略出现的\bibfield{volumes} 域,可以作:
%This command can be used to define a data input handler for \prm{field} when it is read from the \file{.bbl}. Within the \prm{code}, the macro \cmd{NewValue} contains the value of the field. For example, to ignore the \bibfield{volumes} field when it appears, you could do

\begin{ltxexample}
\DeclareFieldInputHandler{volumes}{\def\NewValue{}}
\end{ltxexample}
%
一般情况下,要删除和修改域需要使用\cmd{DeclareSourcemap}(见\secref{aut:ctm:map} 节),而这一替代方法在一些情形下会很有用,例如当强调的是数据的外观而不是数据本身时,因为\prm{code} 可以是任意的\tex 代码。
%Generally, you would want to use \cmd{DeclareSourcemap} (see \secref{aut:ctm:map}) to remove and modify fields but this alternative method may be useful in some circumstances when the emphasis is on appearance rather than data since the \prm{code} can be arbitraty \tex.

\cmditem{DeclareListInputHandler}{list}{code}

类似于\cmd{DeclareFieldInputHandler},但用于列表。在\prm{code} 内,宏\cmd{NewValue} 包含了列表的值,而\cmd{NewCount} 保存列表中项的序号。
%As \cmd{DeclareFieldInputHandler} but for lists. Within the \prm{code}, the macro \cmd{NewValue}
%contains the value of the list and \cmd{NewCount} contains the number of items in the list.

\cmditem{DeclareNameInputHandler}{name}{code}

类似于\cmd{DeclareFieldInputHandler},但用于姓名列表。在\prm{code} 内,宏\cmd{NewValue} 包含了姓名列表的值,而\cmd{NewCount} 保存列表中姓名的序号,\cmd{NewOption} 保存了\file{.bbl} 文件传递的各个姓名相关的任意选项。
%As \cmd{DeclareFieldInputHandler} but for names. Within the \prm{code}, the macro \cmd{NewValue}
%contains the value of the name, \cmd{NewCount} contains the number of individual names in the name and \cmd{NewOption} contains any per-name options passed in the \file{.bbl}.

\end{ltxsyntax}

\subsubsection{格式化指令}%Formatting Directives
\label{aut:bib:fmt}

本节介绍\secref{aut:bib:dat} 节的数据命令所需的格式化指令的定义命令。注意: 所有标注的格式定义在\path{biblatex_.def} 文件中。
%This section introduces the commands used to define the formatting directives required by the data commands from \secref{aut:bib:dat}. Note that all standard formats are defined in \path{biblatex_.def}.

\begin{ltxsyntax}

\cmditem{DeclareFieldFormat}[entrytype, \dots]{format}{code}
\cmditem*{DeclareFieldFormat}*{format}{code}

定义域格式\prm{format}。该格式化指令是由\cmd{printfield} 命令执行的任意\prm{code}。域的值作为仅有的第一个参数传递给\prm{code}。正在处理的域名在\prm{code} 中以\cmd{currentfield} 表示。如果指定一种条目类型(\prm{entrytype}),那么格式是该类型专属的。\prm{entrytype} 可以是一个逗号分隔(comma"=separated)的值列表。带星的命令类似于不带星的命令,区别在于它还将清除所有对具体条目类型做的格式定义。
%Defines the field format \prm{format}. This formatting directive is arbitrary \prm{code} to be executed by \cmd{printfield}. The value of the field will be passed to the \prm{code} as its first and only argument. The name of the field currently being processed is available to the \prm{code} as \cmd{currentfield}. If an \prm{entrytype} is specified, the format is specific to that type. The \prm{entrytype} argument may be a comma"=separated list of values. The starred variant of this command is similar to the regular version, except that all type-specific formats are cleared.

\cmditem{DeclareListFormat}[entrytype, \dots]{format}{code}
\cmditem*{DeclareListFormat}*{format}{code}

定义抄录文本列表\footnote{译者: literal 译为抄录文本} 的格式\prm{format}。格式化指令是\cmd{printlist} 命令处理列表中每一项时执行的任意\prm{code}。当前正在处理的项作为唯一的第一参数传递给\prm{code}。正在处理的文本列表名在\prm{code} 中以\cmd{currentlist} 表示。如果指定了\prm{entrytype},那么格式是该类型专属的。\prm{entrytype} 参数可以是一个逗号分隔(comma"=separated)的值列表。注意格式化指令也会处理在列表各项间插入的标点。需要对当前项处于列表中间或者末尾进行检测,即检查\cnt{listcount} 是否小于或等于\cnt{liststop}。带星的命令类似于不带星的命令,区别在于它还将清除所有对具体条目类型做的格式定义。
%Defines the literal list format \prm{format}. This formatting directive is arbitrary \prm{code} to be executed for every item in a list processed by \cmd{printlist}. The current item will be passed to the \prm{code} as its first and only argument. The name of the literal list currently being processed is available to the \prm{code} as \cmd{currentlist}. If an \prm{entrytype} is specified, the format is specific to that type. The \prm{entrytype} argument may be a comma"=separated list of values. Note that the formatting directive also handles the punctuation to be inserted between the individual items in the list. You need to check whether you are in the middle of or at the end of the list, \ie whether \cnt{listcount} is smaller than or equal to \cnt{liststop}. The starred variant of this command is similar to the regular version, except that all type-specific formats are cleared.

\cmditem{DeclareNameFormat}[entrytype, \dots]{format}{code}
\cmditem*{DeclareNameFormat}*{format}{code}

定义姓名列表的格式\prm{format}。格式化指令是\cmd{printnames} 命令处理列表中每一项时执行的任意\prm{code}。如果指定了\prm{entrytype},那么格式是该类型专属的。\prm{entrytype} 参数可以是一个逗号分隔(comma"=separated)的值列表。单个姓名的各个成分(组成部分)由自动创建的宏表示(见下)。默认数据模型定义了四个成分对应于标准的\bibtex 姓名成分参数。
%Defines the name list format \prm{format}. This formatting directive is arbitrary \prm{code} to be executed for every name in a list processed by \cmd{printnames}. If an \prm{entrytype} is specified, the format is specific to that type. The \prm{entrytype} argument may be a comma"=separated list of values. The individual parts of a name will be available in automatically created macros (see below). The default data mode defines four name part which correspond to the standard \bibtex name parts arguments:

\begin{argumentlist}{00}
\item[family] 姓,\bibtex 中为<last> name成分。当一个姓名只有一个成分时(比如 <Aristotle>),这一成分将被处理为姓。
%The family name(s), know as <last> in \bibtex.  If a name consists of a single part only (for example, <Aristotle>), this part will be treated as the family name.
\item[given] 名。注意名在\bibtex 中为<first> name成分。
%The given name(s). Note that given names are referred to as the <first> names in the \bibtex file format documentation.
\item[prefix] 尊称(前缀),比如von, van, of, da, de, del, della等。注意尊称在\bibtex 格式文件中为<von>成分。
%Any name prefices, for example von, van, of, da, de, del, della, etc. Note that name prefices are referred to as the <von> part of the name in the \bibtex file format documentation.
\item[suffix] 后缀,比如Jr, Sr等。注意后缀在\bibtex 格式文件中为<Jr>成分。
%Any name suffices, for example Jr, Sr. Note that name suffices are referred to as the <Jr> part of the name in the \bibtex file format documentation.
\end{argumentlist}
%
数据模型<nameparts>常量的值(见\secref{aut:bbx:drv})在姓名的数据模型中为每个姓名成分创建了两个宏。比如,在默认数据模型中,姓名格式由如下宏定义:
%The value of the datamodel <nameparts> constant (see \secref{aut:bbx:drv}) creates two macros for each name part in the datamodel for the name. So, for example, in the default data model, name formats will have defined the following macros:

\begin{ltxexample}
\namepartprefix %表示尊称(前缀)部分
\namepartprefixi %表示尊称首字母
\namepartfamily %表示姓
\namepartfamilyi %表示姓首字母
\namepartsuffix %表示后缀
\namepartsuffixi %表示后缀首字母
\namepartgiven %表示名
\namepartgiveni %表示名首字母
\end{ltxexample}
%
如果一个姓名的某些成分没有给出,相应的宏将为空,因此可以使用,比如\sty{etoolbox} 中\cmd{ifdefvoid} 这类的判断来检查姓名的各个成分。正在处理的姓名列表名在\prm{code} 中以\cmd{currentname} 表示。注意格式化指令也会处理在列表各项间插入的标点。需要对当前项是在列表中间或者末尾进行检测,即检查\cnt{listcount} 是否小于或等于\cnt{liststop}(见\secref{use:cav:nam} 节)。带星的命令类似于不带星的命令,区别在于它还将清除所有对具体条目类型做的格式定义。
%If a certain part of a name is not available, the corresponding macro will be empty, hence you may use, for example, the \sty{etoolbox} tests like \cmd{ifdefvoid} to check for the individual parts of a name. The name of the name list currently being processed is available to the \prm{code} as \cmd{currentname}. Note that the formatting directive also handles the punctuation to be inserted between separate names and between the individual parts of a name. You need to check whether you are in the middle of or at the end of the list, \ie whether \cnt{listcount} is smaller than or equal to \cnt{liststop}. See also \secref{use:cav:nam}. The starred variant of this command is similar to the regular version, except that all type-specific formats are cleared.

\cmditem{DeclareListWrapperFormat}[entrytype, \dots]{format}{code}
\cmditem*{DeclareListWrapperFormat}*{format}{code}

%Defines the list wrapper format \prm{format}. This formatting directive is arbitrary \prm{code} to be executed once for the entire list processed by \cmd{printlist}. The name of the literal list currently being processed is available to the \prm{code} as \cmd{currentlist}. If an \prm{entrytype} is specified, the format is specific to that type. The \prm{entrytype} argument may be a comma"=separated list of values. The starred variant of this command is similar to the regular version, except that all type-specific formats are cleared.
定义列表的封套(wrapper)格式。该格式化指令可以是任意代码,在\cmd{printlist}处理整个列表时仅执行一次。
文本列表名在处理过程中可以通过\cmd{currentlist}在\prm{code}中使用。如果制定了条目类型,该格式仅对该类型有效。
\prm{entrytype}参数可以是逗号分隔的列表。该命令带星号版本的差别在于它会清除所有制定条目类型的格式设置。


\cmditem{DeclareNameWrapperFormat}[entrytype, \dots]{format}{code}
\cmditem*{DeclareNameWrapperFormat}*{format}{code}

%Defines the list wrapper format \prm{format}. This formatting directive is arbitrary \prm{code} to be executed once for the entire name list processed by \cmd{printnames}. The name of the literal list currently being processed is available to the \prm{code} as \cmd{currentname}. If an \prm{entrytype} is specified, the format is specific to that type. The \prm{entrytype} argument may be a comma"=separated list of values. The starred variant of this command is similar to the regular version, except that all type-specific formats are cleared.
定义姓名列表的封套(wrapper)格式。该格式化指令可以是任意代码,在\cmd{printlist}处理整个列表时仅执行一次。
文本列表名在处理过程中可以通过\cmd{currentnames}在\prm{code}中使用。如果制定了条目类型,该格式仅对该类型有效。
\prm{entrytype}参数可以是逗号分隔的列表。该命令带星号版本的差别在于它会清除所有制定条目类型的格式设置。


\cmditem{DeclareIndexFieldFormat}[entrytype, \dots]{format}{code}
\cmditem*{DeclareIndexFieldFormat}*{format}{code}

定义域格式\prm{format}。该格式化指令是由\cmd{indexfield} 命令执行的任意\prm{code}。域的值作为仅有的第一个参数传递给\prm{code}。正在处理的域名在\prm{code} 中以\cmd{currentfield} 表示。如果指定一种条目类型(\prm{entrytype}),那么格式是该类型专属的。\prm{entrytype} 可以是一个逗号分隔(comma"=separated)的值列表。该命令类似于\cmd{DeclareFieldFormat},差别在于\prm{code} 处理的数据不是用于打印而是用于索引。注意\cmd{indexfield} 将执行\prm{code} 本身,即\prm{code} 必须包含\cmd{index} 或类似命令。带星的命令类似于不带星的命令,区别在于它还将清除所有对具体条目类型做的格式定义。
%Defines the field format \prm{format}. This formatting directive is arbitrary \prm{code} to be executed by \cmd{indexfield}. The value of the field will be passed to the \prm{code} as its first and only argument. The name of the field currently being processed is available to the \prm{code} as \cmd{currentfield}. If an \prm{entrytype} is specified, the format is specific to that type. The \prm{entrytype} argument may be a comma"=separated list of values. This command is similar to \cmd{DeclareFieldFormat} except that the data handled by the \prm{code} is not intended to be printed but written to the index. Note that \cmd{indexfield} will execute the \prm{code} as is, \ie the \prm{code} must include \cmd{index} or a similar command. The starred variant of this command is similar to the regular version, except that all type-specific formats are cleared.

\cmditem{DeclareIndexListFormat}[entrytype, \dots]{format}{code}
\cmditem*{DeclareIndexListFormat}*{format}{code}

定义抄录文本列表格式\prm{format}。该格式化指令是由\cmd{indexlist} 命令执行的任意\prm{code}。列表中当前值作为唯一参数传递给\prm{code}。正在处理的列表名在\prm{code} 中以\cmd{currentlist} 表示。如果指定一种条目类型(\prm{entrytype}),那么格式是该类型专属的。\prm{entrytype} 可以是一个逗号分隔(comma"=separated)的值列表。该命令类似于\cmd{DeclareListFormat},差别在于\prm{code} 处理的数据不是用于打印而是用于索引。注意\cmd{indexlist} 将执行\prm{code} 本身,即\prm{code} 必须包含\cmd{index} 或类似命令。带星的命令类似于不带星的命令,区别在于它还将清除所有对具体条目类型做的格式定义。
%Defines the literal list format \prm{format}. This formatting directive is arbitrary \prm{code} to be executed for every item in a list processed by \cmd{indexlist}. The current item will be passed to the \prm{code} as its only argument. The name of the literal list currently being processed is available to the \prm{code} as \cmd{currentlist}. If an \prm{entrytype} is specified, the format is specific to that type. The \prm{entrytype} argument may be a comma"=separated list of values. This command is similar to \cmd{DeclareListFormat} except that the data handled by the \prm{code} is not intended to be printed but written to the index. Note that \cmd{indexlist} will execute the \prm{code} as is, \ie the \prm{code} must include \cmd{index} or a similar command. The starred variant of this command is similar to the regular version, except that all type-specific formats are cleared.

\cmditem{DeclareIndexNameFormat}[entrytype, \dots]{format}{code}
\cmditem*{DeclareIndexNameFormat}*{format}{code}

定义姓名列表格式\prm{format}。该格式化指令是由\cmd{indexnames} 命令执行的任意\prm{code}。列表中当前值作为唯一参数传递给\prm{code}。正在处理的列表名在\prm{code} 中以\cmd{currentname} 表示。如果指定一种条目类型(\prm{entrytype}),那么格式是该类型专属的。\prm{entrytype} 可以是一个逗号分隔(comma"=separated)的值列表。该命令类似于\cmd{DeclareNameFormat},差别在于\prm{code} 处理的数据不是用于打印而是用于索引。注意\cmd{indexnames} 将执行\prm{code} 本身,即\prm{code} 必须包含\cmd{index} 或类似命令。带星的命令类似于不带星的命令,区别在于它还将清除所有对具体条目类型做的格式定义。
%Defines the name list format \prm{format}. This formatting directive is arbitrary \prm{code} to be executed for every name in a list processed by \cmd{indexnames}. The name of the name list currently being processed is available to the \prm{code} as \cmd{currentname}. If an \prm{entrytype} is specified, the format is specific to that type. The \prm{entrytype} argument may be a comma"=separated list of values. The parts of the name will be passed to the \prm{code} as separate arguments. This command is very similar to \cmd{DeclareNameFormat} except that the data handled by the \prm{code} is not intended to be printed but written to the index. Note that \cmd{indexnames} will execute the \prm{code} as is, \ie the \prm{code} must include \cmd{index} or a similar command. The starred variant of this command is similar to the regular version, except that all type-specific formats are cleared.

\cmditem{DeclareIndexListWrapperFormat}[entrytype, \dots]{format}{code}
\cmditem*{DeclareIndexListWrapperFormat}*{format}{code}

%Similar to \cmd{DeclareIndexListFormat} but for the list format used for indices.
类似于\cmd{DeclareIndexListFormat},但仅用于索引中。


\cmditem{DeclareIndexNameWrapperFormat}[entrytype, \dots]{format}{code}
\cmditem*{DeclareIndexNameWrapperFormat}*{format}{code}

%Similar to \cmd{DeclareIndexNameFormat} but for the name list format used for indices.
类似于\cmd{DeclareIndexNameFormat},但仅用于索引中。

\cmditem{DeclareFieldAlias}[entry type]{alias}[format entry type]{format}

声明\prm{alias} 作为域格式\prm{format} 的别名。如果指定一种条目类型(\prm{entrytype}),别名是该类型专属的。\prm{format entry type} 是后端格式的条目类型。这仅在声明某一具体条目类型的格式化指令的别名时需要。
%Declares \prm{alias} to be an alias for the field format \prm{format}. If an \prm{entrytype} is specified, the alias is specific to that type. The \prm{format entry type} is the entry type of the backend format. This is only required when declaring an alias for a type"=specific formatting directive.

\cmditem{DeclareListAlias}[entry type]{alias}[format entry type]{format}

声明\prm{alias} 作为抄录文本列表格式\prm{format} 的别名。如果指定一种条目类型(\prm{entrytype}),别名是该类型专属的。\prm{format entry type} 是后端格式的条目类型。这仅在声明某一具体条目类型的格式化指令的别名时需要。
%Declares \prm{alias} to be an alias for the literal list format \prm{format}. If an \prm{entrytype} is specified, the alias is specific to that type. The \prm{format entry type} is the entry type of the backend format. This is only required when declaring an alias for a type"=specific formatting directive.

\cmditem{DeclareNameAlias}[entry type]{alias}[format entry type]{format}

声明\prm{alias} 作为姓名列表格式\prm{format} 的别名。如果指定一种条目类型(\prm{entrytype}),别名是该类型专属的。\prm{format entry type} 是后端格式的条目类型。这仅在声明某一具体条目类型的格式化指令的别名时需要。
%Declares \prm{alias} to be an alias for the name list format \prm{format}. If an \prm{entrytype} is specified, the alias is specific to that type. The \prm{format entry type} is the entry type of the backend format. This is only required when declaring an alias for a type"=specific formatting directive.

\cmditem{DeclareListWrapperAlias}[entry type]{alias}[format entry type]{format}

%Declares \prm{alias} to be an alias for the outer list format \prm{format}. If an \prm{entrytype} is specified, the alias is specific to that type. The \prm{format entry type} is the entry type of the backend format. This is only required when declaring an alias for a type"=specific formatting directive.
声明一个列表封套格式\prm{format}的别名。如果指定了\prm{entrytype},该别名仅对该类型有效。
\prm{format entry type}是基础格式的类型,这类别名通常仅在声明指定类型的格式化指令的别名时需要。


\cmditem{DeclareNameWrapperAlias}[entry type]{alias}[format entry type]{format}

%Declares \prm{alias} to be an alias for the outer name list format \prm{format}. If an \prm{entrytype} is specified, the alias is specific to that type. The \prm{format entry type} is the entry type of the backend format. This is only required when declaring an alias for a type"=specific formatting directive.
声明一个列表封套格式\prm{format}的别名。如果指定了\prm{entrytype},该别名仅对该类型有效。
\prm{format entry type}是基础格式的类型,这类别名通常仅在声明指定类型的格式化指令的别名时需要。


\cmditem{DeclareIndexFieldAlias}[entry type]{alias}[format entry type]{format}

声明\prm{alias} 作为域格式\prm{format} 的别名。如果指定一种条目类型(\prm{entrytype}),别名是该类型专属的。\prm{format entry type} 是后端格式的条目类型。这仅在声明某一具体条目类型的格式化指令的别名时需要。
%Declares \prm{alias} to be an alias for the field format \prm{format}. If an \prm{entrytype} is specified, the alias is specific to that type. The \prm{format entry type} is the entry type of the backend format. This is only required when declaring an alias for a type"=specific formatting directive.

\cmditem{DeclareIndexListAlias}[entry type]{alias}[format entry type]{format}

声明\prm{alias} 作为抄录文本列表格式\prm{format} 的别名。如果指定一种条目类型(\prm{entrytype}),别名是该类型专属的。\prm{format entry type} 是后端格式的条目类型。这仅在声明某一具体条目类型的格式化指令的别名时需要。
%Declares \prm{alias} to be an alias for the literal list format \prm{format}. If an \prm{entrytype} is specified, the alias is specific to that type. The \prm{format entry type} is the entry type of the backend format. This is only required when declaring an alias for a type"=specific formatting directive.

\cmditem{DeclareIndexNameAlias}[entry type]{alias}[format entry type]{format}

声明\prm{alias} 作为姓名列表格式\prm{format} 的别名。如果指定一种条目类型(\prm{entrytype}),别名是该类型专属的。\prm{format entry type} 是后端格式的条目类型。这仅在声明某一具体条目类型的格式化指令的别名时需要。
%Declares \prm{alias} to be an alias for the name list format \prm{format}. If an \prm{entrytype} is specified, the alias is specific to that type. The \prm{format entry type} is the entry type of the backend format. This is only required when declaring an alias for a type"=specific formatting directive.


\cmditem{DeclareIndexListWrapperAlias}[entrytype, \dots]{format}{code}

%Similar to \cmd{DeclareIndexListFormat} but for the list format used for indices.
类似于\cmd{DeclareIndexListFormat},但仅用于索引中。

\cmditem{DeclareIndexNameWrapperAlias}[entrytype, \dots]{format}{code}

%Similar to \cmd{DeclareIndexNameFormat} but for the name list format used for indices.
类似于\cmd{DeclareIndexNameFormat},但仅用于索引中。

\cmditem{DeprecateFieldFormatWithReplacement}[entry type]{alias}[format entry type]{format}

%Declares \prm{alias} to be an alias for the name list format \prm{format} and issue a deprecation warning. If an \prm{entrytype} is specified, the alias is specific to that type. The \prm{format entry type} is the entry type of the backend format. This is only required when declaring an alias for a type"=specific formatting directive.
声明姓名列表格式\prm{format}的别名,并给出一个表示不推荐的警告。
如果指定了\prm{entrytype},该别名仅对该类型有效。
\prm{format entry type}是基础格式的类型,这类别名通常仅在声明指定类型的格式化指令的别名时需要。



\cmditem{DeprecateListFormatWithReplacement}[entry type]{alias}[format entry type]{format}

%Similar to \cmd{DeprecateFieldFormatWithReplacement} but for list formats.
类似于\cmd{DeprecateFieldFormatWithReplacement},但用于列表。

\cmditem{DeprecateNameFormatWithReplacement}[entry type]{alias}[format entry type]{format}

%Similar to \cmd{DeprecateFieldFormatWithReplacement} but for name formats.
类似于\cmd{DeprecateFieldFormatWithReplacement},但用于姓名列表。

\cmditem{DeprecateListWrapperFormatWithReplacement}[entry type]{alias}[format entry type]{format}

%Similar to \cmd{DeprecateFieldFormatWithReplacement} but for outer list formats.
类似于\cmd{DeprecateFieldFormatWithReplacement},但用于列表封套。

\cmditem{DeprecateNameWrapperFormatWithReplacement}[entry type]{alias}[format entry type]{format}

%Similar to \cmd{DeprecateFieldFormatWithReplacement} but for outer name formats.
类似于\cmd{DeprecateFieldFormatWithReplacement},但用于姓名封套。

\cmditem{DeprecateIndexFieldFormatWithReplacement}[entry type]{alias}[format entry type]{format}

%Similar to \cmd{DeprecateFieldFormatWithReplacement} but for index field formats.
类似于\cmd{DeprecateFieldFormatWithReplacement},但用于索引中的域。

\cmditem{DeprecateIndexListFormatWithReplacement}[entry type]{alias}[format entry type]{format}

%Similar to \cmd{DeprecateFieldFormatWithReplacement} but for index list formats.
类似于\cmd{DeprecateFieldFormatWithReplacement},但用于索引中的列表。

\cmditem{DeprecateIndexNameFormatWithReplacement}[entry type]{alias}[format entry type]{format}

%Similar to \cmd{DeprecateFieldFormatWithReplacement} but for index name formats.
类似于\cmd{DeprecateFieldFormatWithReplacement},但用于索引中的姓名列表。

\cmditem{DeprecateIndexListWrapperFormatWithReplacement}[entry type]{alias}[format entry type]{format}

%Similar to \cmd{DeprecateFieldFormatWithReplacement} but for index list formats.
类似于\cmd{DeprecateFieldFormatWithReplacement},但用于索引中的列表封套。

\cmditem{DeprecateIndexNameWrapperFormatWithReplacement}[entry type]{alias}[format entry type]{format}

%Similar to \cmd{DeprecateFieldFormatWithReplacement} but for index name formats.
类似于\cmd{DeprecateFieldFormatWithReplacement},但用于索引中的姓名列表封套。

\end{ltxsyntax}


\subsection{定制}%Customization
\label{aut:ctm}

\subsubsection{关联条目}%Related Entries
\label{aut:ctm:rel}
%The related entries feature comprises the following components:
关联条目相关功能由如下部分构成:
\begin{itemize}
\item 条目中的特殊域用于建立和描述关系
%\item Special fields in an entry to set up and describe relationships
\item 本地化字符串作为关联数据的前缀(可选)
%\item Optionally, localisation strings to prefix the related data
\item 抽取和打印关联数据的宏
%\item Macros to extract and print the related data
\item 用于本地化字符串和关联数据格式化的格式
%\item Formats to format the localisation string and related data
\end{itemize}
%
特殊域是\bibfield{related}, \bibfield{relatedtype}, \bibfield{relatedstring} 和 \bibfield{relatedoptions}:
%The special fields are \bibfield{related}, \bibfield{relatedtype}, \bibfield{relatedstring} and \bibfield{relatedoptions}:
\begin{keymarglist}
\item[related] 与当前条目存在某种程度关联性的条目关键词列表\footnote{译者: 这里separated list 译为分离列表,列表}。注意: 条目关键词\footnote{译者: 这里的key关键词就是是条目关键词,引用关键词,即bibtex 键} 的顺序很重要。来自多个关联条目的数据是按该域中关键词的顺序打印的。
%\item[related] A separated list of keys of entries which are related to this entry in some way. Note the the order of the keys is important. The data from multiple related entries is printed in the order of the keys listed in this field.
\item[relatedtype] 关联类型。主要用于三个目的: 第一,如果该域的值解析为一个本地化字符串的标识,那么得到的本地化字符串将在来自关联条目的数据之前打印。第二,如果存在名为\texttt{related:\prm{relatedtype}} 的宏,它将用于格式化来自关联条目的数据,如果宏不存在,则使用\texttt{related:default} 宏。最后,如果存在名为\texttt{related:\prm{relatedtype}} 的格式,它将用来格式化本地化字符串和关联条目数据。如果没有具体类型的格式,那么使用\texttt{related} 格式。
%The type of relationship. This serves three purposes. If the value of this field resolves to a localisation string identifier, then the resulting localised string is printed before the data from the related entries. Secondly, if there is a macro called \texttt{related:\prm{relatedtype}}, this is used to format the data from the related entries. If no such macro exists, then the macro \texttt{related:default} is used. Lastly, if there is a format named \texttt{related:\prm{relatedtype}}, then it is used to format both the localised string and related entry data. If there is no related type specific format, the \texttt{related} format is used.
\item[relatedstring] 如果一个条目包含该域,如果该域的值解析为一个本地化字符串的标识,那么本地化字符串的键值将在来自关联条目的数据之前打印。如果该域没有指定一个本地化键,则原样打印该域的值。如果\bibfield{relatedtype} 和 \bibfield{relatedstring} 都出现在条目中,\bibfield{relatedstring} 用于数据之前的字符串(但\bibfield{relatedtype} 仍然用于确定打印数据时的格式和宏)。
%If an entry contains this field, then if value of the field resolves to a localisation string identifier, the localisation key value specified is printed before data from the related entries. If the field does not specify a localisation key, its value is printed literally. If both \bibfield{relatedtype} and \bibfield{relatedstring} are present in an entry, \bibfield{relatedstring} is used for the pre-data string (but \bibfield{relatedtype} is still used to determine the macro and format to use when printing the data).
\item[relatedoptions] 设置在关联条目上的各条目的选项列表(实际上,是对关联条目的副本的设置,关联条目的副本作为数据源,而关联条目本身不做任何的改变,因为它自身有可能被引用)。
%A list of per"=entry options to set on the related entry (actually on the clone of the related entry which is used as a data source---the actual related entry is not modified because it might be cited directly itself).
\end{keymarglist}

关联条目功能由\secref{use:opt:pre:gen} 节的\opt{related} 包选项默认启用。来自关联条目的相关信息数据通过一个调用\cmd{usebibmacro\{related\}} 包含进来。标准样式调用该宏直到每个驱动结束。样式作者应该确保(留意)作为\bibfield{relatedtype} 域值的本地化字符串的存在,比如\texttt{translationof} 或者可能的\texttt{translatedas}。本地化键(关键词)的\prm{relatedtype}\texttt{s} 复数形式可以识别。在\bibfield{related} 中给出的超过1个的键的对应字符串都会打印。用于打印由\prm{relatedtype} 关联的条目的参考文献宏和格式化指令应以\texttt{related:\prm{relatedtype}} 为名进行定义。\path{biblatex.def} 包含了通用的关联类型的宏和格式,可以作为模板。特别的,\cmd{entrydata*} 命令在这些宏中是必须的,因为要获取关联条目的数据。应用了该功能的条目数据可以在\biblatex 发布的示例文件\path{biblatex-examples.bib} 中找到。针对该功能的一些用于控制关联条目间的分隔符的具体格式化宏见\secref{aut:fmt:fmt}。
%The related entry feature is enabled by default by the package option \opt{related} from \secref{use:opt:pre:gen}. The related information entry data from the related entries is included via a \cmd{usebibmacro\{related\}} call. Standard styles call this macro towards the end of each driver. Style authors should ensure the existence of (or take note of existing) localisation strings which are useful as values for the \bibfield{relatedtype} field, such as \texttt{translationof} or perhaps \texttt{translatedas}. A plural variant can be identified with the localisation key \prm{relatedtype}\texttt{s}. This key's corresponding string is printed whenever more than one entry is specified in \bibfield{related}. Bibliography macros and formatting directives for printing entries related by \prm{relatedtype} should be defined using the name \texttt{related:\prm{relatedtype}}. The file \path{biblatex.def} contains macros and formats for some common relation types which can be used as templates. In particular, the \cmd{entrydata*} command is essential in such macros in order to make the data of the related entries available. Examples of entries using this feature can be found in the \biblatex distribution examples file \path{biblatex-examples.bib}. There are some specific formatting macros for this feature which control delimiters and separators in related entry information, see \secref{aut:fmt:fmt}.

\subsubsection{数据源的域集合}%Datasource Sets
\label{aut:ctm:dsets}

能给数据源中域的集合命名,在循环等操作中是很有用的。\biber 可以利用这些集合名来对某些特定数据源的域集合应用某些选项或者执行某些操作。下面的宏允许用来定义任意的数据源的域集合,这些集合中的域在\biblatex 中以\sty{etoolbox} 列表表示,并通过\file{.bcf} 文件传递给\biber 。
%It is useful to be able to define named sets of datasource field names for use in loops etc. In addition, \biber can use such sets in order to apply options and perform operations on particular sets of datasource fields. The following macros allow the user to define arbitrary sets of datasource fields, exposed to \biblatex as \sty{etoolbox} lists and to \biber in the \file{.bcf}.


\begin{ltxsyntax}

\cmditem{DeclareDatafieldSet}{name}{specification}

声明一个数据源的域集合,集合名为\prm{name}。
%Declare a set of datasource fields with name \prm{name}.

\begin{optionlist*}
\valitem{name}{set name}

集的名。
%The name of the set.
\end{optionlist*}

\prm{specification} 是一个或更多的\cmd{member}(成员)项:
%The \prm{specification} is one or more \cmd{member} items:

\cmditem{member}

\begin{optionlist*}
\valitem{fieldtype}{fieldtype}
\valitem{datatype}{datatype}
\valitem{field}{fieldname}
\end{optionlist*}

一个\cmd{member} 说明将域添加到集合中。域可以由数据模型\prm{fieldtype} 和/或 \prm{datatype} 指定 (见 \secref{aut:ctm:dm})。 或者,域也可以通过使用\prm{field} 选项显式地以域名添加。一旦完成定义,集就以\sty{etoolbox} 列表的形式存在,命名为\cmd{datafieldset<setname>} 并通过\file{.bcf} 文件传递给\biber。
%A \cmd{member} specification appends fields to the set. Fields can be specified by datamodel \prm{fieldtype} and/or \prm{datatype} (see \secref{aut:ctm:dm}). Alternatively, fields can be explicitly added by name using the \prm{field} option. Once defined, the set is available as an \sty{etoolbox} list called \cmd{datafieldset<setname>} and is also passed via the \file{.bcf} to \biber.

如下示例就是\biblatex 为姓名域和标题域定义的默认集:
%For example, here are the default sets defined by \biblatex for name fields and title fields:

\end{ltxsyntax}

\begin{ltxexample}[style=latex]{}
\DeclareDatafieldSet{setnames}{
  \member[datatype=name, fieldtype=list]
}

\DeclareDatafieldSet{settitles}{
  \member[field=title]
  \member[field=booktitle]
  \member[field=eventtitle]
  \member[field=issuetitle]
  \member[field=journaltitle]
  \member[field=maintitle]
  \member[field=origtitle]
}
\end{ltxexample}
%
这将以\sty{etoolbox} 列表形式定义\cmd{datafieldsetsetnames} 和\cmd{datafieldsetsettitles} 宏,用来包含指定数据源的域成员的名称。
%\footnote{译者: 应用时要注意集的定义名称和使用的名称是不同的。}
%This defines the macros \cmd{datafieldsetsetnames} and \cmd{datafieldsetsettitles} as \sty{etoolbox} lists containing the names of the member datasource fields specified.

\subsubsection{数据动态修改}%Dynamic Modification of Data
\label{aut:ctm:map}

对自动生成或者无法控制的参考文献数据源进行修改在某种程度上会是一个问题。因此,\biber 提供了对它所读取的数据进行修改的能力,这样你可以对源数据流进行修改而不必实际改变它。这种改变可以在\biber 的配置文件(见\biber 文档)中定义,或者通过\biblatex 宏进行定义,通过宏定义的方法你可以在样式中或者以全局定义的方式,将修改应用在具体的文档中。
%Bibliographic data sources which are automatically generated or which you have no control over can be a problem if you need to edit them in some way. For this reason, \biber has the ability to modify data as it is read so that you can apply modifications to the source data stream without actually changing it. The modification can be defined in \biber's config file (see \biber docs), or via \biblatex macros in which case you can apply the modification only for specific documents, styles or globally.

源映射发生在数据解析过程中,因此也在诸如继承和排序等任何其它操作之前。
%Source mapping happens during data parsing and therefore before any other operation such as inheritance and sorting.

源映射可以在不同的层( «levels» )进行定义,各层以某一定义的顺序进行处理。见\biblatex\ 手册,考虑如下这些宏:\\[2ex]
%Source mappings can be defined at different «levels» which are applied
%in a defined order. See the \biblatex\ manual regarding these macros:\\[2ex]

\noindent \cmd{DeclareSourcemap} 命令定义的用户层(\texttt{user}-level)映射$\rightarrow$\\
\hspace*{1em} 在\biber 配置文件定义的用户层(\texttt{user}-level)映射(见 \biber 文档)$\rightarrow$\\
\hspace*{2em}\cmd{DeclareStyleSourcemap} 定义的样式层(\texttt{style}-level)映射$\rightarrow$\\
\hspace*{3em}\cmd{DeclareDriverSourcemap} 定义的驱动层(\texttt{driver}-level)映射\\[2ex]
%\noindent \texttt{user}-level maps defined with \cmd{DeclareSourcemap}$\rightarrow$\\
%\hspace*{1em}\texttt{user}-level maps defined in the \biber config file (see \biber docs)$\rightarrow$\\
%\hspace*{2em}\texttt{style}-level maps defined with \cmd{DeclareStyleSourcemap}$\rightarrow$\\
%\hspace*{3em}\texttt{driver}-level maps defined with \cmd{DeclareDriverSourcemap}\\[2ex]
\begin{ltxsyntax}

\cmditem{DeclareSourcemap}{specification}

定义源数据修改(映射)规则,可以用于执行如下任务或其任意组合:
%Defines source data modification (mapping) rules which can be used to perform any combination of the following tasks:

\begin{itemize}
\item 将数据源条目类型映射为其它类型
%Map data source entrytypes to different entrytypes
\item 将数据源域映射为其它域
%Map datasource fields to different fields
\item 给条目添加新域
%Add new fields to an entry
\item 从条目移除域
%Remove fields from an entry
\item 用标准的Perl 正则表达式匹配和替换,修改域的内容。
%Modify the contents of a field using standard Perl regular expression match and replace
\item 将上述操作限制在来自特定数据源的条目,这些特定数据源在\cmd{addresource} 宏中定义。
%Restrict any of the above operations to entries coming from
%particular datasources which you defined in \cmd{addresource} macros

\item 将上述操作限制在某些条目类型。
%Restrict any of the above operations to entries only of a certain
%entrytype

\item 将上述操作限制在某一特定的参考文献节。
%Restrict any of the above operations to entries in a particular
%reference section
\end{itemize}

\prm{specification} 是一个不限数量的\cmd{maps} 指令的列表,这些指令说明了应用于某一特定数据源类型的映射规则的容器(\secref{use:bib:res})。为实现良好的代码显示效果,可以自由使用空格、制表符、行末符号来整理\prm{specification} 中的代码。但空行是不允许的。这一命令仅能用于导言区并且只能使用一次---后面的命令将覆盖前面的定义。
%The \prm{specification} is an undelimited list of \cmd{maps} directives which specify containers for mappings rules applying to a particular data source type (\secref{use:bib:res}). Spaces, tabs, and line endings may be used freely to visually arrange the \prm{specification}. Blank lines are not permissible. This command may only be used in the preamble and may only be used once---subsequent uses will overwrite earlier definitions.

\cmditem{maps}[options]{elements}

包含\cmd{map} 元素的有序集,每个\cmd{map} 都是应用于数据源的映射步的逻辑相关集。\prm{options} 包括:
%Contains an ordered set of \cmd{map} elements each of which is a logically related set of mapping steps to apply to the data source. The \prm{options} are:
\begin{optionlist*}

\choitem[bibtex]{datatype}{bibtex, biblatexml}

包含的\cmd{map} 应用的数据源的类型(见\secref{use:bib:res})
%Data source type to which the contained \cmd{map} directives apply (\secref{use:bib:res}).

\boolitem[false]{overwrite}

具体说明一个映射规则是否允许覆盖条目中已经存在数据。如果该选项未指定,默认是\texttt{false}。简易形式\opt{overwrite} 等价于\kvopt{overwrite}{true}。
%Specify whether a mapping rule is allowed to overwrite already existing data in an entry. If this option is not specified, the default is \texttt{false}. The short form \opt{overwrite} is equivalent to \kvopt{overwrite}{true}.

\end{optionlist*}

\cmditem{map}[options]{restrictions,steps}

一个包含有序的映射操作\cmd{step}s的容器,可以限制在特定的条目类型或者数据源上。这是一个编组元素允许一组映射操作应用到具体的条目类型或数据源上。映射操作必须包含在\cmd{map} 元素内。\prm{options} 包括:
%A container for an ordered set of map \cmd{step}s, optionally restricted to particular entrytypes or data sources. This is a grouping element to allow a set of mapping steps to apply only to specific entrytypes or data sources. Mapping steps must always be contained within a \cmd{map} element. The \prm{options} are:

\begin{optionlist*}

\boolitem{overwrite}

与父元素\cmd{maps} 的相同的选项。该选项允许map层级的覆盖操作。如果该选项未指定,默认是父元素\cmd{maps} 的选项值。简易形式\opt{overwrite} 等价于\kvopt{overwrite}{true}。
%As the same option on the parent \cmd{maps} element. This option allows an override on a per-map group basis. If this option is not specified, the default is the parent \cmd{maps} element option value. The short form \opt{overwrite} is equivalent to \kvopt{overwrite}{true}.

\valitem{foreach}{loopval}

遍历\cmd{map} 内的\cmd{step}s,将包含在\prm{loopval} 中的逗号分隔的各个值设置到特殊变量|$MAPLOOP|中。%$
\prm{loopval} 可以是一个显式的逗号分隔的值列表,也可以是由\cmd{DeclareDatafieldSet} (见\secref{aut:ctm:dsets})定义的任意数据域集合的集名,biber 从中获取数据域并解析为一个逗号分隔的值列表。\prm{loopval} 以列表中的顺序处理。这使得一组\cmd{step}s操作可以遍历\prm{loopval} 的每个值。使用regexp 映射,可以创建一个CSV 域来配合该功能使用。特殊变量|$MAPUNIQ| %$
也可以在\cmd{step}s中用来随机生成一个唯一的字符串。这可以用于创建新条目的关键词。例如:

%Loop over all \cmd{step}s in this \cmd{map}, setting the special variable |$MAPLOOP| %$
%to each of the comma-separated values contained in \prm{loopval}. \prm{loopval} can either be the name of a datafield set defined with \cmd{DeclareDatafieldSet} (see \secref{aut:ctm:dsets}), a datasource field which is fetched and parsed as a comma"=separated values list or an explicit comma"=separated values list. \prm{loopval} is determined in this order. This allows the user to repeat a group of \cmd{step}s for each value \prm{loopval}. Using regexp maps, it is possible to create a CSV field for use with this functionality. The special variable |$MAPUNIQ| %$
%may also be used in the \cmd{step}s to generate a random unique string. This can be useful when creating keys for new entries. An example:

\begin{ltxexample}[style=latex]{}
\DeclareSourcemap{
  \maps[datatype=bibtex]{
    \map[overwrite, foreach={author,editor, translator}]{
      \step[fieldsource=\regexp{$MAPLOOP}, match={Smith}, replace={Jones}]
    }%数据源域将遍历foreach定义的这些域,而这些域由宏\regexp{$MAPLOOP} 表示
  }
}
\end{ltxexample}
%$<- to stop emacs highlighting breaking

\intitem{refsection}

将包含的\cmd{step} 命令应用于序号等于\prm{refsection} 的参考文献节的条目上。
%Only apply the contained \cmd{step} commands to entries in the reference section with number \prm{refsection}.

\end{optionlist*}

\cmditem{perdatasource}{datasource}

将\cmd{map} 元素内所有的\cmd{step}s限制在来自名为\prm{datasource} 的条目上。\prm{datasource} 名应是在\cmd{addresource} 宏给出的文档的参考文献数据源文件名。一个\cmd{map} 元素内允许出现多个\cmd{perdatasource} 约束。
%Restricts all \cmd{step}s in this \cmd{map} element to entries from the named \prm{datasource}. The \prm{datasource} name should be exactly as given in a \cmd{addresource} macro defining a data source for the document. Multiple \cmd{perdatasource} restrictions are allowed within a \cmd{map} element.

\cmditem{pertype}{entrytype}

将\cmd{map} 元素内所有的\cmd{step}s限制在来自类型为\prm{entrytype} 的条目上。一个\cmd{map} 元素内允许出现多个\cmd{pertype} 约束。
%Restricts all \cmd{step}s in this \cmd{map} element to entries of the named \prm{entrytype}. Multiple \cmd{pertype} restrictions are allowed within a \cmd{map} element.

\cmditem{pernottype}{entrytype}

将\cmd{map} 元素内所有的\cmd{step}s限制在来自类型不是\prm{entrytype} 的条目上。一个\cmd{map} 元素内允许出现多个\cmd{pernottype} 约束。
%Restricts all \cmd{step}s in this \cmd{map} element to entries not of the named \prm{entrytype}. Multiple \cmd{pernottype} restrictions are allowed within a \cmd{map} element.

\cmditem{step}[options]

一个映射步。每一步都顺序地应用于每个相关条目上,其中“相关”意为这些条目满足前述指定的数据源类型,条目类型,数据源文件限制。对条目应用每一映射步都在前面映射步完成之后。映射步执行的映射操作由如下选项(\prm{option}s)确定:
%A mapping step. Each step is applied sequentially to every relevant entry where <relevant> means those entries which correspond to the data source type, entrytype and data source name restrictions mentioned above. Each step is applied to the entry as it appears after the application of all previous steps. The mapping performed by the step is determined by the following \prm{option}s:

\begin{optionlist*}

\valitem{typesource}{entrytype} %源的条目类型
\valitem{typetarget}{entrytype}
\valitem{fieldsource}{entryfield}
\valitem{notfield}{entryfield}
\valitem{fieldtarget}{entryfield}
\valitem{match}{regexp}
\valitem{matchi}{regexp}
\valitem{notmatch}{regexp}
\valitem{notmatchi}{regexp}
\valitem{replace}{regexp}
\valitem{fieldset}{entryfield}
\valitem{fieldvalue}{string}
\valitem{entryclone}{clonekey}
\valitem{entrynew}{entrynewkey}
\valitem{entrynewtype}{string}
\valitem{entrytarget}{string}
\boolitem[false]{entrynocite}
\boolitem[false]{entrynull}
\boolitem[false]{append}
\boolitem[false]{final}
\boolitem[false]{null}
\boolitem[false]{origfield}
\boolitem[false]{origfieldval}
\boolitem[false]{origentrytype}
%
对于所有的布尔映射选项,简易形式\opt{option} 等价于\kvopt{option}{true}。一个映射步的应用规则如下:
%For all boolean \cmd{step} options, the short form \opt{option} is equivalent to \kvopt{option}{true}. The following rules for a mapping step apply:

\renewcommand{\labelitemii}{$\circ$}

\begin{itemize}
\item 如果设置\texttt{entrynew},将创建一个条目关键词为\texttt{entrynewkey} 的新条目,条目的类型在选项\texttt{entrynewtype} 中给出。这一条目仅在当前条目的处理过程中有效,能以\texttt{entrytarget} 进行引用。在\texttt{entrynewkey} 中,可以使用标准的Perl正则表达式来引用从之前的\texttt{match} 步获取的匹配字符串。
%If \texttt{entrynew} is set, a new entry is created with the entry key \texttt{entrynewkey} and the entry type given in the option \texttt{entrynewtype}. This
%entry is only in-scope during the processing of the current entry and can be referenced by
%\texttt{entrytarget}.  In \texttt{entrynewkey}, you may use standard Perl regular expression
%backreferences to captures from a  previous \texttt{match} step.

\item 当一个\texttt{fieldset} 步设置的\texttt{entrytarget} 为一个由\texttt{entrynew} 创建的条目的关键词,域设置的目标将是\texttt{entrytarget} 条目而不是当前正在处理的条目。这使得用户可以创建新的条目并且设置它的域。
%When a \texttt{fieldset} step has \texttt{entrytarget} set to the entrykey of an entry
%created by \texttt{entrynew}, the target for the field set will be the \texttt{entrytarget} entry
%rather than the entry being currently processed. This allows users to create new entries and set
%fields in them.

\item %If \texttt{entrynocite} is used in a \texttt{entrynew} or
  %\texttt{entryclone} step, the new/clone entry will be included in the
  %\file{.bbl} as if the entry/clone had been \cmd{nocite}ed in the document.
如果\texttt{entrynocite}在一个\texttt{entrynew}或\texttt{entryclone}步中使用,
该新/克隆条目将会出现在\file{.bbl}中就好像在正文中对该条目使用了\cmd{nocite}。


\item 如果设置\texttt{entrynull},\cmd{map} 过程立即终止,当前条目不创建。如同该条目不存在于数据源。显然,需要利用前面的映射步来选择需要应用该操作的条目。
%If \texttt{entrynull} is set, processing of the \cmd{map}
%  immediately terminates and the current entry is not created. It is
%  as if it did not exist in the datasource. Obviously, you should
%  select the entries which you want to apply this to using prior
%  mapping steps.

\item 如果设置\texttt{entryclone},将创建一个名为\texttt{clonekey} 的副本条目。显然这会影响作者年等样式中的标签生成,使用需谨慎。副本仅在当前条目的处理过程中有效,将其关键词作为\texttt{entrytarget} 的值可对其进行修改。在\texttt{clonekey} 中,可以使用标准的Perl正则表达式来引用从之前的\texttt{match} 步获取的匹配字符串。
%If \texttt{entryclone} is set, a clone of the entry is created with an entry key
%  \texttt{clonekey}. Obviously this may cause labelling problems in author/year styles etc.
%  and should be used with care. The cloned entry is in-scope during the processing of the
%  current entry and can be modified by passing its key as the value to \texttt{entrytarget}.
%  In \texttt{clonekey}, you may use standard Perl regular expression backreferences to
%  captures from a previous \texttt{match} step.

\item 将\texttt{typesource} \prm{entrytype} 修改到\texttt{typetarget} \prm{entrytype},如果定义存在的话。如果\texttt{final} 为\texttt{true},且条目的\prm{entrytype} 不是\texttt{typesource},父元素\cmd{map} 将立即终止。
%Change the \texttt{typesource} \prm{entrytype} to the
%  \texttt{typetarget} \prm{entrytype}, if defined. If
%  \texttt{final} is \texttt{true} then if the \prm{entrytype} of the entry is not \texttt{typesource}, processing of the parent \cmd{map} immediately terminates.

\item 将\texttt{fieldsource} \prm{entryfield} 修改到\texttt{fieldtarget},如果定义存在的话。如果\texttt{final} 为\texttt{true},且条目中不存在\texttt{fieldsource} \prm{entryfield},父元素\cmd{map} 将立即终止。
%Change the \texttt{fieldsource} \prm{entryfield} to
%  \texttt{fieldtarget}, if defined. If
%  \texttt{final} is \texttt{true} then if there is no \texttt{fieldsource} \prm{entryfield} in the entry, processing of the parent \cmd{map} immediately terminates.

\item 如\prm{entryfield}存在,那么\texttt{notfield}就为true,常与\texttt{final}联用,当条目不包含\prm{entryfield}则映射终止。
%If \texttt{notfield} is true only if the \prm{entryfield} does not
%  exist. Usually used with \texttt{final} so that if an entry does contain
%  \prm{entryfield}, the map terminates.

\item 如果定义了\texttt{match} 但没有定义\texttt{replace},仅当\texttt{fieldsource} \prm{entryfield} 匹配\texttt{match} 正则表达式时
(如果使用\texttt{notmatch} 则逻辑相反,当使用以<i>结尾的版本时是大小写敏感的)
应用该映射步\footnote{正则表达式是Perl 5.16 全集正则表达式。这意味着需要处理特殊字符,见后面的示例}。可以使用圆括号获取匹配内容,并在后面\texttt{fieldvalue} 设置中引用(\$1\ldots\$9)。这使得可以将一些域的部分内容提取出来放入其它一些域中。
%If \texttt{match} is defined but
%  \texttt{replace} is not, only apply the step if the \texttt{fieldsource} \prm{entryfield} matches the
%  \texttt{match} regular expression (logic is reversed if you use \texttt{notmatch} and case-insensitive if you use the versions ending in <i>)\footnote{Regular expressions are full Perl 5.16 regular expressions. This means you may need to deal with special characters, see examples.}. You may use capture parenthesis as usual and refer to these (\$1\ldots\$9) in later \texttt{fieldvalue} specifications. This allows you to pull out parts of some fields and put these parts in other fields.

\item 如果定义了\texttt{match} 和\texttt{replace},对\texttt{fieldsource} \prm{entryfield} 的值执行正则表达式匹配和替换操作。
%Perform a regular expression match and replace on the value of the \texttt{fieldsource} \prm{entryfield} if \texttt{match} and \texttt{replace} are defined.

\item 如果定义了\texttt{fieldset},它的值\prm{entryfield} 将由进一步给出的选项来指定。如果\texttt{overwrite} 是false且该域值已经存在,则该映射步将忽略。如果该映射步的\texttt{final} 也是true,则父元素map在此处终止。如果\texttt{append} 是true,则设置的值将添加到当前\prm{entryfield} 的值中。值仅由如下之一的必选参数设置:
%If \texttt{fieldset} is defined, then its value is \prm{entryfield}
%  which will be set to a value specified by further options. If
%  \texttt{overwrite} is false for this step and the field to set already
%  exists then the map step is ignored. If \texttt{final} is also true for
%  this step, then processing of the parent map stops at this point. If
%  \texttt{append} is true, then the value to set is appended to the current
%  value of \prm{entryfield}. The value to set is specified by a mandatory
%  one and only one of the following options:

  \begin{itemize}
    \item\ \texttt{fieldvalue} --- \texttt{fieldset} \prm{entryfield} 设置为\texttt{fieldvalue} \prm{string}。
    %\texttt{fieldvalue} --- The \texttt{fieldset} \prm{entryfield} is set to the \texttt{fieldvalue} \prm{string}

    \item\ \texttt{null} --- The \texttt{fieldset} \prm{entryfield} 忽略,如果它不存在于数据源中。
    %\texttt{null} --- The \texttt{fieldset} \prm{entryfield} is ignored, as if it did not exist in the datasource

    \item\ \texttt{origentrytype} --- The \texttt{fieldset} \prm{entryfield} 设置为前面最近的\texttt{typesource} \prm{entrytype} 名称。
    %\texttt{origentrytype} --- The \texttt{fieldset} \prm{entryfield} is set to the most recently mentioned \texttt{typesource} \prm{entrytype} name

    \item\ \texttt{origfield} --- The \texttt{fieldset} \prm{entryfield} 设置为前面最近的\texttt{fieldsource} \prm{entryfield} 域名。
    %\texttt{origfield} --- The \texttt{fieldset} \prm{entryfield} is set to the most recently mentioned \texttt{fieldsource} \prm{entryfield} name

    \item\ \texttt{origfieldval} --- The \texttt{fieldset} \prm{entryfield} 设置为前面最近的\texttt{fieldsource} \prm{entryfield} 域值。
    %\texttt{origfieldval} --- The \texttt{fieldset} \prm{entryfield} is set to the most recently mentioned \texttt{fieldsource} value

  \end{itemize}
\end{itemize}
\end{optionlist*}
\end{ltxsyntax}

使用\bibtex\ 数据源,可能设置虚拟域\bibfield{entrykey} 作为\texttt{fieldsource},它就是条目的引用关键词(即bibtex键)。使用\biblatexml\ 数据源的话,\bibfield{entrykey} 是正常属性可以像任何其它属性一样引用。自然,该域<field>是不能(用修改操作\texttt{fieldset}, \texttt{fieldtarget},\texttt{replace})改变的。
%\noindent With \bibtex\ datasources, you may specify the
%pseudo-field \bibfield{entrykey} for \texttt{fieldsource}
%which is the citation key of the entry. With \biblatexml\ the \bibfield{entrykey} is a normal attribute and can be reference like any other attribute. Naturally, this <field> cannot
%be changed (used as \texttt{fieldset}, \texttt{fieldtarget} or changed using \texttt{replace}).

%Macros used in \cmd{step} are expanded. Unexpandable contents should be protected with \cmd{detokenize}, regular expressions can be escaped using the dedicated \cmd{regexp} command (see the examples below).

在\cmd{step}中使用的宏是会展开的,要求不展开的内容应该利用\cmd{detokenize}命令进行表或,
而正则表达式则可以利用下面的\cmd{regexp}命令来避开该问题。

\begin{ltxsyntax}

\cmditem{DeclareStyleSourcemap}{specification}

该命令设置由样式使用的数据源映射。这种映射概念上是与由\cmd{DeclareSourcemap} 定义的用户层映射分离的,并且在用户层映射之后应用。其语法与\cmd{DeclareSourcemap} 完全相同。该命令使得样式作者可以定义样式专属的映射而不会与用户层和驱动层的映射相冲突。这一命令可以在样式文件中多次使用,并根据定义的顺序进行映射操作。
%This command sets the source mappings used by a style. Such mappings are conceptually separate from user mappings defined with \cmd{DeclareSourcemap} and are applied directly after user maps. The syntax is identical to \cmd{DeclareSourcemap}. This command is provided for style authors so that any maps defined for the style do not interfere with user maps or the default driver maps defined with \cmd{DeclareDriverSourcemap}. This command is for use in style files and can be used multiple times, the maps being run in order of definition.

\end{ltxsyntax}

\begin{ltxsyntax}

\cmditem{DeclareDriverSourcemap}[datatype=driver]{specification}

该命令为具体的\prm{driver} 设置驱动默认的数据源映射。这种映射概念上是与由\cmd{DeclareSourcemap} 定义的用户层映射和\cmd{DeclareStyleSourcemap} 定义的样式层映射分离的。它们由一些映射组成,也作为驱动组成部分。用户一般不需要修改它。驱动默认映射在用户层和样式层映射之后应用。默认的设置在附录\secref{apx:maps} 介绍。\prm{specification} 与\cmd{DeclareSourcemap} 的相同,但没有\cmd{maps} 元素: \prm{specification} 是一些\cmd{map} 元素的列表,因为每个\cmd{DeclareDriverSourcemap} 仅应用于某一具体的条目类型的驱动中。默认的定义见附录\secref{apx:maps} 中的示例。
%This command sets the driver default source mappings for the specified \prm{driver}. Such mappings are conceptually separate from user mappings defined with \cmd{DeclareSourcemap} and style mapping defined with \cmd{DeclareStyleSourcemap}. They consist of mappings which are part of the driver setup. Users should not normally need to change these. Driver default mapping are applied after user mappings (\cmd{DeclareSourcemap}) and style mappings (\cmd{DeclareStyleSourcemap}). These defaults are described in Appendix \secref{apx:maps}. The \prm{specification} is identical to that for \cmd{DeclareSourcemap} but without the \cmd{maps} elements: the \prm{specification} is just a list of \cmd{map} elements since each \cmd{DeclareDriverSourcemap} only applies to one datatype driver. See the default definitions in Appendix \secref{apx:maps} for examples.

\end{ltxsyntax}

下面给出一些数据源映射的示例:
%Here are some data source mapping examples:

\begin{ltxexample}
\DeclareSourcemap{
  \maps[datatype=bibtex]{
    \map{
      \perdatasource{<<example1.bib>>}
      \perdatasource{<<example2.bib>>}
      \step[fieldset=<<keywords>>, fieldvalue={<<keyw1, keyw2>>}]
      \step[fieldsource=<<entrykey>>]
      \step[fieldset=<<note>>, origfieldval]
    }
  }
}
\end{ltxexample}
%
这一示例是对能在\texttt{examples1.bib} 和\texttt{examples2.bib} 文件中找到的条目进行处理,增加一个\bibfield{keywords} 域,域值为<keyw1, keyw2>,并且设置\bibfield{note} 域值为条目关键词。
%This would add a \bibfield{keywords} field with value <keyw1, keyw2> and set the \bibfield{note} field to the entry key to all entries which are found in either the
%\texttt{examples1.bib} or \texttt{examples2.bib} files.
%
\begin{ltxexample}
\DeclareSourcemap{
  \maps[datatype=bibtex]{
    \map{
      \step[fieldsource=<<title>>]
      \step[fieldset=<<note>>, origfieldval]
    }
  }
}
\end{ltxexample}
%
这一示例将\bibfield{title} 域复制给\bibfield{note} 域,除非\bibfield{note} 域已经存在。
%Copy the \bibfield{title} field to the \bibfield{note} field unless the
%\bibfield{note} field already exists.
%
\begin{ltxexample}
\DeclareSourcemap{
  \maps[datatype=bibtex]{
    \map{
      \step[typesource=<<chat>>, typetarget=<<customa>>, final]
      \step[fieldset=<<type>>, origentrytype]
    }
  }
}
\end{ltxexample}
%
任何\bibfield{chat} 类型的条目将变成为\bibfield{customa} 条目类型,并且自动设置\bibfield{type} 域为<chat>,除非条目中\bibfield{type} 域已经存在(\texttt{overwrite} 默认是false)。这一映射仅对\bibtype{chat} 类型的条目进行操作,因为第一步中设置了\texttt{final},所以当\texttt{typesource} 不匹配时,\cmd{map} 过程立即终止。
%Any \bibfield{chat} entrytypes would become \bibfield{customa} entrytypes and
%would automatically have a \bibfield{type} field set to
%<chat> unless the \bibfield{type} field already exists in the entry (because
%\texttt{overwrite} is false by default). This mapping applies only to entries of type
%\bibtype{chat} since the first step has \texttt{final} set and so if the
%\texttt{typesource} does not match the entry entrytype, processing of this
%\cmd{map} immediately terminates.
%
\begin{ltxexample}
\DeclareSourcemap{
  \maps[datatype=bibtex]{
    \map{
      \perdatasource{<<examples.bib>>}
      \pertype{<<article>>}
      \pertype{<<book>>}
       \step[fieldset=<<abstract>>, null]
       \step[fieldset=<<note>>, fieldvalue={<<Auto-created this field>>}]
    }
  }
}
\end{ltxexample}
%
任何来自\texttt{examples.bib} 数据源,类型为\bibtype{article} 或\bibtype{book} 的条目的\bibfield{abstract} 域将删除,增加一个\bibfield{note} 域,域值为<Auto-created this field>。
%Any entries of entrytype \bibtype{article} or \bibtype{book} from the
%\texttt{examples.bib} datasource would have their \bibfield{abstract}
%fields removed and a \bibfield{note} field added with value <Auto-created this field>.
%
\begin{ltxexample}
\DeclareSourcemap{
  \maps[datatype=bibtex]{
    \map{
       \step[fieldset=<<abstract>>, null]
       \step[fieldsource=<<conductor>>, fieldtarget=<<namea>>]
       \step[fieldsource=<<gps>>, fieldtarget=<<usera>>]
    }
  }
}
\end{ltxexample}
%
这将删除所有条目的\bibfield{abstract} 域,将\bibfield{conductor} 域修改到\bibfield{namea} 域,将\bibfield{gps} 修改到\bibfield{usera} 域中。
%This removes \bibfield{abstract} fields from any entry, changes
%\bibfield{conductor} fields to \bibfield{namea} fields and changes \bibfield{gps}
%fields to \bibfield{usera} fields.
%
\begin{ltxexample}
\DeclareSourcemap{
  \maps[datatype=bibtex]{
    \map{
       \step[fieldsource=<<pubmedid>>, fieldtarget=<<eprint>>, final]
       \step[fieldset=<<eprinttype>>, origfield]
       \step[fieldset=<<userd>>, fieldvalue={<<Some string of things>>}]
    }
  }
}
\end{ltxexample}
%
仅对有\bibfield{pubmedid} 域的条目做映射,将\bibfield{pubmedid} 域映射到\bibfield{eprint} 域中,设置\bibfield{eprinttype} 值为\bibfield{pubmedid} 域的值,并且设置\bibfield{userd} 的值为字符串<Some string of things>。
%Applies only to entries with \bibfield{pubmed} fields and maps
%\bibfield{pubmedid} fields to \bibfield{eprint} fields, sets the \bibfield{eprinttype}
%field to <pubmedid> and also sets the \bibfield{userd} field to the string
%<Some string of things>.
%
\begin{ltxexample}
\DeclareSourcemap{
  \maps[datatype=bibtex]{
    \map{
       \step[fieldsource=<<series>>,
             match=\regexp{<<\A\d*(.+)>>},
             replace=\regexp{<<\L$1>>}]
    }
  }
}
\end{ltxexample}
%$<- to stop emacs highlighting breaking
这里的正则表示式对\bibfield{series} 域值进行处理,match中 <|\A|> 确定匹配的起始位置为字符串开始处,\cmd{d} 匹配一个数字,*表示做\cmd{d} 匹配0次或多次,即把所有从字符串起始位置的数字匹配出来,(标记一个子表达式开始,)标记一个表达式开始结束,.匹配除换行符外的任意字符,+表示做.匹配1次或任意多次,即将域值数字结束后的所有字符标记为一个子表达式。replace中\cmd{L} 表示将后面所有的字符都转换为大写或小写,\file{\$1}%$
表示引用第一组括号内表达式匹配的字符即match中()内匹配的字符,即将原来\bibfield{series} 域值中起始数字之外的字符全部转换为小写或大写\footnote{译者: 正则表达式参考:正则表达式系统教程和精通正则表达式, \cmd{A} , \cmd{L} ,\file{\$1}%$
见精通正则表达式287,290,298页}。因为正则表达式常包含各种特殊字符,最好将其用宏\cmd{regexp} 包含起来,这样就能将其正确的传递给\biber 。
%Here, the contents of the \bibfield{series} field have leading numbers stripped and the remainder of the contents lowercased. Since regular expressions usually contain all sort of special characters, it is best to enclose them in the provided \cmd{regexp} macro as shown---this will pass the expression through to \biber\ correctly.
%
\begin{ltxexample}
\DeclareSourcemap{
  \maps[datatype=bibtex]{
    \map{
       \step[fieldsource=<<maintitle>>,
             match=\regexp{<<Collected\s+Works.+Freud>>},
             final]
       \step[fieldset=<<keywords>>, fieldvalue=<<freud>>]
    }
  }
}
\end{ltxexample}
%$<- to stop emacs highlighting breaking
如果一个条目中\bibfield{maintitle} 域能匹配一个特殊的正则表达式,则将设置一个特殊的keyword,利用这一keyword可以将一些特定项构成一个参考文献节。(其中<|\s|>匹配任何空白字符,包括空格、制表符、换页符。)
%Here, if for an entry, the \bibfield{maintitle} field matches a particular regular expression, we set a special keyword so we can, for example, make a references section just for certain items.
%
\begin{ltxexample}
\DeclareSourcemap{
  \maps[datatype=bibtex]{
    \map{
       \step[fieldsource=<<lista>>, match=\regexp{<<regexp>>}, final]
       \step[fieldset=<<lista>>, null]
    }
  }
}
\end{ltxexample}
%
如果一个条目的\bibfield{lista} 匹配正则表达式<regexp>,则删除该条目。
%If an entry has a \bibfield{lista} field which matches regular expression <regexp>, then it is removed.
%
\begin{ltxexample}
\DeclareSourcemap{
  \maps[datatype=bibtex]{
    \map[overwrite=false]{
       \step[fieldsource=<<author>>]
       \step[fieldset=<<editor>>, origfieldval, final]
       \step[fieldsource=<<editor>>, match=\regexp{<<\A(.+?)\s+and.*>>}, replace={<<$1>>}]
    }
  }
}
\end{ltxexample}
%$<- to stop emacs highlighting breaking
对于任何有\bibfield{author} 域的条目,将\bibfield{editor} 设置为与\bibfield{author} 相同,但如果\bibfield{editor} 已经存在则不设置并终止映射。如果完成设置,对\bibfield{editor} 进一步匹配和替换。(正则表达式的作用是截取姓名列表中第一个姓名。其match规则为从字符串开始就标记子表达式开始,在遇到空白字符和and之前标记子表达式结束,子表达式匹配任意字符,利用问号结束贪婪过程,即在第一个空白字符和and之前结束匹配。replace则表示提取第一个子表达式的匹配结果。)
%For any entry with an \bibfield{author} field, try to set
%\bibfield{editor} to the same as \bibfield{author}. If this fails because
%\bibfield{editor} already exists, stop, otherwise truncate
%\bibfield{editor} to just the first name in the name list.
%
\begin{ltxexample}
\DeclareSourcemap{
  \maps[datatype=bibtex]{
    \map{
       \step[fieldsource=<<author>>,
             match={<<Smith, Bill>>},
             replace={<<Smith, William>>}]
       \step[fieldsource=<<author>>,
             match={<<Jones, Baz>>},
             replace={<<Jones, Barry>>}]
    }
  }
}
\end{ltxexample}
%
这里,对同一个域使用多个匹配/替换来调整一些姓名。记住,一个\opt{map} 元素中\cmd{step} 步是按顺序处理的,后面的\cmd{step} 步是在前面各步处理结果基础上再处理。注意,当匹配字符中没有一些特殊的字符,则没有必要用\cmd{regexp} 来保护正则表达式。需要注意,因为在\LaTeX 中保护正则表达式的难度,不要在\cmd{regexp} 中出现空格,而要使用空格对应的代码。比如:
%Here, we use multiple match/replace for the same field to regularise some inconstant name variants. Bear in mind that \cmd{step} processing within a \opt{map} element is sequential and so the changes from a previous \cmd{step}s are already committed. Note that we don't need the \cmd{regexp} macro to protect the regular expressions in this example as they contain no characters which need special escaping. Please note that due to the difficulty of protecting regular expressions in \LaTeX, there should be no literal spaces in the argument to \cmd{regexp}. Please use escape code equivalents if spaces are needed. For example, this example, if using \cmd{regexp}, should be:
%
\begin{ltxexample}
\DeclareSourcemap{
  \maps[datatype=bibtex]{
    \map{
       \step[fieldsource=<<author>>,
             match=\regexp{<<Smith,\s+Bill>>},
             replace=\regexp{<<Smith,\x20William>>}]
       \step[fieldsource=<<author>>,
             match=\regexp{<<Jones,\s+Baz>>},
             replace=\regexp{<<Jones,\x20Barry>>}]
    }
  }
}
\end{ltxexample}
%
其中,使用了十六进制符号命令 <|\x20|> 来充当替换字符串中的空格。
%Here, we have used the hexadecimal escape sequence <|\x20|> in place of literal spaces in the replacement strings.
%
\begin{ltxexample}
\DeclareSourcemap{
  \maps[datatype=bibtex]{
    \map[overwrite]{
       \step[fieldsource=<<author>>, match={<<Doe,>>}, final]
       \step[fieldset=<<shortauthor>>, origfieldval]
       \step[fieldset=<<sortname>>, origfieldval]
       \step[fieldsource=<<shortauthor>>,
             match=\regexp{<<Doe,\s*(?:\.|ohn)(?:[-]*)(?:P\.|Paul)*>>},
             replace={<<Doe, John Paul>>}]
       \step[fieldsource=<<sortname>>,
             match=\regexp{<<Doe,\s*(?:\.|ohn)(?:[-]*)(?:P\.|Paul)*>>},
             replace={<<Doe, John Paul>>}]
    }
  }
}
\end{ltxexample}
%
仅对\bibfield{author} 匹配有<Doe,>的条目做处理。首先将\bibfield{author} 域复制给\bibfield{shortauthor} 和\bibfield{sortname} 域,当这两个域已存在仍然进行覆盖。然后对这两个新域做匹配和替换操作,用以规范化一个特定的姓名,因为该名字在数据源中可能有一些不同的形式。
其中<|(?:pattern)|>表示匹配pattern但不获取,第一个括号中只要是.或ohn结尾即可匹配,第二个括号中只要有-字符即可匹配,第三个括号中以P.或Paul结尾即可匹配。
%Only applies to entries with an \bibfield{author} field matching <Doe,>. First the \bibfield{author} field is copied to both the \bibfield{shortauthor} and \bibfield{sortname} fields, overwriting them if they already exist. Then, these two new fields are modified to canonicalise a particular name, which presumably has some variants in the data source.
%
\begin{ltxexample}
\DeclareSourcemap{
  \maps[datatype=bibtex]{
    \map[overwrite]{
      \step[fieldsource=<<verba>>, final]
      \step[fieldset=<<verbb>>, fieldvalue=<</>>, append]
      \step[fieldset=<<verbb>>, origfieldval, append]
      \step[fieldsource=<<verbb>>, final]
      \step[fieldset=<<verbc>>, fieldvalue=<</>>, append]
      \step[fieldset=<<verbc>>, origfieldval, append]
    }
  }
}
\end{ltxexample}
%
该例验证了step步的顺序特性和\opt{append} 选项。如果一个条目有\bibfield{verba} 域,首先在\bibfield{verbb} 域中增加一个斜杠,然后\bibfield{verba} 域值添加到\bibfield{verbb} 中。然后在\bibfield{verbc} 中添加一个斜杠和\bibfield{verbb} 域的值。
%This example demonstrates the sequential nature of the step processing and the \opt{append} option. If an entry has a \bibfield{verba} field then first, a forward slash is appended to the \bibfield{verbb} field. Then, the contents of \bibfield{verba} are appended to the \bibfield{verbb} field. A slash is then appended to the \bibfield{verbc} field and the contents of \bibfield{verbb} are appended to the \bibfield{verbc} field.
%
\begin{ltxexample}
\DeclareSourcemap{
  \maps[datatype=bibtex]{
    \map[overwrite]{
      \step[fieldset=<<autourl>>, fieldvalue={<<http://scholar.google.com/scholar?q=">>}]
      \step[fieldsource=<<title>>]
      \step[fieldset=<<autourl>>, origfieldval, append]
      \step[fieldset=<<autourl>>, fieldvalue={<<"+author:>>}, append]
      \step[fieldsource=<<author>>, match=\regexp{<<\A([^,]+)\s*,>>}]
      \step[fieldset=<<autourl>>, fieldvalue={<<$1>>}, append]
      \step[fieldset=<<autourl>>, fieldvalue={<<&as_ylo=>>}, append]
      \step[fieldsource=<<year>>]
      \step[fieldset=<<autourl>>, origfieldval, append]
      \step[fieldset=<<autourl>>, fieldvalue={<<&as_yhi=>>}, append]
      \step[fieldset=<<autourl>>, origfieldval, append]
    }
  }
}
\end{ltxexample}%$ <- keep AucTeX highlighting happy
该例假设使用\secref{aut:ctm:dm} 中的数据模型创建了一个名为\bibfield{autourl} 的域用来保存比如由条目内容自动创建的Google学术搜索地址。逐步从条目中抽取信息并构建URL。它验证了可以在后面所有的\texttt{fieldvalue} 设置中引用前面最近匹配得到的结果,如果\bibfield{author} 域格式是<family, given>,可以从中抽取姓(family name)。结果域可以用作文献表中文献的标题的一个超链接。(其中 <|[^,]+|> 匹配不包含,的任意字符并获取为\cmd{\$1}。)%$

%This example assumes you have created a field called \bibfield{autourl} using the datamodel macros from \secref{aut:ctm:dm} in order to hold, for example, a Google Scholar query URL auto-created from elements of the entry. The example progressively extracts information from the entry, constructing the URL as it goes. It demonstrates that it is possible to refer to parenthetical matches from the most recent \texttt{match} in any following \texttt{fieldvalue} which allows extracting the family name from the \bibfield{author}, assuming a <family, given> format. The resulting field could then be used as a hyperlink from, for example, the title of the work in the bibliography.
%
\begin{ltxexample}
\DeclareSourcemap{
  \maps[datatype=bibtex]{
    \map{
      \step[fieldsource=<<title>>, match={A Title}, final]
      \step[entrynull]
    }
  }
}
\end{ltxexample}
%
当条目的\bibfield{title} 匹配<A Title>时,忽略该条目(它将不出现在参考文献数据中)。
%Any entry with a \bibfield{title} field matching <A Title> will be completely ignored.
%
\begin{ltxexample}
\DeclareSourcemap{
  \maps[datatype=bibtex]{
    \map{
      \pernottype{book}
      \pernottype{article}
      \step[entrynull]
    }
  }
}
\end{ltxexample}
%
当条目不是\bibtype{book} 或\bibtype{article} 类型将被忽略。
%Any entry which is not a \bibtype{book} or \bibtype{article} will be ignored.
%
\begin{ltxexample}
\DeclareSourcemap{
  \maps[datatype=bibtex]{
    \map{
      \perdatasource{<<biblatex-examples.bib>>}
      \step[entryclone={rel-}]
    }
  }
}
\end{ltxexample}
%
数据源中的条目将被复制。复制条目的关键词为原条目关键词加上\texttt{entryclone} 参数给出的前缀。复制的条目在文档中以其新的关键词进行引用。这种类型的映射需要谨慎使用,因为标签生成可能产生问题,比如在作者年制使用\opt{extrayear} 时。一种应用情况是在顺序编码制中多个文献表包含相同的条目时需要为这一相同同条目生成不同的序号标签。当需要对一个条目生成不同的标签时,这是很麻烦的,而创建具有不同关键词的副本条目能解决这一问题。
%Here, a clone of an entry from the specified data source will be created. The entry key of the clone will be the same as the original but prefixed by the value of the \texttt{entryclone} parameter. The cloned entry would still need to be cited in the document using its new entry key. This type of mapping step should be used with care as it may produce labelling problems in authoryear styles which use, for example, \opt{extrayear}. One use case is for numeric styles which contain multiple bibliographies containing the same entry. In this case, you may need different bibliography number labeld for the same entry and this is very tricky when there is only one entry which needs different labels. Creating clones with different entry keys solves this problem.

\biblatexml\ 数据源比\bibtex\ 更具结构性,因为它们是XML。源的映射也可以对其处理,但源和目标的设置需要支持XPath 1.0路径以便能够处理结构化数据。域的设置可以利用上述的示例,必要时可以转换为XPath 1.0的内部访问方式。比如:
%\biblatexml\ datasources are more structured than \bibtex\ since they are XML. Sourcemapping is possible with \biblatexml\ too but the specifications of source and target fields etc. also support XPath 1.0 paths in order to be able to work with the structured data. Fields can be specified as per the \bibtex\ examples above and these are converted into XPath 1.0 queries internally as necessary. For example:

\begin{ltxexample}
\DeclareSourcemap{
  \maps[datatype=biblatexml]{
    \map{
   \step[fieldsource=\regexp{./bltx:names[@type='author']/bltx:name[2]/bltx:namepart[@type='family']},
      match=\regexp{\ASmith},
      replace={Jones}]
    }
    \map{
      \step[fieldsource=editor, fieldtarget=translator]
    }
    \map{
      \step[fieldsource=\regexp{./bltx:names[@type='editor']},
            fieldtarget=\regexp{./bltx:names[@type='translator']}]
    }
    \map{
      \step[fieldset=\regexp{./bltx:names[@type='author']/bltx:name[2]/@useprefix},
            fieldvalue={false}]
    }
  }
}
\end{ltxexample}
%
%These maps, respectively,
这些映射,分别为:
\begin{itemize}
\item 将\bibfield{author} 域的第二个姓名的姓<Smith>替换为<Jones>。
%Replace the family name <Smith> of the second \bibfield{author} name with <Jones>
\item 将\bibfield{editor} 域值设置到\bibfield{translator}。
%Move the \bibfield{editor} to \bibfield{translator}
\item 将\bibfield{editor} 域值设置到\bibfield{translator},但使用显式的XPaths。这和第二个映射操作结果是一样的。
%Move the \bibfield{editor} to \bibfield{translator} but with explicit XPaths
\item 将\bibfield{author} 域的姓名列表\opt{useprefix} 选项设置为<false>。
%Set the per-namelist \opt{useprefix} option on the \bibfield{author} name list to <false>
\end{itemize}

\subsubsection{数据模型规范}%Data Model Specification
\label{aut:ctm:dm}
\biblatex 使用的数据模型包括4个主要元素:
%The data model which \biblatex uses consists of four main elements:

\begin{itemize}
\item 字符串和字符串列表常量规范
%Specification of constant strings and lists of strings
\item 有效条目类型规范
%Specification of valid Entrytypes
\item 域及其类型、数据类型和其它特殊标识的规范
%Specification of valid Fields along with their type, datatype and any special flags
\item 在什么条目中什么域有效的规范
%Specification of which Fields are valid in which Entrytypes
\item 可用于根据数据模型验证数据的约束规范
%Specification of constraints which can be used to validate data against the data model
\end{itemize}

默认数据模型在核心\biblatex 文件\file{blx-dm.def} 中定义,使用本节给出的宏。默认数据模型详见\secref{bib} 节。数据模型由\biblatex 及后端在内部使用。在实际中,改变数据模型意味着为数据源定义条目类型和域,并可根据数据模型进行验证。自然,这通常只有在样式能支持新的条目类型时有用,并且也可能引起样式间的可移植性问题。(尽管这一问题可以使用\secref{aut:ctm:map} 节介绍的动态数据修改来解决。)
%The default data model is defined in the core \biblatex file \file{blx-dm.def} using the macros described in this section. The default data model is described in detail in \secref{bib}. The data model is used internally by \biblatex and also by the backend. In practice, changing the data model means that you can define the entrytypes and fields for your datasources and validate your data against the data model. Naturally, this is not much use unless your style supports any new entrytypes or fields and it raises issues of portability between styles (although this can be mitigated by using the dynamic data modification functionality described in \secref{aut:ctm:map}).

根据数据模型验证意味着,将数据源映射到数据模型后,\biber (使用\path{--validate_datamodel} 选项)可以检查:
%Validation against the data model means that after mapping your data sources into the data model, \biber (using its \path{--validate_datamodel} option) can check:

\begin{itemize}
\item 所有条目类型是否都有效
%Whether all entrytypes are valid entrytypes
\item 所有的域对于其所属的条目类型是否有效
%Whether all fields are valid fields for their entrytype
\item 域是否都满足指定的不同格式约束
%Whether the fields obey various constraints on their format which you specify
\end{itemize}
%
重定义数据模型可以在多个地方进行。样式作者可以创建一个\file{.dbx} 文件包含需要的数据模型宏,该文件将在\biblatex 宏包根据\opt{style} 选项加载\file{.cbx} 和\file{.bbx} 样式文件后搜索\file{.dbx} 文件来加载\footnote{译者: 可用来定义标准和报纸文章两类条目,但实际上利用数据动态修改可能更为方便}。如果不使用\opt{style} 选项,而使用\opt{citestyle} 和\opt{bibstyle} 选项,宏包将会搜索文件\file{$<$citestyle$>$.dbx} 和\file{$<$bibstyle$>$.dbx} 并加载。另一种方式,数据模型文件名可以不同于样式选项名,但扩展名必须是\file{.dbx},该文件由\opt{datamodel} 包选择加载。在加载完样式的数据模型文件之后,\biblatex 寻找并加载\sty{biblatex-dm.cfg} 配置文件,类似于加载\sty{biblatex.cfg} 的方式。总之,最终数据模型由从这些地方加入的数据模型确定,顺序是:
%Redefining the data model can be done in several places. Style authors can create a \file{.dbx} file which contains the data model macros required and this will be loaded automatically when using the \biblatex package \opt{style} option by looking for a file named after the style with a \file{.dbx} extension (just like the \file{.cbx} and \file{.bbx} files for a style). If the \opt{style} option is not used but rather the \opt{citestyle} and \opt{bibstyle} options, then the package will try to load \file{.dbx} files called \file{$<$citestyle$>$.dbx} and \file{$<$bibstyle$>$.dbx}.
%Alternatively, the name of the data model file can be different from any of the style option names by specifying the name (without \file{.dbx} extension) to the package \opt{datamodel} option. After loading the style data model file, \biblatex then loads, if present, a users \file{biblatex-dm.cfg} which should be put somewhere \biblatex can find it, just like the main configuration file \sty{biblatex.cfg}. To summarise, the data model is determined by adding to the data model from each of these locations, in order:\\

\noindent\file{blx-dm.def}$\rightarrow$\\
\hspace*{1em}\file{$<$datamodel option$>$.dbx} $\rightarrow$\\
\hspace*{2em}\file{$<$style option$>$.dbx} $\rightarrow$\\
\hspace*{3em}\file{$<$citestyle option$>$.dbx} and \file{$<$bibstyle option$>$.dbx} $\rightarrow$\\
\hspace*{4em}\file{biblatex-dm.cfg}\\

在导言区使用宏来加载数据模型是不行的,因为导言内容是在\biblatex 根据数据模型定义关键的内部宏之后读取。在正文中定义的任何数据模型宏都将被忽略并给出一个警告。数据模型由如下宏来定义:
%\noindent It is not possible to add to a loaded data model by using the macros below in your preamble as the preamble is read after \biblatex has defined critical internal macros based on the data model. If any data model macro is used in a document, it will be ignored and a warning will be generated. The data model is defined using the following macros:

\begin{ltxsyntax}

\cmditem{DeclareDatamodelConstant}[options]{name}{constantdef}

声明\prm{name} 作为一个数据模型常量,其定义为\prm{constantdef}。这种常量通常由\biber 在内部使用。
%Declares the \prm{name} as a datamodel constant with definition \prm{constantdef}. Such constants are typically used internally by \biber.

\begin{optionlist*}

\choitem[string]{type}{string, list}

一个常量可以是一个简单的字符串(默认情况下,如果\prm{type} 选项忽略的话)或者一个逗号分隔的字符串列表。
%A constant can be a simple string (default if the \prm{type} option is omitted) or a comma"=separated list of strings.

\end{optionlist*}

\cmditem{DeclareDatamodelEntrytypes}[options]{entrytypes}

声明\prm{entrytypes} 的一个逗号分隔列表,表示数据模型中的有效条目类型。依旧如同在\tex 的csv列表中,请确保每个元素都紧跟着一个逗号或者结束的括号,即不需要额外的空格。
%Declares the comma"=separated list of \prm{entrytypes} to be valid entrytypes in the data model. As usual in \tex csv lists, make sure each element is immediately followed by a comma or the closing brace---no extraneous whitespace.

\begin{optionlist*}

\boolitem[false]{skipout}

该条目类型不输出到\file{.bbl} 中。通常用于后端处理和使用的特殊条目类型比如\bibtype{xdata} 等。
%This entrytype is not output to the \file{.bbl}. Typically used for special entrytypes which are processed and consumed by the backend such as \bibtype{xdata}.

\end{optionlist*}

\cmditem{DeclareDatamodelFields}[options]{fields}

声明\prm{fields} 的一个逗号分隔列表,表示数据模型中与逗号分隔\prm{options} 列表相关的有效域。\prm{type} 和\prm{datatype} 选项是必须的,全部有效选项如下:
%Declares the comma"=separated list of \prm{fields} to be valid fields in the data model with associated comma"=separated \prm{options}. The \prm{type} and \prm{datatype} options are mandatory. All valid \prm{options} are:

\begin{optionlist*}

\valitem{type}{field type}

设置域的类型,典型如<field>或<list>,详见\secref{bib:fld:typ} 节。
%Set the type of the field as described in \secref{bib:fld:typ}, typically <field> or <list>.

\valitem{format}{field format}

该域的任意特殊格式。正常情况下不给出但获取<xsv>值,该值告诉\biber 该域具有多值格式。准确的分隔符由\biber 选项\opt{xsvsep} 控制,默认是由可选的空格包围的逗号。
%Any special format of the field. Normally unspecified but can take the value <xsv> which tells \biber that this field has a separated values format. The exact separator can be controlled with the \biber option \opt{xsvsep} and defaults to the expected comma surrounded by optional whitespace.

\valitem{datatype}{field datatype}

设置域的数据类型,典型如 <name>或<literal>,详见\secref{bib:fld:typ} 节。
%Set the datatype of the field as described in \secref{bib:fld:typ}. For example, <name> or <literal>.

\boolitem[false]{nullok}

该域可以定义为空。
%The field is allowed to be defined but empty.

\boolitem[false]{skipout}

该域不输出到\file{.bbl} 中因此不会出现在\biblatex 样式处理中。依旧如同在\tex 的csv列表中,请确保每个元素都紧跟着一个逗号或者结束的括号,即不需要额外的空格。
%The field is not output to the \file{.bbl} and is therefore not present during \biblatex style processing. As usual in \tex csv lists, make sure each element is immediately followed by a comma or the closing brace---no extraneous whitespace.

\boolitem[false]{label}

该域可以用作文献表的标签。设置该选项会使得\biblatex 为该域创建多个辅助宏,所以存在一些已经定义的内部长度和标题等。
%The field can be used as a label in a bibliography or bibliography list. Specifying this causes \biblatex to create several helper macros for the field so that there are some internal lengths and headings etc. defined.

\end{optionlist*}

\cmditem{DeclareDatamodelEntryfields}[entrytypes]{fields}

声明\prm{fields} 列表对于\prm{entrytypes} 列表是有效的。如果\prm{entrytypes} 不给出,则列表中的这些域对所有条目类型都有效。依旧如同在\tex 的csv列表中,请确保每个元素都紧跟着一个逗号或者结束的括号,即不需要额外的空格。
%Declares that the comma"=separated list of \prm{fields} is valid for the comma"=separated list of \prm{entrytypes}. If \prm{entrytypes} is not given, the fields are valid for all entrytypes. As usual in \tex csv lists, make sure each element is immediately followed by a comma or the closing brace---no extraneous whitespace.

\cmditem{DeclareDatamodelConstraints}[entrytypes]{specification}

如果给出\prm{entrytypes} 的逗号分隔列表,约束仅应用于这些条目。\prm{specification} 是\cmd{constraint} 指令的不分隔列表。为实现良好的代码显示效果,可以自由使用空格、制表符、行末符号来整理\prm{specification} 中的代码,但空行不能使用。
%If a comma"=separated list of \prm{entrytypes} is given, the constraints apply only to those entrytypes. The \prm{specification} is an undelimited list of \cmd{constraint} directives which specify a constraint. Spaces, tabs, and line endings may be used freely to visually arrange the \prm{specification}. Blank lines are not permissible.

\cmditem{constraint}[type=constrainttype]{elements}

指定\prm{constrainttype} 类型的约束,有效的约束类型有:
%Specifies a constraint of type \prm{constrainttype}. Valid constraint types are:

\begin{optionlist*}

\choitem{type}{data, mandatory, conditional}

<data>类型的约束,应用于域的值。<mandatory>类型的约束,指定一个条目类型应具有的域或域的组合。<conditional>类型的约束允许更复杂的条件和量化的域约束。
%Constraints of type <data> put restrictions on the value of a field. Constraints of type <mandatory> specify which fields or combinations of fields an entrytype should have. Constraints of type <conditional> allow more sophisticated conditional and quantified field constraints.

\choitem{datatype}{integer, isbn, issn, ismn, date, pattern}

用于\prm{data} 类型约束,限制域的值为给定的数据类型。
%For constraints of type \prm{data}, constrain field values to be the given datatype.

\valitem{rangemin}{num}

用于\prm{data} 类型和<integer>数据类型的约束,限制域的值最小为\prm{num}。
%For constraints of \prm{type} <data> and \prm{datatype} <integer>, constrain field values to be at least \prm{num}.

\valitem{rangemax}{num}

用于\prm{data} 类型和<integer>数据类型的约束,限制域的值最大为\prm{num}。
%For constraints of \prm{type} <data> and \prm{datatype} <integer>, constrain field values to be at most \prm{num}.

\valitem{pattern}{patt}

用于\prm{data} 类型和<pattern>数据类型的约束,限制域的值匹配正则表达式\prm{patt}。最好利用\cmd{regexp} 宏将正则表达式包围起来,见\secref{aut:ctm:map} 节。
%For constraints of \prm{type} <data> and \prm{datatype} <pattern>, constrain field values to match regular expression pattern \prm{patt}. It is best to wrap any regular expression in the macro \cmd{regexp}, see \secref{aut:ctm:map}.

\end{optionlist*}

一个\cmd{constraint} 宏可以包括如下任意内容:
%A \cmd{constraint} macro may contain any of the following:

\cmditem{constraintfieldsor}{fields}

用于<mandatory>类型的约束,指定一个条目必须包含\cmd{constraintfield}s的一个布尔或(OR)。
%For constraints of \prm{type} <mandatory>, specifies that an entry must contain a boolean OR of the \cmd{constraintfield}s.

\cmditem{constraintfieldsxor}{fields}

用于<mandatory>类型的约束,指定一个条目必须包含\cmd{constraintfield}s的一个布尔异或(XOR)。
%For constraints of \prm{type} <mandatory>, specifies that an entry must contain a boolean XOR of the \cmd{constraintfield}s.

\cmditem{antecedent}[quantifier=quantspec]{fields}

用于<conditional>类型的约束,指定一个在约束的\cmd{consequent} 检查之前必须满足\cmd{constraintfield}s的量化集。\prm{quantspec} 应包含如下值:
%For constraints of \prm{type} <conditional>, specifies a quantified set of \cmd{constraintfield}s which must be satisfied before the \cmd{consequent} of the constraint is checked. \prm{quantspec} should have one of the following values:

\begin{optionlist*}

\choitem{quantifier}{all, one, none}

指定要满足条件约束前提,必须给出的\cmd{antecedent} 内的\cmd{constrainfield} 数量。
%Specifies how many of the \cmd{constrainfield}'s inside the \cmd{antecedent} have to be present to satisfy the antecedent of the conditional constraint.

\end{optionlist*}

\cmditem{consequent}[quantifier=quantspec]{fields}

用于<conditional>类型的约束,指定如果满足前面约束的\cmd{antecedent} 时必须满足的\cmd{constraintfield}s量化集。\prm{quantspec} 应包含如下值:
%For constraints of \prm{type} <conditional>, specifies a quantified set of \cmd{constraintfield}s which must be satisfied if the preceding \cmd{antecedent} of the constraint was satisfied. \prm{quantspec} should have one of the following values:

\begin{optionlist*}

\choitem{quantifier}{all, one, none}

指定要满足的条件约束后件,必须给出的\cmd{consequent} 内的\cmd{constrainfield} 数量。
%Specifies how many of the \cmd{constraintfield}'s inside the \cmd{consequent} have to be present to satisfy the consequent of the conditional constraint.

\end{optionlist*}

\cmditem{constraintfield}{field}

对于<data>类型的约束,约束应用于该\prm{field}。对于<mandatory>类型的约束,条目必须包含\prm{field}。

%For constraints of \prm{type} <data>, the constraint applies to this \prm{field}. For constraints of \prm{type} <mandatory>, the entry must contain this \prm{field}.

数据模型声明宏可以应用多次,用于在前面定义基础上增加定义。如果要替换,修改或移除已有的定义(比如\biblatex 加载的默认模型),需要利用下面的宏重设(清除)当前的定义然后再重设。通常,这些宏在任何\file{biblatex-dm.cfg} 文件中都是首先给出的:
%The data model declaration macros may be used multiple times as they append to the previous definitions. In order to replace, change or remove existing definitions (such as the default model which is loaded with \biblatex), you should reset (clear) the current definition and then set what you want using the following macros. Typically, these macros will be the first things in any \file{biblatex-dm.cfg} file:

\cmditem{ResetDatamodelEntrytypes}

清除所有的数据模型entrytype信息
%Clear all data model entrytype information.

\cmditem{ResetDatamodelFields}

清除所有的数据模型field信息
%Clear all data model field information.

\cmditem{ResetDatamodelEntryfields}

清除所有的数据模型entrytypes的fields信息
%Clear all data model fields for entrytypes information.

\cmditem{ResetDatamodelConstraints}

清除所有的数据模型fields约束信息
%Clear all data model fields Constraints information.

\end{ltxsyntax}

下面是一个简单的数据模型示例。默认数据模型设置参见\biblatex 核心文件\file{blx-dm.def}。
%Here is an example of a simple data model. Refer to the core \biblatex file \file{blx-dm.def} for the default data model specification.

\begin{ltxexample}
\ResetDatamodelEntrytypes
\ResetDatamodelFields
\ResetDatamodelEntryfields
\ResetDatamodelConstraints

\DeclareDatamodelEntrytypes{<<entrytype1, entrytype2>>}

\DeclareDatamodelFields[type=field, datatype=literal]{<<field1,field2,field3,field4>>}

\DeclareDatamodelEntryfields{<<field1>>}
\DeclareDatamodelEntryfields[entrytype1]{<<field2,field3>>}
\DeclareDatamodelEntryfields[entrytype2]{<<field2,field3,field4>>}

\DeclareDatamodelConstraints[<<entrytype1>>]{
  \constraint[type=data, datatype=integer, rangemin=3, rangemax=10]{
    \constraintfield{<<field1>>}
  }
  \constraint[type=mandatory]{
    \constraintfield{<<field1>>}
    \constraintfieldsxor{
      \constraintfield{<<field2>>}
      \constraintfield{<<field3>>}
    }
  }
}
\DeclareDatamodelConstraints{
  \constraint[type=conditional]{
    \antecedent[quantifier=none]{
      \constraintfield{<<field2>>}
    }
    \consequent[quantifier=all]{
      \constraintfield{<<field3>>}
      \constraintfield{<<field4>>}
    }
  }
}
\end{ltxexample}
%
该模型设置了:
%This model specifies:

\begin{itemize}
\item 清除所有默认的数据模型
%Clear the default data model completely
\item 设置两种有效的条目类型\bibtype{entrytype1} and \bibtype{entrytype2}
%Two valid entry types \bibtype{entrytype1} and \bibtype{entrytype2}
\item 设置四个有效的literal域。
%Four valid literal field fields
\item 设置\bibfield{field1} 对所有条目类型都有效。
%\bibfield{field1} is valid for all entrytypes
\item 设置\bibfield{field2} 和\bibfield{field3} 对\bibfield{entrytype1} 有效。
%\bibfield{field2} and \bibfield{field3} are valid for \bibfield{entrytype1}
\item 设置\bibfield{field2}, \bibfield{field3} 和\bibfield{field4} 对\bibfield{entrytype2} 有效。
%\bibfield{field2}, \bibfield{field3} and \bibfield{field4} are valid for \bibtype{entrytype2}
\item 设置\bibtype{entrytype1} 的约束:
%For \bibtype{entrytype1}:
  \begin{itemize}
  \item \bibfield{field1} 必须是一个3和10之间的整数
  %\bibfield{field1} must be an integer between 3 and 10
  \item \bibfield{field1} 必须给出
  %\bibfield{field1} must be present
  \item \bibfield{field2} 或\bibfield{field3} 其中之一必须唯一给出。
  %One and only one of \bibfield{field2} or \bibfield{field3} must be present
  \end{itemize}
\item 对于所有类型,\bibfield{field2} 如果不给出,\bibfield{field3} 和\bibfield{field4} 必须要给出。
%For any entrytype, if \bibfield{field2} is not present, \bibfield{field3} and \bibfield{field4} must be present
\end{itemize}

\subsubsection{标签}%Labels
\label{aut:ctm:lab}
字母顺序制样式使用一个标签来区分参考文献条目。这个标签使用一个描述怎么构建标签的模板由条目的内容构建。该模板可以全局自定义或者根据具体条目类型定义。如何抽取姓名域的成分作为标签则使用一个独立的模板,因为姓名域是相当复杂的域。
%标签的自定义需要用\biber 后端程序而不能用其它后端程序。
%Alphabetic styles use a label to identify bibliography entries. This label is constructed from components of the entry using a template which describes how to build the label. The template can be customised on a global or per-type basis. A separate template is used to specify how to extract parts of name fields for labels, since names can be quite complex fields.

\begin{ltxsyntax}

\cmditem{DeclareLabelalphaTemplate}[entrytype, \dots]{specification}

为指定的条目类型定义字母顺序制标签模板。如果第一个参数中不指定具体的条目类型,则定义的是通用标签模板。\prm{specification} 是\cmd{labelelement} 指令的一个无分隔列表,用来指定构建标签的元素。为实现良好的代码显示效果,可以自由使用空格、制表符、行末符号来整理\prm{specification} 中的代码,但空行不能使用。该命令仅可在导言区使用。
%Defines the alphabetic label template for the given entrytypes. If no entrytypes are specified in the first argument, then the global label template is defined. The \prm{specification} is an undelimited list of \cmd{labelelement} directives which specify the elements used to build the label. Spaces, tabs, and line endings may be used freely to visually arrange the \prm{specification}. Blank lines are not permissible. This command may only be used in the preamble.

\cmditem{labelelement}{elements}

指定用于构建标签的元素。\prm{elements} 是一个\cmd{field} 或\cmd{literal} 命令构成的无分隔列表,这些命令以它们给出的顺序进行处理。从第一个展开为非空字符串的\cmd{field} 或\cmd{literal} 作为\cmd{labelelement} 展开内容开始,后面的\cmd{labelelement} 如果存在的话,接着进行处理。
%Specifies the elements used to build the label. The \prm{elements} are an undelimited list of \cmd{field} or \cmd{literal} commands which are evaluated in the order in which they are given. The first \cmd{field} or \cmd{literal} which expands to a non-empty string is used as the \cmd{labelelement} expansion and the next \cmd{labelelement}, if any, is then processed.

\cmditem{field}[options]{field}

如果\prm{field} 非空,并作为当前标签的\cmd{labelelement},那么受如下选项约束。\prm{field} 的有用参数通常是姓名列表类型的域、日期域和标题域。可以使用虚域`citekey'来将引用关键词作为标签的一部分。姓名列表域需要特殊处理,当给出一个姓名列表域时,用\cmd{DeclareLabelalphaNameTemplate} 命令定义的模板用来从姓名中抽取某些部分,并返回字符串给\cmd{field} 选项使用。
%If \prm{field} is non-empty, use it as the current label \cmd{labelelement}, subject to the options below. Useful values for \prm{field} are typically the name list type fields, date fields, and title fields. You may also use the `citekey' pseudo-field to specify the citation key as part of the label. Name list fields are treated specially and when a name list field is specified, the template defined with \cmd{DeclareLabelalphaNameTemplate} is used to extract parts from the name which then returns the string that the \cmd{field} option uses.

\begin{optionlist*}

\boolitem[false]{final}

该选项标记一个\cmd{field} 指令作为\prm{specification} 中的最后一个。如果\prm{field} 非空,该域用于标签中,\prm{specification} 中后面的剩余内容将被忽略。简写形式\opt{final} 等价于\kvopt{final}{true}。
%This option marks a \cmd{field} directive as the final one in the \prm{specification}. If the \prm{field} is non-empty, then this field is used for the label and the remainder of the \prm{specification} will be ignored. The short form \opt{final} is equivalent to \kvopt{final}{true}.

\boolitem[false]{lowercase}

将从域中得到标签内容转换为小写。默认情况下,直接从域的源数据中取出的内容大小写形式不做改变。
%Forces the label part derived from the field to lowercase. By default, the case is taken from the field source and not modified.

\intitem[1]{strwidth}

使用的\prm{field} 的固定字符数。当从一个姓名中抽取字符时,该设置可能会被一个独立的姓名成分覆盖,见下面的\cmd{DeclareLabelalphaNameTemplate}。
%The number of characters of the \prm{field} to use. This setting may be overridden by an individual name part when extracting characters from a name. See \cmd{DeclareLabelalphaNameTemplate} below.

\choitem[left]{strside}{left, right}

取\texttt{strwidth} 数量字符开始的方向。当从一个姓名中抽取字符时,该设置可能会被一个独立的姓名成分覆盖,见下面的\cmd{DeclareLabelalphaNameTemplate}。
%The side of the string from which to take the \texttt{strwidth} number of characters. This setting may be overridden by an individual name part when extracting characters from a name. See \cmd{DeclareLabelalphaNameTemplate} below.

\choitem[right]{padside}{left, right}

当使用\texttt{padchar} 选项时,向标签部分添加衬垫字符的方向。仅用于固定宽度标签字符串(\texttt{strwidth})。
%Side to pad the label part when using the \texttt{padchar} option. Only for use with fixed-width label strings (\texttt{strwidth}).

\valitem{padchar}{character}

如果存在,则将在标签部分的\texttt{padside} 侧添加衬垫字符直到字符串长度为\texttt{strwidth},仅用于固定宽度标签字符串(\texttt{strwidth})。
%If present, pads the label part on the \texttt{padside} side with the specified character to the length of \texttt{strwidth}. Only for use with fixed-width label strings (\texttt{strwidth}).

\boolitem[false]{uppercase}

将从域中得到标签内容转换为大写。默认情况下,直接从域的源数据中取出的内容大小写形式不做改变。
%Forces the label part derived from the field to uppercase. By default, the case is taken from the field source and not modified.

\boolitem[false]{varwidth}

使用一个宽度变量,并从\prm{field} 返回的字符串的左侧开始抽取字符。字符串的长度由要区分标签相同位置上字符串所需的最小长度来确定。对于姓名列表域,这意味着每个姓名的子字符串要与姓名列表中相同位置的所以其他姓名的字符串相区别(见下面示例)。该选项会覆盖\texttt{strwidth},如果两者同时用。简写形式\opt{varwidth} 等价于\kvopt{varwidth}{true}。对于姓名列表域,设置\opt{pre} 选项的\cmd{namepart}s将添加在非歧义化过程返回的字符串前面。
%Use a variable width, left-side substring of characters from the string returned for \prm{field}. The length of the string is determined by the minimum length needed to disambiguate the substring from all other \prm{field} elements in the same position in the label. For name list fields, this means that each name substring is disambiguated from all other name substrings which occur in the same position in the name list (see examples below). This option overrides \texttt{strwidth} if both are used. The short form \opt{varwidth} is equivalent to \kvopt{varwidth}{true}. For name list fields, the \cmd{namepart}s with the \opt{pre} option set are prepended to the string returned from this disambiguation.

\boolitem[false]{varwidthnorm}

类似\texttt{varwidth},但使得\prm{field} 的可区分子字符串长度为最长的子字符串。这可以用于调整标签的格式,如果需要的话。该选项覆盖\texttt{strwidth},如果两者同时用的话。简写形式\opt{varwidthnorm} 等价于\kvopt{varwidthnorm}{true}。
%As \texttt{varwidth} but will force the disambiguated substrings for the \prm{field} to be the same length as the longest disambiguated substring. This can be used to regularise the format of the labels if desired. This option overrides \texttt{strwidth} if both are used. The short form \opt{varwidthnorm} is equivalent to \kvopt{varwidthnorm}{true}.

\boolitem[false]{varwidthlist}

当域作为一个整体能与在标签相同位置的所有其它域区分时,标签自动非歧义化的替代方法。对于非姓名列表域,这等价于\texttt{varwidth}。对于姓名列表域,姓名列表中某一相同位置的姓名无法区分,但整个列表可以区分。该选项覆盖\texttt{strwidth},如果两者同时用。简写形式\opt{varwidthlist} 等价于\kvopt{varwidthlist}{true}。对于姓名列表域,设置\opt{pre} 选项的\cmd{namepart}s将添加在从非歧义化过程返回的字符串前面。
%Alternative method of automatic label disambiguation where the field as a whole is disambiguated from all other fields in the same label position. For non-name list fields, this is equivalent to \texttt{varwidth}. For name list fields, names in a name list are not disambiguated from other names in the same position in their name lists but instead the entire name list is disambiguated as a whole from other name lists (see examples below). This option overrides \texttt{strwidth} if both are used. The short form \opt{varwidthlist} is equivalent to \kvopt{varwidthlist}{true}.  For name list fields, the \cmd{namepart}s with the \opt{pre} option set are prepended to the string returned from this disambiguation.

\intitem{strwidthmax}

当使用\texttt{varwidth},该选项在可变宽度子字符串上设置一个限制(在字符数上)。
%When using \texttt{varwidth}, this option sets a limit (in number of characters) on the length of variable width substrings. This option can be used to regularise the label.

\intitem[1]{strfixedcount}

当使用\texttt{varwidthnorm},至少存在\texttt{strfixedcount} 个可区分子字符串具有相同的最大长度来促使所有的可区分字符串具有该最大长度。
%When using \texttt{varwidthnorm}, there must be at least \texttt{strfixedcount} disambiguated substrings with the same, maximal length to trigger the forcing of all disambiguated substrings to this same maximal length.

\valitem{ifnames}{range}

当一个姓名列表域具有的姓名数在\texttt{ifnames} 范围内,仅使用这一\cmd{field} 设置。这允许\cmd{labelelement} 可以有姓名长度条件(见下面的示例)。范围可以设置如下:
%Only use this \cmd{field} specification if it is a name list field with a number of names matching the \texttt{ifnames} range value. This allows a \cmd{labelelement} to be conditionalised on name length (see below). The range can specified as in the following examples:

\begin{lstlisting}[language=xml]
ifnames=3     -> Only apply to name lists containing exactly 3 names
ifnames={2-4} -> Only apply to name lists containing minimum 2 and maximum 4 names
ifnames={-3}  -> Only apply to name lists containing at most 3 names
ifnames={2-}  -> Only apply to name lists containing at least 2 names
\end{lstlisting}

\valitem{names}{range}

默认情况下,对于姓名列表域,使用姓名从第一个姓名开始到\cnt{maxalphanames}\slash \cnt{minalphanames} 截止。该选项可以覆盖该设置为一个显式的姓名范围。加号<+>是一个特殊范围终止标记代表max/minalphanames的截断点。范围分隔符可以是任意的数字字符和统一码(unicode)破折号。例如:
%By default, for name list fields, the names used range from the first name to the \cnt{maxalphanames}\slash \cnt{minalphanames} truncation. This option can be used to override this with an explicit range of names to consider. The plus <+> sign is a special end of range marker denoting the truncation point of max/minalphanames. The range separator can be any number of characters with the Unicode Dash property. For example:

\begin{lstlisting}[language=xml]
name=3     -> Use first 3 names in the name list
name={2-3} -> Use second and thirds names only
name={-3}  -> Same as 1-3
name={2-}  -> Use all names starting with the second name (ignoring max/minalphanames truncation)
name={2-+} -> Use all names starting with the second name (respecting max/minalphanames truncation)
\end{lstlisting}

\valitem[empty]{namessep}{string}

在姓名列表的姓名之间插入任意的分隔字符串。
%An arbitrary string separator to put between names in a namelist.

\boolitem[false]{noalphaothers}

默认情况下,当姓名列表中的姓名数大于标签中显示的姓名数时,\cmd{labelalphaothers} 附加在由姓名列表得到的标签部分后面。该选项可以关闭这一默认方式。
%By default, \cmd{labelalphaothers} is appended to label parts derived from name lists if there are more names in the list than are shown in the label part. This option can be used to disable the default behaviour.

\end{optionlist*}

\cmditem{literal}{characters}

在标签的当前位置插入文本\prm{characters}。
%Insert the literal \prm{characters} into the label at this point.

\end{ltxsyntax}
%
当一个姓名列表域已经给出,从中抽取字符串的方法由一个独立模板给出,该模板由如下命令设置:
%When a name list \cmd{field} is specified, the method of extracting the string is specified by a separate template specified by the following command:

\begin{ltxsyntax}

\cmditem{DeclareLabelalphaNameTemplate}[name]{specification}


%Defines the \opt{labelalphaname} template \prm{name}. The \prm{name} is optional and defaults to \prm{<global>}.
%Such templates specify how to extract a label string from a name list when a \cmd{field} specification in \cmd{DeclareLabelalphaTemplate} contains a name list.

定义\opt{labelalphaname}模板\prm{name}。\prm{name}是可选的默认是\prm{<global>}。
当\cmd{DeclareLabelalphaTemplate}的\cmd{field}设置中包含一个姓名列表时,这些模板可以确定从姓名列表中抽取标签字符串的方式。

%老版本信息
%\cmditem{DeclareLabelalphaNameTemplate}[entrytype, \dots]{specification}
%当\cmd{DeclareLabelalphaTemplate} 中的一个\cmd{field} 设置包含有一个姓名列表时,由该命令指出从姓名列表中抽取标签字符串的模板。该模板可以根据具体的条目类型定义。

\cmditem{namepart}[options]{namepart}

\prm{namepart} 是由\cmd{DeclareDatamodelConstant} 命令定义的数据模型的姓名成分之一(见\secref{aut:bbx:drv} 节)。选项有:
%\prm{namepart} is one of the datamodel nameparts defined with the \cmd{DeclareDatamodelConstant} command (see \secref{aut:bbx:drv}). The \opt{options} are:

\begin{optionlist*}

\boolitem[false]{use}

当存在一个相应的选项\opt{use<namepart>}(比如\opt{useprefix})并且值为true,仅使用\prm{namepart} 用于构建标签信息。
%Only use the \prm{namepart} in constructing the label information if there is a corresponding option \opt{use<namepart>} and that option is true.

\boolitem[false]{pre}

当从姓名构建标签字符串时,无\opt{pre} 选项的\cmd{namepart} 将用于构建标签字符串,通过非歧义化处理过程,实施由\cmd{DeclareLabelalpaTemplate} 中的\cmd{field} 选项指定的子字符串操作。然后有\opt{pre} 选项的\cmd{namepart} 将会添加在结果前面(当存在多个这种\cmd{namepart} 时,按给出的顺序处理)。这就允许在姓名标签字符串前面无条件的添加一定的姓名成分信息,比如尊称。注意: \cmd{DeclareLabelalphaTemplate} 中\cmd{field} 的\opt{uppercase} 和\opt{lowercase} 选项应用于从\cmd{DeclareLabelalphaTemplate} 返回的整个标签,无论它是不是带\opt{pre} 选项的\cmd{namepart}。
%When constructing label strings from names, the \cmd{namepart} \emph{without} a \opt{pre} option will be used to construct label string, passing through disambiguation, substring etc. operations as specified by the \cmd{field} options in \cmd{DeclareLabelalpaTemplate}. Then the \cmd{namepart} options \emph{with} the \opt{pre} option set will be prepended to the result, (in the order given, if there are more than one such \cmd{namepart}s). This allows to unconditionally prepend certain namepart information to name label strings, like name prefices. Note that the \opt{uppercase} and \opt{lowercase} options of \cmd{field} in \cmd{DeclareLabelalphaTemplate} are applied to the entire label returned from \cmd{DeclareLabelalphaTemplate}, both \opt{pre} parts and non \opt{pre}.

\boolitem[false]{compound}

对于\cmd{DeclareLabelalphaTemplate} 中静态(非varwidth)区分方法,由空格或连字符(混合姓名)分隔的姓名成分在标签生成中将作为独立姓名成分。这意味着从如<Ballam Forsyth> 这样的姓中形成标签时,当用1个字符,左侧开始抽取子字符串,当\kvopt{compound}{true} 时,该姓名会给出<BF>,而\kvopt{compound}{false} 时会给出<B>。简写形式\opt{compound} 等价于\kvopt{compound}{true}。
%For static (non-varwidth) disambiguation in \cmd{DeclareLabelalphaTemplate}, nameparts separated by whitespace or hyphens (compound names) as separate names for label generation. This means that when forming a label out of, for example the surname <Ballam Forsyth> with a 1 character, left-side substring, this name would give <BF> with \kvopt{compound}{true} and <B> with \kvopt{compound}{false}. The short form \opt{compound} is equivalent to \kvopt{compound}{true}.

\intitem[1]{strwidth}

使用的\prm{namepart} 的字符数。
%The number of characters of the \prm{namepart} to use.

\choitem[left]{strside}{left, right}

获取\texttt{strwidth} 数量字符开始的方向。
%The side of the string from which to take the \texttt{strwidth} number of characters.

\end{optionlist*}

\end{ltxsyntax}

注意标签模板可以根据具体类型分别定义,当使用自动标签非歧义化功能时应注意这一点。非歧义化是不分类型的,因此不同类型的不同标签格式在各自非歧义化过程中可能产生歧义。一般情况,需要针对不同类型使用非常不同的标签格式,使得标签的类型可以明确。
%Note that the templates for labels can be defined per-type and you should be aware of this when using the automatically disambiguated label functionality. Disambiguation is not per-type as this might lead to ambiguity due to different label formats for different types being isolated from each others disambiguation process. Normally, you will want to use very different label formats for different types to make the type obvious by the label.

下面是一些示例。\biblatex 字母顺序标签默认模板全局定义如下。首先\bibfield{shorthand} 具有\kvopt{final}{true},因此当存在\bibfield{shorthand} 域时,它将作为标签,模板的其它内容不再考虑。接着,\bibfield{label} 作为第一个标签元素,如果它存在的话。另外,如果\bibfield{labelname} 列表中仅有一个姓名(\kvopt{ifnames}{1}),那么\bibfield{labelname} 中从姓的左侧取3个字符作为第一个标签元素。如果\bibfield{labelname} 具有超过1个的姓名,从每个名字的姓的左侧取1个字符作为第一个标签元素。第二个标签元素包含从\bibfield{year} 域的右侧取的两个字符。
%Here are some examples. The default global \biblatex alphabetic label template is defined below. Firstly, \bibfield{shorthand} has \kvopt{final}{true} and so if there is a \bibfield{shorthand} field, it is used as the label and nothing more of the template is considered. Next, the \bibfield{label} field is used as the first label element if it exists. Otherwise, if there is only one name (\kvopt{ifnames}{1}) in the \bibfield{labelname} list, then three characters from the left side of the family name in the \bibfield{labelname} are used as the first label element. If the \bibfield{labelname} has more than one name in it, one character from the left side of each family name is used as the first label element. The second label element consists of 2 characters from the right side of the \bibfield{year} field.

示例中也给出了姓名构建标签的默认模板。该模板将任何尊称(如果\opt{useprefix} 选项为true)左侧的第一个字符添加到由姓(根据调用来自\cmd{DeclareLabelalphaTemplate} 的\cmd{field} 的选项实现)中抽取的标签之前,并支持混合(compound)的姓处理。
%The default template for constructing labels from names is also shown. This prepends the first character from the left side of any prefix (if the \opt{useprefix} option is true) to a label extracted from the family name (according to the options on the calling \cmd{field} option from \cmd{DeclareLabelalphaTemplate}), allowing for compound family names.

\begin{ltxexample}
\DeclareLabelalphaTemplate{
  \labelelement{
    \field[final]{<<shorthand>>}
    \field{<<label>>}
    \field[strwidth=3,strside=left,ifnames=1]{<<labelname>>}
    \field[strwidth=1,strside=left]{<<labelname>>}
  }
  \labelelement{
    \field[strwidth=2,strside=right]{<<year>>}
  }
}

\DeclareLabelalphaNameTemplate{
  \namepart[use=true, pre=true, strwidth=1, compound=true]{prefix}
  \namepart{family}
}

\end{ltxexample}
%
要了解标签自动非歧义化(消除歧义)(非模糊化)处理的工作方式,观察如下的作者列表:
%To get an idea of how the label automatic disambiguation works, consider the following author lists:

\begin{lstlisting}{}
Agassi, Chang, Laver   (2000)
Agassi, Connors, Lendl (2001)
Agassi, Courier, Laver (2002)
Borg, Connors, Edberg  (2003)
Borg, Connors, Emerson (2004)
\end{lstlisting}
%
假设一个模板定义如下:
%Assuming a template declaration such as:

\begin{ltxexample}
\DeclareLabelalphaTemplate{
  \labelelement{
    \field[varwidth]{<<labelname>>}
  }
}
\end{ltxexample}
%
那么标签会是:
%Then the labels would be:

\begin{lstlisting}{}
Agassi, Chang, Laver    [AChLa]
Agassi, Connors, Lendl  [AConLe]
Agassi, Courier, Laver  [ACouLa]
Borg, Connors, Edberg   [BConEd]
Borg, Connors, Emerson  [BConEm]
\end{lstlisting}
%
使用规范化可变宽度标签定义:
%With normalised variable width labels defined:

\begin{ltxexample}
\DeclareLabelalphaTemplate{
  \labelelement{
    \field[varwidthnorm]{<<labelname>>}
  }
}
\end{ltxexample}
%
会得到如下结果,其中每个位置的姓名子字符串拓展为相同位置上最长子字符串的长度:
%You would get the following as the substrings of names in each position are extended to the length of the longest substring in that same position:

\begin{lstlisting}{}
Agassi, Chang, Laver    [AChaLa]
Agassi, Connors, Lendl  [AConLe]
Agassi, Courier, Laver  [ACouLa]
Borg, Connors, Edberg   [BConEd]
Borg, Connors, Emerson  [BConEm]
\end{lstlisting}
%
对标签元素的姓名成分有2个字符的限制,定义如下:
%With a restriction to two characters for the name components of the label element defined like this:

\begin{ltxexample}
\DeclareLabelalphaTemplate{
  \labelelement{
    \field[varwidthnorm,strwidthmax=2]{<<labelname>>}
  }
}
\end{ltxexample}
%
得到结果如下(注意每个姓构成的标签部分不再具有非歧义性):
%This would be the result (note that the individual family name label parts are no longer unambiguous):

\begin{lstlisting}{}
Agassi, Chang, Laver    [AChLa]
Agassi, Connors, Lendl  [ACoLe]
Agassi, Courier, Laver  [ACoLa]
Borg, Connors, Edberg   [BCoEd]
Borg, Connors, Emerson  [BCoEm]
\end{lstlisting}
%
或者,可以将姓名列表作为一个整体进行消除歧义处理的:
%Alternatively, you could choose to disambiguate the name lists as a whole with:

\begin{ltxexample}
\DeclareLabelalphaTemplate{
  \labelelement{
    \field[varwidthlist]{<<labelname>>}
  }
}
\end{ltxexample}
%
将得到:
%Which would result in:

\begin{lstlisting}{}
Agassi, Chang, Laver    [AChL]
Agassi, Connors, Lendl  [ACoL]
Agassi, Courier, Laver  [ACL]
Borg, Connors, Edberg   [BCEd]
Borg, Connors, Emerson  [BCE]
\end{lstlisting}
%
或者仅需要考虑最多两个姓名用于标签生成,但又在姓名列表层次进行非歧义化处理:
%Perhaps you only want to consider at most two names for label generation but disambiguate at the whole name list level:
\begin{ltxexample}
\DeclareLabelalphaTemplate{
  \labelelement{
    \field[varwidthlist,names=2]{<<labelname>>}
  }
}
\end{ltxexample}
%
将得到:
%Which would result in:
\begin{lstlisting}{}
Agassi, Chang, Laver    [ACh+]
Agassi, Connors, Lendl  [ACo+]
Agassi, Courier, Laver  [AC+]
Borg, Connors, Edberg   [BC+a]
Borg, Connors, Emerson  [BC+b]
\end{lstlisting}
%
在这最后一个示例中,可以看到已经添加了\cmd{labelalphaothers} 以显示存在更多姓名。最后两个标签需要利用\cmd{extraalpha} 来实现非歧义性,因为仅依靠列表中前两个姓名无法消除歧义。
%In this last example, you can see \cmd{labelalphaothers} has been appended to show that there are more names. The last two labels now require disambiguating with \cmd{extraalpha} as there is no way of disambiguating this label name list with only two names.

最后是一个使用多个标签元素的示例:
%Finally, here is an example using multiple label elements:

\begin{ltxexample}
\DeclareLabelalphaTemplate{
  \labelelement{
    \field[varwidthlist]{<<labelname>>}
  }
  \labelelement{
    \literal{-}
  }
  \labelelement{
    \field[strwidth=3,strside=right]{<<labelyear>>}
  }
}
\end{ltxexample}
%
将得到:
%Which would result in:
\begin{lstlisting}{}
Agassi, Chang, Laver    [AChL-000]
Agassi, Connors, Lendl  [AConL-001]
Agassi, Courier, Laver  [ACouL-002]
Borg, Connors, Edberg   [BCEd-003]
Borg, Connors, Emerson  [BCEm-004]
\end{lstlisting}
%
下面是精心设计的另一个示例,展示当指定衬垫字符或文本时,不需要特别的引用\latex 特殊字符(显然除了<\%>之外):
%Here is another rather contrived example showing that you don't need to specially quote \latex special characters (apart from <\%>, obviously) when specifying padding characters and literals:

\begin{ltxexample}
\DeclareLabelalphaTemplate{
  \labelelement{
    \literal{>}
  }
  \labelelement{
    \literal{\%}
  }
  \labelelement{
    \field[namessep={/}, strwidth=4, padchar=_]{<<labelname>>}
  }
  \labelelement{
    \field[strwidth=3, padchar=&, padside=left]{title}
  }
  \labelelement{
    \field[strwidth=2,strside=right]{<<year>>}
  }
}
\end{ltxexample}
%
当有:
%which given:
\begin{lstlisting}[style=bibtex]{}
@Book{test,
  author    = {XXX YY and WWW ZZ},
  title     = {T},
  year      = {2007},
}
\end{lstlisting}
将得到\footnote{译者:注意这里没有说明姓名模板使用的是哪个?所以对于该结果是有疑惑的,strwidth是应用到一个姓名还是多个姓名?}:
%would resulting a label looking like this:
\begin{verbatim}
[>%YY/ZZ__&&T07]
\end{verbatim}

当域中包含变音符(diacritic),连字符(hyphen),空格(space)等时,从域中生成标签可能存在一些困难。通常,在生成标签时需要忽略分隔符或者空格之类的字符。一个选项可以用来定制正则表达式使得在域传递给标签生成系统前能将其内容取出。
%Generating labels from fields may involve some difficulties when you have fields containing diacritics, hyphens, spaces etc. Often, you want to ignore things like separator characters or spaces when generating labels. An option is provided to customise the regular expression(s) to strip from a field before it is passed to the label generation system.

\begin{ltxsyntax}

\cmditem{DeclareNolabel}{specification}

定义正则表达式实现在生成域的标签部分前从任意域取出其内容。\prm{specification} 是一个\cmd{nolabel} 指令构成的无分隔列表,该列表指定了正则表达式用于将其匹配内容从多个域中去除。可以自由使用空格,制表符,行末符号来整理其中的代码,达到满意的代码显示效果,但空行则不允许。该命令只能用于导言中。
%Defines regular expressions to strip from any field before generating a label part for the field. The \prm{specification} is an undelimited list of \cmd{nolabel} directives which specify the regular expressions to remove from fields. Spaces, tabs and line endings may be used freely to visually arrange the \prm{specification}. Blank lines are not permissible. This command may only be used in the preamble.

\cmditem{nolabel}{regexp}

可以给出任意数量的\cmd{nolabel} 命令,用来指定去除副本域的\prm{regexp},该副本域能被标签生成系统看见。
%Any number of \cmd{nolabel} commands can be given each of which specifies to remove the \prm{regexp} from the copy of the field which the label generation system sees. Since regular expressions usually contain special characters, it is best to enclose them in the provided \cmd{regexp} macro as shown---this will pass the expression through to \biber correctly.

\end{ltxsyntax}

如果不存在\cmd{DeclareNolabel} 设置,\biber 将默认设置:
%If there is no \cmd{DeclareNolabel} specification, \biber will default to:

\begin{ltxexample}
\DeclareNolabel{
  % strip punctuation, symbols, separator and control characters
  \nolabel{\regexp{<<[\p{P}\p{S}\p{C}]+>>}}
}
\end{ltxexample}
%
这一\biber 默认设置将在向标签生成系统传递域字符串之前去除标点,符号,分隔符和控制符。(译者:注意这里正则表达式中的\cmd{p},用于匹配有命名属性的字符。)
%This \biber default strips punctuation, symbol, separator and control characters from fields before passing the field string to the label generation system.

\begin{ltxsyntax}

\cmditem{DeclareNolabelwidthcount}{specification}

当在固定宽度子字符串中统计字符数时,定义任意域中需要忽略的\prm{regexp}(即正则表达式匹配的内容)。\prm{specification} 是一个\cmd{nolabelwidthcount} 指令构成的无分隔列表,该列表指定了正则表达式用于在固定宽度子字符串中统计字符时将其匹配内容忽略。可以自由使用空格,制表符,行末符号来整理其中的代码,达到满意的代码显示效果,但空行则不允许。该命令只能用于导言中。
%Defines regular expressions to ignore from any field when counting characters in fixed-width substrings. The \prm{specification} is an undelimited list of \cmd{nolabelwidthcount} directives which specify the regular expressions to ignore when counting characters for fixed-width substrings. Spaces, tabs and line endings may be used freely to visually arrange the \prm{specification}. Blank lines are not permissible. This command may only be used in the preamble.

\cmditem{nolabelwidthcount}{regexp}

可以给出任意数量的\cmd{nolabelwidthcount} 命令,用来指定在标签生成过程中当生成固定宽度子字符串时忽略的\prm{regexp}。
%Any number of \cmd{nolabelwidthcount} commands can be given each of which specifies to ignore the \prm{regexp} when generating fixed-width substrings during label generation. Since regular expressions usually contain special characters, it is best to enclose them in the provided \cmd{regexp} macro as shown---this will pass the expression through to \biber correctly.

\end{ltxsyntax}

没有默认的\cmd{DeclareNolabelwidthcount} 设置。注意该设置只有在标签部分生成过程中使用固定宽度子字符串时考虑。见\secref{aut:ctm:lab} 节。
%There is no default \cmd{DeclareNolabelwidthcount} specification. Note that
%this setting is only taken into account when using fixed-width substrings
%(non-varwidth) during label part generation. See \secref{aut:ctm:lab}.

\subsubsection{排序}%Sorting
\label{aut:ctm:srt}

除了应用\secref{use:srt} 节讨论的预定义的排序格式外,还可以定义新的排序方式或者修改默认的格式。还可以通过根据具体的条目类型排除某些域和自动添加\bibfield{presort} 域来进一步定义。
%In addition to the predefined sorting templates discussed in \secref{use:srt}, it is possible to define new ones or modify the default definitions. The sorting process may be customized further by excluding certain fields from sorting on a per-type basis and by automatically populating the \bibfield{presort} field on a per-type basis.

\begin{ltxsyntax}

\cmditem{DeclareSortingTemplate}[options]{name}{specification}

定义排序格式\prm{name}。当选择了指定排序格式时,\prm{name} 是传递给\opt{sorting} 选项(见\secref{use:opt:pre:gen})的标识。\cmd{DeclareSortingScheme} 命令支持如下可选参数:
%Defines the sorting Template \prm{name}. The \prm{name} is the identifier passed to the \opt{sorting} option (\secref{use:opt:pre:gen}) when selecting the sorting scheme. The \cmd{DeclareSortingScheme} command supports the following optional arguments:

\begin{optionlist*}

\choitem{locale}{\prm{locale}}

排序格式中的locale用于覆盖\opt{sortlocale} 选项(在\secref{use:opt:pre:gen} 节讨论)给出的全局排序locale。
%The locale for the sorting Template which then overrides the global sorting locale in the \opt{sortlocale} option discussed in \secref{use:opt:pre:gen}.

\end{optionlist*}

\prm{specification} 是\cmd{sort} 指令的无分隔列表,这些\cmd{sort} 指令用于指定需要在排序过程中考虑的元素。可以自由使用空格,制表符,行末符来调整代码呈现格式以达到满意视觉效果,空行不允许。该命令只能用于导言中。
%The \prm{specification} is an undelimited list of \cmd{sort} directives which specify the elements to be considered in the sorting process. Spaces, tabs, and line endings may be used freely to visually arrange the \prm{specification}. Blank lines are not permissible. This command may only be used in the preamble.

\cmditem{sort}{elements}

指定在排序过程中考虑的元素。\prm{elements} 是\cmd{field} ,\cmd{literal},和 \cmd{citeorder} 命令的无分隔列表,这些命令以给定的顺序考虑。\emph{注意:} 一旦定义了一个元素,它将被添加到排序关键词中,然后排序过程跳到下一条\cmd{sort} 指令。如果该元素未定义,则考虑下一个元素。因为文本字符串总是存在,\cmd{literal} 命令应该单独或者最后出现在\cmd{sort} 指令中。所有的\prm{elements} 应该与\secref{bib:fld:dat} 节中描述数据类型一致,因为它们将会与其它条目中的\prm{elements} 进行比较。\cmd{sort} 命令支持如下可选参数:
%Specifies the elements considered in the sorting process. The \prm{elements} are an undelimited list of \cmd{field}, \cmd{literal}, and \cmd{citeorder} commands which are evaluated in the order in which they are given. If an element is defined, it is added to the sort key and the sorting routine skips to the next \cmd{sort} directive. If it is undefined, the next element is evaluated. Since literal strings are always defined, any \cmd{literal} commands should be the sole or the last element in a \cmd{sort} directive. All \prm{elements} should be the same datatype as described in \secref{bib:fld:dat} since they will be potentially compared to any of the other \prm{elements} in other entries. The \cmd{sort} command supports the following optional arguments:

\begin{optionlist*}

\choitem{locale}{\prm{locale}}

在排序元素的一个特定集层级覆盖用于排序的locale。如果给出,locale将覆盖在\cmd{DeclareSortingScheme} 层级和全局设置的locale集。另可参见\secref{use:opt:pre:gen} 节关于全局排序locale选项\opt{sortlocale} 的讨论。
%Override the locale used for sorting at the level of a particular set of sorting elements. If specified, the locale overrides the locale set at the level of \cmd{DeclareSortingScheme} and also the global setting. See also the discussion of the global sorting locale option \opt{sortlocale} in \secref{use:opt:pre:gen}.

\choitem[ascending]{direction}{ascending, descending}

排序方向,可以是\texttt{ascending} 或\texttt{descending}。默认是升序(ascending)。
%The sort direction, which may be either \texttt{ascending} or \texttt{descending}. The default is ascending order.

\boolitem[false]{final}

该选项标记一个\cmd{sort} 指令作为\prm{specification} 中的最后一个。如果\prm{elements} 中的元素存在,\prm{specification} 中剩下的内容将被忽略。简写形式\opt{final} 等价于\kvopt{final}{true}。
%This option marks a \cmd{sort} directive as the final one in the \prm{specification}. If one of the \prm{elements} is available, the remainder of the \prm{specification} will be ignored. The short form \opt{final} is equivalent to \kvopt{final}{true}.

\boolitem{sortcase}

设置排序是否大小写敏感(case"=sensitively)。默认的设置取决于全局选项\opt{sortcase}。
%Whether or not to sort case"=sensitively. The default setting depends on the global \opt{sortcase} option.

\boolitem{sortupper}

设置排序是否大写字母开头的在小写字母开头的条目之前(即<uppercase before lowercase>(\texttt{true})or <lowercase before uppercase> order (\texttt{false}))。 默认设置取决于全局选项\opt{sortupper}。
%Whether or not to sort in <uppercase before lowercase> (\texttt{true}) or <lowercase before uppercase> order (\texttt{false}). The default setting depends on the global \opt{sortupper} option.

\end{optionlist*}

\cmditem{field}[key=value, \dots]{field}

\cmd{field} 元素向排序设置中添加\prm{field}。如果\prm{field} 未定义,则跳过该元素。\cmd{field} 支持如下可选参数:
%The \cmd{field} element adds a \prm{field} to the sorting specification. If the \prm{field} is undefined, the element is skipped. The \cmd{field} command supports the following optional arguments:

\begin{optionlist*}

\choitem[left]{padside}{left, right}

使用\opt{padchar} 在一个域的\texttt{left} 或\texttt{right} 侧添加衬垫字符,使其宽度为\opt{padwidth}。如果padding选项没有设置,则不做任何衬垫。否则将执行衬垫,选项缺省将使用默认值。如果padding和子字符串匹配选项同时存在,那么子字符串匹配首先处理。
%Pads a field on the \texttt{left} or \texttt{right} side using \opt{padchar} so that its width is \opt{padwidth}. If no padding option is set, no padding is done at all. If any padding option is specified, then padding is performed and the missing options are assigned built-in default values. If padding and substring matching are both specified, the substring match is performed first.

\intitem[4]{padwidth}

所需衬垫字符目标宽度
%The target width in characters.

\valitem[0]{padchar}{character}

用于衬垫的字符。
%The character to be used when padding the field.

\choitem[left]{strside}{left, right}

从域的\texttt{left} 或\texttt{right} 侧执行子字符串匹配。匹配的字符数由相应的\texttt{strwidth} 选择指定。如果没有设置substring选项,则不做匹配。如果存在substring选项,将执行匹配,选项缺省将使用默认值。如果padding和子字符串匹配选项同时存在,那么子字符串匹配首先处理。
%Performs a substring match on the \texttt{left} or \texttt{right} side of the field. The number of characters to match is specified by the corresponding \texttt{strwidth} option. If no substring option is set, no substring matching is performed at all. If any substring option is specified, then substring matching is performed and the missing options are assigned built-in default values. If padding and substring matching are both specified, the substring match is performed first.

\intitem[4]{strwidth}

用于匹配的字符数。
%The number of characters to match.

\end{optionlist*}

\cmditem{literal}{string}

\cmd{literal} 元素添加一个文本的\prm{string} 到排序配置中。这在一些域不存在时作为一个备选使用。
%The \cmd{literal} element adds a literal \prm{string} to the sorting specification. This is useful as a fallback if some fields are not available.

\csitem{citeorder}

\cmd{citeorder} 元素有特殊意义。它要求排序基于实际标注的词汇表顺序。对于在一个标注命令中的多个条目,如:
%The \cmd{citeorder} element has a special meaning. It requests a sort based on the lexical order of the actual citations. For entries cited within the same citation command like:

\begin{ltxexample}
\cite{one,two}
\end{ltxexample}
%
词汇表顺序和语义顺序是不同的。这里的«one»和«two»有相同的语义顺序,但有不同的词汇表顺序。语义顺序只关心相同语义顺序下是否指定后续的排序设置来区分条目。比如,一个排序格式\opt{none} 的定义为:
%there is a distinction between the lexical order and the semantic order. Here «one» and «two» have the same semantic order but a unique lexical order. The semantic order only matters if you specify further sorting to disambiguate entries with the same semantic order. For example, this is the definition of the \opt{none} sorting scheme:

\begin{ltxexample}
\DeclareSortingTemplate{none}{
  \sort{\citeorder}
}
\end{ltxexample}
%
这将单纯地根据标注命令中的关键词顺序来排序文献。在上一个示例中, «one»排在«two»前面。如果假设«one»和«two»具有相同的语义顺序因为它们同时被引用,但需要根据year进一步排序,假设«two»条目的年份早于«one»:
%This sorts the bibliography purely lexically by the order of the keys in the citation commands. In the example above, it sorts «one» before «two». However, suppose that you consider «one» and «two» to have the same order (semantic order) since they are cited at the same time and want to further sort these by year. Suppose «two» has an earlier \bibfield{year} than «one»:

\begin{ltxexample}
\DeclareSortingTemplate{noneyear}{
  \sort{\citeorder}
  \sort{<<year>>}
}
\end{ltxexample}
%
这次将«two»排在«one»,尽管从词汇表上看,«one»应排在«two»前面。这是可能的,因为语义顺序可以根据年份进一步排序区分。使用标准的\opt{none} 排序格式,词汇表顺序和语义顺序是相同的,因为没有进一步的排序设置来区分。这意味着可以像使用其它元素一样来使用\cmd{citeorder},选择怎么来对同时引用(在一个标注命令中)的条目进行进一步排序。
%This sorts «two» before «one», even though lexically, «one» would sort before «two». This is possible because the semantic order can be disambiguated by the further sorting on year. With the standard \opt{none} sorting scheme, the lexical order and semantic order are identical because there is nothing further to disambiguate them. This means that you can use \cmd{citeorder} just like any other sorting specification element, choosing how to further sort entries cited at the same time (in the same citation command).

\cmditem{DeclareSortingNamekeyTemplate}[name]{specification}

定义姓名的排序关键词如何构建。这可以用来改变姓名的排序,因为在构建排序要比较的字符串时可以自由选择姓名成分来组合。如此定义的排序关键词的构建格式称为\prm{Templatename},默认是全局的(«global»),如果该可选参数缺省的话。当为姓名构建排序关键词时,构建每个姓名成分的排序关键词,并利用一个专门的内部分隔符构成一个顺序的关键词列表。这一选项的目的是兼容语言或者需要自定义姓名排序的情况(例如,冰岛人的姓名有时是用名排序而不是姓)。该宏可以多次使用为不同姓名的定义格式,以便后面使用。姓名排序关键词格式具有如下作用范围,以优先级增加的顺序包括:
%Defines how the sorting keys for names are constructed. This can change the sorting order of names arbitrarily because you can choose how to put together the name parts when constructing the string to compare when sorting. The sorting key construction scheme so defined is called \prm{Templatename} which defaults to «global» if this optional parameter is absent. When constructing the sorting key for a name, a sorting key for each name part is constructed and the key for each name is formed into an ordered key list with a special internal separator. The point of this option is to accommodate languages or situations where sorting of names needs to be customised (for example, Icelandic names are sometimes sorted by given names rather than by family names). This macro may be used multiple times to define schemes with different names which can then be referred to later. Sorting name key schemes can have the following scopes, in order of increasing precedence:

\begin{itemize}
\item 无可选姓名参数定义的默认格式。
%The default template defined without the optional name argument
\item 为一个参考文献环境(见\secref{use:bib:context})给出\opt{sortingnamekeytemplate} 选项。
%Given as the \opt{sortingnamekeytemplate} option to a reference context (see \secref{use:bib:context})
\item 在参数文献数据源条目中给出具体条目的\opt{sortnamekeytemplate} 选项。
%Given as a per-entry option \opt{sortnamekeyscheme} in a bibliography data source entry
\item 为姓名列表给出的\opt{sortnamekeytemplate} 选项。
%Given as a per-namelist option \opt{sortnamekeytemplate}
\item 为一个姓名给出的\opt{sortnamekeytemplate} 选项。
%Given as a per-name option \opt{sortnamekeytemplate}
\end{itemize}

默认情况下,仅有一个全局格式,\prm{specification} 定义如下:
%By default there is only a global template which has the following \prm{specification}:

\begin{ltxexample}
\DeclareSortingNamekeyTemplate{
  \keypart{
    \namepart[use=true]{<<prefix>>}
  }
  \keypart{
    \namepart{<<family>>}
  }
  \keypart{
    \namepart{<<given>>}
  }
  \keypart{
    \namepart{<<suffix>>}
  }
  \keypart{
    \namepart[use=false]{<<prefix>>}
  }
}
\end{ltxexample}
%
这意味着关键词由如下姓名成分按给出顺序联合构建: 前缀(尊称)(仅当\opt{useprefix} 选项为true),姓,名,后缀和前缀(仅当\opt{useprefix} 选项为false)
%This means that the key is constructed by concatenating, in order, the name prefix (only if the \opt{useprefix} option is true), the family name(s), the given names(s), the name suffix and then the name prefix (only if the \opt{useprefix} option is false).

\cmditem{keypart}{part}

\prm{part} 是一个\cmd{namepart} 和\cmd{literal} 设置构成的有序列表,这些设置联接在一起构建姓名排序关键词的一部分。
%\prm{part} is an ordered list of of \cmd{namepart} and \cmd{literal} specifications which are concatenated together when constructing a part of the name sorting key.

\cmditem{literal}{string}

插入姓名排序关键词中的文本字符串。
%A literal string to insert into the name sorting key.

\cmditem{namepart}{name}

指定姓名成分的\prm{name},用于构建姓名排序关键词。
%Specifies the \prm{name} of a namepart to use in constructing the name sorting key.

\begin{optionlist*}

\boolitem[true]{use}

表示仅当姓名成分\prm{name} 对应的\opt{use<name>} 选项设置为指定的布尔值时使用该\prm{name} 用于构建排序关键词。
%Indicates that the namepart \prm{name} is only to be used in this concatenation position if the corresponding \opt{use<name>} option is set to the specified boolean value.

\boolitem[true]{inits}

表示仅使用姓名成分\prm{name} 的首字母来构建排序关键词。
%Indicates that only the initials of namepart \prm{name} are to be used in
%constructing the sorting specification.

\end{optionlist*}

\end{ltxsyntax}

举个例子,假设用于排序的姓名成分中名要优先于姓,可以定义姓名排序关键词如下:
%As an example, suppose you wanted to be able to sort names by given name rather than family name, you could define a sorting name key scheme like this:

\begin{ltxexample}
\DeclareSortingNamekeyTemplate[givenfirst]{
  \keypart{
    \namepart{<<given>>}
  }
  \keypart{
    \namepart[use=true]{<<prefix>>}
  }
  \keypart{
    \namepart{<<family>>}
  }
  \keypart{
    \namepart[use=false]{<<prefix>>}
  }
}
\end{ltxexample}
%
然后可以在适当的范围内使用\opt{givenfirst} 来使得\biber 根据该格式构建排序关键词。比如,可以用在一个参考文献环境中使用:
%You can then use the name \opt{givenfirst} at the appropriate scope in order to make \biber use this scheme when constructing sorting name keys. For example, you could enable this for one bibliography list like this:

\begin{ltxexample}
\begin{refcontext}[sortnamekeytemplate=givenfirst]
\printbibliography
\end{refcontext}
\end{ltxexample}
%
或者在某一个特定条目中使用:
%or perhaps you only want to do this for a particular entry:

\begin{lstlisting}[style=bibtex]{}
@BOOK{key,
  OPTIONS = {sortnamekeytemplate=givenfirst},
  AUTHOR = {Arnar Vigfusson}
}
\end{lstlisting}
%
或者在一个姓名列表中使用,该选项作为一个假名会自动忽略。
%or just a name list by using the option as a pseudo-name which will be ignored:

\begin{lstlisting}[style=bibtex]{}
@BOOK{key,
  AUTHOR = {sortnamekeytemplate=givenfirst and Arnar Vigfusson}
}
\end{lstlisting}
%
或者在一个姓名中使用,通过将其作为\biber 支持的姓名信息格式的一个扩展成分传递该选项(见\biber 文档):
%or just a single name by passing the option as part of the extended name information
%format which \biber supports (see \biber doc):

\begin{lstlisting}[style=bibtex]{}
@BOOK{key,
  AUTHOR = {given=Arnar, family=Vigfusson, sortnamekeyscheme=givenfirst}
}
\end{lstlisting}
%

下面我们给出一些排序格式的示例。在第一个示例中,我们定义了一个简单的name\slash title\slash year格式。姓名元素可以是\bibfield{author},\bibfield{editor},或\bibfield{translator}。根据这一设置,排序过程将首先使用存在的第一个元素,然后继续处理\bibfield{title}。注意\opt{use$<$name$>$} 类选项在排序过程中自动处理:
%Now we give some examples of sorting templates. In the first example, we define a simple name\slash title\slash year scheme. The name element may be either the \bibfield{author}, the \bibfield{editor}, or the \bibfield{translator}. Given this specification, the sorting routine will use the first element which is available and continue with the \bibfield{title}. Note that the options \opt{use$<$name$>$} options are considered automatically in the sorting process:

\begin{ltxexample}
\DeclareSortingTemplate{sample}{
  \sort{
    \field{<<author>>}
    \field{<<editor>>}
    \field{<<translator>>}
  }
  \sort{
    \field{<<title>>}
  }
  \sort{
    \field{<<year>>}
  }
}
\end{ltxexample}
%
下一个示例中,我们以更详尽的方式定义了一个格式,考虑了\bibfield{presort}, \bibfield{sortkey}, \bibfield{sortname} 等特殊域。因为\bibfield{sortkey} 域指定了排序的主关键词,它需要覆盖所有除\bibfield{presort} 之外的元素。这通过\opt{final} 选项体现。如果\bibfield{sortkey} 域存在,排序过程将停在此处。如果不存在,排序过程将继续下一个\cmd{sort} 指令。这一设置就是\texttt{nty} 格式的默认定义。
%In the next example, we define the same template in a more elaborate way, considering special fields such as \bibfield{presort}, \bibfield{sortkey}, \bibfield{sortname}, etc. Since the \bibfield{sortkey} field specifies the master sort key, it needs to override all other elements except for \bibfield{presort}. This is indicated by the \opt{final} option. If the \bibfield{sortkey} field is available, processing will stop at this point. If not, the sorting routine continues with the next \cmd{sort} directive. This setup corresponds to the default definition of the \texttt{nty} scheme:

\begin{ltxexample}
\DeclareSortingTemplate{nty}{
  \sort{
    \field{<<presort>>}
  }
  \sort[final]{
    \field{<<sortkey>>}
  }
  \sort{
    \field{<<sortname>>}
    \field{<<author>>}
    \field{<<editor>>}
    \field{<<translator>>}
    \field{<<sorttitle>>}
    \field{<<title>>}
  }
  \sort{
    \field{<<sorttitle>>}
    \field{<<title>>}
  }
  \sort{
    \field{<<sortyear>>}
    \field{<<year>>}
  }
}
\end{ltxexample}
%
最后,给出一个覆盖全局的排序locale并在根据\bibfield{origtitle} 域排序时再次覆盖的示例。注意: 在格式层覆盖babel/polyglossia语言名的是一个真实的locale标识。\biber 会将其映射到一个合适的真实locale标识上(本例中是\texttt{sv\_SE})
%Finally, here is an example of a sorting template which overrides the global sorting locale and additionally overrides again when sorting by the \bibfield{origtitle} field. Note the use in the scheme-level override of a babel/polyglossia language name instead of a real locale identifier. \biber will map this to a suitable, real locale identifier (in this case, \texttt{sv\_SE}):

\begin{ltxexample}
\DeclareSortingTemplate[locale=swedish]{custom}{
  \sort{
    \field{<<sortname>>}
    \field{<<author>>}
    \field{<<editor>>}
    \field{<<translator>>}
    \field{<<sorttitle>>}
    \field{<<title>>}
  }
  \sort[locale=de_DE_phonebook]{
    \field{<<origtitle>>}
  }
}
\end{ltxexample}

\begin{ltxsyntax}

\cmditem{DeclareSortExclusion}{entrytype, \dots}{field, \dots}

指定排序具体类型的条目需要排除的域。\prm{entrytype} 和\prm{field} 参数可以是逗号分隔的列表。空的\prm{field} 将会清除该\prm{entrytype} 类型的所有排除设置。\prm{entrytype} 参数如果设置为<*> ,则排除的\prm{field,\dots} 针对所有类型。这等价于从排序设置中简单的删除一些域,且一般仅与\cmd{DeclareSortInclusion} 连用,特别是当希望为显式给出的条目类型排除某个域时候使用。示例见下面的\cmd{DeclareSortInclusion}。该命令只能用于导言区。
%Specifies fields to be excluded from sorting on a per-type basis. The \prm{entrytype} argument and the \prm{field} argument may be a comma"=separated list of values. A blank \prm{field} argument will clear all exclusions for this \prm{entrytype}. A value of <*> for \prm{entrytype} will exclude \prm{field,\dots} for every entrytype. This is equivalent to simply deleting the field from the sorting specification and is only normally used in combination with \cmd{DeclareSortInclusion} when one wishes to exclude a field for all but explicitly included entrytypes. See example in \cmd{DeclareSortInclusion} below. This command may only be used in the preamble.

\cmditem{DeclareSortInclusion}{entrytype, \dots}{field, \dots}

仅与\cmd{DeclareSortExclusion} 一同使用。指定具体条目类型排序需要包含的域。这使得用户可以为所有类型排序排除一个域,然后为某些特定条目类型改变这一设置。这有时比\cmd{DeclareSortExclusion} 更便于为一些条目列出排除项。\prm{entrytype} 和\prm{field} 参数可以是逗号分隔的列表。该命令只能用于导言区。例如,可以设置\bibfield{title} 在排序中仅用于\bibtype{article} 类型:
%Only used along with \cmd{DeclareSortExclusion}. Specifies fields to be included in sorting on a per-type basis. This allows the user to exclude a field from sorting for all entrytypes and then to override this for certain entrytypes. This is easier sometimes than using \cmd{DeclareSortExclusion} to list exclusions for many entrytypes. The \prm{entrytype} argument and the \prm{field} argument may be a comma"=separated list of values. This command may only be used in the preamble. For example, this would use \bibfield{title} during sorting only for \bibtype{article}s:

\begin{ltxexample}
\DeclareSortExclusion{*}{title}
\DeclareSortInclusion{article}{title}
\end{ltxexample}

\cmditem{DeclarePresort}[entrytype, \dots]{string}

定义一个字符串,用于当条目中不存在\bibfield{presort} 域时,复制该字符串到\bibfield{presort} 域。\bibfield{presort} 可以全局定义或者根据具体条目定义。如果给出可选参数\prm{entrytype},\prm{string} 用于指定的条目类型。否则作为全局的默认值。给具体条目\prm{entrytype} 指定一个空的\prm{string},将会清除原来的条目设置。而\prm{entrytype} 参数可以是一个逗号分隔的列表。该命令只能用于导言中。
%Specifies a string to be used to automatically populate the \bibfield{presort} field of entries without a \bibfield{presort} field. The \bibfield{presort} may be defined globally or on a per-type basis. If the optional \prm{entrytype} argument is given, the \prm{string} applies to the respective entry type. If not, it serves as the global default value. Specifying an \prm{entrytype} in conjunction with a blank \prm{string} will clear the type-specific setting. The \prm{entrytype} argument may be a comma"=separated list of values. This command may only be used in the preamble.

\cmditem{DeclareSortTranslit}[entrytype]{specification}

%Languages which can be written in different scripts or alphabets often only have  CLDR sorting tailoring for one script and it is expected that you transliterate into the supported script for sorting purposes. A common example is Sanskrit which is often written in academic contexts in IAST
%romanised script but which needs to be sorted in the <sa> locale which expects the Devanāgarī script. This means that it is necessary to transliterate into the sorting script internally. \cmd{DeclareSortTranslit} declares which parts of an entry you would like to transliterate for sorting purposes. Without the \prm{entrytype} parameter, the \prm{specification} applies to all entrytypes. The \prm{specification} is one or more \cmd{translit} commands:

能用不同文字或字母书写的语言对于一种文字常常仅有CLDR排序方式,为了进行排序往往需要将其转换为支持的文字。一个常见示例是梵文(Sanskrit),它常用罗马化的婆罗米系文字书写正式文献,但需要在<sa> locale中排序,而这需要Devanāgarī文字。这意味着需要在内部转换排序文字。\cmd{DeclareSortTranslit} 声明条目的那个部分需要为排序目的而转化。如果没有\prm{entrytype} 参数,\prm{specification} 用于所有条目。\prm{specification} 包含一个或多个\cmd{translit} 命令。


\cmditem{translit}[langids]{field or fieldset}{from}{to}

指定数据域\bibfield{field} 或由\cmd{DeclareDatafieldSet}(见\secref{aut:ctm:dsets})声明的\prm{fieldset} 集中的所有域为排序目的应从\prm{from} 文字转换为\prm{to} 文字。域和集参数可以是 <*> 使其应用于所有域。有效的\prm{from} 和\prm{to} 在表\ref{tab:translit} 中给出。注意\biblatex 无意于支持通用的转换,而仅做排序目的。如果需要更多的转换,可以在GitHub上的\biblatex\ 查找或询问。
%Specifies that the data field \bibfield{field} or all fields in a fieldset \prm{fieldset} declared with \cmd{DeclareDatafieldSet} (see \secref{aut:ctm:dsets}) should be transliterated from script \prm{from} to script \prm{to} for sorting purposes. The field/set argument can also be <*> to apply transliteration to all fields. The valid \prm{from} and \prm{to} values are given in table \ref{tab:translit}. Note that \biblatex does not aim to support general transliteration, only those which are useful for sorting purposes. Please open a GitHub ticket for \biblatex\ if you think you need additional transliterations.

下面给出一个标题转换的示例,从而可以正确的排序梵文:
%An example of transliterating titles so that they sort correctly in Sanskrit. This example assumes that entries that should have their title fields transliterated have a \bibfield{langid} field set to <sanskrit>.:

\begin{ltxexample}
\DeclareDatafieldSet{settitles}{
  \member[field=title]
  \member[field=booktitle]
  \member[field=eventtitle]
  \member[field=issuetitle]
  \member[field=journaltitle]
  \member[field=maintitle]
  \member[field=origtitle]
}

\DeclareSortTranslit{
  \translit[sanskrit]{settitles}{iast}{devanagari}
}
\end{ltxexample}

\end{ltxsyntax}

\begin{table}
\tablesetup\centering
\begin{tabular}{lll}
\toprule
\sffamily\bfseries\spotcolor From
  & \sffamily\bfseries\spotcolor To
  & Description\\
\midrule
iast & devanagari & Sanskrit IAST to Devanāgarī\\
russian & ala-lc & ALA-LC romanisation for Russian\\
russian & bgn/pcgn-standard & BGN/PCGN:1947 (Standard Variant), Cyrillic to Latin, Russian\\
 \bottomrule
\end{tabular}
\caption{可互相转换的文字}%Valid transliteration pairs%有效的转换文字对
\label{tab:translit}
\end{table}

\subsubsection[参考文献表过滤器]{参考文献表过滤器}%Bibliography List Filters
\label{aut:ctm:bibfilt}

当使用定制的参考文献表时(见\secref{use:bib:biblist}),常需要在\file{.bbl} 中仅写入具有特定域的条目用于在文献表中总结。比如,当打印一个缩略词的常规列表,需要让\biber 仅往\file{.bbl} 中写入一个仅包含具有\bibfield{shorthand} 域的条目的列表。这可以通过利用\cmd{DeclareBiblistFilter} 命令定义一个参考文献表过滤器来实现。这与\cmd{defbibfilter}(见\secref{use:bib:flt})定义的过滤器不同,因为\cmd{defbibfilter} 定义的过滤器在\biblatex 内运行且在\file{.bbl} 被生成之后。而且,\file{.bbl} 中的Bibliography List不包含条目数据,仅是条目的引用关键词,所有\cmd{defbibfilter} 定义的过滤器不能用于bibliography list。\footnote{译者:这里Bibliography List不是简单参考文献表,而是写入到bbl文件中某个列表,仅包含列表中各条目的关键词}
%When using customisable bibliography lists (See \secref{use:bib:biblist}), usually one wants to return in the \file{.bbl} only those entries which have the particular fields which the bibliography list is summarising. For example, when printing a normal list of shorthands, you want the list returned by \biber in the \file{.bbl} to contain only those entries which have a \bibfield{shorthand} field. This is accomplished by defining a bibliography list filter using the \cmd{DeclareBiblistFilter} command. This differs from the filters defined using \cmd{defbibfilter} (see \secref{use:bib:flt}) since the filters defined by \cmd{defbibfilter} run inside \biblatex after the \file{.bbl} has been generated. In addition, bibliography lists in the \file{.bbl} do not contain entry data, only the citation keys for the entries and so no filtering by \biblatex using \cmd{defbibfilter} is possible for bibliography lists.

\begin{ltxsyntax}
\cmditem{DeclareBiblistFilter}{name}{specification}

定义一个参考文献列表过滤器\prm{name}。\prm{specification} 由一个或多个的\cmd{filter} 或\cmd{filteror} 宏构成,对于传递给过滤器的条目所有这些宏都需要满足。
%Defines a bibliography list filter with \prm{name}. The \prm{specification} consists of one or more \cmd{filter} or \cmd{filteror} macros, all of which must be satisfied for the entry to pass the filter:

\cmditem{filter}[filterspec]{filter}

根据\prm{filterspec} 和\prm{filter} 过滤条目。\prm{filterspec} 可以是:
%Filter entries according to the \prm{filterspec} and \prm{filter}. \prm{filterspec} can be one of:

\end{ltxsyntax}

\begin{description}
\item[type/nottype] 条目是/不是\bibfield{entrytype} \prm{filter}
%Entry is/is not of \bibfield{entrytype} \prm{filter}
\item[subtype/notsubtype] 条目是/不是\bibfield{subtype} \prm{filter}
%Entry is/is not of \bibfield{subtype} \prm{filter}
\item[keyword/notkeyword] 条目具有/不具有\bibfield{keyword} \prm{filter}
%Entry has/does not have \bibfield{keyword} \prm{filter}
\item[field/notfield] 条目具有/不具有\prm{filter} 域
%Entry has/does not have a field called \prm{filter}
\end{description}

\begin{ltxsyntax}
\cmditem{filteror}{type}{filters}

在一个或多个\cmd{filter} 命令外的封套,用于指定这些\cmd{filter} 构成了一个并集,即其中任意一个\prm{filters} 都要满足。
%A wrapper around one or more \cmd{filter} commands specifying that they form a disjunctive set, i.e. any one of the \prm{filters} must be satisfied.

\end{ltxsyntax}

数据模型中标记为<Label fields>(见\secref{aut:ctm:dm})的域自动拥有为其定义的过滤器,过滤器名同域名,能将不包含该域的条目过滤掉。比如,\biblatex 自动为\bibfield{shorthand} 域生成一个过滤器:
%Fields in the datamodel which are marked as <Label fields> (see \secref{aut:ctm:dm}) automatically have a filter defined for them with the same name and which filters out any entries which do no contain the field. For example, \biblatex automatically generates a filter for the \bibfield{shorthand} field:

\begin{ltxexample}
\DeclareBiblistFilter{<<shorthand>>}{
  \filter[type=field,filter=shorthand]
}
\end{ltxexample}

\subsubsection{姓名首字母生成控制}%Controlling Name Initials Generation
\label{aut:ctm:noinit}

当姓名中存在前缀,变音符,连字符等时,要从一个给定姓名中生成各姓名成分的首字母存在一些困难。当生成首字母时,我们经常需要忽略像前缀之类的东西,比如«al-Hasan»的首字母就是«H»而不是«a-H»,但当你需要将姓名«Ho-Pun»生成首字母为«H-P»时这就变得相当棘手。
%Generating initials for name parts from a given name involves some difficulties when you have names with prefixes, diacritics, hyphens etc. Often, you want to ignore things like prefixes when generating initials so that the initials for «al-Hasan» is just «H» instead of «a-H». This is tricky when you also have names like «Ho-Pun» where you want the initials to be «H-P», for example.

\begin{ltxsyntax}

\cmditem{DeclareNoinit}{specification}

定义用于在生成首字母前从姓名剥离的匹配正则表达式。\prm{specification} 是\cmd{noinit} 指令的不分隔列表,\cmd{noinit} 指令指定从姓名中移除匹配内容的正则表达式。可以自由使用空格,制表符,行末符来调整代码呈现格式以达到满意代码显示效果,空行不允许。该命令只能用于导言中。
%Defines regular expressions to strip from names before generating initials. The \prm{specification} is an undelimited list of \cmd{noinit} directives which specify the regular expressions to remove from the name. Spaces, tabs and line endings may be used freely to visually arrange the \prm{specification}. Blank lines are not permissible. This command may only be used in the preamble.

\cmditem{noinit}{regexp}

可以给出任意数量的\cmd{noinit} 命令,每一个\cmd{noinit} 都用来指定从首字母生成系统能看到的姓名副本中去除\prm{regexp}。因为正则表达式常包含特殊字符,最好将他们用\cmd{regexp} 宏包围起来---使得能将表达式正确地传递给\biber 。
%Any number of \cmd{noinit} commands can be given each of which specifies to remove the \prm{regexp} from the copy of the name which the initials generation system sees. Since regular expressions usually contain special characters, it is best to enclose them in the provided \cmd{regexp} macro as shown---this will pass the expression through to \biber correctly.

\end{ltxsyntax}

如果没有\cmd{DeclareNoinit} 设置,\biber 采取默认方式为:
%If there is no \cmd{DeclareNoinit} specification, \biber will default to:

\begin{ltxexample}
\DeclareNoinit{
  % strip lowercase prefixes like 'al-' when generating initials from names
  \noinit{\regexp{<<\b\p{Ll}{2}\p{Pd}>>}}
  % strip some common diacritics when generating initials from names
  \noinit{\regexp{<<[\x{2bf}\x{2018}]>>}}
}
\end{ltxexample}
%
\biber 利用这一代码默认在生成首字母前从姓名中剥离小写的前缀和一对变音符。
%This \biber default strips a couple of diacritics and also strips lowercase prefixes from names before generating initials.

\subsubsection{排序微调}% Fine Tuning Sorting
\label{aut:ctm:nosort}
对排序进行微调是有用的,它可以忽略一些特殊域的某些部分。
%It can be useful to fine tune sorting so that it ignores certain parts of particular fields.

\begin{ltxsyntax}

\cmditem{DeclareNosort}{specification}

定义正则表达式用来在排序时对特定的域或者特定类型的域的内容做剥离处理。\prm{specification} 是\cmd{nosort} 指令的无分隔列表,\cmd{nosort} 指令用于定义正则表达式来移除特定域或特定类型的域中的某些内容。可以自由使用空格,制表符,行末符来调整代码呈现格式以达到满意视觉效果,空行不允许。该命令只能用于导言中。
%Defines regular expressions to strip from particular fields or types of fields when sorting. The \prm{specification} is an undelimited list of \cmd{nosort} directives which specify the regular expressions to remove from particular fields or type of field. Spaces, tabs and line endings may be used freely to visually arrange the \prm{specification}. Blank lines are not permissible. This command may only be used in the preamble.

\cmditem{nosort}{field or datafield set}{regexp}

可以给出任意数量的\cmd{nosort} 命令,这些命令将\prm{field} 或\prm{field type} 中的\prm{regexp} 移除。\prm{field type} 简单方便的构建了一个语义类似的域的集,在这些域中希望移除regexp。表\ref{aut:nosort} 给出了可用的域类型。因为正则表达式常包含特殊字符,最好将他们用\cmd{regexp} 宏包围起来---使得能将表达式正确地传递给\biber 。
%Any number of \cmd{nosort} commands can be given each of which specifies to remove the \prm{regexp} from the \prm{field} or \prm{field type}. A \prm{field type} is simple a convenience grouping of semantically similar fields from which you might want to remove a regexp. Table \ref{aut:nosort} shows the available field types. Since regular expressions usually contain special characters, it is best to enclose them in the provided \cmd{regexp} macro as shown---this will pass the expression through to \biber correctly.

\end{ltxsyntax}

默认是:
%The default is:

\begin{ltxexample}
\DeclareNosort{
  % strip prefixes like 'al-' when sorting names
  \nosort{setnames}{\regexp{<<\A\p{L}{2}\p{Pd}>>}}
  % strip some diacritics when sorting names
  \nosort{setnames}{\regexp{<<[\x{2bf}\x{2018}]>>}}
}
\end{ltxexample}
%
排序时,\biber 默认从姓名中去除一对变音符以及前缀。如果排序时需要忽略掉\bibfield{title} 域开头的«The»,那么可以设置如下:
%This \biber default strips a couple of diacritics and also strips two-letter prefixes (like «Al-») from names when sorting. Suppose you wanted to ignore «The» at the beginning of a \bibfield{title} field when sorting:

\begin{ltxexample}
\DeclareNosort{
  \nosort{<<title>>}{\regexp{<<\AThe\s+>>}}
}
\end{ltxexample}
%
或者如果想要忽略任意标题域开头的«The»:
%Or if you wanted to ignore «The» at the beginning of any title field:

\begin{ltxexample}
\DeclareNosort{
  \nosort{<<settitles>>}{\regexp{<<\AThe\s+>>}}
}
\end{ltxexample}

\begin{table}[h]
\tablesetup
\begin{tabular}{@{}V{0.5\textwidth}@{}V{0.5\textwidth}@{}}
\toprule
\multicolumn{1}{@{}H}{Field Type} &
\multicolumn{1}{@{}H}{Fields} \\
\cmidrule(r){1-1}\cmidrule{2-2}
|type_name| & |author| \\
             & |afterword| \\
             & |annotator| \\
             & |bookauthor| \\
             & |commentator| \\
             & |editor| \\
             & |editora| \\
             & |editorb| \\
             & |editorc| \\
             & |foreword| \\
             & |holder| \\
             & |introduction| \\
             & |namea| \\
             & |nameb| \\
             & |namec| \\
             & |shortauthor| \\
             & |shorteditor| \\
             & |translator| \\
|type_title| & |booktitle| \\
              & |eventtitle| \\
              & |issuetitle| \\
              & |journaltitle| \\
              & |maintitle| \\
              & |origtitle| \\
              & |title| \\
\bottomrule
\end{tabular}
\caption{\cmd{nosort} 中使用的域类型}%Field types for \cmd{nosort}
\label{aut:nosort}
\end{table}

\subsubsection{特殊域}% Special Fields
\label{aut:ctm:fld}
\secref{aut:bbx:fld:lab} 节的一些自动生成的域可以重新定制:
%Some of the automatically generated fields from \secref{aut:bbx:fld:lab} may be customized.

\begin{ltxsyntax}

\cmditem{DeclareLabelname}[entrytype, \dots]{specification}

当生成\bibfield{labelname} 域时(见\secref{aut:bbx:fld:lab}),定义要考虑的域。\prm{specification} 是\cmd{field} 命令的一个有序列表。这些域以列出的顺序进行检查,第一个有效的域作为\bibfield{labelname},默认的定义为:
%Defines the fields to consider when generating the \bibfield{labelname} field (see \secref{aut:bbx:fld:lab}). The \prm{specification} is an ordered list of \cmd{field} commands. The fields are checked in the order listed and the first field which is available will be used as \bibfield{labelname}. This is the default definition:

\begin{ltxexample}
\DeclareLabelname{%
  \field{shortauthor}
  \field{author}
  \field{shorteditor}
  \field{editor}
  \field{translator}
}
\end{ltxexample}
%
\bibfield{labelname} 域可以全局的或者根据具体条目类型定制。如果给出了可选参数\prm{entrytype},设置应用于相应的条目类型。如果没有给出,则是全局的。\prm{entrytype} 参数可以是逗号分隔的列表。该命令只能用于导言区。
%The \bibfield{labelname} field may be customized globally or on a per-type basis. If the optional \prm{entrytype} argument is given, the specification applies to the respective entry type. If not, it is applied globally. The \prm{entrytype} argument may be a comma"=separated list of values. This command may only be used in the preamble.

\cmditem{DeclareLabeldate}[entrytype, \dots]{specification}

当生成\bibfield{labelyear}, \bibfield{labelmonth}, \bibfield{labelday}, \bibfield{labelendyear}, \bibfield{labelendmonth} 和\bibfield{labelendday} 域(见\secref{aut:bbx:fld:lab})时,定义要考虑的日期成分。\prm{specification} 是\cmd{field} 或 \cmd{literal} 命令的有序列表。其中各项以列出的顺序检查,可用的第一项将复制为前述生成的域。注意: \cmd{field} 项不一定必须是数据模型中的<date>日期类型,所以可以创建假年标签,例如利用\bibfield{pubstate} 域的内容(如果存在该域),适当地作为定义\cmd{DeclareLabeldate} 的年标签。注意: 当\cmd{literal} 命令被发现时,将总被使用,所以它应该放到列表的最后。如果\cmd{literal} 命令的值是一个有效的本地化字符串,那么它将会解析为当前语言的字符串,否则将作为文本字符串照抄。默认定义如下:
%Defines the date components to consider when generating \bibfield{labelyear}, \bibfield{labelmonth}, \bibfield{labelday}, \bibfield{labelendyear}, \bibfield{labelendmonth} and \bibfield{labelendday} fields (see \secref{aut:bbx:fld:lab}). The \prm{specification} is an ordered list of \cmd{field} or \cmd{literal} commands. The items are checked in the order listed and the first item which is available will be used to popluate the mentioned fields. Note that the \cmd{field} items do not have to be datetype <date> in the data model so that you can create pseudo-year labels by, for example, using a \bibfield{pubstate} field contents, if available, as the year label by defining \cmd{DeclareLabeldate} suitably. Note also that a \cmd{literal} command will always be used when found and so this should always be the last thing in the list. If the value of a \cmd{literal} command is a valid localisation string, then this will be resolved in the current language, otherwise the value is used as a literal string as-is. This is the default definition:

\begin{ltxexample}
\DeclareLabeldate{%
  \field{date}
  \field{year}
  \field{eventdate}
  \field{origdate}
  \field{urldate}
  \literal{nodate}
}
\end{ltxexample}
%
注意: \bibfield{date} 域由后端自动分割成为\bibfield{year}, \bibfield{month}\footnote{译者: 这是需要注意的,在设计样式时判断日期域存在要用year而不是date,因为date已经自动解析分割了。},它们在默认的数据模型中也是有效域。为了支持传统的数据,比如直接设置\bibfield{year} 和/或 \bibfield{month},\cmd{DeclareLabeldate} 中的<\bibfield{date}>设置也将匹配\bibfield{year} 和\bibfield{month},如果它们存在。
%Note that the \bibfield{date} field is split by the backend into \bibfield{year}, \bibfield{month} which are also valid fields in the default data model. In order to support legacy data which directly sets \bibfield{year} and/or \bibfield{month}, the specification <\bibfield{date}> in \cmd{DeclareLabeldate} will also match \bibfield{year} and \bibfield{month} fields, if present.
\bibfield{label*} 域可以全局的或者根据具体条目类型定制。如果给出了可选参数\prm{entrytype},设置应用于相应的条目类型。如果没有给出,则是全局的。\prm{entrytype} 参数可以是逗号分隔的列表。该命令只能用于导言区。另见\secref{aut:bbx:fld:dat} 节。
%The \bibfield{label*} fields may be customized globally or on a per-type basis. If the optional \prm{entrytype} argument is given, the specification applies to the respective entry type. If not, it is applied globally. The \prm{entrytype} argument may be a comma"=separated list of values. This command may only be used in the preamble. See also \secref{aut:bbx:fld:dat}.

\cmditem{DeclareExtradate}{specification}

Defines how \biber tracks information used to construct the \bibfield{extradate} field. This field (see \secref{aut:bbx:fld:lab}) is printed to disambiguate works by the same author which occur in the same date scope. By default, the date scope is the year and so two works by the same author within the same year will have different \bibfield{extradate} values which are used to disambiguate the works in the bibliography in the usual manner seen in many authoryear type styles. The \prm{specification} is one or more \cmd{scope} specifications which can contain one or more \cmd{field} specifications. Within a \cmd{scope}, the existence of each \cmd{field} will be checked and if found, the first \cmd{field} is used and the rest are ignored. This allows a fallback in case certain fields are not available in all entries. All \cmd{scope}s are used to track information and \cmd{scope}s should be specified in decreasing order of generality (e.g. year then month then day etc) The default definition is:

\begin{ltxexample}
\DeclareExtradate{%
  \scope{
    \field{labelyear}
    \field{year}
  }
}
\end{ltxexample}
%
This means that the \bibfield{labelyear} field only (or \bibfield{year} if this does not exist) will be used to track works by the same author. With the following datasource entries:

\begin{lstlisting}[style=bibtex]{}
@BOOK{extra1,
  AUTHOR = {John Doe},
  DATE   = {2001-01}
}

@BOOK{extra2,
  AUTHOR = {John Doe},
  DATE   = {2001-02}
}
\end{lstlisting}
%
The default definition would result in:

\begin{lstlisting}{}
Doe 2001a
Doe 2001b
\end{lstlisting}
%
Here, \bibfield{extradate} only considers the \bibfield((label)year) information and since this is identical, disambiguation is required. However, consider the following definition:
\begin{ltxexample}
\DeclareExtradate{%
  \scope{
    \field{labelyear}
    \field{year}
  }
  \scope{
    \field{labelmonth}
  }
}
\end{ltxexample}
%
The result would be:
\begin{lstlisting}{}
Doe 2001
Doe 2001
\end{lstlisting}
%
If only years were printed, this would be ambiguous because \bibfield{extradate} now considers \bibfield{labelmonth} and since this differs, no disambiguation is necessary. Care should therefore be taken to synchronise the printed information with the \bibfield{extradate} disambiguation settings. Notice that the second definition is <month-in-year> disambiguation and quite different from:
\begin{ltxexample}
\DeclareExtradate{%
  \scope{
    \field{labelmonth}
  }
}
\end{ltxexample}
%
which is just plain <month> disambiguation which is very unlikely to be what you ever want to do since this disambiguation only based on month and ignores the year entirely. \bibfield{extradate} calculation should almost always be based on all information down to the resolution you require. For example, if you wish to disambiguate right down to the hour level (perhaps useful in large bibliographies of rapidly changing online material), you would specify something like this:
\begin{ltxexample}
\DeclareExtradate{%
  \scope{
    \field{labelyear}
    \field{year}
  }
  \scope{
    \field{labelmonth}
  }
  \scope{
    \field{labelday}
  }
  \scope{
    \field{labelhour}
  }
}
\end{ltxexample}
%
Entries without the specified granularity of information will disambiguate at the lowest granularity they contain, so, for example, with:
\begin{ltxexample}
\DeclareExtradate{%
  \scope{
    \field{labelyear}
    \field{year}
  }
  \scope{
    \field{labelmonth}
  }
}
\end{ltxexample}
%
\begin{lstlisting}[style=bibtex]{}
@BOOK{extra1,
  AUTHOR = {John Doe},
  DATE   = {2001}
}

@BOOK{extra2,
  AUTHOR = {John Doe},
  DATE   = {2001}
}
\end{lstlisting}
%
The result would still be:

\begin{lstlisting}{}
Doe 2001a
Doe 2001b
\end{lstlisting}
%
This command may only be used in the preamble.
\cmditem{DeclareLabeltitle}[entrytype, \dots]{specification}

定义生成\bibfield{labeltitle} 域(见\secref{aut:bbx:fld:lab})时要考虑的域。\prm{specification} 是\cmd{field} 命令的一个有序列表。这些域以给出的顺序检查,第一个有效的域作为\bibfield{labeltitle},默认的定义是:
%Defines the fields to consider when generating the \bibfield{labeltitle} field (see \secref{aut:bbx:fld:lab}). The \prm{specification} is an ordered list of \cmd{field} commands. The fields are checked in the order listed and the first field which is available will be used as \bibfield{labeltitle}. This is the default definition:

\begin{ltxexample}
\DeclareLabeltitle{%
  \field{shorttitle}
  \field{title}
}
\end{ltxexample}
%
\bibfield{labeltitle} 可以全局的或者根据具体条目类型定义。如果给出了可选参数\prm{entrytype},设置应用于相应的条目类型。如果没有给出,则是全局的。\prm{entrytype} 参数可以是逗号分隔的列表。该命令只能用于导言区。
%The \bibfield{labeltitle} field may be customized globally or on a per-type basis. If the optional \prm{entrytype} argument is given, the specification applies to the respective entry type. If not, it is applied globally. The \prm{entrytype} argument may be a comma"=separated list of values. This command may only be used in the preamble.

\end{ltxsyntax}

\subsubsection{数据继承(\bibfield{crossref})}% Data Inheritance (\bibfield{crossref})
\label{aut:ctm:ref}

\biber 提供了高度可定制的交叉引用机制和灵活的数据继承规则。本节介绍这些配置接口。默认配置见\apxref{apx:ref} 节。关于术语: \emph{child} 或\emph{target} 是具有\bibfield{crossref} 域的条目,\emph{parent} 或\emph{source} 是\bibfield{crossref} 域指向的条目。子条目从父条目继承数据:
%\biber features a highly customizable cross-referencing mechanism with flexible data inheritance rules. This sections deals with the configuration interface. See \apxref{apx:ref} for the default configuration. A note on terminology: the \emph{child} or \emph{target} is the entry with the \bibfield{crossref} field, the \emph{parent} or \emph{source} is the entry the \bibfield{crossref} field points to. The child inherits data from the parent.

\begin{ltxsyntax}

\cmditem{DefaultInheritance}[exceptions]{options}

配置默认的继承行为。该命令只能用于导言区。默认的继承行为可以通过下面的\prm{options} 定制:
%Configures the default inheritance behavior. This command may only be used in the preamble. The default behavior may be customized be setting the following \prm{options}:

\begin{optionlist*}

\boolitem[true]{all} 是否默认从父条目继承所有的域
%\boolitem[true]{all} Whether or not to inherit all fields from the parent by default.

%\kvopt{all}{true} means that the child entry inherits all fields from the parent, unless a more specific inheritance rule has been set up with \cmd{DeclareDataInheritance}. If an inheritance rule is defined for a field, data inheritance is controlled by that rule. means that no data is inherited from the parent by default and each field to be inherited requires an explicit inheritance rule set up with \cmd{DeclareDataInheritance}. The package default is \kvopt{all}{true}.

\kvopt{all}{true} 意为子条目从父条目继承所有域,除非有\cmd{DeclareDataInheritance} 设置的更具体的继承规则。如果对于一个域定义了继承规则,则数据继承由该规则控制。
\kvopt{all}{false} 意为默认不从父条目继承数据,每个域都需要根据\cmd{DeclareDataInheritance} 设定的明确继承规则来执行继承。包默认是\kvopt{all}{true}。

\boolitem[false]{override} 是否用源域覆盖目标域,当两者都存在时。该选项能作用于自动继承和显式继承两种规则。包默认是\kvopt{override}{false},即子条目存在的域都不覆盖。
%\boolitem[false]{override} Whether or not to overwrite target fields with source fields if both are defined. This applies both to automatic inheritance and to explicit inheritance rules. The package default is \kvopt{override}{false}, \ie existing fields of the child entry are not overwritten.

\valitem{ignore}{csv list of uniqueness options}

该选项的取值是一个逗号分隔的列表,由一个或多个<singletitle>, <uniquetitle>, <uniquebaretitle> 和/或 <uniquework>构成。该选项的目的是,当会触发这些信息追踪的域(表\ref{use:opt:wu})被继承时,忽略这三个选项的追踪信息。一个例子是,假设有多个\bibtype{book} 条目都引用一个\bibtype{mvbook} 条目中的\bibfield{author} 域。你可能需要一个\cmd{ifsingletitle} 判断来返回<true>,这一作者的唯一著作(<work>)是\bibtype{mvbook} 条目。类似的情况也会出现在应用\cmd{ifuniquetitle}, \cmd{ifuniquebaretitle},\cmd{ifuniquework} 判断时。\opt{ignore} 选项列出需要忽略追踪信息的判断,当继承的域会引发它们产生追踪信息时。思路是一个继承的域不参与决定参考文献数据中姓名/标题组合的唯一性。例如,下面修改的默认设置将会忽略\opt{singletitle} 和\opt{uniquetitle} 追踪:
%This option takes a comma-separated list of one of more of <singletitle>, <uniquetitle>, <uniquebaretitle> and/or <uniquework>. The purpose of this option is to ignore tracking information for these three options when the field which would trigger the tracking (\tabref{use:opt:wu}) is inherited. An example---Suppose that you have several \bibtype{book} entries which all crossref a \bibtype{mvbook} from which they get their \bibfield{author} field. You might reasonably want the \cmd{ifsingletitle} test to return <true> for this author as their only <work> is the \bibtype{mvbook}. Similar comments would apply to situations involving the \cmd{ifuniquetitle}, \cmd{ifuniquebaretitle} and \cmd{ifuniquework} tests. The \opt{ignore} option lists which of these should have their tracking information ignored when the fields which would trigger them are inherited. The idea is that the presence of an inherited field does not contribute towards the determination of whether some combination of name/title is unique in the bibliographic data. For example, this modified default setting would ignore \opt{singletitle} and \opt{uniquetitle} tracking:

\begin{ltxexample}
\DefaultInheritance{ignore={singletitle,uniquetitle}, all=true, override=false}
\end{ltxexample}
%
当然,当继承的域不参与信息追踪,追踪忽略不做任何处理。只有表\ref{use:opt:wu} 列出的域与这一选项相关。
%Of course, the ignoring of tracking does nothing if the fields inherited do not play a role in tracking. Only the fields listed in \tabref{use:opt:wu} are relevant to this option.

\end{optionlist*}

可选的\prm{exceptions} 是\cmd{except} 指令的一个不分隔列表。可以自由使用空格,制表符,行末符来调整代码呈现格式以达到满意视觉效果,空行不允许。
%The optional \prm{exceptions} are an undelimited list of \cmd{except} directives. Spaces, tabs, and line endings may be used freely to visually arrange the \prm{exceptions}. Blank lines are not permissible.

\cmditem{except}{source}{target}{options}

定义默认继承规则的例外规则。
%Defines an exception to the default inheritance rules.

\cmd{DeclareDataInheritance} 为一个具体的\prm{source} 和\prm{target} 组合设置继承选项\prm{options}。\prm{source} 和\prm{target} 参数指定了父条目和子条目。星号匹配所有的类型,可用于任一参数中。
%sets the inheritance \prm{options} for a specific \prm{source} and \prm{target} combination. The \prm{source} and \prm{target} arguments specify the parent and the child entry type. The asterisk matches all types and is permissible in either argument.

\cmditem{DeclareDataInheritance}[options]{source, \dots}{target, \dots}{rules}

定义继承规则,\prm{source} 和\prm{target} 参数指定了父条目和子条目。每个参数可以是单个条目,或者一个逗号分隔的类型列表或者星号。星号匹配所有的类型。\prm{rules} 是\cmd{inherit} 和\slash 或\cmd{noinherit} 指令的无分隔列表。可以自由使用空格,制表符,行末符来调整代码呈现格式以达到满意视觉效果,空行不允许。
%Declares inheritance rules. The \prm{source} and \prm{target} arguments specify the parent and the child entry type. Either argument may be a single entry type, a comma"=separated list of types, or an asterisk. The asterisk matches all entry types. The \prm{rules} are an undelimited list of \cmd{inherit} and\slash or \cmd{noinherit} directives. Spaces, tabs, and line endings may be used freely to visually arrange the \prm{rules}. Blank lines are not permissible. This command may only be used in the preamble. The options are:

\begin{optionlist*}

\valitem{ignore}{csv list of uniqueness options}

类似于上述\cmd{DefaultInheritance} 的\opt{ignore} 选项。当给出设置,它将高于由\cmd{DefaultInheritance} 设置的全局选项。下面示例中,当一个\bibtype{book} 条目从\bibtype{mvbook} 条目继承数据时,将忽略\opt{singletitle} 和\opt{uniquetitle} 追踪。
%As the \opt{ignore} option on \cmd{DefaultInheritance} explained above. When set here, it takes precedence over any global options set with \cmd{DefaultInheritance}. For example, this would ignore \opt{singletitle} and \opt{uniquetitle} tracking for a \bibtype{book} inheriting from a \bibtype{mvbook}.

\begin{ltxexample}
\DeclareDataInheritance[ignore={singletitle,uniquetitle}]{mvbook}{book}{<<...>>}
\end{ltxexample}

\end{optionlist*}

\cmditem{inherit}[option]{source}{target}

定义一个继承规则,通过从\prm{source} 域向\prm{target} 域映射实现,\prm{option} 可以是:
%Defines an inheritance rule by mapping a \prm{source} field to a \prm{target} field. \prm{option} can be one of

\begin{optionlist*}

\boolitem[false]{override}

类似于上述\cmd{DefaultInheritance} 中的\opt{override} 选项,当给出设置,它将高于\cmd{DefaultInheritance} 设置的全局选项。
%As the \opt{override} option for \cmd{DefaultInheritance} explained above. When set here, it takes precedence over any global options set with \cmd{DefaultInheritance}.

\end{optionlist*}

\cmditem{noinherit}{source}

无条件阻止从\prm{source} 域的继承。
%Unconditionally prevents inheritance of the \prm{source} field.

\csitem{ResetDataInheritance}

清除由\cmd{DeclareDataInheritance} 定义所有继承规则。该命令只能用于导言中。
%Clears all inheritance rules defined with \cmd{DeclareDataInheritance}. This command may only be used in the preamble.

\end{ltxsyntax}

下面是一些实际示例:
%Here are some practical examples:

\begin{ltxexample}
\DefaultInheritance{<<all=true>>,<<override=false>>}
\end{ltxexample}
%
该示例给出了怎么设置默认的继承行为。该设置是包的默认设置。
%This example shows how to configure the default inheritance behavior. The above settings are the package defaults.

\begin{ltxexample}
\DefaultInheritance[
  \except{<<*>>}{<<online>>}{<<all=false>>}
]{all=true,override=false}
\end{ltxexample}
%
该示例类似于上一示例,差别在于增加了一个例外规则:\bibtype{online} 类型的条目默认将不从任何父条目继承数据。
%This example is similar to the one above but adds one exception: entries of type \bibtype{online} will, by default, not inherit any data from any parent.

\begin{ltxexample}
\DeclareDataInheritance{<<collection>>}{<<incollection>>}{
  \inherit{<<title>>}{<<booktitle>>}
  \inherit{<<subtitle>>}{<<booksubtitle>>}
  \inherit{<<titleaddon>>}{<<booktitleaddon>>}
}
\end{ltxexample}
%
到目前为止,我们已经看到了标准的继承设置。例如\kvopt{all}{true} 意味着一个源条目的\bibfield{publisher} 域将被复制到目标条目的\bibfield{publisher} 域中。然而,在一些情况下,需要非对称的映射。它们通过\cmd{DeclareDataInheritance} 来定义。上面的示例为\bibtype{incollection} 条目引用\bibtype{collection} 信息设置了3条典型规则。将源条目的\bibfield{title} 及其相关域映射到目标条目对应的\bibfield{booktitle} 相关域中。
%So far we have looked at setting up standard inheritance. For example, \kvopt{all}{true} means that the \bibfield{publisher} field of a source entry is copied to the \bibfield{publisher} field of the target entry. In some cases, however, asymmetric mappings are required. They are defined with \cmd{DeclareDataInheritance}. The above example sets up three typical rules for \bibtype{incollection} entries referencing a \bibtype{collection}. We map the \bibfield{title} and related fields of the source to the corresponding \bibfield{booktitle} fields of the target.

\begin{ltxexample}
\DeclareDataInheritance{<<mvbook,book>>}{<<inbook,bookinbook>>}{
  \inherit{<<author>>}{<<author>>}
  \inherit{<<author>>}{<<bookauthor>>}
}
\end{ltxexample}
%
这一规则是一个一对多映射的规则: 为了能运行压缩\bibfield{inbook}\slash \bibfield{bookinbook} 条目,它将源条目的\bibfield{author} 域映射到目标条目的\bibfield{author} 和\bibfield{bookauthor} 域中。源可以是一个\bibtype{mvbook} 或\bibtype{book} 条目,目标可以是一个\bibtype{inbook} 或\bibtype{bookinbook} 条目。
%This rule is an example of one-to-many mapping: it maps the \bibfield{author} field of the source to both the \bibfield{author} and the \bibfield{bookauthor} fields of the target in order to allow for compact \bibfield{inbook}\slash \bibfield{bookinbook} entries. The source may be either a \bibtype{mvbook} or a \bibtype{book} entry, the target either an \bibtype{inbook} or a \bibtype{bookinbook} entry.

\begin{ltxexample}
\DeclareDataInheritance{<<*>>}{<<inbook,incollection>>}{
  \noinherit{<<introduction>>}
}
\end{ltxexample}
%
这一规则阻止对\bibfield{introduction} 域的继承。应用的目标条目是\bibtype{inbook} 或 \bibtype{incollection},源条目则是任意的。
%This rule prevents inheritance of the \bibfield{introduction} field. It applies to all targets of type
%\bibtype{inbook} or \bibtype{incollection}, regardless of the source entry type.

\begin{ltxexample}
\DeclareDataInheritance{<<*>>}{<<*>>}{
  \noinherit{<<abstract>>}
}
\end{ltxexample}
%
该规则应用于所有条目类型,阻止\bibfield{abstract} 域的继承。
%This rule, which applies to all entries, regardless of the source and target entry types, prevents inheritance of the \bibfield{abstract} field.

\begin{ltxexample}
\DefaultInheritance{all=true,override=false}
\ResetDataInheritance
\end{ltxexample}
%
该例展示怎么模拟实现传统的\bibtex 的交叉引用机制。它默认启用继承功能,禁止覆盖,并清除所有的其它规则和映射。
%This example demonstrates how to emulate traditional \bibtex's cross"=referencing mechanism. It enables inheritance by default, disables overwriting, and clears all other inheritance rules and mappings.

在一个参考文献条目中,当值是由\cmd{DeclareDatafieldSet}(\secref{aut:ctm:dsets})定义的数据域的集合,可以给出一个<noinherit>选项。这会阻止具体条目对在该集中的域的继承。例如:
%In a bibliography entry, you can give an option <noinherit> where the value
%is a datafield set defined with \cmd{DeclareDatafieldSet}
%(\secref{aut:ctm:dsets}). This will block inheritance of the fields in the
%set on a per-entry basis. For example:

\begin{ltxexample}
\DeclareDatafieldSet{nobtitle}{
  \member[field=booktitle]
}
\end{ltxexample}

\begin{lstlisting}[style=bibtex]{}
@INBOOK{s1,
  OPTIONS  = {noinherit=nobtitle},
  TITLE    = {Subtitle},
  CROSSREF = {s2}
}

@BOOK{s2,
  TITLE = {Title}
}
\end{lstlisting}
%
这里\bibfield{s2} 的\bibfield{TITLE} 域将不会被继承为 \bibfield{s1} 的\bibfield{BOOKTITLE},这一继承之所以被阻止是因为\opt{noinherit} 选项的值是数据域集合,且集合中包含\bibfield{BOOKTITLE}域。
%Here, \bibfield{s1} will not inherit the \bibfield{TITLE} of \bibfield{s2}
%as \bibfield{BOOKTITLE} as this is blocked by the datafield set given as
%the value to the \opt{noinherit} option.
%

需要重点注意的是,如果他们已经具有某一类型日期的某一成分,子条目不会从父条目继承该类型日期的任何成分域。例如:
%One important thing to note is that children will never inherit any dateparts of a given type if they already contain a datepart of that type. So, for example:

\begin{lstlisting}[style=bibtex]{}
@INBOOK{b1,
  DATE     = {2004-03-03},
  ORIGDATE = {2004-03},
  CROSSREF = {b2}
}

@BOOK{b2,
  DATE      = {2004-03-03/2005-08-09},
  ORIGDATE  = {2004-03/2005-08},
  EVENTDATE = {2004-03/2005-08},
}
\end{lstlisting}
%
这里,\bibfield{b1} 条目将不会继承任何的\bibfield{endyear}, \bibfield{endmonth}, \bibfield{endday}, \bibfield{origendyear} or \bibfield{origendmonth},因为这可能导致与自身日期的混乱。考虑默认继承规则,它将继承所有的\bibfield{event*} 日期成分。
%Here, \bibfield{b1} will not inherit any of \bibfield{endyear}, \bibfield{endmonth}, \bibfield{endday}, \bibfield{origendyear} or \bibfield{origendmonth} as this would make a mess of its own dates. It will, given the inheritance defaults, inherit all of the \bibfield{event*} date parts.



\subsection{辅助命令}%Auxiliary Commands
\label{aut:aux}
本节的工具用来分析和保存参考文献数据而不是对其进行格式化或者打印。
%The facilities in this section are intended for analyzing and saving bibliographic data rather than formatting and printing it.

\subsubsection{数据命令}%Data Commands
\label{aut:aux:dat}
本节的命令允许对未格式化的参考文献数据进行底层访问。这些命令不是用来输出,而是用来将数据保存到临时宏中,可以用于下一步的比较。
%The commands in this section grant low"=level access to the unformatted bibliographic data. They are not intended for typesetting but rather for things like saving data to a temporary macro so that it may be used in a comparison later.

\begin{ltxsyntax}

\cmditem{thefield}{field}

展开为未格式化的\prm{field}。如果\prm{field} 未定义那么展开为一个空字符串。
%Expands to the unformatted \prm{field}. If the \prm{field} is undefined, this command expands to an empty string.

\cmditem{strfield}{field}

类似于\cmd{thefield} 命令,但其值经自动处理(sanitized),以便安全地用于构成控制序列名。
%Similar to \cmd{thefield}, except that the field is automatically sanitized such that its value may safely be used in the formation of a control sequence name.

\cmditem{csfield}{field}

类似于\cmd{thefield} 命令,但禁止展开
%Similar to \cmd{thefield}, but prevents expansion.

\cmditem{usefield}{command}{field}

执行\prm{command} 命令使用未格式化的\prm{field} 作为其参数
%Executes \prm{command} using the unformatted \prm{field} as its argument.

\cmditem{thelist}{literal list}

%Expands to the unformatted \prm{literal list}. If the list is undefined, this command expands to an empty string. Note that this command will dump the \prm{literal list} in the internal format used by this package. This format is not suitable for printing.
展开为未格式化的\prm{literal list}。如果list未定义那么展开为一个空字符串。注意该命令中将\prm{literal list} 转存为本宏包使用的内部格式。这一格式不适合打印。

\cmditem{strlist}{literal list}

类似于\cmd{thelist},差别在于该命令能自动处理列表的内部表示,因此列表的值可以安全地用于控制序列名的构建。
%Similar to \cmd{thelist}, except that the list internal representation is automatically sanitized such that its value may safely be used in the formation of a control sequence name.

\cmditem{thename}{name list}

%Expands to the unformatted \prm{name list}. If the list is undefined, this command expands to an empty string. Note that this command will dump the \prm{name list} in the internal format used by this package. This format is not suitable for printing.
展开为未格式化的\prm{name list}。如果list未定义那么展开为一个空字符串。注意该命令中将\prm{name list} 转存为本宏包使用的内部格式。这一格式不适合打印。

\cmditem{strname}{name list}

类似于\cmd{thename},差别在于该命令能自动处理列表的内部表示,因此列表的值可以安全地用于控制序列名的构建。
%Similar to \cmd{thename}, except that the name internal representation is automatically sanitized such that its value may safely be used in the formation of a control sequence name.

\cmditem{savefield}{field}{macro}
\cmditem*{savefield*}{field}{macro}

将未格式化的\prm{field} 拷贝到一个\prm{macro} 中。不带星的命令全局的定义\prm{macro},而带星的命令是局部定义。
%Copies an unformatted \prm{field} to a \prm{macro}. The regular variant of this command defines the \prm{macro} globally, the starred one works locally.

\cmditem{savelist}{literal list}{macro}
\cmditem*{savelist*}{literal list}{macro}

将未格式化的\prm{literal list} 拷贝到一个\prm{macro} 中。不带星的命令全局的定义\prm{macro},而带星的命令是局部定义。
%Copies an unformatted \prm{literal list} to a \prm{macro}. The regular variant of this command defines the \prm{macro} globally, the starred one works locally.

\cmditem{savename}{name list}{macro}
\cmditem*{savename*}{name list}{macro}

将未格式化的\prm{name list} 拷贝到一个\prm{macro} 中。不带星的命令全局的定义\prm{macro},而带星的命令是局部定义。
%Copies an unformatted \prm{name list} to a \prm{macro}. The regular variant of this command defines the \prm{macro} globally, the starred one works locally.

\cmditem{savefieldcs}{field}{csname}
\cmditem*{savefieldcs*}{field}{csname}

类似于\cmd{savefield} 命令,但将控制序列名\prm{csname}(即没有斜杠)作为参数,而不是宏。
%Similar to \cmd{savefield}, but takes the control sequence name \prm{csname} (without a leading backslash) as an argument, rather than a macro name.

\cmditem{savelistcs}{literal list}{csname}
\cmditem*{savelistcs*}{literal list}{csname}

类似于\cmd{savelist} 命令,但将控制序列名\prm{csname}(即没有斜杠)作为参数,而不是宏。
%Similar to \cmd{savelist}, but takes the control sequence name \prm{csname} (without a leading backslash) as an argument, rather than a macro name.

\cmditem{savenamecs}{name list}{csname}
\cmditem*{savenamecs*}{name list}{csname}

类似于\cmd{savename} 命令,但将控制序列名\prm{csname}(即没有斜杠)作为参数,而不是宏。
%Similar to \cmd{savename}, but takes the control sequence name \prm{csname} (without a leading backslash) as an argument, rather than a macro name.

\cmditem{restorefield}{field}{macro}

从之前用\cmd{savefield} 命令定义的\prm{macro} 中将\prm{field} 恢复回来。该域是在局部范围内恢复。
%Restores a \prm{field} from a \prm{macro} defined with \cmd{savefield} before. The field is restored within a local scope.

\cmditem{restorelist}{literal list}{macro}

从之前用\cmd{savelist} 命令定义的\prm{macro} 中将\prm{literal list} 恢复回来。该list是在局部范围内恢复。
%Restores a \prm{literal list} from a \prm{macro} defined with \cmd{savelist} before. The list is restored within a local scope.

\cmditem{restorename}{name list}{macro}

从之前用\cmd{savename} 命令定义的\prm{macro} 中将\prm{name list} 恢复回来。该list是在局部范围内恢复。
%Restores a \prm{name list} from a \prm{macro} defined with \cmd{savename} before. The list is restored within a local scope.

\cmditem{clearfield}{field}

在局部范围内清除\prm{field}。以这种方式清除的域对于后续的数据命令来说相当于没有定义。
%Clears the \prm{field} within a local scope. A field cleared this way is treated as undefined by subsequent data commands.

\cmditem{clearlist}{literal list}

在局部范围内清除\prm{literal list}。以这种方式清除的list对于后续的数据命令来说相当于没有定义。
%Clears the \prm{literal list} within a local scope. A list cleared this way is treated as undefined by subsequent data commands.

\cmditem{clearname}{name list}

在局部范围内清除\prm{name list}。以这种方式清除的list对于后续的数据命令来说相当于没有定义。
%Clears the \prm{name list} within a local scope. A list cleared this way is treated as undefined by subsequent data commands.

\end{ltxsyntax}

\subsubsection{独立判断命令}%Stand-alone Tests
\label{aut:aux:tst}
本节的命令是不同类型的独立(stand-alone)判断命令,用于参考文献著录和标注样式中。
%The commands in this section are various kinds of stand"=alone tests for use in bibliography and citation styles.

\begin{ltxsyntax}

\cmditem{if$<$datetype$>$julian}{true}{false}

当日期<datetype>date因为\opt{julian} 和\opt{gregorianstart} 选项的设置转换为儒略历(Julian Calendar) 时,展开为\prm{true}。
%Expands to \prm{true} if the date <datetype>date (\opt{date}, \opt{urldate}, \opt{eventdate} etc.) Was converted to the Julian Calendar due to the settings of the \opt{julian}and \opt{gregorianstart}  options.

\cmditem{ifdatejulian}{true}{false}

类似于\cmd{if$<$datetype$>$julian} 但用于\cmd{mkbibdate*} 格式化命令中(\secref{aut:fmt:lng}),在这些格式化命令中恰当使用\cmd{if$<$datetype$>$julian} 命令等价于应用该命令。
%As \cmd{if$<$datetype$>$julian} but for use in \cmd{mkbibdate*} formatting commands (\secref{aut:fmt:lng}) inside which the appropriate \cmd{if$<$datetype$>$julian} command is aliased to this command.

\cmditem{if$<$datetype$>$dateera}{era}{true}{false}

当日期<datetype>date(\opt{date}, \opt{urldate}, \opt{eventdate} 等)指定了一个纪元等于\prm{era},则展开为\prm{true},否则展开为\prm{false}。\biber 确认并在\file{.bbl} 文件中传递的可用\prm{era} 字符串是:
%Expands to \prm{true} if the date <datetype>date (\opt{date}, \opt{urldate}, \opt{eventdate} etc.) has an era specification equal to \prm{era} and \prm{false} otherwise.  The supported \prm{era} strings which \biber determines and passes in the \file{.bbl} are:

\begin{description}
\item[bce]  BCE/BC era
\item[ce]  CE/AD era
\end{description}

该命令用于确定是否打印\secref{aut:lng:key:dt} 节的本地化字符串\footnote{译者:这里的location strings 应是笔误,而应是local strings,这从\secref{aut:lng:key:dt} 节内容可以看出}。
%This command is useful for determining whether to print the location
%strings in \secref{aut:lng:key:dt}.

\cmditem{ifdateera}{era}{true}{false}

类似于\cmd{if$<$datetype$>$dateera},但用于\cmd{mkbibdate*} 格式化命令(\secref{aut:fmt:lng}),在这些格式化命令中恰当使用\cmd{if$<$datetype$>$dateera} 命令等价于应用该命令。
%As \cmd{if$<$datetype$>$dateera} but for use in \cmd{mkbibdate*} formatting commands (\secref{aut:fmt:lng}) inside which the appropriate \cmd{if$<$datetype$>$dateera} command is aliased to this command.

\cmditem{if$<$datetype$>$datecirca}{true}{false}

当日期<datetype>date(\opt{date}, \opt{urldate}, \opt{eventdate} 等)在数据源中具有一个<circa>标记时,则展开为\prm{true},否则展开为\prm{false}。参见\secref{bib:use:dat}。该命令用于确定是否打印\secref{aut:lng:key:dt} 节中的本地化字符串。
%Expands to \prm{true} if the date <datetype>date (\opt{date}, \opt{urldate}, \opt{eventdate} etc.) had a <circa> marker in the source and \prm{false} otherwise.  See \secref{bib:use:dat}. This command is useful for determining whether to print the location strings in \secref{aut:lng:key:dt}.

\cmditem{ifdatecirca}{true}{false}

类似于\cmd{if$<$datetype$>$datecirca},但用于\cmd{mkbibdate*} 格式化命令(\secref{aut:fmt:lng}),在这些格式化命令中恰当使用的\cmd{if$<$datetype$>$datecirca} 命令等价于该命令。
%As \cmd{if$<$datetype$>$datecirca} but for use in \cmd{mkbibdate*} formatting commands (\secref{aut:fmt:lng}) inside which the appropriate \cmd{if$<$datetype$>$datecirca} command is aliased to this command.

\cmditem{if$<$datetype$>$dateuncertain}{true}{false}

当日期<datetype>date(\opt{date}, \opt{urldate}, \opt{eventdate} 等)在数据源中具有一个不确定标记时,则展开为\prm{true},否则展开为\prm{false}。参见\secref{bib:use:dat}。该命令用于确定是否打印例如年份后的一个问号。
%Expands to \prm{true} if the date <datetype>date (\opt{date}, \opt{urldate}, \opt{eventdate} etc.) had an uncertainty marker in the source and \prm{false} otherwise.  See \secref{bib:use:dat}. This command is useful for determining whether to print, for example, a question mark after a year.

\cmditem{ifdateuncertain}{true}{false}

类似于\cmd{if$<$datetype$>$dateuncertain},但用于\cmd{mkbibdate*} 格式化命令(\secref{aut:fmt:lng}),在这些格式化命令中恰当使用\cmd{if$<$datetype$>$dateuncertain} 命令等价于应用该命令。
%As \cmd{if$<$datetype$>$dateuncertain} but for use in \cmd{mkbibdate*} formatting commands (\secref{aut:fmt:lng}) inside which the appropriate \cmd{if$<$datetype$>$dateuncertain} command is aliased to this command.

\cmditem{ifenddateuncertain}{true}{false}

类似于\cmd{ifend$<$datetype$>$dateuncertain},但用于\cmd{mkbibdate*} 格式化命令(\secref{aut:fmt:lng}),在这些格式化命令中恰当使用\cmd{ifend$<$datetype$>$dateuncertain} 命令等价于应用该命令。
%As \cmd{ifend$<$datetype$>$dateuncertain} but for use in \cmd{mkbibdate*} formatting commands (\secref{aut:fmt:lng}) inside which the appropriate \cmd{ifend$<$datetype$>$dateuncertain} command is aliased to this command.
\cmditem{if$<$datetype$>$dateunknown}{true}{false}

%Expands to \prm{true} if the date <datetype>date (\opt{date}, \opt{urldate}, \opt{eventdate} etc.) is marked as unknown (as opposed to open) in the source and \prm{false} otherwise.  See \secref{bib:use:dat}.
展开为\prm{true},如果数据源中日期<datetype>date (\opt{date}, \opt{urldate}, \opt{eventdate} etc.) 标记为未知(与未终止(open)不同),
否则展开为\prm{false},见 \secref{bib:use:dat}节。


\cmditem{ifdateunknown}{true}{false}

%As \cmd{if$<$datetype$>$dateunknown} but for use in \cmd{mkbibdate*} formatting commands (\secref{aut:fmt:lng}) inside which the appropriate \cmd{if$<$datetype$>$dateunknown} command is aliased to this command.
类似于\cmd{if$<$datetype$>$dateunknown},但用在\cmd{mkbibdate*} 格式化命令中(见\secref{aut:fmt:lng}),
在这类命令中类似的\cmd{if$<$datetype$>$dateunknown}命令逻辑与\cmd{ifdateunknown}一致。


\cmditem{ifenddateunknown}{true}{false}

%As \cmd{ifend$<$datetype$>$dateunknown} but for use in \cmd{mkbibdate*} formatting commands (\secref{aut:fmt:lng}) inside which the appropriate \cmd{ifend$<$datetype$>$dateunknown} command is aliased to this command.
类似于\cmd{ifend$<$datetype$>$dateunknown},但用在\cmd{mkbibdate*} 格式化命令中(见\secref{aut:fmt:lng}),
在这类命令中类似的\cmd{ifend$<$datetype$>$dateunknown}命令逻辑与\cmd{ifenddateunknown}一致。


\cmditem{iflabeldateisdate}{true}{false}

%Expands to \prm{true} if labeldate is defined and was obtained from date, and to \prm{false} otherwise.
展开为\prm{true}如果labeldate已定义且从date中获取,否则为\prm{false}



\cmditem{ifdatehasyearonlyprecision}{datetype}{true}{false}

%Expands to \prm{true} if the \prm{datetype}date is defined and would be shown with year precision \cmd{print$<$datetype$>$date}, and to false otherwise.
展开为\prm{true},如果\prm{datetype}date已定义,且将在\cmd{print$<$datetype$>$date}命令中以明确年份显示,否则为\prm{false}。


\cmditem{ifdatehastime}{datetype}{true}{false}

%Expands to \prm{true} if the \prm{datetype}date is defined, has a time component and \opt{$<$datetype$>$dateusetime} is true, and to false otherwise.
展开为\prm{true},如果\prm{datetype}date已定义,且包含时间成分,且\opt{$<$datetype$>$dateusetime}为\prm{true},否则为\prm{false}。


\cmditem{ifdateshavedifferentprecision}{datetype1}{datetype2}{true}{false}

%Expands to \prm{true} if the two dates \prm{datetype1} and \prm{datetype2} would show in different precision when printed with \cmd{print$<$datetype1$>$date} and \cmd{print$<$datetype2$>$date} respectively, and to \prm{false} otherwise.
展开为\prm{true},如果两个日期\prm{datetype1}和\prm{datetype2}以不同精度在\cmd{print$<$datetype1$>$date}和
\cmd{print$<$datetype2$>$date}命令中分别打印时,否则展开为\prm{false}。


\cmditem{ifdateyearsequal}{datetype1}{datetype2}{true}{false}

%Expands to \prm{true} if the two dates \prm{datetype1} and \prm{datetype2} have the same year and era. Since the sign of the date is saved in the era field, years should be compared using this command to avoid confusion when the two years have opposite signs

展开为\prm{true},如果两个日期\prm{datetype1}和\prm{datetype2}具有相同的年份和纪年。
因为日期纪年的符号保存在era域中,可以利用该命令来避免混淆,当两个年份具有相反纪年时,比如公元前和公元后。


\cmditem{ifcaselang}[language]{true}{false}

如果可选的\prm{language} 是\cmd{DeclareCaseLangs}(见\secref{aut:aux:msc})声明的语言之一,展开为\prm{true},否则展开为\prm{false}。当可选参数不给出时,对\cmd{currentlang} 值进行判断。
%Expands to \prm{true} if the the optional \prm{language} is one of those
%declared by \cmd{DeclareCaseLangs} (see \secref{aut:aux:msc}) and to
%\prm{false} otherwise. Without the optional argument, checks the current
%value of \cmd{currentlang}.

\cmditem{ifsortingnamekeytemplatename}{string}{true}{false}

如果\prm{string} 与当前作用范围内姓名排序关键词格式名(见\ref{aut:ctm:srt})相同则展开为\prm{true},否则展开为\prm{false}。
%Expands to \prm{true} if the \prm{string} is equal to the current in scope sorting name key template name (see \ref{aut:ctm:srt}), and to \prm{false} otherwise.

\cmditem{ifuniquenametemplatename}{string}{true}{false}

%Expands to \prm{true} if the \prm{string} is equal to the current in scope uniqueness name key template name (see \ref{aut:ctm:srt}), and to \prm{false} otherwise.
展开为\prm{true},如果\prm{string}等于当前范围内的唯一姓名关键字模板名(见\ref{aut:ctm:srt}),否则展开为\prm{false}。


\cmditem{iflabelalphanametemplatename}{string}{true}{false}

%Expands to \prm{true} if the \prm{string} is equal to the current in scope alphabetic label name template name (see \ref{aut:ctm:srt}), and to \prm{false} otherwise.
展开为\prm{true},如果\prm{string}等于当前范围内的唯一标签姓名模板名(见\ref{aut:ctm:srt}),否则展开为\prm{false}。

\cmditem{iffieldundef}{field}{true}{false}

展开为\prm{true},如果\prm{field} 未定义,否则展开为\prm{false}
%Expands to \prm{true} if the \prm{field} is undefined, and to \prm{false} otherwise.

\cmditem{iflistundef}{literal list}{true}{false}

展开为\prm{true},如果\prm{literal list} 未定义,否则展开为\prm{false}
%Expands to \prm{true} if the \prm{literal list} is undefined, and to \prm{false} otherwise.

\cmditem{ifnameundef}{name list}{true}{false}

展开为\prm{true},如果\prm{name list} 未定义,否则展开为\prm{false}
%Expands to \prm{true} if the \prm{name list} is undefined, and to \prm{false} otherwise.

\cmditem{iffieldsequal}{field 1}{field 2}{true}{false}

展开为\prm{true},如果\prm{field 1} 和\prm{field 2} 相等,否则展开为\prm{false}
%Expands to \prm{true} if the values of \prm{field 1} and \prm{field 2} are equal, and to \prm{false} otherwise.

\cmditem{iflistsequal}{literal list 1}{literal list 2}{true}{false}

展开为\prm{true},如果\prm{literal list 1} 和\prm{literal list 2} 相等,否则展开为\prm{false}
%Expands to \prm{true} if the values of \prm{literal list 1} and \prm{literal list 2} are equal, and to \prm{false} otherwise.

\cmditem{ifnamesequal}{name list 1}{name list 2}{true}{false}

展开为\prm{true},如果\prm{name list 1} 和\prm{name list 2} 相等,否则展开为\prm{false}
%Expands to \prm{true} if the values of \prm{name list 1} and \prm{name list 2} are equal, and to \prm{false} otherwise.

\cmditem{iffieldequals}{field}{macro}{true}{false}

展开为\prm{true},如果\prm{field} 的值和\prm{macro} 的定义相等,否则展开为\prm{false}。\footnote{译者: 比如应用于gb7714-2015中的新闻和标准条目类型的判断}
%Expands to \prm{true} if the value of the \prm{field} is equal to the definition of \prm{macro}, and to \prm{false} otherwise.

\cmditem{iflistequals}{literal list}{macro}{true}{false}

展开为\prm{true},如果\prm{literal list} 的值和\prm{macro} 的定义相等,否则展开为\prm{false}。
%Expands to \prm{true} if the value of the \prm{literal list} is equal to the definition of \prm{macro}, and to \prm{false} otherwise.

\cmditem{ifnameequals}{name list}{macro}{true}{false}

展开为\prm{true},如果\prm{name list} 的值和\prm{macro} 的定义相等,否则展开为\prm{false}。
%Expands to \prm{true} if the value of the \prm{name list} is equal to the definition of \prm{macro}, and to \prm{false} otherwise.

\cmditem{iffieldequalcs}{field}{csname}{true}{false}

类似于\cmd{iffieldequals},但将控制序列名\prm{csname}(不带斜杠)作为参数,而不是一个宏名。
%Similar to \cmd{iffieldequals} but takes the control sequence name \prm{csname} (without a leading backslash) as an argument, rather than a macro name.

\cmditem{iflistequalcs}{literal list}{csname}{true}{false}

类似于\cmd{iflistequals},但将控制序列名\prm{csname}(不带斜杠)作为参数,而不是一个宏名。
%Similar to \cmd{iflistequals} but takes the control sequence name \prm{csname} (without a leading backslash) as an argument, rather than a macro name.

\cmditem{ifnameequalcs}{name list}{csname}{true}{false}

类似于\cmd{ifnameequals},但将控制序列名\prm{csname}(不带斜杠)作为参数,而不是一个宏名。
%Similar to \cmd{ifnameequals} but takes the control sequence name \prm{csname} (without a leading backslash) as an argument, rather than a macro name.

\cmditem{iffieldequalstr}{field}{string}{true}{false}

展开为\prm{true},如果\prm{field} 的值和字符串\prm{string} 的定义相等,否则展开为\prm{false}。该命令是鲁棒的。
%Executes \prm{true} if the value of the \prm{field} is equal to \prm{string}, and \prm{false} otherwise. This command is robust.

\cmditem{iffieldxref}{field}{true}{false}

如果一个条目定义了\bibfield{crossref}\slash \bibfield{xref},该命令检测\prm{field} 是否与cross"=referenced父条目相关联。如果子条目的\prm{field} 与父条目对应的\prm{field} 相等,那么执行\prm{true},否则执行\prm{false}。如果\bibfield{crossref}\slash \bibfield{xref} 未定义,总是执行\prm{false}。该命令是鲁棒的。\bibfield{crossref} 和 \bibfield{xref} 域的描述见\secref{bib:fld:spc},更多关于cross"=referencing的信息见\secref{bib:cav:ref}。
%If the \bibfield{crossref}\slash \bibfield{xref} field of an entry is defined, this command checks if the \prm{field} is related to the cross"=referenced parent entry. It executes \prm{true} if the \prm{field} of the child entry is equal to the corresponding \prm{field} of the parent entry, and \prm{false} otherwise. If the \bibfield{crossref}\slash \bibfield{xref} field is undefined, it always executes \prm{false}. This command is robust. See the description of the \bibfield{crossref} and \bibfield{xref} fields in \secref{bib:fld:spc} as well as \secref{bib:cav:ref} for further information concerning cross"=referencing.

\cmditem{iflistxref}{literal list}{true}{false}

类似于\cmd{iffieldxref} 命令,但检测\prm{literal list} 是否与cross"=referenced父条目相关联。
\bibfield{crossref} 和 \bibfield{xref} 域的描述见\secref{bib:fld:spc},更多关于cross"=referencing的信息见\secref{bib:cav:ref}。
%Similar to \cmd{iffieldxref} but checks if a \prm{literal list} is related to the cross"=referenced parent entry. See the description of the \bibfield{crossref} and \bibfield{xref} fields in \secref{bib:fld:spc} as well as \secref{bib:cav:ref} for further information concerning cross"=referencing.

\cmditem{ifnamexref}{name list}{true}{false}

类似于\cmd{iffieldxref} 命令,但检测\prm{name list} 是否与cross"=referenced父条目相关联。
\bibfield{crossref} 和 \bibfield{xref} 域的描述见\secref{bib:fld:spc},更多关于cross"=referencing的信息见\secref{bib:cav:ref}。
%Similar to \cmd{iffieldxref} but checks if a \prm{name list} is related to the cross"=referenced parent entry. See the description of the \bibfield{crossref} and \bibfield{xref} fields in \secref{bib:fld:spc} as well as \secref{bib:cav:ref} for further information concerning cross"=referencing.

\cmditem{ifcurrentfield}{field}{true}{false}

执行\prm{true},如果当前域为\prm{field},否则执行\prm{false}。该命令是鲁棒的。它主要用于域格式指令中,如果在其它环境中总是执行\prm{false}。
%Executes \prm{true} if the current field is \prm{field}, and \prm{false} otherwise. This command is robust. It is intended for use in field formatting directives and always executes \prm{false} when used in any other context.

\cmditem{ifcurrentlist}{literal list}{true}{false}

执行\prm{true},如果当前list为\prm{literal list},否则执行\prm{false}。该命令是鲁棒的。它主要用于域格式指令中,如果在其它环境中总是执行\prm{false}。
%Executes \prm{true} if the current list is \prm{literal list}, and \prm{false} otherwise. This command is robust. It is intended for use in list formatting directives and always executes \prm{false} when used in any other context.

\cmditem{ifcurrentname}{name list}{true}{false}

执行\prm{true},如果当前list为\prm{name list},否则执行\prm{false}。该命令是鲁棒的。它主要用于域格式指令中,如果在其它环境中总是执行\prm{false}。
%Executes \prm{true} if the current list is \prm{name list}, and \prm{false} otherwise. This command is robust. It is intended for use in list formatting directives and always executes \prm{false} when used in any other context.

\cmditem{ifuseprefix}{true}{false}

执行\prm{true},如果\opt{useprefix} 选项启用(无论是全局的还是针对当前条目),否则执行\prm{false}。该选项的细节见\secref{use:opt:bib}。
%Expands to \prm{true} if the \opt{useprefix} option is enabled (either globally or for the current entry), and \prm{false} otherwise. See \secref{use:opt:bib} for details on this option.

\cmditem{ifuseauthor}{true}{false}

这只是下面的\cmd{ifuse$<$name$>$} 宏的一个特例,因为\bibfield{author} 是默认数据模型的一部分所以放到这里来说。如果\opt{useauthor} 选项启用(无论是全局的还是针对当前条目),执行\prm{true},否则执行\prm{false}。该选项的细节见\secref{use:opt:bib}。
%This is just a particular case of the \cmd{ifuse$<$name$>$} macro below but is mentioned here as \bibfield{author} is part of the default data model. Expands to \prm{true} if the \opt{useauthor} option is enabled (either globally or for the current entry), and \prm{false} otherwise. See \secref{use:opt:bib} for details on this option.

\cmditem{ifuseeditor}{true}{false}

这只是下面的\cmd{ifuse$<$name$>$} 宏的一个特例,因为\bibfield{editor} 是默认数据模型的一部分所以放到这里来说。如果\opt{useeditor} 选项启用(无论是全局的还是针对当前条目),执行\prm{true},否则执行\prm{false}。该选项的细节见\secref{use:opt:bib}。
%This is just a particular case of the \cmd{ifuse$<$name$>$} macro below but is mentioned here as \bibfield{editor} is part of the default data model. Expands to \prm{true} if the \opt{useeditor} option is enabled (either globally or for the current entry), and \prm{false} otherwise. See \secref{use:opt:bib} for details on this option.

\cmditem{ifusetranslator}{true}{false}

这只是下面的\cmd{ifuse$<$name$>$} 宏的一个特例,因为\bibfield{translator} 是默认数据模型的一部分所以放到这里来说。如果\opt{usetranslator} 选项启用(无论是全局的还是针对当前条目),执行\prm{true},否则执行\prm{false}。该选项的细节见\secref{use:opt:bib}。
%This is just a particular case of the \cmd{ifuse$<$name$>$} macro below but is mentioned here as \bibfield{translator} is part of the default data model. Expands to \prm{true} if the \opt{usetranslator} option is enabled (either globally or for the current entry), and \prm{false} otherwise. See \secref{use:opt:bib} for details on this option.

\cmditem{ifuse$<$name$>$}{true}{false}

%Expands to \prm{true} if the \opt{use$<$name$>$} option is enabled (either globally or for the current entry), and \prm{false} otherwise. See \secref{use:opt:bib} for details on this option.
展开为\prm{true},如果选项\opt{use$<$name$>$} 启用(无论全局还是当前条目的选项),否则展开为\prm{false}。这一选项的细节详见第\secref{use:opt:bib} 节。

\cmditem{ifcrossrefsource}{true}{false}

展开为\prm{true},如果包含在\file{.bbl} 中的条目的间接引用(referenced,译者:是交叉引用)次数大于\opt{mincrossrefs},否则展开为\prm{false}。见\secref{use:opt:pre:gen}。如果条目被直接引用则展开为\prm{false}。
%Expands to \prm{true} if the entry was inclued in the \file{.bbl} due to being referenced more than \opt{mincrossrefs} times and false otherwise. See \secref{use:opt:pre:gen}. Also expands to false if the entry was directly cited.

\cmditem{ifxrefsource}{true}{false}

展开为\prm{true},如果包含在\file{.bbl} 中的条目的间接引用(referenced)次数大于\opt{minxrefs},否则展开为\prm{false}。见\secref{use:opt:pre:gen}。如果条目被直接引用则展开为\prm{false}。类似于ifcrossrefsource,但针对xref域。
%Expands to \prm{true} if the entry was inclued in the \file{.bbl} due to being referenced more than \opt{minxrefs} times and false otherwise. See \secref{use:opt:pre:gen}. Also expands to false if the entry was directly cited.

\cmditem{ifsingletitle}{true}{false}

%Expands to \prm{true} if there is only one work by the \opt{labelname} name in the bibliography, and to \prm{false} otherwise. If \opt{labelname} is not set for an entry, this will always expand to \prm{false}. Note that this feature needs to be enabled explicitly with the package option \opt{singletitle}.
展开为\prm{true},如果文献表中只有一篇文献具有\opt{labelname},否则展开为\prm{false}。如果条目的\opt{labelname} 未设置,总是展开为\prm{false}。注意: 使用该功能需要显式启用宏包选项\opt{singletitle}。

\cmditem{ifnocite}{true}{false}

%Expands to \prm{true} if the entry was \emph{only} included in the \file{.bbl} via \cmd{nocite}. That is, returns \prm{false} if an entry was both \cmd{nocite}'d and \cmd{cite}'d.
展开为\prm{true},如果条目仅是通过\cmd{nocite}而被包含在\file{.bbl}中,即如果一个条目均被\cmd{nocite}和\cmd{cite}命令引用时,返回为\prm{false}。

\cmditem{ifuniquetitle}{true}{false}

展开为\prm{true},如果只有一篇文献的题名是\opt{labeltitle},否则展开为\prm{false}。如果条目的\opt{labeltitle} 未设置,总是展开为\prm{false}。注意: 要使用这一功能需要显式地启用包选项\opt{uniquetitle}。
%Expands to \prm{true} if there is only one work with the title \opt{labeltitle} and to \prm{false} otherwise. If \opt{labeltitle} is not set for an entry, this will always expand to \prm{false}. Note that this feature needs to be enabled explicitly with the package option \opt{uniquetitle}.

\cmditem{ifuniquebaretitle}{true}{false}

展开为\prm{true},如果\bibfield{labelname} 域为空且只有一篇文献的题名是\opt{labeltitle},否则展开为\prm{false}。如果条目的\opt{labeltitle} 未设置,总是展开为\prm{false}。注意: 要使用这一功能需要显式地启用包选项\opt{uniquebaretitle}。
%Expands to \prm{true} if \bibfield{labelname} is empty and there is only one work with the title \opt{labeltitle} and to \prm{false} otherwise. If \opt{labeltitle} is not set for an entry, this will always expand to \prm{false}. Note that this feature needs to be enabled explicitly with the package option \opt{uniquebaretitle}.

\cmditem{ifuniquework}{true}{false}

展开为\prm{true},如果文献表中只有一篇文献的标签名是\opt{labelname} 且题名是\opt{labeltitle},否则展开为\prm{false}。如果条目的\opt{labelname} 和\opt{labeltitle} 均未设置,总是展开为\prm{false}。注意:要使用这一功能需要显式地启用包选项\opt{uniquework}。如果同一条目的\bibfield{singletitle} 和\bibfield{uniquetitle} 都是false,可能是因为有其他条目具有相同的\bibfield{labelname},还有其他条目具有相同的\bibfield{labeltitle}。\bibfield{uniquework} 可以让我们知道是否有条目同时具有相同的\bibfield{labelname} 和\bibfield{labeltitle}。这在多人合作时很有用,当多人同时维护一个参考文献数据源时,可能存在添加内容相同但条目关键词不同的文献的风险。这一判断能帮助找到这种存在副本情况。
%Expands to \prm{true} if there is only one work by the \opt{labelname} name with the \opt{labeltitle} title in the bibliography, and to \prm{false} otherwise. If neither \opt{labelname} nor \opt{labeltitle} are set for an entry, this will always expand to \prm{false}. Note that this feature needs to be enabled explicitly with the package option \opt{uniquework}. If both \bibfield{singletitle} and \bibfield{uniquetitle} are false for the same entry, this could be because another entry has the same \bibfield{labdlname} and yet another, different, entry has the same \bibfield{labeltitle}. \bibfield{uniquework} would let you know that there is another entry that has \emph{both} the same \bibfield{labelname} \emph{and} the same \bibfield{labeltitle}. This could be helpful in cases where multiple people maintain bibliography datasources and there is a risk of adding the same work with different keys without other parties realising this. This test could help to find such duplicates.

\cmditem{ifuniqueprimaryauthor}{true}{false}

展开为\prm{true},如果一篇文献的对于其\opt{labelname} 的第一作者的姓是唯一的,否则展开为\prm{false}。如果条目的\opt{labelname} 未设置,将展开为\prm{false}。注意: 使用该功能需要显式启用包选项\opt{uniqueprimaryauthor}。
%Expands to \prm{true} if there is only one work by the primary (first) author name of \opt{labelname} and to \prm{false} otherwise. If \opt{labelname} is not set for an entry, this will always expand to \prm{false}. Note that this feature needs to be enabled explicitly with the package option \opt{uniqueprimaryauthor}.

\cmditem{ifandothers}{list}{true}{false}

展开为\prm{true},如果\prm{list} 已定义并且在\file{bib} 文件中以关键词<\texttt{and others}> 截短了,否则展开为\prm{false}。\prm{list} 可以是literal 或name 列表。
%Expands to \prm{true} if the \prm{list} is defined and has been truncated in the \file{bib} file with the keyword <\texttt{and others}>, and to \prm{false} otherwise. The \prm{list} may be a literal list or a name list.

\cmditem{ifmorenames}{true}{false}

展开为\prm{true},如果当前姓名列表已经截短或将截短,否则展开为\prm{false}。该命令用于姓名列表的格式化指令中,在其它地方使用将展开为\prm{false}。该命令对当前列表执行与\cmd{ifandothers} 判断一样的操作。如果判断结果为否,它还将检测\cnt{listtotal} 是否大于\cnt{liststop}。该命令用于格式化命令中用以决定是否需要在列表默认打印«and others» or «et al.»这样的标注。注意: 还需要检测是否处于列表中间或者末尾时,即\cnt{listcount} 是否小于或等于\cnt{liststop},详见第\secref{aut:bib:dat} 节。
%Expands to \prm{true} if the current name list has been or will be truncated, and to \prm{false} otherwise. This command is intended for use in formatting directives for name lists. It will always expand to \prm{false} when used elsewhere. This command performs the equivalent of an \cmd{ifandothers} test for the current list. If this test is negative, it also checks if the \cnt{listtotal} counter is larger than \cnt{liststop}. This command may be used in a formatting directive to decide if a note such as «and others» or «et al.» is to be printed at the end of the list. Note that you still need to check whether you are in the middle or at the end of the list, \ie whether \cnt{listcount} is smaller than or equal to \cnt{liststop}, see \secref{aut:bib:dat} for details.

\cmditem{ifmoreitems}{true}{false}

类似于\cmd{ifmorenames},但检测literal列表。用于literal列表的格式化指令,在其它地方使用总是展开为\prm{false}。
%This command is similar to \cmd{ifmorenames} but checks the current literal list. It is intended for use in formatting directives for literal lists. It will always expand to \prm{false} when used elsewhere.

\cmditem{if$<$namepart$>$inits}{true}{false}

根据\opt{firstinits} 包选项的状态,展开为\prm{true} 或\prm{false}(见第\secref{use:opt:pre:int} 节)。该命令用于姓名列表的格式化指令。
%Expands to \prm{true} or \prm{false}, depending on the state of the \opt{$<$namepart$>$inits} package option (see \secref{use:opt:pre:int}). This command is intended for use in formatting directives for name lists.

\cmditem{ifterseinits}{true}{false}

根据\opt{terseinits} 包选项的状态,展开为\prm{true} 或\prm{false}(见第\secref{use:opt:pre:int} 节)。该命令用于姓名列表的格式化指令。
%Expands to \prm{true} or \prm{false}, depending on the state of the \opt{terseinits} package option (see \secref{use:opt:pre:int}). This command is intended for use in formatting directives for name lists.

\cmditem{ifentrytype}{type}{true}{false}

如果当前处理条目类型是\prm{type},则展开为\prm{true},否则展开为\prm{false}。
%Executes \prm{true} if the entry type of the entry currently being processed is \prm{type}, and \prm{false} otherwise.

\cmditem{ifkeyword}{keyword}{true}{false}

如果\prm{keyword} 能在当前处理条目的\bibfield{keywords} 域中找到,展开为\prm{true},否则展开为\prm{false}。
%Executes \prm{true} if the \prm{keyword} is found in the \bibfield{keywords} field of the entry currently being processed, and \prm{false} otherwise.

\cmditem{ifentrykeyword}{entrykey}{keyword}{true}{false}

将条目关键词作为第一个参数的\cmd{ifkeyword} 命令的变化形式,用于判断当前处理条目是否是某一条目。
%A variant of \cmd{ifkeyword} which takes an entry key as its first argument. This is useful for testing an entry other than the one currently processed.

\cmditem{ifcategory}{category}{true}{false}

执行\prm{true},如果当前正在处理条目被指派到由\cmd{addtocategory} 命令定义的\prm{category} 类中,否则执行\prm{false}。
%Executes \prm{true} if the entry currently being processed has been assigned to a \prm{category} with \cmd{addtocategory}, and \prm{false} otherwise.

\cmditem{ifentrycategory}{entrykey}{category}{true}{false}

将条目关键词作为第一个参数的\cmd{ifcategory} 命令的变化形式,用于判断当前处理条目是否是某一条目。
%A variant of \cmd{ifcategory} which takes an entry key as its first argument. This is useful for testing an entry other than the one currently processed.

\cmditem{ifciteseen}{true}{false}

展开为\prm{true},如果当前条目之前已经被引用过,否则展开为\prm{false}。该命令是鲁棒的,用于标注样式中。如果文档中有\env{refsection} 环境,引用追踪是基于这些环境的。注意: 引用追踪器需要显式启用包选项\opt{citetracker},如果追踪器未启用,该命令总是展开为\prm{false}。另可参见第\secref{aut:aux:msc} 节的\cmd{citetrackertrue} 和\cmd{citetrackerfalse} 开关。
%Executes \prm{true} if the entry currently being processed has been cited before, and \prm{false} otherwise. This command is robust and intended for use in citation styles. If there are any \env{refsection} environments in the document, the citation tracking is local to these environments. Note that the citation tracker needs to be enabled explicitly with the package option \opt{citetracker}. The behavior of this test depends on the mode the citation tracker is operating in, see \secref{use:opt:pre:int} for details. If the citation tracker is disabled, the test always yields \prm{false}. Also see the \cmd{citetrackertrue} and \cmd{citetrackerfalse} switches in \secref{aut:aux:msc}.

\cmditem{ifentryseen}{entrykey}{true}{false}

将条目关键词作为第一个参数的\cmd{ifciteseen} 命令的变化形式。
因为\prm{entrykey} 先于判断展开,它也可以用来测试在\bibfield{xref} 等域中的条目关键词。
%A variant of \cmd{ifciteseen} which takes an entry key as its first argument. Since the \prm{entrykey} is expanded prior to performing the test, it is possible to test for entry keys in a field such as \bibfield{xref}:

\begin{ltxexample}
\ifentryseen{<<\thefield{xref}>>}{true}{false}
\end{ltxexample}
%
除了一个额外参数,\cmd{ifentryseen} 的操作类似于\cmd{ifciteseen}。 在正文中使用的面向用户的类似命令见\secref{use:eq}。
%Apart from the additional argument, \cmd{ifentryseen} behaves like \cmd{ifciteseen}. A user-facing version of this command is available for use in documents see \secref{use:eq}.

\cmditem{ifentryinbib}{entrykey}{true}{false}

如果\prm{entrykey} 出现当前文献表中,执行\prm{true},否则执行\prm{false}。该命令用于参考文献著录样式。在正文中使用的面向用户的类似命令见\secref{use:eq}。
%Executes \prm{true} if the entry \prm{entrykey} appears in the current bibliography, and \prm{false} otherwise. A user-facing version of this command is available for use in documents see \secref{use:eq}.

\cmditem{iffirstcitekey}{true}{false}

如果当前处理条目是引用列表中的第一个条目,执行\prm{true},否则执行\prm{false}。该命令依赖于\cnt{citecount}, \cnt{citetotal}, \cnt{multicitecount} 和 \cnt{multicitetotal} 计数器(见\secref{aut:fmt:ilc}),因此只能用于\cmd{DeclareCiteCommand} 命令定义的标注命令的循环执行代码\prm{loopcode} 中。
%Executes \prm{true} if the entry currently being processed is the first one in the citation list, and \prm{false} otherwise. This command relies on the \cnt{citecount}, \cnt{citetotal}, \cnt{multicitecount} and \cnt{multicitetotal} counters (\secref{aut:fmt:ilc}) and thus is intended for use only in the \prm{loopcode} of a citation command defined with \cmd{DeclareCiteCommand}.

\cmditem{iflastcitekey}{true}{false}

类似于\cmd{iffirstcitekey},但判断的是是否为引用列表中的最后一个条目。
%Similar \cmd{iffirstcitekey}, but executes \prm{true} if the entry currently being processed is the last one in the citation list, and \prm{false} otherwise.

\cmditem{ifciteibid}{true}{false}

如果当前处理条目与前一条相同,展开为\prm{true},否则展开为\prm{false}。该命令用于标注样式。如果有\env{refsection} 环境,追踪器是基于这些环境的。注意: <ibidem>追踪器需要由\opt{ibidtracker} 包选项显式启用。该判断命令的运行方式与追踪器运行的模式相关,详见\secref{use:opt:pre:int}。如果追踪器未启用,总是展开为\prm{false}。另可参见\secref{aut:aux:msc} 节的\cmd{citetrackertrue} 和\cmd{citetrackerfalse} 开关。
%Expands to \prm{true} if the entry currently being processed is the same as the last one, and to \prm{false} otherwise. This command is intended for use in citation styles. If there are any \env{refsection} environments in the document, the tracking is local to these environments. Note that the <ibidem> tracker needs to be enabled explicitly with the package option \opt{ibidtracker}. The behavior of this test depends on the mode the tracker is operating in, see \secref{use:opt:pre:int} for details. If the tracker is disabled, the test always yields \prm{false}. Also see the \cmd{citetrackertrue} and \cmd{citetrackerfalse} switches in \secref{aut:aux:msc}.

\cmditem{ifciteidem}{true}{false}

如果当前处理条目的责任者(即作者或编者)与前一条目的相同,展开为\prm{true},否则展开为\prm{false}。该命令用于标注样式。如果有\env{refsection} 环境,追踪器是基于这些环境的。注意: <idem> 追踪器需要由\opt{idemtracker} 包选项显式启用。该判断命令的运行方式与追踪器运行的模式相关,详见\secref{use:opt:pre:int}。如果追踪器未启用,总是展开为\prm{false}。另可参见\secref{aut:aux:msc} 节的\cmd{citetrackertrue} 和\cmd{citetrackerfalse} 开关。
%Expands to \prm{true} if the primary name (\ie the author or editor) in the entry currently being processed is the same as the last one, and to \prm{false} otherwise. This command is intended for use in citation styles. If there are any \env{refsection} environments in the document, the tracking is local to these environments. Note that the <idem> tracker needs to be enabled explicitly with the package option \opt{idemtracker}. The behavior of this test depends on the mode the tracker is operating in, see \secref{use:opt:pre:int} for details. If the tracker is disabled, the test always yields \prm{false}. Also see \cmd{citetrackertrue} and \cmd{citetrackerfalse} in \secref{aut:aux:msc}.

\cmditem{ifopcit}{true}{false}

该命令类似于\cmd{ifciteibid},但只要当前处理条目的\emph{作者 或 编者 } 与前一条目相同,则展开为\prm{true}。注意: <opcit> 追踪器需要由\opt{opcittracker} 包选项显式的启用。该判断命令的运行方式与追踪器运行的模式相关,详见\secref{use:opt:pre:int}。如果追踪器未启用,总是展开为\prm{false}。另可参见\secref{aut:aux:msc} 节的\cmd{citetrackertrue} 和\cmd{citetrackerfalse} 开关。
%This command is similar to \cmd{ifciteibid} except that it expands to \prm{true} if the entry currently being processed is the same as the last one \emph{by this author or editor}. Note that the <opcit> tracker needs to be enabled explicitly with the package option \opt{opcittracker}. The behavior of this test depends on the mode the tracker is operating in, see \secref{use:opt:pre:int} for details. If the tracker is disabled, the test always yields \prm{false}. Also see the \cmd{citetrackertrue} and \cmd{citetrackerfalse} switches in \secref{aut:aux:msc}.

\cmditem{ifloccit}{true}{false}

该命令类似于\cmd{ifopcit},但还要比较\prm{postnote} 的参数,如果他们相同且是数值(\secref{aut:aux:tst} 节的\cmd{ifnumerals} 命令判断),则展开为\prm{true}。即:如果引文的页码与前一文献相同则展开为\texttt{true}。 注意: <loccit> 追踪器需要由\opt{loccittracker} 包选项显式启用。该判断命令的运行方式与追踪器运行的模式相关,详见\secref{use:opt:pre:int}。如果追踪器未启用,总是展开为\prm{false}。另可参见\secref{aut:aux:msc} 节的\cmd{citetrackertrue} 和\cmd{citetrackerfalse} 开关。
%This command is similar to \cmd{ifopcit} except that it also compares the \prm{postnote} arguments and expands to \prm{true} only if they match and are numerical (in the sense of \cmd{ifnumerals} from \secref{aut:aux:tst}), \ie \cmd{ifloccit} will yield \texttt{true} if the citation refers to the same page cited before. Note that the <loccit> tracker needs to be enabled explicitly with the package option \opt{loccittracker}. The behavior of this test depends on the mode the tracker is operating in, see \secref{use:opt:pre:int} for details. If the tracker is disabled, the test always yields \prm{false}. Also see the \cmd{citetrackertrue} and \cmd{citetrackerfalse} switches in \secref{aut:aux:msc}.

\cmditem{iffirstonpage}{true}{false}

该命令的运行与\opt{pagetracker} 包选项相关,如果选项设置成\texttt{page},当前项是页中的第一项,展开为\prm{true},否则展开为\prm{false}。如果选项设置成\texttt{spread},当前项是合页(double-page spread)中的第一项,展开为\prm{true},否则展开为\prm{false}。如果选项未启用,总是展开为\prm{false}。根据所处环境不同,<item>可以是一个标注,或者参考文献表中的条目。注意该命令区分正文文本和脚注,例如,当在某页的第一个脚注中使用,即便是文中有一个标注且先于该脚注,它也展开为\prm{true}。另可参见\secref{aut:aux:msc} 节的\cmd{pagetrackertrue} 和\cmd{pagetrackerfalse} 开关。
%The behavior of this command is responsive to the package option \opt{pagetracker}. If the option is set to \texttt{page}, it expands to \prm{true} if the current item is the first one on the page, and to \prm{false} otherwise. If the option is set to \texttt{spread}, it expands to \prm{true} if the current item is the first one on the double-page spread, and to \prm{false} otherwise. If the page tracker is disabled, this test always yields \prm{false}. Depending on the context, the <item> may be a citation or an entry in the bibliography or a bibliography list. Note that this test distinguishes between body text and footnotes. For example, if used in the first footnote on a page, it will expand to \prm{true} even if there is a citation in the body text prior to the footnote. Also see the \cmd{pagetrackertrue} and \cmd{pagetrackerfalse} switches in \secref{aut:aux:msc}.

\cmditem{ifsamepage}{instance 1}{instance 2}{true}{false}

如果两个引用实例位于同于页或者同一合页中,展开为\prm{true},否则为\prm{false}。一个引用实例可以是一个标注也可以是文献表中的条目。这些实例用\cnt{instcount} 计数区分,见\secref{aut:fmt:ilc}。该命令的运行与\opt{pagetracker} 包选项相关,如果选项设置成\texttt{spread},其本质是<if same spread>(是否同一合页)的判断。如果选项未启用,总是展开为\prm{false}。参数\prm{instance 1} 和\prm{instance 2} 以\etex's \cmd{numexpr} 方式当成整数表达式处理。这意味着可以在参数中计算。比如:
%This command expands to \prm{true} if two instances of a reference are located on the same page or double-page spread, and to \prm{false} otherwise. An instance of a reference may be a citation or an entry in the bibliography or a bibliography list. These instances are identified by the value of the \cnt{instcount} counter, see \secref{aut:fmt:ilc}. The behavior of this command is responsive to the package option \opt{pagetracker}. If this option is set to \texttt{spread}, \cmd{ifsamepage} is in fact an <if same spread> test. If the page tracker is disabled, this test always yields \prm{false}. The arguments \prm{instance 1} and \prm{instance 2} are treated as integer expressions in the sense of \etex's \cmd{numexpr}. This implies that it is possible to make calculations within these arguments, for example:

\begin{ltxexample}
\ifsamepage{<<\value>>{instcount}}{<<\value>>{instcount}<<-1>>}{true}{false}
\end{ltxexample}
注意: \cmd{value} 命令不是以\cmd{the} 为前缀,在第二个参数中做了减法运算。如果\prm{instance 1} 或 \prm{instance 2} 是无效数字(比如一个负值),总是展开为\prm{false}。也要注意该命令不区分正文和脚注。另可参见\secref{aut:aux:msc} 节的\cmd{pagetrackertrue} 和\cmd{pagetrackerfalse} 开关。
%Note that \cmd{value} is not prefixed by \cmd{the} and that the subtraction is included in the second argument in the above example. If \prm{instance 1} or \prm{instance 2} is an invalid number (for example, a negative one), the test yields \prm{false}. Also note that this test does not distinguish between body text and footnotes. Also see the \cmd{pagetrackertrue} and \cmd{pagetrackerfalse} switches in \secref{aut:aux:msc}.

\cmditem{ifinteger}{string}{true}{false}

如果\prm{string} 是一个正整数,展开为\prm{true},否则为\prm{false},该命令鲁棒。
%Executes \prm{true} if the \prm{string} is a positive integer, and \prm{false} otherwise. This command is robust.

\cmditem{ifnumeral}{string}{true}{false}

如果\prm{string} 是一个阿拉伯或者罗马数字,展开为\prm{true},否则为\prm{false},该命令鲁棒。另可参见\secref{aut:aux:msc} 节的\cmd{DeclareNumChars} 和\cmd{NumCheckSetup} 命令。
%Executes \prm{true} if the \prm{string} is an Arabic or Roman numeral, and \prm{false} otherwise. This command is robust. See also \cmd{DeclareNumChars} and \cmd{NumCheckSetup} in \secref{aut:aux:msc}.

\cmditem{ifnumerals}{string}{true}{false}

如果\prm{string} 是一个阿拉伯或者罗马数字的范围或列表,展开为\prm{true},否则为\prm{false},该命令鲁棒。相比于\cmd{ifnumeral} 命令,当参数像 «52--58», «14/15», «1,~3,~5» 等时,该命令会执行\prm{true}。
另可参见\secref{aut:aux:msc} 节的\cmd{DeclareNumChars},\cmd{NumCheckSetup},\cmd{DeclareRangeCommands} 和 \cmd{NumCheckSetup} 命令。
%Executes \prm{true} if the \prm{string} is a range or a list of Arabic or Roman numerals, and \prm{false} otherwise. This command is robust. In contrast to \cmd{ifnumeral}, it will also execute \prm{true} with arguments like «52--58», «14/15», «1,~3,~5», and so on. See also \cmd{DeclareNumChars}, \cmd{DeclareRangeChars}, \cmd{DeclareRangeCommands}, and \cmd{NumCheckSetup} in \secref{aut:aux:msc}.

\cmditem{ifpages}{string}{true}{false}

类似于\cmd{ifnumerals},但也考虑\secref{aut:aux:msc} 节的\cmd{DeclarePageCommands} 命令。
%Similar to \cmd{ifnumerals}, but also considers \cmd{DeclarePageCommands} from \secref{aut:aux:msc}.

\cmditem{iffieldint}{field}{true}{false}

类似于\cmd{ifinteger} 命令,但使用\prm{field} 的值而不是一个字符串,如果域未定义,执行\prm{false}。
%Similar to \cmd{ifinteger}, but uses the value of a \prm{field} rather than a literal string in the test. If the \prm{field} is undefined, it executes \prm{false}.

\cmditem{iffieldnum}{field}{true}{false}

类似于\cmd{ifnumeral} 命令,但使用\prm{field} 的值而不是一个字符串,如果域未定义,执行\prm{false}。
%Similar to \cmd{ifnumeral}, but uses the value of a \prm{field} rather than a literal string in the test. If the \prm{field} is undefined, it executes \prm{false}.

\cmditem{iffieldnums}{field}{true}{false}

类似于\cmd{ifnumerals} 命令,但使用\prm{field} 的值而不是一个字符串,如果域未定义,执行\prm{false}。
%Similar to \cmd{ifnumerals}, but uses the value of a \prm{field} rather than a literal string in the test. If the \prm{field} is undefined, it executes \prm{false}.

\cmditem{iffieldpages}{field}{true}{false}

类似于\cmd{ifpages} 命令,但使用\prm{field} 的值而不是一个字符串,如果域未定义,执行\prm{false}。
%Similar to \cmd{ifpages}, but uses the value of a \prm{field} rather than a literal string in the test. If the \prm{field} is undefined, it executes \prm{false}.

\cmditem{ifbibstring}{string}{true}{false}

如果\prm{string} 是已知的本地化关键词,展开为\prm{true},否则\prm{false}。默认定义的本地化字符串见\secref{aut:lng:key}。新的字符串可以用命令\cmd{NewBibliographyString} 定义。
%Expands to \prm{true} if the \prm{string} is a known localisation key, and to \prm{false} otherwise. The localisation keys defined by default are listed in \secref{aut:lng:key}. New ones may be defined with \cmd{NewBibliographyString}.

\cmditem{ifbibxstring}{string}{true}{false}

类似于\cmd{ifbibstring},但\prm{string} 是展开的。
%Similar to \cmd{ifbibstring}, but the \prm{string} is expanded.

\cmditem{iffieldbibstring}{field}{true}{false}

类似于\cmd{ifbibstring},但使用\prm{field} 域的值而不是一个字符串,如果域未定义,执行\prm{false}。
%Similar to \cmd{ifbibstring}, but uses the value of a \prm{field} rather than a literal string in the test. If the \prm{field}  is undefined, it expands to \prm{false}.

\cmditem{iffieldplusstringbibstring}{field}{string}{true}{false}

%Similar to \cmd{iffieldbibstring}, but appends \prm{string} to the value of \prm{field} and checks if the resulting string is a known localisation key. Expands to \prm{false} if \prm{field} is undefined.

类似于\cmd{iffieldbibstring},会把\prm{string}附加到\prm{field} 域的值上,并检测形成的字符串是否是一个已定义的本地化字符串关键字。当\prm{field} 未定义,则展开为\prm{false} 。

\cmditem{ifdriver}{entrytype}{true}{false}

展开为\prm{true} 如果\prm{entrytype} 的驱动存在,否则为\prm{false}。
%Expands to \prm{true} if a driver for the \prm{entrytype} is available, and to \prm{false} otherwise.

\cmditem{ifcapital}{true}{false}

如果\biblatex 的标点追踪器将当前位置的本地化字符串大写,则执行\prm{true},否则执行\prm{false}。该命令鲁棒,用于格式化指令中对姓名的某些成分做有条件的大写。
%Executes \prm{true} if \biblatex's punctuation tracker would capitalize a localisation string at the current location, and \prm{false} otherwise. This command is robust. It may be useful for conditional capitalization of certain parts of a name in a formatting directive.

\cmditem{ifcitation}{true}{false}

当处于标注中则展开为\prm{true},否则为\prm{false}。注意这一命令与其所在的最外层环境有关。比如,当由\cmd{DeclareCiteCommand} 命令定义的标注命令执行一个由\cmd{DeclareBibliographyDriver} 定义的驱动,则任何在该驱动中的\cmd{ifcitation} 都会展开为\prm{true}。一个例子见\secref{aut:cav:mif} 节。
%Expands to \prm{true} when located in a citation, and to \prm{false} otherwise. Note that this command is responsive to the outermost context in which it is used. For example, if a citation command defined with \cmd{DeclareCiteCommand} executes a driver defined with \cmd{DeclareBibliographyDriver}, any \cmd{ifcitation} tests in the driver code will yield \prm{true}. See \secref{aut:cav:mif} for a practical example.

\cmditem{ifbibliography}{true}{false}

当处于文献表中则展开为\prm{true},否则为\prm{false}。注意这一命令与其所在的最外层环境有关。比如,当由\cmd{DeclareBibliographyDriver} 命令定义的驱动执行一个由\cmd{DeclareCiteCommand} 定义的标注,则任何在该标注中的\cmd{ifbibliography} 都会展开为\prm{true}。一个例子见\secref{aut:cav:mif} 节。
%Expands to \prm{true} when located in a bibliography, and to \prm{false} otherwise. Note that this command is responsive to the outermost context in which it is used. For example, if a driver defined with \cmd{DeclareBibliographyDriver} executes a citation command defined with \cmd{DeclareCiteCommand}, any \cmd{ifbibliography} tests in the citation code will yield \prm{true}. See \secref{aut:cav:mif} for a practical example.

\cmditem{ifnatbibmode}{true}{false}

根据\secref{use:opt:ldt} 节的\opt{natbib} 选项展开为\prm{true} 或\prm{false}。
%Expands to \prm{true} or \prm{false} depending on the \opt{natbib} option from \secref{use:opt:ldt}.

\cmditem{ifciteindex}{true}{false}

根据\secref{use:opt:pre:gen} 节的\opt{indexing} 选项展开为\prm{true} 或\prm{false}。
%Expands to \prm{true} or \prm{false} depending on the \opt{indexing} option from \secref{use:opt:pre:gen}.

\cmditem{ifbibindex}{true}{false}

根据\secref{use:opt:pre:gen} 节的\opt{indexing} 选项展开为\prm{true} 或\prm{false}。
%Expands to \prm{true} or \prm{false} depending on the \opt{indexing} option from \secref{use:opt:pre:gen}.

\cmditem{iffootnote}{true}{false}

当处于脚注中时,展开为\prm{true},否则为\prm{false}。注意: 在\env{minipage} 中的脚注被认为正文的一部分。当处于页面底部的脚注中或者由\sty{endnotes} 提供的endnotes中时,只会展开为\prm{true}。
%Expands to \prm{true} when located in a footnote, and to \prm{false} otherwise. Note that footnotes in \env{minipage} environments are considered to be part of the body text. This command will only expand to \prm{true} in footnotes a the bottom of the page and in endnotes as provided by the \sty{endnotes} package.

\cntitem{citecounter}

这一计数器表示当前处理条目在当前reference section中的引用次数。注意: 该功能需要显式启用包选项\opt{citecounter}。如果选项设置为\texttt{context},正文和脚注中的引用分别计数。这种情况下,\cnt{citecounter} 记录其所在环境中的值。
%This counter indicates how many times the entry currently being processed is cited in the current reference section. Note that this feature needs to be enabled explicitly with the package option \opt{citecounter}. If the option is set to \texttt{context}, citations in the body text and in footnotes are counted separately. In this case, \cnt{citecounter} will hold the value of the context it is used in.

\cntitem{maxcitecounter}

%This counter holds the maximum value of \cnt{citecounter} across all entries in the current reference section. Like \cnt{citecounter} it is only available if the \opt{citecounter} option is enabled and tracks footnotes and text separately if the option is set to \texttt{context}.

该计数器保存当前文献节中遍历所有条目后的\cnt{citecounter}值,类似于\cnt{citecounter},它仅在\opt{citecounter}选项启用时有效,
如果该选项设置为\texttt{context},那么脚注和正文中分别计算。


\cntitem{uniquename}
%This counter refers to the \bibfield{labelname} list. It is set on a per-name basis. Its value is \texttt{0} if the base name (by default the <family> part of the name) is unique, \texttt{1} if adding the other parts of the name (as specified in the uniquename template defined by \cmd{DeclareUniquenameTemplate})as initials will make it unique, and \texttt{2} if the full name is required to disambiguate the name. This information is required by author-year and author-title citation schemes which add additional parts of the name when citing different authors with the same last name. For example, (given the default \cmd{DeclareUniquenameTemplate} definition) if there is one <John Doe> and one <Edward Doe> in the list of references, this counter will be set to \texttt{1}. If there is one <John Doe> and one <Jane Doe>, the value of the counter will be \texttt{2}. If the option is set to \texttt{init}\slash \texttt{allinit}\slash \texttt{mininit}, the counter will be limited to \texttt{1}. This is useful for citations styles which use initials to disambiguate names but never print the full name in citations. If adding the initials is not sufficient to disambiguate the name, \cnt{uniquename} will also be set to \texttt{0} for that name. This feature needs to be enabled explicitly with the package option \opt{uniquename}. Note that the \cnt{uniquename} counter is local to \cmd{printnames} and that it is only set for the \bibfield{labelname} list and to the name list \bibfield{labelname} has been derived from (typically \bibfield{author} or \bibfield{editor}). Its value is zero in any other context, i.e., it must be evaluated in the name formatting directives handling name lists. See \secref{aut:cav:amb} for further details and practical examples.
这一计数器用于\bibfield{labelname} 列表。它以每个名字为基础进行设置。如果姓不同,它的值设置为0,当增加姓名的其它成分的首字母使得姓名能区分时,则设置为1,如果需要完整的姓名才能区分,则设置为2。作者年制和作者标题制的标注格式需要这一信息来增加姓名的其它成分以对同姓的不同作者进行引用。比如: (考虑默认的\cmd{DeclareUniquenameTemplate} 定义)当引用列表中有一个<John Doe> 和一个<Edward Doe>,该计数器将设置为1。如果有一个<John Doe>和一个<Jane Doe>,该计数器将设置为2。如果选项设置成\texttt{init}\slash \texttt{allinit}\slash \texttt{mininit},那么计数器将限制值最大为\texttt{1}。
这对于标注样式不打印全名而使用首字母来区分姓名很有用。如果添加首字母还无法区分姓名,\cnt{uniquename} 也将设置为\texttt{0}。该功能需要显式启用包选项\opt{uniquename}。注意\cnt{uniquename} 是\cmd{printnames} 局部使用的,仅根据\bibfield{labelname} 列表或其来源姓名列表(典型如\bibfield{author} 或\bibfield{editor})设置。它的值在其它环境中都是0,即它仅在处理姓名的格式化指令中计算,更多细节和实例见\secref{aut:cav:amb}。

\cntitem{uniquelist}
%This counter refers to the \bibfield{labelname} list. It is set on a per-field basis. Its value indicates the number of names required to disambiguate the name list if automatic \cnt{maxnames}\slash \cnt{minnames} truncation would lead to ambiguous citations. For example, if there is one work by <Doe\slash Smith\slash Johnson> and another one by <Doe\slash Edwards\slash Williams>, setting \kvopt{maxnames}{1} would lead to <Doe et al.> in both cases. In this case, \cnt{uniquelist} would be set to \texttt{2} on the \bibfield{labelname} lists of both entries because at least the first two names are required to disambiguate them. Note that the \cnt{uniquelist} counter is local to \cmd{printnames} and that it is only set for the \bibfield{labelname} list and to the name list \bibfield{labelname} has been derived from (typically \bibfield{author} or \bibfield{editor}). Its value is zero in any other context. If available, the \cnt{uniquelist} value will be used automatically by \cmd{printnames} when processing the name list, \ie it will automatically override \cnt{maxnames}\slash \cnt{minnames}. This feature needs to be enabled explicitly with the package option \opt{uniquelist}. See \secref{aut:cav:amb} for further details and practical examples.
该计数器用于\bibfield{labelname} 列表。它以每个域为基础进行设置。它的值表示当使用\cnt{maxnames}\slash \cnt{minnames} 自动将姓名列表截短后导致标注歧义时,消除歧义需要的最小姓名数。比如,有一篇作者是<Doe\slash Smith\slash Johnson> 的文献和另一篇作者是<Doe\slash Edwards\slash Williams>的文献,设置\kvopt{maxnames}{1} 将导致两篇的作者都是<Doe et al.>。 这种情况下,两个条目的\bibfield{labelname} 列表的\cnt{uniquelist} 将设置成\texttt{2},因为至少需要两个名字来区分。
注意\cnt{uniquelist} 是\cmd{printnames} 命令局部使用的,仅根据\bibfield{labelname} 列表或其来源姓名列表(典型如\bibfield{author} 或\bibfield{editor})设置。它的值在其它环境中都是0,即它仅在处理姓名的格式化指令中计算,如果该值存在,则\cmd{printnames} 命令在处理姓名列表时将自动应用,即自动覆盖\cnt{maxnames}\slash \cnt{minnames}。该功能需要显式启用选项\opt{uniquelist}。更多细节和实例见\secref{aut:cav:amb}。


\cntitem{parenlevel}

圆括号和/或方括号的嵌套层级。该信息仅在\secref{use:opt:pre:int} 的\opt{parentracker} 选项启用的情况下提供。
%The current nesting level of parentheses and\slash or brackets. This information is only available if the \opt{parentracker} from \secref{use:opt:pre:int} is enabled.

\end{ltxsyntax}

\subsubsection{使用\cmd{ifboolexpr} 和\cmd{ifthenelse} 的判断}%Tests with \cmd{ifboolexpr} and \cmd{ifthenelse}
\label{aut:aux:ife}

第\secref{aut:aux:tst} 节介绍的判断可以与\sty{etoolbox} 宏包提供的\cmd{ifboolexpr} 命令和\sty{ifthen} 宏包提供的\cmd{ifthenelse} 命令一同使用。这种情况下,其语法略有差异,判断命令的\prm{true} 和\prm{false} 参数自动省略,而直接传递给\cmd{ifboolexpr} 或 \cmd{ifthenelse}。 注意,使用这些命令需要一些计算代价。如果不需要任何布尔运算,使用\secref{aut:aux:tst} 节的独立判断命令更高效。
%The tests introduced in \secref{aut:aux:tst} may also be used with the \cmd{ifboolexpr} command provided by the \sty{etoolbox} package and the \cmd{ifthenelse} command provided by the \sty{ifthen} package. The syntax of the tests is slightly different in this case: the \prm{true} and \prm{false} arguments are omitted from the test itself and passed to the \cmd{ifboolexpr} or \cmd{ifthenelse} command instead. Note that the use of these commands implies some processing overhead. If you do not need any boolean operators, it is more efficient to use the stand"=alone tests from \secref{aut:aux:tst}.

\begin{ltxsyntax}

\cmditem{ifboolexpr}{expression}{true}{false}

该\sty{etoolbox} 包命令允许进行包括布尔运算和编组的复杂判断。
%\sty{etoolbox} command which allows for complex tests with boolean operators and grouping:

\begin{lstlisting}[style=ifthen]{}
\ifboolexpr{ (
	       test {\ifnameundef{editor}}
	       and
	       not test {\iflistundef{location}}
	     )
	     or test {\iffieldundef{year}}
  }
  {...}
  {...}
\end{lstlisting}

\cmditem{ifthenelse}{tests}{true}{false}

该\sty{ifthen} 包命令允许进行包括布尔运算和编组的复杂判断。
%\sty{ifthen} command which allows for complex tests with boolean operators and grouping:

\begin{lstlisting}[style=ifthen]{}
\ifthenelse{ \(
		\ifnameundef{editor}
		\and
		\not \iflistundef{location}
	     \)
	     \or \iffieldundef{year}
  }
  {...}
  {...}
\end{lstlisting}
%
\biblatex 提供的附加判断命令,仅在标注命令和文献表中使用\cmd{ifboolexpr} 或\cmd{ifthenelse} 时可用。
%The additional tests provided by \biblatex are only available when \cmd{ifboolexpr} or \cmd{ifthenelse} are used in citation commands and in the bibliography.

\end{ltxsyntax}

\subsubsection{综合命令}%Miscellaneous Commands
\label{aut:aux:msc}

本节介绍参考文献著录和标注样式中使用的一些综合命令和小助手。
%The section introduced miscellaneous commands and little helpers for use in bibliography and citation styles.

\begin{ltxsyntax}

\cmditem{newbibmacro}{name}[arguments][optional]{definition}
\cmditem*{newbibmacro*}{name}[arguments][optional]{definition}

定义一个用于后面\cmd{usebibmacro} 调用的宏。该命令的语法类似于\cmd{newcommand},除了\prm{name} 可以包含一些数字或标点,但不以斜杠开头。可选参数\prm{arguments} 是一个整数用于指定宏需要处理的参数数量。如果\prm{optional} 给出,它指定了该宏的第一个参数的默认值,这第一个参数自动变成为可选参数。相比于\cmd{newcommand},当宏已经定义时,\cmd{newbibmacro} 命令会给出一个警告信息,并自动转换为\cmd{renewbibmacro} 命令。类似于\cmd{newcommand},该命令的常规形式在定义中使用\cmd{long} 前缀,而带星的命令则没有。如果一个宏声明为long,它的参数可以包含\cmd{par} 记号。提供\cmd{newbibmacro} 和\cmd{renewbibmacro} 命令是为了方便使用,样式作者也可以使用\cmd{newcommand} 或\cmd{def}。然而,需要注意,共享文件 \path{biblatex.def} 中的绝大多数定义都是用\cmd{newbibmacro} 定义的,因此,要使用和修改它们要用相应的方式处理。
%Defines a macro to be executed via \cmd{usebibmacro} later. The syntax of this command is very similar to \cmd{newcommand} except that \prm{name} may contain characters such as numbers and punctuation marks and does not start with a backslash. The optional argument \prm{arguments} is an integer specifying the number of arguments taken by the macro. If \prm{optional} is given, it specifies a default value for the first argument of the macro, which automatically becomes an optional argument. In contrast to \cmd{newcommand}, \cmd{newbibmacro} issues a warning message if the macro is already defined, and automatically falls back to \cmd{renewbibmacro}. As with \cmd{newcommand}, the regular variant of this command uses the \cmd{long} prefix in the definition while the starred one does not. If a macro has been declared to be long, it may take arguments containing \cmd{par} tokens. \cmd{newbibmacro} and \cmd{renewbibmacro} are provided for convenience. Style authors are free to use \cmd{newcommand} or \cmd{def} instead. However, note that most shared definitions found in \path{biblatex.def} are defined with \cmd{newbibmacro}, hence they must be used and modified accordingly.

\cmditem{renewbibmacro}{name}[arguments][optional]{definition}
\cmditem*{renewbibmacro*}{name}[arguments][optional]{definition}

类似于\cmd{newbibmacro},但用于重定义\prm{name}。相比于\cmd{newcommand},当宏未定义时,\cmd{renewbibmacro} 命令给出一个警告信息,并自动转换为\cmd{newbibmacro} 命令。
%Similar to \cmd{newbibmacro} but redefines \prm{name}. In contrast to \cmd{renewcommand}, \cmd{renewbibmacro} issues a warning message if the macro is undefined, and automatically falls back to \cmd{newbibmacro}.

\cmditem{providebibmacro}{name}[arguments][optional]{definition}
\cmditem*{providebibmacro*}{name}[arguments][optional]{definition}

类似于\cmd{newbibmacro},但仅在\prm{name} 未定义时定义宏。该命令概念上类似于\cmd{providecommand}。
%Similar to \cmd{newbibmacro} but only defines \prm{name} if it is undefined. This command is similar in concept to \cmd{providecommand}.

\cmditem{letbibmacro}{alias}{name}
\cmditem*{letbibmacro*}{alias}{name}

This command defines the macro \prm{alias} to be an alias of the macro \prm{name}. The definition is perfomed by \cmd{csletcs}.
An error is issued if \prm{name} is undefined.
The regular variant of this command sanitizes \prm{name} while the starred variant does not.

\cmditem{usebibmacro}{name}
\cmditem*{usebibmacro*}{name}

该命令执行由\cmd{newbibmacro} 定义的宏\prm{name}。如果宏带参数,只要简单的跟在\prm{name} 后面即可。该命令的常规形式会处理(净化,改造,sanitize)\prm{name},而带星的命令不会。
%This command executes the macro \prm{name}, as defined with \cmd{newbibmacro}. If the macro takes any arguments, they are simply appended after \prm{name}. The regular variant of this command sanitizes
%\prm{name} while the starred variant does not.

\cmditem{savecommand}{command}
\cmditem{restorecommand}{command}

这两个命令用来保存和恢复\prm{command},其中\prm{command} 必须是以斜杠开头的命令。两个命令都在局部范围内起作用。它们主要用于本地化文件中。
%These commands save and restore any \prm{command}, which must be a command name starting with a backslash. Both commands work within a local scope. They are mainly provided for use in localisation files.

\cmditem{savebibmacro}{name}
\cmditem{restorebibmacro}{name}

这两个命令用来保存和恢复宏\prm{name},其中\prm{name} 由\cmd{newbibmacro} 定义的宏的标识。两个命令都在局部范围内起作用。它们主要用于本地化文件中。
%These commands save and restore the macro \prm{name}, where \prm{name} is the identifier of a macro defined with \cmd{newbibmacro}. Both commands work within a local scope. They are mainly provided for use in localisation files.

\cmditem{savefieldformat}[entry type]{format}
\cmditem{restorefieldformat}[entry type]{format}

这两个命令用来保存和恢复格式化指令\prm{format},其中\prm{format} 由\cmd{DeclareFieldFormat} 定义。两个命令都在局部范围内起作用。它们主要用于本地化文件中。
%These commands save and restore the formatting directive \prm{format}, as defined with \cmd{DeclareFieldFormat}. Both commands work within a local scope. They are mainly provided for use in localisation files.

\cmditem{savelistformat}[entry type]{format}
\cmditem{restorelistformat}[entry type]{format}

这两个命令用来保存和恢复格式化指令\prm{format},其中\prm{format} 由\cmd{DeclareListFormat} 定义。两个命令都在局部范围内起作用。它们主要用于本地化文件中。
%These commands save and restore the formatting directive \prm{format}, as defined with \cmd{DeclareListFormat}. Both commands work within a local scope. They are mainly provided for use in localisation files.

\cmditem{savenameformat}[entry type]{format}
\cmditem{restorenameformat}[entry type]{format}

这两个命令用来保存和恢复格式化指令\prm{format},其中\prm{format} 由\cmd{DeclareNameFormat} 定义。两个命令都在局部范围内起作用。它们主要用于本地化文件中。
%These commands save and restore the formatting directive \prm{format}, as defined with \cmd{DeclareNameFormat}. Both commands work within a local scope. They are mainly provided for use in localisation files.

\cmditem{savelistwrapperformat}[entry type]{format}
\cmditem{restorelistwrapperformat}[entry type]{format}

%These commands save and restore the formatting directive \prm{format}, as defined with \cmd{DeclareListWrapperFormat}. Both commands work within a local scope. They are mainly provided for use in localisation files.
这些命令用于保存和恢复\cmd{DeclareListWrapperFormat}定义的格式,通常在局部范围使用,主要用于本地化文件处理中。


\cmditem{savenamewrapperformat}[entry type]{format}
\cmditem{restorenamewrapperformat}[entry type]{format}

%These commands save and restore the formatting directive \prm{format}, as defined with \cmd{DeclareNameWrapperFormat}. Both commands work within a local scope. They are mainly provided for use in localisation files.
这些命令用于保存和恢复\cmd{DeclareNameWrapperFormat}定义的格式,通常在局部范围使用,主要用于本地化文件处理中。



\cmditem{ifbibmacroundef}{name}{true}{false}

如果参考文献宏\prm{name} 未定义,展开为\prm{true} 否则为\prm{false}。
%Expands to \prm{true} if the bibliography macro \prm{name} is undefined, and to \prm{false} otherwise.

\cmditem{iffieldformatundef}[entry type]{name}{true}{false}
\cmditem{iflistformatundef}[entry type]{name}{true}{false}
\cmditem{ifnameformatundef}[entry type]{name}{true}{false}
\cmditem{iflistwrapperformatundef}[entry type]{name}{true}{false}
\cmditem{ifnamewrapperformatundef}[entry type]{name}{true}{false}
如果参考文献格式化指令\prm{format} 未定义,展开为\prm{true} 否则为\prm{false}。
%Expands to \prm{true} if the formatting directive \prm{format} is undefined, and to \prm{false}
%otherwise.

\cmditem{usedriver}{code}{entrytype}

执行\prm{entrytype} 类条目的参考文献驱动。在由\cmd{DeclareCiteCommand} 定义的标注命令的\prm{loopcode} 中调用该命令是打印类似于一个参考文献条目的完整标注的简单方法。诸如\cmd{newblock} 等命令无法用于标注,自动省略。附加的初始化命令可以通过\prm{code} 参数传递。该参数在一个编组内执行,这一编组用于运行相应驱动。注意: 该参数语法上是必须的,但可以留空。也要注意如果\opt{autolang} 包选项启用的话,该命令会自动切换语言。
%Executes the bibliography driver for an \prm{entrytype}. Calling this command in the \prm{loopcode} of a citation command defined with \cmd{DeclareCiteCommand} is a simple way to print full citations similar to a bibliography entry. Commands such as \cmd{newblock}, which are not applicable in a citation, are disabled automatically. Additional initialization commands may be passed as the \prm{code} argument. This argument is executed inside the group in which \cmd{usedriver} runs the respective driver. Note that it is mandatory in terms of the syntax but may be left empty. Also note that this command will automatically switch languages if the \opt{autolang} package option is enabled.

\cmditem{bibhypertarget}{name}{text}

\sty{hyperref} 的\cmd{hypertarget} 命令的封套(包围命令)。\prm{name} 是超链接锚的名字,\prm{text} 的内容作为超链接锚,可以是任意可打印文字或代码。如果文档中存在\env{refsection} 环境,\prm{name} 是基于当前refsection环境。如果\opt{hyperref} 包选项未启用或者\sty{hyperref} 包未加载,该命令简单的传递\prm{text} 变量。另可参见\secref{aut:fmt:ich} 节的格式化指令\texttt{bibhypertarget}。
%A wrapper for \sty{hyperref}'s \cmd{hypertarget} command. The \prm{name} is the name of the anchor, the \prm{text} is arbitrary printable text or code which serves as an anchor. If there are any \env{refsection} environments in the document, the \prm{name} is local to the current environment. If the \opt{hyperref} package option is disabled or the \sty{hyperref} package has not been loaded, this command will simply pass on its \prm{text} argument. See also the formatting directive \texttt{bibhypertarget} in \secref{aut:fmt:ich}.

\cmditem{bibhyperlink}{name}{text}

\sty{hyperref} 的\cmd{hyperlink} 命令的封套。\prm{name} 是由\cmd{bibhypertarget} 定义的超链接锚的名字,\prm{text} 的内容将转变成超链接,可以是任意可打印文字或代码。如果文档中存在\env{refsection} 环境,\prm{name} 是基于当前refsection环境。如果\opt{hyperref} 包选项未启用或者\sty{hyperref} 包未加载,该命令简单的传递\prm{text} 变量。另可参见\secref{aut:fmt:ich} 节的格式化指令\texttt{bibhyperlink}。
%A wrapper for \sty{hyperref}'s \cmd{hyperlink} command. The \prm{name} is the name of an anchor defined with \cmd{bibhypertarget}, the \prm{text} is arbitrary printable text or code to be transformed into a link. If there are any \env{refsection} environments in the document, the \prm{name} is local to the current environment. If the \opt{hyperref} package option is disabled or the \sty{hyperref} package has not been loaded, this command will simply pass on its \prm{text} argument. See also the formatting directive \texttt{bibhyperlink} in \secref{aut:fmt:ich}.

\cmditem{bibhyperref}[entrykey]{text}

将\prm{text} 转变为指向参考文献表中的\prm{entrykey}(即某一条目)的内部链接。如果\prm{entrykey} 省略,该命令使用当前正在处理的条目的引用关键词。该命令用于将标注转换为可点击的超链接,可以链接到参考文献表中的相应条目。链接目标由\biblatex 自动标记。如果文档中有多个文献表,链接目标将是所有文献表中第一个出现的\prm{entrykey} 条目。如果文档中存在\env{refsection} 环境,则超链接基于当前refsection环境。另可参见\secref{aut:fmt:ich} 节的格式化指令\texttt{bibhyperref}。
%Transforms \prm{text} into an internal link pointing to \prm{entrykey} in the bibliography. If \prm{entrykey} is omitted, this command uses the key of the entry currently being processed. This command is employed to transform citations into clickable links pointing to the corresponding entry in the bibliography. The link target is marked automatically by \biblatex. If there are multiple bibliographies in a document, the target will be the first occurence of \prm{entrykey} in one of the bibliographies. If there are \env{refsection} environments, the links are local to the environment. See also the formatting directive \texttt{bibhyperref} in \secref{aut:fmt:ich}.

\cmditem{ifhyperref}{true}{false}

展开为\prm{true},如果\opt{hyperref} 包选项已启用(意味着\sty{hyperref} 包已加载),否则展开为\prm{false}。
%Expands to \prm{true} if the \opt{hyperref} package option is enabled (which implies that the \sty{hyperref} package has been loaded), and to \prm{false} otherwise.

\cmditem{docsvfield}{field}

类似于\sty{etoolbox} 包的\cmd{docsvlist} 命令,差别在于它的参数是一个域名。域的值将以一个英文逗号分隔(comma-separated)的列表进行解析。如果\prm{field} 未定义,该命令展开为空字符串。
%Similar to the \cmd{docsvlist} command from the \sty{etoolbox} package, except that it takes a field name as its argument. The value of this field is parsed as a comma"=separated list. If the \prm{field} is undefined, this command expands to an empty string.

\cmditem{forcsvfield}{handler}{field}

类似于\sty{etoolbox} 包的\cmd{forcsvlist} 命令,差别在于它的参数是一个域名。域的值将以一个英文逗号分隔(comma-separated)的列表进行解析。如果\prm{field} 未定义,该命令展开为空字符串。
%Similar to the \cmd{forcsvlist} command from the \sty{etoolbox} package, except that it takes a field name as its argument. The value of this field is parsed as a comma"=separated list. If the \prm{field} is undefined, this command expands to an empty string.

\cmditem{MakeCapital}{text}

类似于\cmd{MakeUppercase},但仅将\prm{text} 的第一个可打印字符转换为大写。注意: \cmd{MakeUppercase} 命令的限制也适用于这一命令。即: \prm{text} 中的所有命令必须是鲁棒的或者以\cmd{protect} 为前缀,因为在大写操作中\prm{text} 需要展开。除了Ascii字符和标准重音命令外,该命令也处理\sty{inputenc} 包的活动字符和\sty{babel} 包的缩略词。如果\prm{text} 以一个控制序列开头,不做任何大写操作。该命令是鲁棒的。
%Similar to \cmd{MakeUppercase} but only converts the first printable character in \prm{text} to uppercase. Note that the restrictions that apply to \cmd{MakeUppercase} also apply to this command. Namely, all commands in \prm{text} must either be robust or prefixed with \cmd{protect} since the \prm{text} is expanded during capitalization. Apart from Ascii characters and the standard accent commands, this command also handles the active characters of the \sty{inputenc} package as well as the shorthands of the \sty{babel} package. If the \prm{text} starts with a control sequence, nothing is capitalized. This command is robust.

\cmditem{MakeSentenceCase}{text}
\cmditem*{MakeSentenceCase*}{text}

将\prm{text} 参数转换为sentence case(句子模式),即字符串中的第一个单词首字母大写而剩下其他部分转换为小写。该命令是鲁棒的。带星号的命令与常规命令(不带星号)的差别在于它能考虑条目的语言,根据\bibfield{langid} 域指定。只有当\bibfield{langid} 未定义或者值为由\cmd{DeclareCaseLangs} 命令(见后面)声明的某种语言时,它才将\prm{text} 转换为句子模式。\footnote{默认情况下,如下语言支持转换: \texttt{american}, \texttt{british}, \texttt{canadian}, \texttt{english}, \texttt{australian}, \texttt{newzealand} as well as the aliases \texttt{USenglish} and \texttt{UKenglish}. 要扩展或修改该列表请使用\cmd{DeclareCaseLangs} 命令。} 否则\prm{text} 不做任何改变。推荐使用\cmd{MakeSentenceCase*} 而不是常规命令。两个命令都支持\file{bib} 文件的传统\bibtex 规范,即: 遇到任何以花括号包围的内容大小写都不作变化,例如:
%Converts its \prm{text} argument to sentence case, \ie the first word is capitalized and the remainder of the string is converted to lowercase. This command is robust. The starred variant differs from the regular version in that it considers the language of the entry, as specified in the \bibfield{langid} field. If the \bibfield{langid} field is defined and holds a language declared with \cmd{DeclareCaseLangs} (see below)\footnote{By default, converting to sentence case is enabled for the following language identifiers: \texttt{american}, \texttt{british}, \texttt{canadian}, \texttt{english}, \texttt{australian}, \texttt{newzealand} as well as the aliases \texttt{USenglish} and \texttt{UKenglish}. Use \cmd{DeclareCaseLangs} to extend or change this list.}, then the sentence case conversion is performed. If the \bibfield{langid} field is undefined, then the language list declared with \cmd{DeclareCaseLangs} is checked for the presence of the main document language derived from the \opt{language} option. If found, sentence case conversion is performed, if not, the \prm{text} is not altered in any way. It is recommended to use \cmd{MakeSentenceCase*} rather than the regular variant in formatting directives. Both variants support the traditional \bibtex convention for \file{bib} files that anything wrapped in a pair of curly braces is not modified when changing the case. For example:

\begin{ltxexample}
\MakeSentenceCase{an Introduction to LaTeX}
\MakeSentenceCase{an Introduction to {LaTeX}}
\end{ltxexample}
%
将得到:
%would yield:

\begin{lstlisting}[style=plain]{}
An introduction to latex
An introduction to LaTeX
\end{lstlisting}
%
在以传统\bibtex 方式设计的\file{bib} 文件中,为阻止字母的大小写变化(case-changing),将单个字母用花括号包围是一种相当常见的方法。
%In \file{bib} files designed with traditional \bibtex in mind, it has been fairly common to only wrap single letters in braces to prevent case"=changing:

\begin{lstlisting}[style=bibtex]{}
title = {An Introduction to {L}a{T}e{X}}
\end{lstlisting}
%
这种方式存在一个问题是括号会压缩被包围字母两侧的字距。最好的方式是如第一个示例所示的那样,将整个单词都包围起来。标题中的宏也必须要用花括号仅保护。
%The problem with this convention is that the braces will suppress the kerning on both sides of the enclosed letter. It is preferable to wrap the entire word in braces as shown in the first example.
%Macros in titles must also be protected with braces

\begin{lstlisting}[style=bibtex]{}
title = {The {\TeX book}},
\end{lstlisting}
%
%Due to its complex implementation this command can not accept arbitrary input, it only safely operates on raw text or field data. In the standard styles the \texttt{title} and other \texttt{title}-like field formats do not work together with \cmd{MakeSentenceCase} because of their argument structure, so the standard styles offer a dedicated \texttt{titlecase} field format to apply this command. To enable sentence casing in standard styles for languages that support it you would use:

因为实现的复杂性,该命令不接受任意的输入,它仅在处理原始文本或域数据时才是安全。
在标准样式中,\texttt{title}类的域格式因为其参数结构不做\cmd{MakeSentenceCase}格式处理,
所以标准样式特意提供了\texttt{titlecase}域格式来实现这类命令。要在标准样式中对支持的语言启用句子模式,那么需要:

\begin{ltxexample}
\DeclareFieldFormat{titlecase}{<<\MakeSentenceCase*{#1}>>}
\end{ltxexample}
%
%Sentence casing can then be disabled by resetting that field format to
然后,句子模式设置可以通过如下域格式来取消:

\begin{ltxexample}
\DeclareFieldFormat{titlecase}{<<#1>>}
\end{ltxexample}

%Custom styles may follow a different approach, but style authors are encouraged to apply the same general ideas to their styles.
定制样式可以采用另一种方法,但鼓励样式作者采用相同的通用方法。

\cmditem{mkpageprefix}[pagination][postpro]{text}

该命令用于域格式化指令中,对标注命令的\prm{postnote} 参数和文献条目的\bibfield{pages} 域进行格式化。默认情况下,它将会解析\prm{text} 参数,并且以<p.> or <pp.>做为前缀。可选参数\prm{pagination} 保存指示pagination类型的域名,可以是\bibfield{pagination} 或\bibfield{bookpagination},默认是\bibfield{pagination}。前缀与\prm{text} 之间的间距可以通过重定义\cmd{ppspace} 命令来调整。默认是一个不可断行的词内空格。详见\secref{bib:use:pag, use:cav:pag}。另可参见\cmd{DeclareNumChars}, \cmd{DeclareRangeChars}, \cmd{DeclareRangeCommands}, 和\cmd{NumCheckSetup}。可选参数\prm{postpro} 指定了用于对\prm{text} 后处理的宏。如果只给出一个可选参数,将作为\prm{pagination},下面是两个典型示例:
%This command is intended for use in field formatting directives which format the page numbers in the \prm{postnote} argument of citation commands and the \bibfield{pages} field of bibliography entries. It will parse its \prm{text} argument and prefix it with <p.> or <pp.> by default. The optional \prm{pagination} argument holds the name of a field indicating the pagination type. This may be either \bibfield{pagination} or \bibfield{bookpagination}, with \bibfield{pagination} being the default. The spacing between the prefix and the \prm{text} may be modified by redefining \cmd{ppspace}. The default is an unbreakable interword space. See \secref{bib:use:pag, use:cav:pag} for further details. See also \cmd{DeclareNumChars}, \cmd{DeclareRangeChars}, \cmd{DeclareRangeCommands}, and \cmd{NumCheckSetup}. The optional \prm{postpro} argument specifies a macro to be used for post-processing the \prm{text}. If only one optional argument is given, it is taken as \prm{pagination}. Here are two typical examples:

\begin{ltxexample}
\DeclareFieldFormat{postnote}{<<\mkpageprefix[pagination][\mknormrange]{#1}>>}
\DeclareFieldFormat{pages}{<<\mkpageprefix[bookpagination]{#1}>>}
\end{ltxexample}
%
第一个示例中的可选参数\bibfield{pagination} 可以省略。
%The optional argument \bibfield{pagination} in the first example is omissible.

\cmditem{mkpagetotal}[pagination][postpro]{text}

该命令类似于\cmd{mkpageprefix},差别在于它用于条目的\bibfield{pagetotal} 域,即它将打印«123 pages»而不是«page 123»。可选参数\prm{pagination} 默认是\bibfield{bookpagination}。在\prm{text} 和后缀之间的间距可由对\cmd{ppspace} 重定义进行调整。可选参数\prm{postpro} 指定了用于对\prm{text} 后处理的宏。如果只给出一个可选参数,将作为\prm{pagination},下面是一个典型示例:
%This command is similar to \cmd{mkpageprefix} except that it is intended for the \bibfield{pagetotal} field of bibliography entries, \ie it will print «123 pages» rather than «page 123». The optional \prm{pagination} argument defaults to \bibfield{bookpagination}. The spacing inserted between the pagination suffix and the \prm{text} may be modified by redefining the macro \cmd{ppspace}. The optional \prm{postpro} argument specifies a macro to be used for post-processing the \prm{text}. If only one optional argument is given, it is taken as \prm{pagination}. Here is a typical example:

\begin{ltxexample}
\DeclareFieldFormat{pagetotal}{<<\mkpagetotal[bookpagination]{#1}>>}
\end{ltxexample}
%
在本例中可选参数\bibfield{bookpagination} 可省略。pagination本地化字符串取自
\texttt{$<$pagination$>$total} 和 \texttt{$<$pagination$>$totals}。
%The optional argument \bibfield{bookpagination} is omissible in this case.
%The pagination strings are taken from \texttt{$<$pagination$>$total} and \texttt{$<$pagination$>$totals}.

\begin{table}
\tablesetup\lnstyle
\begin{tabularx}{\textwidth}{@{}>{\ttfamily}X@{}p{0.25\textwidth}@{}p{0.25\textwidth}@{}p{0.25\textwidth}@{}}
\toprule
\multicolumn{1}{@{}H}{Input} &
\multicolumn{3}{@{}H}{Output} \\
\cmidrule(r){1-1}\cmidrule{2-4}
& \multicolumn{1}{@{}H}{\ttfamily mincomprange=10}
& \multicolumn{1}{@{}H}{\ttfamily mincomprange=100}
& \multicolumn{1}{@{}H}{\ttfamily mincomprange=1000} \\
\cmidrule(r){2-2}\cmidrule(r){3-3}\cmidrule{4-4}
11--15		& 11--5		& 11--15	& 11--15	\\
111--115	& 111--5	& 111--5	& 111--115	\\
1111--1115	& 1111--5	& 1111--5	& 1111--5	\\
\cmidrule{2-4}
& \multicolumn{1}{@{}H}{\ttfamily maxcomprange=1000}
& \multicolumn{1}{@{}H}{\ttfamily maxcomprange=100}
& \multicolumn{1}{@{}H}{\ttfamily maxcomprange=10} \\
\cmidrule(r){2-2}\cmidrule(r){3-3}\cmidrule{4-4}
1111--1115	& 1111--5	& 1111--5	& 1111--5	\\
1111--1155	& 1111--55	& 1111--55	& 1111--1155	\\
1111--1555	& 1111--555	& 1111--1555	& 1111--1555	\\
\cmidrule{2-4}
& \multicolumn{1}{@{}H}{\ttfamily mincompwidth=1}
& \multicolumn{1}{@{}H}{\ttfamily mincompwidth=10}
& \multicolumn{1}{@{}H}{\ttfamily mincompwidth=100} \\
\cmidrule(r){2-2}\cmidrule(r){3-3}\cmidrule{4-4}
1111--1115	& 1111--5	& 1111--15	& 1111--115	\\
1111--1155	& 1111--55	& 1111--55	& 1111--155	\\
1111--1555	& 1111--555	& 1111--555	& 1111--555	\\
\bottomrule
\end{tabularx}
\caption{\cmd{mkcomprange} 设置}%\cmd{mkcomprange} setup
\label{aut:aux:tab1}
\end{table}

\cmditem{mkcomprange}[postpro]{text}
\cmditem*{mkcomprange*}[postpro]{text}

%This command, which is intended for use in field formatting directives, will parse its \prm{text} argument for page ranges and compress them. For example, «125--129» may be formatted as «125--9». You may configure the behavior of \cmd{mkcomprange} by adjusting the \latex counters \cnt{mincomprange}, \cnt{maxcomprange}, and \cnt{mincompwidth}, as illustrated in \tabref{aut:aux:tab1}. The default settings are \texttt{10}, \texttt{100000}, and \texttt{1}, respectively. This means that the command tries to compress as much as possible by default. Use \cmd{setcounter} to adjust the parameters. The scanner recognizes \cmd{bibrangedash} and hyphens as range dashes. It will normalize the dash by replacing any number of consecutive hyphens with \cmd{bibrangedash}. Lists of ranges delimited with \cmd{bibrangessep} are also supported. The backend will normalise any comma or semi-colons surrounded by optional space by replacing them with \cmd{bibrangessep}. If you want to hide a character from the list/range scanner for some reason, wrap the character or the entire string in curly braces. The optional \prm{postpro} argument specifies a macro to be used for post-processing the \prm{text}. This is important if you want to combine \cmd{mkcomprange} with other formatting macros which also need to parse their \prm{text} argument, such as \cmd{mkpageprefix}. Simply nesting these commands will not work as expected. Use the \prm{postpro} argument to set up the processing chain as follows:
该命令用于域格式化指令,将\prm{text} 参数解析为页码范围并且压缩这些范围。例如,«125--129»可能格式化为«125--9»。可以通过调整\latex 计数器\cnt{mincomprange}, \cnt{maxcomprange} 和 \cnt{mincompwidth} 来设置\cmd{mkcomprange} 的操作方式,如表\ref{aut:aux:tab1} 所示。默认的设置分别是\texttt{10}, \texttt{100000}, and \texttt{1}。
这意味着该命令默认是尽可能的压缩。使用\cmd{setcounter} 来调整参数。扫描程序将\cmd{bibrangedash} 和hyphens作为范围间隔符。它通过将使用\cmd{bibrangedash} 的任意数量连续连字符替换破折号(dash)实现正规化。支持以\cmd{bibrangessep} 分隔的多范围列表。后端会将逗号或冒号(commas/semicolon)的多范围分隔符转换为\cmd{bibrangessep}。如果因为某些原因需要对list/range扫描程序隐藏一个字符,那么可以将该字符或者整个字符串用花括号包围起来。可选参数\prm{postpro} 指定了一个用于对\prm{text} 进行后处理的宏。这对于需要将\cmd{mkcomprange} 和其它也要解析它们自身的\prm{text} 参数的格式化宏(比如\cmd{mkpageprefix})联合使用非常重要。简单的嵌套可能无法如期正常工作。使用\prm{postpro} 参数构建的处理链,如下:


\begin{ltxexample}
\DeclareFieldFormat{postnote}{\mkcomprange[<<{>>\mkpageprefix[pagination]<<}>>]{#1}}
\end{ltxexample}
%
注意:\cmd{mkcomprange} 命令首先处理,\cmd{mkpageprefix} 则作为后处理器。也要注意\prm{postpro} 被额外的一对花括号包围。这仅在特殊情况下需要,为阻止\latex 的可选参数扫描器与嵌套的方括号混淆。带星的命令与不带星命令的差别是它的\prm{postpro} 参数应用于列表中的各项,例如:
%Note that \cmd{mkcomprange} is executed first, using \cmd{mkpageprefix} as post-processor. Also note that the \prm{postpro} argument is wrapped in an additional pair of braces. This is only required in this particular case to prevent \latex's optional argument scanner from getting confused by the nested brackets. The starred version of this command differs from the regular one in the way the \prm{postpro} argument is applied to a list of values. For example:

\begin{ltxexample}
\mkcomprange[\mkpageprefix]{5, 123-129, 423-439}
\mkcomprange*[\mkpageprefix]{5, 123-129, 423-439}
\end{ltxexample}
%
将输出:
%will output:

\begin{ltxexample}
pp. 5, 123-9, 423-39
p. 5, pp. 123-9, pp. 423-39
\end{ltxexample}

\cmditem{mknormrange}[postpro]{text}
\cmditem*{mknormrange*}[postpro]{text}

%This command, which is intended for use in field formatting directives, will parse its \prm{text} argument for page ranges and will normalise them. The command is similar to \cmd{mkcomprange} except that the page ranges will not be compressed. The scanner recognises \cmd{bibrangedash} and hyphens as range dashes. It will normalize the dash by replacing any number of consecutive hyphens with \cmd{bibrangedash}. Lists of ranges delimited with \cmd{bibrangessep} are also supported. The scanner will normalise any comma or semi-colons surrounded by optional space by replacing them with \cmd{bibrangessep}. If you want to hide a character from the list/range scanner for some reason, wrap the character or the entire string in curly braces. The optional \prm{postpro} argument specifies a macro to be used for post-processing the \prm{text}. See \cmd{mkcomprange} on how to use this argument. The starred version of this command differs from the regular one in the way the \prm{postpro} argument is applied to a list of values.
该命令用于域的格式化指令中,将会解析其\prm{text}参数来实现和规范页码范围。
该命令类似于\cmd{mkcomprange},不同在于页码范围不被压缩。
扫描器识别\cmd{bibrangedash},并处理为一个范围破折号。
它会规范破折号来代替连续的\cmd{bibrangedash}。以\cmd{bibrangessep}分隔的范围列表,也支撑这一处理。
扫描器将规范任意逗号或分号为\cmd{bibrangessep}。如果想要隐藏他们,那么将其或整个字符串用花括号包围起来。
可选参数\prm{postpro}指定了\prm{text}后处理中用的宏。如何使用该参数见\cmd{mkcomprange}。带星号的版本差别在于\prm{postpro}参数应用于列表。


\cmditem{mkfirstpage}[postpro]{text}
\cmditem*{mkfirstpage*}[postpro]{text}

该命令用于域格式化指令,将\prm{text} 参数解析为页码范围并且仅打印这些范围的起始页码。扫描程序将\cmd{bibrangedash} 和hyphens作为范围间隔符。支持范围列表以\cmd{bibrangessep} 分隔。如果因为某些原因需要对list/range扫描程序隐藏一个字符,那么可以将该字符或者整个字符串用花括号包围起来。可选参数\prm{postpro} 指定了一个用于对\prm{text} 进行后处理的宏。怎么使用该参数见\cmd{mkcomprange} 命令。带星的命令的差别在于\prm{postpro} 参数应用于列表的各项。例如:
%This command, which is intended for use in field formatting directives, will parse its \prm{text} argument for page ranges and print the start page of the range only. The scanner recognizes \cmd{bibrangedash} and hyphens as range dashes. Lists of ranges delimited with \cmd{bibrangessep} are also supported. If you want to hide a character from the list/range scanner for some reason, wrap the character or the entire string in curly braces. The optional \prm{postpro} argument specifies a macro to be used for post-processing the \prm{text}. See \cmd{mkcomprange} on how to use this argument. The starred version of this command differs from the regular one in the way the \prm{postpro} argument is applied to a list of values. For example:

\begin{ltxexample}
\mkfirstpage[\mkpageprefix]{5, 123-129, 423-439}
\mkfirstpage*[\mkpageprefix]{5, 123-129, 423-439}
\end{ltxexample}
%
将输出:
%will output:

\begin{ltxexample}
p. 5, 123, 423
p. 5, p. 123, p. 423
\end{ltxexample}

\cmditem{rangelen}{rangefield}
该命令将其参数解析为一个范围,并返回范围的长度。计算由\biber 负责可以处理很多特殊情况。对于无终点的(开口的)范围将返回-1。特别的,\cmd{rangelen} 可以:
%这可以作为样式中一些判断的一部分,例如将<f>作为只有两页的范围的后缀,比如范围<36-37>将打印<36f>。这可以通过命令\cmd{ifnumcomp} 实现:
%Takes the name of a bibfield declared as a range field in the data model and returns the length of the range. This is calculated by \biber and can handle many special cases. It will return $-1$ for open ended ranges. Specifically  can:

\begin{itemize}
\item 计算同一个域中多个范围比如<1-10, 20-30>的总数
%Calculate the total of multiple ranges in the same field such as <1-10, 20-30>
\item 处理隐含的范围比如<22-4> and <130-33>
%Handle implicit ranges such as <22-4> and <130-33>
\item 处理大小写的罗马数字范围,处理ASCII码和统一码表示的罗马数字范围。
%Handle roman numeral ranges in upper and lower case and consisting of both ASCII and Unicode roman numeral representations.
\end{itemize}
%
下面是一些示例:
%Here are some examples:

\begin{tabular}{ll}
pages = <10> & |\rangelen{pages}| returns '1'\\
pages = <10-15> & |\rangelen{pages}| returns '6'\\
pages = <10-15,47-53> & |\rangelen{pages}| returns '13'\\
pages = <10-> & |\rangelen{pages}| returns '-1'\\
pages = <-10> & |\rangelen{pages}| returns '-1'\\
pages = <48-9> & |\rangelen{pages}| returns '2'\\
pages = <172-77> & |\rangelen{pages}| returns '6'\\
pages = <i-vi> & |\rangelen{pages}| returns '6'\\
pages = <X-XX> & |\rangelen{pages}| returns '11'\\
pages = <ⅥⅠ-ⅻ> & |\rangelen{pages}| returns '6'\\
pages = <ⅥⅠ-ⅻ, 145-7, 135-39> & |\rangelen{pages}| returns '14'
\end{tabular}

\cmd{rangelen} 命令可以用于判断中:
%The \cmd{rangelen} command can be used in tests:

\begin{ltxexample}
\ifnumcomp{\rangelen{pages}}{=}{1}{add 'f'}{do nothing}
\end{ltxexample}

\cmditem{DeclareNumChars}{characters}
\cmditem*{DeclareNumChars*}{characters}

该命令设置\secref{aut:aux:tst} 节的\cmd{ifnumeral}, \cmd{ifnumerals} 和\cmd{ifpages} 命令。该设置也将影响\cmd{iffieldnum}, \cmd{iffieldnums}, \cmd{iffieldpages}, \cmd{mkpageprefix} 和 \cmd{mkpagetotal} 命令。\prm{characters} 参数是一个无分隔符的符号列表,将作为数字的一部分进行处理。不带星命令将替换当前设置,带星命令则将其参数附加到当前列表中。默认设置为:
%This command configures the \cmd{ifnumeral}, \cmd{ifnumerals}, and \cmd{ifpages} tests from \secref{aut:aux:tst}. The setup will also affect \cmd{iffieldnum}, \cmd{iffieldnums}, \cmd{iffieldpages} as well as \cmd{mkpageprefix} and \cmd{mkpagetotal}. The \prm{characters} argument is an undelimited list of characters which are to be considered as being part of a number. The regular version of this command replaces the current setting, the starred version appends its argument to the current list. The default setting is:

\begin{ltxexample}
\DeclareNumChars{.}
\end{ltxexample}
%
这意味着(节或者其他)数值比如 <3.4.5> 将被认为是一个数字。注意,默认检测的是阿拉伯和罗马数字,没有必要对此做显式声明。
%This means that a (section or other) number like <3.4.5> will be considered as a number. Note that Arabic and Roman numerals are detected by default, there is no need to declare them explicitly.

\cmditem{DeclareRangeChars}{characters}
\cmditem*{DeclareRangeChars*}{characters}

该命令设置\secref{aut:aux:tst} 的\cmd{ifnumerals} 和\cmd{ifpages} 命令。其设置还将影响\cmd{iffieldnums}, \cmd{iffieldpages} ,\cmd{mkpageprefix} 和\cmd{mkpagetotal}。\prm{characters} 参数是一个无分隔符的符号列表,将作为范围指示符进行处理。不带星命令将替换当前设置,带星命令则将其参数附加到当前列表中。默认设置为:
%This command configures the \cmd{ifnumerals} and \cmd{ifpages} tests from \secref{aut:aux:tst}. The setup will also affect \cmd{iffieldnums} and \cmd{iffieldpages} as well as \cmd{mkpageprefix} and \cmd{mkpagetotal}. The \prm{characters} argument is an undelimited list of characters which are to be considered as range indicators. The regular version of this command replaces the current setting, the starred version appends its argument to the current list. The default setting is:

\begin{ltxexample}
\DeclareRangeChars{~,;-+/}
\end{ltxexample}

For engines that fully support Unicode these defaults are extended with
\begin{ltxexample}[escapeinside={(*@}{@*)}]
\DeclareRangeChars*{(*@–—@*)}
\end{ltxexample}
%
这意味着比如<3--5>, <35+>, <8/9>等字符串会被\cmd{ifnumerals} 和\cmd{ifpages} 认为是一个范围。这些字符串中的非范围字符将被认为是数字。因此,类似于<3a--5a>和<35b+>之类的字符串默认情况下不被认为是范围。更多细节详见\secref{bib:use:pag, use:cav:pag}。
%This means that strings like <3--5>, <35+>, <8/9> and so on will be considered as a range by \cmd{ifnumerals} and \cmd{ifpages}. Non-range characters in such strings are recognized as numbers. So strings like <3a--5a> and <35b+> are not deemed to be ranges by default. See also \secref{bib:use:pag, use:cav:pag} for further details.

\cmditem{DeclareRangeCommands}{commands}
\cmditem*{DeclareRangeCommands*}{commands}

该命令类似于\cmd{DeclareRangeChars},差别在于\prm{commands} 参数是一个无分隔符的命令列表,将被视为范围指示符。不带星命令将替换当前设置,带星命令则将其参数附加到当前列表中。默认列表相当长,应该覆盖所有一般情况。下面是一个简单示例:
%This command is similar to \cmd{DeclareRangeChars}, except that the \prm{commands} argument is an undelimited list of commands which are to be considered as range indicators. The regular version of this command replaces the current setting, the starred version appends its argument to the current list. The default list is rather long and should cover all common cases; here is a shorter example:

\begin{ltxexample}
\DeclareRangeCommands{\&\bibrangedash\textendash\textemdash\psq\psqq}
\end{ltxexample}
%
更多细节参见\secref{bib:use:pag, use:cav:pag}。
%See also \secref{bib:use:pag, use:cav:pag} for further details.

\cmditem{DeclarePageCommands}{commands}
\cmditem*{DeclarePageCommands*}{commands}

该命令类似于\cmd{DeclareRangeCommands},差别在于它仅影响\cmd{ifpages} 和\cmd{iffieldpages} 判断,而不影响\cmd{ifnumerals} 和\cmd{iffieldnums}。默认设置为:
%This command is similar to \cmd{DeclareRangeCommands}, except that it only affects the \cmd{ifpages} and \cmd{iffieldpages} tests but not \cmd{ifnumerals} and \cmd{iffieldnums}. The default setting is:

\begin{ltxexample}
\DeclarePageCommands{\pno\ppno}
\end{ltxexample}

\cmditem{NumCheckSetup}{code}

该命令用于临时重定义一些将与\secref{aut:aux:tst} 节的\cmd{ifnumeral}, \cmd{ifnumerals}, \cmd{ifpages} 命令执行的判断产生冲突的命令。该设置也将影响\cmd{iffieldnum}, \cmd{iffieldnums}, \cmd{iffieldpages}, \cmd{mkpageprefix} 和\cmd{mkpagetotal}。这些命令将在组内执行\prm{code}。因为上述命令将展开为字符串用于分析,可以利用将冲突命令展开为空字符串(将被判断命令忽略)的方式来移除这些命令。更多细节参见\secref{bib:use:pag, use:cav:pag}。
%Use this command to temporarily redefine any commands which interfere with the tests performed by \cmd{ifnumeral}, \cmd{ifnumerals}, and \cmd{ifpages} from \secref{aut:aux:tst}. The setup will also affect \cmd{iffieldnum}, \cmd{iffieldnums}, \cmd{iffieldpages} as well as \cmd{mkpageprefix} and \cmd{mkpagetotal}. The \prm{code} will be executed in a group by these commands. Since the above mentioned commands will expand the string to be analyzed, it is possible to remove commands to be ignored by the tests by making them expand to an empty string. See also \secref{bib:use:pag, use:cav:pag} for further details.

\cmditem{DeclareCaseLangs}{languages}
\cmditem*{DeclareCaseLangs*}{languages}

定义语言列表,该列表在\cmd{MakeSentenceCase*} 命令将一个字符串转换成句子模式时考虑。\prm{languages} 参数是一个由\sty{babel}/\sty{polyglossia} 语言标识构成的逗号分隔的列表。不带星命令用于替换当前设置,而带星的命令用于附加当前列表。默认的设置为:
%Defines the list of languages which are considered by the \cmd{MakeSentenceCase*} command as it converts a string to sentence case. The \prm{languages} argument is a comma"=separated list of \sty{babel}/\sty{polyglossia} language identifiers. The regular version of this command replaces the current setting, the starred version appends its argument to the current list. The default setting is:

\begin{ltxexample}
\DeclareCaseLangs{%
  american,british,canadian,english,australian,newzealand,USenglish,UKenglish}
\end{ltxexample}
%
语言标识的列表见\sty{babel}/\sty{polyglossia} 手册和表\ref{bib:fld:tab1}。
%See the \sty{babel}/\sty{polyglossia} manuals and \tabref{bib:fld:tab1} for a list of languages identifiers.

\cmditem{BibliographyWarning}{message}

该命令类似于\cmd{PackageWarning},但打印内容除了输入行号外还有当前处理条目的引用关键词。如果\prm{message} 相当长,可以使用\cmd{MessageBreak} 命令来断行。注意: 标准的\cmd{PackageWarning} 命令在参考文献中使用时无法提供一个有意义的提示,因为其打印的输入行号只是\cmd{printbibliography} 命令所在的行号。
%This command is similar to \cmd{PackageWarning} but prints the entry key of the entry currently being processed in addition to the input line number. It may be used in the bibliography as well as in citation commands. If the \prm{message} is fairly long, use \cmd{MessageBreak} to include line breaks. Note that the standard \cmd{PackageWarning} command does not provide a meaningful clue when used in the bibliography since the input line number is the line on which the \cmd{printbibliography} command was given.

%该命令用于\file{cbx}\slash\file{bbx} 文件和\file{bib} 文件的\texttt{@preamble}(导言)中。它检测选择的后端程序,如果后端不是\biber 则发出警告。可选参数\prm{severity} 是一个整数用于指定警告的严重程度。当值等于1时将触发一个消息陈述推荐使用\biber;当值等于2触发一个警告陈述需要使用\biber ,否则样式或\slash\file{bib} 文件可能无法正常工作;当值等于3触发一个错误信息陈述\biber 是严格必须的,样式或\slash\file{bib} 文件在其它后端下将无法正常工作。如果\cmd{RequireBiber} 命令多次使用,\prm{severity} 应取最大值。\file{cbx}\slash\file{bbx} 文件和所有\file{bib} 文件的\texttt{@preamble}(导言)部分作为两个方面,其追踪是分别进行的。如果可选参数\prm{severity} 缺省,那么默认值是2(触发一个警告信息)。

\boolitem{pagetracker}
\leavevmode\vspace{\numexpr2\baselineskip}% fix margin spilling into the text

这些命令将启用或关闭局部引用追踪器(这将影响来自\secref{aut:aux:tst} 节的\cmd{iffirstonpage} 和\cmd{ifsamepage} 判断)。可在标注命令定义或者正文中的任意位置使用。要使标注命令完全排除页码追踪,可以在\cmd{DeclareCiteCommand} 命令的\prm{precode} 参数中使用\cmd{pagetrackerfalse}。详见\secref{aut:cbx:cbx}。注意: 当全局页码追踪关闭时,这些命令无效。
%These commands activate or deactivate the citation tracker locally (this will affect the \cmd{iffirstonpage} and \cmd{ifsamepage} test from \secref{aut:aux:tst}). They are intended for use in the definition of citation commands or anywhere in the document body. If a citation command is to be excluded from page tracking, use \cmd{pagetrackerfalse} in the \prm{precode} argument of \cmd{DeclareCiteCommand}. See \secref{aut:cbx:cbx} for details. Note that these commands have no effect if page tracking has been disabled globally.

\boolitem{citetracker}
\leavevmode\vspace{\numexpr2\baselineskip}% fix margin spilling into the text

这些命令将启用或关闭所有的局部引用追踪器(这将影响来自\secref{aut:aux:tst} 节的 \cmd{ifciteseen}, \cmd{ifentryseen}, \cmd{ifciteibid}, 和\cmd{ifciteidem} 判断)。可在标注命令定义或者正文中的任意位置使用。要使标注命令完全排除页码追踪,可以在\cmd{DeclareCiteCommand} 命令的\prm{precode} 参数中使用\cmd{citetrackerfalse}。详见\secref{aut:cbx:cbx}。注意: 当全局追踪关闭时,这些命令无效。
%These commands activate or deactivate all citation trackers locally (this will affect the \cmd{ifciteseen}, \cmd{ifentryseen}, \cmd{ifciteibid}, and \cmd{ifciteidem} tests from \secref{aut:aux:tst}). They are intended for use in the definition of citation commands or anywhere in the document body. If a citation command is to be excluded from tracking, use \cmd{citetrackerfalse} in the \prm{precode} argument of \cmd{DeclareCiteCommand}. See \secref{aut:cbx:cbx} for details. Note that these commands have no effect if tracking has been disabled globally.

\boolitem{backtracker}
\leavevmode\vspace{\numexpr2\baselineskip}% fix margin spilling into the text

这些命令将启用或关闭所有的局部\texttt{backref} 追踪器。可在标注命令定义或者正文中的任意位置使用。要使标注命令完全排除反向链接追踪,可以在\cmd{DeclareCiteCommand} 命令的\prm{precode} 参数中使用\cmd{backtrackerfalse}。注意: 当\texttt{backref} 选项未进行全局设置,这些命令无效。
%These commands activate or deactivate the \texttt{backref} tracker locally. They are intended for use in the definition of citation commands or anywhere in the document body. If a citation command is to be excluded from backtracking, use \cmd{backtrackerfalse} in the \prm{precode} argument of \cmd{DeclareCiteCommand}. Note that these commands have no effect if the \texttt{backref} option has been not been set globally.

\end{ltxsyntax}

\subsection[标点和间距]{标点和间距}%Punctuation and Spacing
\label{aut:pct}

\biblatex 宏包提供了设计用来在文献著录表和标注中管理和追踪标点与空格的精致工具。这些工具在两个层面工作。\secref{aut:pct:new} 节讨论的高层(high-level)命令处理由著录样式在一个参考文献条目的不同部分插入的标点和空格。而\secref{aut:pct:chk, aut:pct:pct, aut:pct:spc} 节中的命令在更低一层工作。它们以一种健壮高效的方式使用\tex 的\gls{间距因子} 并间距因子修改代码来追踪标点。采用这种方法,不仅可以检测在域间显式插入的标点符号,也可以检测在域内末尾的标点符号。相同的技术也用在本地化字符串的自动大写处理过程中,详见\secref{aut:pct:cfg} 节的\cmd{DeclareCapitalPunctuation} 命令以及\secref{aut:str} 节。注意: 这些工具仅在标注和文献著录表内局部使用。它们不会影响正文中其他任何部分。
%The \biblatex package provides elaborate facilities designed to manage and track punctuation and spacing in the bibliography and in citations. These facilities work on two levels. The high"=level commands discussed in \secref{aut:pct:new} deal with punctuation and whitespace inserted by the bibliography style between the individual segments of a bibliography entry. The commands in \secref{aut:pct:chk, aut:pct:pct, aut:pct:spc} work at a lower level. They use \tex's space factor and modified space factor codes to track punctuation in a robust and efficient way. This way it is possible to detect trailing punctuation marks within fields, not only those explicitly inserted between fields. The same technique is also used for automatic capitalization of localisation strings, see \cmd{DeclareCapitalPunctuation} in \secref{aut:pct:cfg} as well as \secref{aut:str} for details. Note that these facilities are only made available locally in citations and bibliographies. They will not affect any other part of a document.

\subsubsection{块和单元的标点}%Block and Unit Punctuation
\label{aut:pct:new}

参考文献条目的主要组成部分是块(<blocks>)和单元(<units>)。其中块更大,单元相对较小,至多长度上与块相等。例如,诸如\bibfield{title} 或\bibfield{note} 等域的值常构成一个单元,并且用一个句号或逗号与后面的数据分隔开来。一个块可以由多个域构成,这些域被认为是独立的单元,例如\bibfield{publisher}, \bibfield{location} 和\bibfield{year}。一个条目如何划分成块和单元完全是由著录样式决定的。条目内容的划分通过在合适位置插入\cmd{newblock} 和\cmd{newunit} 命令,在末尾插入\cmd{finentry} 命令(见\secref{aut:bbx:drv} 的示例)来实现。一些实用提示见\secref{aut:cav:pct} 节。
%The major segments of a bibliography entry are <blocks> and <units>. A block is the larger segment of the two, a unit is shorter or at most equal in length. For example, the values of fields such as \bibfield{title} or \bibfield{note} usually form a unit which is separated from subsequent data by a period or a comma. A block may comprise several fields which are treated as separate units, for example \bibfield{publisher}, \bibfield{location}, and \bibfield{year}. The segmentation of an entry into blocks and units is at the discretion of the bibliography style. An entry is segmented by inserting \cmd{newblock} and \cmd{newunit} commands at suitable places and \cmd{finentry} at the very end (see \secref{aut:bbx:drv} for an example). See also \secref{aut:cav:pct} for some practical hints.

\begin{ltxsyntax}

\csitem{newblock}

标记一个块的终点。该命令不打印任何内容,仅标记块的结束。块的分隔符\cmd{newblockpunct} 将由后面跟着的\cmd{printtext}, \cmd{printfield}, \cmd{printlist}, \cmd{printnames},或\cmd{bibstring} 命令插入。可以在合适的位置使用 \cmd{newblock} 而不需要担心产生多余的块。一个新的块仅在其后是\cmd{printfield}(或类似)命令且打印一些内容的情况下开始。详见\secref{aut:cav:pct}。
%Records the end of a block. This command does not print anything, it merely marks the end of the block. The block delimiter \cmd{newblockpunct} will be inserted by a subsequent \cmd{printtext}, \cmd{printfield}, \cmd{printlist}, \cmd{printnames}, or \cmd{bibstring} command. You may use \cmd{newblock} at suitable places without having to worry about spurious blocks. A new block will only be started by the next \cmd{printfield} (or similar) command if this command prints anything. See \secref{aut:cav:pct} for further details.

\csitem{newunit}

记录一个单元的终点,并将默认分隔符 \cmd{newunitpunct} 放入标点缓存中。该命令不打印任何内容,仅标记单元的结束。标点缓存将由接下来的\cmd{printtext}, \cmd{printfield}, \cmd{printlist}, \cmd{printnames}, 或\cmd{bibstring} 命令插入。可以在类似\cmd{printfield} 的命令后面使用\cmd{newunit},而不需要担心带来多余的标点和空格。标点缓存只能由后面的\cmd{printfield} 或者类似命令且打印的域非空时插入。这一方式也应用于\cmd{printtext}, \cmd{printlist}, \cmd{printnames} 和\cmd{bibstring} 命令。详见\secref{aut:cav:pct} 节。
%Records the end of a unit and puts the default delimiter \cmd{newunitpunct} in the punctuation buffer. This command does not print anything, it merely marks the end of the unit. The punctuation buffer will be inserted by the next \cmd{printtext}, \cmd{printfield}, \cmd{printlist}, \cmd{printnames}, or \cmd{bibstring} command. You may use \cmd{newunit} after commands like \cmd{printfield} without having to worry about spurious punctuation and whitespace. The buffer will only be inserted by the next \cmd{printfield} or similar command if \emph{both} fields are non"=empty. This also applies to \cmd{printtext}, \cmd{printlist}, \cmd{printnames}, and \cmd{bibstring}. See \secref{aut:cav:pct} for further details.

\csitem{finentry}

插入 \cmd{finentrypunct}。该命令应在每个参考文献条目的最末尾使用。
%Inserts \cmd{finentrypunct}. This command should be used at the very end of every bibliography entry.

\cmditem{setunit}{punctuation}
\cmditem*{setunit*}{punctuation}

\cmd{setunit} 类似于\cmd{newunit},除了它使用\prm{punctuation} 代替\cmd{newunitpunct}。带星号命令与不带星号命令的差别在于它检查前面最近的\cmd{printtext}, \cmd{printfield}, \cmd{printlist}, \cmd{printnames} 或 \cmd{bibstring} 命令是否实际打印内容。如果没有打印,该命令不做任何处理。
%The \cmd{setunit} command is similar to \cmd{newunit} except that it uses \prm{punctuation} instead of \cmd{newunitpunct}. The starred variant differs from the regular version in that it checks if the last \cmd{printtext}, \cmd{printfield}, \cmd{printlist}, \cmd{printnames}, or \cmd{bibstring} command did actually print anything. If not, it does nothing.

\cmditem{printunit}{punctuation}
\cmditem*{printunit*}{punctuation}

\cmd{printunit} 命令类似于\cmd{setunit},但\prm{punctuation} 继续存留在缓存中。这能确保\prm{punctuation} 能在由\cmd{printtext}, \cmd{printfield}, \cmd{printlist}, \cmd{printnames}, or \cmd{bibstring} 命令打印下一个非空域前插入---不管任何立即调用的\cmd{newunit} 或\cmd{setunit} 命令。(译者: 即该命令设置的标点只要后面还有非空域需要打印都会插入。)
%The \cmd{printunit} command is similar to \cmd{setunit} except that \prm{punctuation} persists in the buffer. This ensures that \prm{punctuation} is inserted before the next non"=empty field printed by the \cmd{printtext}, \cmd{printfield}, \cmd{printlist}, \cmd{printnames}, or \cmd{bibstring} commands---regardless of any intermediate calls to \cmd{newunit} or \cmd{setunit}.

\cmditem{setpunctfont}{command}

该命令用于域格式化指令中,提供了处理打印为另一种字体(比如斜体打印的标题)的域后面的单元标点的替换方式。标准的\latex 处理方式是在后面添加一个小的空格,称为斜体修正。该命令允许将标点调整为之前的域采用的字体。\prm{command} 应是带一个参数的text字体命令,比如\cmd{emph} 或\cmd{textbf}。该命令仅影响由\secref{aut:pct:pct} 节的命令插入的标点符号。字体调整仅应用于接下来的标点符号,处理之后将自动重设。如果希望在它生效之前手动重设,使用\cmd{resetpunctfont}。如果\opt{punctfont} 选项关闭,该命令不做任何处理。注意\secref{aut:fmt:ich} 节的\cmd{mkbibemph}, \cmd{mkbibitalic} 和\cmd{mkbibbold} 封套命令自动包含该功能。
%This command, which is intended for use in field formatting directives, provides an alternative way of dealing with unit punctuation after a field printed in a different font (for example, a title printed in italics). The standard \latex way of dealing with this is adding a small amount of space, the so-called italic correction. This command allows adapting the punctuation to the font of the preceding field. The \prm{command} should be a text font command which takes one argument, such as \cmd{emph} or \cmd{textbf}. This command will only affect punctuation marks inserted by one of the commands from \secref{aut:pct:pct}. The font adaption is applied to the next punctuation mark only and will be reset automatically thereafter. If you want to reset it manually before it takes effect, issue \cmd{resetpunctfont}. If the \opt{punctfont} package option is disabled, this command does nothing. Note that the \cmd{mkbibemph}, \cmd{mkbibitalic} and \cmd{mkbibbold}  wrappers from \secref{aut:fmt:ich} incorporate this feature by default.

\csitem{resetpunctfont}

该命令在其生效之前重设由\cmd{setpunctfont} 定义的单元标点字体。如果\opt{punctfont} 选项关闭,该命令不做任何处理。
%This command resets the unit punctuation font defined with \cmd{setpunctfont} before it takes effect. If the \opt{punctfont} package option is disabled, this command does nothing.

\end{ltxsyntax}

\subsubsection{标点判断}%Punctuation Tests
\label{aut:pct:chk}

下面的命令可以用来在标注和文献表的任意位置判断前面的标点符号。
%The following commands may be used to test for preceding punctuation marks at any point in citations and the bibliography.

\begin{ltxsyntax}

\cmditem{ifpunct}{true}{false}

如果前面的是除缩略点外的标点符号,执行\prm{true},否则执行\prm{false}。
%Executes \prm{true} if preceded by any punctuation mark except for an abbreviation dot, and \prm{false} otherwise.

\cmditem{ifterm}{true}{false}

如果前面的是一个终结标点(terminal punctuation mark),执行\prm{true},否则执行\prm{false}。终结标点是设置用来自动大写的任意标点符号,可以使用默认符号或者用\cmd{DeclareCapitalPunctuation} 设定,详见 \secref{aut:pct:cfg}。默认情况下,用句号,叹号和问号。
%Executes \prm{true} if preceded by a terminal punctuation mark, and \prm{false} otherwise. A terminal punctuation mark is any punctuation mark which has been registered for automatic capitalization, either with \cmd{DeclareCapitalPunctuation} or by default, see \secref{aut:pct:cfg} for details. By default, this applies to periods, exclamation marks, and question marks.

\cmditem{ifpunctmark}{character}{true}{false}

如果前面的是一个\prm{character} 标点符,执行\prm{true},否则执行\prm{false}。\prm{character} 可以是逗号,分号,冒号,句号,叹号,问号或星号。注意,一个句号代表一个结束句子的句号。使用星号用来判断缩略词后的点号。如果该命令用于姓名列表的格式化指令,即在\cmd{DeclareNameFormat} 的参数中,\prm{character} 也可以是撇号。
%Executes \prm{true} if preceded by the punctuation mark \prm{character}, and \prm{false} otherwise. The \prm{character} may be a comma, a semicolon, a colon, a period, an exclamation mark, a question mark, or an asterisk. Note that a period denotes an end-of"=sentence period. Use the asterisk to test for the dot after an abbreviation. If this command is used in a formatting directive for name lists, \ie in the argument to \cmd{DeclareNameFormat}, the \prm{character} may also be an apostrophe.

\cmditem{ifprefchar}{true}{false}

如果前面的是一个由\cmd{DeclarePrefChars} 声明的前缀符,执行\prm{true},否则执行\prm{false}。
%Executes \prm{true} if preceded by any prefix character declared by \cmd{DeclarePrefChars}.

\end{ltxsyntax}

\subsubsection{添加标点}%Adding Punctuation
\label{aut:pct:pct}
下面的命令设计用来防止重复标点。参考文献著录和标注样式总需要使用这些命令来代替原样输出的标点符号。本节中所有的\cmd{add...} 命令自动利用\cmd{unspace} 命令移除前面的空白(见\secref{aut:pct:spc})。注意:下面讨论的所有的\cmd{add...} 命令的作用是宏包默认的,无论\biblatex 换哪种语言都会重新恢复。其作用可以通过\cmd{DeclarePunctuationPairs} 命令进行调整,见\secref{aut:pct:cfg} 节。
%The following commands are designed to prevent double punctuation marks. Bibliography and citation styles should always use these commands instead of literal punctuation marks. All \cmd{add...} commands in this section automatically remove preceding whitespace with \cmd{unspace} (see \secref{aut:pct:spc}). Note that the behavior of all \cmd{add...} commands discussed below is the package default, which is restored whenever \biblatex switches languages. This behavior may be adjusted with \cmd{DeclarePunctuationPairs} from \secref{aut:pct:cfg}.

\begin{ltxsyntax}

\csitem{adddot}

除非前面输出的是任何一种标点符号,否则添加一个句点(period)。该命令的目的是在一个缩写后面插入点(dot)。以这种方式插入的点被认为与其它标点命令插入的标点性质相同。该命令也用来将前面插入的文本句点(原样输出的句点,literal period) 转换成一个缩写的点。
%Adds a period unless it is preceded by any punctuation mark. The purpose of this command is inserting the dot after an abbreviation. Any dot inserted this way is recognized as such by the other punctuation commands. This command may also be used to turn a previously inserted literal period into an abbreviation dot.

\csitem{addcomma}

除非前面输出的是一个逗号(comma)、分号(semicolon)、冒号(colon)和句点(period),否则添加一个逗号。
%Adds a comma unless it is preceded by another comma, a semicolon, a colon, or a period.

\csitem{addsemicolon}

除非前面输出的是一个逗号,另一个分号,冒号和句号,否则添加一个分号。
%Adds a semicolon unless it is preceded by a comma, another semicolon, a colon, or a period.

\csitem{addcolon}

除非前面输出的是一个逗号,分号,另一个冒号,或句号,否则添加一个冒号。
%Adds a colon unless it is preceded by a comma, a semicolon, another colon, or a period.

\csitem{addperiod}

除非前面输出的是一个缩写点或其它任意标点符号,否则添加一个句号。该命令也可以用来将前面插入的缩写点转换为句号,比如在句子的末尾(译者:这种转换应该是经过某种判断的,比如在末尾就转换,不再末尾的话缩写点则不转换)。
%Adds a period unless it is preceded by an abbreviation dot or any other punctuation mark. This command may also be used to turn a previously inserted abbreviation dot into a period, for example at the end of a sentence.

\csitem{addexclam}
除非前面输出的是缩写点之外的任意标点,否则添加一个叹号。
%Adds an exclamation mark unless it is preceded by any punctuation mark except for an abbreviation dot.

\csitem{addquestion}

除非前面输出的是缩写点之外的任意标点,否则添加一个问号。
%Adds a question mark unless it is preceded by any punctuation mark except for an abbreviation dot.

\csitem{isdot}

将前面输出的是句号转换为缩写点。与\cmd{adddot} 相反,如果前面不是一个句号,则不添加任何符号。
%Turns a previously inserted literal period into an abbreviation dot. In contrast to \cmd{adddot}, nothing is inserted if this command is not preceded by a period.

\csitem{nopunct}

添加一个内部标记,将导致接下来的标点命令不打印任何标点。
%Adds an internal marker which will cause the next punctuation command to print nothing.

\end{ltxsyntax}

\subsubsection{添加空格}%Adding Whitespace
\label{aut:pct:spc}

下面的命令设计用来防止多余的空格。参考文献著录和标注样式总需要使用这些命令来代替原样输出的空格。与\secref{aut:pct:chk, aut:pct:pct} 节的命令不同,这些命令不仅限在标注和著录样式中使用,而是可以全局使用的。
%The following commands are designed to prevent spurious whitespace. Bibliography and citation styles should always use these commands instead of literal whitespace. In contrast to the commands in \secref{aut:pct:chk, aut:pct:pct}, they are not restricted to citations and the bibliography but available globally.

\begin{ltxsyntax}

\csitem{unspace}

移除前面的空格,即消除来自当前水平列末尾的所有间距(skip)和阀值。下面的所有命令都隐含执行该命令。
%Removes preceding whitespace, \ie removes all skips and penalties from the end of the current horizontal list. This command is implicitly executed by all of the following commands.

\csitem{addspace}

添加一个可断行的词内空格。
%Adds a breakable interword space.

\csitem{addnbspace}

添加一个不可断行(non-breakable)的词内空格。
%Adds a non"=breakable interword space.

\csitem{addthinspace}

添加一个\emph{可断行} 的短空格(thin space)。
%Adds a \emph{breakable} thin space.

\csitem{addnbthinspace}

添加一个\emph{不可断行} 的短空格(thin space),类似于命令\cmd{,} 和\cmd{thinspace}。
%Adds a non"=breakable thin space. This is similar to \cmd{,} and \cmd{thinspace}.

\csitem{addlowpenspace}

添加一个由\cnt{lownamepenalty} 计数器值作为阀值控制的空格,详见\secref{use:fmt:len, aut:fmt:len}。
%Adds a space penalized by the value of the \cnt{lownamepenalty} counter, see \secref{use:fmt:len, aut:fmt:len} for details.

\csitem{addhighpenspace}

添加一个由\cnt{highnamepenalty} 计数器值作为阀值控制的空格,详见\secref{use:fmt:len, aut:fmt:len}。
%Adds a space penalized by the value of the \cnt{highnamepenalty} counter, see \secref{use:fmt:len, aut:fmt:len} for details.

\csitem{addlpthinspace}

类似于\cmd{addlowpenspace},但添加的是可断行的短空格。
%Similar to \cmd{addlowpenspace} but adds a breakable thin space.

\csitem{addhpthinspace}

类似于\cmd{addhighpenspace},但添加的是可断行的短空格。
%Similar to \cmd{addhighpenspace} but adds a breakable thin space.

\csitem{addabbrvspace}

添加一个由\cnt{abbrvpenalty} 计数器值作为阀值控制的空格,详见\secref{use:fmt:len, aut:fmt:len}。
%Adds a space penalized by the value of the \cnt{abbrvpenalty} counter, see \secref{use:fmt:len, aut:fmt:len} for details.

\csitem{addabthinspace}

类似于\cmd{addabbrvspace},但添加一个短空格。
%Similar to \cmd{addabbrvspace} but using a thin space.

\csitem{adddotspace}

执行\cmd{adddot} 并且添加一个由\cnt{abbrvpenalty} 计数器值作为阀值控制的空格,详见\secref{use:fmt:len, aut:fmt:len}。
%Executes \cmd{adddot} and adds a space penalized by the value of the \cnt{abbrvpenalty} counter, see \secref{use:fmt:len, aut:fmt:len} for details.

\csitem{addslash}

添加一个可断行的斜杠,该命令与latex内核提供的\cmd{slash} 命令不同,因为\cmd{slash} 之后的断行完全没有阀值控制。
%Adds a breakable slash. This command differs from the \cmd{slash} command in the \latex kernel in that a linebreak after the slash is not penalized at all.

\end{ltxsyntax}

注意:本节的命令隐式地执行\cmd{unspace} 来去除多余的空格,因此它们可以互相覆盖。比如,可以使用\cmd{addnbspace} 将前面插入的词内空格转换为一个不可断行的空格,而\cmd{addspace} 可以将一个不可断行空格转换为可断行空格。\footnote{译者: 注意有的时候\cmd{unspace} 看似能够起到作用,但其实并不能随意使用的。在beamer中printtext老是有些问题,可能是实现printtext命令的依赖命令,在beamer中重定义了,跟aritcle文档类中的情况差别很大。}
%Note that the commands in this section implicitly execute \cmd{unspace} to remove spurious whitespace, hence they may be used to override each other. For example, you may use \cmd{addnbspace} to transform a previously inserted interword space into a non"=breakable one and \cmd{addspace} to turn a non"=breakable space into a breakable one.


\subsubsection{配置标点和大写}%Configuring Punctuation and Capitalization
\label{aut:pct:cfg}
下面的命令配置与标点和自动大写相关的各种功能。
%The following commands configure various features related to punctuation and automatic capitalization.


\begin{ltxsyntax}

\cmditem{DeclarePrefChars}{characters}
\cmditem*{DeclarePrefChars*}{characters}

该命令声明,测试是否将在姓名前缀和姓之间插入\cmd{bibnamedelimc} 时,需做特殊处理的字符。如果一个字符在\prm{characters} 中,那么\cmd{bibnamedelimc} 不会插入。它用于一些前缀特殊的缩写名中比如<d'Argent>,其中撇号之后没有空格,默认设置为:
%This command declares characters that are to be treated specially when testing to see if \cmd{bibnamedelimc} is to be inserted between a name prefix and a family name. If a character is in the list of \prm{characters}, \cmd{bibnamedelimc} is not inserted. It is used to allow abbreviated name prefices like <d'Argent> where no space should be inserted after the apostrophe. The starred version appends its argument to the list of prefix characters, the unstarred version replaces current setting. The default setting is:

\begin{ltxexample}
\DeclarePrefChars{'-}
\end{ltxexample}

For engines that fully support Unicode these defaults are extended with
\begin{ltxexample}[escapeinside={(*@}{@*)}]
\DeclarePrefChars*{(*@’@*)}
\end{ltxexample}
\cmditem{DeclareAutoPunctuation}{characters}

该命令定义当标注命令扫描标点时需要考虑的标点符号。注意,\prm{characters} 是一个不分隔字符列表。有效的\prm{characters} 包括句号、逗号、分号、冒号、叹号和问号。默认设置为:
%This command defines the punctuation marks to be considered by the citation commands as they scan ahead for punctuation. Note that \prm{characters} is an undelimited list of characters. Valid \prm{characters} are period, comma, semicolon, colon, exclamation and question mark. The default setting is:

\begin{ltxexample}
\DeclareAutoPunctuation{.,;:!?}
\end{ltxexample}
%
当\opt{autopunct} 包选项设置为\texttt{true} 时,该定义将自动恢复。执行|\DeclareAutoPunctuation{}|等价于设置\kvopt{autopunct}{false},即它会关闭该功能。
%This definition is restored automatically whenever the \opt{autopunct} package option is set to \texttt{true}. Executing |\DeclareAutoPunctuation{}| is equivalent to setting \kvopt{autopunct}{false}, \ie it disables this feature.

\cmditem{DeclareCapitalPunctuation}{characters}

当\biblatex 插入本地化字符串时,即关键项如 <edition>或<volume>等,该命令将把终结标点后的字符自动大写。该命令定义的一些标点符号将导致一些本地化字符串大写,如果某个标点符号在一个字符串之前。注意\prm{characters} 是一个不分隔字符列表,有效的\prm{characters} 包括句号、逗号、分号、冒号、叹号和问号。默认设置为:
%When \biblatex inserts localisation strings, \ie key terms such as <edition> or <volume>, it automatically capitalizes them after terminal punctuation marks. This command defines the punctuation marks which will cause localisation strings to be capitalized if one of them precedes a string. Note that \prm{characters} is an undelimited list of characters. Valid \prm{characters} are period, comma, semicolon, colon, exclamation and question mark. The package default is:

\begin{ltxexample}
\DeclareCapitalPunctuation{.!?}
\end{ltxexample}
%
使用参数为空的\cmd{DeclareCapitalPunctuation} 命令等价于关闭自动大写功能。因为该功能与语言相关,所以该命令必须用在\cmd{DefineBibliographyExtras} 命令(当用在导言区时)或者\cmd{DeclareBibliographyExtras}(当用在本地化模型中)的参数中。详见\secref{use:lng, aut:lng}。默认情况下,在句号、叹号和问号后面的本地化字符串将大写。所有段落开头的字符串一般都需要大写(事实上无论\tex 是否在垂直模式中)。
%Using \cmd{DeclareCapitalPunctuation} with an empty argument is equivalent to disabling automatic capitalization. Since this feature is language specific, this command must be used in the argument to \cmd{DefineBibliographyExtras} (when used in the preamble) or \cmd{DeclareBibliographyExtras} (when used in a localisation module). See \secref{use:lng, aut:lng} for details. By default, strings are capitalized after periods, exclamation marks, and question marks. All strings are generally capitalized at the beginning of a paragraph (in fact whenever \tex is in vertical mode).

\cmditem{DeclarePunctuationPairs}{identifier}{characters}

使用该命令来声明有效的标点符号对。这将影响\secref{aut:pct:pct} 节讨论的标点命令。例如,\cmd{addcomma} 命令的解释是: 该命令添加一个逗号,除非前面的字符是另一个逗号,分号,冒号或句号。换句话说,在缩写点,叹号,问号之后加入逗号是允许的。这些标点有效对声明如下:
%Use this command to declare valid pairs of punctuation marks. This will affect the punctuation commands discussed in \secref{aut:pct:pct}. For example, the description of \cmd{addcomma} states that this command adds a comma unless it is preceded by another comma, a semicolon, a colon, or a period. In other words, commas after abbreviation dots, exclamation marks, and question marks are permitted. These valid pairs are declared as follows:

\begin{ltxexample}
\DeclarePunctuationPairs{comma}{*!?}
\end{ltxexample}
%
\prm{identifier} 选择需要配置的命令。标识对应于\secref{aut:pct:pct} 节标点命令的标点名称(即去除\cmd{add} 前缀后的命令名),即有效的\prm{identifier} 字符串包括\texttt{dot}, \texttt{comma}, \texttt{semicolon}, \texttt{colon}, \texttt{period}, \texttt{exclam}, \texttt{question}。\prm{characters} 是标点符号的不分隔列表。有效的\prm{characters} 是逗号,分号,冒号,句号,叹号,问号和星号。\prm{characters} 中的句号代表一个句末点号,星号代表缩写后的点号。下面是默认的设置,当\biblatex 切换语言时总是自动回复默认设置,并且与\secref{aut:pct:pct} 描述的行为对应:
%The \prm{identifier} selects the command to be configured. The identifiers correspond to the names of the punctuation commands from \secref{aut:pct:pct} without the \cmd{add} prefix, \ie valid \prm{identifier} strings are \texttt{dot}, \texttt{comma}, \texttt{semicolon}, \texttt{colon}, \texttt{period}, \texttt{exclam}, \texttt{question}. The \prm{characters} argument is an undelimited list of punctuation marks. Valid \prm{characters} are comma, semicolon, colon, period, exclamation mark, question mark, and asterisk. A period in the \prm{characters} argument denotes an end-of"=sentence period, an asterisk the dot after an abbreviation. This is the default setup, which is automatically restored whenever \biblatex switches languages and corresponds to the behavior described in \secref{aut:pct:pct}:

\begin{ltxexample}
\DeclarePunctuationPairs{dot}{}
\DeclarePunctuationPairs{comma}{*!?}
\DeclarePunctuationPairs{semicolon}{*!?}
\DeclarePunctuationPairs{colon}{*!?}
\DeclarePunctuationPairs{period}{}
\DeclarePunctuationPairs{exclam}{*}
\DeclarePunctuationPairs{question}{*}
\end{ltxexample}
%
因为该功能与语言相关,\cmd{DeclarePunctuationPairs} 必须用在\cmd{DefineBibliographyExtras}(当在导言区使用)或\cmd{DeclareBibliographyExtras}(当在本地化模块中使用)的参数中。详见\secref{use:lng, aut:lng}。注意: 一些本地化模块可能使用不同于宏包默认的设置。\footnote{到本文档写作时,\texttt{american} 模块针对美语样式(<American-style>)标点使用不同的设置。}
%Since this feature is language specific, \cmd{DeclarePunctuationPairs} must be used in the argument to \cmd{DefineBibliographyExtras} (when used in the preamble) or \cmd{DeclareBibliographyExtras} (when used in a localisation module). See \secref{use:lng, aut:lng} for details. Note that some localisation modules may use a setup which is different from the package default.\footnote{As of this writing, the \texttt{american} module uses different settings for <American-style> punctuation.}

\cmditem{DeclareQuotePunctuation}{characters}

该命令控制<American-style>标点。\secref{aut:fmt:ich} 节的\cmd{mkbibquote} 封套可以与\secref{aut:pct:new, aut:pct:pct, aut:pct:spc} 节讨论的标点工具交互。\cmd{mkbibquote} 之后的标点将移动到引号内,如果它们已在\cmd{DeclareQuotePunctuation} 声明过。注意\prm{characters} 是一个字符的不分隔列表。有效的\prm{characters} 包括: 句号,逗号,分号,冒号,叹号和问号。下面是一个示例:
%This command controls <American-style> punctuation. The \cmd{mkbibquote} wrapper from \secref{aut:fmt:ich} can interact with the punctuation facilities discussed in \secref{aut:pct:new, aut:pct:pct, aut:pct:spc}. Punctuation marks after \cmd{mkbibquote} will be moved inside the quotes if they have been registered with \cmd{DeclareQuotePunctuation}. Note that \prm{characters} is an undelimited list of characters. Valid \prm{characters} are period, comma, semicolon, colon, exclamation and question mark. Here is an example:

\begin{ltxexample}
\DeclareQuotePunctuation{.,}
\end{ltxexample}
%
执行|\DeclareQuotePunctuation{}|等价于关闭该功能。这是包的默认情形。因为该功能与语言相关,该命令必须用在\cmd{DefineBibliographyExtras}(当在导言区使用)或\cmd{DeclareBibliographyExtras}(当在本地化模块中使用)的参数中。详见\secref{use:lng, aut:lng}。另可参见\secref{use:loc:us}。
%Executing |\DeclareQuotePunctuation{}| is equivalent to disabling this feature. This is the package default. Since this feature is language specific, this command must be used in the argument to \cmd{DefineBibliographyExtras} (when used in the preamble) or \cmd{DeclareBibliographyExtras} (when used in a localisation module). See \secref{use:lng, aut:lng} for details. See also \secref{use:loc:us}.

\csitem{uspunctuation}

使用底层命令\cmd{DeclareQuotePunctuation} 和\cmd{DeclarePunctuationPairs} 激活美语样式标点的缩略命令。详见\secref{use:loc:us}。提供该缩略命令是为了使用方便。有效的设置由底层命令应用。
%A shorthand using the lower-level commands \cmd{DeclareQuotePunctuation} and \cmd{DeclarePunctuationPairs} to activate <American-style> punctuation. See \secref{use:loc:us} for details. This shorthand is provided for convenience only. The effective settings are applied by the lower-level commands.

\csitem{stdpunctuation}

取消由\cmd{uspunctuation} 命令产生的设置,恢复到标准的标点样式。因为标准标点样式是默认设置,仅需要使用该命令来覆盖前面执行的\cmd{uspunctuation} 命令即可。详见\secref{use:loc:us}。
%Undoes the settings applied by \cmd{uspunctuation}, restoring standard punctuation. As standard punctuation is the default setting, you only need this command to override a previously executed \cmd{uspunctuation} command. See \secref{use:loc:us} for details.

\end{ltxsyntax}

\subsubsection{修正标点追踪}%Correcting Punctuation Tracking
\label{aut:pct:ctr}

在一般情况下标点追踪和自动大写的工具是很可靠的,但总是存在一些少量情况可能需要手动干预。典型的问题包括当本地化字符串作为脚注(就大写而言脚注常被认为是在一个段落的开始,但\tex 在此时并不处于垂直模式中)的第一个单词打印时,或者句号(不是真正的句末点号,例如在一个像«[\dots\unkern]»的省略号之后诸如\cmd{addperiod} 等命令将不会进行任何操作,因为圆括号和方括号对于标点追踪器来说是透明的)之后的标点。\footnote{译者:重点注意括号对于标点追踪器是透明的,这个问题在样式设计时经常会碰到。} 在这些情况中,如果有需要,可在著录和标注样式中使用如下命令来标记句子的开头或者中间位置:
%The facilities for punctuation tracking and automatic capitalization are very reliable under normal circumstances, but there are always marginal cases which may require manual intervention. Typical cases are localisation strings printed as the first word in a footnote (which is usually treated as the beginning of a paragraph as far as capitalization is concerned, but \tex is not in vertical mode at this point) or punctuation after periods which are not really end"=of"=sentence periods (for example, after an ellipsis like «[\dots\unkern]» a command such as \cmd{addperiod} would do nothing since parentheses and brackets are transparent to the punctuation tracker). In such cases, use the following commands in bibliography and citation styles to mark the beginning or middle of a sentence if and where required:

\begin{ltxsyntax}

\csitem{bibsentence}

该命令标记句子的开头。紧跟在其后面的一个本地化字符串将会大写并且标点追踪器也会重设,即该命令对标点追踪器隐藏前面所有的标点并且强制大写。
%This command marks the beginning of a sentence. A localisation string immediately after this command will be capitalized and the punctuation tracker is reset, \ie this command hides all preceding punctuation marks from the punctuation tracker and enforces capitalization.

\csitem{midsentence}

该命令标记句子的中间位置。紧跟在其后面的一个本地化字符串将不会大写并且标点追踪器也会重设,即该命令对标点追踪器隐藏前面所有的标点并且抑制大写。
%This command marks the middle of a sentence. A localisation string immediately after this command will not be capitalized and the punctuation tracker is reset, \ie this command hides all preceding punctuation marks from the punctuation tracker and suppresses capitalization.

\csitem*{midsentence*}

\cmd{midsentence} 命令的带星号版本,差别在于,对于该命令前面的缩写点不会隐藏,即\cmd{midsentence*} 之后的任何代码都能看见前面的缩写点。而所有其他标点将对标点追踪器隐藏,大写也将被抑制。
%The starred variant of \cmd{midsentence} differs from the regular one in that a preceding abbreviation dot is not hidden from the the punctuation tracker, \ie any code after \cmd{midsentence*} will see a preceding abbreviation dot. All other punctuation marks are hidden from the punctuation tracker and capitalization is suppressed.

\end{ltxsyntax}

\subsection{本地化字符串}%Localization Strings
\label{aut:str}

本地化字符串如<edition>或<volume>之类的关键词项将由\biblatex 本地化模块自动转换。本地化字符串概述见\secref{aut:lng},所有默认支持的字符串列表见\secref{aut:lng:key} 节。本节的命令用于打印本地化的项。
%Localization strings are key terms such as <edition> or <volume> which are automatically translated by \biblatex's localisation modules. See \secref{aut:lng} for an overview and \secref{aut:lng:key} for a list of all strings supported by default. The commands in this section are used to print the localised term.

\begin{ltxsyntax}

\cmditem{bibstring}[wrapper]{key}

打印本地化字符串\prm{key},其中\prm{key} 是一个以小写字母书写的标识(见\secref{aut:lng:key})。需要时字符串会大写,详见\secref{aut:pct:cfg} 节。根据\secref{use:opt:pre:gen} 节的\opt{abbreviate} 包选项,\cmd{bibstring} 打印短字符串或长字符串。如果本地化字符串处于嵌套中,即\cmd{bibstring} 用在另一个字符串中,它的作用类似于\cmd{bibxstring}。如果给出\prm{wrapper} 参数,字符串将传递给\prm{wrapper} 用于格式化。这常用于字体命令比如\cmd{emph} 等。
%Prints the localisation string \prm{key}, where \prm{key} is an identifier in lowercase letters (see \secref{aut:lng:key}). The string will be capitalized as required, see \secref{aut:pct:cfg} for details.
%Depending on the \opt{abbreviate} package option from \secref{use:opt:pre:gen}, \cmd{bibstring} prints the short or the long version of the string. If localisation strings are nested, \ie if \cmd{bibstring} is used in another string, it will behave like \cmd{bibxstring}.
%If the \prm{wrapper} argument is given, the string is passed to the \prm{wrapper} for formatting. This is intended for font commands such as \cmd{emph}.

\cmditem{biblstring}[wrapper]{key}

类似于\cmd{bibstring},但总是打印长字符串,忽略\opt{abbreviate} 选项。
%Similar to \cmd{bibstring} but always prints the long string, ignoring the \opt{abbreviate} option.

\cmditem{bibsstring}[wrapper]{key}

类似于\cmd{bibstring},但总是打印短字符串,忽略\opt{abbreviate} 选项。
%Similar to \cmd{bibstring} but always prints the short string, ignoring the \opt{abbreviate} option.

\cmditem{bibcpstring}[wrapper]{key}

类似于\cmd{bibstring},但字符串总是首字母大写
%Similar to \cmd{bibstring} but the term is always capitalized.

\cmditem{bibcplstring}[wrapper]{key}

类似于\cmd{biblstring},但字符串总是首字母大写
%Similar to \cmd{biblstring} but the term is always capitalized.

\cmditem{bibcpsstring}[wrapper]{key}

类似于\cmd{bibsstring},但字符串总是首字母大写
%Similar to \cmd{bibsstring} but the term is always capitalized.

\cmditem{bibucstring}[wrapper]{key}

类似于\cmd{bibstring},但字符串总是大写
%Similar to \cmd{bibstring} but the whole term is uppercased.

\cmditem{bibuclstring}[wrapper]{key}

类似于\cmd{biblstring},但字符串总是大写
%Similar to \cmd{biblstring} but the whole term is uppercased.

\cmditem{bibucsstring}[wrapper]{key}

类似于\cmd{bibsstring},但字符串总是大写
%Similar to \cmd{bibsstring} but the whole term is uppercased.

\cmditem{biblcstring}[wrapper]{key}

类似于\cmd{bibstring},但字符串总是小写
%Similar to \cmd{bibstring} but the whole term is lowercased.

\cmditem{biblclstring}[wrapper]{key}

类似于\cmd{biblstring},但字符串总是小写
%Similar to \cmd{biblstring} but the whole term is lowercased.

\cmditem{biblcsstring}[wrapper]{key}

类似于\cmd{bibsstring},但字符串总是小写
%Similar to \cmd{bibsstring} but the whole term is lowercased.

\cmditem{bibxstring}{key}

\cmd{bibstring} 命令一个简化的但可展开的版本。注意,这一命令不自动首字母大写,也不与标点追踪器交互。用于当字符串嵌套或者一个判断中需要一个可展开的本地化字符串等特殊情况。
%A simplified but expandable version of \cmd{bibstring}. Note that this variant does not capitalize automatically, nor does it hook into the punctuation tracker. It is intended for special cases in which strings are nested or an expanded localisation string is required in a test.

\cmditem{bibxlstring}[wrapper]{key}

类似于\cmd{bibxstring},但总是打印长字符串,忽略\opt{abbreviate} 选项。
%Similar to \cmd{bibxstring} but always uses the long string, ignoring the \opt{abbreviate} option.

\cmditem{bibxsstring}[wrapper]{key}

类似于\cmd{bibxstring},但总是打印短字符串,忽略\opt{abbreviate} 选项。
%Similar to \cmd{bibxstring} but always uses the short string, ignoring the \opt{abbreviate} option.

\cmditem{mainlang}

从当前语言转换到主文档语言。可以用在上述本地化字符串命令的\prm{wrapper} 参数中。
%Switches from the current language to the main document language. This can be used the \prm{wrapper} argument in the localisation string commands above.

\end{ltxsyntax}

\subsection{本地化模块}%Localization Modules
\label{aut:lng}

本地化模块提供了诸如<edition>或<volume>等关键项,或者一些具体语言相关功能定义比如日期格式和序数等的转换(翻译)。这些定义在后缀为\file{lbx} 的文件中给出。文件的基本名称必须是\sty{babel}/\sty{polyglossia} 包已定义的语言名称。\file{lbx} 文件也用于将\sty{babel}/\sty{polyglossia} 语言名映射为\biblatex 包的后端模块。所有的本地化模型根据需要在正文中加载。注意文件的内容在一个编组内处理,\texttt{@} 字符的类别码临时设置为<letter>。
%A localisation module provides translations for key terms such as <edition> or <volume> as well as definitions for language specific features such as the date format and ordinals. These definitions are provided in files with the suffix \file{lbx}. The base name of the file must be a language name known to the \sty{babel}/\sty{polyglossia} packages. The \file{lbx} files may also be used to map \sty{babel}/\sty{polyglossia} language names to the backend modules of the \biblatex package. All localisation modules are loaded on demand in the document body. Note that the contents of the file are processed in a group and that the category code of the character \texttt{@} is temporarily set to <letter>.

\subsubsection{本地化命令}%Localization Commands
\label{aut:lng:cmd}

用户层(user-level)的本地化命令已经在\secref{use:lng} 节介绍过了。然而,用在\file{lbx} 文件中的本地化命令的语法与用在导言区和配置文件中的语法略有不同。当在本地化文件中使用,不需要指定\prm{language},因为字符串要映射的语言已经由\file{lbx} 文件名给出。
%The user"=level versions of the localisation commands were already introduced in \secref{use:lng}. When used in \file{lbx} files, however, the syntax of localisation commands is different from the user syntax in the preamble and the configuration file. When used in localisation files, there is no need to specify the \prm{language} because the mapping of strings to a language is already provided by the name of the \file{lbx} file.

\begin{ltxsyntax}

\cmditem{DeclareBibliographyStrings}{definitions}

该命令仅在\file{lbx} 文件中提供。用于定义本地化字符串。\prm{definitions} 由\keyval 对构成,用于给一个标识赋予一个表达式。默认支持的完整关键词(键)列表在\secref{aut:lng:key} 节给出。注意:\file{lbx} 文件中值的语法是不同的。\footnote{译者: 重点注意定义本地化字符串的语法在lbx文件和样式文件中是不同的,这在设计样式文件是会碰到。} 赋予的键值由两个短语构成,每个部分都包含在一个括号中。下面是一个示例:
%This command is only available in \file{lbx} files. It is used to define localisation strings. The \prm{definitions} consist of \keyval pairs which assign an expression to an identifier. A complete list of all keys supported by default is given is \secref{aut:lng:key}. Note that the syntax of the value is different in \file{lbx} files. The value assigned to a key consists of two expressions, each of which is wrapped in an additional pair of brackets. This is best shown by example:

\begin{ltxexample}
\DeclareBibliographyStrings{%
  bibliography  = {{Bibliography}{Bibliography}},
  shorthands    = {{List of Abbreviations}{Abbreviations}},
  editor        = {{editor}{ed.}},
  editors       = {{editors}{eds.}},
}
\end{ltxexample}
%
第一个值是用于写出的长短语,第二个是缩写或简易形式。两个字符串都必须给出即便它们是相同的,比如当一个短语总是要(或不要)缩写时。根据\opt{abbreviate} 包选项设置(见\secref{use:opt:pre:gen}),当加载\opt{abbreviate} 时\biblatex 选择一个短语。也存在一个名为\texttt{inherit} 特殊键,从另一种语言中拷贝字符串。这用于仅有部分短语不同的语言间,比如德国和奥地利,或者美国和英国英语。例如,下面给出的奥地利语完整定义:
%The first value is the long, written out expression, the second one is an abbreviated or short form. Both strings must always be given even though they may be identical if an expression is always (or never) abbreviated. Depending on the setting of the \opt{abbreviate} package option (see \secref{use:opt:pre:gen}), \biblatex selects one expression when loading the \file{lbx} file. There is also a special key named \texttt{inherit} which copies the strings from a different language. This is intended for languages which only differ in a few expressions, such as German and Austrian or American and British English. For example, here are the complete definitions for Austrian:

\begin{ltxexample}
\DeclareBibliographyStrings{%
  inherit       = {german},
  january       = {{J\"anner}{J\"an.}},
}
\end{ltxexample}

这里示例略微简化了。真实的本地化文件应该使用\secref{aut:pct:pct, use:fmt} 节讨论的标点和格式化命令而不是文本标点(literal punctuation)\footnote{译者: 在样式文件中使用文本标点时可能会遇到一些不可预料的问题,应尽可能的使用biblatex提供的标点命令}。下面的内容摘录自一个真实的本地化文件:
%The above examples are slightly simplified. Real localisation files should use the punctuation and formatting commands discussed in \secref{aut:pct:pct, use:fmt} instead of literal punctuation. Here is an excerpt from a real localisation file:

\begin{ltxexample}
  bibliography     = {{Bibliography}{Bibliography}},
  shorthands       = {{List of Abbreviations}{Abbreviations}},
  editor           = {{editor}{ed\adddot}},
  editors          = {{editors}{eds\adddot}},
  byeditor         = {{edited by}{ed\adddotspace by}},
  mathesis         = {{Master's thesis}{MA\addabbrvspace thesis}},
\end{ltxexample}
%
注意上例中缩略点的处理,缩略短语中的空格以及首字母大写等。所有的短语在一个句子中使用时,都是首字母大写的。\biblatex 包将自动将句首的第一个单词的首字母大写,详见\secref{aut:pct:cfg} 节的\cmd{DeclareCapitalPunctuation}。用作标题的短语是特殊的。它们常使用一种适合标题的方式并且不应使用缩略形式(但它们可能具有简易形式)。
%Note the handling of abbreviation dots, the spacing in abbreviated expressions, and the capitalization in the example above. All expressions should be capitalized as they usually are when used in the middle of a sentence. The \biblatex package will automatically capitalize the first word when required at the beginning of a sentence, see \cmd{DeclareCapitalPunctuation} in \secref{aut:pct:cfg} for details. Expressions intended for use in headings are special. They should be capitalized in a way that is suitable for titling and should not be abbreviated (but they may have a short form).

\cmditem{InheritBibliographyStrings}{language}

该命令仅在\file{lbx} 文件中提供。它将\prm{language} 语言的本地化字符串复制到\file{lbx} 文件名指出的当前语言中。
%This command is only available in \file{lbx} files. It copies the localisation strings for \prm{language} to the current language, as specified by the name of the \file{lbx} file.

\cmditem{DeclareBibliographyExtras}{code}

该命令仅在\file{lbx} 文件中提供。用于调整语言相关的功能比如日期格式和序号。任意的\latex 代码\prm{code},常由\secref{aut:fmt:lng} 节的格式化命令的重定义构成。
%This command is only available in \file{lbx} files. It is used to adapt language specific features such as the date format and ordinals. The \prm{code}, which may be arbitrary \latex code, will usually consist of redefinitions of the formatting commands from \secref{aut:fmt:lng}.

\cmditem{UndeclareBibliographyExtras}{code}

该命令仅在\file{lbx} 文件中提供。用于恢复由\cmd{DeclareBibliographyExtras} 命令修改的格式化命令。如果一个重定义命令包含在\secref{aut:fmt:lng} 中,则没有必要恢复前一定义,因为这些命令总是会根据所有语言模块进行本地化。
%This command is only available in \file{lbx} files. It is used to restore any formatting commands modified with \cmd{DeclareBibliographyExtras}. If a redefined command is included in \secref{aut:fmt:lng}, there is no need to restore its previous definition since these commands are localised by all language modules anyway.

\cmditem{InheritBibliographyExtras}{language}

该命令仅在\file{lbx} 文件中提供。它将\prm{language} 语言的参考文献附加规则复制到\file{lbx} 文件名指出的当前语言中。
%This command is only available in \file{lbx} files. It copies the bibliography extras for \prm{language} to the current language, as specified by the name of the \file{lbx} file.

\cmditem{DeclareHyphenationExceptions}{text}

该命令对应于\secref{use:lng} 节的\cmd{DefineHyphenationExceptions}。差别在于它仅在\file{lbx} 文件中提供,并且没有\prm{language} 参数。连字符例外规则将影响正在处理的\file{lbx} 文件的语言。
%This command corresponds to \cmd{DefineHyphenationExceptions} from \secref{use:lng}. The difference is that it is only available in \file{lbx} files and that the \prm{language} argument is omitted. The hyphenation exceptions will affect the language of the \file{lbx} file currently being processed.

\cmditem{DeclareRedundantLanguages}{language, language, ...}{langid, langid, ...}

该命令提供了\secref{use:opt:pre:gen} 节\opt{clearlang} 语言选项要求的语言映射。\prm{language} 是\bibfield{language} 域给出的字符串(不需要可选的\texttt{lang} 前缀),\prm{langid} 是\sty{babel}/\sty{polyglossia} 的语言标识,在加载\sty{babel} 包的\cmd{usepackage} 命令可选参数或者使用\sty{polyglossia} 包的\cmd{setdefaultlanguage} 或\cmd{setotherlanguages} 命令中给出。这一命令可以用于\file{lbx} 文件中或者文档导言区中。下面是一些示例:
%This command provides the language mappings required by the \opt{clearlang} option from \secref{use:opt:pre:gen}.
%The \prm{language} is the string given in the \bibfield{language} field (without the optional \texttt{lang} prefix); \prm{langid} is \sty{babel}/\sty{polyglossia}'s language identifier, as given in the optional argument of \cmd{usepackage} when loading \sty{babel} or the argument of \cmd{setdefaultlanguage} or \cmd{setotherlanguages} when using \sty{polyglossia}. This command may be used in \file{lbx} files or in the document preamble. Here are some examples:

\begin{ltxexample}
\DeclareRedundantLanguages{french}{french}
\DeclareRedundantLanguages{german}{german,ngerman,austrian,naustrian,
        nswissgerman,swissgerman}
\DeclareRedundantLanguages{english,american}{english,american,british,
	canadian,australian,newzealand,USenglish,UKenglish}
\end{ltxexample}
%
注意这一功能需要全局地启用\secref{use:opt:pre:gen} 节的\opt{clearlang} 选项。如果关闭该选项,所有的映射将被忽略。如果\prm{langid} 参数为空,\biblatex 将清除相应语言的映射,即仅关闭该\prm{language} 的功能。
%Note that this feature needs to be enabled globally with the \opt{clearlang} option from \secref{use:opt:pre:gen}. If it is disabled, all mappings will be ignored. If the \prm{langid} parameter is blank, \biblatex will clear the mappings for the corresponding \prm{language}, \ie the feature will be disabled for this \prm{language} only.

\cmditem{DeclareLanguageMapping}{language}{file}

该命令将\sty{babel}/\sty{polyglossia} 语言标识映射到一个 \file{lbx} 文件中。\prm{language} 必须是一个\sty{babel}/\sty{polyglossia} 已经定义的语言,即表\ref{bib:fld:tab1} 中列出的标识之一。\prm{file} 参数是选择的\file{lbx} 文件的出\texttt{.lbx} 后缀外的文件名。可以多次声明相同的映射。后面的声明将覆盖前面的声明。该命令只能用在导言中。详见\secref{aut:cav:lng}。
%This command maps a \sty{babel}/\sty{polyglossia} language identifier to an \file{lbx} file. The \prm{language} must be a language name known to the \sty{babel}/\sty{polyglossia} package, \ie one of the identifiers listed in \tabref{bib:fld:tab1}. The \prm{file} argument is the name of an alternative \file{lbx} file without the \texttt{.lbx} suffix. Declaring the same mapping more than once is possible. Subsequent declarations will simply overwrite any previous ones. This command may only be used in the preamble. See \secref{aut:cav:lng} for further details.

\cmditem{DeclareLanguageMappingSuffix}{suffix}

%This command defines a language file suffix which will be added when looking for \file{.lbx} language string definition files. This is intended for styles which provide their own \file{.lbx} files so that they will be used automatically. For example, the APA style defines:
该命令定义了语言文件的后缀,将在搜索\file{.lbx}语言字符串定义文件时加入。用于一些样式提供了自身的\file{.lbx}文件的情况,使其可以自动加载使用,
例如APA样式定义了:

\begin{ltxexample}
\DeclareLanguageMappingSuffix{-apa}
\end{ltxexample}
%
%When the document language is <german>, \biblatex will look for the file \file{german-apa.lbx} which defines some APA specific strings and in turn loads \file{german.lbx}. If \cmd{DeclareLanguageMapping} is defined for a language, this overrides \cmd{DeclareLanguageMappingSuffix}.

当文档语言是<german>,\biblatex 会搜索 \file{german-apa.lbx}文件,该文件定义了一些特殊的字符串,然后再加载\file{german.lbx}。
如果\cmd{DeclareLanguageMapping}是未某种语言定义的,那么将会覆盖\cmd{DeclareLanguageMappingSuffix}的定义。


%The suffix will be applied to other language files loaded recursively by the loading of a language file. For example, given the suffix defined above, when loading <ngerman>, \biblatex will look for the file \file{ngerman-apa.lbx} and if this recursively loads <german>, then biblatex will look for \file{german-apa.lbx}. Infinite recursion is of course avoided.
后缀将会应用到语言文件中调用的其它语言文件中。对于上面的例子,当加载<ngerman>,也会搜索\file{ngerman-apa.lbx} ,
如果该文件还调用了<german>,那么biblatex还会搜索\file{german-apa.lbx},当然会避免无限循环的情况。


\cmditem{NewBibliographyString}{key}

该命令可以用在导言区(包括\file{cbx} 和\file{bbx} 文件)或者\file{lbx} 文件中,用于声明新的本地化字符串,即它初始化一个关键词(键)可以用于 \cmd{DefineBibliographyStrings} 或 \cmd{DeclareBibliographyStrings} 命令的 \prm{definitions} 中。\prm{key} 选项可以是一个逗号分隔的列表。当在\file{lbx} 中使用时,\prm{key} 只完成\file{lbx} 文件名指定的语言的初始化。默认的键在\secref{aut:lng:key} 节列出。
%This command, which may be used in the preamble (including \file{cbx} and \file{bbx} files) as well as in \file{lbx} files, declares new localisation strings, \ie it initializes a new \prm{key} to be used in the \prm{definitions} of \cmd{DefineBibliographyStrings} or \cmd{DeclareBibliographyStrings}. The \prm{key} argument may also be a comma"=separated list of key names. When used in an \file{lbx}, the \prm{key} is initialized only for the language specified by the name of the \file{lbx} file. The keys listed in \secref{aut:lng:key} are defined by default.

\end{ltxsyntax}

\subsubsection{本地化关键词(键)}%Localization Keys
\label{aut:lng:key}

本节中的本地化关键词是默认定义的,由\biblatex 附带的本地化文件提供。注意这些字符串仅在标注、著录表和文献列表中使用。所有的短语当在句中使用时常需要首字母大写。而\biblatex 会自动将句首的字符串首字母大写。唯一的例外规则是三个用于标题中的字符串。
%The localisation keys in this section are defined by default and covered by the localisation files which come with \biblatex. Note that these strings are only available in citations, the bibliography and bibliography lists. All expressions should be capitalized as they usually are when used in the middle of a sentence. \biblatex will capitalize them automatically at the beginning of a sentence. The only exceptions to these rules are the three strings intended for use in headings.

\paragraph{标题}%Headings
\label{aut:lng:key:bhd}

下面的字符串比较特殊是因为他们用在标题中,并通过宏可以全局使用。因此,通常他们在标题中使用时需要首字母大写,并且不能包含任何作为\biblatex 作者接口的本地命令。
%The following strings are special because they are intended for use in headings and made available globally via macros. For this reason, they should be capitalized for use in headings and they must not include any local commands which are part of \biblatex's author interface.

\begin{keymarglist}
\item[bibliography] 词(术语)<bibliography>,也作\cmd{bibname} 使用。
%The term <bibliography>, also available as \cmd{bibname}.
\item[references] 词<references>,也作\cmd{refname} 使用。
%The term <references>, also available as \cmd{refname}.
\item[shorthands] 词<list of shorthands>或<list of abbreviations>,也作\cmd{biblistname} 使用。
%The term <list of shorthands> or <list of abbreviations>, also available as \cmd{biblistname}.
\end{keymarglist}

\paragraph{角色,解释为职业}%Roles, Expressed as Functions
\label{aut:lng:key:efn}

下面的关键词指出的角色,可以解释为职业(<editor>, <translator>)而不是行动(<edited by>, <translated by>)。
%The following keys refer to roles which are expressed as a function (<editor>, <translator>) rather than as an action (<edited by>, <translated by>).

\begin{keymarglist}
\item[editor] The term <editor>, referring to the main editor. This is the most generic editorial role.
\item[editors] The plural form of \texttt{editor}.
\item[compiler] The term <compiler>, referring to an editor whose task is to compile a work.
\item[compilers] The plural form of \texttt{compiler}.
\item[founder] The term <founder>, referring to a founding editor.
\item[founders] The plural form of \texttt{founder}.
\item[continuator] An expression like <continuator>, <continuation>, or <continued>, referring to a past editor who continued the work of the founding editor but was subsequently replaced by the current editor.
\item[continuators] The plural form of \texttt{continuator}.
\item[redactor] The term <redactor>, referring to a secondary editor.
\item[redactors] The plural form of \texttt{redactor}.
\item[reviser] The term <reviser>, referring to a secondary editor.
\item[revisers] The plural form of \texttt{reviser}.
\item[collaborator] A term like <collaborator>, <collaboration>, <cooperator>, or <cooperation>, referring to a secondary editor.
\item[collaborators] The plural form of \texttt{collaborator}.
\item[translator] The term <translator>.
\item[translators] The plural form of \texttt{translator}.
\item[commentator] The term <commentator>, referring to the author of a commentary to a work.
\item[commentators] The plural form of \texttt{commentators}.
\item[annotator] The term <annotator>, referring to the author of annotations to a work.
\item[annotators] The plural form of \texttt{annotators}.
\item[organizer] The term <organizer>, referring to the organizer of an
  event or work.
\item[organizers] The plural form of \texttt{organizer}.
\end{keymarglist}

\paragraph{合并的编辑角色,解释为职业}%Concatenated Editor Roles, Expressed as Functions
\label{aut:lng:key:cef}

下面的这些关键词类似于\texttt{editor}, \texttt{translator} 等的作用。它们常用来说明编辑的附加角色,比如<editor and translator>, <editor and foreword>。
%The following keys are similar in function to \texttt{editor}, \texttt{translator}, etc. They are used to indicate additional roles of the editor, \eg\ <editor and translator>, <editor and foreword>.

\begin{keymarglist}
\item[editortr] Used if \bibfield{editor}\slash \bibfield{translator} are identical.
\item[editorstr] The plural form of \texttt{editortr}.
\item[editorco] Used if \bibfield{editor}\slash \bibfield{commentator} are identical.
\item[editorsco] The plural form of \texttt{editorco}.
\item[editoran] Used if \bibfield{editor}\slash \bibfield{annotator} are identical.
\item[editorsan] The plural form of \texttt{editoran}.
\item[editorin] Used if \bibfield{editor}\slash \bibfield{introduction} are identical.
\item[editorsin] The plural form of \texttt{editorin}.
\item[editorfo] Used if \bibfield{editor}\slash \bibfield{foreword} are identical.
\item[editorsfo] The plural form of \texttt{editorfo}.
\item[editoraf] Used if \bibfield{editor}\slash \bibfield{aftword} are identical.
\item[editorsaf] The plural form of \texttt{editoraf}.
\end{keymarglist}
%
Keys for \bibfield{editor}\slash \bibfield{translator}\slash \prm{role} combinations:

\begin{keymarglist}
\item[editortrco] Used if \bibfield{editor}\slash \bibfield{translator}\slash \bibfield{commentator} are identical.
\item[editorstrco] The plural form of \texttt{editortrco}.
\item[editortran] Used if \bibfield{editor}\slash \bibfield{translator}\slash \bibfield{annotator} are identical.
\item[editorstran] The plural form of \texttt{editortran}.
\item[editortrin] Used if \bibfield{editor}\slash \bibfield{translator}\slash \bibfield{introduction} are identical.
\item[editorstrin] The plural form of \texttt{editortrin}.
\item[editortrfo] Used if \bibfield{editor}\slash \bibfield{translator}\slash \bibfield{foreword} are identical.
\item[editorstrfo] The plural form of \texttt{editortrfo}.
\item[editortraf] Used if \bibfield{editor}\slash \bibfield{translator}\slash \bibfield{aftword} are identical.
\item[editorstraf] The plural form of \texttt{editortraf}.
\end{keymarglist}
%
Keys for \bibfield{editor}\slash \bibfield{commentator}\slash \prm{role} combinations:

\begin{keymarglist}
\item[editorcoin] Used if \bibfield{editor}\slash \bibfield{commentator}\slash \bibfield{introduction} are identical.
\item[editorscoin] The plural form of \texttt{editorcoin}.
\item[editorcofo] Used if \bibfield{editor}\slash \bibfield{commentator}\slash \bibfield{foreword} are identical.
\item[editorscofo] The plural form of \texttt{editorcofo}.
\item[editorcoaf] Used if \bibfield{editor}\slash \bibfield{commentator}\slash \bibfield{aftword} are identical.
\item[editorscoaf] The plural form of \texttt{editorcoaf}.
\end{keymarglist}
%
Keys for \bibfield{editor}\slash \bibfield{annotator}\slash \prm{role} combinations:

\begin{keymarglist}
\item[editoranin] Used if \bibfield{editor}\slash \bibfield{annotator}\slash \bibfield{introduction} are identical.
\item[editorsanin] The plural form of \texttt{editoranin}.
\item[editoranfo] Used if \bibfield{editor}\slash \bibfield{annotator}\slash \bibfield{foreword} are identical.
\item[editorsanfo] The plural form of \texttt{editoranfo}.
\item[editoranaf] Used if \bibfield{editor}\slash \bibfield{annotator}\slash \bibfield{aftword} are identical.
\item[editorsanaf] The plural form of \texttt{editoranaf}.
\end{keymarglist}
%
Keys for \bibfield{editor}\slash \bibfield{translator}\slash \bibfield{commentator}\slash \prm{role} combinations:

\begin{keymarglist}
\item[editortrcoin] Used if \bibfield{editor}\slash \bibfield{translator}\slash \bibfield{commentator}\slash \bibfield{introduction} are identical.
\item[editorstrcoin] The plural form of \texttt{editortrcoin}.
\item[editortrcofo] Used if \bibfield{editor}\slash \bibfield{translator}\slash \bibfield{commentator}\slash \bibfield{foreword} are identical.
\item[editorstrcofo] The plural form of \texttt{editortrcofo}.
\item[editortrcoaf] Used if \bibfield{editor}\slash \bibfield{translator}\slash \bibfield{commentator}\slash \bibfield{aftword} are identical.
\item[editorstrcoaf] The plural form of \texttt{editortrcoaf}.
\end{keymarglist}
%
Keys for \bibfield{editor}\slash \bibfield{annotator}\slash \bibfield{commentator}\slash \prm{role} combinations:

\begin{keymarglist}
\item[editortranin] Used if \bibfield{editor}\slash \bibfield{annotator}\slash \bibfield{commentator}\slash \bibfield{introduction} are identical.
\item[editorstranin] The plural form of \texttt{editortranin}.
\item[editortranfo] Used if \bibfield{editor}\slash \bibfield{annotator}\slash \bibfield{commentator}\slash \bibfield{foreword} are identical.
\item[editorstranfo] The plural form of \texttt{editortranfo}.
\item[editortranaf] Used if \bibfield{editor}\slash \bibfield{annotator}\slash \bibfield{commentator}\slash \bibfield{aftword} are identical.
\item[editorstranaf] The plural form of \texttt{editortranaf}.
\end{keymarglist}

\paragraph{合并的译者角色,解释为职业}%Concatenated Translator Roles, Expressed as Functions
\label{aut:lng:key:ctf}

下面的这些关键词类似于\texttt{translator} 的作用。它们常用来说明编辑的附加角色,比如<translator and commentator>, <translator and introduction>。
%The following keys are similar in function to \texttt{translator}. They are used to indicate additional roles of the translator, \eg\ <translator and commentator>, <translator and introduction>.

\begin{keymarglist}
\item[translatorco] Used if \bibfield{translator}\slash \bibfield{commentator} are identical.
\item[translatorsco] The plural form of \texttt{translatorco}.
\item[translatoran] Used if \bibfield{translator}\slash \bibfield{annotator} are identical.
\item[translatorsan] The plural form of \texttt{translatoran}.
\item[translatorin] Used if \bibfield{translator}\slash \bibfield{introduction} are identical.
\item[translatorsin] The plural form of \texttt{translatorin}.
\item[translatorfo] Used if \bibfield{translator}\slash \bibfield{foreword} are identical.
\item[translatorsfo] The plural form of \texttt{translatorfo}.
\item[translatoraf] Used if \bibfield{translator}\slash \bibfield{aftword} are identical.
\item[translatorsaf] The plural form of \texttt{translatoraf}.
\end{keymarglist}
%
Keys for \bibfield{translator}\slash \bibfield{commentator}\slash \prm{role} combinations:

\begin{keymarglist}
\item[translatorcoin] Used if \bibfield{translator}\slash \bibfield{commentator}\slash \bibfield{introduction} are identical.
\item[translatorscoin] The plural form of \texttt{translatorcoin}.
\item[translatorcofo] Used if \bibfield{translator}\slash \bibfield{commentator}\slash \bibfield{foreword} are identical.
\item[translatorscofo] The plural form of \texttt{translatorcofo}.
\item[translatorcoaf] Used if \bibfield{translator}\slash \bibfield{commentator}\slash \bibfield{aftword} are identical.
\item[translatorscoaf] The plural form of \texttt{translatorcoaf}.
\end{keymarglist}
%
Keys for \bibfield{translator}\slash \bibfield{annotator}\slash \prm{role} combinations:

\begin{keymarglist}
\item[translatoranin] Used if \bibfield{translator}\slash \bibfield{annotator}\slash \bibfield{introduction} are identical.
\item[translatorsanin] The plural form of \texttt{translatoranin}.
\item[translatoranfo] Used if \bibfield{translator}\slash \bibfield{annotator}\slash \bibfield{foreword} are identical.
\item[translatorsanfo] The plural form of \texttt{translatoranfo}.
\item[translatoranaf] Used if \bibfield{translator}\slash \bibfield{annotator}\slash \bibfield{aftword} are identical.
\item[translatorsanaf] The plural form of \texttt{translatoranaf}.
\end{keymarglist}

\paragraph{角色,解释为行为}%Roles, Expressed as Actions
\label{aut:lng:key:eac}
下面的关键词指的角色解释为行为(<edited by>, <translated by>)而不是职业(<editor>, <translator>)。
%The following keys refer to roles which are expressed as an action (<edited by>, <translated by>) rather than as a function (<editor>, <translator>).

\begin{keymarglist}
\item[byauthor] The expression <[created] by \prm{name}>.
\item[byeditor] The expression <edited by \prm{name}>.
\item[bycompiler] The expression <compiled by \prm{name}>.
\item[byfounder] The expression <founded by \prm{name}>.
\item[bycontinuator] The expression <continued by \prm{name}>.
\item[byredactor] The expression <redacted by \prm{name}>.
\item[byreviser] The expression <revised by \prm{name}>.
\item[byreviewer] The expression <reviewed by \prm{name}>.
\item[bycollaborator] An expression like <in collaboration with \prm{name}> or <in cooperation with \prm{name}>.
\item[bytranslator] The expression <translated by \prm{name}> or <translated from \prm{language} by \prm{name}>.
\item[bycommentator] The expression <commented by \prm{name}>.
\item[byannotator] The expression <annotated by \prm{name}>.
\item[byorganizer] The expression <[organized] by \prm{name}>.
\end{keymarglist}

\paragraph{合并的编者角色,解释为行为}%Concatenated Editor Roles, Expressed as Actions
\label{aut:lng:key:cea}

下面的这些关键词类似于\texttt{byeditor}, \texttt{bytranslator} 等的作用。它们常用来说明编辑的附加角色,比如<edited and translated by>, <edited and furnished with an introduction by>, <edited, with a foreword, by>。
%The following keys are similar in function to \texttt{byeditor}, \texttt{bytranslator}, etc. They are used to indicate additional roles of the editor, \eg\ <edited and translated by>, <edited and furnished with an introduction by>, <edited, with a foreword, by>.

\begin{keymarglist}
\item[byeditortr] Used if \bibfield{editor}\slash \bibfield{translator} are identical.
\item[byeditorco] Used if \bibfield{editor}\slash \bibfield{commentator} are identical.
\item[byeditoran] Used if \bibfield{editor}\slash \bibfield{annotator} are identical.
\item[byeditorin] Used if \bibfield{editor}\slash \bibfield{introduction} are identical.
\item[byeditorfo] Used if \bibfield{editor}\slash \bibfield{foreword} are identical.
\item[byeditoraf] Used if \bibfield{editor}\slash \bibfield{aftword} are identical.
\end{keymarglist}
%
Keys for \bibfield{editor}\slash \bibfield{translator}\slash \prm{role} combinations:

\begin{keymarglist}
\item[byeditortrco] Used if \bibfield{editor}\slash \bibfield{translator}\slash \bibfield{commentator} are identical.
\item[byeditortran] Used if \bibfield{editor}\slash \bibfield{translator}\slash \bibfield{annotator} are identical.
\item[byeditortrin] Used if \bibfield{editor}\slash \bibfield{translator}\slash \bibfield{introduction} are identical.
\item[byeditortrfo] Used if \bibfield{editor}\slash \bibfield{translator}\slash \bibfield{foreword} are identical.
\item[byeditortraf] Used if \bibfield{editor}\slash \bibfield{translator}\slash \bibfield{aftword} are identical.
\end{keymarglist}
%
Keys for \bibfield{editor}\slash \bibfield{commentator}\slash \prm{role} combinations:

\begin{keymarglist}
\item[byeditorcoin] Used if \bibfield{editor}\slash \bibfield{commentator}\slash \bibfield{introduction} are identical.
\item[byeditorcofo] Used if \bibfield{editor}\slash \bibfield{commentator}\slash \bibfield{foreword} are identical.
\item[byeditorcoaf] Used if \bibfield{editor}\slash \bibfield{commentator}\slash \bibfield{aftword} are identical.
\end{keymarglist}
%
Keys for \bibfield{editor}\slash \bibfield{annotator}\slash \prm{role} combinations:

\begin{keymarglist}
\item[byeditoranin] Used if \bibfield{editor}\slash \bibfield{annotator}\slash \bibfield{introduction} are identical.
\item[byeditoranfo] Used if \bibfield{editor}\slash \bibfield{annotator}\slash \bibfield{foreword} are identical.
\item[byeditoranaf] Used if \bibfield{editor}\slash \bibfield{annotator}\slash \bibfield{aftword} are identical.
\end{keymarglist}
%
Keys for \bibfield{editor}\slash \bibfield{translator}\slash \bibfield{commentator}\slash \prm{role} combinations:

\begin{keymarglist}
\item[byeditortrcoin] Used if \bibfield{editor}\slash \bibfield{translator}\slash \bibfield{commentator}\slash \bibfield{introduction} are identical.
\item[byeditortrcofo] Used if \bibfield{editor}\slash \bibfield{translator}\slash \bibfield{commentator}\slash \bibfield{foreword} are identical.
\item[byeditortrcoaf] Used if \bibfield{editor}\slash \bibfield{translator}\slash \bibfield{commentator}\slash \bibfield{aftword} are identical.
\end{keymarglist}
%
Keys for \bibfield{editor}\slash \bibfield{translator}\slash \bibfield{annotator}\slash \prm{role} combinations:

\begin{keymarglist}
\item[byeditortranin] Used if \bibfield{editor}\slash \bibfield{annotator}\slash \bibfield{commentator}\slash \bibfield{introduction} are identical.
\item[byeditortranfo] Used if \bibfield{editor}\slash \bibfield{annotator}\slash \bibfield{commentator}\slash \bibfield{foreword} are identical.
\item[byeditortranaf] Used if \bibfield{editor}\slash \bibfield{annotator}\slash \bibfield{commentator}\slash \bibfield{aftword} are identical.
\end{keymarglist}

\paragraph{合并的译者角色,解释为行为}%Concatenated Translator Roles, Expressed as Actions
\label{aut:lng:key:cta}

下面的这些关键词类似于\texttt{bytranslator} 的作用。它们常用来说明译者的附加角色,比如<translated and commented by>, <translated and furnished with an introduction by>, <translated, with a foreword, by>。

%The following keys are similar in function to \texttt{bytranslator}. They are used to indicate additional roles of the translator, \eg\ <translated and commented by>, <translated and furnished with an introduction by>, <translated, with a foreword, by>.

\begin{keymarglist}
\item[bytranslatorco] Used if \bibfield{translator}\slash \bibfield{commentator} are identical.
\item[bytranslatoran] Used if \bibfield{translator}\slash \bibfield{annotator} are identical.
\item[bytranslatorin] Used if \bibfield{translator}\slash \bibfield{introduction} are identical.
\item[bytranslatorfo] Used if \bibfield{translator}\slash \bibfield{foreword} are identical.
\item[bytranslatoraf] Used if \bibfield{translator}\slash \bibfield{aftword} are identical.
\end{keymarglist}
%
Keys for \bibfield{translator}\slash \bibfield{commentator}\slash \prm{role} combinations:

\begin{keymarglist}
\item[bytranslatorcoin] Used if \bibfield{translator}\slash \bibfield{commentator}\slash \bibfield{introduction} are identical.
\item[bytranslatorcofo] Used if \bibfield{translator}\slash \bibfield{commentator}\slash \bibfield{foreword} are identical.
\item[bytranslatorcoaf] Used if \bibfield{translator}\slash \bibfield{commentator}\slash \bibfield{aftword} are identical.
\end{keymarglist}
%
Keys for \bibfield{translator}\slash \bibfield{annotator}\slash \prm{role} combinations:

\begin{keymarglist}
\item[bytranslatoranin] Used if \bibfield{translator}\slash \bibfield{annotator}\slash \bibfield{introduction} are identical.
\item[bytranslatoranfo] Used if \bibfield{translator}\slash \bibfield{annotator}\slash \bibfield{foreword} are identical.
\item[bytranslatoranaf] Used if \bibfield{translator}\slash \bibfield{annotator}\slash \bibfield{aftword} are identical.
\end{keymarglist}

\paragraph{角色,解释为对象}%Roles, Expressed as Objects
\label{aut:lng:key:rob}
与补充材料相关的角色可以解释为对象(<with a commentary by>)而不是职业(<commentator>)或行为(<commented by>)。
%Roles which are related to supplementary material may also be expressed as objects (<with a commentary by>) rather than as functions (<commentator>) or as actions (<commented by>).

\begin{keymarglist}
\item[withcommentator] The expression <with a commentary by \prm{name}>.
\item[withannotator] The expression <with annotations by \prm{name}>.
\item[withintroduction] The expression <with an introduction by \prm{name}>.
\item[withforeword] The expression <with a foreword by \prm{name}>.
\item[withafterword] The expression <with an afterword by \prm{name}>.
\end{keymarglist}

\paragraph{补充材料}%Supplementary Material
\label{aut:lng:key:mat}

\begin{keymarglist}
\item[commentary] The term <commentary>.
\item[annotations] The term <annotations>.
\item[introduction] The term <introduction>.
\item[foreword] The term <foreword>.
\item[afterword] The term <afterword>.
\end{keymarglist}

\paragraph{出版信息细节}%Publication Details
\label{aut:lng:key:pdt}

\begin{keymarglist}
\item[volume] The term <volume>, referring to a book.
\item[volumes] The plural form of \texttt{volume}.
\item[involumes] The term <in>, as used in expressions like <in \prm{number of volumes} volumes>.
\item[jourvol] The term <volume>, referring to a journal.
\item[jourser] The term <series>, referring to a journal.
\item[book] The term <book>, referring to a document division.
\item[part] The term <part>, referring to a part of a book or a periodical.
\item[issue] The term <issue>, referring to a periodical.
\item[newseries] The expression <new series>, referring to a journal.
\item[oldseries] The expression <old series>, referring to a journal.
\item[edition] The term <edition>.
\item[in] The term <in>, referring to the title of a work published as part of another one, \eg\ <\prm{title of article} in \prm{title of journal}>.
\item[inseries] The term <in>, as used in expressions like <volume \prm{number} in \prm{name of series}>.
\item[ofseries] The term <of>, as used in expressions like <volume \prm{number} of \prm{name of series}>.
\item[number] The term <number>, referring to an issue of a journal.
\item[chapter] The term <chapter>, referring to a chapter in a book.
\item[version] The term <version>, referring to a revision number.
\item[reprint] The term <reprint>.
\item[reprintof] The expression <reprint of \prm{title}>.
\item[reprintas] The expression <reprinted as \prm{title}>.
\item[reprintfrom] The expression <reprinted from \prm{title}>.
\item[translationof] The expression <translation of \prm{title}>.
\item[translationas] The expression <translated as \prm{title}>.
\item[translationfrom] The expression <translated from [the] \prm{language}>.
\item[reviewof] The expression <review of \prm{title}>.
\item[origpubas] The expression <originally published as \prm{title}>.
\item[origpubin] The expression <originally published in \prm{year}>.
\item[astitle] The term <as>, as used in expressions like <published by \prm{publisher} as \prm{title}>.
\item[bypublisher] The term <by>, as used in expressions like <published by \prm{publisher}>.
\end{keymarglist}

\paragraph{出版状态}%Publication State
\label{aut:lng:key:pst}

\begin{keymarglist}
\item[inpreparation] The expression <in preparation> (the manuscript is being prepared for publication).
\item[submitted] The expression <submitted> (the manuscript has been submitted to a journal or conference).
\item[forthcoming] The expression <forthcoming> (the manuscript has been accepted by a press or journal).
\item[inpress] The expression <in press> (the manuscript is fully copyedited and out of the author's hands; it is in the final stages of the production process).
\item[prepublished] The expression <pre-published> (the manuscript is published in a preliminary form or location, such as online version in advance of print publication).
\end{keymarglist}

\paragraph{页码}%Pagination
\label{aut:lng:key:pag}

\begin{keymarglist}
\item[page] The term <page>.
\item[pages] The plural form of \texttt{page}.
\item[column] The term <column>, referring to a column on a page.
\item[columns] The plural form of \texttt{column}.
\item[section] The term <section>, referring to a document division (usually abbreviated as \S).
\item[sections] The plural form of \texttt{section} (usually abbreviated as \S\S).
\item[paragraph] The term <paragraph> (\ie a block of text, not to be confused with \texttt{section}).
\item[paragraphs] The plural form of \texttt{paragraph}.
\item[verse] The term <verse> as used when referring to a work which is cited by verse numbers.
\item[verses] The plural form of \texttt{verse}.
\item[line] The term <line> as used when referring to a work which is cited by line numbers.
\item[lines] The plural form of \texttt{line}.
\item[pagetotal] The term <page> as used in \cmd{mkpageprefix}.
\item[pagetotals] The plural form of \texttt{pagetotal}.
\item[columntotal] The term <column>, referring to a column on a page, as used in \cmd{mkpageprefix}.
\item[columntotals] The plural form of \texttt{columntotal}.
\item[sectiontotal] The term <section>, referring to a document division (usually abbreviated as \S),  as used in \cmd{mkpageprefix}.
\item[sectiontotals] The plural form of \texttt{sectiontotal} (usually abbreviated as \S\S).
\item[paragraphtotal] The term <paragraph> (\ie a block of text, not to be confused with \texttt{section}) as used in \cmd{mkpageprefix}.
\item[paragraphtotals] The plural form of \texttt{paragraphtotal}.
\item[versetotal] The term <verse> as used when referring to a work which is cited by verse numbers when used in \cmd{mkpageprefix}.
\item[versetotals] The plural form of \texttt{versetotal}.
\item[linetotal] The term <line> as used when referring to a work which is cited by line numbers when used in \cmd{mkpageprefix}.
\item[linetotals] The plural form of \texttt{linetotal}.
\end{keymarglist}

\paragraph{类型}%Types
\label{aut:lng:key:typ}

下面的关键词常用于\bibtype{thesis},\bibtype{report}, \bibtype{misc} 和其它一些条目的\bibfield{type} 域中:
%The following keys are typically used in the \bibfield{type} field of \bibtype{thesis}, \bibtype{report}, \bibtype{misc}, and other entries:

\begin{keymarglist}
\item[bathesis] An expression equivalent to the term <Bachelor's thesis>.
\item[mathesis] An expression equivalent to the term <Master's thesis>.
\item[phdthesis] The term <PhD thesis>, <PhD dissertation>, <doctoral thesis>, etc.
\item[candthesis] An expression equivalent to the term <Candidate thesis>. Used for <Candidate> degrees that have no clear equivalent to the Master's or doctoral level.
\item[techreport] The term <technical report>.
\item[resreport] The term <research report>.
\item[software] The term <computer software>.
\item[datacd] The term <data \textsc{cd}> or <\textsc{cd-rom}>.
\item[audiocd] The term <audio \textsc{cd}>.
\end{keymarglist}

\paragraph{杂项}%Miscellaneous
\label{aut:lng:key:msc}

\begin{keymarglist}
\item[nodate] The term to use in place of a date when there is no date for an entry \eg\ <n.d.>
\item[and] The term <and>, as used in a list of authors or editors, for example.
\item[andothers] The expression <and others> or <et alii>, used to mark the truncation of a name list.
\item[andmore] Like \texttt{andothers} but used to mark the truncation of a literal list.
\end{keymarglist}

\paragraph{标签}%Labels
\label{aut:lng:key:lab}

下面的字符串用于形成标签,比如<Address: \prm{url}>或者<Abstract: \prm{abstract}>等。
%The following strings are intended for use as labels, \eg\ <Address: \prm{url}> or <Abstract: \prm{abstract}>.

\begin{keymarglist}
\item[url] The term <address> in the sense of an internet address.
\item[urlfrom] An expression like <available from \prm{url}> or <available at \prm{url}>.
\item[urlseen] An expression like <accessed on \prm{date}>, <retrieved on \prm{date}>, <visited on \prm{date}>, referring to the access date of an online resource.
\item[file] The term <file>.
\item[library] The term <library>.
\item[abstract] The term <abstract>.
\item[annotation] The term <annotations>.
\end{keymarglist}

\paragraph{标注}%Citations
\label{aut:lng:key:cit}

标注中使用的传统学术短语:
%Traditional scholarly expressions used in citations:

\begin{keymarglist}
\item[idem] The term equivalent to the Latin <idem> (<the same [person]>).
\item[idemsf] The feminine singular form of \texttt{idem}.
\item[idemsm] The masculine singular form of \texttt{idem}.
\item[idemsn] The neuter singular form of \texttt{idem}.
\item[idempf] The feminine plural form of \texttt{idem}.
\item[idempm] The masculine plural form of \texttt{idem}.
\item[idempn] The neuter plural form of \texttt{idem}.
\item[idempp] The plural form of \texttt{idem} suitable for a mixed gender list of names.
\item[ibidem] The term equivalent to the Latin <ibidem> (<in the same place>).
\item[opcit] The term equivalent to the Latin term <opere citato> (<[in] the work [already] cited>).
\item[loccit] The term equivalent to the Latin term <loco citato> (<[at] the place [already] cited>).
\item[confer] The term equivalent to the Latin <confer> (<compare>).
\item[sequens] The term equivalent to the Latin <sequens> (<[and] the following [page]>), as used to indicate a range of two pages when only the starting page is provided (\eg\ <25\,sq.> or <25\,f.> instead of <25--26>).
\item[sequentes] The term equivalent to the Latin <sequentes> (<[and] the following [pages]>), as used to indicate an open"=ended range of pages when only the starting page is provided (\eg\ <25\,sqq.> or <25\,ff.>).
\item[passim] The term equivalent to the Latin <passim> (<throughout>, <here and there>, <scatteredly>).
\end{keymarglist}
%
其他在标注中常用的短语:
%Other expressions frequently used in citations:

\begin{keymarglist}
\item[see] The term <see>.
\item[seealso] The expression <see also>.
\item[seenote] An expression like <see note \prm{footnote}> or <as in \prm{footnote}>, used to refer to a previous footnote in a citation.
\item[backrefpage] An expression like <see page \prm{page}> or <cited on page \prm{page}>, used to introduce back references in the bibliography.
\item[backrefpages] The plural form of \texttt{backrefpage}, \eg\ <see pages \prm{pages}> or <cited on pages \prm{pages}>.
\item[quotedin] An expression like <quoted in \prm{citation}>, used when quoting a passage which was already a quotation in the cited work.
\item[citedas] An expression like <henceforth cited as \prm{shorthand}>, used to introduce a shorthand in a citation.
\item[thiscite] The expression used in some verbose citation styles to differentiate between the page range of the cited item (typically an article in a journal, collection, or conference proceedings) and the page number the citation refers to. For example: \enquote{Author, Title, in: Book, pp. 45--61, \texttt{thiscite} p. 52.}
\end{keymarglist}

\paragraph{月份名}%Month Names
\label{aut:lng:key:mon}

\begin{keymarglist}
\item[january] The name <January>.
\item[february] The name <February>.
\item[march] The name <March>.
\item[april] The name <April>.
\item[may] The name <May>.
\item[june] The name <June>.
\item[july] The name <July>.
\item[august] The name <August>.
\item[september] The name <September>.
\item[october] The name <October>.
\item[november] The name <November>.
\item[december] The name <December>.
\end{keymarglist}

\paragraph{语言名}%Language Names
\label{aut:lng:key:lng}

\begin{keymarglist}
\item[langamerican] The language <American> or <American English>.
\item[langbrazilian] The language <Brazilian> or <Brazilian Portuguese>.
\item[langbulgarian] The language <Bulgarian>.
\item[langcatalan] The language <Catalan>.
\item[langcroatian] The language <Croatian>.
\item[langczech] The language <Czech>.
\item[langdanish] The language <Danish>.
\item[langdutch] The language <Dutch>.
\item[langenglish] The language <English>.
\item[langestonian] The language <Estonian>.
\item[langfinnish] The language <Finnish>.
\item[langfrench] The language <French>.
\item[langgerman] The language <German>.
\item[langgreek] The language <Greek>.
\item[langhungarian] The language <Hungarian>.
\item[langitalian] The language <Italian>.
\item[langjapanese] The language <Japanese>.
\item[langlatin] The language <Latin>.
\item[langlatvian] The language <Latvian>.
\item[langnorwegian] The language <Norwegian>.
\item[langpolish] The language <Polish>.
\item[langportuguese] The language <Portuguese>.
\item[langrussian] The language <Russian>.
\item[langslovak] The language <Slovak>.
\item[langslovene] The language <Slovene>.
\item[langspanish] The language <Spanish>.
\item[langswedish] The language <Swedish>.
\item[langukrainian] The language <Ukrainian>.
\end{keymarglist}
%
下面的字符串用于像<translated from [the] English by \prm{translator}>之类的短语中:
%The following strings are intended for use in phrases like <translated from [the] English by \prm{translator}>:

\begin{keymarglist}
\item[fromamerican] The expression <from [the] American> or <from [the] American English>.
\item[frombrazilian] The expression <from [the] Brazilian> or <from [the] Brazilian Portuguese>.
\item[frombulgarian] The expression <from [the] Bulgarian>.
\item[fromcatalan] The expression <from [the] Catalan>.
\item[fromcroatian] The expression <from [the] Croatian>.
\item[fromczech] The expression <from [the] Czech>.
\item[fromdanish] The expression <from [the] Danish>.
\item[fromdutch] The expression <from [the] Dutch>.
\item[fromenglish] The expression <from [the] English>.
\item[fromestonian] The expression <from [the] Estonian>.
\item[fromfinnish] The expression <from [the] Finnish>.
\item[fromfrench] The expression <from [the] French>.
\item[fromgerman] The expression <from [the] German>.
\item[fromgreek] The expression <from [the] Greek>.
\item[fromhungarian] The language <from [the] Hungarian>.
\item[fromitalian] The expression <from [the] Italian>.
\item[fromjapanese] The expression <from [the] Japanese>.
\item[fromlatin] The expression <from [the] Latin>.
\item[fromlatvian] The expression <from [the] Latvian>.
\item[fromnorwegian] The expression <from [the] Norwegian>.
\item[frompolish] The expression <from [the] Polish>.
\item[fromportuguese] The expression <from [the] Portuguese>.
\item[fromrussian] The expression <from [the] Russian>.
\item[fromslovak] The expression <from [the] Slovak>.
\item[fromslovene] The expression <from [the] Slovene>.
\item[fromspanish] The expression <from [the] Spanish>.
\item[fromswedish] The expression <from [the] Swedish>.
\item[fromukrainian] The expression <from [the] Ukrainian>.
\end{keymarglist}

\paragraph{国名}%Country Names
\label{aut:lng:key:cnt}

国名利用\texttt{country} 加\acr{ISO}-3166国家代码实现本地化。不同语言的译文的简易版本应是\acr{ISO}-3166国家代码。注意:仅默认定义了少量国名关键词,用来说明这一方法。这些关键词用在\bibtype{patent} 条目的\bibfield{location} 域中,但也可以用于其他地方。
%Country names are localised by using the string \texttt{country} plus the \acr{ISO}-3166 country code as the key. The short version of the translation should be the \acr{ISO}-3166 country code. Note that only a small number of country names is defined by default, mainly to illustrate this scheme. These keys are used in the \bibfield{location} list of \bibtype{patent} entries but they may be useful for other purposes as well.

\begin{keymarglist}
\item[countryde] 名称<Germany>,缩写为\texttt{DE}
%The name <Germany>, abbreviated as \texttt{DE}.
\item[countryeu] 名称<European Union>,缩写为\texttt{EU}
%The name <European Union>, abbreviated as \texttt{EU}.
\item[countryep] 类似于\opt{countryeu},但缩写为\texttt{EP}。用于\bibfield{patent} 条目。
%Similar to \opt{countryeu} but abbreviated as \texttt{EP}. This is intended for \bibfield{patent} entries.
\item[countryfr] 名称<France>,缩写为\texttt{FR}
%The name <France>, abbreviated as \texttt{FR}.
\item[countryuk] 名称<United Kingdom>,缩写为\texttt{GB}(根据\acr{ISO}-3166)
%The name <United Kingdom>, abbreviated (according to \acr{ISO}-3166) as \texttt{GB}.
\item[countryus] 名称<United States of America>,缩写为\texttt{US}
%The name <United States of America>, abbreviated as \texttt{US}.
\end{keymarglist}

\paragraph{专利和专利申请}%Patents and Patent Requests
\label{aut:lng:key:pat}

与专利相关的字符串通过使用术语\texttt{patent} 加\acr{ISO}-3166国家代码作为关键词实现本地化。注意:仅默认定义了少量专利关键词,用来说明这一方法。这些关键词用在\bibtype{patent} 的\bibfield{type} 域中。
%Strings related to patents are localised by using the term \texttt{patent} plus the \acr{ISO}-3166 country code as the key. Note that only a small number of patent keys is defined by default, mainly to illustrate this scheme. These keys are used in the \bibfield{type} field of \bibtype{patent} entries.

\begin{keymarglist}
\item[patent] 通用术语<patent>
%The generic term <patent>.
\item[patentde] 短语<German patent>
%The expression <German patent>.
\item[patenteu] 短语<European patent>
%The expression <European patent>.
\item[patentfr] 短语<French patent>
%The expression <French patent>.
\item[patentuk] 短语<British patent>
%The expression <British patent>.
\item[patentus] 短语<U.S. patent>
%The expression <U.S. patent>.
\end{keymarglist}
%
专利申请以类似方式处理,使用字符串\texttt{patreq} 作为关键词的基本名称:
%Patent requests are handled in a similar way, using the string \texttt{patreq} as the base name of the key:

\begin{keymarglist}
\item[patreq] 通用术语<patent request>
%The generic term <patent request>.
\item[patreqde] 短语<German patent request>
%The expression <German patent request>.
\item[patreqeu] 短语<European patent request>
%The expression <European patent request>.
\item[patreqfr] 短语<French patent request>
%The expression <French patent request>.
\item[patrequk] 短语<British patent request>
%The expression <British patent request>.
\item[patrequs] 短语<U.S. patent request>
%The expression <U.S. patent request>.
\end{keymarglist}

\paragraph{日期和时间}%Dates and Times
\label{aut:lng:key:dt}
标准纪元的缩略词。支持世俗的或基督教两种版本。
%Abbreviation strings for standard eras. Both secular and Christian variants are supported.

\begin{keymarglist}
\item[commonera] 纪元<CE>(表示:公元)
%The era <CE>
\item[beforecommonera] 纪元<BCE>(表示:公元前)
%The era <BCE>
\item[annodomini] 纪元<AD>(表示:公元)
%The era <AD>
\item[beforechrist] 纪元<BC>(表示:公元前)
%The era <BC>
\end{keymarglist}

<circa>日期的缩略词:
%Abbreviation strings for <circa> dates:

\begin{keymarglist}
\item[circa] 词<circa>(表示:大约在)
%The string <circa>
\end{keymarglist}

从\acr{EDTF} 日期解析的季节的缩略词:
%Abbreviation strings for seasons parsed from \acr{ISO8601-2} Extended Format dates:

\begin{keymarglist}
\item[spring] 词<spring>(表示:春)
\item[summer] 词<summer>(表示:夏)
\item[autumn] 词<autumn>(表示:秋)
\item[winter] 词<winter>(表示:冬)
\end{keymarglist}

AM/PM的缩略词:
%Abbreviation strings for AM/PM:

\begin{keymarglist}
\item[am] 词<AM>(表示:上午)
\item[pm] 词<PM>(表示:下午)
\end{keymarglist}

\subsection{格式化命令}%Formatting Commands
\label{aut:fmt}

本节对应于用户指南部分的\secref{use:fmt} 节。著录和标注样式需要一些本节讨论的命令和工具来提供一定程度的高层可配置性。用户不需要非得写一个新的样式,如果仅要求修改文献表中的空格以及标注中的标点的话。
%This section corresponds to \secref{use:fmt} in the user part of this manual. Bibliography and citation styles should incorporate the commands and facilities discussed in this section in order to provide a certain degree of high"=level configurability. Users should not be forced to write new styles if all they want to do is modify the spacing in the bibliography or the punctuation used in citations.

\subsubsection{用户可定义的命令和钩子}%User-definable Commands and Hooks
\label{aut:fmt:fmt}

本节对应用户指南部分的\secref{use:fmt:fmt} 节。这里讨论的命令和钩子可以由用户重定义,但著录和标注样式可能会提供一个不同于宏包定义的默认定义。这些命令在\path{biblatex.def} 中定义。注意所有的这些命令以\cmd{mk\dots} 开头具有一个必选参数。
%This section corresponds to \secref{use:fmt:fmt} in the user part of the manual. The commands and hooks discussed here are meant to be redefined by users, but bibliography and citation styles may provide a default definition which is different from the package default. These commands are defined in \path{biblatex.def}. Note that all commands starting with \cmd{mk\dots} take one mandatory argument.

\begin{ltxsyntax}

\csitem{bibnamedelima}
这一分隔符控制构成姓名成分的元素间的间距。它由后端自动添加,位于第一个元素后面如果该元素少于三个字符长度和最后一个元素之前。默认的定义为\cmd{addhighpenspace},即一个由\cnt{highnamepenalty} 计数器(\secref{use:fmt:len} 节)值控制的空间,更多细节见\secref{use:cav:nam}。
%This delimiter controls the spacing between the elements which make up a name part. It is inserted automatically by the backend after the first name element if the element is less than three characters long and before the last element. The default definition is \cmd{addhighpenspace}, \ie a space penalized by the value of the \cnt{highnamepenalty} counter (\secref{use:fmt:len}). Please refer to \secref{use:cav:nam} for further details.

\csitem{bibnamedelimb}
这一分隔符控制构成姓名成分的元素间的间距。它由后端自动添加,位于所有元素之间,但存在\cmd{bibnamedelima} 时不添加。默认的定义为\cmd{addlowpenspace},即一个由\cnt{lownamepenalty} 计数器(\secref{use:fmt:len} 节)值控制的空间,更多细节见\secref{use:cav:nam}。
%This delimiter controls the spacing between the elements which make up a name part. It is inserted automatically by the backend between all name elements where \cmd{bibnamedelima} does not apply. The default definition is \cmd{addlowpenspace}, \ie a space penalized by the value of the \cnt{lownamepenalty} counter (\secref{use:fmt:len}). Please refer to \secref{use:cav:nam} for further details.

\csitem{bibnamedelimc}
这一分隔符控制构成姓名成分的元素间的间距。它由后端自动添加,位于前缀和姓之间,当\kvopt{useprefix}{true} 时。默认的定义为\cmd{addhighpenspace},即一个由\cnt{highnamepenalty} 计数器(\secref{use:fmt:len} 节)值控制的空间,更多细节见\secref{use:cav:nam}。
%This delimiter controls the spacing between name parts. The default name formats use it between the name prefix and the last name if \kvopt{useprefix}{true}. The default definition is \cmd{addhighpenspace}, \ie a space penalized by the value of the \cnt{highnamepenalty} counter (\secref{use:fmt:len}). Please refer to \secref{use:cav:nam} for further details.

\csitem{bibnamedelimd}
这一分隔符控制构成姓名成分的元素间的间距。它由后端自动添加,位于所有元素之间,但存在\cmd{bibnamedelimc} 时不添加。默认的定义为\cmd{addlowpenspace},即一个由\cnt{lownamepenalty} 计数器(\secref{use:fmt:len} 节)值控制的空间,更多细节见\secref{use:cav:nam}。
%This delimiter controls the spacing between name parts. The default name formats use it between all name parts where \cmd{bibnamedelimc} does not apply. The default definition is \cmd{addlowpenspace}, \ie a space penalized by the value of the \cnt{lownamepenalty} counter (\secref{use:fmt:len}). Please refer to \secref{use:cav:nam} for further details.

\csitem{bibnamedelimi}
这一分隔符代替首字母后的\cmd{bibnamedelima/b}。注意: 这仅应用于在\file{bib} 文件中给出的首字母后,而不是由\biblatex 自动生成的首字母后,因为它使用自己的分隔符。
%This delimiter replaces \cmd{bibnamedelima/b} after initials. Note that this only applies to initials given as such in the \file{bib} file, not to the initials automatically generated by \biblatex which use their own set of delimiters.

\csitem{bibinitperiod}

当不应用\cmd{bibinithyphendelim} 时由后端自动在所有缩写首字母后插入的标点。默认的定义是句点(\cmd{adddot})。更多细节见\secref{use:cav:nam}。
%The punctuation inserted automatically by the backend after all initials unless \cmd{bibinithyphendelim} applies. The default definition is a period (\cmd{adddot}). Please refer to \secref{use:cav:nam} for further details.

\csitem{bibinitdelim}
当不应用\cmd{bibinithyphendelim} 时由后端自动在多个缩写首字母见插入空间。默认的定义是不可断行的词内空格。更多细节见\secref{use:cav:nam}。
%The spacing inserted automatically by the backend between multiple initials unless \cmd{bibinithyphendelim} applies. The default definition is an unbreakable interword space. Please refer to \secref{use:cav:nam} for further details.

\csitem{bibinithyphendelim}
由后端自动在连字符连接的姓名成分的缩写首字母间插入的标点,代替\cmd{bibinitperiod} 和\cmd{bibinitdelim}。默认的定义时句点加一个不可断行连字符。更多细节见\secref{use:cav:nam}。
%The punctuation inserted automatically by the backend between the initials of hyphenated name parts, replacing \cmd{bibinitperiod} and \cmd{bibinitdelim}. The default definition is a period followed by an unbreakable hyphen. Please refer to \secref{use:cav:nam} for further details.

\csitem{bibindexnamedelima}
用于在索引中取代\cmd{bibnamedelima}。
%Replaces \cmd{bibnamedelima} in the index.

\csitem{bibindexnamedelimb}
用于在索引中取代\cmd{bibnamedelimb}。
%Replaces \cmd{bibnamedelimb} in the index.

\csitem{bibindexnamedelimc}
用于在索引中取代\cmd{bibnamedelimc}。
%Replaces \cmd{bibnamedelimc} in the index.

\csitem{bibindexnamedelimd}
用于在索引中取代\cmd{bibnamedelimd}。
%Replaces \cmd{bibnamedelimd} in the index.

\csitem{bibindexnamedelimi}
用于在索引中取代\cmd{bibnamedelimi}。
%Replaces \cmd{bibnamedelimi} in the index.

\csitem{bibindexinitperiod}
用于在索引中取代\cmd{bibinitperiod}。
%Replaces \cmd{bibinitperiod} in the index.

\csitem{bibindexinitdelim}
用于在索引中取代\cmd{bibinitdelim}。
%Replaces \cmd{bibinitdelim} in the index.

\csitem{bibindexinithyphendelim}
用于在索引中取代\cmd{bibinithyphendelim}。
%Replaces \cmd{bibinithyphendelim} in the index.

\csitem{revsdnamepunct}
当姓和名顺序相反时两者之间插入的标点。默认是逗号。该命令应在姓名列表的格式化指令中使用。更多细节见\secref{use:cav:nam}。
%The punctuation to be printed between the given and family name parts when a name is reversed. The default is a comma. This command should be incorporated in formatting directives for name lists.  Please refer to \secref{use:cav:nam} for further details.

\csitem{bibnamedash}
用于代替参考文献表中接连再出现的责任者的破折号。默认是一个<em>或<en>破折号,根据文献表的缩进选取。
%The dash to be used as a replacement for recurrent authors or editors in the bibliography. The default is an <em> or an <en> dash, depending on the indentation of the list of references.

\csitem{labelnamepunct}\DeprecatedMark

%A separator to be printed after the name used for alphabetizing in the bibliography (\bibfield{author} or \bibfield{editor}, if the \bibfield{author} field is undefined) instead of \cmd{newunitpunct}. The default is \cmd{newunitpunct}, \ie it is not handled differently from regular unit punctuation but permits convenient reconfiguration. This punctuation command is deprecated and has been superseded by the context-sensitive \cmd{nametitledelim} (see \secref{use:fmt:csd}). For backwards compatibility reasons, however, \cmd{nametitledelim} still defaults to \cmd{labelnamepunct} in the \texttt{bib} and \texttt{biblist} contexts. Style authors may want to consider replacing \cmd{labelnampunct} with \texttt{\textbackslash printdelim\{nametitledelim\}} and users may want to prefer modifying the context-sensitive \texttt{nametitledelim} with \cmd{DeclareDelimFormat} over redefining \cmd{labelnamepunct}.


该分隔符在文献表中用来按字母顺序排列的责任者(\bibfield{author} 或\bibfield{editor},如果\bibfield{author} 域未定义)之后打印。使用该分隔符代替该位置的\cmd{newunitpunct}。默认是\cmd{newunitpunct},即它与一般的单元标点并无不同,但允许方便地重设。


\csitem{subtitlepunct}
该分隔符在\bibfield{title} 和\bibfield{subtitle} 域,\bibfield{booktitle} 和\bibfield{booksubtitle},以及\bibfield{maintitle} 和 \bibfield{mainsubtitle} 之间打印。替代该位置处的\cmd{newunitpunct}。默认是\cmd{newunitpunct},即它与一般的单元标点并无不同,但允许方便地重设。
%The separator to be printed between the fields \bibfield{title} and \bibfield{subtitle}, \bibfield{booktitle} and \bibfield{booksubtitle}, as well as \bibfield{maintitle} and \bibfield{mainsubtitle}. Use this separator instead of \cmd{newunitpunct} at this location. The default is \cmd{newunitpunct}, \ie it is not handled differently from regular unit punctuation but permits convenient reconfiguration.

\csitem{intitlepunct}
该分隔符在«in»与其后面的一些条目类型的标题之间打印。替代该位置处的\cmd{newunitpunct}。默认是一个冒号加词内空格。
%The separator to be printed between the word «in» and the following title in entry types such as \bibtype{article}, \bibtype{inbook}, \bibtype{incollection}, etc. Use this separator instead of \cmd{newunitpunct} at this location. The default definition is a colon plus an interword space.

\csitem{bibpagespunct}
该分隔符在\bibfield{pages} 域前打印。替代该位置处的\cmd{newunitpunct}。默认是一个逗号加词内空格。
%The separator to be printed before the \bibfield{pages} field. Use this separator instead of \cmd{newunitpunct} at this location. The default is a comma plus an interword space.

\csitem{bibpagerefpunct}
该分隔符在\bibfield{pageref} 域前打印。替代该位置处的\cmd{newunitpunct}。默认是一个词内空格。
%The separator to be printed before the \bibfield{pageref} field. Use this separator instead of \cmd{newunitpunct} at this location. The default is an interword space.

\csitem{multinamedelim}
该分隔符在\bibfield{author} 或\bibfield{editor} 之类的姓名列表的各项之间打印,如果列表中存在超过2个姓名的话。如果列表中仅有两个姓名,则使用\cmd{finalnamedelim}。该命令应在姓名列表的所有格式化指令中使用。
%The delimiter to be printed between multiple items in a name list like \bibfield{author} or \bibfield{editor} if there are more than two names in the list. If there are only two names in the list, use the \cmd{finalnamedelim} instead. This command should be incorporated in all formatting directives for name lists.

\csitem{finalnamedelim}
用于替代姓名列表中最后一个名字前的\cmd{multinamedelim}。
%Use this command instead of \cmd{multinamedelim} before the final name in a name list.

\csitem{revsdnamedelim}

在由两个姓名构成的姓名列表中第一个姓名之后打印的额外分隔符(加在\cmd{finalnamedelim} 后面),如果第一个姓名的姓和名顺序相反的话。该命令应在姓名列表的所有格式化指令中使用。
%The extra delimiter to be printed after the first name in a name list consisting of two names (in addition to \cmd{finalnamedelim}) if the first name is reversed. This command should be incorporated in all formatting directives for name lists.

\csitem{andothersdelim}

一个\bibfield{author} 或\bibfield{editor} 之类的姓名列表被截短后在本地化字符串<\texttt{andothers}>前打印的分隔符。该命令应在姓名列表的所有格式化指令中配合使用。
%The delimiter to be printed before the localisation string <\texttt{andothers}> if a name list like \bibfield{author} or \bibfield{editor} is truncated. This command should be incorporated in all formatting directives for name lists.

\csitem{multilistdelim}
该分隔符在\bibfield{publisher} 或\bibfield{location} 之类的文本列表的各项之间打印,如果列表中存在超过2个文本项的话。如果列表中仅有两项,则使用\cmd{finallistdelim}。该命令应在文本列表的所有格式化指令中使用。
%The delimiter to be printed between multiple items in a literal list like \bibfield{publisher} or \bibfield{location} if there are more than two names in the list. If there are only two items in the list, use the \cmd{finallistdelim} instead. This command should be incorporated in all formatting directives for literal lists.
%这里原文有笔误。

\csitem{finallistdelim}
用于替代文本列表中最后一个项前的\cmd{multilistdelim}。
%Use this command instead of \cmd{multilistdelim} before the final item in a literal list.

\csitem{andmoredelim}
一个\bibfield{publisher} 或\bibfield{location} 之类的文本列表被截短后在本地化字符串<\texttt{andmore}>前打印的分隔符。该命令应在文本列表的所有格式化指令中使用。
%The delimiter to be printed before the localisation string <\texttt{andmore}> if a literal list like \bibfield{publisher} or \bibfield{location} is truncated. This command should be incorporated in all formatting directives for literal lists.

\csitem{multicitedelim}
该分隔符在传递给当个标注命令的多个条目关键词之间打印。该命令应在标注命令定义中使用,例如在传递给\cmd{DeclareCiteCommand} 的\prm{sepcode} 参数中。详见\secref{aut:cbx:cbx}。
%The delimiter printed between citations if multiple entry keys are passed to a single citation command. This command should be incorporated in the definition of all citation commands, for example in the \prm{sepcode} argument passed to \cmd{DeclareCiteCommand}. See \secref{aut:cbx:cbx} for details.

\csitem{supercitedelim}
类似于\cmd{multicitedelim},但仅用于\cmd{supercite} 命令中。
%Similar to \cmd{multinamedelim}, but intended for the \cmd{supercite} command only.
%这里原文有笔误。

\csitem{compcitedelim}
类似于\cmd{multicitedelim},但仅用于压缩(<compress>)多个引用的标注样式中,即打印作者一次,如果后面接着的引用文献的作者相同的话。
%Similar to \cmd{multicitedelim}, but intended for citation styles that <compress> multiple citations, \ie print the author only once if subsequent citations share the same author etc.

\csitem{textcitedelim}
类似于\cmd{multicitedelim},但仅用于\cmd{textcite} 和相关命令(\secref{use:cit:cbx})中。
%Similar to \cmd{multicitedelim}, but intended for \cmd{textcite} and related commands (\secref{use:cit:cbx}).

\csitem{nametitledelim}
在责任者和标题之间打印的分隔符。该命令应在作者标题制和一些长标注样式的所有标注命令定义中使用。
%The delimiter to be printed between the author\slash editor and the title. This command should be incorporated in the definition of all citation commands of author-title and some verbose citation styles.

\csitem{nameyeardelim}
在责任者和年份之间打印的分隔符。该命令应在作者年制标注样式的所有标注命令定义中使用。
%The delimiter to be printed between the author\slash editor and the year. This command should be incorporated in the definition of all citation commands of author-year citation styles.

\csitem{namelabeldelim}
在name\slash title和标签之间打印的分隔符。该命令应在字母顺序编码和数字顺序编码标注样式的所有标注命令定义中使用。
%The delimiter printed between the name\slash title and the label. This command should be incorporated in the definition of all citation commands of alphabetic and numeric citation styles.

\csitem{nonameyeardelim}
作者年制标注样式中labelname的替代者(当labelname不存在时)与年份之间打印的分隔符。仅用于无labelname的情况,因为当其存在时使用的是\cmd{nameyeardelim}。
%The delimiter printed between the substitute for the labelname when it does not exist (usually the label or title in standard styles) and the year in author-year citation styles. This is only used when there is no labelname since when the labelname exists, \cmd{nameyeardelim} is used.

\csitem{authortypedelim}
The delimiter printed between the author and the \texttt{authortype}.

\csitem{editortypedelim}
The delimiter printed between the editor and the \texttt{editor} or \texttt{editortype} string.

\csitem{translatortypedelim}
The delimiter printed between the translator and the \texttt{translator} string.

\csitem{volcitedelim}
在\cmd{volcite} 和相关命令的卷部分和页码/文本部分之间打印的分隔符(见\secref{use:cit:spc})。
%The delimiter to be printed between the volume portion and the page/text portion of \cmd{volcite} and related commands (\secref{use:cit:spc}).

\csitem{prenotedelim}
在标注命令的\prm{prenote} 参数后面打印的分隔符。
%The delimiter to be printed after the \prm{prenote} argument of a citation command.

\csitem{postnotedelim}
在标注命令的\prm{postnote} 参数前面打印的分隔符。
%The delimiter to be printed before the \prm{postnote} argument of a citation command.

\csitem{extpostnotedelim}
当postnote出现在标注括号外时,标注命令中在标注和插入的\prm{postnote} 参数之间打印的分隔符。在标准样式中,这仅发生在标注使用条目的缩略域时。
%The delimiter printed between the citation and the parenthetical \prm{postnote} argument of a citation command when the postnote occurs outside of the citation parentheses. In the standard styles, this occurs when the citation uses the shorthand field of the entry.

\cmditem{mkbibnamefamily}{text}
姓的格式化钩子,用于姓名列表的所有格式化指令中。
%Formatting hook for the family name, to be used in all formatting directives for name lists.

\cmditem{mkbibnamegiven}{text}
类似于\cmd{mkbibnamefamily},当用于名。
%Similar to \cmd{mkbibnamefamily}, but intended for the given name.

\cmditem{mkbibnameprefix}{text}
类似于\cmd{mkbibnamefamily},当用于姓名前缀。
%Similar to \cmd{mkbibnamefamily}, but intended for the name prefix.

\cmditem{mkbibnamesuffix}{text}
类似于\cmd{mkbibnamefamily},当用于姓名后缀。
%Similar to \cmd{mkbibnamefamily}, but intended for the name suffix.

\csitem{relatedpunct}
在相关类型参考文献本地化字符串和第一个关联条目数据之间的分隔符。
%The separator between the relatedtype bibliography localisation string and the data from the first related entry.

\csitem{relateddelim}
在多个关联条目数据之间打印的分隔符。默认是断行。
%The separator between the data of multiple related entries. The default definition is a linebreak.

\csitem{relateddelim$<$relatedtype$>$}
在<relatedtype>类型的关联条目中的多个关联条目数据之间打印的分隔符。没有默认设置,如果不指定具体类型的分隔符,则使用\cmd{relateddelim}。
%The separator between the data of multiple related entries inside related entries of type <relatedtype>. There is no default, if such a type-specific delimiter does not exist, \cmd{relateddelim} is used.

\csitem{begrelateddelim}
%The generic separator before the block of related entries. The default definition is \cmd{newunitpunct}.
在关联条目块前面的一般分隔符。默认定义是\cmd{newunitpunct}.

\csitem{begrelateddelim$<$relatedtype$>$}
%The separator between the block of related entries of type <relatedtype>. There is no default, if such a type-specific delimiter does not exist, \cmd{relateddelim} is used.
在<relatedtype>类型关联条目块前面的一般分隔符。无默认定义,当具体类型的定义不存在,则使用\cmd{relateddelim} 。



\end{ltxsyntax}

\subsubsection{具体语言的命令}%Language-specific Commands
\label{aut:fmt:lng}

本节对应用户指南部分的\secref{use:fmt:lng} 节。下面讨论的命令常在本地化模型中处理,但用户可能根据具体的语言重定义。注意,所有的命令以\cmd{mk\dots} 开头,具有一个或更多的必选参数。
%This section corresponds to \secref{use:fmt:lng} in the user part of the manual. The commands discussed here are usually handled by the localisation modules, but may also be redefined by users on a per"=language basis. Note that all commands starting with \cmd{mk\dots} take one or more mandatory arguments.

\begin{ltxsyntax}

\csitem{bibrangedash}

具体语言的范围破折号,默认是\cmd{textendash}。
%The language specific range dash. Defaults to \cmd{textendash}.

\csitem{bibrangessep}

用于多个范围间的具体语言的分隔符。默认是逗号加一个空格。
%The language specific separator to be used between multiple ranges. Defaults to a comma followed by a space.

\csitem{bibdatesep}

简洁日期格式中各日期成分之间使用的具体语言的分隔符。默认是连字符(\cmd{hyphen})。
%The language specific separator used between date components in terse date formats. Defaults to \cmd{hyphen}.

\csitem{bibdaterangesep}

用于日期范围之间的具体语言的分隔符。除了\opt{ymd} 格式默认是\cmd{slash} 外,其它所有日期格式中默认是\cmd{textendash}。\opt{edtf} 选项的日期格式是硬编码(不轻易改变的)为\cmd{slash},因为这是一种需符合标准的格式。
%The language specific separator to be used for date ranges. Defaults to \cmd{textendash} for all date formats apart from \opt{ymd} which defaults to a \cmd{slash}. The date format option \opt{edtf} is hard-coded to \cmd{slash} since this is a standards compliant format.

\csitem{mkbibdatelong}

取三个域的名作为参数,对应三个日期成分(以year\slash month\slash day的顺序),并使用这些域的值以具体语言的长日期格式打印日期。
%Takes the names of three field as arguments which correspond to three date components (in the order year\slash month\slash day) and uses their values to print the date in the language specific long date format.

\csitem{mkbibdateshort}

类似于\cmd{mkbibdatelong},但使用具体语言的短日期格式。
%Similar to \cmd{mkbibdatelong} but using the language specific short date format.

\csitem{mkbibtimezone}

修改作为唯一参数传递进来的时区。默认情况下,修改<Z>为\cmd{bibtimezone} 的值。
%Modifies a timezone string passed in as the only argument. By default this changes <Z> to the value of \cmd{bibtimezone}.

\csitem{bibdateuncertain}

当全局选项\opt{dateuncertain} 启用时,在不确定日期后用的具体语言的标记。默认是一个空格加一个问号。
%The language specific marker to be used after uncertain dates when the global option \opt{dateuncertain} is enabled. Defaults to a space followed by a question mark.


\csitem{bibdateeraprefix}

当\opt{dateera} 设为<astronomical>时,在日期范围中作为起始BCE/BC日期前缀打印的具体语言标记。有定义的话,默认是\cmd{textminus},否则是\cmd{textendash}。
%The language specific marker which is printed as a prefix to beginning BCE/BC dates in a date range when the option \opt{dateera} is set to <astronomical>. Defaults to \cmd{textminus}, if defined and \cmd{textendash} otherwise.

\csitem{bibdateeraendprefix}

当\opt{dateera} 设为<astronomical>时,在日期范围中作为终点BCE/BC日期前缀打印的具体语言标记。当\cmd{bibdaterangesep} 设置为破折号(dash)时默认是短空格(thin space),否则是\cmd{bibdateeraprefix}。这是一个独立宏,所以可以在一个负日期标记(比如跟在一个破折号日期范围标记之后的)前添加额外的空格,因为它看起来有点奇特。
%The language specific marker which is printed as a prefix to end BCE/BC dates in a date range when the option \opt{dateera} is set to <astronomical>. Defaults to a thin space followed by \cmd{bibdateeraprefix} when \cmd{bibdaterangesep} is set to a dash and to \cmd{bibdateeraprefix} otherwise.  This is a separate macro so that you may add extra space before a negative date marker which, for example follows a dash date range marker as this can look a little odd.

\csitem{bibtimesep}

分隔时间成分的具体语言标记,默认是分号。
%The language specific marker which separates time components. Default to a colon.

\csitem{bibutctimezone}

UTC时区的具体语言的打印字符串,默认是<Z>。
%The language specific string printed for the UTC timezone. Defaults to <Z>.

\csitem{bibtimezonesep}

分隔时间的可选时区成分的具体语言的标记,默认为空。
%The language specific marker which separates an optional time zone component from a time. Empty by default.

\csitem{bibdatetimesep}

当时间和日期同时打印时分隔时间成分和日期成分的具体语言的分隔符。(见\secref{use:opt:pre:gen} 节的\opt{$<$datetype$>$dateusetime} 选项)。默认是一个空格对于non-EDTF输出格式,对于EDTF输出格式则是'T'。
%The language specific separator printed between date and time components when printing time components along with date components (see the \opt{$<$datetype$>$dateusetime} option in \secref{use:opt:pre:gen}). Defaults to a space for non-EDTF output formats, and 'T' for EDTF output format.

\csitem{finalandcomma}

在枚举中最后的<and>前插入的逗号,如果可以用于具体的语言。
%Prints the comma to be inserted before the final <and> in an enumeration, if applicable in the respective language.

\csitem{finalandsemicolon}

在枚举中最后的<and>前插入的分号,如果可以用于具体的语言。
%Prints the semicolon to be inserted before the final <and> in an enumeration, if applicable in the respective language.

\cmditem{mkbibordinal}{integer}

取一个整数参数并打印成一般数字。
%Takes an integer argument and prints it as an ordinal number.

\cmditem{mkbibmascord}{integer}

类似于\cmd{mkbibordinal},但打印一个男性用的序号,如果可以用于具体的语言。
%Similar to \cmd{mkbibordinal}, but prints a masculine ordinal, if applicable in the respective language.

\cmditem{mkbibfemord}{integer}

类似于\cmd{mkbibordinal},但打印一个女性用的序号,如果可以用于具体的语言。
%Similar to \cmd{mkbibordinal}, but prints a feminine ordinal, if applicable in the respective language.

\cmditem{mkbibneutord}{integer}

类似于\cmd{mkbibordinal},但打印一个中性用的序号,如果可以用于具体的语言。
%Similar to \cmd{mkbibordinal}, but prints a neuter ordinal, if applicable in the respective language.

\cmditem{mkbibordedition}{integer}

类似于\cmd{mkbibordinal},但与术语<edition>连用。
%Similar to \cmd{mkbibordinal}, but intended for use with the term <edition>.

\cmditem{mkbibordseries}{integer}

类似于\cmd{mkbibordinal},但与术语<series>连用。
%Similar to \cmd{mkbibordinal}, but intended for use with the term <series>.

\end{ltxsyntax}

\subsubsection{用户可定义的长度和计数器}%User-definable Lengths and Counters
\label{aut:fmt:len}

本节对应用户指南部分的\secref{use:fmt:len} 节。下面讨论的长度和计数器用户可以修改。著录和标注样式需要的时候应该使用它们,也可以提供不同于biblatex包提供的默认设置
%This section corresponds to \secref{use:fmt:len} in the user part of the manual. The length registers and counters discussed here are meant to be altered by users. Bibliography and citation styles should incorporate them where applicable and may also provide a default setting which is different from the package default.

\begin{ltxsyntax}

\lenitem{bibhang}

如果用的话,是参考文献表的悬挂缩进。该长度在加载时初始化为\cmd{parindent}。如果\cmd{parindent} 因为某些原因设置为0,\cmd{bibhang} 将默认为\texttt{1em}。
%The hanging indentation of the bibliography, if applicable. This length is initialized to \cmd{parindent} at load-time. If \cmd{parindent} is zero length for some reason, \cmd{bibhang} will default to \texttt{1em}.

\lenitem{biblabelsep}

条目和对应标签之间的水平间距。使用\env{list} 环境并打印标签的参考文献样式应在环境定义中设置\len{labelsep} 为\len{biblabelsep}。
%The horizontal space between entries and their corresponding labels. Bibliography styles which use \env{list} environments and print a label should set \len{labelsep} to \len{biblabelsep} in the definition of the respective environment.

\lenitem{bibitemsep}

文献表中各条目间的垂直间距。使用\env{list} 环境的参考文献样式应在环境定义中设置\len{itemsep} 为\len{bibitemsep}。
%The vertical space between the individual entries in the bibliography. Bibliography styles using \env{list} environments should set \len{itemsep} to \len{bibitemsep} in the definition of the respective environment.

\lenitem{bibparsep}

文献表中条目内段落间的垂直间距。使用\env{list} 环境的参考文献样式应在环境定义中设置\len{parsep} 为\len{bibparsep}。
%The vertical space between paragraphs within an entry in the bibliography. Bibliography styles using \env{list} environments should set \len{parsep} to \len{bibparsep} in the definition of the respective environment.

\cntitem{abbrvpenalty}

用于\cmd{addabbrvspace}, \cmd{addabthinspace} 和\cmd{adddotspace} 的阀值,详见\secref{aut:pct:spc} 节。
%The penalty used by \cmd{addabbrvspace}, \cmd{addabthinspace}, and \cmd{adddotspace}, see \secref{aut:pct:spc} for details.

\cntitem{lownamepenalty}

用于\cmd{addlowpenspace} 和\cmd{addlpthinspace} 的阀值,详见\secref{aut:pct:spc} 节。
%The penalty used by \cmd{addlowpenspace} and \cmd{addlpthinspace}, see \secref{aut:pct:spc} for details.

\cntitem{highnamepenalty}

用于\cmd{addhighpenspace} 和\cmd{addhpthinspace} 的阀值,详见\secref{aut:pct:spc} 节。
%The penalty used by \cmd{addhighpenspace} and \cmd{addhpthinspace}, see \secref{aut:pct:spc} for details.

\cntitem{biburlnumpenalty}

如果该计数器设置为大于0的值,\biblatex 将允许在以\sty{url} 包的\cmd{url} 命令格式化的所有字符串中允许数字后面的断行。这将影响文献表中的\acr{url}s和\acr{doi}s。断行点阀值将由该计数器的值确定。如果文献表中的\acr{url}s and/or \acr{doi}s超出到页边中,尽可能设置该计数器值大于0但小于10000(通常需要使用一个大值如9000)。设置该计数器为0将关闭该功能。这是默认设置。\footnote{译者: url、doi超出页边时往往需要用到。}
%If this counter is set to a value greater than zero, \biblatex will permit linebreaks after numbers in all strings formatted with the \cmd{url} command from the \sty{url} package. This will affect \acr{url}s and \acr{doi}s in the bibliography. The breakpoints will be penalized by the value of this counter. If \acr{url}s and/or \acr{doi}s in the bibliography run into the margin, try setting this counter to a value greater than zero but less than 10000 (you normally want to use a high value like 9000). Setting the counter to zero disables this feature. This is the default setting.

\cntitem{biburlucpenalty}

类似于\cnt{biburlnumpenalty},差别在于它将会在所有大写字母后面添加断点。
%Similar to \cnt{biburlnumpenalty}, except that it will add a breakpoint after all uppercase letters.

\cntitem{biburllcpenalty}

类似于\cnt{biburlnumpenalty},差别在于它将会在所有小写字母后面添加断点。
%Similar to \cnt{biburlnumpenalty}, except that it will add a breakpoint after all lowercase letters.

\end{ltxsyntax}

\subsubsection{辅助命令和钩子}%Auxiliary Commands and Hooks
\label{aut:fmt:ich}

本节的辅助命令和钩子具有特殊用途。从某种意义上说,其中一些用于\biblatex 与著录和标注样式之间的通信。
%The auxiliary commands and facilities in this section serve a special purpose. Some of them are used by \biblatex to communicate with bibliography and citation styles in some way or other.

\begin{ltxsyntax}

\cmditem{mkbibemph}{text}

通用命令将其参数打印为强调的文本内容。这是一个包围标准\cmd{emph} 命令的简单封套。除此之外,它使用\secref{aut:pct:new} 节的\cmd{setpunctfont} 来调整紧接在设为斜体的文本后的标点符号的字体。如果\opt{punctfont} 包选项未启用,该命令作用同\cmd{emph}。
%A generic command which prints its argument as emphasized text. This is a simple wrapper around the standard \cmd{emph} command. Apart from that, it uses \cmd{setpunctfont} from \secref{aut:pct:new} to adapt the font of the next punctuation mark following the text set in italics. If the \opt{punctfont} package option is disabled, this command behaves like \cmd{emph}.

\cmditem{mkbibitalic}{text}

类似于\cmd{mkbibemph} 的概念,但打印斜体文本。这是在一个标准\cmd{textit} 命令的简单封套,其中包含\cmd{setpunctfont} 命令。如果\opt{punctfont} 包选项未启用,该命令作用同\cmd{textit}。
%Similar in concept to \cmd{mkbibemph} but prints italicized text. This is a simple wrapper around the standard \cmd{textit} command which incorporates \cmd{setpunctfont}. If the \opt{punctfont} package option is disabled, this command behaves like \cmd{textit}.

\cmditem{mkbibbold}{text}

类似于\cmd{mkbibemph} 的概念,但打印斜体文本。这是在一个标准\cmd{textbf} 命令的简单封装,其中包含\cmd{setpunctfont} 命令。如果\opt{punctfont} 包选项未启用,该命令作用同\cmd{textbf}。
%Similar in concept to \cmd{mkbibemph} but prints bold text. This is a simple wrapper around the standard \cmd{textbf} command which incorporates \cmd{setpunctfont}. If the \opt{punctfont} package option is disabled, this command behaves like \cmd{textbf}.

\cmditem{mkbibquote}{text}

将其参数用引号包围起来的通用命令。如果加载了\sty{csquotes} 包,该命令使用该包提供的具体语言的引号。\cmd{mkbibquote} 也支持<American-style>的标点,详见\secref{aut:pct:cfg} 节的\cmd{DeclareQuotePunctuation} 命令。
%A generic command which wraps its argument in quotation marks. If the \sty{csquotes} package is loaded, this command uses the language sensitive quotation marks provided by that package. \cmd{mkbibquote} also supports <American-style> punctuation, see \cmd{DeclareQuotePunctuation} in \secref{aut:pct:cfg} for details.

\cmditem{mkbibparens}{text}

将其参数用圆括号包围起来的通用命令。该命令可以嵌套。当嵌套时,它将根据嵌套的层级交替使用圆括号和方括号。
%A generic command which wraps its argument in parentheses. This command is nestable. When nested, it will alternate between parentheses and brackets, depending on the nesting level.

\cmditem{mkbibbrackets}{text}

将其参数用方括号包围起来的通用命令。该命令可以嵌套。当嵌套时,它将根据嵌套的层级交替使用圆括号和方括号。
%A generic command which wraps its argument in square brackets. This command is nestable. When nested, it will alternate between brackets and parentheses, depending on the nesting level.

\cmditem{bibopenparen}<text>|{\ltxsyntaxlabelfont\cmd{bibcloseparen}}|

\cmd{mkbibparens} 命令的替代语法。这能跨编组使用。注意 \cmd{bibopenparen} 和 \cmd{bibcloseparen} 必须配套使用。
%Alternative syntax for \cmd{mkbibparens}. This will also work across groups. Note that \cmd{bibopenparen} and \cmd{bibcloseparen} must always be balanced.

\cmditem{bibopenbracket}<text>|{\ltxsyntaxlabelfont\cmd{bibclosebracket}}|

\cmd{mkbibbrackets} 命令的替代语法。这能跨编组使用。注意 \cmd{bibopenbracket} 和 \cmd{bibclosebracket} 必须配套使用。
%Alternative syntax for \cmd{mkbibbrackets}. This will also work across groups. Note that \cmd{bibopenbracket} and \cmd{bibclosebracket} must always be balanced.

\cmditem{mkbibfootnote}{text}

将其参数作为脚注的通用命令。它是标准\latex \cmd{footnote} 命令的封套,并能消除脚注标记前的多余空格,阻止嵌套脚注。默认情况下,\cmd{mkbibfootnote} 需要在脚注内容开始时大写并在结束时自动添加一个句号。可以重定义下面介绍的\cmd{bibfootnotewrapper} 宏来修改其作用。
%A generic command which prints its argument as a footnote. This is a wrapper around the standard \latex \cmd{footnote} command which removes spurious whitespace preceding the footnote mark and prevents nested footnotes. By default, \cmd{mkbibfootnote} requests capitalization at the beginning of the note and automatically adds a period at the end. You may change this behavior by redefining the \cmd{bibfootnotewrapper} macro introduced below.

\cmditem{mkbibfootnotetext}{text}

类似于\cmd{mkbibfootnote},但使用\cmd{footnotetext} 命令。
%Similar to \cmd{mkbibfootnote} but uses the \cmd{footnotetext} command.

\cmditem{mkbibendnote}{text}

类似于\cmd{mkbibfootnote} 的概念,但将其参数打印为尾注。\cmd{mkbibendnote} 能消除尾注标记前的多余空格,并阻止嵌套。它支持由\sty{endnotes} 包提供的\cmd{endnote} 命令和\sty{pagenote} 包和\sty{memoir} 类提供的\cmd{pagenote} 命令。如果两个命令都可用,
\cmd{endnote} 优先。如果没有可用的尾注命令,\cmd{mkbibendnote} 将报错并回退为\cmd{footnote}。默认情况下,\cmd{mkbibendnote} 需要在尾注内容开始时大写并在结束时自动添加一个句号。可以重定义下面介绍的\cmd{bibfootnotewrapper} 宏来修改其作用。
%Similar in concept to \cmd{mkbibfootnote} except that it prints its argument as an endnote. \cmd{mkbibendnote} removes spurious whitespace preceding the endnote mark and prevents nested notes. It supports the \cmd{endnote} command provided by the \sty{endnotes} package as well as the \cmd{pagenote} command provided by the \sty{pagenote} package and the \sty{memoir} class. If both commands are available, \cmd{endnote} takes precedence. If no endnote support is available, \cmd{mkbibendnote} issues an error and falls back to \cmd{footnote}. By default, \cmd{mkbibendnote} requests capitalization at the beginning of the note and automatically adds a period at the end. You may change this behavior by redefining the \cmd{bibendnotewrapper} macro introduced below.

\cmditem{mkbibendnotetext}{text}

类似于\cmd{mkbibendnote},但使用\cmd{endnotetext} 命令。请注意,对于这种写法,\sty{pagenote} 包和\sty{memoir} 都不提供相应的\cmd{pagenotetext} 命令。这种情况下,\cmd{mkbibendnote} 将报错并回退为\cmd{footnotetext}。
%Similar to \cmd{mkbibendnote} but uses the \cmd{endnotetext} command. Please note that as of this writing, neither the \sty{pagenote} package nor the \sty{memoir} class provide a corresponding \cmd{pagenotetext} command. In this case, \cmd{mkbibendnote} will issue an error and fall back to \cmd{footnotetext}.

\cmditem{bibfootnotewrapper}{text}

一个内部封套,将 \cmd{mkbibfootnote} 和 \cmd{mkbibfootnotetext} 命令的\prm{text} 参数包围起来。例如,\cmd{mkbibfootnote} 最终归结为:
%An inner wrapper which encloses the \prm{text} argument of \cmd{mkbibfootnote} and \cmd{mkbibfootnotetext}. For example, \cmd{mkbibfootnote} eventually boils down to this:

\begin{ltxexample}
\footnote{<<\bibfootnotewrapper{>>text<<}>>}
\end{ltxexample}
%
该封套确保注文内容开始时大写并在结束时自动添加一个句号,默认定义为:
%The wrapper ensures capitalization at the beginning of the note and adds a period at the end. The default definition is:

\begin{ltxexample}
\newcommand{\bibfootnotewrapper}[1]{<<\bibsentence>> #1<<\addperiod>>}
\end{ltxexample}
%
如果不想大写首字母或者在注文尾部添加句号,不修改\cmd{mkbibfootnote},而要重定义\cmd{bibfootnotewrapper}。
%If you don't want capitalization at the beginning or a period at the end of the note, do not modify \cmd{mkbibfootnote} but redefine \cmd{bibfootnotewrapper} instead.

\cmditem{bibendnotewrapper}{text}

类似于\cmd{bibfootnotewrapper} 的概念,但对应于\cmd{mkbibendnote} 和\cmd{mkbibendnotetext} 命令。
%Similar in concept to \cmd{bibfootnotewrapper} but related to the \cmd{mkbibendnote} and \cmd{mkbibendnotetext} commands.

\cmditem{mkbibsuperscript}{text}

一个将参数转换成上标的通用命令。它是标准\latex \cmd{textsuperscript} 命令的封套,并能消除多余空格,允许前面的单词使用连字符。
%A generic command which prints its argument as superscripted text. This is a simple wrapper around the standard \latex \cmd{textsuperscript} command which removes spurious whitespace and allows hyphenation of the preceding word.

\cmditem{mkbibmonth}{integer}

该命令根据其整数参数打印月份名。尽管该命令的输出与具体语言相关,但它的定义不是,因此在本地化模型中通常不重定义。
%This command takes an integer argument and prints it as a month name. Even though the output of this command is language specific, its definition is not, hence it is normally not redefined in localisation modules.

\cmditem{mkbibseason}{string}

该命令根据季节本地化字符串打印与包选项\opt{dateabbrev} 对应版本的字符串。尽管该命令的输出与具体语言相关,但其定义并非如此,因此一般情况下不用在本地化模型中重定义。
%This command takes a season localisation string and prints the version of the string corresponding to the setting of the \opt{dateabbrev} package option. Even though the output of this command is language specific, its definition is not, hence it is normally not redefined in localisation modules.

\cmditem{mkyearzeros}{integer}

该命令根据\opt{datezeros} 包选项(\secref{use:opt:pre:gen})设置移除或增添年份的前导零串。用于在日期格式化宏的定义中。
%This command strips leading zeros from a year or enforces them, depending on the \opt{datezeros} package option (\secref{use:opt:pre:gen}). It is intended for use in the definition of date formatting macros.

\cmditem{mkmonthzeros}{integer}

该命令根据\opt{datezeros} 包选项(\secref{use:opt:pre:gen})设置移除或增添月份的前导零串。用于在日期格式化宏的定义中。
%This command strips leading zeros from a month or enforces them, depending on the \opt{datezeros} package option (\secref{use:opt:pre:gen}). It is intended for use in the definition of date formatting macros.

\cmditem{mkdayzeros}{integer}

该命令根据\opt{datezeros} 包选项(\secref{use:opt:pre:gen})设置移除或增添日的前导零串。用于在日期格式化宏的定义中。
%This command strips leading zeros from a day or enforces them, depending on the \opt{datezeros} package option (\secref{use:opt:pre:gen}). It is intended for use in the definition of date formatting macros.

\cmditem{mktimezeros}{integer}

该命令根据\opt{timezeros} 包选项(\secref{use:opt:pre:gen})设置移除或增添时间的前导零串。用于在日期格式化宏的定义中。
%This command strips leading zeros from a number or preserves them, depending on the \opt{timezeros} package option (\secref{use:opt:pre:gen}). It is intended for use in the definition of time formatting macros.

\cmditem{forcezerosy}{integer}

该命令将零串添加到年份中(或者任何4位数的数字中)。用于日期格式化和序数中。
%This command adds zeros to a year (or any number supposed to be 4-digits). It is intended for date formatting and ordinals.

\cmditem{forcezerosmdt}{integer}

该命令将零串添加到月份、日或时间成分中(或者任何2位数的数字中)。用于日期/时间格式化和序数中。
%This command adds zeros to a month, day or time part (or any number supposed to be 2-digits). It is intended for date/time formatting and ordinals.

\cmditem{stripzeros}{integer}

该命令移除数字中的前导零串。用于日期格式化和序数中。
%This command strips leading zeros from a number. It is intended for date formatting and ordinals.

\optitem{$<$labelfield$>$width}

对于数据模型中任何标记为<Label field>的域,根据上述的\texttt{shorthandwidth} 自动创建一个格式化指令。因为默认的数据模型中\bibfield{shorthand} 就是如此标记的,所以该功能是\texttt{shorthandwidth} 功能的父集。
%For every field marked as a <Label field> in the data model, a formatting directive is created as per \texttt{shorthandwidth} above. Since \bibfield{shorthand} is so marked in the default data model, this functionality is a superset of that described for \texttt{shorthandwidth}.

\optitem{labelnumberwidth}

类似于\texttt{shorthandwidth},但指的是\bibfield{labelnumber} 域和长度\cmd{labelnumberwidth}。顺序编码样式应该调整该指令以便与参考文献表中应用的格式一致。
%Similar to \texttt{shorthandwidth}, but referring to the \bibfield{labelnumber} field and the length register \cmd{labelnumberwidth}. Numeric styles should adjust this directive such that it corresponds to the format used in the bibliography.

\optitem{labelalphawidth}

类似于\texttt{shorthandwidth},但指的是\bibfield{labelalpha} 域和长度\cmd{labelalphawidth}。字母顺序样式应该调整该指令以便与参考文献表中应用的格式一致。\footnote{译者:注意这个命令和上一个命令的差别,Alphabetic和Numeric样式的差别}
%Similar to \texttt{shorthandwidth}, but referring to the \bibfield{labelalpha} field and the length register \cmd{labelalphawidth}. Alphabetic styles should adjust this directive such that it corresponds to the format used in the bibliography.

\optitem{bibhyperref}

与\cmd{printfield} 和\cmd{printtext} 配合使用的一个特殊格式化指令。该指令将其参数包含在\cmd{bibhyperref} 命令中,详见\secref{aut:aux:msc}。
%A special formatting directive for use with \cmd{printfield} and \cmd{printtext}. This directive wraps its argument in a \cmd{bibhyperref} command, see \secref{aut:aux:msc} for details.

\optitem{bibhyperlink}

与\cmd{printfield} 和\cmd{printtext} 配合使用的一个特殊格式化指令。该指令将其参数包含在\cmd{bibhyperlink} 命令中,详见\secref{aut:aux:msc}。\prm{name} 传递给\cmd{bibhyperlink} 命令的是\bibfield{entrykey} 域的值。
%A special formatting directive for use with \cmd{printfield} and \cmd{printtext}. It wraps its argument in a \cmd{bibhyperlink} command, see \secref{aut:aux:msc} for details. The \prm{name} argument passed to \cmd{bibhyperlink} is the value of the \bibfield{entrykey} field.

\optitem{bibhypertarget}

与\cmd{printfield} 和\cmd{printtext} 配合使用的一个特殊格式化指令。该指令将其参数包含在\cmd{bibhypertarget} 命令中,详见\secref{aut:aux:msc}。\prm{name} 传递给\cmd{bibhypertarget} 命令的是\bibfield{entrykey} 域的值。
%A special formatting directive for use with \cmd{printfield} and \cmd{printtext}. It wraps its argument in a \cmd{bibhypertarget} command, see \secref{aut:aux:msc} for details. The \prm{name} argument passed to \cmd{bibhypertarget} is the value of the \bibfield{entrykey} field.

\optitem{volcitepages}

控制类似\cmd{volcite} 等标注命令参数中的页码或文本部分格式的一个特殊格式化指令。
%A special formatting directive which controls the format of the page\slash text portion in the argument of citation commands like \cmd{volcite}.

\optitem{volcitevolume}

控制类似\cmd{volcite} 等标注命令参数中的卷部分格式的一个特殊格式化指令。
%A special formatting directive which controls the format of the volume portion in the argument of citation commands like \cmd{volcite}.

\optitem{date}

控制\cmd{printdate} 格式的一个特殊格式化指令(\secref{aut:bib:dat})。注意,日期格式(long/short等) 由\secref{use:opt:pre:gen} 节的包选项\opt{date} 控制。该格式化指令仅控制如字体等额外的格式。
%A special formatting directive which controls the format of \cmd{printdate} (\secref{aut:bib:dat}). Note that the date format (long/short etc.) is controlled by the package option \opt{date} from \secref{use:opt:pre:gen}. This formatting directive only controls additional formatting such as fonts etc.

\optitem{labeldate}

类似于\texttt{date},当控制\cmd{printlabeldate} 的格式。
%As \texttt{date} but controls the format of \cmd{printlabeldate}.

\optitem{$<$datetype$>$date}

类似于\texttt{date},当控制\cmd{print$<$datetype$>$date} 的格式。
As \texttt{date} but controls the format of \cmd{print$<$datetype$>$date}.

\optitem{time}

控制\cmd{printtime} (\secref{aut:bib:dat})格式的一个特殊格式化指令。注意:时间格式(24h/12h 等)由\secref{use:opt:pre:gen} 节的包选项\opt{time} 控制。该格式化指令仅控制如字体等额外的格式。
%A special formatting directive which controls the format of \cmd{printtime} (\secref{aut:bib:dat}). Note that the time format (24h/12h etc.) is controlled by the package option \opt{time} from \secref{use:opt:pre:gen}. This formatting directive only controls additional formatting such as fonts etc.

\optitem{labeltime}

类似于\texttt{time},但控制\cmd{printlabeltime} 的格式。
%As \texttt{time} but controls the format of \cmd{printlabeltime}.

\optitem{$<$datetype$>$time}

类似于\texttt{time},但控制\cmd{print$<$datetype$>$time} 的格式。
As \texttt{time} but controls the format of \cmd{print$<$datetype$>$time}.

\end{ltxsyntax}

\subsubsection{辅助长度、计数器和其它功能}% Auxiliary Lengths, Counters, and Other Features
\label{aut:fmt:ilc}
这里讨论的长度和计数器用于在\biblatex 中项著录和标注样式传递信息。可以将它们认为是只读的(read"=only)。注意:所有的计数器都是\latex 计数器。使用|\value{counter}|来读取当前值。
%The length registers and counters discussed here are used by \biblatex to pass information to bibliography and citation styles. Think of them as read"=only registers. Note that all counters are \latex counters. Use |\value{counter}| to read out the current value.

\begin{ltxsyntax}

\lenitem{$<$labelfield$>$width}

对于数据模型中任何标记为<label>的域,根据上述的\texttt{shorthandwidth} 自动创建一个长度。因为\bibfield{shorthand} 在默认数据模型中就是如此标记的,所以该功能是\texttt{shorthandwidth} 描述功能的父集。
%For every field marked as a <label> field in the data model, a length register is created as per \texttt{shorthandwidth} above. Since \bibfield{shorthand} is so marked in the default data model, this functionality is a superset of that described for \texttt{shorthandwidth}.

\lenitem{labelnumberwidth}

表示最宽\bibfield{labelnumber} 的长度。顺序编码著录样式应在参考文献表环境的定义中考虑该长度。
%This length register indicates the width of the widest \bibfield{labelnumber}. Numeric bibliography styles
%should incorporate this length in the definition of the bibliography environment.

\lenitem{labelalphawidth}

表示最宽\bibfield{labelalpha} 的长度。字母顺序编码著录样式应在参考文献表环境的定义中考虑该长度。
%This length register indicates the width of the widest \bibfield{labelalpha}. Alphabetic bibliography styles should incorporate this length in the definition of the bibliography environment.

\cntitem{maxextraalpha}

该计数器保存在\bibfield{extraalpha} 域中能找到的最大数值。
%This counter holds the highest number found in any \bibfield{extraalpha} field.

\cntitem{maxextradate}

该计数器保存在\bibfield{extradate} 域中能找到的最大数值。
%This counter holds the highest number found in any \bibfield{extradate} field.

\cntitem{maxextraname}

%This counter holds the highest number found in any \bibfield{extraname} field.
该计数器保存任意\bibfield{extraname}域中保存的最大值。

\cntitem{maxextratitle}

%This counter holds the highest number found in any \bibfield{extratitle} field.
该计数器保存任意\bibfield{extratitle}域中保存的最大值。

\cntitem{maxextratitleyear}

%This counter holds the highest number found in any \bibfield{extratitleyear} field.
该计数器保存任意\bibfield{extratitleyear}域中保存的最大值。

\cntitem{refsection}

该计数器表示当前的\env{refsection} 环境。当在一个文献列表标题中请求时,该计数器返回传递给\cmd{printbibliography} 命令的\opt{refsection} 选项的值。
%This counter indicates the current \env{refsection} environment. When queried in a bibliography heading, the counter returns the value of the \opt{refsection} option passed to \cmd{printbibliography}.

\cntitem{refsegment}

该计数器表示当前的\env{refsegment} 环境。当在一个文献列表标题中请求时,该计数器返回传递给\cmd{printbibliography} 命令的\opt{refsegment} 选项的值。
%This counter indicates the current \env{refsegment} environment. When queried in a bibliography heading, this counter returns the value of the \opt{refsegment} option passed to \cmd{printbibliography}.

\cntitem{maxnames}

该计数器保存\opt{maxnames} 包选项的设置。
%This counter holds the setting of the \opt{maxnames} package option.

\cntitem{minnames}

该计数器保存\opt{minnames} 包选项的设置。
%This counter holds the setting of the \opt{minnames} package option.

\cntitem{maxitems}

该计数器保存\opt{maxitems} 包选项的设置。
%This counter holds the setting of the \opt{maxitems} package option.

\cntitem{minitems}

该计数器保存\opt{minitems} 包选项的设置。
%This counter holds the setting of the \opt{minitems} package option.

\cntitem{instcount}

该计数器由\biblatex 根据每个出现的引用自动增加,在文献列表中则根据出现的条目自动增加。该计数器的值唯一的确定文档中一篇文献对象。\footnote{译者:a single instance of a reference in the document}
%This counter is incremented by \biblatex for every citation as well as for every entry in the bibliography and bibliography lists. The value of this counter uniquely identifies a single instance of a reference in the document.

\cntitem{citetotal}

该计数器,仅在\cmd{DeclareCiteCommand} 定义的标注命令的\prm{loopcode} 中提供,用于保存传递给标注命令的有效条目关键词总数。
%This counter, which is only available in the \prm{loopcode} of a citation command defined with \cmd{DeclareCiteCommand}, holds the total number of valid entry keys passed to the citation command.

\cntitem{citecount}

该计数器,仅在\cmd{DeclareCiteCommand} 定义的标注命令的\prm{loopcode} 中提供,用于保存\prm{loopcode} 正在处理的条目的序号。
%This counter, which is only available in the \prm{loopcode} of a citation command defined with \cmd{DeclareCiteCommand}, holds the number of the entry key currently being processed by the \prm{loopcode}.

\cntitem{multicitetotal}

该命令类似于\cnt{citetotal},但仅在multicite类命令中提供。它保存传递给multicite类命令的标注命令总数。注意,其包含的各个标注命令可能包含多于一个条目关键词。这一信息由\cnt{citetotal} 计数器提供。
%This counter is similar to \cnt{citetotal} but only available in multicite commands. It holds the total number of citations passed to the multicite command. Note that each of these citations may consist of more than one entry key. This information is provided by the \cnt{citetotal} counter.

\cntitem{multicitecount}

该命令类似于\cnt{citecount},但仅在multicite类命令中提供。它保存正在处理的标注命令序号。注意,其包含的各个标注命令可能包含多于一个条目关键词。这一信息由\cnt{citetotal} 和\cnt{citecount} 计数器提供。
%This counter is similar to \cnt{citecount} but only available in multicite commands. It holds the number of the citation currently being processed. Note that this citation may consist of more than one entry key. This information is provided by the \cnt{citetotal} and \cnt{citecount} counters.

\cntitem{listtotal}

该计数器保存当前列表中项的总数。用于列表的格式化指令中,在其它任何地方使用时不保存一个有意义的数值。作为一个特例,它可能用在\cmd{printnames} 和\cmd{printlist} 命令的第二个参数中,详见\secref{aut:bib:dat}。对于每个列表,都有一个与其名称相同的计数器用于保存相应列表的项的总数。例如,\cnt{author} 计数器保存了\bibfield{author} 列表中的项的总数。无论姓名列表还是文本列表都是如此。这些计数器有点类似\cnt{listtotal},差别在于他们可以用于列表格式化命令外独立使用。例如,一个参考文献著录样式可能会检查\cnt{editor} 计数器来决定是否在编者列表之后打印术语«editor»或者其复数形式«editors»
%This counter holds the total number of items in the current list. It is intended for use in list formatting directives and does not hold a meaningful value when used anywhere else. As an exception, it may also be used in the second optional argument to \cmd{printnames} and \cmd{printlist}, see \secref{aut:bib:dat} for details. For every list, there is also a counter by the same name which holds the total number of items in the corresponding list. For example, the \cnt{author} counter holds the total number of items in the \bibfield{author} list. This applies to both name lists and literal lists. These counters are similar to \cnt{listtotal} except that they may also be used independently of list formatting directives. For example, a bibliography style might check the \cnt{editor} counter to decide Whether or not to print the term «editor» or rather its plural form «editors» after the list of editors.

\cntitem{listcount}

该计数器保存列表当前正在处理的项的序号。用于列表格式化指令,在其它任何地方使用无意义。
%This counter holds the number of the list item currently being processed. It is intended for use in list formatting directives and does not hold a meaningful value when used anywhere else.

\cntitem{liststart}

该计数器保存传递给\cmd{printnames} 和\cmd{printlist} 命令的\prm{start} 参数。用于列表格式化指令,在其它任何地方使用无意义。
%This counter holds the \prm{start} argument passed to \cmd{printnames} or \cmd{printlist}. It is intended for use in list formatting directives and does not hold a meaningful value when used anywhere else.

\cntitem{liststop}

该计数器保存传递给\cmd{printnames} 和\cmd{printlist} 命令的\prm{stop} 参数。用于列表格式化指令,在其它任何地方使用无意义。
%This counter holds the \prm{stop} argument passed to \cmd{printnames} or \cmd{printlist}. It is intended for use in list formatting directives and does not hold a meaningful value when used anywhere else.

\csitem{currentlang}

\biblatex 中当前活动语言的名称。可以用于任何地方,默认为文档主体语言。它能在定义\bibfield{langid} 的条目内部自动转换,如果\opt{autolang} 和\opt{language} 选项设置合适的话。注意,它不能追踪文档中所有的语言改变,仅用于当前的\biblatex\ 设置。
%The name of the currently active language for \biblatex. Can be used anywhere and
%defaults to the main document language. This is automatically switched
%inside entries which define \bibfield{langid}, given suitable settings of the
%\opt{autolang} and \opt{language} options. Note that this does not track
%all document language changes, only the current \biblatex\ setting.

\csitem{currentfield}

\cmd{printfield} 命令正在处理的域的名称。这一信息仅在局部的域格式化指令中提供。
%The name of the field currently being processed by \cmd{printfield}. This information is only available locally in field formatting directives.

\csitem{currentlist}

\cmd{printlist} 命令正在处理的文本列表的名称。这一信息仅在局部的域格式化指令中提供。
%The name of the literal list currently being processed by \cmd{printlist}. This information is only available locally in list formatting directives.

\csitem{currentname}

\cmd{printnames} 命令正在处理的姓名列表的名称。这一信息仅在局部的域格式化指令中提供。
%The name of the name list currently being processed by \cmd{printnames}. This information is only available locally in name formatting directives.

\end{ltxsyntax}

\subsubsection{多用途钩子}% General Purpose Hooks
\label{aut:fmt:hok}

\begin{ltxsyntax}

\cmditem{AtBeginBibliography}{code}

向在打印文献表开始时执行的内部钩子添加\prm{code}。\prm{code} 在引文列表开始处执行,紧跟在\cmd{defbibenvironment} 环境的\prm{begin code} 后面。该命令只能在导言区中使用。
%Appends the \prm{code} to an internal hook executed at the beginning of the bibliography. The \prm{code} is executed at the beginning of the list of references, immediately after the \prm{begin code} of \cmd{defbibenvironment}. This command may only be used in the preamble.

\cmditem{AtBeginShorthands}{code}

向在缩略列表开始时执行的内部钩子添加\prm{code}。\prm{code} 在缩略列表开始处执行,紧跟在\cmd{defbibenvironment} 环境的\prm{begin code} 后面。该命令只能在导言区中使用。
%Appends the \prm{code} to an internal hook executed at the beginning of the list of shorthands. The \prm{code} is executed at the beginning of the list of shorthands, immediately after the \prm{begin code} of \cmd{defbibenvironment}. This command may only be used in the preamble.

该命令也等价于:
%This is just an alias for:

\begin{ltxexample}
\AtBeginBiblist{shorthand}{code}
\end{ltxexample}

\cmditem{AtBeginBiblist}{biblistname}{code}

向在打印\prm{biblistname} 开始时执行的内部钩子添加\prm{code}。\prm{code} 在引文列表开始处执行,紧跟在\cmd{defbibenvironment} 环境的\prm{begin code} 后面。该命令只能在导言区中使用。
%Appends the \prm{code} to an internal hook executed at the beginning of the bibliography list \prm{biblistname}. The \prm{code} is executed at the beginning of the bibliography list, immediately after the \prm{begin code} of \cmd{defbibenvironment}. This command may only be used in the preamble.

\cmditem{AtEveryBibitem}{code}

向在文献表中打印各条目开始时执行的内部钩子添加\prm{code}。\prm{code} 紧跟在\cmd{defbibenvironment} 环境的\prm{item code} 后面。各条目的数据此时已经提供。该命令只能在导言区中使用。
%Appends the \prm{code} to an internal hook executed at the beginning of every item in the bibliography. The \prm{code} is executed immediately after the \prm{item code} of \cmd{defbibenvironment}. The bibliographic data of the respective entry is available at this point. This command may only be used in the preamble.

\cmditem{AtEveryLositem}{code}

向在缩略表中打印各项开始时执行的内部钩子添加\prm{code}。\prm{code} 紧跟在\cmd{defbibenvironment} 环境的\prm{item code} 后面。各条目的数据此时已经提供。该命令只能在导言区中使用。
%Appends the \prm{code} to an internal hook executed at the beginning of every item in the list of shorthands. The \prm{code} is executed immediately after the \prm{item code} of \cmd{defbibenvironment}. The bibliographic data of the respective entry is available at this point. This command may only be used in the preamble.

该命令也等价于:
%This is just an alias for:

\begin{ltxexample}
\AtEveryBiblistitem{shorthand}{code}
\end{ltxexample}

\cmditem{AtEveryBiblistitem}{biblistname}{code}

向在列表\prm{biblistname} 中打印各项开始时执行的内部钩子添加\prm{code}。\prm{code} 紧跟在\cmd{defbibenvironment} 环境的\prm{item code} 后面。各条目的数据此时已经提供。该命令只能在导言区中使用。
%Appends the \prm{code} to an internal hook executed at the beginning of every item in the bibliography list named \prm{biblistname}. The \prm{code} is executed immediately after the \prm{item code} of \cmd{defbibenvironment}. The bibliographic data of the respective entry is available at this point. This command may only be used in the preamble.

\cmditem{AtNextBibliography}{code}

类似于\cmd{AtBeginBibliography},但仅影响下一个\cmd{printbibliography}。一旦执行该命令,原内部钩子定义将被清除。该命令可以用于正文中。
%Similar to \cmd{AtBeginBibliography} but only affecting the next \cmd{printbibliography}. The internal hook is cleared after being executed once. This command may be used in the document body.

\cmditem{AtUsedriver}{code}
\cmditem*{AtUsedriver}*{code}

%Appends the \prm{code} to an internal hook executed when initializing \cmd{uisedriver}. The starred variant of the command clears the initialisation hook, so the defaults can be overwritten. This command may only be used in the preamble.
%The default setting is:
为初始化\cmd{uisedriver}时执行的一个内部钩子添加代码\prm{code}。带星号的命令会清除该钩子的初始化设置,所以它能覆盖默认的设置。该命令只能用于导言区中,默认的设置为:

\begin{ltxexample}
\AtUsedriver{%
  \let\finentry\blx@finentry@usedrv
  \let\newblock\relax
  \let\abx@macro@bibindex\@empty
  \let\abx@macro@pageref\@empty}
\end{ltxexample}

\cmditem{AtEveryCite}{code}

向在每个标注命令开始时执行的内部钩子添加\prm{code}。\prm{code} 在标注命令的\prm{precode} 前执行(见\secref{aut:cbx:cbx})。各条目的数据此时未提供。该命令只能在导言区中使用。
%Appends the \prm{code} to an internal hook executed at the beginning of every citation command. The \prm{code} is executed immediately before the \prm{precode} of the command (see \secref{aut:cbx:cbx}). No bibliographic data is available at this point. This command may only be used in the preamble.

\cmditem{AtEveryCitekey}{code}

向在把各条目关键词传递给一个标注命令时执行一次的内部钩子添加\prm{code}。\prm{code} 在标注命令的\prm{loopcode} 前执行(见\secref{aut:cbx:cbx})。各条目的数据此时已经提供。该命令只能在导言区中使用。
%Appends the \prm{code} to an internal hook executed once for every entry key passed to a citation command. The \prm{code} is executed immediately before the \prm{loopcode} of the command (see \secref{aut:cbx:cbx}). The bibliographic data of the respective entry is available at this point. This command may only be used in the preamble.

\cmditem{AtEveryMultiCite}{code}

向在每个multicite命令开始时执行的内部钩子添加\prm{code}。\prm{code} 在\bibfield{multiprenote} 域(见\secref{aut:cbx:fld})之前执行。各条目的数据此时未提供。该命令只能在导言区中使用。
%Appends the \prm{code} to an internal hook executed at the beginning of every multicite command. The \prm{code} is executed immediately before the \bibfield{multiprenote} field (\secref{aut:cbx:fld}) is printed. No bibliographic data is available at this point. This command may only be used in the preamble.

\cmditem{AtNextCite}{code}

类似于\cmd{AtEveryCite},但仅影响下一个标注命令。一旦执行该命令,原内部钩子定义将被清除。该命令可以用于正文中。
%Similar to \cmd{AtEveryCite} but only affecting the next citation command. The internal hook is cleared after being executed once. This command may be used in the document body.

\cmditem{AtEachCitekey}{code}

类似于\cmd{AtEveryCitekey},但仅影响当前标注命令。该命令可以用于正文中。当在一个标注中时,\prm{code} 添加到一个局部的内容部钩子中,可以用\cmd{ifcitation} 判断是否是在一个标注中。
%Similar to \cmd{AtEveryCitekey} but only affecting the current citation command. This command may be used in the document body. The \prm{code} is appended to the internal hook locally when located in a citation, as determined by \cmd{ifcitation}.

\cmditem{AtNextCitekey}{code}

类似于\cmd{AtEveryCitekey},但仅影响下一个条目关键词。一旦执行该命令,原内部钩子定义将被清除。该命令可以用于正文中。
%Similar to \cmd{AtEveryCitekey} but only affecting the next entry key. The internal hook is cleared after being executed once. This command may be used in the document body.

\cmditem{AtNextMultiCite}{code}

类似于\cmd{AtEveryMultiCite},但仅影响下一个multicite命令。一旦执行该命令,原内部钩子定义将被清除。该命令可以用于正文中。
%Similar to \cmd{AtEveryMultiCite} but only affecting the next multicite command. The internal hook is cleared after being executed once. This command may be used in the document body.

\cmditem{AtDataInput}[entrytype]{code}

向参考文献数据从\file{bbl} 文件导入后每个条目执行一次的内部钩子添加\prm{code}。\prm{entrytype} 是\prm{code} 应用的条目类型。如果要应用于所有条目类型,则忽略该可选参数。\prm{code} 在条目导入后立即执行。该命令只能用于导言中。注意:\prm{code} 对于一个条目可能被执行多次,这发生在当一个相同条目在不同的\env{refsection} 环境中引用或者\opt{sorting} 选项设置包含多余一个的排序格式时。当数据导入时,\cnt{refsection} 计数器保存各参考文献节的序号。
%Appends the \prm{code} to an internal hook executed once for every entry as the bibliographic data is imported from the \file{bbl} file. The \prm{entrytype} is the entry type the \prm{code} applies to. If it applies to all entry types, omit the optional argument. The \prm{code} is executed immediately after the entry has been imported. This command may only be used in the preamble. Note that \prm{code} may be executed multiple times for an entry. This occurs when the same entry is cited in different \env{refsection} environments or the \opt{sorting} option settings incorporate more than one sorting scheme. The \cnt{refsection} counter holds the number of the respective reference section while the data is imported.

\cmditem{UseBibitemHook}

执行对应\cmd{AtEveryBibitem} 的内部钩子。
%Executes the internal hook corresponding to \cmd{AtEveryBibitem}.

\cmditem{UseUsedriverHook}

执行对应\cmd{AtUsedriver} 的内部钩子。
%Executes the internal hook corresponding to \cmd{AtUsedriver}.

\cmditem{UseEveryCiteHook}

执行对应\cmd{AtEveryCite} 的内部钩子。
%Executes the internal hook corresponding to \cmd{AtEveryCite}.

\cmditem{UseEveryCitekeyHook}

执行对应\cmd{AtEveryCitekey} 的内部钩子。
%Executes the internal hook corresponding to \cmd{AtEveryCitekey}.

\cmditem{UseEveryMultiCiteHook}

执行对应\cmd{AtMultiEveryCite} 的内部钩子。
%Executes the internal hook corresponding to \cmd{AtMultiEveryCite}.

\cmditem{UseNextCiteHook}

执行对应\cmd{AtNextCite} 的内部钩子。
%Executes and clears the internal hook corresponding to \cmd{AtNextCite}.

\cmditem{UseNextCitekeyHook}

执行对应\cmd{AtNextCitekey} 的内部钩子。
%Executes and clears the internal hook corresponding to \cmd{AtNextCitekey}.

\cmditem{UseNextMultiCiteHook}

执行对应\cmd{AtNextMultiCite} 的内部钩子。
%Executes and clears the internal hook corresponding to \cmd{AtNextMultiCite}.

\cmditem{DeferNextCitekeyHook}

局部地取消由\cmd{AtNextCitekey} 设定的内部钩子。其本质是当钩子在\cmd{DeclareCiteCommand} (见\secref{aut:cbx:cbx})的\prm{precode} 参数中执行时,将该钩子延迟到标注列表中的下一个条目关键词。
%Locally un-defines the internal hook specified by \cmd{AtNextCitekey}. This essentially defers the hook to the next entry key in the citation list, when executed in the \prm{precode} argument of \cmd{DeclareCiteCommand} (\secref{aut:cbx:cbx}).

\end{ltxsyntax}

\subsection{提示与警告}%Hints and Caveats
\label{aut:cav}

本节提供了关于biblatex宏包接口的一些附加提示,也将论述一些普遍性的问题和容易误解的概念。
%This section provides some additional hints concerning the author interface of this package. It also addresses common problems and potential misconceptions.

\subsubsection{条目集}%Entry Sets
\label{aut:cav:set}

条目集已经在\secref{use:use:set} 节介绍过,本节主要讨论怎么在著录样式中处理条目集。从驱动的角度看,静态和动态的条目集并无差别。两者都以相同方式处理。只需要使用\secref{aut:bib:dat} 的\cmd{entryset} 命令遍历集的所有成员(以在\bibtype{set} 条目的\bibfield{entryset} 域中的出现的顺序,或者它们传递给\cmd{defbibentryset} 命令的顺序进行遍历),并在最后加上\cmd{finentry} 命令即可。格式化则由集的成员各自条目类型的驱动控制。
%Entry sets have already been introduced in \secref{use:use:set}. This section discusses how to process entry sets in a bibliography style. From the perspective of the driver, there is no difference between static and dynamic entry sets. Both types are handled in the same way. You will normally use the \cmd{entryset} command from \secref{aut:bib:dat} to loop over all set members (in the order in which they are listed in the \bibfield{entryset} field of the \bibtype{set} entry, or in the order in which they were passed to \cmd{defbibentryset}, respectively) and append \cmd{finentry} at the end. That's it. The formatting is handled by the drivers for the entry types of the individual set members:

\begin{ltxexample}
\DeclareBibliographyDriver{set}{%
  <<\entryset>>{}{}%
  \finentry}
\end{ltxexample}
%
需要注意: 本宏包附带的\texttt{numeric} 样式支持条目集细分,即集成员以一个字母或者其他记号来标记,标注命令可以引用整个集或者其中的某一具体成员。记号由样式文件以如下方式生成:
%You may have noticed that the \texttt{numeric} styles which ship with this package support subdivided entry sets, \ie the members of the set are marked with a letter or some other marker such that citations may either refer to the entire set or to a specific set member. The markers are generated as follows by the bibliography style:

\begin{ltxexample}
\DeclareBibliographyDriver{set}{%
  \entryset
    {<<\printfield{entrysetcount}>>%
     <<\setunit*{\addnbspace}>>}
    {}%
  \finentry}
\end{ltxexample}
%
\bibfield{entrysetcount} 域保存了一个整数用于指示集成员在整个集中的位置。数字是转换为一个字母还是其他记号由域格式\bibfield{entrysetcount} 控制。所有驱动需要做的是打印域和一些空格(或者换行)。在标注中打印记号的方式类似。当顺序编码制样式给出|\printfield{labelnumber}|时,可以简单地加上\bibfield{entrysetcount} 域。
%The \bibfield{entrysetcount} field holds an integer indicating the position of a set member in the entry set. The conversion of this number to a letter or some other marker is handled by the formatting directive of the \bibfield{entrysetcount} field. All the driver needs to do is print the field and add some white space (or start a new line). Printing the markers in citations works in a similar way. Where a numeric style normally says |\printfield{labelnumber}|, you simply append the \bibfield{entrysetcount} field:

\begin{ltxexample}
\printfield{labelnumber}<<\printfield{entrysetcount}>>
\end{ltxexample}
%
因为该域仅在处理标注指向一个集成员时定义,所以没有必要添加任何更多的判断。
%Since this field is only defined when processing citations referring to a set member, there is no need to add any additional tests.

%Citing entry sets directly requires that a meaningful way of identifying sets is available in the style. This is obvious for styles based on numeric or alphabetic labels but not obvious (and rarely required) in styles which construct citations based on textual names/titles/dates etc. The default provided styles which no not construct citations based on labels (\texttt{authoryear}, \texttt{authortitle}, \texttt{verbose} etc.) therefore do not support citing sets directly as there is no obvious default identifier to use in such cases and such styles rarely, if ever, employ sets anyway. Custom styles may of course choose to define and print a citation identifier for directly cited sets.

引用一个条目集,需要样式提供一个有效识别的条目集的方式。
这在基于数字顺序和字母顺序编码的样式中是显然的,但在基于文本比如names/titles/dates等的标签的样式中通常是无法显著区分的,或者极少需要的。
在不基于标签(label)构造标注的样式(\texttt{authoryear}, \texttt{authortitle}, \texttt{verbose} etc.)中,通常不支持直接的条目集引用,
因为没有一个默认的显著的标识符来区分,且这类样式记号以这种方式使用。
定制样式当然可以定义或打印一个标注标识符用于直接引用。



\subsubsection{电子出版信息}%Electronic Publishing Information
\label{aut:cav:epr}
标准样式主要支持arXiv网站的文献\footnote{译者:arXiv 原先是由物理学家保罗·金斯巴格在1991年建立的网站,本意在收集物理学的论文预印本,随后括及天文、数学等其它领域。金斯巴格因为这个网站获得了2002年的麦克阿瑟奖。arXiv 原先挂在洛斯阿拉莫斯国家实验室,是故早期被称为「LANL 预印本数据库」。目前的 arXiv 落脚于康乃尔大学,并在全球各地设有镜像站点。网站在1999年改名为 arXiv.org 。}。其它资源的支持很容易增加。标准样式以如下方式处理\bibfield{eprint} 域:
%The standard styles feature dedicated support for arXiv references. Support for other resources is easily added. The standard styles handle the \bibfield{eprint} field as follows:

\begin{ltxexample}
\iffieldundef{eprinttype}
  {\printfield{eprint}}
  {\printfield[<<eprint:\strfield{eprinttype}>>]{eprint}}
\end{ltxexample}
%
如果\bibfield{eprinttype} 域存在,上述代码将使用域格式\texttt{eprint:\prm{eprinttype}}。如果该格式未定义,\cmd{printfield} 自动退回到使用域格式\texttt{eprint}。有两种预定义的域格式,具体类型的域格式\texttt{eprint:arxiv} 和通用域格式\texttt{eprint}。
%If an \bibfield{eprinttype} field is available, the above code tries to use the field format \texttt{eprint:\prm{eprinttype}}. If this format is undefined, \cmd{printfield} automatically falls back to the field format \texttt{eprint}. There are two predefined field formats, the type"=specific format \texttt{eprint:arxiv} and the fallback format \texttt{eprint}:

\begin{ltxexample}
\DeclareFieldFormat{<<eprint>>}{...}
\DeclareFieldFormat{<<eprint:arxiv>>}{...}
\end{ltxexample}
%
换句话说,增加其他数据源的支持只需要定义一个名为\texttt{eprint:\prm{resource}} 的与格式,其中\prm{resource} 是在\bibfield{eprinttype} 域中使用的标识。
%In other words, adding support for additional resources is as easy as defining a field format named \texttt{eprint:\prm{resource}} where \prm{resource} is an identifier to be used in the \bibfield{eprinttype} field.

\subsubsection{外部摘要和注解}% External Abstracts and Annotations
\label{aut:cav:prf}

外部摘要和注解已经在\secref{use:use:prf} 节讨论过,本节为样式作者提供更多的背景知识。标准样式使用如下的宏(来自 \path{biblatex.def})来处理摘要和注解:
%External abstracts and annotations have been discussed in \secref{use:use:prf}. This section provides some more background for style authors. The standard styles use the following macros (from \path{biblatex.def}) to handle abstracts and annotations:

\begin{ltxexample}
\newbibmacro*{annotation}{%
  \iffieldundef{annotation}
    {\printfile[annotation]{<<\bibannotationprefix\thefield{entrykey}.tex>>}}%
    {\printfield{annotation}}}
\newcommand*{<<\bibannotationprefix>>}{bibannotation-}

\newbibmacro*{abstract}{%
  \iffieldundef{abstract}
    {\printfile[abstract]{<<\bibabstractprefix\thefield{entrykey}.tex>>}}%
    {\printfield{abstract}}}
\newcommand*{<<\bibabstractprefix>>}{bibabstract-}
\end{ltxexample}
%
如果\bibfield{abstract}\slash \bibfield{annotation} 域未定义,上述代码将从外部文件中加载摘要和注解。\cmd{printfile} 将根据用户定义的前缀来搜索文件名。注意:必须显式地设置\secref{use:opt:pre:gen} 节的\opt{loadfiles} 包选项来启用\cmd{printfile} 功能。基于性能原因该功能默认是关闭的。
%If the \bibfield{abstract}\slash \bibfield{annotation} field is undefined, the above code tries to load the abstracts\slash annotations from an external file. The \cmd{printfile} commands also incorporate file name prefixes which may be redefined by users. Note that you must enable \cmd{printfile} explicitly by setting the \opt{loadfiles} package option from \secref{use:opt:pre:gen}. This feature is disabled by default for performance reasons.

\subsubsection[消除姓名歧义]{消除姓名歧义}%Name Disambiguation
\label{aut:cav:amb}
在\secref{use:opt:pre:int} 节引入的\opt{uniquename} 和\opt{uniquelist} 选项支持多种操作模式。本节用举例方式介绍不同模式的差别。\opt{uniquename} 选项消除\bibfield{labelname} 列表中各姓名间的歧义,而\opt{uniquelist} 则消除因\opt{maxnames}\slash \opt{minnames} 截短导致的\bibfield{labelname} 列表歧义。两个选项可以单独使用也可以联合使用:
%The \opt{uniquename} and \opt{uniquelist} options introduced in \secref{use:opt:pre:int} support various modes of operation. This section explains the differences between these modes by way of example. The \opt{uniquename} option disambiguates individual names in the \bibfield{labelname} list. The \opt{uniquelist} option disambiguates the \bibfield{labelname} list if it has become ambiguous after \opt{maxnames}\slash \opt{minnames} truncation. You can use either option stand-alone or combine both.

消除姓名歧义原理是根据由一个或多个姓名成分构成的<base>,来确定需要在其基础上添加什么(如果存在的话),使得姓名在当前参考文献节中是唯一的。消除姓名歧义由如下命令声明的 uniquename 模板控制:
%Name disambiguation works by taking a <base> which is composed of one or more nameparts and then determining what needs to be added, if anything, to this <base> to make the name unique in the current refsection. Name disambiguation is controlled by the uniquename template declared with the following command:

\begin{ltxsyntax}

\cmditem{DeclareUniquenameTemplate}[name]{specification}

Defines the \opt{uniquename} template \prm{name}. The \prm{name} is optional and defaults to \prm{<global>}.
\prm{specification} 是\cmd{namepart} 命令的一个有序列表,定义了确定 uniquename 信息所使用的姓名成分。
%The \prm{specification} is an ordered list of \cmd{namepart} commands which define the nameparts to use in determining the uniquename information

\cmditem{namepart}[options]{namepart}

\prm{namepart} 是数据模型中的姓名成分,由\cmd{DeclareDatamodelConstant} 命令定义(见\secref{aut:bbx:drv})。选项包括:
%\prm{namepart} is one of the datamodel nameparts defined with the \cmd{DeclareDatamodelConstant} command (see \secref{aut:bbx:drv}). The \opt{options} are:

\begin{optionlist*}

\boolitem[false]{use}

在构建 uniquename 信息中仅使用\prm{namepart},如果存在相应的选项\opt{use<namepart>} 并且值为true。
%Only use the \prm{namepart} in constructing the uniquename information if there is a corresponding option \opt{use<namepart>} and that option is true.

\boolitem[false]{base}


%The \prm{namepart} is part of the <base> which is the main piece of namepart(s) information which is being disambiguated by uniqueness information. For example, a family name which may be disambiguated by further given names. <base> \prm{namepart}s must occur before any non-<base> \prm{nameparts}.

\prm{namepart} 是<base>的部分,<base>是用作唯一性区分的namepart(s)信息的主段。

\choitem{disambiguation}{none, init, initorfull, full}

The \prm{namepart} will be disambiguated at most by information at the given value. If this option is not present then the default is inferred from the \opt{uniquename} package option setting (see \secref{use:opt:wu}). The <disambiguation> option is ignored for \prm{namepart}s which have the <base> option set to <true> since it is these nameparts which are being disambiguated by the value of the non-base \prm{namepart}s and therefore <disambiguation> does not apply.

\begin{description}
\item[none]~Do not use the \prm{namepart} to perform any name disambiguation
\item[init]~Use only the initials of the \prm{namepart} to perform name disambiguation
\item[initorfull]~Use initials and if necessary the full \prm{namepart} to perform name disambiguation
\item[full]~Use only the full \prm{namepart} to perform name
  disambiguation even if initials would suffice
\end{description}

\end{optionlist*}

\end{ltxsyntax}
%
默认的uniquename模板是:
%The default uniquename template is:

\begin{ltxexample}
\DeclareUniquenameTemplate{
  \namepart[use=true, base=true]{prefix}
  \namepart[base=true]{family}
  \namepart{given}
}
\end{ltxexample}
%
这意味着要区分的<base>由姓(<family>)和前缀(如果\opt{useprefix} 选项是true)构成。消除歧义主要通过增加模板中任何非<base>姓名成分来实现,这里就是名(<given>)成分。
%This means that the <base> to be disambiguated consists of the <family> namepart, along with any prefix, if the \opt{useprefix} option is true. The disambiguation is performed by adding aspects of any non <base> nameparts in the specification, here just the <given> namepart.

\paragraph{单个姓名(姓名间的区分)(\opt{uniquename})}%Individual Names (\opt{uniquename})

%Let's start off with some \opt{uniquename} examples. Consider the following data:
下面从一些\opt{uniquename} 示例开始,考虑如下数据:

\begin{lstlisting}{}
John Doe   2008
Edward Doe 2008
John Smith 2008
Jane Smith 2008
\end{lstlisting}
%
%Let's assume we're using an author-year style and set \kvopt{uniquename}{false}. In this case, we would get the following citations:
假设我们使用作者年制且设置\kvopt{uniquename}{false},这种情况下,我们得到如下引用标注:
\begin{lstlisting}{}
Doe 2008a
Doe 2008b
Smith 2008a
Smith 2008b
\end{lstlisting}
%
%Since the family names are ambiguous and all works have been published in the same year, an extra letter is appended to the year to disambiguate the citations. Many style guides, however, mandate that the extra letter be used to disambiguate works by the same authors only, not works by different authors with the same family name. In order to disambiguate the author's family name, you are expected to add additional parts of the name, either as initials or in full. This requirement is addressed by the \opt{uniquename} option. Here are the same citations with \kvopt{uniquename}{init}:
因为姓有歧义,且所有的年都相同,所以年后附加的字符用来区分并消除歧义。然而,很多样式指南强制要求附加字符只能用于相同作者的区分,而不能用于作者相同的姓的区分。为了消除作者姓的歧义,需要增加姓名的其它完整部分或者缩写来区分。这一需要由\opt{uniquename} 选项处理,下面是使用了\kvopt{uniquename}{init} 的引用标注:
\begin{lstlisting}{}
J. Doe 2008
E. Doe 2008
Smith 2008a
Smith 2008b
\end{lstlisting}
%
%\kvopt{uniquename}{init} restricts name disambiguation to initials. Since <J. Smith> would still be ambiguous, no additional name parts are added for the <Smiths>. With \kvopt{uniquename}{full}, names are printed in full where required:
\kvopt{uniquename}{init} 限制了用缩写来区分姓名。但因为<J. Smith>仍然有歧义,所以没有增加。而使用\kvopt{uniquename}{full},标注如下:
\begin{lstlisting}{}
J. Doe 2008
E. Doe 2008
John Smith 2008
Jane Smith 2008
\end{lstlisting}
%
%In order to illustrate the difference between \kvopt{uniquename}{init\slash full} and \texttt{allinit\slash allfull}, we need to introduce the notion of a <visible> name. In the following, <visible> names are all names at a position before the \opt{maxnames}\slash \opt{minnames}\slash \opt{uniquelist} truncation point. For example, given this data:
为了说明\kvopt{uniquename}{init\slash full} 和\texttt{allinit\slash allfull} 的差别,我们下面介绍 <visible> 姓名的概念。<visible> 姓名是位于\opt{maxnames}\slash \opt{minnames}\slash \opt{uniquelist} 截短点前的姓名,比如,给出数据:
\begin{lstlisting}{}
William Jones/Edward Doe/Jane Smith
John Doe
John Smith
\end{lstlisting}
%
%and \kvopt{maxnames}{1}, \kvopt{minnames}{1}, \kvopt{uniquename}{init/full}, we would get the following names in citations:
当\kvopt{maxnames}{1}, \kvopt{minnames}{1}, \kvopt{uniquename}{init/full} 时,我们得到如下的引用标注:
\begin{lstlisting}{}
Jones et al.
Doe
Smith
\end{lstlisting}
%
%When disambiguating names, \kvopt{uniquename}{init/full} only consider the visible names. Since all visible last names are distinct in this example, no further name parts are added. Let's compare that to the output of \kvopt{uniquename}{allinit}:
在消除歧义的时候,\kvopt{uniquename}{init/full} 仅考虑可见的姓名。因为本例中所有的可见姓名的姓都是不同的,所有没有姓名的其他部分附加进来。比较一下使用\kvopt{uniquename}{allinit} 的输出:
\begin{lstlisting}{}
Jones et al.
J. Doe
Smith
\end{lstlisting}
%
%\texttt{allinit} considers all names in all \bibfield{labelname} lists, including those which are hidden and replaced by <et al.> as the list is truncated. In this example, <John Doe> is disambiguated from <Edward Doe>. Since the ambiguity of the two <Smiths> can't be resolved by adding initials, no initials are added in this case. Now let's compare that to the output of \kvopt{uniquename}{allfull} which also disambiguates <John Smith> from <Jane Smith>:
\texttt{allinit} 认为所有在\bibfield{labelname} 列表中的姓名,包括列表截短后已经隐藏并且由<et al.>代替的姓名。在本例中,<John Doe> 与 <Edward Doe>存在歧义。因为两个 <Smiths> 无法通过添加缩写的方式区分,所以没有添加。现在来比较一下\kvopt{uniquename}{allfull} 的输出:
\begin{lstlisting}{}
Jones et al.
J. Doe
John Smith
\end{lstlisting}
%
%The options \kvopt{uniquename}{mininit/minfull} are similar to \texttt{init\slash full} in that they only consider visible names, but they perform minimal disambiguation. That is, they will disambiguate individual names only if they occur in identical lists of last names. Consider the following data:
\kvopt{uniquename}{mininit/minfull} 选项类似于\texttt{init\slash full} 仅考虑可见姓名,但仅执行最小的歧义消除。即,仅对姓列表的歧义进行处理,考虑如下数据:
\begin{lstlisting}{}
John Doe/William Jones
Edward Doe/William Jones
John Smith/William Edwards
Edward Smith/Allan Johnson
\end{lstlisting}
%
%With \kvopt{uniquename}{init/full}, we would get:
使用\kvopt{uniquename}{init/full},得到:
\begin{lstlisting}{}
J. Doe and Jones
E. Doe and Jones
J. Smith and Edwards
E. Smith and Johnson
\end{lstlisting}
%
%With \kvopt{uniquename}{mininit/minfull}:
使用\kvopt{uniquename}{mininit/minfull},得到:
\begin{lstlisting}{}
J. Doe and Jones
E. Doe and Jones
Smith and Edwards
Smith and Johnson
\end{lstlisting}
%
%The <Smiths> are not disambiguated because the visible name lists are not ambiguous and the \opt{mininit/minfull} options serve to disambiguate names occurring in identical last name lists only. Another way of looking at this is that they globally disambiguate family name lists. When it comes to ambiguous lists, note that a truncated list is considered to be distinct from an untruncated one even if the visible names are identical. For example, consider the following data:
<Smiths> 并无歧义,因为姓名列表时没有歧义。\opt{mininit/minfull} 选项仅对姓的列表相同情况进行处理。全局的看姓的列表,注意当未截短的列表的可见名相同的时候,截短的列表时也可能是不同的,比如下面的数据:
\begin{lstlisting}{}
John Doe/William Jones
Edward Doe
\end{lstlisting}
%
%With \kvopt{maxnames}{1}, \kvopt{uniquename}{init/full}, we would get:
使用\kvopt{maxnames}{1}, \kvopt{uniquename}{init/full}:

\begin{lstlisting}{}
J. Doe et al.
E. Doe
\end{lstlisting}
%
%With \kvopt{uniquename}{mininit/minfull}:
使用\kvopt{uniquename}{mininit/minfull}:

\begin{lstlisting}{}
Doe et al.
Doe
\end{lstlisting}
%
%Because the lists differ in the <et al.>, the names are not disambiguated.
因为列表有 <et al.> 的不同,姓名列表就不歧义。

\paragraph{姓名列表(列表间的区分) (\opt{uniquelist})}%Lists of Names (\opt{uniquelist})

%Ambiguity is also an issue with name lists. If the \bibfield{labelname} list is truncated by the \opt{maxnames}\slash \opt{minnames} options, it may become ambiguous. This type of ambiguity is addressed by the \opt{uniquelist} option. Consider the following data:

姓名列表也可能存在歧义问题。如果\bibfield{labelname} 列表由\opt{maxnames}\slash \opt{minnames} 选项截短就可能产生歧义。这类问题由\opt{uniquelist} 选处理,考虑如下数据:
\begin{lstlisting}{}
Doe/Jones/Smith   2005
Smith/Johnson/Doe 2005
Smith/Doe/Edwards 2005
Smith/Doe/Jones   2005
\end{lstlisting}
%
%Many author-year styles truncate long author/editor lists in citations. For example, with \kvopt{maxnames}{1} we would get:

很多作者年制样式需要在标注中截短,比如使用\kvopt{maxnames}{1} 选项,得到:
\begin{lstlisting}{}
Doe et al. 2005
Smith et al. 2005a
Smith et al. 2005b
Smith et al. 2005c
\end{lstlisting}
%
%Since the authors are ambiguous after truncation, the extra letter is added to the year to ensure unique citations. Here again, many style guides mandate that the extra letter be used to disambiguate works by the same authors only. In order to disambiguate author lists, you are usually required to add more names, exceeding the \opt{maxnames}\slash \opt{minnames} truncation point. The \opt{uniquelist} feature addresses this requirement. With \kvopt{uniquelist}{true}, we would get:
因为截短后作者存在歧义,所以添加额外字符确保引用标注的唯一性。同样的,一些样式强制要求额外字符只能用于所有作者都相同的情况。为了区分作者列表,必须增加更多的姓名,这样就会超出\opt{maxnames}\slash \opt{minnames} 选项设定的截短点。\opt{uniquelist} 选项即描述这一需求,当\kvopt{uniquelist}{true},有:

\begin{lstlisting}{}
Doe et al. 2005
Smith, Johnson et al. 2005
Smith, Doe and Edwards 2005
Smith, Doe and Jones 2005
\end{lstlisting}
%
%The \opt{uniquelist} option overrides \opt{maxnames}\slash \opt{minnames} on a per-entry basis. Essentially, what happens is that the <et al.> part of the citation is expanded to the point of no ambiguity~-- but no further than that. \opt{uniquelist} may also be combined with \opt{uniquename}. Consider the following data:
\opt{uniquelist} 选项以条目为限重设\opt{maxnames}\slash \opt{minnames}。大体上,标注的 <et al.> 部分扩展到无歧义的点--而且也基本到此为止。\opt{uniquelist} 也可以与\opt{uniquename} 联合使用,考虑如下数据:

\begin{lstlisting}{}
John Doe/Allan Johnson/William Jones  2009
John Doe/Edward Johnson/William Jones 2009
John Doe/Jane Smith/William Jones     2009
John Doe/John Smith/William Jones     2009
John Doe/John Edwards/William Jones   2009
John Doe/John Edwards/Jack Johnson    2009
\end{lstlisting}
%
%With \kvopt{maxnames}{1}:
使用\kvopt{maxnames}{1},得到:

\begin{lstlisting}{}
Doe et al. 2009a
Doe et al. 2009b
Doe et al. 2009c
Doe et al. 2009d
Doe et al. 2009e
Doe et al. 2009f
\end{lstlisting}
%
%With \kvopt{maxnames}{1}, \kvopt{uniquename}{full}, \kvopt{uniquelist}{true}:
使用\kvopt{maxnames}{1}, \kvopt{uniquename}{full}, \kvopt{uniquelist}{true} 则有:

\begin{lstlisting}{}
Doe, A. Johnson et al. 2009
Doe, E. Johnson et al. 2009
Doe, Jane Smith et al. 2009
Doe, John Smith et al. 2009
Doe, Edwards and Jones 2009
Doe, Edwards and Johnson 2009
\end{lstlisting}
%
%With \kvopt{uniquelist}{minyear}, list disambiguation only happens if the visible list is identical to another visible list with the same \bibfield{labelyear}. This is useful for author-year styles which only require that the citation as a whole be unique, but do not guarantee unambiguous authorship information in citations. This mode is conceptually related to \kvopt{uniquename}{mininit/minfull}. Consider this example:

使用\kvopt{uniquelist}{minyear},消除列表歧义仅在可见列表和\bibfield{labelyear} 相同的时候。这对于仅仅需要整个标注整体具有唯一性的作者年制样式是很有用的,但是不保证作者姓名的非歧义性。这一模式概念上域\kvopt{uniquename}{mininit/minfull} 选项相关。考虑如下示例:
\begin{lstlisting}{}
Smith/Jones   2000
Smith/Johnson 2001
\end{lstlisting}
%
%With \kvopt{maxnames}{1} and \kvopt{uniquelist}{true}, we would get:
使用\kvopt{maxnames}{1} 和\kvopt{uniquelist}{true},得到:

\begin{lstlisting}{}
Smith and Jones 2000
Smith and Johnson 2001
\end{lstlisting}
%
%With \kvopt{uniquelist}{minyear}:
使用\kvopt{uniquelist}{minyear},则得到:

\begin{lstlisting}{}
Smith et al. 2000
Smith et al. 2001
\end{lstlisting}
%
%With \kvopt{uniquelist}{minyear}, it is not clear that the authors are different for the two works but the citations as a whole are still unambiguous since the year is different. In contrast to that, \kvopt{uniquelist}{true} disambiguates the authorship even if this information is not required to uniquely locate the works in the bibliography. Let's consider another example:

使用\kvopt{uniquelist}{minyear},两个文献的作者是否相同并不清楚,但标注的整体是非歧义的,因为年份的不同。与此相反,\kvopt{uniquelist}{true} 需要消除作者列表的歧义即便这一信息对于参考文献表的唯一引用是不必要的,看看如下示例:
\begin{lstlisting}{}
Vogel/Beast/Garble/Rook  2000
Vogel/Beast/Tremble/Bite 2000
Vogel/Beast/Acid/Squeeze 2001
\end{lstlisting}
%
%With \kvopt{maxnames}{3}, \kvopt{minnames}{1}, \kvopt{uniquelist}{true}, we would get:
使用\kvopt{maxnames}{3}, \kvopt{minnames}{1}, \kvopt{uniquelist}{true},得到

\begin{lstlisting}{}
Vogel, Beast, Garble et al. 2000
Vogel, Beast, Tremble et al. 2000
Vogel, Beast, Acid et al. 2001
\end{lstlisting}
%
%With \kvopt{uniquelist}{minyear}:
使用\kvopt{uniquelist}{minyear} 选项,则有:

\begin{lstlisting}{}
Vogel, Beast, Garble et al. 2000
Vogel, Beast, Tremble et al. 2000
Vogel et al. 2001
\end{lstlisting}
%
%In the last citation, \kvopt{uniquelist}{minyear} does not override \opt{maxnames}\slash \opt{minnames} as the citation does not need disambiguating from the other two because the year is different.
在最后一个引用中,\kvopt{uniquelist}{minyear} 不重写\opt{maxnames}\slash \opt{minnames},因为年份的不同,所以不需要消除与其它两个间的歧义。

\subsubsection{浮动体和\acr{TOC}/\acr{LOT}/\acr{LOF} 中的追踪器}%Trackers in Floats and \acr{TOC}/\acr{LOT}/\acr{LOF}
\label{aut:cav:flt}

当引用命令出现在浮动体(比如图和表的题注)中,因为浮动体无论是物理上还是逻辑上都在文本流之外,这会导致的学术反向引用(比如<ibidem>,意为“出处同上”)和基于页码追踪器的反向引用难以区分,因此这种引用的逻辑很难在其中应用。为避免这种问题,标注(引用)和页码追踪器在所有的浮动体中临时关闭。并且,这些追踪器加上反向引用追踪器(\opt{backref})在目录,图和表目录中也临时关闭。
%If a citation is given in a float (typically in the caption of a figure or table), scholarly back references like <ibidem> or back references based on the page tracker get ambiguous because floats are objects which are (physically and logically) placed outside the flow of text, hence the logic of such references applies poorly to them. To avoid any such ambiguities, the citation and page trackers are temporarily disabled in all floats. In addition to that, these trackers plus the back reference tracker (\opt{backref}) are temporarily disabled in the table of contents, the list of figures, and the list of tables.

\subsubsection{混合编程接口}%Mixing Programming Interfaces
\label{aut:cav:mif}

%The \biblatex package provides two main programming interfaces for style authors. The \cmd{DeclareBibliographyDriver} command, which defines a handler for an entry type, is typically used in \file{bbx} files. \cmd{DeclareCiteCommand}, which defines a new citation command, is typically used in \file{cbx} files. However, in some cases it is convenient to mix these two interfaces. For example, the \cmd{fullcite} command prints a verbose citation similar to the full bibliography entry. It is essentially defined as follows:

\biblatex 宏包给样式作者提供了2个主要的编程接口即: \file{bbx} 文件中使用的\cmd{DeclareBibliographyDriver} 命令用来定义各类参考文献条目的驱动(即条目的格式处理器),\file{cbx} 文件中使用的\cmd{DeclareCiteCommand} 命令用来定义新的标注命令。然而有时候,混合使用这两个接口会很方便。比如\cmd{fullcite} 命令就可以打印类似于完整参考文献条目的长串标注,该命令定义大体如下:
\begin{ltxexample}
\DeclareCiteCommand{\fullcite}
  {...}
  {<<\usedriver>>{...}{<<\thefield{entrytype}>>}}
  {...}
  {...}
\end{ltxexample}
%
%As you can see, the core code which prints the citations simply executes the bibliography driver defined with \cmd{DeclareBibliographyDriver} for the type of the current entry. When writing a citation style for a verbose citation scheme, it is often convenient to use the following structure:
如上所见,打印标注的核心代码简单地为当前的条目类型
执行了\cmd{DeclareBibliographyDriver} 定义的驱动命令。当为长标注格式(verbose)编写标注样式文件的时候,使用下面的结构是非常方便的:

\begin{ltxexample}
\ProvidesFile{example.cbx}[2007/06/09 v1.0 biblatex citation style]

\DeclareCiteCommand{\cite}
  {...}
  {<<\usedriver>>{...}{<<cite:\thefield{entrytype}>>}}
  {...}
  {...}

\DeclareBibliographyDriver{<<cite:article>>}{...}
\DeclareBibliographyDriver{<<cite:book>>}{...}
\DeclareBibliographyDriver{<<cite:inbook>>}{...}
...
\end{ltxexample}
%
%Another case in which mixing interfaces is helpful are styles using cross"=references within the bibliography. For example, when printing an \bibtype{incollection} entry, the data inherited from the \bibtype{collection} parent entry would be replaced by a short pointer to the respective parent entry:

混合接口的另一个有用情况是在参考文献表中使用交叉引用(cross"=references)时。比如当打印\bibtype{incollection} 类型的条目,数据继承自\bibtype{collection} 父条目,可由一个指向对应父条目的简短指针来代替。
\begin{enumerate}
\renewcommand*\labelenumi{[\theenumi]}
\setlength{\leftskip}{0.5em}
\item Audrey Author: \emph{Title of article}. In: [\textln{2}], pp.~134--165.
\item Edward Editor, ed.: \emph{Title of collection}. Publisher: Location, 1995.
\end{enumerate}

%One way to implement such cross"=references within the bibliography is to think of them as citations which use the value of the \bibfield{xref} or \bibfield{crossref} field as the entry key. Here is an example:
实现参考文献表内的这种交叉引用的一种方法是将它们当成标注,并使用\bibfield{xref} 或 \bibfield{crossref} 域的值作为条目关键词(条目bibtex键),示例如下:

\begin{ltxexample}
\ProvidesFile{example.bbx}[2007/06/09 v1.0 biblatex bibliography style]

\DeclareCiteCommand{<<\bbx@xref>>}
  {}
  {...}% code for cross-references
  {}
  {}

\DeclareBibliographyDriver{incollection}{%
  ...
  \iffieldundef{xref}
    {...}% code if no cross-reference
    {<<\bbx@xref>>{<<\thefield{xref}>>}}%
  ...
}
\end{ltxexample}
%
%When defining \cmd{bbx@xref}, the \prm{precode}, \prm{postcode}, and \prm{sepcode} arguments of \cmd{DeclareCiteCommand} are left empty in the above example because they will not be used anyway. The cross"=reference is printed by the \prm{loopcode} of \cmd{bbx@xref}. For further details on the \bibfield{xref} field, refer to \secref{bib:fld:spc} and to the hints in \secref{bib:cav:ref}. Also see the \cmd{iffieldxref}, \cmd{iflistxref}, and \cmd{ifnamexref} tests in \secref{aut:aux:tst}. The above could also be implemented using the \cmd{entrydata} command from \secref{aut:bib:dat}:
当定义\cmd{bbx@xref} 命令时,\cmd{DeclareCiteCommand} 命令的\prm{precode}, \prm{postcode}, 和 \prm{sepcode} 参数留空,是因为上面示例中没有用到。交叉引用由\cmd{bbx@xref} 命令的\prm{loopcode} 参数打印。更多的关于\bibfield{xref} 域的细节见\secref{bib:fld:spc} 节以及\secref{bib:cav:ref} 节中的注意事项。在\secref{aut:aux:tst} 节我们也看到了\cmd{iffieldxref}, \cmd{iflistxref}, 和\cmd{ifnamexref} 测试命令。这些都也可以用\secref{aut:bib:dat} 节的\cmd{entrydata} 命令来实现。

\begin{ltxexample}
\ProvidesFile{example.bbx}[2007/06/09 v1.0 biblatex bibliography style]

\DeclareBibliographyDriver{incollection}{%
  ...
  \iffieldundef{xref}
    {...}% code if no cross-reference
    {<<\entrydata>>{<<\thefield{xref}>>}{%
      % code for cross-references
      ...
    }}%
  ...
}
\end{ltxexample}

\subsubsection{使用标点追踪}%Using the Punctuation Tracker
\label{aut:cav:pct}

\paragraph{标点基础}%The Basics

%There is one fundamental principle style authors should keep in mind when designing a bibliography driver: block and unit punctuation is handled asynchronously. This is best explained by way of example. Consider the following code snippet:

样式作者设计参考文献驱动时需要记住一点原则:块和单元的标点是异步处理的。用示例最容易解释这一点,看下面一段代码:
\begin{ltxexample}
\printfield{title}%
\newunit
\printfield{edition}%
\newunit
\printfield{note}%
\end{ltxexample}
%
%If there is no \bibfield{edition} field, this piece of code will not print:
如果没有\bibfield{edition} 域,那么这段代码的打印结果不会是:

\begin{lstlisting}[style=highlight]{}
Title. . Note
\end{lstlisting}
%
%but rather:
而会是

\begin{lstlisting}[style=highlight]{}
Title. Note
\end{lstlisting}
%
因为单元的标点追踪器是异步方式工作的。\cmd{newunit} 命令将不会立即打印标点。它仅是记录一个单元的边界并且将\cmd{newunitpunct} 命令放入标点缓存中。该缓存会有\emph{接下来的}\cmd{printfield}、\cmd{printlist} 或类似命令进行处理,且仅当这些命令各自处理的域或列表已定义的时候才会处理。像\cmd{printfield} 这样的命令在插入任何块和单元的标点之前将首先考虑3个因素:
%because the unit punctuation tracker works asynchronously. \cmd{newunit} will not print the unit punctuation immediately. It merely records a unit boundary and puts \cmd{newunitpunct} on the punctuation buffer. This buffer will be handled by \emph{subsequent} \cmd{printfield}, \cmd{printlist}, or similar commands but only if the respective field or list is defined. Commands like \cmd{printfield} will consider three factors prior to inserting any block or unit punctuation:

\begin{itemize}
\item 是否有新的单元/块的输出请求?\par
=前面是否有\cmd{newunit} 或者\cmd{newblock} 命令?
%Has a new unit/block been requested at all?\par
%= Is there any preceding \cmd{newunit} or \cmd{newblock} command?

\item 前面的命令是否有打印输出?\par
=前面是否有\cmd{printfield} 或者相似命令?\par
=该命令是否实际打印了任何东西?\par
%Did the preceding commands print anything?\par
%= Is there any preceding \cmd{printfield} or similar command?\par
%= Did this command actually print anything?\par

\item 现在是否要打印一些东西?\par
要进行打印处理的域或列表是否已定义?
%Are we about to print anything now?\par
%= Is the field/list to be processed now defined?

\end{itemize}
%
%Block and unit punctuation will only be inserted if \emph{all} of these conditions apply. Let's reconsider the above example:
块和单元的标点只会在上述\emph{所有} 条件满足的时候才会输出。让我们再次考虑上面的示例:
\begin{ltxexample}
\printfield{title}%
\newunit
\printfield{edition}%
\newunit
\printfield{note}%
\end{ltxexample}
%
如果\bibfield{edition} 域没有定义会发生什么呢?第一个\cmd{printfield} 命令打印了标题并设置一个内部的<new~text>标志。第一个\cmd{newunit} 命令设置一个内部的<new~unit>标志。这使没有任何标点输出。第二个\cmd{printfield} 命令不进行任何处理因为\bibfield{edition} 域未定义。接下来的\cmd{newunit} 命令再次设置<new unit>标志,仍然没有标点输出。第三个\cmd{printfield} 命令检测\bibfield{note} 域是否已定义,如果是,它会寻找<new~text>和<new~unit>标志。如果两个标志都存在,那么它会在打印note前插入标点缓存。然后它会清除<new~unit>标志然后再次设置<new~text>标志。
%Here's what happens if the \bibfield{edition} field is undefined. The first \cmd{printfield} command prints the title and sets an internal <new~text> flag. The first \cmd{newunit} sets an internal <new~unit> flag. No punctuation has been printed at this point. The second \cmd{printfield} does nothing because the \bibfield{edition} field is undefined. The next \cmd{newunit} command sets the internal flag <new unit> again. Still no punctuation has been printed. The third \cmd{printfield} checks if the \bibfield{note} field is defined. If so, it looks at the <new~text> and <new~unit> flags. If both are set, it inserts the punctuation buffer before printing the note. It then clears the <new~unit> flag and sets the <new~text> flag again.

所有这些听起来似乎很复杂,但实际上,这意味着可以用顺序的方式写一个具有很多部件的参考文献驱动。这种方法的优势在不使用标点追踪而实现上述代码功能时会体现的很明显。如果不用标点追踪,那么会因为大量对所有可能存在域的判断产生一个复杂的\cmd{iffieldundef} 判断命令集合。
%This may all sound more complicated than it is. In practice, it means that it is possible to write large parts of a bibliography driver in a sequential way. The advantage of this approach becomes obvious when trying to write the above code without using the punctuation tracker. Such an attempt will lead to a rather convoluted set of \cmd{iffieldundef} tests required to check for all possible field combinations (note that the code below handles three fields; a typical driver may need to cater for some two dozen fields):

\begin{ltxexample}
\iffieldundef{title}%
  {\iffieldundef{edition}
     {\printfield{note}}
     {\printfield{edition}%
      \iffieldundef{note}%
	{}
	{. \printfield{note}}}}
  {\printfield{title}%
   \iffieldundef{edition}
     {}
     {. \printfield{edition}}%
   \iffieldundef{note}
     {}
     {. \printfield{note}}}%
\end{ltxexample}

\paragraph{常见错误}%Common Mistakes

把单元的标点处理认为是同步处理的是一个相当常见的误解。这会导致当驱动中包含抄录文本
%\footnote{这里literal text 理解为原样文本,如实文本,逐字文本,抄录文本,照抄文本}
时出现一些典型错误。考虑下面导致标点错位的错误代码段:
%It is a fairly common misconception to think of the unit punctuation as something that is handled synchronously. This typically causes problems if the driver includes any literal text. Consider this erroneous code snippet which will generate misplaced unit punctuation:

\begin{ltxexample}
\printfield{title}%
\newunit
<<(>>\printfield{series} \printfield{number}<<)>>%
\end{ltxexample}
%
%This code will yield the following result:
这段代码将产生下面的结果:

\begin{lstlisting}[style=highlight]{}
Title <<(.>> Series Number<<)>>
\end{lstlisting}
%
这里发生了什么呢?第一个\cmd{printfield} 命令打印了标题,然后\cmd{newunit} 命令标记了一个新的单元边界但不打印任何内容。单元的标点由\emph{下一个}\cmd{printfield} 命令打印。这是前面提过的异步机制。然而因为左括号在下一个
\cmd{printfield} 命令插入标点前立即打印,所以导致了错误的句点。当插入\emph{任何} 原样文本比如括号(还包括由
\cmd{bibopenparen} 和\cmd{mkbibparens} 命令打印的括号)时,总需要将这些文本用\cmd{printtext} 命令包起来。要让标点追踪正常运转,需要让驱动知道所有插入的原样文本。这是\cmd{printtext} 命令的作用所在。\cmd{printtext} 命令联系标点追踪器确保标点缓存在原样文本打印前插入。它也设定内部 <new~text> 标志。注意本例中还有第三处原样文本即|\printfield{series}|后面的空格。在改正的示例中,我们将使用标点追踪器来处理该空格。
%Here's what happens. The first \cmd{printfield} prints the title. Then \cmd{newunit} marks a unit boundary but does not print anything. The unit punctuation is printed by the \emph{next} \cmd{printfield} command. That's the asynchronous part mentioned before. However, the opening parenthesis is printed immediately before the next \cmd{printfield} inserts the unit punctuation, leading to a misplaced period. When inserting \emph{any} literal text such as parentheses (including those printed by commands such as \cmd{bibopenparen} and \cmd{mkbibparens}), always wrap the text in a \cmd{printtext} command. For the punctuation tracker to work as expected, it needs to know about all literal text inserted by a driver. This is what \cmd{printtext} is all about. \cmd{printtext} interfaces with the punctuation tracker and ensures that the punctuation buffer is inserted before the literal text gets printed. It also sets the internal <new~text> flag. Note there is in fact a third piece of literal text in this example: the space after |\printfield{series}|. In the corrected example, we will use the punctuation tracker to handle that space.

\begin{ltxexample}
\printfield{title}%
\newunit
<<\printtext{(}>>%
\printfield{series}%
<<\setunit*{\addspace}>>%
\printfield{number}%
<<\printtext{)}>>%
\end{ltxexample}
%
尽管上面的代码能够如常工作,但处理括号、引号和其它包围某个域的标点是,推荐的方式是定义一个域格式:
%While the above code will work as expected, the recommended way to handle parentheses, quotes, and other things which enclose more than one field, is to define a field format:

\begin{ltxexample}
\DeclareFieldFormat{<<parens>>}{<<\mkbibparens{#1}>>}
\end{ltxexample}
%
域格式可以同时用于\cmd{printfield} 和\cmd{printtext} 命令,因此我们可以利用它对若干个域用一堆括号进行包裹。
%Field formats may be used with both \cmd{printfield} and \cmd{printtext}, hence we can use them to enclose several fields in a single pair of parentheses:

\begin{ltxexample}
<<\printtext[parens]{>>%
  \printfield{series}%
  \setunit*{\addspace}%
  \printfield{number}%
<<}>>%
\end{ltxexample}
%
这里我们还需要处理没有series信息时的情况,因此进一步改进代码如下:
%We still need to handle cases in which there is no series information at all, so let's improve the code some more:

\begin{ltxexample}
<<\iffieldundef{series}>>
  {}
  {\printtext[parens]{%
     \printfield{series}%
     \setunit*{\addspace}%
     \printfield{number}}}%
\end{ltxexample}
%
最后的一点提示: 本地化字符串对于标点追踪器来说不是原样文本。因为\cmd{bibstring} 和相似命令能联系标点追踪器,因此就不需要用\cmd{printtext} 包裹起来。
%One final hint: localisation strings are not literal text as far as the punctuation tracker is concerned. Since \cmd{bibstring} and similar commands interface with the punctuation tracker, there is no need to wrap them in a \cmd{printtext} command.

\paragraph{高级用法}%Advanced Usage
标点追踪器也可用来处理更复杂的情况。比如,考虑需要对\bibfield{location}、 \bibfield{publisher} 和 \bibfield{year} 根据数据是否提供以如下的格式打印:
%The punctuation tracker may also be used to handle more complex scenarios. For example, suppose that we want the fields \bibfield{location}, \bibfield{publisher}, and \bibfield{year} to be rendered in one of the following formats, depending on the available data:

\begin{ltxexample}
...text<<. Location: Publisher, Year.>> Text...
...text<<. Location: Publisher.>> Text...
...text<<. Location: Year.>> Text...
...text<<. Publisher, Year.>> Text...
...text<<. Location.>> Text...
...text<<. Publisher.>> Text...
...text<<. Year.>> Text...
\end{ltxexample}
%
这个问题可以用一个相当复杂的\cmd{iflistundef} 和\cmd{iffieldundef} 判断集进行处理,通过这些判断可以确定所有可能的域的组合:
%This problem can be solved with a rather convoluted set of \cmd{iflistundef} and \cmd{iffieldundef} tests which check for all possible field combinations:

\begin{ltxexample}
\iflistundef{location}
  {\iflistundef{publisher}
     {\printfield{year}}
     {\printlist{publisher}%
      \iffieldundef{year}
        {}
        {, \printfield{year}}}}
  {\printlist{location}%
   \iflistundef{publisher}%
     {\iffieldundef{year}
        {}
        {: \printfield{year}}}
     {: \printlist{publisher}%
      \iffieldundef{year}
        {}
        {, \printfield{year}}}}%
\end{ltxexample}
%
可以应用\cmd{ifthenelse} 命令和\secref{aut:aux:ife} 讨论的布尔运算可使上面的代码更具可读性。但本质上是一样的。然而,也可以按顺序写成如下方式:
%The above could be written in a somewhat more readable way by employing \cmd{ifthenelse} and the boolean operators discussed in \secref{aut:aux:ife}. The approach would still be essentially the same. However, it may also be written sequentially:

\begin{ltxexample}
\newunit
\printlist{location}%
\setunit*{\addcolon\space}%
\printlist{publisher}%
\setunit*{\addcomma\space}%
\printfield{year}%
\newunit
\end{ltxexample}
%
在实际使用中,你会经常使用标点追踪器执行一些显式或隐式的组合判断,比如,考虑如下格式(注意当没有publisher时location后面的标点)
%In practice, you will often use a combination of explicit tests and the implicit tests performed by the punctuation tracker. For example, consider the following format (note the punctuation after the location if there is no publisher):

\begin{ltxexample}
...text. Location: Publisher, Year. Text...
...text. Location: Publisher. Text...
...text<<. Location, Year.>> Text...
...text. Publisher, Year. Text...
...text. Location. Text...
...text. Publisher. Text...
...text. Year. Text...
\end{ltxexample}
%
这可以用如下代码进行处理:
%This can be handled by the following code:

\begin{ltxexample}
\newunit
\printlist{location}%
\iflistundef{publisher}
  {\setunit*{\addcomma\space}}
  {\setunit*{\addcolon\space}}%
\printlist{publisher}%
\setunit*{\addcomma\space}%
\printfield{year}%
\newunit
\end{ltxexample}
%
因为当没有publisher时location后面的标点的特殊性,我们需要用一个\cmd{iflistundef} 判断来确保正确性。剩下其它的则有标点追踪器处理。
%Since the punctuation after the location is special if there is no publisher, we need one \cmd{iflistundef} test to catch this case. Everything else is handled by the punctuation tracker.

\subsubsection{本地化定制模型}%Custom Localization Modules
\label{aut:cav:lng}

参考文献格式要求可能包含某些规定,比如像<edition>之类的字符串怎么缩写或者需要表示成一些固定的形式。例如\acr{mla} 格式要求作者在参考文献表的标题中使用<Works~Cited>而不是<Bibliography>或<References>。本地化命令比如\secref{use:lng} 节的\cmd{DefineBibliographyStrings} 可以在\file{cbx} 和\file{bbx} 文件中处理这些情况。然而,用翻译内容重载样式文件是相当不便的。而这正是\secref{aut:lng:cmd} 节的\cmd{DeclareLanguageMapping} 命令发挥作用的地方。这一命令将一个\file{lbx} 文件映射为某一\sty{babel}/\sty{polyglossia} 语言的替代翻译。例如,可以创建一个名为\path{french-humanities.lbx} 的文件提供适用于人文学科的法语翻译并在导言区或者配置文件中将其映射为\sty{babel}/\sty{polyglossia} 语言\texttt{french}。
%Style guides may include provisions as to how strings like <edition> should be abbreviated or they may mandate certain fixed expressions. For example, the \acr{mla} style guide requires authors to use the term <Works~Cited> rather than <Bibliography> or <References> in the heading of the bibliography. Localization commands such as \cmd{DefineBibliographyStrings} from \secref{use:lng} may indeed be used in \file{cbx} and \file{bbx} files to handle such cases. However, overloading style files with translations is rather inconvenient. This is where \cmd{DeclareLanguageMapping} from \secref{aut:lng:cmd} comes into play. This command maps an \file{lbx} file with alternative translations to a \sty{babel}/\sty{polyglossia} language. For example, you could create a file named \path{french-humanities.lbx} which provides French translations adapted for use in the humanities and map it to the \sty{babel}/\sty{polyglossia} language \texttt{french} in the preamble or in the configuration file:

\begin{ltxexample}
\DeclareLanguageMapping{french}{french-humanities}
\end{ltxexample}
%
如果正文的语言设置为\texttt{french},\path{french-humanities.lbx} 将替换\path{french.lbx}。回到前述的\acr{mla} 示例,一个\acr{mla} 样式可能带有一个\path{american-mla.lbx} 文件来提供符合\acr{mla} 格式规要求的字符串。它将在\file{cbx} 和/或 \file{bbx} 文件中声明如下映射:
%If the document language is set to \texttt{french}, \path{french-humanities.lbx} will replace \path{french.lbx}. Coming back to the \acr{mla} example mentioned above, an \acr{mla} style may come with an \path{american-mla.lbx} file to provide strings which comply with the \acr{mla} style guide. It would declare the following mapping in the \file{cbx} and/or \file{bbx} file:

\begin{ltxexample}
\DeclareLanguageMapping{american}{american-mla}
\end{ltxexample}
%
%Use \cmd{DeclareLanguageMappingSuffix} (see \secref{aut:lng:cmd}) to define such a mapping for all languages.
使用\cmd{DeclareLanguageMappingSuffix}(see \secref{aut:lng:cmd})来定义针对所有语言的映射。

因为替换的\file{lbx} 文件可以从标准的\path{american.lbx} 模型中继承字符串,所以\path{american-mla.lbx} 可以简化为:
%Since the alternative \file{lbx} file can inherit strings from the standard \path{american.lbx} module, \path{american-mla.lbx} may be as short as this:

\begin{ltxexample}
\ProvidesFile{american-mla.lbx}[2008/10/01 v1.0 biblatex localization]
<<\InheritBibliographyExtras>>{<<american>>}
\DeclareBibliographyStrings{%
  <<inherit>>          = {<<american>>},
  bibliography     = {{Works Cited}{Works Cited}},
  references       = {{Works Cited}{Works Cited}},
}
\endinput
\end{ltxexample}
%
替换的\file{lbx} 文件必须保证本地化模型是完整的。这可以通过从相应的标准模型中继承来实现。如果语言\texttt{american} 映射为\path{american-mla.lbx},\biblatex 将不会加载\path{american.lbx} 除非该模型被明确要求加载。在上述示例中,继承<strings>和<extras>将使得\biblatex 在\path{american-mla.lbx} 中应用修改前加载\path{american.lbx}。
%Alternative \file{lbx} files must ensure that the localisation module is complete. They should do so by inheriting data from the corresponding standard module. If the language \texttt{american} is mapped to \path{american-mla.lbx}, \biblatex will not load \path{american.lbx} unless this module is requested explicitly. In the above example, inheriting <strings> and <extras> will cause \biblatex to load \path{american.lbx} before applying the modifications in \path{american-mla.lbx}.

注意:\cmd{DeclareLanguageMapping} 不用于处理语言的变体(比如American English 和 British English))或者\sty{babel}/\sty{polyglossia} 语言别名(比如\texttt{USenglish} vs. \texttt{american})。例如,\sty{babel}/\sty{polyglossia} 提供了\texttt{USenglish} 选项类似于\texttt{american}。因此\biblatex 附带一个\path{USenglish.lbx} 文件,该文件简单的从\path{american.lbx} 文件中继承所有数据(而该文件又从\path{english.lbx} 文件中获取字符串)。换句话说,语言变体和\sty{babel}/\sty{polyglossia} 语言别名的映射发射在文件层,因此\biblatex 的语言支持可以简单地增加额外的\file{lbx} 文件来实现拓展。没有必要进行集中的映射,如果需要支持比如Portuguese (babel/polyglossia: \file{portuges}),可以创建一个名为\path{portuges.lbx} 的文件。如果\sty{babel}/\sty{polyglossia} 提供一个名为\texttt{brasil} 的别名,可以创建一个\path{brasil.lbx} 文件并可从\path{portuges.lbx} 中继承数据。相比之下,\cmd{DeclareLanguageMapping} 主要用于处理\emph{文体} 上的变化,像<humanities 对 natural sciences> or <\acr{mla} 对 \acr{apa}> 等,这通常是建立在现有的\file{lbx} 文件基础上的。
%Note that \cmd{DeclareLanguageMapping} is not intended to handle language variants (\eg American English vs. British English) or \sty{babel}/\sty{polyglossia} language aliases (\eg \texttt{USenglish} vs. \texttt{american}). For example, \sty{babel}/\sty{polyglossia} offers the \texttt{USenglish} option which is similar to \texttt{american}. Therefore, \biblatex ships with an \path{USenglish.lbx} file which simply inherits all data from \path{american.lbx} (which in turn gets the <strings> from \path{english.lbx}). In other words, the mapping of language variants and \sty{babel}/\sty{polyglossia} language aliases happens on the file level, the point being that \biblatex's language support can be extended simply by adding additional \file{lbx} files. There is no need for centralized mapping. If you need support for, say, Portuguese (babel/polyglossia: \file{portuges}), you create a file named \path{portuges.lbx}. If \sty{babel}/\sty{polyglossia} offered an alias named \texttt{brasil}, you would create \path{brasil.lbx} and inherit the data from \path{portuges.lbx}. In contrast to that, the point of \cmd{DeclareLanguageMapping} is handling \emph{stylistic} variants like <humanities vs. natural sciences> or <\acr{mla} vs. \acr{apa}> etc. which will typically be built on top of existing \file{lbx} files.

\subsubsection{编组}%Grouping
\label{aut:cav:grp}

在标注和著录样式中,可能需要设置标志或保存一些值以便后面使用。这种情况下,理解宏包执行的基本编组结构很关键。一条经验法则是,无论诸如\secref{aut:aux} 节讨论的样式作者命令是否存在,所有的工作都是在一个大的编组内进行的,因为本宏包的作者接口都是局部的。当存在参考文献数据时,至少存在一个额外的编组,下面是一些基本规则:
%In a citation or bibliography style, you may need to set flags or store certain values for later use. In this case, it is crucial to understand the basic grouping structure imposed by this package. As a rule of thumb, you are working in a large group whenever author commands such as those discussed in \secref{aut:aux} are available because the author interface of this package is only enabled locally. If any bibliographic data is available, there is at least one additional group. Here are some general rules:

\begin{itemize}

\item 由\cmd{printbibliography} 或类似命令打印的整个文献表是在一个编组中处理。表中的每个条目都是在一个额外的编组中,这一编组包含了\cmd{defbibenvironment} 的\prm{item code} 和所有的驱动代码。
%The entire list of references printed by \cmd{printbibliography} and similar commands is processed in a group. Each entry in the list is processed in an additional group which encloses the \prm{item code} of \cmd{defbibenvironment} as well as all driver code.

\item 由\cmd{printbiblist} 命令打印的整个文献表是在一个编组中处理。表中的每个条目都是在一个额外的编组中,这一编组包含了\cmd{defbibenvironment} 的\prm{item code} 和所有的驱动代码。
%The entire bibliography list printed by \cmd{printbiblist} is processed in a group. Each entry in the list is processed in an additional group which encloses the \prm{item code} of \cmd{defbibenvironment} as well as all driver code.

\item 所有由\cmd{DeclareCiteCommand} 命令定义的标注命令都是在一个编组中处理,该编组包含由\prm{precode}, \prm{sepcode}, \prm{loopcode}, 和\prm{postcode} 等参数构成的完整标注代码。每次执行的\prm{loopcode} 都包含在一个额外的编组中。如果指定了任何的\prm{wrapper},包含\prm{wrapper} 代码和标注代码的整个单元都在一个额外的编组中。
%All citation commands defined with \cmd{DeclareCiteCommand} are processed in a group holding the complete citation code consisting of the \prm{precode}, \prm{sepcode}, \prm{loopcode}, and \prm{postcode} arguments. The \prm{loopcode} is enclosed in an additional group every time it is executed. If any \prm{wrapper} code has been specified, the entire unit consisting of the wrapper code and the citation code is wrapped in an additional group.

\item 除了由\cmd{DeclareCiteCommand} 定义的所有后端命令会产生编组外,所有的<autocite> 和 <multicite>定义也都会产生一个额外的编组。
%In addition to the grouping imposed by all backend commands defined with \cmd{DeclareCiteCommand}, all <autocite> and <multicite> definitions imply an additional group.

\item \cmd{printfile}, \cmd{printtext}, \cmd{printfield}, \cmd{printlist}, 和\cmd{printnames} 命令也形成编组。这意味着所有的格式化指令都是在它们自身的编组中处理。
%\cmd{printfile}, \cmd{printtext}, \cmd{printfield}, \cmd{printlist}, and \cmd{printnames} form groups. This implies that all formatting directives will be processed within a group of their own.

\item 所有的\file{lbx} 文件都是在一个编组中加载和处理。如果一个\file{lbx} 文件包含的一些代码不是\cmd{DeclareBibliographyExtras} 的一部分,那么这些定义是全局的。
%All \file{lbx} files are loaded and processed in a group. If an \file{lbx} file contains any code which is not part of \cmd{DeclareBibliographyExtras}, the definitions must be global.

\end{itemize}

注意:在标注和著录样式中使用\cmd{aftergroup} 是不可靠的,因为在一定环境中应用的编组的精确层数在宏包的未来版本中可能发生变化。上述说明中如果说某些东西在一个编组中处理,这意味着至少存在一个编组,也可能存在多层嵌套的编组。
%Note that using \cmd{aftergroup} in citation and bibliography styles is unreliable because the precise number of groups employed in a certain context may change in future versions of this package. If the above list states that something is processed in a group, this means that there is \emph{at least one} group. There may also be several nested ones.

\subsubsection{命名空间}%Namespaces
\label{aut:cav:nam}
为减小命名冲突的风险, \latex 宏包通常在其内部宏名前加上一个代表该宏包的短字符串。例如: 如果\sty{foobar} 宏包需要一个内部使用的宏,它通常会命名为\cmd{FB@macro} 或\cmd{foo@macro} 而不是\cmd{macro} or \cmd{@macro}。下面是\biblatex 使用或推荐使用的前缀字符串:
%In order to minimize the risk of name clashes, \latex packages typically prefix the names of internal macros with a short string specific to the package. For example, if the \sty{foobar} package requires a macro for internal use, it would typically be called \cmd{FB@macro} or \cmd{foo@macro} rather than \cmd{macro} or \cmd{@macro}. Here is a list of the prefixes used or recommended by \biblatex:

\begin{marglist}

\item[\texttt{blx}] 所有的宏名像\cmd{blx@name} 的宏严格作为内部使用。这也应用于计数器名,长度名,布尔开关等。这些宏可能会以非后向兼容的方式改变,它们可能会重命名设置删除掉而不会有更多的说明。这种改变也不会在版本修改历史和版本发布信息中出现。简而言之: 不要在任何样式中使用以\texttt{blx} 字符串开头的宏。
%All macros with names like \cmd{blx@name} are strictly reserved for internal use. This also applies to counter names, length registers, boolean switches, and so on. These macros may be altered in backwards"=incompatible ways, they may be renamed or even removed at any time without further notice. Such changes will not even be mentioned in the revision history or the release notes. In short: never use any macros with the string \texttt{blx} in their name in any styles.

\item[\texttt{abx}] 以\texttt{abx} 为前缀的宏也是内部使用,但会相当稳定。但仍应优先使用正式的作者接口提供的工具,但有些情况下使用某一\texttt{abx} 宏可能会比较方便。
%Macros prefixed with \texttt{abx} are also internal macros but they are fairly stable. It is always preferable to use the facilities provided by the official author interface, but there may be cases in which using an \texttt{abx} macro is convenient.

\item[\texttt{bbx}] 建议在著录样式中内部使用的宏名的前缀
%This is the recommended prefix for internal macros defined in bibliography styles.

\item[\texttt{cbx}] 建议在标注样式中内部使用的宏名的前缀
%This is the recommended prefix for internal macros defined in citation styles.

\item[\texttt{lbx}] 建议在本地化模型中内部使用的宏名的前缀。本地化模型需要添加第二个前缀来指定语言,比如一个为西班牙语本地化模型定义的内部宏可以命名为\cmd{lbx@es@macro}。
%This is the recommended base prefix for internal macros defined in localisation modules. The localisation module should add a second prefix to specify the language. For example, an internal macro defined by the Spanish localisation module would be named \cmd{lbx@es@macro}.

\end{marglist}






\appendix
\section*{附录 Appendix}
\addcontentsline{toc}{section}{附录}

\section{驱动层的默认数据源映射}%默认的驱动源映射
%\section{Default Driver Source Mappings}
\label{apx:maps}

These are the driver default source mappings.

\subsection{\opt{bibtex}}
The \opt{bibtex} driver is of course the most comprehensive and mature of the \biblatex/\biber supported data formats. These source mapping defaults are how the aliases from sections \secref{bib:typ:als} and \secref{bib:fld:als} are implemented.

\begin{ltxexample}
\DeclareDriverSourcemap[datatype=bibtex]{
  \map{
    \step[typesource=conference, typetarget=inproceedings]
    \step[typesource=electronic, typetarget=online]
    \step[typesource=www,        typetarget=online]
  }
  \map{
    \step[typesource=mastersthesis, typetarget=thesis, final]
    \step[fieldset=type,            fieldvalue=mathesis]
  }
  \map{
    \step[typesource=phdthesis, typetarget=thesis, final]
    \step[fieldset=type,        fieldvalue=phdthesis]
  }
  \map{
    \step[typesource=techreport, typetarget=report, final]
    \step[fieldset=type,         fieldvalue=techreport]
  }
  \map{
    \step[fieldsource=address,       fieldtarget=location]
    \step[fieldsource=school,        fieldtarget=institution]
    \step[fieldsource=annote,        fieldtarget=annotation]
    \step[fieldsource=archiveprefix, fieldtarget=eprinttype]
    \step[fieldsource=journal,       fieldtarget=journaltitle]
    \step[fieldsource=primaryclass,  fieldtarget=eprintclass]
    \step[fieldsource=key,           fieldtarget=sortkey]
    \step[fieldsource=pdf,           fieldtarget=file]
  }
}
\end{ltxexample}

\section{默认继承设置 Default Inheritance Setup}
\label{apx:ref}

The following table shows the \biber cross-referencing rules defined by default.
Please refer to \secref{bib:cav:ref, aut:ctm:ref} for explanation.

\begingroup
\tablesetup
\def\sep{\textrm{, }}
\def\skip{\textrm{--}}
\def\note#1{\textrm{#1}}
\begin{longtable}[l]{%
	@{}V{0.2\textwidth}%
	@{}V{0.4\textwidth}%
	@{}V{0.3\textwidth}%
	@{}V{0.3\textwidth}@{}}
\toprule
\multicolumn{2}{@{}H}{Types} & \multicolumn{2}{@{}H}{Fields} \\
\cmidrule(r){1-2}\cmidrule{3-4}
\multicolumn{1}{@{}H}{Source} & \multicolumn{1}{@{}H}{Target} &
\multicolumn{1}{@{}H}{Source} & \multicolumn{1}{@{}H}{Target} \\
\cmidrule(r){1-1}\cmidrule(r){2-2}\cmidrule(r){3-3}\cmidrule{4-4}
\endhead
\bottomrule
\endfoot
\textasteriskcentered & \textasteriskcentered &
  ids\par
	crossref\par
	xref\par
	entryset\par
	entrysubtype\par
	execute\par
	label\par
	options\par
	presort\par
	related\par
	relatedoptions\par
	relatedstring\par
	relatedtype\par
	shorthand\par
	shorthandintro\par
	sortkey &
	\skip\par \skip\par \skip\par \skip\par
	\skip\par \skip\par \skip\par \skip\par
	\skip\par \skip\par \skip\par \skip\par
	\skip\par \skip \\
mvbook\sep book &
	inbook\sep bookinbook\sep suppbook &
	author\par author &
	author\par bookauthor \\
mvbook &
	book\sep inbook\sep bookinbook\sep suppbook &
	title\par subtitle\par titleaddon\par
        shorttitle\par sorttitle\par indextitle\par indexsorttitle &
	maintitle\par mainsubtitle\par maintitleaddon\par
	\skip\par \skip\par \skip\par \skip \\
mvcollection\sep mvreference &
	collection\sep reference\sep incollection\sep inreference\sep suppcollection &
	title\par subtitle\par titleaddon\par
        shorttitle\par sorttitle\par indextitle\par indexsorttitle &
	maintitle\par mainsubtitle\par maintitleaddon\par
	\skip\par \skip\par \skip\par \skip \\
mvproceedings	&
	proceedings\sep inproceedings &
	title\par subtitle\par titleaddon\par
        shorttitle\par sorttitle\par indextitle\par indexsorttitle &
	maintitle\par mainsubtitle\par maintitleaddon\par
	\skip\par \skip\par \skip\par \skip \\
book &
	inbook\sep bookinbook\sep suppbook &
	title\par subtitle\par titleaddon\par
        shorttitle\par sorttitle\par indextitle\par indexsorttitle &
	booktitle\par booksubtitle\par booktitleaddon\par
	\skip\par \skip\par \skip\par \skip \\
collection\sep reference &
	incollection\sep inreference\sep suppcollection &
	title\par subtitle\par titleaddon\par
        shorttitle\par sorttitle\par indextitle\par indexsorttitle &
	booktitle\par booksubtitle\par booktitleaddon\par
	\skip\par \skip\par \skip\par \skip \\
proceedings &
	inproceedings &
	title\par subtitle\par titleaddon\par
        shorttitle\par sorttitle\par indextitle\par indexsorttitle &
	booktitle\par booksubtitle\par booktitleaddon\par
	\skip\par \skip\par \skip\par \skip \\
periodical &
	article\sep suppperiodical &
	title\par subtitle\par
        shorttitle\par sorttitle\par indextitle\par indexsorttitle &
	journaltitle\par journalsubtitle\par
	\skip\par \skip\par \skip\par \skip \\
\end{longtable}
\endgroup

\section{默认的排序方式}
%\section{Default Sorting Schemes}
\label{apx:srt}

\subsection[Alphabetic 1]{Alphabetic Schemes 1}
\label{apx:srt:a1}

The following table shows the standard alphabetic sorting schemes defined by default. Please refer to \secref{use:srt} for explanation.

\begingroup
\sorttablesetup
\begin{longtable}[l]{@{}%
	V{0.100\textwidth}@{}%
	L{0.100\textwidth}@{}%
	L{0.175\textwidth}@{}%
	L{0.175\textwidth}@{}%
	L{0.150\textwidth}@{}%
	L{0.300\textwidth}@{}}
\toprule
\multicolumn{1}{@{}H}{Option} & \multicolumn{5}{@{}H}{Sorting scheme} \\
\cmidrule(r){1-1}\cmidrule{2-6}
\endhead
\bottomrule
\endfoot
nty &	presort\alt mm &
	\new sortname\alt author\alt editor\alt translator\alt sorttitle\alt title &
	\new sorttitle\alt title &
	\new sortyear\alt year &
	\new volume\\
nyt &	presort\alt mm &
	\new sortname\alt author\alt editor\alt translator\alt sorttitle\alt title &
	\new sortyear\alt year &
	\new sorttitle\alt title &
	\new volume\\
nyvt &	presort\alt mm &
	\new sortname\alt author\alt editor\alt translator\alt sorttitle\alt title &
	\new sortyear\alt year &
	\new volume &
	\new sorttitle\alt title \\
\textrm{all} & presort\alt mm &
	\new sortkey \\
\end{longtable}
\endgroup

\subsection[Alphabetic 2]{Alphabetic Schemes 2}
\label{apx:srt:a2}

The following table shows the alphabetic sorting schemes for \texttt{alphabetic} styles defined by default. Please refer to \secref{use:srt} for explanation.

\begingroup
\sorttablesetup
\begin{longtable}[l]{@{}%
	V{0.100\textwidth}@{}%
	L{0.100\textwidth}@{}%
	L{0.175\textwidth}@{}%
	L{0.175\textwidth}@{}%
	L{0.150\textwidth}@{}%
	L{0.150\textwidth}@{}%
	L{0.150\textwidth}@{}}
\toprule
\multicolumn{1}{@{}H}{Option} & \multicolumn{6}{@{}H}{Sorting scheme} \\
\cmidrule(r){1-1}\cmidrule{2-7}
\endhead
\bottomrule
\endfoot
anyt &	presort\alt mm &
	\new labelalpha &
	\new sortname\alt author\alt editor\alt translator\alt sorttitle\alt title &
	\new sortyear\alt year &
	\new sorttitle\alt title &
	\new volume\\
anyvt &	presort\alt mm &
	\new labelalpha &
	\new sortname\alt author\alt editor\alt translator\alt sorttitle\alt title &
	\new sortyear\alt year &
	\new volume &
	\new sorttitle\alt title \\
\textrm{all} & presort\alt mm &
	\new labelalpha &
	\new sortkey \\
\end{longtable}
\endgroup

\subsection[Chronological]{Chronological Schemes}
\label{apx:srt:chr}

The following table shows the chronological sorting schemes defined by default. Please refer to \secref{use:srt} for explanation.

\begingroup
\sorttablesetup
\begin{longtable}[l]{@{}%
	V{0.100\textwidth}@{}%
	L{0.100\textwidth}@{}%
	L{0.225\textwidth}@{}%
	L{0.175\textwidth}@{}%
	L{0.400\textwidth}@{}}
\toprule
\multicolumn{1}{@{}H}{Option} & \multicolumn{4}{@{}H}{Sorting scheme} \\
\cmidrule(r){1-1}\cmidrule{2-5}
\endhead
\bottomrule
\endfoot
ynt &	presort\alt mm &
	\new sortyear\alt year \alt 9999 &
	\new sortname\alt author\alt editor\alt translator\alt sorttitle\alt title &
	\new sorttitle\alt title \\
ydnt &	presort\alt mm &
	\new sortyear\note{ (desc.)}\alt year\note{ (desc.)} \alt 9999 &
	\new sortname\alt author\alt editor\alt translator\alt sorttitle\alt title &
	\new sorttitle\alt title \\
\textrm{all} & presort\alt mm &
	\new sortkey \\
\end{longtable}
\endgroup

\section{\biblatexml}
\label{apx:biblatexml}
The \biblatexml\ XML datasource format is designed to be an extensible and modern data source format for \biblatex\ users. There are limitations with \bibtex\ format \file{.bib} files, in particular one might mention UTF-8 support and name formats. \biber\ goes some way to addressing the UTF-8 limitations by using a modified version of the \texttt{btparse} C library but the rather archaic name parsing rules for \bibtex\ are hard-coded and specific to simple Western names.

\biblatexml is an XML format for bibliographic data. When \biber\ either reads or writes \biblatexml\ format datasources, it automatically writes a RelaXNG XML schema for the datasources which is dynamically generated from the active \biblatex\ datamodel. There is no static schema for \biblatexml\ datasources because the allowable fields etc. depend on the data model. The format of \biblatexml\ datasources is relatively self-explanatory---it is usually only necessary to generate a \biblatexml\ datasource from existing \bibtex\ format datasources (using \biber's <tool> mode) in order to understand the format. \biber also allows users to validate \biblatexml\ datasources against the data model generated schema.

Since the \biblatexml\ format is XML and depends on the data model and the data model is extensible by the user (see \secref{aut:ctm:dm}), the \biblatexml\ format can deal with extensions that \bibtex\ format data sources cannot, e.g. new nameparts, options at sub-entry scope. Since it is an XML format, it is relatively easy to transform it into other XML formats or HTML using standard XML processing libraries and tools.

Here is an explanation of the format with examples. By convention, \biblatexml\ files have a \file{.bltxml} extension and \file{kpsewhich} understands this file extension.

\subsection{Header}
\biblatexml\ files begin with the standard XML header:

\begin{lstlisting}[language=xml]
<?xml version="1.0" encoding="UTF-8"?>
\end{lstlisting}
%
The schema model, type and schema type namespace are given in the following line:

\begin{lstlisting}[language=xml]
<?xml-model href="biblatexml.rng"
            type="application/xml"
            schematypens="http://relaxng.org/ns/structure/1.0"?>
\end{lstlisting}
%
When \biber\ generates \biblatexml\ data sources, it automatically adds this line and points the schema model (href) attribute at the automatically generated RelaXNG XML schema for ease of validation.

\subsection{Body}

The body of a \biblatexml\ data source looks like:

\begin{lstlisting}[language=xml]
<bltx:entries
  xmlns:bltx="http://biblatex-biber.sourceforge.net/biblatexml">

  <bltx:entry id="" entrytype="">
  </bltx:entry>
       .
       .
       .
  <bltx:entry id="" entrytype="">
  </bltx:entry>

</bltx:entries>
\end{lstlisting}
%
The body is one or more \bibfield{entry} elements inside the top-level \bibfield{entries} element and everything is in the \bibfield{bltx} namespace. An entry has an \bibfield{id} attribute corresponding to a \bibtex\ entry key and a \bibfield{entrytype} attribute corresponding to a \bibtex\ entrytype. For example, the \biblatexml\

\begin{lstlisting}[language=xml]
<?xml version="1.0" encoding="UTF-8"?>
<?xml-model href="biblatexml.rng"
            type="application/xml"
            schematypens="http://relaxng.org/ns/structure/1.0"?>
<bltx:entries
  xmlns:bltx="http://biblatex-biber.sourceforge.net/biblatexml">
  <bltx:entry id="key1" entrytype="book">
  </bltx:entry>
</bltx:entries>
\end{lstlisting}
%
Corresponds to the \bibtex\ \file{.bib}

\begin{lstlisting}[style=bibtex]{}
@book{key1,
}
\end{lstlisting}
%
In general, the XML elements in a \biblatexml\ format datasource file have names corresponding to the fields in the datamodel, just like \bibtex\ format datasources. So for example, the \bibtex\ format source

\begin{lstlisting}[style=bibtex]{}
@book{key1,
  TITLE = {...},
  ISSUE = {...},
  NOTE = {...}
}
\end{lstlisting}
%
would be, in \biblatexml

\begin{lstlisting}[language=xml]
  <bltx:entry id="key1" entrytype="book">
    <bltx:title>...</bltx:title>
    <bltx:issue>...</bltx:issue>
    <bltx:note>...</bltx:note>
  </bltx:entry>
\end{lstlisting}
%
The following exceptions to this simple mapping are to be noted

\subsubsection{Key aliases}
Citation key aliases are specified like this:

\begin{lstlisting}[language=xml]
    <bltx:ids>
      <bltx:key>alias1</bltx:key>
      <bltx:key>alias2</bltx:key>
    </bltx:ids>
\end{lstlisting}
%
this corresponds to the \bibtex\ format

\begin{lstlisting}[style=bibtex]{}
@book{key1,
  IDS = {alias1,alias2}
}
\end{lstlisting}

\subsubsection{Names}

Name specifications in \biblatexml\ are somewhat more complex in order to generalise the name handling abilities of \biblatex. The user has to be more explicit about the name parts and this allows a much great scope for the handling of different types of names and name parts. A name in \biblatexml\ format looks like this

\begin{lstlisting}[language=xml]
    <bltx:names type="author" morenames="1" useprefix="true">
      <bltx:name gender="sm">
        <bltx:namepart type="given">
          <bltx:namepart initial="J">John</bltx:namepart>
          <bltx:namepart initial="A">Arthur</bltx:namepart>
        </bltx:namepart>
        <bltx:namepart type="family">Smith</bltx:namepart>
        <bltx:namepart type="prefix" initial="v">von</bltx:namepart>
      </bltx:name>
      <bltx:name useprefix="false">
        <bltx:namepart type="given">
          <bltx:namepart>Raymond</bltx:namepart>
        </bltx:namepart>
        <bltx:namepart type="family">Brown</bltx:namepart>
      </bltx:name>
    </bltx:names>
\end{lstlisting}
%
A name list field is contained in the \bibfield{names} element with the mandatory \bibfield{type} attribute giving the name of the name list. Things to note:

\begin{itemize}
  \item The optional \bibfield{morenames} attribute performs the same task as the \bibtex\ datasource format <and others> string at the end of a name.
  \item Note that optional \opt{useprefix} option can be specified can be specified at the level of a name list or an individual name in the name list. This is impossible with \bibtex\ datasources.
  \item Individual names may have an optional \bibfield{gender} attribute which must be one of those defined in the datamodel <gender> constant list. This is currently not used by standard styles but is available in \biblatex name formats if necessary.
  \item A name list is composed of one or more \bibfield{name} elements.
  \item Each name is composed of name parts of a \bibfield{type} defined by the data model <nameparts> constant.
  \item Each name part may have an option \bibfield{initial} attribute which makes explicit the initial of the name part. If this is not present, \biber\ attempts to automatically determine the initial from the name part.
  \item Name parts may have name parts so that compound names can be handled.
\end{itemize}
%
Ignoring the \biblatexml-only features, a corresponding \bibtex\ format datasource would look like this:

\begin{lstlisting}[style=bibtex]{}
  AUTHOR = {von Smith, John Arthur and Brown, Raymond and others}
\end{lstlisting}

\subsubsection{Lists}
Datasource list fields (see \secref{bib:fld:typ}) can be represented in two ways, depending on whether there is more than one element in the list:

\begin{lstlisting}[language=xml]
    <bltx:publisher>London</bltx:publisher>
    <bltx:location>
      <bltx:item>London</bltx:item>
      <bltx:item>Moscow</bltx:item>
    </bltx:location>
\end{lstlisting}

\subsubsection{Ranges}
Datasource range fields (see \secref{bib:fld:typ}) are represented like this:

\begin{lstlisting}[language=xml]
    <bltx:pages>
      <bltx:item>
        <bltx:start>1</bltx:start>
        <bltx:end>10</bltx:end>
      </bltx:item>
      <bltx:item>
        <bltx:start>30</bltx:start>
        <bltx:end>34</bltx:end>
      </bltx:item>
    </bltx:pages>
\end{lstlisting}
%
A range field is a list of ranges, each with its own \bibfield{item}. A range item has a \bibfield{start} element and an optional \bibfield{end} element, since ranges can be open-ended.

\subsubsection{Dates}
Datasource date fields (see \secref{bib:fld:typ}) can be represented in two ways, depending on whether they constitute a date range:

\begin{lstlisting}[language=xml]
    <bltx:date>1985-04-02</bltx:date>
    <bltx:date type="event">
      <bltx:start>1990-05-16</bltx:start>
      <bltx:end>1990-05-17</bltx:end>
    </bltx:date>
\end{lstlisting}
%
The \bibfield{type} attribute on a date element corresponds to a particular type of date defined in the data model.

\subsubsection{Related Entries}

Related entries are specified as follows:
\begin{lstlisting}[language=xml]
    <bltx:related>
      <bltx:item type="reprint"
                 ids="rel1,rel2"
                 string="Somestring"
                 options="skipbiblist"/>
    </bltx:related>
\end{lstlisting}
%
This corresponds to the \bibtex\ format:

\begin{lstlisting}[style=bibtex]{}
@book{key1,
  RELATED         = {rel2,rel2},
  RELATEDTYPE     = {reprint},
  RELATEDSTRING   = {Somestring},
  RELATEDOPTIONS  = {skipbiblist}
}
\end{lstlisting}
%
As per \secref{aut:ctm:rel}, the \bibfield{string} and \bibfield{options} attributes are optional.

\section{选项范围 Option Scope}
\label{apx:opt}

The following table provides an overview of the scope (global\slash per-type\slash per-entry) of various package options.

\begingroup
\tablesetup
\let\+\tickmarkyes
\let\_\tickmarkno
\begin{longtable}[l]{@{}%
	V{0.4\textwidth}@{}%
	C{0.15\textwidth}@{}%
	C{0.15\textwidth}@{}%
	C{0.15\textwidth}@{}%
	C{0.15\textwidth}@{}}
\toprule
\multicolumn{1}{@{}H}{Option} &
\multicolumn{4}{@{}H}{Scope} \\
\cmidrule{2-5}
& \multicolumn{1}{@{}H}{Load-time} &
\multicolumn{1}{@{}H}{Global} &
\multicolumn{1}{@{}H}{Per-type} &
\multicolumn{1}{@{}H}{Per-entry} \\
\cmidrule(r){1-1}\cmidrule(r){2-2}\cmidrule(r){3-3}\cmidrule(r){4-4}\cmidrule{5-5}
\endhead
\bottomrule
\endfoot
abbreviate	&\+&\+&\_&\_\\
alldates	&\+&\+&\_&\_\\
alldatesusetime	&\+&\+&\_&\_\\
alltimes	&\+&\+&\_&\_\\
arxiv		&\+&\+&\_&\_\\
autocite	&\+&\+&\_&\_\\
autopunct	&\+&\+&\_&\_\\
autolang		&\+&\+&\_&\_\\
backend		&\+&\_&\_&\_\\
backref		&\+&\+&\_&\_\\
backrefsetstyle	&\+&\+&\_&\_\\
backrefstyle	&\+&\+&\_&\_\\
bibencoding	&\+&\+&\_&\_\\
bibstyle	&\+&\_&\_&\_\\
bibwarn		&\+&\+&\_&\_\\
block		&\+&\+&\_&\_\\
citecounter	&\+&\+&\_&\_\\
citereset	&\+&\+&\_&\_\\
citestyle	&\+&\_&\_&\_\\
citetracker	&\+&\+&\_&\_\\
clearlang	&\+&\+&\_&\_\\
datamodel &\+&\_&\_&\_\\
dataonly	&\_&\_&\+&\+\\
date		&\+&\+&\_&\_\\
labeldate		&\+&\+&\_&\_\\
$<$datetype$>$date		&\+&\+&\_&\_\\
dateabbrev	&\+&\+&\_&\_\\
datecirca &\+&\+&\_&\_\\
dateera &\+&\+&\_&\_\\
dateerauto &\+&\+&\_&\_\\
dateuncertain &\+&\+&\_&\_\\
datezeros	&\+&\+&\_&\_\\
defernumbers	&\+&\+&\_&\_\\
doi		&\+&\+&\_&\_\\ % style
eprint		&\+&\+&\_&\_\\ % style
$<$namepart$>$inits	&\+&\+&\_&\_\\
gregorianstart &\+&\+&\_&\_\\
hyperref	&\+&\+&\_&\_\\
ibidtracker	&\+&\+&\_&\_\\
idemtracker	&\+&\+&\_&\_\\
indexing	&\+&\+&\+&\+\\
isbn		&\+&\+&\_&\_\\ % style
julian &\+&\+&\_&\_\\
labelalpha	&\+&\+&\+&\_\\
labelnamefield  &\_&\_&\_&\+\\
labelnumber	&\+&\+&\+&\_\\
labeltitle	&\+&\+&\+&\_\\
labeltitlefield &\_&\_&\_&\+\\
labeltitleyear	&\+&\+&\+&\_\\
labeldateparts	&\+&\+&\+&\_\\
labeltime	&\+&\+&\_&\_\\
labeldateusetime	&\+&\+&\_&\_\\
$<$datetype$>$time	&\+&\+&\_&\_\\
$<$datetype$>$dateusetime	&\+&\+&\_&\_\\
language	&\+&\+&\_&\_\\
loadfiles	&\+&\+&\_&\_\\
loccittracker	&\+&\+&\_&\_\\
maxalphanames	&\+&\+&\+&\+\\
maxbibnames	&\+&\+&\+&\+\\
maxcitenames	&\+&\+&\+&\+\\
maxitems	&\+&\+&\+&\+\\
maxnames	&\+&\+&\+&\+\\
maxparens	&\+&\+&\_&\_\\
mcite		&\+&\_&\_&\_\\
minalphanames	&\+&\+&\+&\+\\
minbibnames	&\+&\+&\+&\+\\
mincitenames	&\+&\+&\+&\+\\
mincrossrefs	&\+&\+&\_&\_\\
minxrefs	&\+&\+&\_&\_\\
minitems	&\+&\+&\+&\+\\
minnames	&\+&\+&\+&\+\\
natbib		&\+&\_&\_&\_\\
noinherit		&\_&\_&\_&\+\\
notetype	&\+&\+&\_&\_\\
opcittracker	&\+&\+&\_&\_\\
openbib		&\+&\+&\_&\_\\
pagetracker	&\+&\+&\_&\_\\
parentracker	&\+&\+&\_&\_\\
punctfont	&\+&\+&\_&\_\\
refsection	&\+&\+&\_&\_\\
refsegment	&\+&\+&\_&\_\\
safeinputenc	&\+&\+&\_&\_\\
seconds	&\+&\+&\_&\_\\
singletitle	&\+&\+&\+&\_\\
skipbib		&\_&\_&\+&\+\\
skipbiblist	&\_&\_&\+&\+\\
skiplab		&\_&\_&\+&\+\\
sortcase	&\+&\+&\_&\_\\
sortcites	&\+&\+&\_&\_\\
sorting		&\+&\+&\_&\_\\
sortnamekeyscheme 	&\_&\_&\_&\+\\
sortlocale	&\+&\+&\_&\_\\
sortlos		&\+&\+&\_&\_\\
sortupper	&\+&\+&\_&\_\\
style		&\+&\_&\_&\_\\
terseinits	&\+&\+&\_&\_\\
texencoding	&\+&\+&\_&\_\\
timezeros	&\+&\+&\_&\_\\
timezones	&\+&\+&\_&\_\\
uniquelist	&\+&\+&\+&\+\\
uniquename	&\+&\+&\+&\+\\
uniquetitle	&\+&\+&\+&\_\\
uniquebaretitle	&\+&\+&\+&\_\\
uniquework	&\+&\+&\+&\_\\
uniqueprimaryauthor	&\+&\+&\_&\_\\
url		&\+&\+&\_&\_\\
useprefix	&\+&\+&\+&\+\\
use$<$name$>$	&\+&\+&\+&\+\\
\end{longtable}
\endgroup

\section{更新历史}
%\section{Revision History}
\label{apx:log}

This revision history is a list of changes relevant to users of this package. Changes of a more technical nature which do not affect the user interface or the behavior of the package are not included in the list. More technical details are to be found in the \file{CHANGES.org} file. The numbers on the right indicate the relevant section of this manual.

\begin{changelog}

\begin{release}{3.7}{2016-12-08}
\item Corrected default for \cmd{bibdateeraprefix}\see{aut:fmt:lng}
\item Added \cmd{DeclareSortInclusion}\see{aut:ctm:srt}
\item Added \cmd{relateddelim$<$relatedtype$>$}\see{use:fmt:fmt}
\end{release}

\begin{release}{3.6}{2016-09-15}
\item Corrected some documentation and fixed a bug with labeldate
  localisation strings.
\end{release}

\begin{release}{3.5}{2016-09-10}
\item Added \cmd{ifuniquebaretitle} test\see{aut:aux:tst}
\item Documented \cmd{labelnamesource} and \cmd{labeltitlesource}\see{aut:bbx:fld:gen}
\item Added \cmd{bibdaterangesep}\see{use:fmt:lng}
\item Added \opt{refsection} option to \cmd{DeclareSourcemap}\see{aut:ctm:map}
\item Added \opt{suppress} option to inheritance specifications\see{aut:ctm:ref}
\item Added \cmd{ifuniquework}\see{aut:aux:tst}
\item Changed \cmd{DeclareStyleSourcemap} so that it can be used multiple times\see{aut:ctm:map}
\item Added \cmd{forcezerosy} and \cmd{forcezerosmdt}\see{aut:fmt:ich}
\item Changed \cmd{mkdatezeros} to \cmd{mkyearzeros}, \cmd{mkmonthszeros}
  and \cmd{mkdayzeros}\see{aut:fmt:ich}
\item Added \bibfield{namehash} and \bibfield{fullhash} for all name list fields\see{aut:bbx:fld:gen}
\item Generalised \opt{giveninits} option to all nameparts\see{use:opt:pre:int}
\item Added \opt{inits} option to \cmd{DeclareSortingNamekeyScheme}\see{aut:ctm:srt}
\item Added \cmd{DeclareLabelalphaNameTemplate}\see{aut:ctm:lab}
\item Added full \acr{EDTF} Levels 0 and 1 compliance for parsing and printing times\see{bib:use:dat}
\item Changed dates to be fully \acr{EDTF} Levels 0 and 1 compliant. Associated tests and localisation strings\see{bib:use:dat}
\item Added \opt{timezeros}\see{use:opt:pre:gen}
\item Added \opt{mktimezeros}\see{aut:fmt:ich}
\item Changed \opt{iso8601} to \opt{edtf}\see{use:opt:pre:gen}
\item Added \cmd{DeclareUniquenameTemplate}\see{aut:cav:amb}
\item Removed experimental RIS support
\item \opt{sortnamekeyscheme} and \opt{useprefix} can be now be set per-namelist and per-name for
  \bibtex datasources\see{aut:ctm:srt}
\item Added \cmd{DeclareDelimcontextAlias}\see{use:fmt:csd}
\item Added Estonian localisation (Benson Muite)
\item Reference contexts may now be named\see{use:bib:context}
\item Added \opt{notfield} step in Sourcemaps\see{aut:ctm:map}
\end{release}

\begin{release}{3.4}{2016-05-10}
\item Added \cmd{ifcrossrefsource} and \cmd{ifxrefsource}\see{aut:aux:tst}
\item Added data annotation feature\see{use:annote}
\item Added package option \opt{minxrefs}\see{use:opt:pre:gen}
\item Added \cmd{ifuniqueprimaryauthor} and associated global option\see{aut:aux:tst}
\item Added \cmd{DeprecateField}, \cmd{DeprecateList} and \cmd{DeprecateName}\see{aut:bib:dat}
\item Added \cmd{ifcaselang}\see{aut:aux:tst}
\item Added \cmd{DeclareSortTranslit}\see{aut:ctm:srt}
\item Added \opt{uniquetitle} test\see{aut:aux:tst}
\item Added \cmd{namelabeldelim}\see{use:fmt:fmt}
\item New starred variants of the \cmd{assignrefcontext*} macros\see{use:bib:context}
\item New context-sensitive delimiter interface\see{use:fmt:csd}
\item Moved \opt{prefixnumbers} option to \cmd{newrefcontext} and renamed to \opt{labelprefix}\see{use:bib:context}
\item Added \cmd{DeclareDatafieldSet}\see{aut:ctm:dsets}
\end{release}

\begin{release}{3.3}{2016-03-01}
\item New macros for auto-assignment of refcontexts\see{use:bib:context}
\item Schema documentation for \biblatexml\see{apx:biblatexml}
\item Sourcemapping documentation and examples for \biblatexml\see{aut:ctm:map}
\item Changes for name formats to generalise available name parts\see{aut:bib:fmt}
\item \opt{useprefix} can now be specified per-namelist and per-name in \biblatexml datasources
\item New sourcemapping options for creating new entries dynamically and looping over map steps\see{aut:ctm:map}
\item Added \opt{noalphaothers} and enhanced name range selection in \cmd{DeclareLabelalphaTemplate}\see{aut:ctm:lab}
\item Added \cmd{DeclareDatamodelConstant}\see{aut:ctm:dm}
\item Renamed \opt{firstinits} and \opt{sortfirstinits}
\item Added \cmd{DeclareSortingNamekeyScheme}\see{aut:ctm:srt}
\item Removed messy experimental endnote and zoterordf support for \biber
\item Added \cmd{nonameyeardelim}\see{use:fmt:fmt}
\item Added \cmd{extpostnotedelim}\see{use:fmt:fmt}
\end{release}

\begin{release}{3.2}{2015-12-28}
\item Added \opt{pstrwidth} and \opt{pcompound} to \cmd{DeclareLabelalphaTemplate}\see{aut:ctm:lab}
\item Added \cmd{AtEachCitekey}\see{aut:fmt:hok}
\end{release}

\begin{release}{3.1}{2015-09}
\item Added \cmd{DeclareNolabel}\see{aut:ctm:lab}
\item Added \cmd{DeclareNolabelwidthcount}\see{aut:ctm:lab}
\end{release}

\begin{release}{3.0}{2015-04-20}
\item Improved Danish (Jonas Nyrup) and Spanish (ludenticus) translations
\item \bibfield{labelname} and \bibfield{labeltitle} are now resolved by \biblatex instead of \biber for more flexibility and future extensibility
\item New \cmd{entryclone} sourcemap verb for cloning entries during sourcemapping\see{aut:ctm:map}
\item New \cmd{pernottype} negated per-type sourcemap verb\see{aut:ctm:map}
\item New range calculation command \cmd{frangelen}\see{aut:aux:msc}
\item New bibliography context functionality\see{use:bib:context}
\item Name lists in the data model now automatically create internals for \cmd{ifuse$<$name$>$} tests and booleans\see{use:opt:bib:hyb} and \see{aut:aux:tst}
\end{release}

\begin{release}{2.9a}{2014-06-25}
\item \texttt{resetnumbers} now allows passing a number to reset to\see{use:bib:bib}
\end{release}

\begin{release}{2.9}{2014-02-25}
\item Generalised shorthands facility\see{use:bib:biblist}
\item Sorting locales can now be defined as part of a sorting scheme\see{aut:ctm:srt}
\item Added \bibfield{sortinithash}\see{aut:bbx:fld:gen}
\item Added Slovene localisation (Tea Tušar and Bogdan Filipič)
\item Added \cmd{mkbibitalic}\see{aut:fmt:ich}
\item Recommend \texttt{begentry} and \texttt{finentry} bibliography macros\see{aut:bbx:drv}
\end{release}

\begin{release}{2.8a}{2013-11-25}
\item Split option \opt{language=auto} into \opt{language=autocite} and \opt{language=autobib}\see{use:opt:pre:gen}
\end{release}

\begin{release}{2.8}{2013-10-21}
\item New \bibfield{langidopts}\see{bib:fld:spc}
\item \bibfield{hyphenation} field renamed to \bibfield{langid}\see{bib:fld:spc}
\item \sty{polyglossia} support
\item Renamed \opt{babel} option to \opt{autolang}\see{use:opt:pre:gen}
\item Corrected Dutch localisation
\item Added \opt{datelabel=year} option\see{use:opt:pre:gen}
\item Added \bibfield{datelabelsource} field\see{aut:bbx:fld:gen}
\end{release}

\begin{release}{2.7a}{2013-07-14}
\item Bugfix - respect maxnames and uniquelist in \cmd{finalandsemicolon}
\item Corrected French localisation
\end{release}

\begin{release}{2.7}{2013-07-07}
\item Added field \bibfield{eventtitleaddon} to default datamodel and styles\see{bib:fld:dat}
\item Added \cmd{ifentryinbib}, \cmd{iffirstcitekey} and \cmd{iflastcitekey}\see{aut:aux:tst}
\item Added \bibfield{postpunct} special field, documented \bibfield{multiprenote} and \bibfield{multipostnote} special fields\see{aut:cbx:fld}
\item Added \cmd{UseBibitemHook}, \cmd{AtEveryMultiCite}, \cmd{AtNextMultiCite}, \cmd{UseEveryCiteHook}, \cmd{UseEveryCitekeyHook}, \cmd{UseEveryMultiCiteHook}, \cmd{UseNextCiteHook}, \cmd{UseNextCitekeyHook}, \cmd{UseNextMultiCiteHook}, \cmd{DeferNextCitekeyHook}\see{aut:fmt:hok}
\item Fixed \cmd{textcite} and related commands in the numeric and verbose styles\see{use:cit:cbx}
\item Added multicite variants of \cmd{volcite} and related commands\see{use:cit:spc}
\item Added \cmd{finalandsemicolon}\see{use:fmt:lng}
\item Added citation delimiter \cmd{textcitedelim} for \cmd{textcite} and related commands to styles\see{aut:fmt:fmt}
\item Updated Russian localisation (Oleg Domanov)
\item Fixed Brazilian and Finnish localisation
\end{release}

\begin{release}{2.6}{2013-04-30}
\item Added \cmd{printunit}\see{aut:pct:new}
\item Added field \bibfield{clonesourcekey}\see{aut:bbx:fld:gen}
\item New options for \cmd{DeclareLabelalphaTemplate}\see{aut:ctm:lab}
\item Added \cmd{DeclareLabeldate} and retired \cmd{DeclareLabelyear}\see{aut:ctm:fld}
\item Added \texttt{nodate} localisation string\see{aut:lng:key:msc}
\item Added \cmd{rangelen}\see{aut:aux:msc}
\item Added starred variants of \cmd{citeauthor} and \cmd{Citeauthor}\see{use:cit:txt}
\item Restored original \texttt{url} format. Added \texttt{urlfrom} localisation key\see{aut:lng:key:lab}
\item Added \cmd{AtNextBibliography}\see{aut:fmt:hok}
\item Fixed related entry processing to allow nested and cyclic related entries
\item Added Croatian localisation (Ivo Pletikosić)
\item Added Polish localisation (Anastasia Kandulina, Yuriy Chernyshov)
\item Fixed Catalan localisation
\item Added smart ``of'' for titles to Catalan and French localisation
\item Misc bug fixes
\end{release}

\begin{release}{2.5}{2013-01-10}
\item Made \texttt{url} work as a localisation string, defaulting to previously hard-coded value <\textsc{URL}>.
\item Changed some \biber\ option names to cohere with \biber\ 1.5.
\item New sourcemap step for conditionally removing entire entries\see{aut:ctm:map}
\item Updated Catalan localisation (Sebastià Vila-Marta)
\end{release}

\begin{release}{2.4}{2012-11-28}
\item Added \bibfield{relatedoptions} field\see{aut:ctm:rel}
\item Added \cmd{DeclareStyleSourcemap}\see{aut:ctm:map}
\item Renamed \cmd{DeclareDefaultSourcemap} to \cmd{DeclareDriverSourcemap}\see{aut:ctm:map}
\item Documented \cmd{DeclareFieldInputHandler}, \cmd{DeclareListInputHandler} and \cmd{DeclareNameInputHandler}.
\item Added Czech localisation (Michal Hoftich)
\item Updated Catalan localisation (Sebastià Vila-Marta)
\end{release}

\begin{release}{2.3}{2012-11-01}
\item Better detection of situations which require a \biber\ or \LaTeX\ re-run
\item New append mode for \cmd{DeclareSourcemap} so that fields can be combined\see{aut:ctm:map}
\item Extended auxiliary indexing macros
\item Added support for plural localisation strings with \bibfield{relatedtype}\see{aut:ctm:rel}
\item Added \cmd{csfield} and \cmd{usefield}\see{aut:aux:dat}
\item Added starred variant of \cmd{usebibmacro}\see{aut:aux:msc}
\item Added \cmd{ifbibmacroundef}, \cmd{iffieldformatundef}, \cmd{iflistformatundef}
  and \cmd{ifnameformatundef}\see{aut:aux:msc}
\item Added Catalan localisation (Sebastià Vila-Marta)
\item Misc bug fixes
\end{release}

\begin{release}{2.2}{2012-08-17}
\item Misc bug fixes
\item Added \cmd{revsdnamepunct}\see{use:fmt:fmt}
\item Added \cmd{ifterseinits}\see{aut:aux:tst}
\end{release}

\begin{release}{2.1}{2012-08-01}
\item Misc bug fixes
\item Updated Norwegian localisation (Håkon Malmedal)
\item Increased data model auto-loading possibilities\see{aut:ctm:dm}
\end{release}

\begin{release}{2.0}{2012-07-01}
\item Misc bug fixes
\item Generalised \opt{singletitle} test a little\see{aut:aux:tst}
\item Added new special field \bibfield{extratitleyear}\see{aut:bbx:fld}
\item Customisable data model\see{aut:ctm:dm}
\item Added \cmd{DeclareDefaultSourcemap}\see{aut:ctm:map}
\item Added \opt{labeltitle} option\see{use:opt:pre:int}
\item Added new special field \bibfield{extratitle}\see{aut:bbx:fld}
\item Made special field \bibfield{labeltitle} customisable\see{aut:bbx:fld}
\item Removed field \bibfield{reprinttitle}\see{use:rel}
\item Added related entry feature\see{use:rel}
\item Added \cmd{DeclareNoinit}\see{aut:ctm:noinit}
\item Added \cmd{DeclareNosort}\see{aut:ctm:nosort}
\item Added \opt{sorting} option for \cmd{printbibliography} and \cmd{printshorthands}\see{use:bib:bib}
\item Added \texttt{ids} field for citekey aliasing\see{bib:fld}
\item Added \opt{sortfirstinits} option\see{use:opt:pre:int}
\item Added data stream modification feature\see{aut:ctm:map}
\item Added customisable labels feature\see{aut:ctm:lab}
\item Added \cmd{citeyear*} and \cmd{citedate*}\see{use:cit:txt}
\end{release}

% \begin{release}{1.7}{2011-11-13}
% \item Added \texttt{xdata} containers\see{use:use:xdat}
% \item Added special entry type \bibfield{xdata}\see{bib:typ:blx}
% \item Added special field \bibfield{xdata}\see{bib:fld:spc}
% \item Support \opt{maxnames}/\opt{minnames} globally/per-type/per-entry\see{use:opt:pre:gen}
% \item Support \opt{maxbibnames}/\opt{minbibnames} globally/per-type/per-entry\see{use:opt:pre:gen}
% \item Support \opt{maxcitenames}/\opt{mincitenames} globally/per-type/per-entry\see{use:opt:pre:gen}
% \item Support \opt{maxitems}/\opt{minitems} globally/per-type/per-entry\see{use:opt:pre:gen}
% \item Support \opt{maxalphanames}/\opt{minalphanames} globally/per-type/per-entry\see{use:opt:pre:int}
% \item Support \opt{uniquelist} globally/per-type/per-entry\see{use:opt:pre:int}
% \item Support \opt{uniquename} globally/per-type/per-entry\see{use:opt:pre:int}
% \item Added \cmd{textcite} and \cmd{textcites} to all \texttt{verbose} citation styles\see{use:xbx:cbx}
% \item Added special field formats \texttt{date}, \texttt{urldate}, \texttt{origdate}, \texttt{eventdate}\see{aut:fmt:ich}
% \item Added \cmd{mkcomprange*}\see{aut:aux:msc}
% \item Added \cmd{mkfirstpage*}\see{aut:aux:msc}
% \item Added \cmd{volcitedelim}\see{aut:fmt:fmt}
% \item Added missing documentation for \cmd{ifentrytype}\see{aut:aux:tst}
% \item Added \cmd{mkbibneutord}\see{use:fmt:lng}
% \item Added counter \cnt{biburlnumpenalty}\see{aut:fmt:len}
% \item Added counter \cnt{biburlucpenalty}\see{aut:fmt:len}
% \item Added counter \cnt{biburllcpenalty}\see{aut:fmt:len}
% \item Added localisation keys \texttt{book}, \texttt{part}, \texttt{issue}, \texttt{forthcoming}\see{aut:lng:key}
% \item Added some \texttt{lang...} and \texttt{from...} localisation keys\see{aut:lng:key}
% \item Expanded documentation\see{apx:opt}
% \item Added support for Russian (Oleg Domanov)
% \item Updated support for Dutch (Pieter Belmans)
% \item Fixed compatibility issue with \sty{textcase} package
% \item Fixed some bugs
% \end{release}

% \begin{release}{1.6}{2011-07-29}
% \item Added special field \bibfield{sortshorthand}\see{bib:fld:spc}
% \item Revised options \opt{maxnames}/\opt{minnames}\see{use:opt:pre:gen}
% \item Options \opt{maxcitenames}/\opt{mincitenames} now supported by backend\see{use:opt:pre:gen}
% \item Options \opt{maxbibnames}/\opt{minbibnames} now supported by backend\see{use:opt:pre:gen}
% \item Added options \opt{maxalphanames}/\opt{minalphanames}\see{use:opt:pre:int}
% \item Removed local options \opt{maxnames}/\opt{minnames} from \cmd{printbibliography}\see{use:bib:bib}
% \item Removed local options \opt{maxitems}/\opt{minitems} from \cmd{printbibliography}\see{use:bib:bib}
% \item Removed local options \opt{maxnames}/\opt{minnames} from \cmd{bibbysection}\see{use:bib:bib}
% \item Removed local options \opt{maxitems}/\opt{minitems} from \cmd{bibbysection}\see{use:bib:bib}
% \item Removed local options \opt{maxnames}/\opt{minnames} from \cmd{bibbysegment}\see{use:bib:bib}
% \item Removed local options \opt{maxitems}/\opt{minitems} from \cmd{bibbysegment}\see{use:bib:bib}
% \item Removed local options \opt{maxnames}/\opt{minnames} from \cmd{bibbycategory}\see{use:bib:bib}
% \item Removed local options \opt{maxitems}/\opt{minitems} from \cmd{bibbycategory}\see{use:bib:bib}
% \item Removed local options \opt{maxnames}/\opt{minnames} from \cmd{printshorthands}\see{use:bib:biblist}
% \item Removed local options \opt{maxitems}/\opt{minitems} from \cmd{printshorthands}\see{use:bib:biblist}
% \item Added special field format \bibfield{volcitevolume}\see{use:cit:spc}
% \item Added special field format \bibfield{volcitepages}\see{use:cit:spc}
% \item Added special field \bibfield{hash}\see{aut:bbx:fld:gen}
% \item Added \cmd{mkcomprange}\see{aut:aux:msc}
% \item Added \cmd{mkfirstpage}\see{aut:aux:msc}
% \item Removed \cmd{mkpagefirst}\see{aut:aux:msc}
% \item Fixed some bugs
% \end{release}

% \begin{release}{1.5a}{2011-06-17}
% \item Fixed some bugs
% \end{release}

% \begin{release}{1.5}{2011-06-08}
% \item Added option \kvopt{uniquename}{mininit/minfull}\see{use:opt:pre:int}
% \item Added option \kvopt{uniquelist}{minyear}\see{use:opt:pre:int}
% \item Updated documentation of \cnt{uniquename} counter\see{aut:aux:tst}
% \item Updated documentation of \cnt{uniquelist} counter\see{aut:aux:tst}
% \item Expanded documentation for \opt{uniquename/uniquelist} options\see{aut:cav:amb}
% \item Added editorial role \texttt{reviser}\see{bib:use:edr}
% \item Added localisation keys \texttt{reviser}, \texttt{revisers}, \texttt{byreviser}\see{aut:lng:key}
% \item Added bibliography heading \texttt{none}\see{use:bib:hdg}
% \item Fixed some \sty{memoir} compatibility issues
% \end{release}

% \begin{release}{1.4c}{2011-05-12}
% \item Fixed some bugs
% \end{release}

% \begin{release}{1.4b}{2011-04-12}
% \item Fixed some bugs
% \end{release}

% \begin{release}{1.4a}{2011-04-06}
% \item Enable \opt{uniquename} and \opt{uniquelist} in all \texttt{authortitle} styles\see{use:xbx:cbx}
% \item Enable \opt{uniquename} and \opt{uniquelist} in all \texttt{authoryear} styles\see{use:xbx:cbx}
% \item Fixed some bugs
% \end{release}

% \begin{release}{1.4}{2011-03-31}
% \item Added package option \opt{uniquelist}\see{use:opt:pre:int}
% \item Added special counter \cnt{uniquelist}\see{aut:aux:tst}
% \item Revised and improved package option \opt{uniquename}\see{use:opt:pre:int}
% \item Revised and improved special counter \cnt{uniquename}\see{aut:aux:tst}
% \item Added \cmd{bibnamedelimi}\see{use:fmt:fmt}
% \item Added \cmd{bibindexnamedelima}\see{use:fmt:fmt}
% \item Added \cmd{bibindexnamedelimb}\see{use:fmt:fmt}
% \item Added \cmd{bibindexnamedelimc}\see{use:fmt:fmt}
% \item Added \cmd{bibindexnamedelimd}\see{use:fmt:fmt}
% \item Added \cmd{bibindexnamedelimi}\see{use:fmt:fmt}
% \item Added \cmd{bibindexinitperiod}\see{use:fmt:fmt}
% \item Added \cmd{bibindexinitdelim}\see{use:fmt:fmt}
% \item Added \cmd{bibindexinithyphendelim}\see{use:fmt:fmt}
% \item Fixed conflict with some \acr{AMS} classes
% \end{release}

% \begin{release}{1.3a}{2011-03-18}
% \item Fixed some bugs
% \end{release}

% \begin{release}{1.3}{2011-03-14}
% \item Support \bibtype{thesis} with \bibfield{isbn}\see{bib:typ:blx}
% \item Updated \opt{terseinits} option\see{use:opt:pre:gen}
% \item Allow macros in argument to \cmd{addbibresource} and friends\see{use:bib:res}
% \item Allow macros in argument to \cmd{bibliography}\see{use:bib:res}
% \item Introducing experimental support for Zotero \acr{RDF}/\acr{XML}\see{use:bib:res}
% \item Introducing experimental support for EndNote \acr{XML}\see{use:bib:res}
% \item Added option \opt{citecounter}\see{use:opt:pre:int}
% \item Added \cnt{citecounter}\see{aut:aux:tst}
% \item Added \cmd{smartcite} and \cmd{Smartcite}\see{use:cit:cbx}
% \item Added \cmd{smartcites} and \cmd{Smartcites}\see{use:cit:mlt}
% \item Added \cmd{svolcite} and \cmd{Svolcite}\see{use:cit:spc}
% \item Added \cmd{bibnamedelima}\see{use:fmt:fmt}
% \item Added \cmd{bibnamedelimb}\see{use:fmt:fmt}
% \item Added \cmd{bibnamedelimc}\see{use:fmt:fmt}
% \item Added \cmd{bibnamedelimd}\see{use:fmt:fmt}
% \item Added \cmd{bibinitperiod}\see{use:fmt:fmt}
% \item Added \cmd{bibinitdelim}\see{use:fmt:fmt}
% \item Added \cmd{bibinithyphendelim}\see{use:fmt:fmt}
% \item Expanded documentation\see{use:cav:nam}
% \item Added \prm{position} parameter \texttt{f} to \cmd{DeclareAutoCiteCommand}\see{aut:cbx:cbx}
% \end{release}

% \begin{release}{1.2a}{2011-02-13}
% \item Fix in \cmd{mkbibmonth}\see{aut:fmt:ich}
% \end{release}

% \begin{release}{1.2}{2011-02-12}
% \item Added entry type \bibtype{mvbook}\see{bib:typ:blx}
% \item Added entry type \bibtype{mvcollection}\see{bib:typ:blx}
% \item Added entry type \bibtype{mvproceedings}\see{bib:typ:blx}
% \item Added entry type \bibtype{mvreference}\see{bib:typ:blx}
% \item Introducing remote resources\see{use:bib:res}
% \item Introducing experimental \acr{RIS} support\see{use:bib:res}
% \item Added \cmd{addbibresource}\see{use:bib:res}
% \item \cmd{bibliography} now deprecated\see{use:bib:res}
% \item \cmd{bibliography*} replaced by \cmd{addglobalbib}\see{use:bib:res}
% \item Added \cmd{addsectionbib}\see{use:bib:res}
% \item Updated and expanded documentation\see{bib:cav:ref}
% \item Introducing smart \bibfield{crossref} data inheritance\see{bib:cav:ref:bbr}
% \item Introducing \bibfield{crossref} configuration interface\see{aut:ctm:ref}
% \item Added \cmd{DefaultInheritance}\see{aut:ctm:ref}
% \item Added \cmd{DeclareDataInheritance}\see{aut:ctm:ref}
% \item Added \cmd{ResetDataInheritance}\see{aut:ctm:ref}
% \item Added \cmd{ifkeyword}\see{aut:aux:tst}
% \item Added \cmd{ifentrykeyword}\see{aut:aux:tst}
% \item Added \cmd{ifcategory}\see{aut:aux:tst}
% \item Added \cmd{ifentrycategory}\see{aut:aux:tst}
% \item Added \cmd{ifdriver}\see{aut:aux:tst}
% \item Added \cmd{forcsvfield}\see{aut:aux:msc}
% \item Extended \cmd{mkpageprefix}\see{aut:aux:msc}
% \item Extended \cmd{mkpagetotal}\see{aut:aux:msc}
% \item Extended \cmd{mkpagefirst}\see{aut:aux:msc}
% \item Added localisation key \texttt{inpreparation}\see{aut:lng:key}
% \item Rearranged manual slightly, moving some tables to the appendix
% \end{release}

% \begin{release}{1.1b}{2011-02-04}
% \item Added option \opt{texencoding}\see{use:opt:pre:gen}
% \item Added option \opt{safeinputenc}\see{use:opt:pre:gen}
% \item Expanded documentation\see{bib:cav:enc:enc}
% \item Improved \opt{mergedate} option of bibliography style \texttt{authoryear}\see{use:xbx:bbx}
% \item Removed \opt{pass} option of \cmd{DeclareSortingScheme}\see{aut:ctm:srt}
% \item Fixed some bugs
% \end{release}

% \begin{release}{1.1a}{2011-01-08}
% \item Added unsupported entry type \bibtype{bibnote}\see{bib:typ:ctm}
% \item Added \cmd{bibliography*}\see{use:bib:res}
% \item Fixed some bugs
% \end{release}

% \begin{release}{1.1}{2011-01-05}
% \item Added option \opt{maxbibnames}\see{use:opt:pre:gen}
% \item Added option \opt{minbibnames}\see{use:opt:pre:gen}
% \item Added option \opt{maxcitenames}\see{use:opt:pre:gen}
% \item Added option \opt{mincitenames}\see{use:opt:pre:gen}
% \item Fixed \kvopt{idemtracker}{strict} and \kvopt{idemtracker}{constrict}\see{use:opt:pre:int}
% \item Added option \opt{mergedate} to bibliography style \texttt{authoryear}\see{use:xbx:bbx}
% \item Added support for \opt{prefixnumbers} to bibliography style \texttt{alphabetic}\see{use:xbx:bbx}
% \item Made option \bibfield{useprefix} settable on a per-type basis\see{use:opt:bib}
% \item Made option \bibfield{useauthor} settable on a per-type basis\see{use:opt:bib}
% \item Made option \bibfield{useeditor} settable on a per-type basis\see{use:opt:bib}
% \item Made option \opt{usetranslator} settable on a per-type basis\see{use:opt:bib}
% \item Made option \opt{skipbib} settable on a per-type basis\see{use:opt:bib}
% \item Made option \opt{skiplos} settable on a per-type basis\see{use:opt:bib}
% \item Made option \opt{skiplab} settable on a per-type basis\see{use:opt:bib}
% \item Made option \opt{dataonly} settable on a per-type basis\see{use:opt:bib}
% \item Made option \opt{labelalpha} settable on a per-type basis\see{use:opt:pre:int}
% \item Made option \opt{labelnumber} settable on a per-type basis\see{use:opt:pre:int}
% \item Made option \opt{labelyear} settable on a per-type basis\see{use:opt:pre:int}
% \item Made option \opt{singletitle} settable on a per-type basis\see{use:opt:pre:int}
% \item Made option \opt{uniquename} settable on a per-type basis\see{use:opt:pre:int}
% \item Made option \opt{indexing} settable on a per-type basis\see{use:opt:pre:gen}
% \item Made option \opt{indexing} settable on a per-entry basis\see{use:opt:pre:gen}
% \item Extended \cmd{ExecuteBibliographyOptions}\see{use:cfg:opt}
% \item Added \cmd{citedate}\see{use:cit:txt}
% \item Improved static entry sets\see{use:use:set}
% \item Introducing dynamic entry sets\see{use:use:set}
% \item Added \cmd{defbibentryset}\see{use:bib:set}
% \item Added option \opt{mcite}\see{use:opt:ldt}
% \item Added \sty{mcite}\slash\sty{mciteplus}-like commands\see{use:cit:mct}
% \item Added \cmd{sortalphaothers}\see{use:fmt:fmt}
% \item Extended \cmd{DeclareNameFormat}\see{aut:bib:fmt}
% \item Extended \cmd{DeclareIndexNameFormat}\see{aut:bib:fmt}
% \item Extended \cmd{DeclareListFormat}\see{aut:bib:fmt}
% \item Extended \cmd{DeclareIndexListFormat}\see{aut:bib:fmt}
% \item Extended \cmd{DeclareFieldFormat}\see{aut:bib:fmt}
% \item Extended \cmd{DeclareIndexFieldFormat}\see{aut:bib:fmt}
% \item Added \cmd{DeclareNameFormat*}\see{aut:bib:fmt}
% \item Added \cmd{DeclareIndexNameFormat*}\see{aut:bib:fmt}
% \item Added \cmd{DeclareListFormat*}\see{aut:bib:fmt}
% \item Added \cmd{DeclareIndexListFormat*}\see{aut:bib:fmt}
% \item Added \cmd{DeclareFieldFormat*}\see{aut:bib:fmt}
% \item Added \cmd{DeclareIndexFieldFormat*}\see{aut:bib:fmt}
% \item Introducing configurable sorting schemes
% \item Added \cmd{DeclareSortingScheme}\see{aut:ctm:srt}
% \item Added \cmd{DeclarePresort}\see{aut:ctm:srt}
% \item Added \cmd{DeclareSortExclusion}\see{aut:ctm:srt}
% \item Added \cmd{DeclareLabelname}\see{aut:ctm:fld}
% \item Added \cmd{DeclareLabelyear}\see{aut:ctm:fld}
% \item Improved special field \bibfield{labelname}\see{aut:bbx:fld}
% \item Improved special field \bibfield{labelyear}\see{aut:bbx:fld}
% \item Added \cmd{entrydata*}\see{aut:bib:dat}
% \item Added \cmd{RequireBiber}\see{aut:aux:msc}
% \item Added option \opt{check} to \cmd{printbibliography}\see{use:bib:bib}
% \item Added option \opt{check} to \cmd{printshorthands}\see{use:bib:biblist}
% \item Added \cmd{defbibcheck}\see{use:bib:flt}
% \item Updated support for Portuguese (José Carlos Santos)
% \item Fixed conflict with \sty{titletoc}
% \item Fixed some bugs
% \end{release}

% \begin{release}{1.0}{2010-11-19}
% \item First officially stable release
% \item Renamed option \kvopt{bibencoding}{inputenc} to \kvopt{bibencoding}{auto}\see{use:opt:pre:gen}
% \item Made \kvopt{bibencoding}{auto} the package default\see{use:opt:pre:gen}
% \item Added option \kvopt{backend}{bibtexu}\see{use:opt:pre:gen}
% \item Slightly updated documentation\see{bib:cav:enc}
% \item Updated support for Dutch (Alexander van Loon)
% \item Updated support for Italian (Andrea Marchitelli)
% \end{release}

%\begin{release}{0.9e}{2010-10-09}
%\item Updated and expanded manual\see{bib:cav:enc}
%\item Added option \opt{sortupper}\see{use:opt:pre:gen}
%\item Added option \opt{sortlocale}\see{use:opt:pre:gen}
%\item Added option \opt{backrefsetstyle}\see{use:opt:pre:gen}
%\item Added \cmd{bibpagerefpunct}\see{use:fmt:fmt}
%\item Added \cmd{backtrackertrue} and \cmd{backtrackerfalse}\see{aut:aux:msc}
%\item Disable back reference tracking in \acr{TOC}/\acr{LOT}/\acr{LOF}\see{aut:cav:flt}
%\item Improved back reference tracking for \bibtype{set} entries
%\item Fixed some bugs
%\end{release}
%
%\begin{release}{0.9d}{2010-09-03}
%\item Added workaround for \sty{hyperref} space factor issue
%\item Added workaround for \sty{xkeyval}'s flawed class option inheritance
%\item Added workaround for \sty{fancyvrb}'s flawed brute-force \cmd{VerbatimFootnotes}
%\item Removed option \kvopt{date}{none}\see{use:opt:pre:gen}
%\item Removed option \kvopt{urldate}{none}\see{use:opt:pre:gen}
%\item Removed option \kvopt{origdate}{none}\see{use:opt:pre:gen}
%\item Removed option \kvopt{eventdate}{none}\see{use:opt:pre:gen}
%\item Removed option \kvopt{alldates}{none}\see{use:opt:pre:gen}
%\item Added option \kvopt{date}{iso8601}\see{use:opt:pre:gen}
%\item Added option \kvopt{urldate}{iso8601}\see{use:opt:pre:gen}
%\item Added option \kvopt{origdate}{iso8601}\see{use:opt:pre:gen}
%\item Added option \kvopt{eventdate}{iso8601}\see{use:opt:pre:gen}
%\item Added option \kvopt{alldates}{iso8601}\see{use:opt:pre:gen}
%\end{release}
%
%\begin{release}{0.9c}{2010-08-29}
%\item Added field \bibfield{eprintclass}\see{bib:fld:dat}
%\item Added field alias \bibfield{archiveprefix}\see{bib:fld:als}
%\item Added field alias \bibfield{primaryclass}\see{bib:fld:als}
%\item Updated documentation\see{use:use:epr}
%\item Tweaked package option \kvopt{babel}{other*}\see{use:opt:pre:gen}
%\item Updated support for Brazilian (Mateus Araújo)
%\item Fixed some bugs
%\end{release}
%
%\begin{release}{0.9b}{2010-08-04}
%
%\item New dependency on \sty{logreq} package\see{int:pre:req}
%\item Improved separator masking in literal lists\see{bib:use:and}
%\item Added citation style \texttt{authortitle-ticomp}\see{use:xbx:cbx}
%\item Added option \opt{citepages} to all \texttt{verbose} citation styles\see{use:xbx:cbx}
%\item Added support for prefixes to all \texttt{numeric} citation styles\see{use:xbx:cbx}
%\item Added support for prefixes to all \texttt{numeric} bibliography styles\see{use:xbx:bbx}
%\item Renamed package option \opt{defernums} to \opt{defernumbers}\see{use:opt:pre:gen}
%\item Added option \opt{sortcase}\see{use:opt:pre:gen}
%\item Added option \opt{dateabbrev}\see{use:opt:pre:gen}
%\item Added option \kvopt{date}{none}\see{use:opt:pre:gen}
%\item Added option \kvopt{urldate}{none}\see{use:opt:pre:gen}
%\item Added option \kvopt{origdate}{none}\see{use:opt:pre:gen}
%\item Added option \kvopt{eventdate}{none}\see{use:opt:pre:gen}
%\item Added option \kvopt{alldates}{none}\see{use:opt:pre:gen}
%\item Added option \opt{clearlang}\see{use:opt:pre:gen}
%\item Added option \opt{prefixnumbers} to \cmd{printbibliography}\see{use:bib:bib}
%\item Added option \opt{resetnumbers} to \cmd{printbibliography}\see{use:bib:bib}
%\item Added option \opt{omitnumbers} to \cmd{printbibliography}\see{use:bib:bib}
%\item Added special field \bibfield{prefixnumber}\see{aut:bbx:fld}
%\item Added \cmd{DeclareRedundantLanguages}\see{aut:lng:cmd}
%\item Added support for handles (\acr{HDL}s)\see{use:use:epr}
%\item Extended \cmd{defbibfilter}\see{use:bib:flt}
%\item Added \cmd{nametitledelim}\see{use:fmt:fmt}
%\item Improved \cmd{newbibmacro}\see{aut:aux:msc}
%\item Improved \cmd{renewbibmacro}\see{aut:aux:msc}
%\item Added \cmd{biblstring}\see{aut:str}
%\item Added \cmd{bibsstring}\see{aut:str}
%\item Added \cmd{bibcplstring}\see{aut:str}
%\item Added \cmd{bibcpsstring}\see{aut:str}
%\item Added \cmd{bibuclstring}\see{aut:str}
%\item Added \cmd{bibucsstring}\see{aut:str}
%\item Added \cmd{biblclstring}\see{aut:str}
%\item Added \cmd{biblcsstring}\see{aut:str}
%\item Added \cmd{bibxlstring}\see{aut:str}
%\item Added \cmd{bibxsstring}\see{aut:str}
%\item Added \cmd{mkbibbold}\see{aut:fmt:ich}
%\item Modified and extended log messages\see{bib:cav:ide}
%\item Fixed some bugs
%
%\end{release}
%
%\begin{release}{0.9a}{2010-03-19}
%
%\item Modified citation style \texttt{numeric}\see{use:xbx:cbx}
%\item Modified citation style \texttt{numeric-comp}\see{use:xbx:cbx}
%\item Modified citation style \texttt{numeric-verb}\see{use:xbx:cbx}
%\item Modified citation style \texttt{alphabetic}\see{use:xbx:cbx}
%\item Modified citation style \texttt{alphabetic-verb}\see{use:xbx:cbx}
%\item Modified citation style \texttt{authoryear}\see{use:xbx:cbx}
%\item Modified citation style \texttt{authoryear-comp}\see{use:xbx:cbx}
%\item Modified citation style \texttt{authoryear-ibid}\see{use:xbx:cbx}
%\item Modified citation style \texttt{authoryear-icomp}\see{use:xbx:cbx}
%\item Modified citation style \texttt{authortitle}\see{use:xbx:cbx}
%\item Modified citation style \texttt{authortitle-comp}\see{use:xbx:cbx}
%\item Modified citation style \texttt{authortitle-ibid}\see{use:xbx:cbx}
%\item Modified citation style \texttt{authortitle-icomp}\see{use:xbx:cbx}
%\item Modified citation style \texttt{authortitle-terse}\see{use:xbx:cbx}
%\item Modified citation style \texttt{authortitle-tcomp}\see{use:xbx:cbx}
%\item Modified citation style \texttt{draft}\see{use:xbx:cbx}
%\item Modified citation style \texttt{debug}\see{use:xbx:cbx}
%\item Added option \opt{bibwarn}\see{use:opt:pre:gen}
%\item Added \cmd{printbibheading}\see{use:bib:bib}
%\item Added option \opt{env} to \cmd{printbibliography}\see{use:bib:bib}
%\item Added option \opt{env} to \cmd{printshorthands}\see{use:bib:biblist}
%\item Added \cmd{defbibenvironment}\see{use:bib:hdg}
%\item Removed \env{thebibliography}\see{aut:bbx:bbx}
%\item Removed \env{theshorthands}\see{aut:bbx:bbx}
%\item Removed \cmd{thebibitem}\see{aut:bbx:bbx}
%\item Removed \cmd{thelositem}\see{aut:bbx:bbx}
%\item Updated documentation\see{aut:bbx:bbx}
%\item Updated documentation\see{aut:bbx:env}
%\item Added \cmd{intitlepunct}\see{use:fmt:fmt}
%\item Added option \opt{parentracker}\see{use:opt:pre:int}
%\item Added option \opt{maxparens}\see{use:opt:pre:int}
%\item Added counter \cnt{parenlevel}\see{aut:aux:tst}
%\item Added \cmd{parentext}\see{use:cit:txt}
%\item Added \cmd{brackettext}\see{use:cit:txt}
%\item Improved \cmd{mkbibparens}\see{aut:fmt:ich}
%\item Improved \cmd{mkbibbrackets}\see{aut:fmt:ich}
%\item Added \cmd{bibopenparen} and \cmd{bibcloseparen}\see{aut:fmt:ich}
%\item Added \cmd{bibopenbracket} and \cmd{bibclosebracket}\see{aut:fmt:ich}
%\item Added special field \bibfield{childentrykey}\see{aut:bbx:fld}
%\item Added special field \bibfield{childentrytype}\see{aut:bbx:fld}
%\item Added \cmd{ifnatbibmode}\see{aut:aux:tst}
%\item Added missing documentation of \cmd{ifbibxstring}\see{aut:aux:tst}
%\item Added \cmd{providebibmacro}\see{aut:aux:msc}
%\item Added localisation key \texttt{backrefpage}\see{aut:lng:key}
%\item Added localisation key \texttt{backrefpages}\see{aut:lng:key}
%\item Slightly expanded documentation\see{bib:use:dat}
%\item Slightly expanded documentation\see{aut:bbx:fld:dat}
%\item Added support for Finnish (translations by Hannu Väisänen)
%\item Updated support for Greek (translations by Prokopis)
%
%\end{release}
%
%\begin{release}{0.9}{2010-02-14}
%
%\item Added entry type \bibtype{bookinbook}\see{bib:typ:blx}
%\item Support \bibfield{eventtitle}/\bibfield{eventdate}/\bibfield{venue} in \bibtype{proceedings}\see{bib:typ:blx}
%\item Support \bibfield{eventtitle}/\bibfield{eventdate}/\bibfield{venue} in \bibtype{inproceedings}\see{bib:typ:blx}
%\item Added support for multiple editorial roles\see{bib:use:edr}
%\item Added field \bibfield{editora}\see{bib:fld:dat}
%\item Added field \bibfield{editorb}\see{bib:fld:dat}
%\item Added field \bibfield{editorc}\see{bib:fld:dat}
%\item Added field \bibfield{editoratype}\see{bib:fld:dat}
%\item Added field \bibfield{editorbtype}\see{bib:fld:dat}
%\item Added field \bibfield{editorctype}\see{bib:fld:dat}
%\item Removed field \bibfield{redactor}\see{bib:fld:dat}
%\item Added field \bibfield{pubstate}\see{bib:fld:dat}
%\item Support \bibfield{pubstate} in all entry types\see{bib:typ:blx}
%\item Support full dates in all entry types\see{bib:typ:blx}
%\item Modified and extended date handling\see{bib:use:dat}
%\item Updated documentation\see{bib:use:iss}
%\item Removed field \bibfield{day}\see{bib:fld:dat}
%\item Modified data type of field \bibfield{year}\see{bib:fld:dat}
%\item Extended field \bibfield{date}\see{bib:fld:dat}
%\item Removed field \bibfield{origyear}\see{bib:fld:dat}
%\item Added field \bibfield{origdate}\see{bib:fld:dat}
%\item Added field \bibfield{eventdate}\see{bib:fld:dat}
%\item Removed fields \bibfield{urlday}/\bibfield{urlmonth}/\bibfield{urlyear}\see{bib:fld:dat}
%\item Updated documentation\see{bib:use:dat}
%\item Extended option \opt{date}\see{use:opt:pre:gen}
%\item Extended option \opt{urldate}\see{use:opt:pre:gen}
%\item Added option \opt{origdate}\see{use:opt:pre:gen}
%\item Added option \opt{eventdate}\see{use:opt:pre:gen}
%\item Added option \opt{alldates}\see{use:opt:pre:gen}
%\item Added option \opt{datezeros}\see{use:opt:pre:gen}
%\item Added option \opt{language}\see{use:opt:pre:gen}
%\item Added option \cnt{notetype}\see{use:opt:pre:gen}
%\item Added option \cnt{backrefstyle}\see{use:opt:pre:gen}
%\item Modified option \opt{indexing}\see{use:opt:pre:gen}
%\item Made option \kvopt{hyperref}{auto} the default\see{use:opt:pre:gen}
%\item Added option \kvopt{backend}{biber}\see{use:opt:pre:gen}
%\item Updated documentation\see{bib:cav:enc}
%\item Added option \opt{isbn}\see{use:opt:pre:bbx}
%\item Added option \opt{url}\see{use:opt:pre:bbx}
%\item Added option \opt{doi}\see{use:opt:pre:bbx}
%\item Added option \opt{eprint}\see{use:opt:pre:bbx}
%\item Improved citation style \texttt{authortitle-comp}\see{use:xbx:cbx}
%\item Improved citation style \texttt{authortitle-icomp}\see{use:xbx:cbx}
%\item Improved citation style \texttt{authortitle-tcomp}\see{use:xbx:cbx}
%\item Improved citation style \texttt{authoryear-comp}\see{use:xbx:cbx}
%\item Added citation style \texttt{authoryear-icomp}\see{use:xbx:cbx}
%\item Added citation style \texttt{verbose-trad3}\see{use:xbx:cbx}
%\item Improved bibliography style \texttt{authortitle}\see{use:xbx:bbx}
%\item Improved bibliography style \texttt{authoryear}\see{use:xbx:bbx}
%\item Improved bibliography style \texttt{verbose}\see{use:xbx:bbx}
%\item Added option \opt{title} to \cmd{printbibliography}\see{use:bib:bib}
%\item Added option \opt{title} to \cmd{printshorthands}\see{use:bib:biblist}
%\item Extended \cmd{defbibheading}\see{use:bib:hdg}
%\item Added options \opt{subtype}/\opt{notsubtype} to \cmd{printbibliography}\see{use:bib:biblist}
%\item Added options \opt{subtype}/\opt{notsubtype} to \cmd{printshorthands}\see{use:bib:biblist}
%\item Added test \opt{subtype} to \cmd{defbibfilter}\see{use:bib:flt}
%\item Added option \opt{segment} to \cmd{printshorthands}\see{use:bib:biblist}
%\item Added options \opt{type}/\opt{nottype} to \cmd{printshorthands}\see{use:bib:biblist}
%\item Added options \opt{keyword}/\opt{notkeyword} to \cmd{printshorthands}\see{use:bib:biblist}
%\item Added options \opt{category}/\opt{notcategory} to \cmd{printshorthands}\see{use:bib:biblist}
%\item Added option \opt{filter} to \cmd{printshorthands}\see{use:bib:biblist}
%\item Added \cmd{footcitetext}\see{use:cit:std}
%\item Added \cmd{footcitetexts}\see{use:cit:mlt}
%\item Added \cmd{ftvolcite}\see{use:cit:spc}
%\item Added \cmd{textcites} and \cmd{Textcites}\see{use:cit:mlt}
%\item Added \cmd{nohyphenation}\see{use:fmt:aux}
%\item Added \cmd{textnohyphenation}\see{use:fmt:aux}
%\item Added \cmd{mkpagefirst}\see{aut:aux:msc}
%\item Added \cmd{pagenote} support to \cmd{mkbibendnote}\see{aut:fmt:ich}
%\item Added \cmd{mkbibfootnotetext}\see{aut:fmt:ich}
%\item Added \cmd{mkbibendnotetext}\see{aut:fmt:ich}
%\item Added \cmd{bibfootnotewrapper}\see{aut:fmt:ich}
%\item Added \cmd{bibendnotewrapper}\see{aut:fmt:ich}
%\item Added \cmd{mkdatezeros}\see{aut:fmt:ich}
%\item Added \cmd{stripzeros}\see{aut:fmt:ich}
%\item Added support for \acr{jstor} links\see{use:use:epr}
%\item Added support for PubMed links\see{use:use:epr}
%\item Added support for Google Books links\see{use:use:epr}
%\item Improved \cmd{DeclareBibliographyDriver}\see{aut:bbx:bbx}
%\item Improved \cmd{DeclareBibliographyAlias}\see{aut:bbx:bbx}
%\item Added special fields \bibfield{day}/\bibfield{month}/\bibfield{year}\see{aut:bbx:fld}
%\item Added special fields \bibfield{endday}/\bibfield{endmonth}/\bibfield{endyear}\see{aut:bbx:fld}
%\item Added special fields \bibfield{origday}/\bibfield{origmonth}/\bibfield{origyear}\see{aut:bbx:fld}
%\item Added special fields \bibfield{origendday}/\bibfield{origendmonth}/\bibfield{origendyear}\see{aut:bbx:fld}
%\item Added special fields \bibfield{eventday}/\bibfield{eventmonth}/\bibfield{eventyear}\see{aut:bbx:fld}
%\item Added special fields \bibfield{eventendday}/\bibfield{eventendmonth}/\bibfield{eventendyear}\see{aut:bbx:fld}
%\item Added special fields \bibfield{urlday}/\bibfield{urlmonth}/\bibfield{urlyear}\see{aut:bbx:fld}
%\item Added special fields \bibfield{urlendday}/\bibfield{urlendmonth}/\bibfield{urlendyear}\see{aut:bbx:fld}
%\item Renamed special field \bibfield{labelyear} to \bibfield{extrayear}\see{aut:bbx:fld}
%\item Added new special field \bibfield{labelyear}\see{aut:bbx:fld}
%\item Renamed \cnt{maxlabelyear} to \cnt{maxextrayear}\see{aut:fmt:ilc}
%\item Renamed \cmd{bibdate} to \cmd{printdate}, modified \cmd{printdate}\see{aut:bib:dat}
%\item Added \cmd{printdateextra}\see{aut:bib:dat}
%\item Renamed \cmd{biburldate} to \cmd{printurldate}, modified \cmd{printurldate}\see{aut:bib:dat}
%\item Added \cmd{printorigdate}\see{aut:bib:dat}
%\item Added \cmd{printeventdate}\see{aut:bib:dat}
%\item Added \cmd{bibdatedash}\see{use:fmt:lng}
%\item Added \cmd{mkbibdatelong} and \cmd{mkbibdateshort}\see{use:fmt:lng}
%\item Removed \cmd{bibdatelong} and \cmd{bibdateshort}\see{use:fmt:lng}
%\item Removed \cmd{biburldatelong} and \cmd{biburldateshort}\see{use:fmt:lng}
%\item Added \cmd{ifciteindex}\see{aut:aux:tst}
%\item Added \cmd{ifbibindex}\see{aut:aux:tst}
%\item Added \cmd{iffieldint}\see{aut:aux:tst}
%\item Added \cmd{iffieldnum}\see{aut:aux:tst}
%\item Added \cmd{iffieldnums}\see{aut:aux:tst}
%\item Added \cmd{ifpages}\see{aut:aux:tst}
%\item Added \cmd{iffieldpages}\see{aut:aux:tst}
%\item Added \cmd{DeclarePageCommands} and \cmd{DeclarePageCommands*}\see{aut:aux:msc}
%\item Improved \cmd{NewBibliographyString}\see{aut:lng:cmd}
%\item Removed localisation key \texttt{editor}\see{aut:lng:key}
%\item Removed localisation key \texttt{editors}\see{aut:lng:key}
%\item Renamed localisation key \texttt{typeeditor} to \texttt{editor}\see{aut:lng:key}
%\item Renamed localisation key \texttt{typeeditors} to \texttt{editors}\see{aut:lng:key}
%\item Renamed localisation key \texttt{typecompiler} to \texttt{compiler}\see{aut:lng:key}
%\item Renamed localisation key \texttt{typecompilers} to \texttt{compilers}\see{aut:lng:key}
%\item Added localisation key \texttt{founder}\see{aut:lng:key}
%\item Added localisation key \texttt{founders}\see{aut:lng:key}
%\item Added localisation key \texttt{continuator}\see{aut:lng:key}
%\item Added localisation key \texttt{continuators}\see{aut:lng:key}
%\item Added localisation key \texttt{collaborator}\see{aut:lng:key}
%\item Added localisation key \texttt{collaborators}\see{aut:lng:key}
%\item Removed localisation key \texttt{byauthor}\see{aut:lng:key}
%\item Renamed localisation key \texttt{bytypeauthor} to \texttt{byauthor}\see{aut:lng:key}
%\item Removed localisation key \texttt{byeditor}\see{aut:lng:key}
%\item Renamed localisation key \texttt{bytypeeditor} to \texttt{byeditor}\see{aut:lng:key}
%\item Renamed localisation key \texttt{bytypecompiler} to \texttt{bycompiler}\see{aut:lng:key}
%\item Added localisation key \texttt{byfounder}\see{aut:lng:key}
%\item Added localisation key \texttt{bycontinuator}\see{aut:lng:key}
%\item Added localisation key \texttt{bycollaborator}\see{aut:lng:key}
%\item Added localisation key \texttt{inpress}\see{aut:lng:key}
%\item Added localisation key \texttt{submitted}\see{aut:lng:key}
%\item Added support for Dutch (translations by Alexander van Loon)
%\item Added support for Greek (translations by Apostolos Syropoulos)
%\item Added notes on Greek\see{use:loc:grk}
%
%\end{release}
%
%\begin{release}{0.8i}{2009-09-20}
%
%\item Fixed some bugs
%
%\end{release}
%
%\begin{release}{0.8h}{2009-08-10}
%
%\item Fixed some bugs
%
%\end{release}
%
%\begin{release}{0.8g}{2009-08-06}
%
%\item Fixed some bugs
%
%\end{release}
%
%\begin{release}{0.8f}{2009-07-25}
%
%\item Fixed some bugs
%
%\end{release}
%
%\begin{release}{0.8e}{2009-07-04}
%
%\item Added \cmd{mkbibordedition}\see{use:fmt:lng}
%\item Added \cmd{mkbibordseries}\see{use:fmt:lng}
%\item Added \cmd{mkbibendnote}\see{aut:fmt:ich}
%\item Added several localisation keys related to \texttt{editor}\see{aut:lng:key}
%\item Added several localisation keys related to \texttt{translator}\see{aut:lng:key}
%\item Added localisation key \texttt{thiscite}\see{aut:lng:key}
%\item Removed several \texttt{country...} localisation keys\see{aut:lng:key}
%\item Removed several \texttt{patent...} localisation keys\see{aut:lng:key}
%\item Removed several \texttt{patreq...} localisation keys\see{aut:lng:key}
%\item Updated and clarified documentation\see{aut:lng:key}
%\item Added support for Brazilian Portuguese (by Augusto Ritter Stoffel)
%\item Added preliminary support for Portuguese/Portugal\see{use:opt:pre:gen}
%\item Added revised Swedish translations (by Per Starbäck)\see{use:opt:pre:gen}
%\item Expanded documentation\see{aut:cav:nam}
%\item Improved concatenation of editorial and other roles
%\item Fixed some bugs
%
%\end{release}
%
%\begin{release}{0.8d}{2009-05-30}
%
%\item Removed package option \opt{bibtex8}\see{use:opt:pre:gen}
%\item Added package option \opt{backend}\see{use:opt:pre:gen}
%\item Slightly modified package option \bibfield{loccittracker}\see{use:opt:pre:int}
%\item Added \cmd{volcite} and \cmd{Volcite}\see{use:cit:spc}
%\item Added \cmd{pvolcite} and \cmd{Pvolcite}\see{use:cit:spc}
%\item Added \cmd{fvolcite}\see{use:cit:spc}
%\item Added \cmd{tvolcite} and \cmd{Tvolcite}\see{use:cit:spc}
%\item Added \cmd{avolcite} and \cmd{Avolcite}\see{use:cit:spc}
%\item Added \cmd{notecite} and \cmd{Notecite}\see{use:cit:spc}
%\item Added \cmd{Pnotecite} and \cmd{Pnotecite}\see{use:cit:spc}
%\item Added \cmd{fnotecite}\see{use:cit:spc}
%\item Added \cmd{addabthinspace}\see{aut:pct:spc}
%\item Disable citation and page trackers in \acr{TOC}/\acr{LOT}/\acr{LOF}\see{aut:cav:flt}
%\item Disable citation and page trackers in floats\see{aut:cav:flt}
%\item Improved on-demand loading of localisation modules
%\item Fixed some bugs
%
%\end{release}
%
%\begin{release}{0.8c}{2009-01-10}
%
%\item Added <idem> tracker\see{use:opt:pre:int}
%\item Added package option \opt{idemtracker}\see{use:opt:pre:int}
%\item Added \cmd{ifciteidem}\see{aut:aux:tst}
%\item Added \cmd{ifentryseen}\see{aut:aux:tst}
%\item Improved citation style \texttt{verbose-trad1}\see{use:xbx:cbx}
%\item Improved citation style \texttt{verbose-trad2}\see{use:xbx:cbx}
%\item Renamed \len{bibitemextrasep} to \len{bibnamesep}\see{use:fmt:len}
%\item Slightly modified \len{bibnamesep}\see{use:fmt:len}
%\item Added \len{bibinitsep}\see{use:fmt:len}
%\item Increased default value of \cnt{highnamepenalty}\see{use:fmt:len}
%\item Increased default value of \cnt{lownamepenalty}\see{use:fmt:len}
%\item Updated documentation\see{use:loc:us}
%\item Added \cmd{uspunctuation}\see{aut:pct:cfg}
%\item Added \cmd{stdpunctuation}\see{aut:pct:cfg}
%\item Added \cmd{midsentence*}\see{aut:pct:ctr}
%\item Fixed some bugs
%
%\end{release}
%
%\begin{release}{0.8b}{2008-12-13}
%
%\item Added package/entry option \opt{usetranslator}\see{use:opt:bib}
%\item Added \cmd{ifusetranslator}\see{aut:aux:tst}
%\item Consider \bibfield{translator} when sorting\see{use:srt}
%\item Consider \bibfield{translator} when generating \bibfield{labelname}\see{aut:bbx:fld}
%\item Added field \bibfield{eventtitle}\see{bib:fld:dat}
%\item Support \bibfield{eventtitle} in \bibtype{proceedings} entries\see{bib:typ:blx}
%\item Support \bibfield{eventtitle} in \bibtype{inproceedings} entries\see{bib:typ:blx}
%\item Added unsupported entry type \bibtype{commentary}\see{bib:typ:ctm}
%\item Permit \cmd{NewBibliographyString} in \file{lbx} files\see{aut:lng:cmd}
%\item Improved behavior of \cmd{mkbibquote} in <American-punctuation> mode\see{aut:fmt:ich}
%\item Fixed some bugs
%
%\end{release}
%
%\begin{release}{0.8a}{2008-11-29}
%
%\item Updated documentation (important, please read)\see{int:feb}
%\item Added package option \kvopt{hyperref}{auto}\see{use:opt:pre:gen}
%\item Improved bibliography style \texttt{reading}\see{use:xbx:bbx}
%\item Updated \acr{KOMA}-Script support for version 3.x\see{use:cav:scr}
%\item Slightly modified special field \bibfield{fullhash}\see{aut:bbx:fld}
%\item Added documentation of \cmd{DeclareNumChars*}\see{aut:aux:msc}
%\item Added documentation of \cmd{DeclareRangeChars*}\see{aut:aux:msc}
%\item Added documentation of \cmd{DeclareRangeCommands*}\see{aut:aux:msc}
%\item Added \cmd{MakeSentenceCase}\see{aut:aux:msc}
%\item Added \cmd{DeclareCaseLangs}\see{aut:aux:msc}
%\item Support nested \cmd{mkbibquote} with American punctuation\see{aut:fmt:ich}
%\item Improved \cmd{ifpunctmark}\see{aut:pct:chk}
%\item Improved punctuation tracker\see{aut:pct:pct}
%\item Added \cmd{DeclarePunctuationPairs}\see{aut:pct:cfg}
%\item Added \cmd{DeclareLanguageMapping}\see{aut:lng:cmd}
%\item Added support for custom localisation modules\see{aut:cav:lng}
%\item Added extended \pdf bookmarks to this manual
%\item Fixed various bugs
%
%\end{release}
%
%\begin{release}{0.8}{2008-10-02}
%
%\item Added \cmd{DefineHyphenationExceptions}\see{use:lng}
%\item Added \cmd{DeclareHyphenationExceptions}\see{aut:lng:cmd}
%\item Added \cmd{mkpagetotal}\see{aut:aux:msc}
%\item Improved \acr{KOMA}-Script support\see{use:cav:scr}
%\item Added \cmd{ifkomabibtotoc}\see{use:cav:scr}
%\item Added \cmd{ifkomabibtotocnumbered}\see{use:cav:scr}
%\item Added \cmd{ifmemoirbibintoc}\see{use:cav:mem}
%\item Updated documentation\see{use:bib:hdg}
%\item Updated documentation of \cmd{iffootnote}\see{aut:aux:tst}
%\item Added several new localisation keys\see{aut:lng:key}
%\item Rearranged some localisation keys (\texttt{section} vs. \texttt{paragraph})\see{aut:lng:key}
%\item Added unsupported entry type \bibtype{letter}\see{bib:typ:ctm}
%\item Added entry type \bibtype{suppbook}\see{bib:typ:blx}
%\item Added entry type \bibtype{suppcollection}\see{bib:typ:blx}
%\item Added entry type \bibtype{suppperiodical}\see{bib:typ:blx}
%\item Support \bibtype{reference} and \bibtype{inreference}\see{bib:typ:blx}
%\item Support \bibtype{review} as an alias\see{bib:typ:ctm}
%\item Added field \bibfield{origpublisher}\see{bib:fld:dat}
%\item Added field alias \bibfield{annote}\see{bib:fld:als}
%\item Expanded documentation\see{bib:cav:enc}
%\item Added \cmd{DeclareCapitalPunctuation}\see{aut:pct:cfg}
%\item Removed \cmd{EnableCapitalAfter} and \cmd{DisableCapitalAfter}\see{aut:pct:cfg}
%\item Added support for <American-style> punctuation\see{aut:pct:cfg}
%\item Added \cmd{DeclareQuotePunctuation}\see{aut:pct:cfg}
%\item Improved \cmd{mkbibquote}\see{aut:fmt:ich}
%\item Expanded documentation\see{use:loc:us}
%\item Improved all \texttt{numeric} citation styles\see{use:xbx:cbx}
%\item Improved \texttt{numeric} bibliography style\see{use:xbx:bbx}
%\item Added citation style \texttt{authoryear-ibid}\see{use:xbx:cbx}
%\item Improved all \texttt{authoryear} citation styles\see{use:xbx:cbx}
%\item Improved \texttt{authoryear} bibliography style\see{use:xbx:bbx}
%\item Added \opt{pageref} option to \texttt{verbose-note} style\see{use:xbx:cbx}
%\item Added \opt{pageref} option to \texttt{verbose-inote} style\see{use:xbx:cbx}
%\item Added citation style \texttt{reading}\see{use:xbx:cbx}
%\item Added bibliography style \texttt{reading}\see{use:xbx:bbx}
%\item Added citation style \texttt{draft}\see{use:xbx:cbx}
%\item Added bibliography style \texttt{draft}\see{use:xbx:bbx}
%\item Improved \sty{natbib} compatibility style\see{use:cit:nat}
%\item Added \cmd{ifcitation}\see{aut:aux:tst}
%\item Added \cmd{ifbibliography}\see{aut:aux:tst}
%\item Added \cmd{printfile}\see{aut:bib:dat}
%\item Added package option \opt{loadfiles}\see{use:opt:pre:gen}
%\item Added support for bibliographic data in external files\see{use:use:prf}
%\item Expanded documentation\see{aut:cav:prf}
%\item Modified field \bibfield{edition}\see{bib:fld:dat}
%\item Modified special field \bibfield{labelyear}\see{aut:bbx:fld}
%\item Modified special field \bibfield{labelalpha}\see{aut:bbx:fld}
%\item Added special field \bibfield{extraalpha}\see{aut:bbx:fld}
%\item Added counter \cnt{maxlabelyear}\see{aut:fmt:ilc}
%\item Added counter \cnt{maxextraalpha}\see{aut:fmt:ilc}
%\item Added \cmd{mknumalph}\see{use:fmt:aux}
%\item Added \cmd{mkbibacro}\see{use:fmt:aux}
%\item Added \cmd{autocap}\see{use:fmt:aux}
%\item Added package option \opt{firstinits}\see{use:opt:pre:gen}
%\item Added \cmd{iffirstinits}\see{aut:aux:tst}
%\item Added support for eprint data\see{use:use:epr}
%\item Added support for arXiv\see{use:use:epr}
%\item Expanded documentation \see{aut:cav:epr}
%\item Added field \bibfield{eprint}\see{bib:fld:dat}
%\item Added field \bibfield{eprinttype}\see{bib:fld:dat}
%\item Added eprint support to all standard entry types\see{bib:typ:blx}
%\item Added package option \opt{arxiv}\see{use:opt:pre:gen}
%\item Introduced concept of an entry set\see{use:use:set}
%\item Expanded documentation\see{aut:cav:set}
%\item Added entry type \bibtype{set}\see{bib:typ:blx}
%\item Added field \bibfield{entryset}\see{bib:fld:spc}
%\item Added special field \bibfield{entrysetcount}\see{aut:bbx:fld}
%\item Added \cmd{entrydata}\see{aut:bib:dat}
%\item Expanded documentation\see{aut:cav:mif}
%\item Added \cmd{entryset}\see{aut:bib:dat}
%\item Added \cmd{strfield}\see{aut:aux:dat}
%\item Improved \cmd{usedriver}\see{aut:aux:msc}
%\item Added \cmd{bibpagespunct}\see{use:fmt:fmt}
%\item Expanded documentation\see{aut:cav:pct}
%\item Added entry option \opt{skipbib}\see{use:opt:bib}
%\item Added entry option \opt{skiplos}\see{use:opt:bib}
%\item Added entry option \opt{skiplab}\see{use:opt:bib}
%\item Added entry option \opt{dataonly}\see{use:opt:bib}
%\item Modified special field \bibfield{namehash}\see{aut:bbx:fld}
%\item Added special field \bibfield{fullhash}\see{aut:bbx:fld}
%\item Added \cmd{DeclareNumChars}\see{aut:aux:msc}
%\item Added \cmd{DeclareRangeChars}\see{aut:aux:msc}
%\item Added \cmd{DeclareRangeCommands}\see{aut:aux:msc}
%\item Added support for Swedish (translations by Per Starbäck and others)
%\item Updated various localisation files
%\item Various minor improvements throughout
%\item Fixed some bugs
%
%\end{release}
%
%\begin{release}{0.7}{2007-12-09}
%
%\item Expanded documentation\see{int:feb}
%\item New dependency on \sty{etoolbox} package\see{int:pre:req}
%\item Made \sty{url} a required package\see{int:pre:req}
%\item Modified package option \opt{sorting}\see{use:opt:pre:gen}
%\item Introduced concept of an entry option\see{use:opt:bib}
%\item Added option \bibfield{useauthor}\see{use:opt:bib}
%\item Added option \bibfield{useeditor}\see{use:opt:bib}
%\item Modified option \bibfield{useprefix}\see{use:opt:bib}
%\item Removed field \bibfield{useprefix}\see{bib:fld:spc}
%\item Added field \bibfield{options}\see{bib:fld:spc}
%\item Updated documentation\see{use:srt}
%\item Added citation style \texttt{authortitle-ibid}\see{use:xbx:cbx}
%\item Added citation style \texttt{authortitle-icomp}\see{use:xbx:cbx}
%\item Renamed citation style \texttt{authortitle-cterse} to \texttt{authortitle-tcomp}\see{use:xbx:cbx}
%\item Renamed citation style \texttt{authortitle-verb} to \texttt{verbose}\see{use:xbx:cbx}
%\item Renamed citation style \texttt{authortitle-cverb} to \texttt{verbose-ibid}\see{use:xbx:cbx}
%\item Added citation style \texttt{verbose-note}\see{use:xbx:cbx}
%\item Added citation style \texttt{verbose-inote}\see{use:xbx:cbx}
%\item Renamed citation style \texttt{authortitle-trad} to \texttt{verbose-trad1}\see{use:xbx:cbx}
%\item Removed citation style \texttt{authortitle-strad}\see{use:xbx:cbx}
%\item Added citation style \texttt{verbose-trad2}\see{use:xbx:cbx}
%\item Improved citation style \texttt{authoryear}\see{use:xbx:cbx}
%\item Improved citation style \texttt{authoryear-comp}\see{use:xbx:cbx}
%\item Improved citation style \texttt{authortitle-terse}\see{use:xbx:cbx}
%\item Improved citation style \texttt{authortitle-tcomp}\see{use:xbx:cbx}
%\item Improved all verbose citation styles\see{use:xbx:cbx}
%\item Expanded documentation\see{bib:fld:typ}
%\item Modified entry type \bibtype{article}\see{bib:typ:blx}
%\item Added entry type \bibtype{periodical}\see{bib:typ:blx}
%\item Added entry type \bibtype{patent}\see{bib:typ:blx}
%\item Extended entry types \bibfield{proceedings} and \bibfield{inproceedings}\see{bib:typ:blx}
%\item Extended entry type \bibfield{article}\see{bib:typ:blx}
%\item Extended entry type \bibfield{booklet}\see{bib:typ:blx}
%\item Extended entry type \bibfield{misc}\see{bib:typ:blx}
%\item Added entry type alias \bibtype{electronic}\see{bib:typ:als}
%\item Added new custom types\see{bib:typ:ctm}
%\item Support \bibfield{pagetotal} field where applicable\see{bib:typ:blx}
%\item Added field \bibfield{holder}\see{bib:fld:dat}
%\item Added field \bibfield{venue}\see{bib:fld:dat}
%\item Added field \bibfield{version}\see{bib:fld:dat}
%\item Added field \bibfield{journaltitle}\see{bib:fld:dat}
%\item Added field \bibfield{journalsubtitle}\see{bib:fld:dat}
%\item Added field \bibfield{issuetitle}\see{bib:fld:dat}
%\item Added field \bibfield{issuesubtitle}\see{bib:fld:dat}
%\item Removed field \bibfield{journal}\see{bib:fld:dat}
%\item Added field alias \bibfield{journal}\see{bib:fld:als}
%\item Added field \bibfield{shortjournal}\see{bib:fld:dat}
%\item Added field \bibfield{shortseries}\see{bib:fld:dat}
%\item Added field \bibfield{shorthandintro}\see{bib:fld:dat}
%\item Added field \bibfield{xref}\see{bib:fld:spc}
%\item Added field \bibfield{authortype}\see{bib:fld:dat}
%\item Added field \bibfield{editortype}\see{bib:fld:dat}
%\item Added field \bibfield{reprinttitle}\see{bib:fld:dat}
%\item Improved handling of field \bibfield{type}\see{bib:fld:dat}
%\item Improved handling of field \bibfield{series}\see{bib:fld:dat}
%\item Updated documentation\see{bib:use:ser}
%\item Renamed field \bibfield{id} to \bibfield{eid}\see{bib:fld:dat}
%\item Added field \bibfield{pagination}\see{bib:fld:dat}
%\item Added field \bibfield{bookpagination}\see{bib:fld:dat}
%\item Added special field \bibfield{sortinit}\see{aut:bbx:fld}
%\item Introduced concept of a multicite command\see{use:cit:mlt}
%\item Added \cmd{cites}\see{use:cit:mlt}
%\item Added \cmd{Cites}\see{use:cit:mlt}
%\item Added \cmd{parencites}\see{use:cit:mlt}
%\item Added \cmd{Parencites}\see{use:cit:mlt}
%\item Added \cmd{footcites}\see{use:cit:mlt}
%\item Added \cmd{supercites}\see{use:cit:mlt}
%\item Added \cmd{Autocite}\see{use:cit:aut}
%\item Added \cmd{autocites}\see{use:cit:aut}
%\item Added \cmd{Autocites}\see{use:cit:aut}
%\item Added \cmd{DeclareMultiCiteCommand}\see{aut:cbx:cbx}
%\item Added counter \cnt{multicitecount}\see{aut:fmt:ilc}
%\item Added counter \cnt{multicitetotal}\see{aut:fmt:ilc}
%\item Renamed \cmd{citefulltitle} to \cmd{citetitle*}\see{use:cit:txt}
%\item Added \cmd{cite*}\see{use:cit:cbx}
%\item Added \cmd{citeurl}\see{use:cit:txt}
%\item Added documentation of field \bibfield{nameaddon}\see{bib:fld:dat}
%\item Added field \bibfield{entrysubtype}\see{bib:fld:spc}
%\item Added field \bibfield{execute}\see{bib:fld:spc}
%\item Added custom fields \bibfield{verb{[a--c]}}\see{bib:fld:ctm}
%\item Added custom fields \bibfield{name{[a--c]}type}\see{bib:fld:ctm}
%\item Consider \bibfield{sorttitle} field when falling back to \bibfield{title}\see{use:srt}
%\item Removed package option \opt{labelctitle}\see{use:opt:pre:int}
%\item Removed field \opt{labelctitle}\see{aut:bbx:fld}
%\item Added package option \opt{singletitle}\see{use:opt:pre:int}
%\item Added \cmd{ifsingletitle}\see{aut:aux:tst}
%\item Added \cmd{ifuseauthor}\see{aut:aux:tst}
%\item Added \cmd{ifuseeditor}\see{aut:aux:tst}
%\item Added \cmd{ifopcit}\see{aut:aux:tst}
%\item Added \cmd{ifloccit}\see{aut:aux:tst}
%\item Added package option \opt{uniquename}\see{use:opt:pre:int}
%\item Added special counter \cnt{uniquename}\see{aut:aux:tst}
%\item Added package option \opt{natbib}\see{use:opt:ldt}
%\item Added compatibility commands for the \sty{natbib} package\see{use:cit:nat}
%\item Added package option \opt{defernums}\see{use:opt:pre:gen}
%\item Improved support for numeric labels\see{use:cav:lab}
%\item Added package option \opt{mincrossrefs}\see{use:opt:pre:gen}
%\item Added package option \opt{bibencoding}\see{use:opt:pre:gen}
%\item Expanded documentation\see{bib:cav:enc}
%\item Updated documentation\see{bib:cav:ide}
%\item Added package option \opt{citetracker}\see{use:opt:pre:int}
%\item Added package option \opt{ibidtracker}\see{use:opt:pre:int}
%\item Added package option \bibfield{opcittracker}\see{use:opt:pre:int}
%\item Added package option \bibfield{loccittracker}\see{use:opt:pre:int}
%\item Added \cmd{citetrackertrue} and \cmd{citetrackerfalse}\see{aut:aux:msc}
%\item Modified package option \opt{pagetracker}\see{use:opt:pre:int}
%\item Added \cmd{pagetrackertrue} and \cmd{pagetrackerfalse}\see{aut:aux:msc}
%\item Text commands now excluded from tracking\see{use:cit:txt}
%\item Updated documentation of \cmd{iffirstonpage}\see{aut:aux:tst}
%\item Updated documentation of \cmd{ifsamepage}\see{aut:aux:tst}
%\item Removed package option \opt{keywsort}\see{use:opt:pre:gen}
%\item Added package option \opt{refsection}\see{use:opt:pre:gen}
%\item Added package option \opt{refsegment}\see{use:opt:pre:gen}
%\item Added package option \opt{citereset}\see{use:opt:pre:gen}
%\item Added option \opt{section} to \cmd{bibbysegment}\see{use:bib:bib}
%\item Added option \opt{section} to \cmd{bibbycategory}\see{use:bib:bib}
%\item Added option \opt{section} to \cmd{printshorthands}\see{use:bib:biblist}
%\item Extended documentation of \env{refsection} environment\see{use:bib:sec}
%\item Added \cmd{newrefsection}\see{use:bib:sec}
%\item Added \cmd{newrefsegment}\see{use:bib:seg}
%\item Added heading definition \texttt{subbibliography}\see{use:bib:hdg}
%\item Added heading definition \texttt{subbibintoc}\see{use:bib:hdg}
%\item Added heading definition \texttt{subbibnumbered}\see{use:bib:hdg}
%\item Make all citation commands scan ahead for punctuation\see{use:cit}
%\item Updated documentation of \cmd{DeclareAutoPunctuation}\see{aut:pct:cfg}
%\item Removed \cmd{usecitecmd}\see{aut:cbx:cbx}
%\item Updated documentation of \opt{autocite} package option\see{use:opt:pre:gen}
%\item Updated documentation of \opt{autopunct} package option\see{use:opt:pre:gen}
%\item Added \cmd{citereset}\see{use:cit:msc}
%\item Added \cmd{citereset*}\see{use:cit:msc}
%\item Added \cmd{mancite}\see{use:cit:msc}
%\item Added \cmd{citesetup}\see{use:fmt:fmt}
%\item Added \cmd{compcitedelim}\see{use:fmt:fmt}
%\item Added \cmd{labelnamepunct}\see{use:fmt:fmt}
%\item Added \cmd{subtitlepunct}\see{use:fmt:fmt}
%\item Added \cmd{finallistdelim}\see{use:fmt:fmt}
%\item Added \cmd{andmoredelim}\see{use:fmt:fmt}
%\item Added \cmd{labelalphaothers}\see{use:fmt:fmt}
%\item Added \len{bibitemextrasep}\see{use:fmt:len}
%\item Renamed \cmd{blxauxprefix} to \cmd{blxauxsuffix}\see{use:use:aux}
%\item Added \cmd{DeclareBibliographyOption}\see{aut:bbx:bbx}
%\item Added \cmd{DeclareEntryOption}\see{aut:bbx:bbx}
%\item Renamed \cmd{InitializeBibliographyDrivers} to \cmd{InitializeBibliographyStyle}\see{aut:bbx:bbx}
%\item Added \cmd{InitializeCitationStyle}\see{aut:cbx:cbx}
%\item Added \cmd{OnManualCitation}\see{aut:cbx:cbx}
%\item Extended documentation of \cmd{DeclareCiteCommand}\see{aut:cbx:cbx}
%\item Modified \cmd{DeclareAutoCiteCommand}\see{aut:cbx:cbx}
%\item Improved \cmd{printtext}\see{aut:bib:dat}
%\item Improved \cmd{printfield}\see{aut:bib:dat}
%\item Improved \cmd{printlist}\see{aut:bib:dat}
%\item Improved \cmd{printnames}\see{aut:bib:dat}
%\item Improved \cmd{indexfield}\see{aut:bib:dat}
%\item Improved \cmd{indexlist}\see{aut:bib:dat}
%\item Improved \cmd{indexnames}\see{aut:bib:dat}
%\item Modified \cmd{DeclareFieldFormat}\see{aut:bib:fmt}
%\item Modified \cmd{DeclareListFormat}\see{aut:bib:fmt}
%\item Modified \cmd{DeclareNameFormat}\see{aut:bib:fmt}
%\item Modified \cmd{DeclareFieldAlias}\see{aut:bib:fmt}
%\item Modified \cmd{DeclareListAlias}\see{aut:bib:fmt}
%\item Modified \cmd{DeclareNameAlias}\see{aut:bib:fmt}
%\item Modified \cmd{DeclareIndexFieldFormat}\see{aut:bib:fmt}
%\item Modified \cmd{DeclareIndexListFormat}\see{aut:bib:fmt}
%\item Modified \cmd{DeclareIndexNameFormat}\see{aut:bib:fmt}
%\item Modified \cmd{DeclareIndexFieldAlias}\see{aut:bib:fmt}
%\item Modified \cmd{DeclareIndexListAlias}\see{aut:bib:fmt}
%\item Modified \cmd{DeclareIndexNameAlias}\see{aut:bib:fmt}
%\item Improved \cmd{iffirstonpage}\see{aut:aux:tst}
%\item Improved \cmd{ifciteseen}\see{aut:aux:tst}
%\item Improved \cmd{ifandothers}\see{aut:aux:tst}
%\item Added \cmd{ifinteger}\see{aut:aux:tst}
%\item Added \cmd{ifnumeral}\see{aut:aux:tst}
%\item Added \cmd{ifnumerals}\see{aut:aux:tst}
%\item Removed \cmd{ifpage}\see{aut:aux:tst}
%\item Removed \cmd{ifpages}\see{aut:aux:tst}
%\item Moved \cmd{ifblank} to \sty{etoolbox} package\see{aut:aux:tst}
%\item Removed \cmd{xifblank}\see{aut:aux:tst}
%\item Moved \cmd{docsvlist} to \sty{etoolbox} package\see{aut:aux:msc}
%\item Updated documentation of \cmd{docsvfield}\see{aut:aux:msc}
%\item Added \cmd{ifciteibid}\see{aut:aux:tst}
%\item Added \cmd{iffootnote}\see{aut:aux:tst}
%\item Added \cmd{iffieldxref}\see{aut:aux:tst}
%\item Added \cmd{iflistxref}\see{aut:aux:tst}
%\item Added \cmd{ifnamexref}\see{aut:aux:tst}
%\item Added \cmd{ifmoreitems}\see{aut:aux:tst}
%\item Added \cmd{ifbibstring}\see{aut:aux:tst}
%\item Added \cmd{iffieldbibstring}\see{aut:aux:tst}
%\item Added \cmd{mkpageprefix}\see{aut:aux:msc}
%\item Added \cmd{NumCheckSetup}\see{aut:aux:msc}
%\item Added \cmd{pno}\see{use:cit:msc}
%\item Added \cmd{ppno}\see{use:cit:msc}
%\item Added \cmd{nopp}\see{use:cit:msc}
%\item Added \cmd{ppspace}\see{aut:aux:msc}
%\item Added \cmd{psq}\see{use:cit:msc}
%\item Added \cmd{psqq}\see{use:cit:msc}
%\item Added \cmd{sqspace}\see{aut:aux:msc}
%\item Expanded documentation\see{bib:use:pag}
%\item Expanded documentation\see{use:cav:pag}
%\item Added \cmd{RN}\see{use:cit:msc}
%\item Added \cmd{Rn}\see{use:cit:msc}
%\item Added \cmd{RNfont}\see{use:cit:msc}
%\item Added \cmd{Rnfont}\see{use:cit:msc}
%\item Added package option \opt{punctfont}\see{use:opt:pre:gen}
%\item Added \cmd{setpunctfont}\see{aut:pct:new}
%\item Added \cmd{resetpunctfont}\see{aut:pct:new}
%\item Added \cmd{nopunct}\see{aut:pct:pct}
%\item Added \cmd{bibxstring}\see{aut:str}
%\item Added \cmd{mkbibemph}\see{aut:fmt:ich}
%\item Added \cmd{mkbibquote}\see{aut:fmt:ich}
%\item Added \cmd{mkbibfootnote}\see{aut:fmt:ich}
%\item Added \cmd{mkbibsuperscript}\see{aut:fmt:ich}
%\item Added \cmd{currentfield}\see{aut:fmt:ilc}
%\item Added \cmd{currentlist}\see{aut:fmt:ilc}
%\item Added \cmd{currentname}\see{aut:fmt:ilc}
%\item Added \cmd{AtNextCite}\see{aut:fmt:hok}
%\item Added \cmd{AtNextCitekey}\see{aut:fmt:hok}
%\item Added \cmd{AtDataInput}\see{aut:fmt:hok}
%\item Added several new localisation keys\see{aut:lng:key}
%\item Added support for Norwegian (translations by Johannes Wilm)
%\item Added support for Danish (translations by Johannes Wilm)
%\item Expanded documentation\see{aut:cav:grp}
%\item Expanded documentation\see{aut:cav:mif}
%\item Numerous improvements under the hood
%\item Fixed some bugs
%
%\end{release}
%
%\begin{release}{0.6}{2007-01-06}
%
%\item Added package option \kvopt{sorting}{none}\see{use:opt:pre:gen}
%\item Renamed package option \kvopt{block}{penalty} to \kvopt{block}{ragged}\see{use:opt:pre:gen}
%\item Changed data type of \bibfield{origlanguage} back to field\see{bib:fld:dat}
%\item Support \bibfield{origlanguage} field if \bibfield{translator} is present\see{bib:typ:blx}
%\item Renamed field \bibfield{articleid} to \bibfield{id}\see{bib:fld:dat}
%\item Support \bibfield{id} field in \bibfield{article} entries\see{bib:typ:blx}
%\item Support \bibfield{series} field in \bibfield{article} entries\see{bib:typ:blx}
%\item Support \bibfield{doi} field\see{bib:typ:blx}
%\item Updated documentation of all entry types\see{bib:typ:blx}
%\item Updated documentation of field \bibfield{series}\see{bib:fld:dat}
%\item Added field \bibfield{redactor}\see{bib:fld:dat}
%\item Added field \bibfield{shortauthor}\see{bib:fld:dat}
%\item Added field \bibfield{shorteditor}\see{bib:fld:dat}
%\item Improved support for corporate authors and editors\see{bib:use:inc}
%\item Updated documentation of field \bibfield{labelname}\see{aut:bbx:fld}
%\item Added field alias \bibfield{key}\see{bib:fld:als}
%\item Added package option \opt{autocite}\see{use:opt:pre:gen}
%\item Added package option \opt{autopunct}\see{use:opt:pre:gen}
%\item Added \cmd{autocite}\see{use:cit:aut}
%\item Added \cmd{DeclareAutoCiteCommand}\see{aut:cbx:cbx}
%\item Added \cmd{DeclareAutoPunctuation}\see{aut:pct:cfg}
%\item Added option \opt{filter} to \cmd{printbibliography}\see{use:bib:bib}
%\item Added \cmd{defbibfilter}\see{use:bib:flt}
%\item Added package option \opt{maxitems}\see{use:opt:pre:gen}
%\item Added package option \opt{minitems}\see{use:opt:pre:gen}
%\item Added option \opt{maxitems} to \cmd{printbibliography}\see{use:bib:bib}
%\item Added option \opt{minitems} to \cmd{printbibliography}\see{use:bib:bib}
%\item Added option \opt{maxitems} to \cmd{bibbysection}\see{use:bib:bib}
%\item Added option \opt{minitems} to \cmd{bibbysection}\see{use:bib:bib}
%\item Added option \opt{maxitems} to \cmd{bibbysegment}\see{use:bib:bib}
%\item Added option \opt{minitems} to \cmd{bibbysegment}\see{use:bib:bib}
%\item Added option \opt{maxitems} to \cmd{bibbycategory}\see{use:bib:bib}
%\item Added option \opt{minitems} to \cmd{bibbycategory}\see{use:bib:bib}
%\item Added option \opt{maxitems} to \cmd{printshorthands}\see{use:bib:biblist}
%\item Added option \opt{minitems} to \cmd{printshorthands}\see{use:bib:biblist}
%\item Added counter \cnt{maxitems}\see{aut:fmt:ilc}
%\item Added counter \cnt{minitems}\see{aut:fmt:ilc}
%\item Added adapted headings for \sty{scrartcl}, \sty{scrbook}, \sty{scrreprt}\see{int:pre:cmp}
%\item Added adapted headings for \sty{memoir}\see{int:pre:cmp}
%\item Added \cmd{Cite}\see{use:cit:std}
%\item Added \cmd{Parencite}\see{use:cit:std}
%\item Added \cmd{Textcite}\see{use:cit:cbx}
%\item Added \cmd{parencite*}\see{use:cit:cbx}
%\item Added \cmd{supercite}\see{use:cit:cbx}
%\item Added \cmd{Citeauthor}\see{use:cit:txt}
%\item Added \cmd{nameyeardelim}\see{use:fmt:fmt}
%\item Added \cmd{multilistdelim}\see{use:fmt:fmt}
%\item Completed documenation\see{use:fmt:fmt}
%\item Completed documenation\see{aut:fmt:fmt}
%\item Added \cmd{usecitecmd}\see{aut:cbx:cbx}
%\item Added \cmd{hyphenate}\see{use:fmt:aux}
%\item Added \cmd{hyphen}\see{use:fmt:aux}
%\item Added \cmd{nbhyphen}\see{use:fmt:aux}
%\item Improved \cmd{ifsamepage}\see{aut:aux:tst}
%\item Removed \cmd{ifnameequalstr}\see{aut:aux:tst}
%\item Removed \cmd{iflistequalstr}\see{aut:aux:tst}
%\item Added \cmd{ifcapital}\see{aut:aux:tst}
%\item Added documentation of \cmd{MakeCapital}\see{aut:aux:msc}
%\item Added starred variant to \cmd{setunit}\see{aut:pct:new}
%\item Improved \cmd{ifterm}\see{aut:pct:chk}
%\item Straightened out documentation of \cmd{thelist}\see{aut:aux:dat}
%\item Straightened out documentation of \cmd{thename}\see{aut:aux:dat}
%\item Added \cmd{docsvfield}\see{aut:aux:msc}
%\item Added \cmd{docsvlist}\see{aut:aux:msc}
%\item Removed \cmd{CopyFieldFormat}\see{aut:bib:fmt}
%\item Removed \cmd{CopyIndexFieldFormat}\see{aut:bib:fmt}
%\item Removed \cmd{CopyListFormat}\see{aut:bib:fmt}
%\item Removed \cmd{CopyIndexListFormat}\see{aut:bib:fmt}
%\item Removed \cmd{CopyNameFormat}\see{aut:bib:fmt}
%\item Removed \cmd{CopyIndexNameFormat}\see{aut:bib:fmt}
%\item Added \cmd{savefieldformat}\see{aut:aux:msc}
%\item Added \cmd{restorefieldformat}\see{aut:aux:msc}
%\item Added \cmd{savelistformat}\see{aut:aux:msc}
%\item Added \cmd{restorelistformat}\see{aut:aux:msc}
%\item Added \cmd{savenameformat}\see{aut:aux:msc}
%\item Added \cmd{restorenameformat}\see{aut:aux:msc}
%\item Added \cmd{savebibmacro}\see{aut:aux:msc}
%\item Added \cmd{restorebibmacro}\see{aut:aux:msc}
%\item Added \cmd{savecommand}\see{aut:aux:msc}
%\item Added \cmd{restorecommand}\see{aut:aux:msc}
%\item Added documentation of \texttt{shorthands} driver\see{aut:bbx:bbx}
%\item Rearranged, renamed, and extended localisation keys\see{aut:lng:key}
%\item Renamed counter \cnt{citecount} to \cnt{instcount}\see{aut:fmt:ilc}
%\item Added new counter \cnt{citecount}\see{aut:fmt:ilc}
%\item Added counter \cnt{citetotal}\see{aut:fmt:ilc}
%\item Rearranged and expanded documentation\see{bib:use}
%\item Expanded documentation\see{bib:cav}
%\item Expanded documentation\see{use:cav:scr}
%\item Expanded documentation\see{use:cav:mem}
%\item Completed support for Spanish\see{use:loc:esp}
%\item Added support for Italian (translations by Enrico Gregorio)
%\item Added language alias \opt{australian}\see{bib:fld:spc}
%\item Added language alias \opt{newzealand}\see{bib:fld:spc}
%\item Various minor improvements throughout
%
%\end{release}
%
%\begin{release}{0.5}{2006-11-12}
%
%\item Added \cmd{usedriver}\see{aut:aux:msc}
%\item Added package option \opt{pagetracker}\see{use:opt:pre:gen}
%\item Added \cmd{iffirstonpage}\see{aut:aux:tst}
%\item Added \cmd{ifsamepage}\see{aut:aux:tst}
%\item Corrected documentation of \cmd{ifciteseen}\see{aut:aux:tst}
%\item Added package option \opt{terseinits}\see{use:opt:pre:gen}
%\item Modified default value of package option \opt{maxnames}\see{use:opt:pre:gen}
%\item Renamed package option \opt{index} to \opt{indexing}\see{use:opt:pre:gen}
%\item Extended package option \opt{indexing}\see{use:opt:pre:gen}
%\item Removed package option \opt{citeindex}\see{use:opt:pre:gen}
%\item Removed package option \opt{bibindex}\see{use:opt:pre:gen}
%\item Added package option \opt{labelalpha}\see{use:opt:pre:int}
%\item Updated documentation of field \bibfield{labelalpha}\see{aut:bbx:fld}
%\item Added package option \opt{labelctitle}\see{use:opt:pre:int}
%\item Updated documentation of field \bibfield{labelctitle}\see{aut:bbx:fld}
%\item Added package option \opt{labelnumber}\see{use:opt:pre:int}
%\item Updated documentation of field \bibfield{labelnumber}\see{aut:bbx:fld}
%\item Added package option \opt{labelyear}\see{use:opt:pre:int}
%\item Updated documentation of field \bibfield{labelyear}\see{aut:bbx:fld}
%\item Added citation style \texttt{authortitle-verb}\see{use:xbx:cbx}
%\item Added citation style \texttt{authortitle-cverb}\see{use:xbx:cbx}
%\item Renamed citation style \texttt{traditional} to \texttt{authortitle-trad}\see{use:xbx:cbx}
%\item Improved citation style \texttt{authortitle-trad}\see{use:xbx:cbx}
%\item Added citation style \texttt{authortitle-strad}\see{use:xbx:cbx}
%\item Improved bibliography style \texttt{authoryear}\see{use:xbx:bbx}
%\item Improved bibliography style \texttt{authortitle}\see{use:xbx:bbx}
%\item Added option \opt{maxnames} to \cmd{printbibliography}\see{use:bib:bib}
%\item Added option \opt{minnames} to \cmd{printbibliography}\see{use:bib:bib}
%\item Added option \opt{maxnames} to \cmd{bibbysection}\see{use:bib:bib}
%\item Added option \opt{minnames} to \cmd{bibbysection}\see{use:bib:bib}
%\item Added option \opt{maxnames} to \cmd{bibbysegment}\see{use:bib:bib}
%\item Added option \opt{minnames} to \cmd{bibbysegment}\see{use:bib:bib}
%\item Added option \opt{maxnames} to \cmd{bibbycategory}\see{use:bib:bib}
%\item Added option \opt{minnames} to \cmd{bibbycategory}\see{use:bib:bib}
%\item Added option \opt{maxnames} to \cmd{printshorthands}\see{use:bib:biblist}
%\item Added option \opt{minnames} to \cmd{printshorthands}\see{use:bib:biblist}
%\item Renamed \env{bibsection} to \env{refsection} (conflict with \sty{memoir})\see{use:bib:sec}
%\item Renamed \env{bibsegment} to \env{refsegment} (consistency)\see{use:bib:seg}
%\item Extended \env{refsection} environment\see{use:bib:sec}
%\item Renamed \env{bibsection} counter to \env{refsection}\see{aut:fmt:ilc}
%\item Renamed \env{bibsegment} counter to \env{refsegment}\see{aut:fmt:ilc}
%\item Updated documentation\see{use:use:mlt}
%\item Added counter \cnt{citecount}\see{aut:fmt:ilc}
%\item Modified default definition of \cmd{blxauxprefix}\see{use:use:aux}
%\item Added \cmd{CopyFieldFormat}\see{aut:bib:fmt}
%\item Added \cmd{CopyIndexFieldFormat}\see{aut:bib:fmt}
%\item Added \cmd{CopyListFormat}\see{aut:bib:fmt}
%\item Added \cmd{CopyIndexListFormat}\see{aut:bib:fmt}
%\item Added \cmd{CopyNameFormat}\see{aut:bib:fmt}
%\item Added \cmd{CopyIndexNameFormat}\see{aut:bib:fmt}
%\item Added \cmd{clearfield}\see{aut:aux:dat}
%\item Added \cmd{clearlist}\see{aut:aux:dat}
%\item Added \cmd{clearname}\see{aut:aux:dat}
%\item Added \cmd{restorefield}\see{aut:aux:dat}
%\item Added \cmd{restorelist}\see{aut:aux:dat}
%\item Added \cmd{restorename}\see{aut:aux:dat}
%\item Renamed \cmd{bibhyperlink} to \cmd{bibhyperref}\see{aut:aux:msc}
%\item Added new command \cmd{bibhyperlink}\see{aut:aux:msc}
%\item Added \cmd{bibhypertarget}\see{aut:aux:msc}
%\item Renamed formatting directive \texttt{bibhyperlink} to \texttt{bibhyperref}\see{aut:fmt:ich}
%\item Added new formatting directive \texttt{bibhyperlink}\see{aut:fmt:ich}
%\item Added formatting directive \texttt{bibhypertarget}\see{aut:fmt:ich}
%\item Added \cmd{addlpthinspace}\see{aut:pct:spc}
%\item Added \cmd{addhpthinspace}\see{aut:pct:spc}
%\item Added field \bibfield{annotator}\see{bib:fld:dat}
%\item Added field \bibfield{commentator}\see{bib:fld:dat}
%\item Added field \bibfield{introduction}\see{bib:fld:dat}
%\item Added field \bibfield{foreword}\see{bib:fld:dat}
%\item Added field \bibfield{afterword}\see{bib:fld:dat}
%\item Updated documentation of field \bibfield{translator}\see{bib:fld:dat}
%\item Added field \bibfield{articleid}\see{bib:fld:dat}
%\item Added field \bibfield{doi}\see{bib:fld:dat}
%\item Added field \bibfield{file}\see{bib:fld:dat}
%\item Added field alias \bibfield{pdf}\see{bib:fld:als}
%\item Added field \bibfield{indextitle}\see{bib:fld:dat}
%\item Added field \bibfield{indexsorttitle}\see{bib:fld:spc}
%\item Changed data type of \bibfield{language}\see{bib:fld:dat}
%\item Changed data type of \bibfield{origlanguage}\see{bib:fld:dat}
%\item Updated documentation of entry type \bibfield{book}\see{bib:typ:blx}
%\item Updated documentation of entry type \bibfield{collection}\see{bib:typ:blx}
%\item Updated documentation of entry type \bibfield{inbook}\see{bib:typ:blx}
%\item Updated documentation of entry type \bibfield{incollection}\see{bib:typ:blx}
%\item Extended entry type \bibfield{misc}\see{bib:typ:blx}
%\item Added \cmd{UndefineBibliographyExtras}\see{use:lng}
%\item Added \cmd{UndeclareBibliographyExtras}\see{aut:lng:cmd}
%\item Added \cmd{finalandcomma}\see{use:fmt:lng}
%\item Added localisation key \texttt{citedas}\see{aut:lng:key}
%\item Renamed localisation key \texttt{editby} to \texttt{edited}\see{aut:lng:key}
%\item Renamed localisation key \texttt{transby} to \texttt{translated}\see{aut:lng:key}
%\item Added localisation key \texttt{annotated}\see{aut:lng:key}
%\item Added localisation key \texttt{commented}\see{aut:lng:key}
%\item Added localisation key \texttt{introduced}\see{aut:lng:key}
%\item Added localisation key \texttt{foreworded}\see{aut:lng:key}
%\item Added localisation key \texttt{afterworded}\see{aut:lng:key}
%\item Added localisation key \texttt{commentary}\see{aut:lng:key}
%\item Added localisation key \texttt{annotations}\see{aut:lng:key}
%\item Added localisation key \texttt{introduction}\see{aut:lng:key}
%\item Added localisation key \texttt{foreword}\see{aut:lng:key}
%\item Added localisation key \texttt{afterword}\see{aut:lng:key}
%\item Added localisation key \texttt{doneby}\see{aut:lng:key}
%\item Added localisation key \texttt{itemby}\see{aut:lng:key}
%\item Added localisation key \texttt{spanish}\see{aut:lng:key}
%\item Added localisation key \texttt{latin}\see{aut:lng:key}
%\item Added localisation key \texttt{greek}\see{aut:lng:key}
%\item Modified localisation key \texttt{fromenglish}\see{aut:lng:key}
%\item Modified localisation key \texttt{fromfrench}\see{aut:lng:key}
%\item Modified localisation key \texttt{fromgerman}\see{aut:lng:key}
%\item Added localisation key \texttt{fromspanish}\see{aut:lng:key}
%\item Added localisation key \texttt{fromlatin}\see{aut:lng:key}
%\item Added localisation key \texttt{fromgreek}\see{aut:lng:key}
%\item Expanded documentation\see{bib:use}
%\item Updated documentation\see{use:xbx:cbx}
%\item Updated documentation\see{use:xbx:bbx}
%\item Updated documentation\see{use:fmt:fmt}
%\item Updated documentation\see{aut:fmt:fmt}
%\item Updated and completed documentation\see{use:fmt:lng}
%\item Updated and completed documentation\see{aut:fmt:lng}
%\item Added support for Spanish (translations by Ignacio Fernández Galván)
%\item Various memory-related optimizations in \path{biblatex.bst}
%
%\end{release}
%
%\begin{release}{0.4}{2006-10-01}
%
%\item Added package option \opt{sortlos}\see{use:opt:pre:gen}
%\item Added package option \opt{bibtex8}\see{use:opt:pre:gen}
%\item Made \bibfield{pageref} field local to \env{refsection} environment\see{aut:bbx:fld}
%\item Renamed field \bibfield{labeltitle} to \bibfield{labelctitle}\see{aut:bbx:fld}
%\item Added new field \bibfield{labeltitle}\see{aut:bbx:fld}
%\item Added new field \bibfield{sortkey}\see{bib:fld:spc}
%\item Updated documentation\see{use:srt}
%\item Removed \cmd{iffieldtrue}\see{aut:aux:tst}
%\item Renamed counter \cnt{namepenalty} to \cnt{highnamepenalty}\see{use:fmt:len}
%\item Added counter \cnt{lownamepenalty}\see{use:fmt:len}
%\item Added documentation of \cmd{noligature}\see{use:fmt:aux}
%\item Added \cmd{addlowpenspace}\see{aut:pct:spc}
%\item Added \cmd{addhighpenspace}\see{aut:pct:spc}
%\item Added \cmd{addabbrvspace}\see{aut:pct:spc}
%\item Added \cmd{adddotspace}\see{aut:pct:spc}
%\item Added \cmd{addslash}\see{aut:pct:spc}
%\item Expanded documentation\see{use:cav}
%\item Various minor improvements throughout
%\item Fixed some bugs
%
%\end{release}
%
%\begin{release}{0.3}{2006-09-24}
%
%\item Renamed citation style \texttt{authortitle} to \texttt{authortitle-terse}\see{use:xbx:cbx}
%\item Renamed citation style \texttt{authortitle-comp} to \texttt{authortitle-cterse}\see{use:xbx:cbx}
%\item Renamed citation style \texttt{authortitle-verb} to \texttt{authortitle}\see{use:xbx:cbx}
%\item Added new citation style \texttt{authortitle-comp}\see{use:xbx:cbx}
%\item Citation style \texttt{traditional} now supports <loc.~cit.>\see{use:xbx:cbx}
%\item Added package option \opt{date}\see{use:opt:pre:gen}
%\item Added package option \opt{urldate}\see{use:opt:pre:gen}
%\item Introduced new data type: literal lists\see{bib:fld}
%\item Renamed \cmd{citename} to \cmd{citeauthor}\see{use:cit:txt}
%\item Renamed \cmd{citelist} to \cmd{citename}\see{use:cit:low}
%\item Added new \cmd{citelist} command\see{use:cit:low}
%\item Renamed \cmd{printlist} to \cmd{printnames}\see{aut:bib:dat}
%\item Added new \cmd{printlist} command\see{aut:bib:dat}
%\item Renamed \cmd{indexlist} to \cmd{indexnames}\see{aut:bib:dat}
%\item Added new \cmd{indexlist} command\see{aut:bib:dat}
%\item Renamed \cmd{DeclareListFormat} to \cmd{DeclareNameFormat}\see{aut:bib:fmt}
%\item Added new \cmd{DeclareListFormat} command\see{aut:bib:fmt}
%\item Renamed \cmd{DeclareListAlias} to \cmd{DeclareNameAlias}\see{aut:bib:fmt}
%\item Added new \cmd{DeclareListAlias} command\see{aut:bib:fmt}
%\item Renamed \cmd{DeclareIndexListFormat} to \cmd{DeclareIndexNameFormat}\see{aut:bib:fmt}
%\item Added new \cmd{DeclareIndexListFormat} command\see{aut:bib:fmt}
%\item Renamed \cmd{DeclareIndexListAlias} to \cmd{DeclareIndexNameAlias}\see{aut:bib:fmt}
%\item Added new \cmd{DeclareIndexListAlias} command\see{aut:bib:fmt}
%\item Renamed \cmd{biblist} to \cmd{thename}\see{aut:aux:dat}
%\item Added new \cmd{thelist} command\see{aut:aux:dat}
%\item Renamed \cmd{bibfield} to \cmd{thefield}\see{aut:aux:dat}
%\item Renamed \cmd{savelist} to \cmd{savename}\see{aut:aux:dat}
%\item Added new \cmd{savelist} command\see{aut:aux:dat}
%\item Renamed \cmd{savelistcs} to \cmd{savenamecs}\see{aut:aux:dat}
%\item Added new \cmd{savelistcs} command\see{aut:aux:dat}
%\item Renamed \cmd{iflistundef} to \cmd{ifnameundef}\see{aut:aux:tst}
%\item Added new \cmd{iflistundef} test\see{aut:aux:tst}
%\item Renamed \cmd{iflistsequal} to \cmd{ifnamesequal}\see{aut:aux:tst}
%\item Added new \cmd{iflistsequal} test\see{aut:aux:tst}
%\item Renamed \cmd{iflistequals} to \cmd{ifnameequals}\see{aut:aux:tst}
%\item Added new \cmd{iflistequals} test\see{aut:aux:tst}
%\item Renamed \cmd{iflistequalcs} to \cmd{ifnameequalcs}\see{aut:aux:tst}
%\item Added new \cmd{iflistequalcs} test\see{aut:aux:tst}
%\item Renamed \cmd{iflistequalstr} to \cmd{ifnameequalstr}\see{aut:aux:tst}
%\item Added new \cmd{iflistequalstr} test\see{aut:aux:tst}
%\item Renamed \cmd{ifcurrentlist} to \cmd{ifcurrentname}\see{aut:aux:tst}
%\item Added new \cmd{ifcurrentlist} test\see{aut:aux:tst}
%\item Entry type alias \bibtype{conference} now resolved by \bibtex\see{bib:typ:als}
%\item Entry type alias \bibtype{mastersthesis} now resolved by \bibtex\see{bib:typ:als}
%\item Entry type alias \bibtype{phdthesis} now resolved by \bibtex\see{bib:typ:als}
%\item Entry type alias \bibtype{techreport} now resolved by \bibtex\see{bib:typ:als}
%\item Entry type alias \bibtype{www} now resolved by \bibtex\see{bib:typ:als}
%\item Added new custom fields \bibfield{lista} through \bibfield{listf}\see{bib:fld:ctm}
%\item Changed data type of \bibfield{location}\see{bib:fld:dat}
%\item Changed data type of \bibfield{origlocation}\see{bib:fld:dat}
%\item Changed data type of \bibfield{publisher}\see{bib:fld:dat}
%\item Changed data type of \bibfield{institution}\see{bib:fld:dat}
%\item Changed data type of \bibfield{organization}\see{bib:fld:dat}
%\item Modified values of \bibfield{gender} field for \sty{jurabib} compatibility\see{bib:fld:spc}
%\item Modified and extended \texttt{idem\dots} keys for \sty{jurabib} compatibility\see{aut:lng:key}
%\item Improved \cmd{addtocategory}\see{use:bib:cat}
%\item Removed formatting command \cmd{mkshorthand}\see{use:fmt:fmt}
%\item Added field formatting directive \texttt{shorthandwidth}\see{aut:fmt:ich}
%\item Added documentation of \cmd{shorthandwidth}\see{aut:fmt:ilc}
%\item Removed formatting command \cmd{mklabelnumber}\see{use:fmt:fmt}
%\item Added field formatting directive \texttt{labelnumberwidth}\see{aut:fmt:ich}
%\item Added documentation of \cmd{labelnumberwidth}\see{aut:fmt:ilc}
%\item Removed formatting command \cmd{mklabelalpha}\see{use:fmt:fmt}
%\item Added field formatting directive \texttt{labelalphawidth}\see{aut:fmt:ich}
%\item Added documentation of \cmd{labelalphawidth}\see{aut:fmt:ilc}
%\item Renamed \cmd{bibitem} to \cmd{thebibitem}\see{aut:bbx:env}
%\item Renamed \cmd{lositem} to \cmd{thelositem}\see{aut:bbx:env}
%\item Modified \cmd{AtBeginBibliography}\see{aut:fmt:hok}
%\item Added \cmd{AtBeginShorthands}\see{aut:fmt:hok}
%\item Added \cmd{AtEveryLositem}\see{aut:fmt:hok}
%\item Extended \sty{showkeys} compatibility to list of shorthands\see{int:pre:cmp}
%\item Added compatibility code for the \sty{hyperref} package\see{int:pre:cmp}
%\item Added package option \opt{hyperref}\see{use:opt:pre:gen}
%\item Added package option \opt{backref}\see{use:opt:pre:gen}
%\item Added field \bibfield{pageref}\see{aut:bbx:fld}
%\item Added \cmd{ifhyperref}\see{aut:aux:msc}
%\item Added \cmd{bibhyperlink}\see{aut:aux:msc}
%\item Added field formatting directive \texttt{bibhyperlink}\see{aut:fmt:ich}
%\item Renamed \cmd{ifandothers} to \cmd{ifmorenames}\see{aut:aux:tst}
%\item Added new \cmd{ifandothers} test\see{aut:aux:tst}
%\item Removed field \bibfield{moreauthor}\see{aut:bbx:fld}
%\item Removed field \bibfield{morebookauthor}\see{aut:bbx:fld}
%\item Removed field \bibfield{moreeditor}\see{aut:bbx:fld}
%\item Removed field \bibfield{morelabelname}\see{aut:bbx:fld}
%\item Removed field \bibfield{moretranslator}\see{aut:bbx:fld}
%\item Removed field \bibfield{morenamea}\see{aut:bbx:fld}
%\item Removed field \bibfield{morenameb}\see{aut:bbx:fld}
%\item Removed field \bibfield{morenamec}\see{aut:bbx:fld}
%\item Updated documentation\see{aut:int}
%\item Updated documentation\see{aut:bbx:bbx}
%\item Updated documentation\see{aut:bbx:env}
%\item Updated documentation\see{aut:bbx:drv}
%\item Expanded documentation\see{aut:fmt}
%\item Modified internal \bibtex interface
%\item Fixed some typos in the manual
%\item Fixed some bugs
%
%\end{release}
%
%\begin{release}{0.2}{2006-09-06}
%
%\item Added bibliography categories\see{use:bib:cat}
%\item Added \cmd{DeclareBibliographyCategory}\see{use:bib:cat}
%\item Added \cmd{addtocategory}\see{use:bib:cat}
%\item Added \texttt{category} and \texttt{notcategory} filters\see{use:bib:bib}
%\item Added \cmd{bibbycategory}\see{use:bib:bib}
%\item Added usage examples for bibliography categories\see{use:use:div}
%\item Added documentation of configuration file\see{use:cfg:cfg}
%\item Added documentation of \cmd{ExecuteBibliographyOptions}\see{use:cfg:opt}
%\item Added documentation of \cmd{AtBeginBibliography}\see{aut:fmt:hok}
%\item Added \cmd{AtEveryBibitem}\see{aut:fmt:hok}
%\item Added \cmd{AtEveryCite}\see{aut:fmt:hok}
%\item Added \cmd{AtEveryCitekey}\see{aut:fmt:hok}
%\item Added optional argument to \cmd{printtext}\see{aut:bib:dat}
%\item Added \cmd{ifpage}\see{aut:aux:tst}
%\item Added \cmd{ifpages}\see{aut:aux:tst}
%\item Added field \bibfield{titleaddon}\see{bib:fld:dat}
%\item Added field \bibfield{booktitleaddon}\see{bib:fld:dat}
%\item Added field \bibfield{maintitleaddon}\see{bib:fld:dat}
%\item Added field \bibfield{library}\see{bib:fld:dat}
%\item Added field \bibfield{part}\see{bib:fld:dat}
%\item Added field \bibfield{origlocation}\see{bib:fld:dat}
%\item Added field \bibfield{origtitle}\see{bib:fld:dat}
%\item Added field \bibfield{origyear}\see{bib:fld:dat}
%\item Added field \bibfield{origlanguage}\see{bib:fld:dat}
%\item Modified profile of field \bibfield{language}\see{bib:fld:dat}
%\item Extended entry type \bibtype{book}\see{bib:typ:blx}
%\item Extended entry type \bibtype{inbook}\see{bib:typ:blx}
%\item Extended entry type \bibtype{collection}\see{bib:typ:blx}
%\item Extended entry type \bibtype{incollection}\see{bib:typ:blx}
%\item Extended entry type \bibtype{proceedings}\see{bib:typ:blx}
%\item Extended entry type \bibtype{inproceedings}\see{bib:typ:blx}
%\item Added entry type alias \bibtype{www}\see{bib:typ:als}
%\item Added compatibility code for the \sty{showkeys} package\see{int:pre:cmp}
%\item Support printable characters in \bibfield{keyword} and \texttt{notkeyword} filters\see{use:bib:bib}
%\item Support printable characters in \bibfield{keywords} field\see{bib:fld:spc}
%\item Ignore spaces after commas in \bibfield{keywords} field\see{bib:fld:spc}
%\item Internal rearrangement of all bibliography styles
%\item Fixed various bugs
%
%\end{release}
%
%\begin{release}{0.1}{2006-09-02}
%
%\item Initial public release
%
%\end{release}

\end{changelog} 

\printglossary[type=trans,style=longragged3colborder,nopostdot=true,nogroupskip]%long3colborder

\section{后记}

\setlength{\parskip}{2mm}\setlength{\itemindent}{2em}

之所以摘译biblatex,一是出于对keep texing的兴趣,二是考虑到中文资料里总体介绍latex的资料其实较多,
反而是在一些专项部分中文资料较少,看宏包的英文说明当然没有问题,但有的文档内容长达几百页,一般用户真心没有精力去看,即便是找一些自己需要的功能也比较麻烦,所以考虑对参考文献的biblatex宏包文档进行摘译,是希望能在这一方面有所贡献。关于这一点Wenbo兄也很有同感,在几年前biblatex还是2.x版本的时候他就深入研究了biblatex并翻译了1--4节很多内容。我在biblatex-gb7714-2015样式宏包中提议翻译biblatex文档之后,我们决定合作来推进这个事情,尽管都只有部分业余时间可以利用,但我们认为只要有空就积累一点,那么终有完成的时候。

如同我在biblatex-gb7714-2015样式宏包说明文档中介绍的那样,biblatex宏包具有很多强大功能比如参考文献表划分、文献集、样式定制、动态数据处理等等,在科技论文或书籍写作中特别有用(尤其是在对参考文献著录和标注格式有特殊要求的情况下)。可以说,biblatex是latex文档写作中参考文献问题的完整解决方案,也一定程度上代表了这一方面的未来趋势。总之,本项目的总体任务是完成biblatex宏包说明文档关键内容的摘译,希望能对使用biblatex和对参考文献样式有深度定制要求的用户有所帮助,当然也希望能使中文latex资料库更为全面和深入。需要说明的是,限于作者水平,其中难免存在一些错误和理解不到位的地方,欢迎批评指正,欢迎@译者邮箱。最后感谢CTEX和Latexstudio论坛,感谢论坛上各位作者关于biblatex参考文献方面的工作分享和经验介绍。

\end{document}
%%% Local Variables:
%%% coding: utf-8
%%% eval: (auto-fill-mode -1)
%%% eval: (visual-line-mode)
%%% End:
