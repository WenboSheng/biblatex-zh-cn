% !TeX encoding = UTF-8
% UserGuide.tex

%\section{User Guide}
\section{用户使用手册}
\label{use}

%This part of the manual documents the user interface of the \biblatex package. The user guide covers everything you need to know in order to use \biblatex with the default styles that come with this package. You should read the user guide first in any case. If you want to write your own citation and\slash or bibliography styles, continue with the author guide afterwards.

本部分介绍了 \biblatex 宏包的用户接口,
涵盖要使用 \biblatex 自带的标准样式所需的所有信息。
无论如何,用户都应首先阅读这一部分内容。
如果你想编写你自己的著录和标注(引用)样式,请继续阅读随后的作者指南。

\subsection{宏包选项}
%Package Options
\label{use:opt}

%All package options are given in \keyval notation. The value \texttt{true} is omissible with all boolean keys. For example, giving \opt{sortcites} without a value is equivalent to \kvopt{sortcites}{true}.

所有的宏包选项都以 \keyval 记法给出。
对于所有的布尔型键值,\texttt{true} 是可以省略的。
例如,给出不带值的 \opt{sortcites} 等价于 \kvopt{sortcites}{true}。

\subsubsection{载入时选项}
%Load-time Options
\label{use:opt:ldt}

%The following options must be used as \biblatex is loaded, \ie in the optional argument to \cmd{usepackage}.

以下的选项必须在 \biblatex 载入时使用,即作为 \cmd{usepackage} 的可选参数。

\begin{optionlist}

\optitem[biber]{backend}{\opt{bibtex}, \opt{bibtex8}, \opt{biber}}

%Specifies the database backend. The following backends are supported:

指定数据库后端程序。支持以下后端:

\begin{valuelist}
	
\item[biber] %	\biber, the default backend of \biblatex, supports Ascii, 8-bit encodings, \utf, on-the-fly reencoding, locale"=specific sorting, and many other features. Locale"=specific sorting, case"=sensitive sorting, and upper\slash lowercase precedence are controlled by the options \opt{sortlocale}, \opt{sortcase}, and \opt{sortupper}, respectively.
\biber{},\biblatex 的默认后端程序,支持Ascii、8比特编码、\utf{}、实时重新编码、本地化定制排序,及许多其它特性。
本地化定制排序,大小写敏感排序和大小写优先排序分别由选项 \opt{sortlocale}、\opt{sortcase} 和 \opt{sortupper} 控制。

\item[bibtex] %Legacy \bibtex. Traditional \bibtex supports Ascii encoding only. Sorting is always case"=insensitive.
遗留下来的 \BibTeX{}。传统的 \BibTeX 只支持Ascii编码,并且排序总是大小写不敏感的。
	
\item[bibtex8] %\bin{bibtex8}, the 8-bit implementation of \bibtex, supports Ascii and 8-bit encodings such as Latin~1.
\bin{bibtex8} 是 \BibTeX 的8比特实现,支持Ascii和Latin~1等8比特编码。

\end{valuelist}

%See \secref{use:bibtex} for details of using \bibtex as a backend.
使用 \BibTeX 作为后端的细节见 \secref{use:bibtex} 节。

\valitem[numeric]{style}{file}

%Loads the bibliography style \prm{file}\path{.bbx} and the citation style \prm{file}\path{.cbx}. See \secref{use:xbx} for an overview of the standard styles.

加载著录样式文件 \prm{file}\path{.bbx} 和标注(引用)样式文件 \prm{file}\path{.cbx}。
标准样式概览见 \secref{use:xbx} 节。

\valitem[numeric]{bibstyle}{file}

%Loads the bibliography style \prm{file}\path{.bbx}. See \secref{use:xbx:bbx} for an overview of the standard bibliography styles.

加载著录样式文件 \prm{file}\path{.bbx}。
标准著录样式概览见 \secref{use:xbx:bbx} 节。

\valitem[numeric]{citestyle}{file}

%Loads the citation style \prm{file}\path{.cbx}. See \secref{use:xbx:cbx} for an overview of the standard citation styles.

加载标注(引用)样式文件 \prm{file}\path{.cbx}。
标准标注样式概览见 \secref{use:xbx:cbx} 节。

\boolitem[false]{natbib}

%Loads compatibility module which provides aliases for the citation commands of the \sty{natbib} package. See \secref{use:cit:nat} for details.

加载兼容性模块,该模块提供 \sty{natbib} 宏包引用命令的同名替代命令。
详见 \secref{use:cit:nat} 节。

\boolitem[false]{mcite}

%Loads a citation module which provides \sty{mcite}\slash\sty{mciteplus}-like citation commands. See \secref{use:cit:mct} for details.

加载一个标注(引用)模块,该模块提供 \sty{mcite}\slash 类 \sty{mciteplus} 的标注(引用)命令。
详见 \secref{use:cit:mct} 节。

\end{optionlist}

\subsubsection{导言区选项}
%Preamble Options
\label{use:opt:pre}

\paragraph{一般选项}
%General
\label{use:opt:pre:gen}

%The following options may be used in the optional argument to \cmd{usepackage} as well as in the configuration file and the document preamble. The default value listed to the right is the package default. Note that bibliography and citation styles may modify the default setting at load time, see \secref{use:xbx} for details.

下列选项既可以作为 \cmd{usepackage} 的可选项,也可以在配置文件和导言区中使用。
右侧列出的是选项的宏包默认值。
请注意,著录和标注样式可以在载入时修改默认设置,详见 \secref{use:xbx} 节。

\begin{optionlist}

\optitem[nty]{sorting}{\opt{nty}, \opt{nyt}, \opt{nyvt}, \opt{anyt}, \opt{anyvt}, \opt{ynt}, \opt{ydnt}, \opt{none}, \opt{debug}, \prm{name}}

%The sorting order of the bibliography. Unless stated otherwise, the entries are sorted in ascending order. The following choices are available by default:
参考文献的排序方式。除非另加说明,文献条目总是按升序排列。
默认提供以下可选值:

\begin{valuelist}
\item[nty] %Sort by name, title, year.
按照姓名、标题、年份排序。
\item[nyt] %Sort by name, year, title.
按照姓名、年份、标题排序。
\item[nyvt] %Sort by name, year, volume, title.
按照姓名、年份、卷数、标题排序。
\item[anyt] %Sort by alphabetic label, name, year, title.
按照字母标签、姓名、年份、标题排序。
\item[anyvt] %Sort by alphabetic label, name, year, volume, title.
按照字母标签、姓名、年份、卷数、标题排序。
\item[ynt] %Sort by year, name, title.
按照年份、姓名、标题排序。
\item[ydnt] %Sort by year (descending), name, title.
按照年份(降序)、姓名、标题排序。
\item[none] %Do not sort at all. All entries are processed in citation order.
不进行排序。所有的条目按照引用顺序处理。
\item[debug] %Sort by entry key. This is intended for debugging only.
按照条目键值排列。该选项只用于程序调试。
\item[\prm{name}] %Use \prm{name}, as defined with \cmd{DeclareSortingTemplate} (\secref{aut:ctm:srt})
使用由 \cmd{DeclareSortingTemplate}(\secref{aut:ctm:srt})定义的 \prm{name}。
\end{valuelist}

%Using any of the <alphabetic> sorting templates only makes sense in conjunction with a bibliography style which prints the corresponding labels. Note that some bibliography styles initialize this package option to a value different from the package default (\opt{nty}). See \secref{use:xbx:bbx} for details. Please refer to \secref{use:srt} for an in"=depth explanation of the above sorting options as well as the fields considered in the sorting process. See also \secref{aut:ctm:srt} on how to adapt the predefined templates or define new ones.

只有与打印相应标签的参考文献样式配合使用,“字母顺序”排序格式才有意义。
请注意,一些参考文献样式会将宏包选项从宏包的默认值(\opt{nty})初始化为另外一个值。
详见 \secref{use:xbx:bbx} 节。
关于以上排序选项的深度解读以及排序过程中涉及的域,请参考 \secref{use:srt} 节。
关于如何调整预定义格式或者定义新格式也可参考 \secref{aut:ctm:srt}。

\boolitem[true]{sortcase}

%Whether or not to sort the bibliography and the list of shorthands case"=sensitively.

文献和缩略语列表的排序是否是大小写敏感的。

\boolitem[true]{sortupper}

%This option corresponds to \biber's \opt{--sortupper} command-line option. If enabled, the bibliography is sorted in <uppercase before lowercase> order. Disabling this option means <lowercase before uppercase> order.

对应于 \biber 的 |--sortupper| 命令行选项。
当该选项激活时,文献表会按照“大写在前小写在后”的顺序排列。
关闭该选项则使用“小写在前大写在后”的顺序。

\optitem{sortlocale}{\opt{auto}, \prm{locale}}

%This option sets the global sorting locale. Every sorting template inherits this locale if none is specified using the \prm{locale} option to \cmd{printbibliography}. Setting this to \opt{auto} requests that it be set to the \sty{babel}/\sty{polyglossia} main document language identifier, if these packages are used and \texttt{en\_US} otherwise. \biber will map \sty{babel}/\sty{polyglossia} language identifiers into sensible locale identifiers (see the \biber documentation). You can therefore specify either a normal locale identifier like \texttt{de\_DE\_phonebook}, \texttt{es\_ES} or one of the supported \sty{babel}/\sty{polyglossia} language identifiers if the mapping \biber makes of this is fine for you.
该选项设置全局的排序区域语言(本地语言)。
只要\cmd{printbibliography}时未对 \prm{locale} 选项做设置,那么各排序格式都会继承该区域语言设置。
当该选项值设置为 \opt{auto},如果使用 \sty{babel}/\sty{polyglossia} 宏包,那么\prm{locale}选项会设置为主文档语言的标识符,否则\prm{locale}将设置为 \texttt{en\_US}。
\biber 会将 \sty{babel}/\sty{polyglossia} 语言标识符映射为有意义的本地化标识符(参考 \biber 文档)。
因此,你可以指定标准的本地化标识符,例如 \texttt{de\_DE\_phonebook}、\texttt{es\_ES};
或者指定本宏包支持的 \sty{babel}/\sty{polyglossia} 语言标识符,如果你对\biber 的语言映射还满意的话。

\boolitem[true]{related}

%Whether or not to use information from related entries or not. See \secref{use:rel}.

是否使用来自关联条目的信息。参考 \secref{use:rel} 节。

\boolitem[false]{sortcites}

%Whether or not to sort citations if multiple entry keys are passed to a citation command. If this option is enabled, citations are sorted according to the current bibliography context sorting scheme (see \secref{use:bib:context}). This feature works with all citation styles.

当多个条目的引用关键字传给一个标注(引用)命令时,是否对标注进行排序。
如果该选项激活,就会根据当前的文献表的排序格式(见 \secref{use:bib:context} 节)对标注进行排序。
该功能对所有的标注样式都有效。

\boolitem[false]{sortsets}

是否根据当前参考文献文境排序格式对条目集成员排序。默认设置为false,条目集成员以数据源给出的顺序排列。
%Whether or not to sort set members according to the active reference context sorting scheme. By default this is false and set members appear in the order given in the data source.


\intitem[3]{maxnames}

%A threshold affecting all lists of names (\bibfield{author}, \bibfield{editor}, etc.). If a list exceeds this threshold, \ie if it holds more than \prm{integer} names, it is automatically truncated according to the setting of the \opt{minnames} option. \opt{maxnames} is the master option which sets both \opt{maxbibnames} and \opt{maxcitenames}.
影响所有名称列表(\bibfield{author}、\bibfield{editor} 等)的阈值。如果某个列表超过了该阈值,即,它包含的姓名数量超过 \prm{integer},那么就会根据 \opt{minnames} 选项的设置进行自动截断。\opt{maxnames} 是设置 \opt{maxbibnames} 和 \opt{maxcitenames} 两个选项的支配选项。

\intitem[1]{minnames}

%A limit affecting all lists of names (\bibfield{author}, \bibfield{editor}, etc.). If a list holds more than \prm{maxnames} names, it is automatically truncated to \prm{minnames} names. The \prm{minnames} value must be smaller than or equal to \prm{maxnames}. \opt{minnames} is the master option which sets both \opt{minbibnames} and \opt{mincitenames}.
影响所有名称列表(\bibfield{author}、\bibfield{editor} 等)的限制值。如果某个列表包含的姓名数量超过 \prm{integer},
那么就会自动截断至\opt{minnames}个姓名。\prm{minnames} 的值必须小于或等于 \prm{maxnames}。
\opt{minnames} 是设置 \opt{minbibnames} 和 \opt{mincitenames} 两个选项的支配选项。

\intitem[\prm{maxnames}]{maxbibnames}

%Similar to \opt{maxnames} but affects only the bibliography.

类似于 \opt{maxnames} 但只影响参考文献表。

\intitem[\prm{minnames}]{minbibnames}

%Similar to \opt{minnames} but affects only the bibliography.

类似于  \opt{minnames} 但只影响参考文献表。

\intitem[\prm{maxnames}]{maxcitenames}

%Similar to \opt{maxnames} but affects only the citations in the document body.

类似于 \opt{maxnames} 但只影响正文中的标注(引用)。

\intitem[\prm{minnames}]{mincitenames}

%Similar to \opt{minnames} but affects only the citations in the document body.

类似于 \opt{minnames} 但只影响正文中的标注(引用)。

\intitem[3]{maxitems}

%Similar to \opt{maxnames}, but affecting all literal lists (\bibfield{publisher}, \bibfield{location}, etc.).

类似于 \opt{maxnames} 但影响所有的文本列表(\bibfield{publisher}、\bibfield{location}等)。

\intitem[1]{minitems}

%Similar to \opt{minnames}, but affecting all literal lists (\bibfield{publisher}, \bibfield{location}, etc.).

类似于 \opt{minnames} 但影响所有的文本列表(\bibfield{publisher}、\bibfield{location}等)。

\optitem{autocite}{\opt{plain}, \opt{inline}, \opt{footnote}, \opt{superscript}, \opt{...}}

%This option controls the behavior of the \cmd{autocite} command discussed in \secref{use:cit:aut}. The \opt{plain} option makes \cmd{autocite} behave like \cmd{cite}, \opt{inline} makes it behave like \cmd{parencite}, \opt{footnote} makes it behave like \cmd{footcite}, and \opt{superscript} makes it behave like \cmd{supercite}. The options \opt{plain}, \opt{inline}, and \opt{footnote} are always available, the \opt{superscript} option is only provided by the numeric citation styles which come with this package. The citation style may also define additional options. The default setting of this option depends on the selected citation style, see \secref{use:xbx:cbx}.
该选项控制 \secref{use:cit:aut} 节中讨论的 \cmd{autocite} 命令的行为。\opt{plain} 选项使得 \cmd{autocite} 相当于 \cmd{cite};\opt{inline} 选项使得它相当于 \cmd{parencite};\opt{footnote} 选项使得它相当于 \cmd{footcite};\opt{superscript} 选项使得它相当于 \cmd{supercite}。
选项 \opt{plain}、\opt{inline} 和 \opt{footnote} 总是可用的,而  \opt{superscript} 只能用于本宏包所带的顺序编码制标注样式中。标注样式也可以定义其它选项值。该选项的默认设置取决于所选的标注样式,参考 \secref{use:xbx:cbx} 节。

\boolitem[true]{autopunct}

%This option controls whether the citation commands scan ahead for punctuation marks. See \secref{use:cit} and \cmd{DeclareAutoPunctuation} in \secref{aut:pct:cfg} for details.

该选项控制标注(引用)命令是否区分标点。详见 \secref{use:cit} 节和 \secref{aut:pct:cfg} 节中的命令 \cmd{DeclareAutoPunctuation}。

\optitem[autobib]{language}{\opt{autobib}, \opt{autocite}, \opt{auto}, \prm{language}}

%This option controls multilingual support. When set to \opt{autobib}, \opt{autocite} or \opt{auto}, \biblatex will try to get the main document language from the \sty{babel}/\sty{polyglossia} package (and fall back to English if \sty{babel}/\sty{polyglossia} is not available). It is also possible to select the package language manually. In this case, the language chosen will override the \bibfield{langid} of entries and you should still choose a language switching environment with the \opt{autolang} option to select how the switch to the manually chosen language is handled. Please refer to \tabref{bib:fld:tab1} for a list of supported languages and the corresponding identifiers. \opt{autobib} switches the language for each entry in the bibliography using the \bibfield{langid} field and the language environment specified by the \opt{autolang} option. \opt{autocite} switches the language for each citation using the \bibfield{langid} field and the language environment specified by the \opt{autolang} option. \opt{auto} is a shorthand to set both \opt{autobib} and \opt{autocite}.
该选项控制多语种支持功能。当其设置为 \opt{autobib}、\opt{autocite} 或 \opt{auto} 时,\biblatex 会尝试从 \sty{babel}/\sty{polyglossia} 宏包中获取文档的主语言(如果\sty{babel}/\sty{polyglossia} 宏包不可用则设置为后备的英语)。也可以手动选择文档的语言,手动选择的语言会覆盖条目的 \bibfield{langid} 域,并且仍需要使用 \opt{autolang} 选项选择语言切换环境,以确定如何处理手动选择语言的切换方式。关于所支持的语言和相应的标识符请参考 \tabref{bib:fld:tab1}。
\opt{autobib} 使用条目中的 \bibfield{langid} 域和 \opt{autolang} 选项确定的语言环境为文献表中每个条目切换语言。
\opt{autocite} 使用条目中的 \bibfield{langid} 域和 \opt{autolang} 选项确定的语言环境为每个标注(引用)切换语言。
而 \opt{auto} 是同时设置 \opt{autobib} 和 \opt{autocite} 的缩略语。

\boolitem[true]{clearlang}

%If this option is enabled, \biblatex will automatically clear the \bibfield{language} field of all entries whose language matches the \sty{babel}/\sty{polyglossia} language of the document (or the language specified explicitly with the \opt{language} option) in order to omit redundant language specifications. The language mappings required by this feature are provided by the \cmd{DeclareRedundantLanguages} command from \secref{aut:lng:cmd}.

如果激活该选项,\biblatex 会自动清除那些与文档的 \sty{babel}/\sty{polyglossia} 语言(或者由 \opt{language} 选项显式指定的语言)相匹配的所有条目的 \bibfield{language} 域,以便略去冗余的语言设定。
该功能所需的语言映射由 \secref{aut:lng:cmd} 节的 \cmd{DeclareRedundantLanguages} 命令提供。

\optitem[none]{autolang}{\opt{none}, \opt{hyphen}, \opt{other}, \opt{other*}, \opt{langname}}

%This option controls which \sty{babel} language environment\footnote{\sty{polyglossia} understands the \sty{babel} language environments too and so this option controls both the \sty{babel} and \sty{polyglossia} language environments.} is used if the \sty{babel}/\sty{polyglossia} package is loaded and a bibliography entry includes a \bibfield{langid} field (see \secref{bib:fld:spc}). Note that \biblatex automatically adjusts to the main document language if \sty{babel}/\sty{polyglossia} is loaded. In multilingual documents, it will also continually adjust to the current language as far as citations and the default language of the bibliography is concerned. The effect of language adjustment depends on the language environment selected by this option. The possible choices are:
如果载入了 \sty{babel}/\sty{polyglossia} 宏包并且文献条目包含 \bibfield{langid} 域
(见 \secref{bib:fld:spc} 节),那么该选项可以控制使用哪种 \sty{babel} 语言环境
\footnote{	\sty{polyglossia} 宏包也可以理解 \sty{babel} 的语言环境,	因此本选项可以同时控制 \sty{babel}/\sty{polyglossia} 语言环境。}。
请注意,当载入 \sty{babel}/\sty{polyglossia} 宏包时 \biblatex 会自动适应主文档的语言。
在多语言文档中,只要涉及到标注(引用)和文献表的默认语言,本宏包也会持续地调整以适应当前语言。
切换语言的效果取决于该选项选择的语言环境,可用的选择有:

\begin{valuelist}

\item[none]
%Disable this feature, \ie do not use any language environment at all.

关闭该功能,也就是不使用任何语言环境。

\item[hyphen]
%Enclose the entry in a \env{hyphenrules} environment. This will load hyphenation patterns for the language specified in the \env{hyphenation} field of the entry, if available.
将条目装入  \env{hyphenrules} 环境中。如果可用的话,会为条目的 \env{hyphenation} 域指定的语言导入断词模式。

\item[other]
%Enclose the entry in an \env{otherlanguage} environment. This will load hyphenation patterns for the specified language, enable all extra definitions which \sty{babel}/\sty{polyglossia} and \biblatex provide for the respective language, and translate key terms such as <editor> and <volume>. The extra definitions include localisations of the date format, of ordinals, and similar things.

将条目装入 \env{otherlanguage} 环境中。
这将导入特定语言的断词模式,激活 \sty{babel}/\sty{polyglossia} 和 \biblatex 为相应语言提供的所有额外定义,并翻译“editor”和“volume”等键项。这些额外定义包括日期格式、序数和其它类似内容的本地化。

\item[other*]
%Enclose the entry in an \env{otherlanguage*} environment. Please note that \sty{biblatex} treats \env{otherlanguage*} like \env{otherlanguage} but other packages may make a distinction in this case.
将条目装入 \env{otherlanguage*} 环境中。
请注意,此时 \biblatex 将 \env{otherlanguage*} 环境视为 \env{otherlanguage} 环境但其它宏包不会。

\item[langname]
%\sty{polyglossia} only. Enclose the entry in a \env{$<$languagename$>$} environment. The benefit of this option value for \sty{polyglossia} users is that it takes note of the \bibfield{langidopts} field so that you can add per-language options to an entry (like selecting a language variant). When using \sty{babel}, this option does the same as the \opt{other} option value.
只用于 \sty{polyglossia} 宏包。将条目装入 \env{\prm{languagename}} 环境中。
该选项值对 \sty{polyglossia} 用户的好处是它会留意 \bibfield{langidopts} 域,
这样可以为一个条目增加语言相关选项(类似于选择语言变种)。当使用 \sty{babel} 时,该选项值与 \opt{other} 选项值相同。

\end{valuelist}

\optitem[none]{block}{\opt{none}, \opt{space}, \opt{par}, \opt{nbpar}, \opt{ragged}}

%This option controls the extra spacing between blocks, \ie larger segments of a bibliography entry. The possible choices are:
该选项用于控制块之间的额外空白,即文献表中条目内大的分段(块(block)相比于单元(unit)更大)。可用的选项值有:

\begin{valuelist}

\item[none] %Do not add anything at all.
不添加任何东西。

\item[space] %Insert additional horizontal space between blocks. This is similar to the default behavior of the standard \latex document classes.
在块之间添加额外的水平间距,类似于标准 \LaTeX 文档类的默认行为。

\item[par] %Start a new paragraph for every block. This is similar to the \opt{openbib} option of the standard \latex document classes.
每一块都另起一段,类似于标准 \LaTeX 文档类的 \opt{openbib} 选项。

\item[nbpar] %Similar to the \opt{par} option, but disallows page breaks at block boundaries and within an entry.
类似于 \opt{par} 选项但是不允许在块的边界和条目内部分页。

\item[ragged] %Inserts a small negative penalty to encourage line breaks at block boundaries and sets the bibliography ragged right.
插入一个小的负的断行罚值以鼓励在块边界处断行并设置左对齐。

\end{valuelist}

%The \cmd{newblockpunct} command may also be redefined directly to achieve different results, see \secref{use:fmt:fmt}. Also see \secref{aut:pct:new} for additional information.

也可以直接重新定义 \cmd{newblockpunct} 命令实现不同的效果,见  \secref{use:fmt:fmt} 节。
更多信息见 \secref{aut:pct:new} 节。

\optitem[foot+end]{notetype}{\opt{foot+end}, \opt{footonly}, \opt{endonly}}

%This option controls the behavior of \cmd{mkbibfootnote}, \cmd{mkbibendnote}, and similar wrappers from \secref{aut:fmt:ich}. The possible choices are:

该选项控制 \secref{aut:fmt:ich} 节中 \cmd{mkbibfootnote}、\cmd{mkbibendnote} 和类似封套的行为。
可用的选项值有:

\begin{valuelist}
\item[foot+end] %Support both footnotes and endnotes, \ie \cmd{mkbibfootnote} will generate footnotes and \cmd{mkbibendnote} will generate endnotes.
同时支持脚注和尾注,即,\cmd{mkbibfootnote} 会生成脚注而 \cmd{mkbibendnote} 会生成尾注。
\item[footonly] %Force footnotes, \ie make \cmd{mkbibendnote} generate footnotes.
强制脚注,即,\cmd{mkbibendnote} 也生成脚注。
\item[endonly] %Force endnotes, \ie make \cmd{mkbibfootnote} generate endnotes.
强制尾注,即,\cmd{mkbibfootnote} 也生成尾注。
\end{valuelist}

\optitem[auto]{hyperref}{\opt{true}, \opt{false}, \opt{auto}}

%Whether or not to transform citations and back references into clickable hyperlinks. This feature requires the \sty{hyperref} package. It also requires support by the selected citation style. All standard styles which ship with this package support hyperlinks. \kvopt{hyperref}{auto} automatically detects if the \sty{hyperref} package has been loaded.
是否将标注(引用)和反向引用转化为可点击的超链接。这一功能需要 \sty{hyperref} 宏包。也需要标注样式的支持。
本宏包所带的所有标准样式都支持超链接。\kvopt{hyperref}{auto} 会自动检测 \sty{hyperref} 宏包是否已载入。

\boolitem[false]{backref}

%Whether or not to print back references in the bibliography. The back references are a list of page numbers indicating the pages on which the respective bibliography entry is cited. If there are \env{refsection} environments in the document, the back references are local to the reference sections. Strictly speaking, this option only controls whether the \biblatex package collects the data required to print such references. This feature still has to be supported by the selected bibliography style. All standard styles which ship with this package do so.
是否在文献表中打印出反向引用。反向引用是一组标明引用相应文献所在页码构成的列表。
如果在文档中有 \env{refsection} 环境,反向引用是局部的,针对相应的参考文献分节(refsection)。
严格地讲,该选项只控制 \biblatex 是否收集所需的数据。该功能也需要文献样式的支持。
本宏包所带的所有标准样式都支持该功能。

\optitem[three]{backrefstyle}{\opt{none}, \opt{three}, \opt{two}, \opt{two+}, \opt{three+}, \opt{all+}}

%This option controls how sequences of consecutive pages in the list of back references are formatted. The following styles are available:
该选项控制反向引用中的连续页码的格式。可用的样式有:

\begin{valuelist}

\item[none] %Disable this feature, \ie do not compress the page list.
不启用该特性,即,不压缩页码列表。

\item[three] %Compress any sequence of three or more consecutive pages to a range, \eg the list <1, 2, 11, 12, 13, 21, 22, 23, 24> is compressed to <1, 2, 11--13, 21--24>.
将任意连续三页或更多页缩写为页码范围,例如,“1, 2, 11, 12, 13, 21, 22, 23, 24” 会压缩成“1, 2, 11--13, 21--24”。

\item[two] %Compress any sequence of two or more consecutive pages to a range, \eg the above list is compressed to <1--2, 11--13, 21--24>.
将任意连续两页或更多页缩写成页码范围,例如上面的例子会变成“1--2, 11--13, 21--24”。

\item[two+] %Similar in concept to \opt{two} but a sequence of exactly two consecutive pages is printed using the starting page and the localization string \texttt{sequens}, \eg the above list is compressed to <1\,sq., 11--13, 21--24>.
概念类似于 \opt{two},但是连续两页的打印格式为开始页和本地化字符串 \texttt{sequens},例如上面的例子会变成“1\,sq., 11--13, 21--24”。

\item[three+] %Similar in concept to \opt{two+} but a sequence of exactly three consecutive pages is printed using the starting page and the localization string \texttt{sequentes}, \eg the above list is compressed to <1\,sq., 11\,sqq., 21--24>.
概念类似于 \opt{two+},但是连续三页的打印格式为开始页可本地化字符串 \texttt{sequentes},例如上面的例子会变成“1\,sq., 11\,sqq., 21--24”。

\item[all+] %Similar in concept to \opt{three+} but any sequence of consecutive pages is printed as an open-ended range, \eg the above list is compressed to <1\,sq., 11\,sqq., 21\,sqq.>.
概念类似于 \opt{three+},但是任意连续页码将打印成末端不封闭的形式,例如上面的例子会变成“1\,sq., 11\,sqq., 21\,sqq.”。

\end{valuelist}

%All styles support both Arabic and Roman numerals. In order to avoid potentially ambiguous lists, different sets of numerals will not be mixed when generating ranges, \eg the list <iii, iv, v, 6, 7, 8> is compressed to <iii--v, 6--8>.
所有这些样式都同时支持阿拉伯和罗马数字。为了避免可能的歧义,不同数字集在生成数字范围时不会混在一起,
例如,“iii, iv, v, 6, 7, 8”将缩写为“iii--v, 6--8”。

\optitem[setonly]{backrefsetstyle}{\opt{setonly}, \opt{memonly}, \opt{setormem}, \opt{setandmem}, \opt{memandset}, \opt{setplusmem}}

%This option controls how back references to \bibtype{set} entries and their members are handled. The following options are available:
该选项控制怎样处理指向 \bibtype{set} 条目及其成员的反向引用。可用选项有:

\begin{valuelist}

\item[setonly] %All back references are added to the \bibtype{set} entry. The \bibfield{pageref} lists of set members remain blank.
所有的反向引用都添加到 \bibtype{set} 条目中。
而其成员的 \bibfield{pageref} 列表为空。

\item[memonly] %References to set members are added to the respective member. References to the \bibtype{set} entry are added to all members. The \bibfield{pageref} list of the \bibtype{set} entry remains blank.
条目集成员的引用添加到各自成员中。
\bibtype{set} 条目的引用添加到所有成员中。
\bibtype{set} 条目的 \bibfield{pageref} 列表为空。

\item[setormem] %References to the \bibtype{set} entry are added to the \bibtype{set} entry. References to set members are added to the respective member.
\bibtype{set} 条目的引用添加到 \bibtype{set} 条目中。
其成员的引用添加到各自成员中。

\item[setandmem] %References to the \bibtype{set} entry are added to the \bibtype{set} entry. References to set members are added to the respective member and to the \bibtype{set} entry.
\bibtype{set} 条目的引用添加到  \bibtype{set} 条目中。
其成员的引用添加到各自成员和 \bibtype{set} 条目中。

\item[memandset] %References to the \bibtype{set} entry are added to the \bibtype{set} entry and to all members. References to set members are added to the respective member.
\bibtype{set} 条目的引用添加到  \bibtype{set} 条目和所有成员中。
其成员的引用添加到各自成员中。

\item[setplusmem] %References to the \bibtype{set} entry are added to the \bibtype{set} entry and to all members. References to set members are added to the respective member and to the \bibtype{set} entry.
\bibtype{set} 条目的引用添加到  \bibtype{set} 条目和所有成员中。
其成员的引用添加到各自成员和 \bibtype{set} 条目中。

\end{valuelist}

\optitem[false]{indexing}{\opt{true}, \opt{false}, \opt{cite}, \opt{bib}}

%This option controls indexing in citations and in the bibliography. More precisely, it affects the \cmd{ifciteindex} and \cmd{ifbibindex} commands from \secref{aut:aux:tst}. The option is settable on a global, a per-type, or on a per-entry basis. The possible choices are:
该选项控制文献表和标注(引用)中的索引。
更准确地说,它影响 \secref{aut:aux:tst} 节的 \cmd{ifciteindex} 和 \cmd{ifbibindex} 命令。
该选项可以全局、针对某一类型或针对某一条目进行设置。
可用的选择有:

\begin{valuelist}
\item[true] %Enable indexing globally.
全局激活索引。
\item[false] %Disable indexing globally.
全局不激活索引。
\item[cite] %Enable indexing in citations only.
只在标注中激活索引。
\item[bib] %Enable indexing in the bibliography only.
只在文献表中激活索引。
\end{valuelist}

%This feature requires support by the selected citation style. All standard styles which ship with this package support indexing of both citations and entries in the bibliography. Note that you still need to enable indexing globally with \cmd{makeindex} to get an index.
该特性需要标注样式的支持。本宏包所带的所有标准样式都支持标注和文献条目中的索引。
请注意,为了得到索引表,仍然需要用 \cmd{makeindex} 命令全局激活索引模式。

\boolitem[false]{loadfiles}

%This option controls whether external files requested by way of the \cmd{printfile} command are loaded. See also \secref{use:use:prf} and \cmd{printfile} in \secref{aut:bib:dat}. Note that this feature is disabled by default for performance reasons.
该选项控制是否载入 \cmd{printfile} 命令请求的外部文件。另参考 \secref{use:use:prf} 和 \secref{aut:bib:dat} 节中的 \cmd{printfile} 命令。请注意,出于性能考虑,该特性默认不激活。

\optitem[none]{refsection}{\opt{none}, \opt{part}, \opt{chapter}, \opt{section}, \opt{subsection}}

%This option automatically starts a new reference section at a document division such as a chapter or a section. This is equivalent to the \cmd{newrefsection} command, see \secref{use:bib:sec} for details. The following choice of document divisions is available:
该选项自动在文档划分时(例如一章或一节)开始一个新的参考文献分节。这等价于 \cmd{newrefsection} 命令,参考 \secref{use:bib:sec} 节。可用的文档划分如下:

\begin{valuelist}
\item[none] %Disable this feature.
不启用该特性。
\item[part] %Start a reference section at every \cmd{part} command.
在每个 \cmd{part} 命令处开始一个参考文献分节。
\item[chapter] %Start a reference section at every \cmd{chapter} command.
在每个 \cmd{chapter} 命令处开始一个参考文献分节。
\item[section] %Start a reference section at every \cmd{section} command.
在每个 \cmd{section} 命令处开始一个参考文献分节。
\item[subsection] %Start a reference section at every \cmd{subsection} command.
在每个 \cmd{subsection} 命令处开始一个参考文献分节。
\end{valuelist}
%
%The starred versions of these commands will not start a new reference section.
对应这些命令的带星号版本不会开始新的参考文献分节。

\optitem[none]{refsegment}{\opt{none}, \opt{part}, \opt{chapter}, \opt{section}, \opt{subsection}}

%Similar to the \opt{refsection} option but starts a new reference segment. This is equivalent to the \cmd{newrefsegment} command, see \secref{use:bib:seg} for details. When using both options, note that you can only apply this option to a lower"=level document division than the one \opt{refsection} is applied to and that nested reference segments will be local to the enclosing reference section.
类似于 \opt{refsection} 选项,但是开始一个新的参考文献片段。这等价于 \cmd{newrefsegment} 命令,详见 \secref{use:bib:seg} 节。当两个选项都使用时,请注意该选项只能应用到比 \opt{refsection} 选项应用的文档划分更低层划分中,同时,嵌套的参考文献片段相对于所附属的参考文献分节来讲是局部的。

\optitem[none]{citereset}{\opt{none}, \opt{part}, \opt{chapter}, \opt{section}, \opt{subsection}}

%This option automatically executes the \cmd{citereset} command from \secref{use:cit:msc} at a document division such as a chapter or a section. The following choice of document divisions is available:

该选项在文档划分处(例如一章或一节)自动执行 \secref{use:cit:msc} 节介绍的 \cmd{citereset} 命令。可用的文档分段有:

\begin{valuelist}
\item[none] %Disable this feature.
不启用该特性。
\item[part] %Perform a reset at every \cmd{part} command.
在每个 \cmd{part}  命令后执行重置。
\item[chapter] %Perform a reset at every \cmd{chapter} command.
在每个 \cmd{chapter}  命令后执行重置。
\item[section] %Perform a reset at every \cmd{section} command.
在每个 \cmd{section}  命令后执行重置。
\item[subsection] %Perform a reset at every \cmd{subsection} command.
在每个 \cmd{subsection}  命令后执行重置。
\end{valuelist}
%
%The starred versions of these commands will not trigger a reset.
这些命令对应的带星号版本不会引起重置。

\boolitem[true]{abbreviate}

%Whether or not to use long or abbreviated strings in citations and in the bibliography. This option affects the localisation modules. If this option is enabled, key terms such as <editor> are abbreviated. If not, they are written out.
是否在标注和文献表中使用长字符串或缩写字符串。该选项会影响本地化模块。如果启用该选项,“editor”等键值项会被缩写,反之则会全部写出。

\optitem[comp]{date}{\opt{year}, \opt{short}, \opt{long}, \opt{terse}, \opt{comp}, \opt{ymd}, \opt{iso}}

%This option controls the basic format of printed date specifications. The following choices are available:

该选项控制日期规范的基本格式。有以下选择:

\begin{valuelist}
\item[year] %Use only years, for example:\par
只使用年份,例如:\par
2010\par
2010--2012\par
\item[short] %Use the short format with verbose ranges, for example:\par
使用详细日期范围的短格式,例如:\par
01/01/2010\par
21/01/2010--30/01/2010\par
01/21/2010--01/30/2010
\item[long] %Use the long format with verbose ranges, for example:\par
使用详细的日期范围的长格式,例如:\par
1st January 2010\par
21st January 2010--30th January 2010\par
January 21, 2010--January 30, 2010\par
\item[terse] %Use the short format with compact ranges, for example:\par
使用压缩日期范围的短格式,例如:\par
21--30/01/2010\par
01/21--01/30/2010
\item[comp] %Use the long format with compact ranges, for example:\par
使用压缩日期范围长格式,例如:\par
21st--30th January 2010\par
January 21--30, 2010\par
\item[iso] Use ISO8601 Extended Format (\texttt{yyyy-mm-dd}), for example:\par
使用ISO8601扩展格式(\texttt{yyyy-mm-dd}),例如:\par
2010-01-01\par
2010-01-21/2010-01-30
\item[ymd] %A year-month-day format which can be modified by other options unlike strict \acr{ISO8601-2}, for example:\par
不同于严格的\acr{ISO8601-2}格式,使用可以被其它选项修改的年-月-日格式,例如:\par
2010-1-1\par
2010-1-21/2010-1-30
\end{valuelist}
%
%Note that \opt{iso} format will enforce \kvopt{dateera}{astronomical}, \kvopt{datezeros}{true}, \kvopt{timezeros}{true}, \kvopt{seconds}{true}, \kvopt{$<$datetype$>$time}{24h} and \kvopt{julian}{false}. \opt{ymd} is an ETFT-like format but which can change the various options which the strict \opt{iso} option does not allow for.
注意,\opt{iso} 格式会强制开启 \kvopt{dateera}{astronomical}, \kvopt{datezeros}{true}, \kvopt{timezeros}{true}, \kvopt{seconds}{true}, \kvopt{\prm{datetype}time}{24h} 以及 \kvopt{julian}{false} 等键值。\opt{ymd} 是ETFT类格式,但可以由不同的选项做出改变,而这是\opt{iso}选项不允许的。

%As seen in the above examples, the actual date format is language specific. Note that the month name in all long formats is responsive to the \opt{abbreviate} package option. The leading zeros for months and days in all short formats may be controlled separately with the \opt{datezeros} package option. The leading zeros for hours, minutes and seconds in all short formats may be controlled separately with the \opt{timezeros} package option. If outputting times, the printing of seconds and timezones is controlled by the \opt{seconds} and \opt{timezones} options respectively.
从上述例子可以看出,实际的日期格式是与语言相关的。请注意,所有长格式中的月份名受 \opt{abbreviate} 宏包选项影响。所有短格式中月和日的首位零可以另外由 \opt{datezeros} 宏包选项控制。所有短格式中时分秒的首位零可以另外由 \opt{timezeros} 宏包选项控制。如果要输出时刻,那么秒和时区的打印分别由 \opt{seconds} 和 \opt{timezones} 选项控制。

%The options \opt{julian} and \opt{gregorianstart}  may be used to control when to output Julian Calendar dates.

\opt{julian} 和 \opt{gregorianstart} 选项可以用于控制何时输出儒略历日期。

\optitem[year]{labeldate}{\opt{year}, \opt{short}, \opt{long}, \opt{terse}, \opt{comp}, \opt{ymd}, \opt{iso}}

%Similar to the \opt{date} option but controls the format of the date field selected with \cmd{DeclareLabeldate}.
类似于 \opt{date} 选项但是控制由 \cmd{DeclareLabeldate} 选择的日期域的格式。

\optitem[comp]{\prm{datetype}date}{\opt{year}, \opt{short}, \opt{long}, \opt{terse}, \opt{comp}, \opt{ymd}, \opt{iso}}

%Similar to the \opt{date} option but controls the format of the \bibfield{$<$datetype$>$date} field in the datamodel.
类似于 \opt{date} 选项但是控制数据模型中 \bibfield{\prm{datetype}date} 域的格式。

\optitem{alldates}{\opt{year}, \opt{short}, \opt{long}, \opt{terse}, \opt{comp}, \opt{iso}}

%Sets the option for all dates in the datamodel to the same value. The date fields in the default data model are \bibfield{date}, \bibfield{origdate}, \bibfield{eventdate} and \bibfield{urldate}.
将数据模型中所有日期的选项设置为同一值。默认数据模型中的日期域为 \bibfield{date}、\bibfield{origdate}、\bibfield{eventdate} 和 \bibfield{urldate}。

\boolitem[false]{julian}

%This option controls whether dates before the date specified in the \opt{gregorianstart} option will be converted automatically to the Julian Calendar. Dates so changed will return <true> for the \cmd{ifdatejulian} and \cmd{if$<$datetype$>$datejulian} tests (see \secref{aut:aux:tst}). Please bear in mind that dates consisting of just a year like <1565> will never be converted to a Julian Calendar date because a date without a month and day has an ambiguous Julian Calendar representation\footnote{This is potentially true for dates missing times too but this is not relevant for bibliographic work.}. For example, in the case of <1565>, this is Julian year <1564> until after the Gregorian date <10th January 1565> when the Julian year becomes <1565>.
该选项控制是否将由 \opt{gregorianstart} 选项指定日期之前的日期自动转换为儒略历。改变的日期在 \cmd{ifdatejulian} 和 \cmd{if\prm{datetype}datejulian} 测试下会返回“true”(见 \secref{aut:aux:tst} 节)。请谨记,只包含年份的日期不会转换为儒略历,例如“1565”,这是因为没有月日的日期在儒略历表示下会引起混乱
\footnote{缺失时刻的日期也有可能出现这一问题,不过对于文献作品问题不大。}
例如,在“1565”的例子中,在公历(格里高利历)“1565年1月10日”之后的日期是儒略历“1565”年,而之前的日期是儒略历“1564”年。

\valitem{gregorianstart}{YYYY-MM-DD}

%This option controls the date before which dates are converted to the Julian Calendar. It is a strict format string, 4-digit year, 2-digit month and day, separated by a single dash character (any valid Unicode character with the <Dash> property). The default is '1582-10-15', the date of the instigation of the standard Gregorian Calendar. This option does not nothing unless \opt{julian} is set to <true>.
该选项控制在哪一日期之前的日期可以转换为儒略历。选项值有严格的字符串格式:4位的年份、2位的月份和日期,并且由短划线(或者具有“Dash”属性的任何有效Unicode字符)分隔。默认值是“1582-10-15”,即标准公历(格里高利历)的颁布日期。只有 \opt{julian} 设置为 “true”时本选项才起作用。

\boolitem[true]{datezeros}

%This option controls whether \texttt{short} and \texttt{terse} date components are printed with leading zeros unless overridden by specific formatting.
该选项控制当没有特定格式覆盖时,打印 \texttt{short} 和 \texttt{terse} 日期成分是否首位补零。

\boolitem[true]{timezeros}

%This option controls whether time components are printed with leading zeros unless overridden by specific formatting.
该选项控制没有覆盖特定格式时,打印时刻成分是否首位补零。

\boolitem[false]{timezones}

%This option controls whether timezones are printed when printing times.
该选项控制打印时刻时是否输出时区。

\boolitem[false]{seconds}

%This option controls whether seconds are printed when printing times.
该选项控制打印时刻时是否输出秒。

\boolitem[true]{dateabbrev}

%This option controls whether \texttt{long} and \texttt{comp} dates are printed with long or abbreviated month/season names. The option is similar to the generic \opt{abbreviate} option but specific to the date formatting.
该选项控制打印\texttt{long} 和 \texttt{comp} 日期格式时,使用完整的或是缩写的月份名。该选项类似于一般的 \opt{abbreviate} 选项但是只针对日期格式。

\boolitem[false]{datecirca}

%This option controls whether to output <circa> information about dates. If set to \opt{true}, dates will be preceded by the expansion of the \cmd{datecircaprint} macro (\secref{use:fmt:fmt}).
该选项控制是否输出日期的“circa”信息。如果设置为 \opt{true},那么日期将由\secref{use:fmt:fmt} 节的\cmd{datecircaprint} 宏的展开来引导。

\boolitem[false]{dateuncertain}

%This option controls whether to output uncertainty information about dates. If set to \opt{true}, dates will be followed by the expansion of the \cmd{dateuncertainprint} macro and end dates will be followed by the \cmd{enddateuncertainprint} macro (\secref{use:fmt:fmt}).
该选项控制是否输出日期的不确定信息。如果设置为 \opt{true},那么日期将由 \cmd{dateuncertainprint} 宏的展开来引导,
并由 \cmd{enddateuncertainprint} 宏结束,见 \secref{use:fmt:fmt} 节。

\optitem[astronomical]{dateera}{\opt{astronomical}, \opt{secular}, \opt{christian}}

%This option controls how date era information is printed. <astronomical> uses \cmd{dateeraprintpre} to print era information \emph{before} start/end dates. <secular> and <christian> uses \cmd{dateeraprint} to print era information \emph{after} the start/end/dates. By default <astronomical> results in a minus sign before BCE/BC dates and <secular>/<christian> results in the relevant localisation strings like <BCE> or <BC> after BCE/BC dates. See the relevant comments in \secref{use:fmt:fmt} and the localisation strings in \secref{aut:lng:key:dt}.
该选项控制如何打印日期纪元信息。选项值“astronomical”使用 \cmd{dateeraprintpre} 命令在起止日期\emph{之前}打印纪元信息。
而选项值“secular”和“christian”使用 \cmd{dateeraprint} 命令在起止日期\emph{之后}打印纪元信息。
缺省情况下,使用“astronomical”效果是在公元前日期之前使用负号,而使用“<secular>”或“<christian>”的效果是在公元前日期之后加上“BCE”或“BC”等相关的本地化字符串。见 \secref{use:fmt:fmt} 节的有关评论以及 \secref{aut:lng:key:dt} 节的本地化字符串。

\intitem[0]{dateeraauto}

%This option sets the astronomical year, below which era localisation strings are automatically added. This option does nothing without \opt{dateera} being set to <secular> or <christian>.
该选项设置回归年,在其之前自动添加纪元的本地化字符串。只有当 \opt{dateera} 设置为“secular”或者“christian”时本选项才起作用。

\optitem[24h]{time}{\opt{12h}, \opt{24h}, \opt{24hcomp}}

%This option controls the basic format of printed time specifications. The following choices are available:

该选项控制时刻规范的基本格式。
可用的选择有:

\begin{valuelist}
	\item[24h] %24-hour format, for example:\par
	24小时格式,例如:\par
	14:03:23\par
	14:3:23\par
	14:03:23+05:00\par
	14:03:23Z\par
	14:21:23--14:23:45\par
	14:23:23--14:23:45\par
	\item[24hcomp] %24-hour format with compressed ranges, for example:\par
	带有缩写范围的24小时格式,例如:\par
	14:21--23 (小时相同)\par %(hours are the same)\par
	14:23:23--45 (小时和分钟相同)\par %(hour and minute are the same)\par
	\item[12h] %12-hour format with (localised) AM/PM markers, for example:\par
	带有本地化上下午标识符的12小时格式,例如:\par
	2:34 PM\par
	2:34 PM--3:50 PM\par
\end{valuelist}
%
%As seen in the above examples, the actual time format is language specific. Note that the AM/PM string is responsive to the \opt{abbreviate} package option, if this makes a difference in the specific locale. The leading zeros in the 24-hour formats may be controlled separately with the \opt{timezeros} package option. The separator between time components (\cmd{bibtimesep} and \cmd{bibtzminsep})and between the time and any timezone (\cmd{bibtimezonesep}) are also language specific and customisable, see \secref{use:fmt:lng}. There are global package options which determine whether seconds and timezones are printed (\opt{seconds} and \opt{timezones}, respectively, see \secref{use:opt:pre:gen}). Timezones, if present, are either <Z> or a numeric positive or negative offset. No default styles print time information. Custom styles may print times by using the \cmd{print<datetype>time} commands, see \secref{aut:bib:dat}.
从以上例子可以看出,实际的时刻格式是与语言相关的。请注意,如果与指定的区域不同,那么上下午(AM/PM)字符串受 \opt{abbreviate} 宏包选项影响。24小时格式首位补零的话可以单独用 \opt{timezeros} 宏包选项控制。此外与语言相关并可以单独定制的还有时刻成分之间的分隔符\cmd{bibtimesep}、\cmd{bibtzminsep},以及时刻与时区之间的分隔符 \cmd{bibtimezonesep},见 \secref{use:fmt:lng} 节。还有一些全局的宏包选项可以控制是否打印秒和时区(分别是 \opt{seconds} 和 \opt{timezones},见 \secref{use:opt:pre:gen} 节)。如果有时区的话,要么是字符`Z',要么是表示正负偏移量的数值。标准样式不打印时刻信息。
定制样式可以使用 \cmd{print\prm{datetype}time} 命令打印时刻,见 \secref{aut:bib:dat} 节。

\optitem[24h]{labeltime}{\opt{12h}, \opt{24h}, \opt{24hcomp}}

%Similar to the \opt{time} option but controls the format of the time part fields obtained from the field selected with \cmd{DeclareLabeldate}.
类似于 \opt{time} 选项但是控制由 \cmd{DeclareLabeldate} 选择的域中得到的时刻部分的格式。

\optitem[24h]{\prm{datetype}time}{\opt{12h}, \opt{24h}, \opt{24hcomp}}

%Similar to the \opt{time} option but controls the format of the time part fields obtained from the \bibfield{$<$datetype$>$date} field in the datamodel.
类似于 \opt{time} 选项但是控制数据模型中 \bibfield{\prm{datetype}date} 域中得到的时刻部分格式。

\optitem{alltimes}{\opt{12h}, \opt{24h}, \opt{24hcomp}}

%Sets \opt{labeltime} and the \opt{$<$datetype$>$time} option for all times in the datamodel to the same value. The date fields supporting time parts in the default data model are \bibfield{date}, \bibfield{origdate}, \bibfield{eventdate} and \bibfield{urldate}.
为数据模型中所有的时刻设置相同的 \opt{labeltime} 和 \opt{\prm{datetype}time} 值。默认数据模型中支持时刻部分的日期域有 \bibfield{date}、\bibfield{origdate}、\bibfield{eventdate} 和 \bibfield{urldate}。

\boolitem[false]{dateusetime}

%Specifies whether to print any time component of a date field after the date component. The separator between the date and time components is \cmd{bibdatetimesep} from \secref{use:fmt:lng}. This option does nothing if a compact date format is being used (see \secref{use:opt:pre:gen}) as this would be very confusing.
确定在日期域的日期成分后是否打印时刻成分。日期和时刻成分的分隔符是  \cmd{bibdatetimesep},见 \secref{use:fmt:lng} 节。如果使用压缩的日期格式(见  \secref{use:opt:pre:gen} 节),那么该选项则不起作用,否则会引起混乱。

\boolitem[false]{labeldateusetime}

%Similar to the \opt{dateusetime} option but controls the whether to print time components for the field selected with \cmd{DeclareLabeldate}.
类似于 \opt{dateusetime} 选项,但是控制是否打印 \cmd{DeclareLabeldate} 选择的域中的时刻成分。

\boolitem[false]{\prm{datetype}dateusetime}

%Similar to the \opt{dateusetime} option but controls the whether to print time components for the \bibfield{\prm{datetype}date} field in the datamodel.
类似于 \opt{dateusetime} 选项,但是控制是否打印数据模型中 \bibfield{\prm{datetype}date} 域的时刻成分。

\boolitem[false]{alldatesusetime}

%Sets \opt{labeldateusetime} and the \opt{\prm{datetype}dateusetime} option for all \bibfield{\prm{datetype}date} fields in the datamoel.
为数据模型中所有的 \bibfield{\prm{datetype}date} 域设置 \opt{labeldateusetime} 和 \opt{\prm{datetype}dateusetime} 选项值。

\boolitem[false]{defernumbers}

%In contrast to standard \latex, the numeric labels generated by this package are normally assigned to the full list of references at the beginning of the document body. If this option is enabled, numeric labels (\ie the \bibfield{labelnumber} field discussed in \secref{aut:bbx:fld}) are assigned the first time an entry is printed in any bibliography. See \secref{use:cav:lab} for further explanation.  This option requires two \latex runs after the data has been exported to the \file{bbl} file by the backend (in addition to any other runs required by page breaks changing etc.). An important thing to note is that if you change the value of this option in your document (or the value of options which depend on this like some of the options to the \cmd{printbibliography} macro, see \secref{use:bib:bib}), then it is likely that you will need to delete your current \file{aux} file and re-run \latex to obtain the correct numbering. See \secref{aut:int}.
与标准 \LaTeX 不同,默认情况下,本宏包生成的顺序编码标签在文档正文开始处就分配给文献表的全体文献。但如果该选项被激活,各文献的顺序编码标签(也就是 \secref{aut:bbx:fld} 中讨论的 \bibfield{labelnumber} 域)只有该文献在任意文献表中被打印时才会做第一次分配。进一步解释见 \secref{use:cav:lab} 节。该选项需要在后端将数据导出到 \file{bbl} 文件后再运行两次 \LaTeX (除由分页变化等要求的编译外)。需注意的一个要点是,如果你在文档中改变了该选项值(或者那些依赖于本选项的选项值,例如与\cmd{printbibliography} 宏相关的选项,见\secref{use:bib:bib} 节),那么很可能需要删除当前的 \file{aux} 文件然后重新运行 \LaTeX 来获得正确的顺序编码,见 \secref{aut:int} 节。(\emph{需要注意:该选项只与顺序编码标签有关,而与文献表中文献的排序无关。主要用于解决做文献筛选后顺序编码不连续的问题,具体见\secref{use:cav:lab}节。-译者注})

\boolitem[false]{punctfont}

%This option enables an alternative mechanism for dealing with unit punctuation after a field printed in a different font (for example, a title printed in italics). See \cmd{setpunctfont} in \secref{aut:pct:new} for details.
该选项激活一个可选机制,用来处理以不同字体打印的域之后的单元标点(例如以斜体打印的标题后的标点)。详见 \secref{aut:pct:new} 节中的 \cmd{setpunctfont}。

\optitem[abs]{arxiv}{\opt{abs}, \opt{ps}, \opt{pdf}, \opt{format}}

%Path selector for arXiv links. If hyperlink support is enabled, this option controls which version of the document the arXiv \bibfield{eprint} links will point to. The following choices are available:
arXiv 链接的路径选择。如果启用超链接支持,该选项会控制 arXiv 的 \bibfield{eprint} 链接指向该文件的哪个版本。以下是可用的选择:

\begin{valuelist}
\item[abs] %Link to the abstract page.
链接到摘要页面。
\item[ps] %Link to the PostScript version.
链接到 PostScript 版本。
\item[pdf] %Link to the \pdf version.
链接到 \pdf 版本。
\item[format] %Link to the format selector page.
链接到格式选择页面。
\end{valuelist}

%See \secref{use:use:epr} for details on support for arXiv and electronic publishing information.
关于 arXiv 和电子出版信息的支持详见 \secref{use:use:epr} 节。

\optitem[auto]{texencoding}{\opt{auto}, \prm{encoding}}

%Specifies the encoding of the \file{tex} file. This option affects the data transferred from the backend to \biblatex. This corresponds to \biber's \opt{--output-encoding} option. The following choices are available:
指定 \file{tex} 文件的编码。该选项影响从后端传递给 \biblatex  的数据。该选项对应于 \biber 的 \opt{--output-encoding} 选项。可用的选择有:

\begin{valuelist}

\item[auto] %Try to auto-detect the input encoding. If the \sty{inputenc}\slash \sty{inputenx}\slash \sty{luainputenc} package is available, \biblatex will get the main encoding from that package. If not, it assumes \utf encoding if \xetex or \luatex has been detected, and Ascii otherwise.
尝试自动识别输入的编码。如果有 \sty{inputenc}\slash\sty{inputenx}\slash\sty{luainputenc} 等宏包,
\biblatex 会从宏包中获取主要编码。否则,当探测到 \XeTeX 或 \LuaTeX 引擎时设定为 \utf 编码,其余情况设为 Ascii。

\item[\prm{encoding}] %Specifies the \prm{encoding} explicitly. This is for odd cases in which auto-detection fails or you want to force a certain encoding for some reason.
显式指定编码为 \prm{encoding}。少数情况下自动检测失败,或你出于某种原因想强制使用某一编码,那么此时可以使用该选项。

\end{valuelist}
%
%Note that setting \kvopt{texencoding}{\prm{encoding}} will also affect the \opt{bibencoding} option if \kvopt{bibencoding}{auto}.
请注意如果\kvopt{bibencoding}{auto},那么设置 \kvopt{texencoding}{\prm{encoding}} 也会影响 \opt{bibencoding} 选项。

\optitem[auto]{bibencoding}{\opt{auto}, \prm{encoding}}

%Specifies the encoding of the \file{bib} files. This corresponds to \biber's \opt{--input-encoding} option. The following choices are available:
指定 \file{bib} 文件的编码。该选项对应于 \biber 程序的 |--output-encoding| 选项。可用选择有:

\begin{valuelist}

\item[auto] %Use this option if the workflow is transparent, \ie if the encoding of the \file{bib} file is identical to the encoding of the \file{tex} file.
当工作流是透明时,即,当 \file{bib} 文件的编码与 \file{tex} 文件的编码相同时使用该选项。

\item[\prm{encoding}] %If the encoding of the \file{bib} file is different from the one of the \file{tex} file, you need to specify it explicitly.
如果 \file{bib} 文件的编码与 \file{tex} 不同,你需要显式地指定编码。

\end{valuelist}

%By default, \biblatex assumes that the \file{tex} file and the \file{bib} file use the same encoding (\kvopt{bibencoding}{auto}).
默认情况下,\biblatex 假定 \file{tex} 和 \file{bib} 文件使用相同的编码(\kvopt{bibencoding}{auto})。

\boolitem[false]{safeinputenc}

%If this option is enabled, \biblatex will automatically force \kvopt{texencoding}{ascii} if the \sty{inputenc}\slash \sty{inputenx} package has been loaded and the input encoding is \utf, \ie it will ignore any macro-based \utf support and use Ascii only. \biber will then try to convert any non-Ascii data in the \file{bib} file to Ascii. For example, it will convert \texttt{\d{S}} to |\d{S}|. See \secref{bib:cav:enc:enc} for an explanation of why you may want to enable this option.
如果启用该选项,\biblatex 会在载入 \sty{inputenc}\slash \sty{inputenx} 宏包并且输入代码是 \utf 时自动加入 \kvopt{texencoding}{ascii},也就是说,它会忽略任何基于宏的 \utf 支持而只是用Ascii。然后 \biber 会尝试将 \file{bib} 文件中的任意非Ascii数据转化为 Ascii。例如,它会将 \texttt{\d{S}} 转化为 |\d{S}|。关于为什么需要启用该选项的原因,请参考 \secref{bib:cav:enc:enc} 节。

\boolitem[true]{bibwarn}

%By default, \biblatex will report warnings issued by the backend concerning the data in the \file{bib} file as \latex warnings. Use this option to suppress such warnings.
默认情况下,\biblatex 会报告后端产生的关于 \file{bib} 文件数据的警告,就像 \LaTeX 警告一样。使用该选项会取消此警告。

\intitem[2]{mincrossrefs}

%Sets the minimum number of cross references to \prm{integer} when requesting a backend run.\footnote{If an entry which is cross-referenced by other entries in the \file{bib} file hits this threshold, it is included in the bibliography even if it has not been cited explicitly. This is a standard feature of the \bibtex format and not specific to \biblatex. See the description of the \bibfield{crossref} field in \secref{bib:fld:spc} for further information.} This option also affects the handling of the \bibfield{xref} field. See the field description in \secref{bib:fld:spc} as well as \secref{bib:cav:ref} for details.
当需要运行后端时,设置交叉引用的最小数目为 \prm{integer}
\footnote{如果一个条目被 \file{bib} 文件中的其它条目引用的数目达到这个阈值,它就会载入到参考文献中,即使没有显式地被引用。	这是 \bibtex  格式的标准特性,并不是 \biblatex 特有的。更多信息见 \secref{bib:fld:spc} 节中关于 \bibfield{crossref} 域的描述。}。该选项也影响 \bibfield{xref} 域的处理。详见 \secref{bib:fld:spc} 节以及 \secref{bib:cav:ref} 节对该域的描述。

\intitem[2]{minxrefs}

%As \opt{mincrossrefs} but for \bibfield{xref} fields.
类似于 \opt{mincrossrefs} 但针对于 \bibfield{xref} 域。

\end{optionlist}

\paragraph{特定样式选项}%\paragraph{Style-specific}
\label{use:opt:pre:bbx}

%The following options are provided by the standard styles (as opposed to the core package). Technically, they are preamble options like those in \secref{use:opt:pre:gen}.
下列选项由标准样式(而不是宏包内核)提供。技术上讲,它们和 \secref{use:opt:pre:gen} 中的选项一样也是导言区选项。

\begin{optionlist}

\boolitem[true]{isbn}

%This option controls whether the fields \bibfield{isbn}\slash \bibfield{issn}\slash \bibfield{isrn} are printed.
该选项控制是否打印 \bibfield{isbn}\slash \bibfield{issn}\slash \bibfield{isrn} 等域。

\boolitem[true]{url}

%This option controls whether the \bibfield{url} field and the access date is printed. The option only affects entry types whose \bibfield{url} information is optional. The \bibfield{url} field of \bibtype{online} entries is always printed.
该选项控制是否打印 \bibfield{url} 域和访问日期。该选项只影响 \bibfield{url} 信息是可选的那些条目类型。而 \bibtype{online} 条目的 \bibfield{url} 域总是打印出来的。

\boolitem[true]{doi}

%This option controls whether the field \bibfield{doi} is printed.
该选项控制是否打印 \bibfield{doi} 域。

\boolitem[true]{eprint}

%This option controls whether \bibfield{eprint} information is printed.
该选项控制是否打印 \bibfield{eprint} 信息。

\end{optionlist}

\paragraph{内部选项}%\paragraph{Internal}
\label{use:opt:pre:int}

%The default settings of the following preamble options are controlled by bibliography and citation styles. Apart from the \opt{pagetracker} and \opt{$<$name$>$inits} options, which you may want to adapt, there is normally no need to set them explicitly.
下列导言区选项的默认设置由著录和标注样式控制。除了 \opt{pagetracker} 和 \opt{\prm{name}inits} 这两个你可能想调整的选项外,一般没必要显式设置。

\begin{optionlist}

\optitem[false]{pagetracker}{\opt{true}, \opt{false}, \opt{page}, \opt{spread}}

%This option controls the page tracker which is required by the \cmd{ifsamepage} and \cmd{iffirstonpage} tests from \secref{aut:aux:tst}. The possible choices are:
该选项控制 \secref{aut:aux:tst} 节的  \cmd{ifsamepage} 和 \cmd{iffirstonpage} 测试所需的页码跟踪器。可用选择有:

\begin{valuelist}
\item[true] %Enable the tracker in automatic mode. This is like \opt{spread} if \latex is in twoside mode, and like \opt{page} otherwise.
在自动模式下激活。该选项在 \LaTeX 处于双面模式时类似于 \opt{spread},否则类似于 \opt{page}。
\item[false] %Disable the tracker.
不激活跟踪器。
\item[page] %Enable the tracker in page mode. In this mode, tracking works on a per"=page basis.
在页面模式下激活。此时跟踪器基于每一页。
\item[spread] %Enable the tracker in spread mode. In this mode, tracking works on a per"=spread (double page) basis.
在跨页模式下激活。此时跟踪器基于每一页面(双页)。
\end{valuelist}

%Note that this tracker is disabled in all floats, see \secref{aut:cav:flt}.
注意所有的浮动体都禁用该跟踪器,见 \secref{aut:cav:flt} 节。

\optitem[false]{citecounter}{\opt{true}, \opt{false}, \opt{context}}

%This option controls the citation counter which is required by \cnt{citecounter} from \secref{aut:aux:tst}. The possible choices are:
该选项控制 \secref{aut:aux:tst} 节的 \cnt{citecounter} 所需的引用计数器。可用的选择有:

\begin{valuelist}
\item[true] %Enable the citation counter in global mode.
在全局模式下启用引用计数器。
\item[false] %Disable the citation counter.
禁用引用计数器。
\item[context] %Enable the citation counter in context"=sensitive mode. In this mode, citations in footnotes and in the body text are counted independently.
在环境敏感模式下启用引用计数器。此时,脚注和正文中的引用将独立计数。
\end{valuelist}

\optitem[false]{citetracker}{\opt{true}, \opt{false}, \opt{context}, \opt{strict}, \opt{constrict}}

%This option controls the citation tracker which is required by the \cmd{ifciteseen} and \cmd{ifentryseen} tests from \secref{aut:aux:tst}. The possible choices are:
该选项控制 \secref{aut:aux:tst} 节的 \cmd{ifciteseen} 和 \cmd{ifentryseen} 测试所需的引用跟踪器。可用选择有:

\begin{valuelist}
\item[true] %Enable the tracker in global mode.
在全局模式下启用跟踪器。
\item[false] %Disable the tracker.
禁用跟踪器。
\item[context] %Enable the tracker in context"=sensitive mode. In this mode, citations in footnotes and in the body text are tracked independently.
在环境敏感模式下启用跟踪器。此时,脚注和正文中的引用将独立跟踪。
\item[strict] %Enable the tracker in strict mode. In this mode, an item is only considered by the tracker if it appeared in a stand-alone citation, \ie if a single entry key was passed to the citation command.
在严格模式下启用跟踪器。此时,跟踪器只考虑独立的引用,即,引用命令只传递单个的条目键。
\item[constrict] %This mode combines the features of \opt{context} and \opt{strict}.
该模式是 \opt{context} 和 \opt{strict} 的结合。
\end{valuelist}

%Note that this tracker is disabled in all floats, see \secref{aut:cav:flt}.
注意所有的浮动体都禁用该跟踪器,见 \secref{aut:cav:flt} 节。

\optitem[false]{ibidtracker}{\opt{true}, \opt{false}, \opt{context}, \opt{strict}, \opt{constrict}}

%This option controls the <ibidem> tracker which is required by the \cmd{ifciteibid} test from \secref{aut:aux:tst}. The possible choices are:

该选项控制 \secref{aut:aux:tst} 节的 \cmd{ifciteibid} 测试所需的“如前所述”(ibidem)跟踪器。可用的选择有:

\begin{valuelist}
\item[true] %Enable the tracker in global mode.
在全局模式下启用跟踪器。
\item[false] %Disable the tracker.
禁用跟踪器。
\item[context] %Enable the tracker in context"=sensitive mode. In this mode, citations in footnotes and in the body text are tracked separately.
在环境敏感模式下启用跟踪器。此时,脚注和正文中的引用将独立跟踪。
\item[strict] %Enable the tracker in strict mode. In this mode, potentially ambiguous references are suppressed. A reference is considered ambiguous if either the current citation (the one including the <ibidem>) or the previous citation (the one the <ibidem> refers to) consists of a list of references.\footnote{For example, suppose the initial citation is «Jones, \emph{Title}; Williams, \emph{Title}» and the following one «ibidem». From a technical point of view, it is fairly clear that the <ibidem> refers to <Williams> because this is the last reference processed by the previous citation command. To a human reader, however, this may not be obvious because the <ibidem> may also refer to both titles. The strict mode avoids such ambiguous references.}
在严格模式下启用跟踪器。此时将不考虑那些模糊不清的参考文献。如果当前引用(包含“ibidem”)或者之前引用(“ibidem”的指向)包含多个文献时,就认为是模糊不清的。
\footnote{例如,假设一开始的引用是“Jones, \emph{Title}; Williams, \emph{Title}”,而接下来的是“ibidem”。
	从技术角度来看,“ibidem”指向“Williams”是相当清晰的,因为这是之前引用命令处理的最后文献。然而对于用户而言,“ibidem”也可以同时指向这两个标题,因此含义不清。	严格模式就避免这种含义不清的文献。}
\item[constrict] %This mode combines the features of \opt{context} and \opt{strict}. It also keeps track of footnote numbers and detects potentially ambiguous references in footnotes in a stricter way than the \opt{strict} option. In addition to the conditions imposed by the \opt{strict} option, a reference in a footnote will only be considered as unambiguous if the current citation and the previous citation are given in the same footnote or in immediately consecutive footnotes.
该模式是 \opt{context} 和 \opt{strict} 的结合。它也保持对脚注数量的跟踪,不过检测脚注中含义不清的文献时比 \opt{strict} 更加严格。除了 \opt{strict} 选项的条件外,只有当前引用和之前引用都在同一个或者连续的脚注中时,脚注中的文献才认为是含义清晰的。
\end{valuelist}

%Note that this tracker is disabled in all floats, see \secref{aut:cav:flt}.
注意所有的浮动体都禁用该跟踪器,见 \secref{aut:cav:flt} 节。

\optitem[false]{opcittracker}{\opt{true}, \opt{false}, \opt{context}, \opt{strict}, \opt{constrict}}

%This option controls the <opcit> tracker which is required by the \cmd{ifopcit} test from \secref{aut:aux:tst}. This feature is similar to the <ibidem> tracker, except that it tracks citations on a per-author/editor basis, \ie \cmd{ifopcit} will yield \texttt{true} if the cited item is the same as the last one by this author\slash editor. The possible choices are:
该选项控制 \secref{aut:aux:tst} 节的 \cmd{ifopcit} 测试所需的“opcit”跟踪器。该特性类似于“ibidem”跟踪器,不同之处在于它跟踪的是基于某一作者或编辑的引用,即,如果引用项目与之前项目的作者或编者相同,那么 \cmd{ifopcit} 的结果为 \texttt{true}。可用选择有:

\begin{valuelist}
\item[true] %Enable the tracker in global mode.
在全局模式下启用该跟踪器。
\item[false] %Disable the tracker.
禁用该跟踪器。
\item[context] %Enable the tracker in context"=sensitive mode. In this mode, citations in footnotes and in the body text are tracked separately.
在环境敏感模式下启用该跟踪器。此时,脚注和正文中的引用会独立跟踪。
\item[strict] %Enable the tracker in strict mode. In this mode, potentially ambiguous references are suppressed. See \kvopt{ibidtracker}{strict} for details.
在严格模式下启用该跟踪器。此时将不跟踪那些含义不清的引用。详见 \kvopt{ibidtracker}{strict} 选项。
\item[constrict] %This mode combines the features of \opt{context} and \opt{strict}. See the explanation of \kvopt{ibidtracker}{constrict} for details.
该模式是 \opt{context} 和 \opt{strict} 的结合。详见  \kvopt{ibidtracker}{constrict} 选项的解释。
\end{valuelist}

%Note that this tracker is disabled in all floats, see \secref{aut:cav:flt}.
注意所有的浮动体都禁用该跟踪器,见 \secref{aut:cav:flt} 节。

\optitem[false]{loccittracker}{\opt{true}, \opt{false}, \opt{context}, \opt{strict}, \opt{constrict}}

%This option controls the <loccit> tracker which is required by the \cmd{ifloccit} test from \secref{aut:aux:tst}. This feature is similar to the <opcit> tracker except that it also checks whether the \prm{postnote} arguments match, \ie \cmd{ifloccit} will yield \texttt{true} if the citation refers to the same page cited before. The possible choices are:
该选项控制 \secref{aut:aux:tst} 节的 \cmd{ifloccit} 测试所需的“loccit”跟踪器。该特性类似于“opcit”跟踪器,不同之处在于它也会检查 \prm{postnote} 选项是否匹配,即,如果引用项目与之前引用指向的页数相同,那么 \cmd{ifloccit} 的结果为 \texttt{true}。可用选择有:

\begin{valuelist}
\item[true] %Enable the tracker in global mode.
在全局模式下启用该跟踪器。
\item[false] %Disable the tracker.
禁用该跟踪器。
\item[context] %Enable the tracker in context"=sensitive mode. In this mode, citations in footnotes and in the body text are tracked separately.
在环境敏感模式下启用该跟踪器。此时,脚注和正文中的引用会独立跟踪。
\item[strict] %Enable the tracker in strict mode. In this mode, potentially ambiguous references are suppressed. See \kvopt{ibidtracker}{strict} for details. In addition to that, this mode also checks if the \prm{postnote} argument is numerical (based on \cmd{ifnumerals} from \secref{aut:aux:tst}).
在严格模式下启用该跟踪器。此时将不跟踪那些含义不清的引用。详见 \kvopt{ibidtracker}{strict} 选项。此外,该模式也会检查 \prm{postnote} 选项是否是数值型的(基于 \secref{aut:aux:tst} 节的 \cmd{ifnumerals} 命令)。
\item[constrict] %This mode combines the features of \opt{context} and \opt{strict}. See the explanation of \kvopt{ibidtracker}{constrict} for details. In addition to that, this mode also checks if the \prm{postnote} argument is numerical (based on \cmd{ifnumerals} from \secref{aut:aux:tst}).
该模式是 \opt{context} 和 \opt{strict} 的结合。详见  \kvopt{ibidtracker}{constrict} 选项的解释。此外,该模式也会检查 \prm{postnote} 选项是否是数值型的(基于 \secref{aut:aux:tst} 节的 \cmd{ifnumerals} 命令)。
\end{valuelist}

%Note that this tracker is disabled in all floats, see \secref{aut:cav:flt}.
注意所有的浮动体都禁用该跟踪器,见 \secref{aut:cav:flt} 节。

\optitem[false]{idemtracker}{\opt{true}, \opt{false}, \opt{context}, \opt{strict}, \opt{constrict}}

%This option controls the <idem> tracker which is required by the \cmd{ifciteidem} test from \secref{aut:aux:tst}. The possible choices are:
该选项控制 \secref{aut:aux:tst} 节的 \cmd{ifciteidem} 测试所需的“idem”跟踪器。
可用选择有:

\begin{valuelist}
\item[true] %Enable the tracker in global mode.
在全局模式下启用该跟踪器。
\item[false] %Disable the tracker.
禁用该跟踪器。
\item[context] %Enable the tracker in context"=sensitive mode. In this mode, citations in footnotes and in the body text are tracked separately.
在环境敏感模式下启用该跟踪器。此时,脚注和正文中的引用会独立跟踪。
\item[strict] %This is an alias for \texttt{true}, provided only for consistency with the other trackers. Since <idem> replacements do not get ambiguous in the same way as <ibidem> or <op.~cit.>, the \texttt{strict} tracking mode does not apply to them.
该选项是 \texttt{true} 的别名。提供该选项只是为了和其它跟踪器保持一致。由于“idem”不会像“ibidem”或“op.~cit.”那样引起歧义,因此不必使用 \texttt{strict} 跟踪模式。
\item[constrict] %This mode is similar to \opt{context} with one additional condition: a reference in a footnote will only be considered as unambiguous if the current citation and the previous citation are given in the same footnote or in immediately consecutive footnotes.
该模式类似于 \opt{context},只有一个额外条件:对于脚注中的引用,只有当前引用和之前引用位于同一个或相连的脚注中才会认为是含义明确的。
\end{valuelist}

%Note that this tracker is disabled in all floats, see \secref{aut:cav:flt}.
注意所有的浮动体都禁用该跟踪器,见 \secref{aut:cav:flt} 节。

\boolitem[true]{parentracker}

%This option controls the parenthesis tracker which keeps track of nested parentheses and brackets. This information is used by \cmd{parentext} and \cmd{brackettext} from \secref{use:cit:txt}, \cmd{mkbibparens} and \cmd{mkbibbrackets} from \secref{aut:fmt:ich} and \cmd{bibopenparen}, \cmd{bibcloseparen}, \cmd{bibopenbracket}, \cmd{bibclosebracket} (also \secref{aut:fmt:ich}).

该选项控制括号跟踪器,
用于对嵌套的圆括号和方括号的跟踪。
所得信息用于 \secref{use:cit:txt} 节的  \cmd{parentext} 和 \cmd{brackettext} 命令、
\secref{aut:fmt:ich} 节的 \cmd{mkbibparens} 和 \cmd{mkbibbrackets} 命令,
以及同样来自于 \secref{aut:fmt:ich} 节的 \cmd{bibopenparen}, \cmd{bibcloseparen}, \cmd{bibopenbracket}, \cmd{bibclosebracket} 等命令。

\intitem[3]{maxparens}

%The maximum permitted nesting level of parentheses and brackets. If parentheses and brackets are nested deeper than this value, \biblatex will issue errors.

圆括号和方括号嵌套的最大层级。
如果嵌套深度大于该值则会报错。

\boolitem[false]{\prm{namepart}inits}

%When enabled, all $<$namepart$>$ name parts will be rendered as initials. The option will affect the \cmd{if$<$namepart$>$inits} test from \secref{aut:aux:tst}. The valid name parts are defined in the data model by the \cmd{DeclareDatamodelConstant} command (\secref{aut:bbx:drv}).

启用该选项时所有的 \prm{namepart} 姓名部分都会用首字母表示。
该选项会影响 \secref{aut:aux:tst} 节的 \cmd{if\prm{namepart}inits} 测试。
在数据模型中有效的姓名部分由 \cmd{DeclareDatamodelConstant} 命令定义,见 \secref{aut:bbx:drv} 节。

\boolitem[false]{terseinits}

%This option controls the format of all initials generated by \biblatex. If enabled, initials are rendered using a terse format without dots and spaces. For example, the initials of Donald Ervin Knuth would be rendered as <D.~E.> by default, and as <DE> if this option is enabled. The option will affect the \cmd{ifterseinits} test from \secref{aut:aux:tst}. The option works by redefining some macros which control the format of initials. See \secref{use:cav:nam} for details.

该选项控制 \biblatex 生成的首字母格式。
当启用时,首字母缩写将采用没有点号和空格的紧凑形式。
例如 Donald Ervin Knuth 的首字母缩写默认是“D.~E.”,而在此选项启用时会是“DE”。
该选项会影响 \secref{aut:aux:tst} 中的 \cmd{ifterseinits} 测试。
该选项会重新定义一些控制首字母格式的宏。详见 \secref{use:cav:nam} 节。

\boolitem[false]{labelalpha}

%Whether or not to provide the special fields \bibfield{labelalpha} and \bibfield{extraalpha}, see \secref{aut:bbx:fld} for details.
%This option is also settable on a per-type basis. See also \opt{maxalphanames} and \opt{minalphanames}. Table \ref{use:opt:tab1} summarises the various \opt{extra*} disambiguation counters and what they track.

是否提供特殊的域 \bibfield{labelalpha} 和 \bibfield{extraalpha},详见 \secref{aut:bbx:fld} 节。
该选项可以基于每一条目类型(per-type)设置。
另见 \opt{maxalphanames} 和 \opt{minalphanames} 选项。
表 \ref{use:opt:tab1} 总结了各种 \opt{extra*} 歧义消除的计数器以及它们所跟踪的选项。

\intitem[3]{maxalphanames}

%Similar to the \opt{maxnames} option but customizes the format of the \bibfield{labelalpha} field.

类似于 \opt{maxnames} 但用于定制 \bibfield{labelalpha} 域的格式。

\intitem[1]{minalphanames}

%Similar to the \opt{minnames} option but customizes the format of the \bibfield{labelalpha} field.
类似于 \opt{minnames} 但用于定制 \bibfield{labelalpha} 域的格式。

\boolitem[false]{labelnumber}

%Whether or not to provide the special field \bibfield{labelnumber}, see \secref{aut:bbx:fld} for details.
%This option is also settable on a per-type basis.

是否提供特殊域 \bibfield{labelnumber},详见  \secref{aut:bbx:fld} 节。
该选项可基于每一类型而设置。

\boolitem[false]{labeltitle}

%Whether or not to provide the special field \bibfield{extratitle}, see \secref{aut:bbx:fld} for details. Note that the special field \bibfield{labeltitle} is always provided and this option controls rather whether \bibfield{labeltitle} is used to generate \bibfield{extratitle} information. This option is also settable on a per-type basis. Table \ref{use:opt:tab1} summarises the various \opt{extra*} disambiguation counters and what they track.

是否提供特殊域\bibfield{extratitle},详见 \secref{aut:bbx:fld} 节。
请注意,特殊域 \bibfield{labeltitle} 总是提供的,
而该选项控制是否利用 \bibfield{labeltitle} 生成  \bibfield{extratitle} 信息。
该选项也可基于每一类型而设置。
表 \ref{use:opt:tab1} 总结了各种 \opt{extra*} 消除歧义的计数器以及所跟踪的选项。

\boolitem[false]{labeltitleyear}

%Whether or not to provide the special field \bibfield{extratitleyear}, see \secref{aut:bbx:fld} for details. Note that the special field \bibfield{labeltitle} is always provided and this option controls rather whether \bibfield{labeltitle} is used to generate \bibfield{extratitleyear} information. This option is also settable on a per-type basis. Table \ref{use:opt:tab1} summarises the various \opt{extra*} disambiguation counters and what they track.

是否提供特殊域  \bibfield{extratitleyear},详见 \secref{aut:bbx:fld} 节。
请注意,特殊域 \bibfield{labeltitle} 总是提供的,
而该选项控制是否利用 \bibfield{labeltitle} 生成  \bibfield{extratitleyear} 信息。
该选项也可基于每一类型而设置。
表 \ref{use:opt:tab1} 总结了各种 \opt{extra*} 消除歧义的计数器以及所跟踪的选项。

\boolitem[false]{labeldateparts}

%Whether or not to provide the special fields \bibfield{labelyear}, \bibfield{labelmonth}, \bibfield{labelday}, \bibfield{labelendyear}, \bibfield{labelendmonth}, \bibfield{labelendday}, \bibfield{labelhour}, \bibfield{labelendhour}, \bibfield{labelminute}, \bibfield{labelendminute}, \bibfield{labelsecond}, \bibfield{labelendsecond}, \bibfield{labelseason}, \bibfield{labelendseason}, \bibfield{labeltimezone}, \bibfield{labelendtimeone} and \bibfield{extrayear}, see \secref{aut:bbx:fld} for details.
%This option is also settable on a per-type basis. Table \ref{use:opt:tab1} summarises the various \opt{extra*} disambiguation counters and what they track.

是否提供特殊域 \bibfield{labelyear}, \bibfield{labelmonth}, \bibfield{labelday}, \bibfield{labelendyear}, \bibfield{labelendmonth}, \bibfield{labelendday}, \bibfield{labelhour}, \bibfield{labelendhour}, \bibfield{labelminute}, \bibfield{labelendminute}, \bibfield{labelsecond}, \bibfield{labelendsecond}, \bibfield{labelseason}, \bibfield{labelendseason}, \bibfield{labeltimezone}, \bibfield{labelendtimeone} 以及 \bibfield{extrayear},详见 \secref{aut:bbx:fld} 节。
该选项也可基于每一类型而设置。
表 \ref{use:opt:tab1} 总结了各种 \opt{extra*} 消除歧义的计数器以及所跟踪的选项。


\begin{table}
	\footnotesize
	\ttfamily
	\tablesetup
	\begin{tabularx}{\textwidth}{XXX}
		\toprule
		\multicolumn{1}{@{}H}{选项} &
		\multicolumn{1}{@{}H}{测试} &
		\multicolumn{1}{@{}H}{跟踪的域} \\
		\cmidrule(r){1-1}\cmidrule(r){2-2}\cmidrule(r){3-3}
		singletitle & \cmd{ifsingletitle} & labelname\\
		uniquetitle & \cmd{ifuniquetitle} & labeltitle\\
		uniquebaretitle & \cmd{ifuniquebaretitle} & labeltitle (当 labelname 为空时) \\
		uniquework  & \cmd{ifuniquework}  & labelname+labeltitle\\
		\bottomrule
	\end{tabularx}
	\caption{惟一性选项}%Work Uniqueness options
	\label{use:opt:wu}
\end{table}

\boolitem[false]{singletitle}

%Whether or not to provide the data required by the \cmd{ifsingletitle} test, see \secref{aut:aux:tst} for details. See \tabref{use:opt:wu} for details on what determines the data for this test.
%This option is also settable on a per-type basis.

是否提供 \cmd{ifsingletitle} 测试所需的数据,详见 \secref{aut:aux:tst} 节。
关于该测试中数据的控制因素详见 \tabref{use:opt:wu}。
该选项也可基于每一类型而设置。

\boolitem[false]{uniquetitle}

%Whether or not to provide the data required by the \cmd{ifuniquetitle} test, see \secref{aut:aux:tst} for details. See \tabref{use:opt:wu} for details on what determines the data for this test.
%This option is also settable on a per-type basis.

是否提供 \cmd{ifuniquetitle} 测试所需的数据,详见 \secref{aut:aux:tst} 节。
关于该测试中数据的控制因素详见 \tabref{use:opt:wu}。
该选项也可基于每一类型而设置。

\boolitem[false]{uniquebaretitle}

%Whether or not to provide the data required by the \cmd{ifuniquebaretitle} test, see \secref{aut:aux:tst} for details. See \tabref{use:opt:wu} for details on what determines the data for this test.
%This option is also settable on a per-type basis.

是否提供 \cmd{ifuniquebaretitle} 测试所需的数据,详见 \secref{aut:aux:tst} 节。
关于该测试中数据的控制因素详见 \tabref{use:opt:wu}。
该选项也可基于每一类型而设置。

\boolitem[false]{uniquework}

%Whether or not to provide the data required by the \cmd{ifuniquework} test, see \secref{aut:aux:tst} for details. See \tabref{use:opt:wu} for details on what determines the data for this test.
%This option is also settable on a per-type basis.

是否提供 \cmd{ifuniquework} 测试所需的数据,详见 \secref{aut:aux:tst} 节。
关于该测试中数据的控制因素详见 \tabref{use:opt:wu}。
该选项也可基于每一类型而设置。

\boolitem[false]{uniqueprimaryauthor}

%Whether or not to provide the data required by the \cmd{ifuniqueprimaryauthor} test, see \secref{aut:aux:tst} for details.

是否提供 \cmd{ifuniqueprimaryauthor} 测试所需的数据,详见 \secref{aut:aux:tst} 节。

\optitem[false]{uniquename}{\opt{true}, \opt{false}, \opt{init}, \opt{full}, \opt{allinit}, \opt{allfull},
\opt{mininit}, \opt{minfull}}

%Whether or not to update the \cnt{uniquename} counter, see \secref{aut:aux:tst} for details. This feature will disambiguate individual names in the \bibfield{labelname} list. This option is also settable on a per-type basis. The possible choices are:

是否更新 \cnt{uniquename} 计数器,详见 \secref{aut:aux:tst} 节。
该特性会消除 \bibfield{labelname} 列表中各个姓名的歧义。
该选项也可基于每一类型而设置。可用的选择有:

\begin{valuelist}
\item[true] %An alias for \opt{full}.
\opt{full} 的别称。
\item[false] %Disable this feature.
禁用该特性。
\item[init] %Disambiguate using initials only.
只使用首字母消除歧义。
\item[full] %Disambiguate using initials or full names, as required.
根据要求使用首字母或全名消除歧义。
\item[allinit] %Similar to \opt{init} but disambiguates all names in the \bibfield{labelname} list, beyond \opt{maxnames}\slash \opt{minnames}\slash \opt{uniquelist}.
类似于 \opt{init},但是会对 \bibfield{labelname} 列表中所有姓名消除歧义,
即便超出了 \opt{maxnames}\slash \opt{minnames}\slash \opt{uniquelist} 选项。
\item[allfull] %Similar to \opt{full} but disambiguates all names in the \bibfield{labelname} list, beyond \opt{maxnames}\slash \opt{minnames}\slash \opt{uniquelist}.
类似于 \opt{full},但是会对 \bibfield{labelname} 列表中所有姓名消除歧义,
即便超出了 \opt{maxnames}\slash \opt{minnames}\slash \opt{uniquelist} 选项。
\item[mininit] %A variant of \texttt{init} which only disambiguates names in lists with identical last names.
\texttt{init} 的变种,只对列表中有同一姓(last name)的姓名消除歧义。
\item[minfull] %A variant of \texttt{full} which only disambiguates names in lists with identical last names.
\texttt{full} 的变种,只对列表中有同一姓(last name)的姓名消除歧义。
\end{valuelist}
%
%Note that the \opt{uniquename} option will also affect \opt{uniquelist}, the \cmd{ifsingletitle} test, and the \bibfield{extrayear} field. See \secref{aut:cav:amb} for further details and practical examples.
请注意,\opt{uniquename} 选项也会影响 \opt{uniquelist} 选项、
\cmd{ifsingletitle} 测试,以及 \bibfield{extrayear} 域。
更多细节和实例见 \secref{aut:cav:amb} 节。

\optitem[false]{uniquelist}{\opt{true}, \opt{false}, \opt{minyear}}

%Whether or not to update the \cnt{uniquelist} counter, see \secref{aut:aux:tst} for details. This feature will disambiguate the \bibfield{labelname} list if it has become ambiguous after \opt{maxnames}\slash \opt{minnames} truncation. Essentially, it overrides \opt{maxnames}\slash \opt{minnames} on a per-field basis. This option is also settable on a per-type basis. The possible choices are:

是否更新 \cnt{uniquelist} 计数器,详见 \secref{aut:aux:tst} 节。
如果 \bibfield{labelname} 列表在 \opt{maxnames}\slash \opt{minnames} 截断后含义不清,
那么该特性会消除 \bibfield{labelname} 列表中的歧义。
本质上,该选项会覆盖基于每一域的 \opt{maxnames}\slash \opt{minnames} 设置。
该选项也可基于每一类型而设置。可用的选择有:

\begin{valuelist}
\item[true] %Disambiguate the \bibfield{labelname} list.
消除 \bibfield{labelname} 列表的歧义。
\item[false] %Disable this feature.
禁用该特性。
\item[minyear] %Disambiguate the \bibfield{labelname} list only if the truncated list is identical to another one with the same \bibfield{labelyear}. This mode of operation is useful for author-year styles and requires \kvopt{labeldateparts}{true}.
只有当被截断的列表与带有相同 \bibfield{labelyear} 的另一列表相同时才会消除 \bibfield{labelname} 列表的歧义。
该操作模式适用于作者---年份样式中,如果要求 \kvopt{labeldateparts}{true} 的场合。
\end{valuelist}
%
%Note that the \opt{uniquelist} option will also affect the \cmd{ifsingletitle} test and the \bibfield{extrayear} field. See \secref{aut:cav:amb} for further details and practical examples.
请注意,\opt{uniquelist} 选项也会影响 \cmd{ifsingletitle} 测试和 \bibfield{extrayear} 域。
更多细节和实例见 \secref{aut:cav:amb} 节。

\end{optionlist}

\begin{table}
\footnotesize
\ttfamily
\tablesetup
\begin{tabularx}{\textwidth}{XXXX}
\toprule
\multicolumn{1}{@{}H}{选项} &
\multicolumn{1}{@{}H}{激活域} & %Enabled field(s)
\multicolumn{1}{@{}H}{激活计数器} & %Enabled counter
\multicolumn{1}{@{}H}{计数器跟踪} \\ %Counter tracks
\cmidrule(r){1-1}\cmidrule(r){2-2}\cmidrule(r){3-3}\cmidrule{4-4}
labelalpha     & labelalpha       & extraalpha     &  label\\
labeldateparts & labelyear        & extradate      &  labelname+\\
		& labelmonth       &                &  labelyear\\
		& labelday         &                &  \\
		& labelendyear     &                &  \\
		& labelendmonth    &                &  \\
		& labelendday      &                &  \\
		& labelhour        &                &  \\
		& labelminute      &                &  \\
		& labelsecond      &                &  \\
		& labelendhour     &                &  \\
		& labelendminute   &                &  \\
		& labelendsecond   &                &  \\
		& labelseason      &                &  \\
		& labelendseason   &                &  \\
		& labeltimezone    &                &  \\
		& labelendtimezone &                &  \\
labeltitle     & \rmfamily{---}   & extratitle     &  labelname+labeltitle\\
labeltitleyear & \rmfamily{---}   & extratitleyear &  labeltitle+labelyear\\
\bottomrule
\end{tabularx}
\caption{歧义消除计数器}%Disambiguation counters
\label{use:opt:tab1}
\end{table}

\subsubsection{条目选项}%\subsubsection{Entry Options}
\label{use:opt:bib}

%Entry options are package options which determine how bibliography data entries are handled. They may be set at various scopes defined below.

条目选项是控制参考文献数据条目处理的包选项。
可以在以下不同尺度上设置。

\paragraph{导言区/类型/条目选项}%\paragraph{Preamble/Type/Entry Options}
\label{use:opt:bib:hyb}

%The following options are settable on a per"=type basis or on a per"=entry in the \bibfield{options} field. In addition to that, they may also be used in the optional argument to \cmd{usepackage} as well as in the configuration file and the document preamble. This is useful if you want to change the default behaviour globally.

下列选项可以基于类型或条目在 \bibfield{options} 域中设置。
此外还可以在 \cmd{usepackage} 的可选项以及配置文件和导言区中使用。
这可用于全局改变默认样式。

\begin{optionlist}

\boolitem[true]{useauthor}

%Whether the \bibfield{author} is used in labels and considered during sorting. This may be useful if an entry includes an \bibfield{author} field but is usually not cited by author for some reason. Setting \kvopt{useauthor}{false} does not mean that the \bibfield{author} is ignored completely. It means that the \bibfield{author} is not used in labels and ignored during sorting. The entry will then be alphabetized by \bibfield{editor} or \bibfield{title}. With the standard styles, the \bibfield{author} is printed after the title in this case. See also \secref{use:srt}.
%This option is also settable on a per-type and per-entry basis.

是否在标签和排序中使用 \bibfield{author} 域。
如果一个条目包含 \bibfield{author} 域但出于某种原因通常不以作者引用,该选项是很有用的。
设置 \kvopt{useauthor}{false} 并不意味着 \bibfield{author} 被完全忽略了,而只是在标签和排序中不使用。
该条目将按照 \bibfield{editor} 或 \bibfield{title} 域的字母顺序排列。
在标准样式中,此时 \bibfield{author} 会在标题后打印。参考 \secref{use:srt} 节。
该选项可以基于每一类型设置。

\boolitem[true]{useeditor}

%Whether the \bibfield{editor} replaces a missing \bibfield{author} in labels and during sorting. This may be useful if an entry includes an \bibfield{editor} field but is usually not cited by editor. Setting \kvopt{useeditor}{false} does not mean that the \bibfield{editor} is ignored completely. It means that the \bibfield{editor} does not replace a missing \bibfield{author} in labels and during sorting. The entry will then be alphabetized by \bibfield{title}. With the standard styles, the \bibfield{editor} is printed after the title in this case. See also \secref{use:srt}.
%This option is also settable on a per-type and per-entry basis.

在标签和排序中是否用 \bibfield{editor} 域来代替缺失的 \bibfield{author} 域。
如果一个条目包含 \bibfield{editor} 域但是通常不以作者引用,那么该选项是很有用的。
设置 \kvopt{useeditor}{false} 并不意味着 \bibfield{editor} 被完全忽略了,而只是在标签和排序中不用 \bibfield{editor} 来代替缺失的 \bibfield{author}。
该条目会以 \bibfield{title} 的字母顺序排列。
在标准样式中,此时 \bibfield{editor} 会在标题之后打印。参考 \secref{use:srt} 节。
该选项可以基于每一类型或条目而设置。

\boolitem[false]{usetranslator}

%Whether the \bibfield{translator} replaces a missing \bibfield{author}\slash \bibfield{editor} in labels and during sorting. Setting \kvopt{usetranslator}{true} does not mean that the \bibfield{translator} overrides the \bibfield{author}\slash \bibfield{editor}. It means that the \bibfield{translator} is considered as a fallback if the \bibfield{author}\slash \bibfield{editor} is missing or if \opt{useauthor} and \opt{useeditor} are set to \texttt{false}. In other words, in order to cite a book by translator rather than by author, you need to set the following options:
%This option is also settable on a per-type and per-entry basis.

在标签和排序中是否用 \bibfield{translator} 代替缺失的 \bibfield{author}\slash \bibfield{editor}。
设置 \kvopt{usetranslator}{true} 并不意味着 \bibfield{translator} 会覆盖 \bibfield{author}\slash \bibfield{editor},而只是当 \bibfield{author}\slash \bibfield{editor} 缺失或者  \opt{useauthor} 和 \opt{useeditor} 选项设置为 \texttt{false} 时作为后备。
也就是说,如果要按译者而不是作者引用一本书,你需要设置如下选项:

\begin{lstlisting}[style=bibtex]{}
@Book{...,
  options    = {useauthor=false,usetranslator=true},
  author     = {...},
  translator = {...},
  ...
\end{lstlisting}
%
%With the standard styles, the \bibfield{translator} is printed after the title by default. See also \secref{use:srt}.
该选项也可以基于每一类型或条目而设置。
在标准样式中,\bibfield{translator} 默认在标题之后打印,参考 \secref{use:srt} 节。

\boolitem[true]{use\prm{name}}

%As per \opt{useauthor}, \opt{useeditor} and \opt{usetranslator}, all name lists defined in the data model have an option controlling their behaviour in sorting and labelling automatically defined. Global, per-type and per-entry options called <use$<$name$>$>are automatically created.

按照 \opt{useauthor}、\opt{useeditor} 和 \opt{usetranslator},
数据模型中定义的所有的姓名列表都有一个选项用于控制自动定义的排序和标签行为。
此时会自动创建全局、基于类型和基于条目选项而调用的 \opt{use\prm{name}}。

\boolitem[false]{useprefix}

%Whether the default date model name part <prefix> (von, van, of, da, de, della, etc.) is considered when:

是否在以下几种情况下考虑默认数据模型中的姓名前缀部分(von、van、of,、da、de、della 等):

\begin{itemize}
\item %Printing the family name in citations
在引用中打印姓
\item %Sorting
排序
\item %Generation of certain types of labels
生成标签的某些类型
\item %Generating name uniqueness information
生成姓名惟一性信息。
\item %Formatting aspects of the bibliography
参考文献的格式方面
\end{itemize}
%
%For example, if this option is enabled, \biblatex precedes the family name with the prefix---Ludwig van Beethoven would be cited as «van Beethoven» and alphabetized as «Van Beethoven, Ludwig». If this option is disabled (the default), he is cited as «Beethoven» and alphabetized as «Beethoven, Ludwig van» instead.
%This option is also settable on a per-type scope. With \biblatexml datasources and the \bibtex extended name format supported by \biber, this is also settable on per-namelist and per-name scopes.
如果激活该选项,\biblatex 总是在姓氏前面加上该前缀。
例如 Ludwig van Beethoven 将引作“van Beethoven”并按照“Van Beethoven, Ludwig” 排序,
而如果未激活该选项(默认情况),将引作“Beethoven”并按照“Beethoven, Ludwig van”排序。
该选项也可基于每一类型设置。
当使用 \biblatexml 数据源以及 \biber 支持的 \BibTeX 扩展名格式时,
还可以基于每一姓名列表或姓名而设置。

\optitem{indexing}{\opt{true}, \opt{false}, \opt{cite}, \opt{bib}}

%The \opt{indexing} option is also settable per-type or per-entry basis. See \secref{use:opt:pre:gen} for details.
\opt{indexing} 选项也可以基于每一类型或条目设置。详见 \secref{use:opt:pre:gen} 节。

\end{optionlist}

\paragraph{类型/条目选项}%\paragraph{Type/Entry Options}
\label{use:opt:bib:ded}

%The following options are settable on a per"=type basis or on a per"=entry in the \bibfield{options} field. They are not available globally.
下列选项只能基于类型或条目在 \bibfield{options} 域中设置,而不能全局设置。

\begin{optionlist}

\boolitem[false]{skipbib}

%If this option is enabled, the entry is excluded from the bibliography but it may still be cited.
%This option is also settable on a per-type basis.

如果激活该选项,该条目在参考文献中将被排除在外但仍然可以引用。
该选项也可以基于每一类型设置。

\boolitem[false]{skipbiblist}

%If this option is enabled, the entry is excluded from and bibliography lists. It is still included in the bibliography and it may also be cited by shorthand etc.
%This option is also settable on a per-type basis.

如果激活该选项,该选项将在文献列表中被排除在外,
但仍然包含在参考文献中并可以被shorthand引用。
该选项也可以基于每一类型设置。

\boolitem[false]{skiplab}

%If this option is enabled, \biblatex will not assign any labels to the entry. It is not required for normal operation. Use it with care. If enabled, \biblatex can not guarantee unique citations for the respective entry and citations styles which require labels may fail to create valid citations for the entry.
%This option is also settable on a per-type basis.

如果激活该选项,\biblatex 不会给该条目分配标签。
正常操作不需要该选项。要小心使用。
当激活时,\biblatex 不能保证相应条目有唯一的引用,
而且那些需要标签的引用样式可能不能为该条目创建有效的引用。
该选项也可以基于每一类型设置。

\boolitem[false]{dataonly}

%Setting this option is equivalent to \kvopt{uniquename}{false}, \kvopt{uniquelist}{false, }\opt{skipbib}, \opt{skipbiblist}, and \opt{skiplab}. It is not required for normal operation. Use it with care.
%This option is also settable on a per-type basis.

设置该选项等价于设置 \kvopt{uniquename}{false}、\kvopt{uniquelist}{false}、\opt{skipbib}、\opt{skipbiblist} 和 \opt{skiplab}。
正常操作不需要该选项。要小心使用。
该选项也可以基于每一类型设置。

\end{optionlist}

\paragraph{条目选项}%\paragraph{Entry Only Options}
\label{use:opt:bib:entry}

%The following options are settable only on a per"=entry in the \bibfield{options} field. They are not available globally or per"=type.
下列选项只能基于条目在 \bibfield{options} 域中而不能全局或基于类型设置。

\begin{optionlist}

\valitem{labelnamefield}{fieldname}

%Specifies the field to consider first when looking for a \bibfield{labelname} candidate. It is essentially prepended to the search list created by \cmd{DeclareLabelname} for just this entry.
指定搜索 \bibfield{labelname} 时首先考虑的域。
本质上,只在该条目中该域会放到 \cmd{DeclareLabelname} 创建的搜索列表之前。

\valitem{labeltitlefield}{fieldname}

%Specifies the field to consider first when looking for a \bibfield{labeltitle} candidate. It is essentially prepended to the search list created by \cmd{DeclareLabeltitle} for just this entry.
指定搜索 \bibfield{labeltitle} 时首先考虑的域。
本质上,只在该条目中该域会放到 \cmd{DeclareLabeltitle} 创建的搜索列表之前。

\end{optionlist}

\subsubsection{遗留选项}

%The following legacy option may be used globally in the optional argument to \cmd{documentclass} or locally in the optional argument to \cmd{usepackage}:
下面的遗留选项可以全局地在 \cmd{documentclass} 的可选项中使用,
也可以局部地在 \cmd{usepackage} 的可选项中使用:

\begin{optionlist}
	
\legitem{openbib}\DeprecatedMark  %This option is provided for backwards compatibility with the standard LaTeX document classes. \opt{openbib} is similar to \kvopt{block}{par}.
该选项用于向后兼容标准 \LaTeX 文档类。
\opt{openbib} 类似于 \kvopt{block}{par}。

\end{optionlist}

\subsection{全局定制}%\subsection{Global Customization}
\label{use:cfg}

%Apart from writing new citation and bibliography styles, there are numerous ways to customize the styles which ship with this package. Customization will usually take place in the preamble, but there is also a configuration file for permanent adaptions. The configuration file may also be used to initialize the package options to a value different from the package default.

除了编写新的引用和文献样式,本宏包中还有很多定制样式的方法。
定制通常在导言区中进行,但也可以在配置文件中进行以便长期使用。
配置文件也可以用于将宏包选项从默认值初始化为不同的值。

\subsubsection{配置文件}%\subsubsection{Configuration File}
\label{use:cfg:cfg}

%If available, this package will load the configuration file \path{biblatex.cfg}. This file is read at the end of the package, immediately after the citation and bibliography styles have been loaded.

当可用时,本宏包会导入配置文件 \path{biblatex.cfg}。
该文件会在宏包的末尾,紧跟在引用和文献样式导入之后立即被读取。

\subsubsection{设置宏包选项}%\subsubsection{Setting Package Options}
\label{use:cfg:opt}

%The load-time package options in \secref{use:opt:ldt} must be given in the optional argument to \cmd{usepackage}. The package options in \secref{use:opt:pre} may also be given in the preamble. The options are executed with the following command:

\secref{use:opt:ldt} 节中的实时载入宏包选项必须在 \cmd{usepackage} 的可选项中给出。
\secref{use:opt:pre} 节中的宏包选项同样可以在导言区中给出。
以下命令用于执行选项:

\begin{ltxsyntax}

\cmditem{ExecuteBibliographyOptions}[entrytype, \dots]{key=value, \dots}

%This command may also be used in the configuration file to modify the default setting of a package option. Certain options are also settable on a per-type basis. In this case, the optional \prm{entrytype} argument specifies the entry type. The \prm{entrytype} argument may be a comma"=separated list of values.

该命令也可以在配置文件中使用,以修改宏包选项的默认设置。
某些选项还可以基于每一条目而设置。
此时,可选的 \prm{entrytype} 选项用来确定条目类型。
\prm{entrytype} 选项可以是逗号分隔的值列表。

\end{ltxsyntax}

\subsection{biblatex提供的标准参考文献样式}%Standard Styles
\label{use:xbx}

%This section provides a short description of all bibliography and citation styles which ship with the \biblatex package. If you want to write your own styles, see \secref{aut}.

我们常说的参考文献样式,在biblatex中,其实是著录样式和标注(引用)样式的统称。其中著录样式控制文献列表比如书后的文献表的格式,而标注样式则控制正文中引用参考文献所产生的标注的样式。

本节将简要描述本宏包附带的所有著录样式和标注(引用)样式。
如果你想自己写样式文件,请参考 \secref{aut} 节。

\subsubsection{标注样式}%\subsubsection{Citation Styles}
\label{use:xbx:cbx}

%The citation styles which come with this package implement several common citation schemes. All standard styles cater for the \bibfield{shorthand} field and support hyperlinks as well as indexing.

本宏包所带的引用样式实现了一些常见的引用格式。
所有的标准样式都支持 \bibfield{shorthand} 域,并且支持超链接和索引。


\begin{marglist}

\item[numeric]
%This style implements a numeric citation scheme similar to the standard bibliographic facilities of \latex. It should be employed in conjunction with a numeric bibliography style which prints the corresponding labels in the bibliography. It is intended for in-text citations. The style will set the following package options at load time: \kvopt{autocite}{inline}, \kvopt{labelnumber}{true}. This style also provides an additional preamble option called \opt{subentry} which affects the handling of entry sets. If this option is disabled, citations referring to a member of a set will point to the entire set. If it is enabled, the style supports citations like «[5c]» which point to a subentry in a set (the third one in this example). See the style example for details.
该样式实现与 \LaTeX 的标准文献工具类似的数值式引用格式。
它应当与某种能在参考文献中打印出相应标签的数值式文献样式一起使用,
并用于文内引用。
该样式在宏包载入时设置如下的宏包选项:\kvopt{autocite}{inline}、\kvopt{labelnumber}{true}。
该样式还额外提供了一个导言区选项 \opt{subentry},这会影响条目集的处理。
如果该选项被禁止,指向条目集中某一成员的引用会指向整个的条目集。
如果该选项被激活,该样式会支持类似于“[5c]”这样指向条目集中的子条目的引用(这个例子是第三个条目)。
详见样式例子。

\item[numeric-comp]
%A compact variant of the \texttt{numeric} style which prints a list of more than two consecutive numbers as a range. This style is similar to the \sty{cite} package and the \opt{sort\&compress} option of the \sty{natbib} package in numerical mode. For example, instead of «[8, 3, 1, 7, 2]» this style would print «[1--3, 7, 8]». It is intended for in-text citations. The style will set the following package options at load time: \kvopt{autocite}{inline}, \kvopt{sortcites}{true}, \kvopt{labelnumber}{true}. It also provides the \opt{subentry} option.
\texttt{numeric} 样式紧凑形式的变种,会将两个以上的连续数字打印成一个区间。
该样式类似于 \sty{cite} 宏包和数值模式中的 \sty{natbib} 宏包的 \opt{sort\&compress} 选项。
例如,“[8, 3, 1, 7, 2]”会变成“[1--3, 7, 8]”。
它用于文中引用。
该样式在宏包载入时设置如下的宏包选项:
\kvopt{autocite}{inline}、\kvopt{sortcites}{true}、\kvopt{labelnumber}{true}。
它也提供了 \opt{subentry} 选项。

\item[numeric-verb]
%A verbose variant of the \texttt{numeric} style. The difference affects the handling of a list of citations and is only apparent when multiple entry keys are passed to a single citation command. For example, instead of «[2, 5, 6]» this style would print «[2]; [5]; [6]». It is intended for in-text citations. The style will set the following package options at load time: \kvopt{autocite}{inline}, \kvopt{labelnumber}{true}. It also provides the \opt{subentry} option.
\texttt{numeric} 样式详细形式的变种。
不同之处在于对一组引用的处理,
并且只当不同的条目键值传递给单个引用命令时才会显示差异。
例如,“[2, 5, 6]”会变成“[2]; [5]; [6]”。
它用于文中引用。
该样式在宏包载入时设置如下的宏包选项:
\kvopt{autocite}{inline}、\kvopt{labelnumber}{true}。
它也提供了 \opt{subentry} 选项。

\item[alphabetic]
%This style implements an alphabetic citation scheme similar to the \path{alpha.bst} style of traditional \bibtex. The alphabetic labels resemble a compact author"=year style to some extent, but the way they are employed is similar to a numeric citation scheme. For example, instead of «Jones 1995» this style would use the label «[Jon95]». «Jones and Williams 1986» would be rendered as «[JW86]». This style should be employed in conjunction with an alphabetic bibliography style which prints the corresponding labels in the bibliography. It is intended for in-text citations. The style will set the following package options at load time: \kvopt{autocite}{inline}, \kvopt{labelalpha}{true}.

该样式实现的字母顺序引用格式类似于传统 \BibTeX 的 \path{alpha.bst} 样式。
字母标签某种程度上类似于紧凑的作者---年份样式,但是使用的方式类似于数字引用格式。
例如,“Jones 1995” 会是“[Jon95]”;而“Jones and Williams 1986”会缩写为“[JW86]”。
该样式应当与一种字母顺序文献样式一起使用,从而可以在参考文献中打印出相应的标签。
它用于文内引用。
该样式在载入时设置如下的宏包选项:\kvopt{autocite}{inline}、\kvopt{labelalpha}{true}。

\item[alphabetic-verb]
%A verbose variant of the \texttt{alphabetic} style. The difference affects the handling of a list of citations and is only apparent when multiple entry keys are passed to a single citation command. For example, instead of «[Doe92; Doe95; Jon98]» this style would print «[Doe92]; [Doe95]; [Jon98]». It is intended for in-text citations. The style will set the following package options at load time: \kvopt{autocite}{inline}, \kvopt{labelalpha}{true}.
\texttt{alphabetic} 样式详细格式的变种。
不同之处在于对一组引用的处理,
并且只当不同的条目键值传递给单个引用命令时才会显示差异。
例如“[Doe92; Doe95; Jon98]”会变成“[Doe92]; [Doe95]; [Jon98]”。
它用于文内引用。
该样式在载入时设置如下的宏包选项:\kvopt{autocite}{inline}、\kvopt{labelalpha}{true}。

\item[authoryear]
%This style implements an author"=year citation scheme. If the bibliography contains two or more works by the same author which were all published in the same year, a letter is appended to the year. For example, this style would print citations such as «Doe 1995a; Doe 1995b; Jones 1998». This style should be employed in conjunction with an author"=year bibliography style which prints the corresponding labels in the bibliography. It is primarily intended for in-text citations, but it could also be used with citations given in footnotes. The style will set the following package options at load time: \kvopt{autocite}{inline}, \kvopt{labeldate}{true}, \kvopt{uniquename}{full}, \kvopt{uniquelist}{true}.
该样式实现了作者---年份引用格式。
如果参考文献中包含多个由同一作者同一年份发表的作品,那么年份后会附加一个字母用以区分。
例如该样式会打印出“Doe 1995a; Doe 1995b; Jones 1998”这样的引用。
该样式应当与一种作者---年份文献样式一起使用,从而可以在参考文献中打印出相应的标签。
它起初用于文内引用,但也可以用在脚注中。
该样式在载入时设置如下的宏包选项:
\kvopt{autocite}{inline}、\kvopt{labeldate}{true}、\kvopt{uniquename}{full}、\kvopt{uniquelist}{true}。

\item[authoryear-comp]
%A compact variant of the \texttt{authoryear} style which prints the author only once if subsequent references passed to a single citation command share the same author. If they share the same year as well, the year is also printed only once. For example, instead of «Doe 1995b; Doe 1992; Jones 1998; Doe 1995a» this style would print «Doe 1992, 1995a,b; Jones 1998». It is primarily intended for in-text citations, but it could also be used with citations given in footnotes. The style will set the following package options at load time: \kvopt{autocite}{inline}, \kvopt{sortcites}{true}, \kvopt{labeldate}{true}, \kvopt{uniquename}{full}, \kvopt{uniquelist}{true}.
\texttt{authoryear} 样式紧凑格式的变种。
如果传递给单个引用命令的一列文献作者相同,那么该作者只会打印一次。
如果它们年份也相同,那么年份也只会打印一次。
例如,“Doe 1995b; Doe 1992; Jones 1998; Doe 1995a”在该样式下会变成“Doe 1992, 1995a,b; Jones 1998”。
它起初用于文中引用,但也可以用在脚注中。
该样式在载入时设置如下的宏包选项:
\kvopt{autocite}{inline}、\kvopt{sortcites}{true}、\kvopt{labeldate}{true}、\kvopt{uniquename}{full}、\kvopt{uniquelist}{true}。

\item[authoryear-ibid]
%A variant of the \texttt{authoryear} style which replaces repeated citations by the abbreviation \emph{ibidem} unless the citation is the first one on the current page or double-page spread, or the \emph{ibidem} would be ambiguous in the sense of the package option \kvopt{ibidtracker}{constrict}. The style will set the following package options at load time: \kvopt{autocite}{inline}, \kvopt{labeldate}{true}, \kvopt{uniquename}{full}, \kvopt{uniquelist}{true}, \kvopt{ibidtracker}{constrict}, \kvopt{pagetracker}{true}. This style also provides an additional preamble option called \opt{ibidpage}. See the style example for details.
\texttt{authoryear} 样式的变种,会用缩略语 \emph{ibidem} 替代重复的引用,
除非该引用在当前页或跨页是第一次出现,
或者 \emph{ibidem} 在宏包选项 \kvopt{ibidtracker}{constrict} 的意义下表意不清。
该样式在载入时设置如下的宏包选项:
\kvopt{autocite}{inline}、\kvopt{labeldate}{true}、\kvopt{uniquename}{full}、\kvopt{uniquelist}{true}、\kvopt{ibidtracker}{constrict}、\kvopt{pagetracker}{true}。
该样式还额外提供了一个导言区选项 \opt{ibidpage}。
详见样式例子。

\item[authoryear-icomp]
%A style combining \texttt{authoryear-comp} and \texttt{authoryear-ibid}. The style will set the following package options at load time: \kvopt{autocite}{inline}, \kvopt{labeldate}{true}, \kvopt{uniquename}{full}, \kvopt{uniquelist}{true}, \kvopt{ibidtracker}{constrict}, \kvopt{pagetracker}{true}, \kvopt{sortcites}{true}. This style also provides an additional preamble option called \opt{ibidpage}. See the style example for details.
一个结合了 \texttt{authoryear-comp} 和 \texttt{authoryear-ibid} 的样式。
该样式在载入时设置如下的宏包选项:
\kvopt{autocite}{inline}、\kvopt{labeldate}{true}、\kvopt{uniquename}{full}、\kvopt{uniquelist}{true}、\kvopt{ibidtracker}{constrict}、\kvopt{pagetracker}{true}、\kvopt{sortcites}{true}。
该样式还额外提供了一个导言区选项 \opt{ibidpage}。
详见样式例子。

\item[authortitle]
%This style implements a simple author"=title citation scheme. It will make use of the \bibfield{shorttitle} field, if available. It is intended for citations given in footnotes. The style will set the following package options at load time: \kvopt{autocite}{footnote}, \kvopt{uniquename}{full}, \kvopt{uniquelist}{true}.
该样式实现了一个简单的作者---标题引用格式。
如果可用的话,它会使用 \bibfield{shorttitle} 域。
它用于脚注中给出的引用。
该样式在载入时设置如下的宏包选项:
\kvopt{autocite}{footnote}、\kvopt{uniquename}{full}、\kvopt{uniquelist}{true}。

\item[authortitle-comp]
%A compact variant of the \texttt{authortitle} style which prints the author only once if subsequent references passed to a single citation command share the same author. For example, instead of «Doe, \emph{First title}; Doe, \emph{Second title}» this style would print «Doe, \emph{First title}, \emph{Second title}». It is intended for citations given in footnotes. The style will set the following package options at load time: \kvopt{autocite}{footnote}, \kvopt{sortcites}{true}, \kvopt{uniquename}{full}, \kvopt{uniquelist}{true}.
\texttt{authortitle} 样式紧凑格式的变种。
如果传递给单个引用命令的一列文献作者相同,那么该作者只会打印一次。
例如,“Doe, \emph{First title}; Doe, \emph{Second title}”在此样式下会变成“Doe, \emph{First title}, \emph{Second title}”。
它用于脚注中给出的引用。
该样式在载入时设置如下的宏包选项:
\kvopt{autocite}{footnote}、\kvopt{sortcites}{true}、\kvopt{uniquename}{full}、\kvopt{uniquelist}{true}。

\item[authortitle-ibid]
%A variant of the \texttt{authortitle} style which replaces repeated citations by the abbreviation \emph{ibidem} unless the citation is the first one on the current page or double-page spread, or the \emph{ibidem} would be ambiguous in the sense of the package option \kvopt{ibidtracker}{constrict}. It is intended for citations given in footnotes. The style will set the following package options at load time: \kvopt{autocite}{footnote}, \kvopt{uniquename}{full}, \kvopt{uniquelist}{true}, \kvopt{ibidtracker}{constrict}, \kvopt{pagetracker}{true}. This style also provides an additional preamble option called \opt{ibidpage}. See the style example for details.
\texttt{authortitle} 样式的变种,会用缩略语 \emph{ibidem} 替代重复的引用,
除非该引用在当前页或跨页是第一次出现,
或者 \emph{ibidem} 在宏包选项 \kvopt{ibidtracker}{constrict} 的意义下表意不清。
它用于脚注中给出的引用。
该样式在载入时设置如下的宏包选项:
\kvopt{autocite}{footnote}、\kvopt{uniquename}{full}、\kvopt{uniquelist}{true}、\kvopt{ibidtracker}{constrict}、\kvopt{pagetracker}{true}。
该样式还额外提供了一个导言区选项 \opt{ibidpage}。
详见样式例子。

\item[authortitle-icomp]
%A style combining the features of \texttt{authortitle-comp} and \texttt{authortitle-ibid}. The style will set the following package options at load time: \kvopt{autocite}{footnote}, \kvopt{uniquename}{full}, \kvopt{uniquelist}{true}, \kvopt{ibidtracker}{constrict}, \kvopt{pagetracker}{true}, \kvopt{sortcites}{true}. This style also provides an additional preamble option called \opt{ibidpage}. See the style example for details.
结合了 \texttt{authortitle-comp} 和 \texttt{authortitle-ibid} 特性的样式。
该样式在载入时设置如下的宏包选项:
\kvopt{autocite}{footnote}、\kvopt{uniquename}{full}、\kvopt{uniquelist}{true}、\kvopt{ibidtracker}{constrict}、\kvopt{pagetracker}{true}、\kvopt{sortcites}{true}。
该样式还额外提供了一个导言区选项 \opt{ibidpage}。
详见样式例子。

\item[authortitle-terse]
%A terse variant of the \texttt{authortitle} style which only prints the title if the bibliography contains more than one work by the respective author\slash editor. This style will make use of the \bibfield{shorttitle} field, if available. It is suitable for in-text citations as well as citations given in footnotes. The style will set the following package options at load time: \kvopt{autocite}{inline}, \kvopt{singletitle}{true}, \kvopt{uniquename}{full}, \kvopt{uniquelist}{true}.
\texttt{authortitle} 样式简明格式的变种,
如果文献中包含多个相应作者/编辑的作品,那么只会打印出标题。
如果可用的话,该样式会使用 \bibfield{shorttitle} 域。
它在文中引用和脚注中引用都适用。
该样式在载入时设置如下的宏包选项:\kvopt{autocite}{inline}、\kvopt{singletitle}{true}、\kvopt{uniquename}{full}、\kvopt{uniquelist}{true}。

\item[authortitle-tcomp]
%A style combining the features of \texttt{authortitle-comp} and \texttt{authortitle-terse}. This style will make use of the \bibfield{shorttitle} field, if available. It is suitable for in-text citations as well as citations given in footnotes. The style will set the following package options at load time: \kvopt{autocite}{inline}, \kvopt{sortcites}{true}, \kvopt{singletitle}{true}, \kvopt{uniquename}{full}, \kvopt{uniquelist}{true}.
结合了 \texttt{authortitle-comp} 和 \texttt{authortitle-terse} 特性的样式。
如果可用的话,该样式会使用 \bibfield{shorttitle} 域。
它在文内引用和脚注中引用都适用。
该样式在载入时设置如下的宏包选项:
\kvopt{autocite}{inline}、\kvopt{sortcites}{true}、\kvopt{singletitle}{true}、\kvopt{uniquename}{full}、\kvopt{uniquelist}{true}。

\item[authortitle-ticomp]
%A style combining the features of \texttt{authortitle-icomp} and \texttt{authortitle-terse}. In other words: a variant of the \texttt{authortitle-tcomp} style with an \emph{ibidem} feature. This style is suitable for in-text citations as well as citations given in footnotes. It will set the following package options at load time: \kvopt{autocite}{inline}, \kvopt{ibidtracker}{constrict}, \kvopt{pagetracker}{true}, \kvopt{sortcites}{true}, \kvopt{singletitle}{true}, \kvopt{uniquename}{full}, \kvopt{uniquelist}{true}. This style also provides an additional preamble option called \opt{ibidpage}. See the style example for details.
结合了 \texttt{authortitle-icomp} 和 \texttt{authortitle-terse} 特性的样式。
换句话说就是带有 \emph{ibidem} 特性的 \texttt{authortitle-tcomp} 样式变种。
它在文中引用和脚注中引用都适用。
该样式在载入时设置如下的宏包选项:
\kvopt{autocite}{inline}、\kvopt{ibidtracker}{constrict}、\kvopt{pagetracker}{true}、\kvopt{sortcites}{true}、\kvopt{singletitle}{true}、\kvopt{uniquename}{full}、\kvopt{uniquelist}{true}。
该样式还额外提供了一个导言区选项 \opt{ibidpage}。
详见样式例子。

\item[verbose]
%A verbose citation style which prints a full citation similar to a bibliography entry when an entry is cited for the first time, and a short citation afterwards. If available, the \bibfield{shorttitle} field is used in all short citations. If the \bibfield{shorthand} field is defined, the shorthand is introduced on the first citation and used as the short citation thereafter. This style may be used without a list of references and shorthands since all bibliographic data is provided on the first citation. It is intended for citations given in footnotes. The style will set the following package options at load time: \kvopt{autocite}{footnote}, \kvopt{citetracker}{context}. This style also provides an additional preamble option called \opt{citepages}. See the style example for details.
完整信息样式(多词标注样式、详细标注样式),在第一次引用某条目时会打印出类似于参考文献那样的长标注格式,并且在之后打印出短格式。
如果可用的话,\bibfield{shorttitle} 域会用在所有的短格式中。
如果 \bibfield{shorthand} 域有定义,该shorthand会在第一次引用时被引入并在之后作为短格式被使用。
由于在第一次引用时提供了所有的文献数据,因此该样式的使用不需要参考文献和shorthand列表。
它用于脚注中给出的引用。
该样式在载入时设置如下的宏包选项:\kvopt{autocite}{footnote}、\kvopt{citetracker}{context}。
该样式还额外提供了一个导言区选项 \opt{citepages}。
详见样式例子。

\item[verbose-ibid]
%A variant of the \texttt{verbose} style which replaces repeated citations by the abbreviation \emph{ibidem} unless the citation is the first one on the current page or double-page spread, or the \emph{ibidem} would be ambiguous in the sense of \kvopt{ibidtracker}{strict}. This style is intended for citations given in footnotes. The style will set the following package options at load time: \kvopt{autocite}{footnote}, \kvopt{citetracker}{context}, \kvopt{ibidtracker}{constrict}, \kvopt{pagetracker}{true}. This style also provides additional preamble options called \opt{ibidpage} and \opt{citepages}. See the style example for details.
\texttt{verbose} 样式的变种,会用缩略语 \emph{ibidem} 替代重复的引用,
除非该引用在当前页或跨页是第一次出现,
或者 \emph{ibidem} 在宏包选项 \kvopt{ibidtracker}{strict} 的意义下表意不清。
它用于脚注中给出的引用。
该样式在载入时设置如下的宏包选项:
\kvopt{autocite}{footnote}、\kvopt{citetracker}{context}、\kvopt{ibidtracker}{constrict}、\kvopt{pagetracker}{true}。
该样式还额外提供了导言区选项 \opt{ibidpage} 和 \opt{citepages}。
详见样式例子。

\item[verbose-note]
%This style is similar to the \texttt{verbose} style in that it prints a full citation similar to a bibliography entry when an entry is cited for the first time, and a short citation afterwards. In contrast to the \texttt{verbose} style, the short citation is a pointer to the footnote with the full citation. If the bibliography contains more than one work by the respective author\slash editor, the pointer also includes the title. If available, the \bibfield{shorttitle} field is used in all short citations. If the \bibfield{shorthand} field is defined, it is handled as with the \texttt{verbose} style. This style may be used without a list of references and shorthands since all bibliographic data is provided on the first citation. It is exclusively intended for citations given in footnotes. The style will set the following package options at load time: \kvopt{autocite}{footnote}, \kvopt{citetracker}{context}, \kvopt{singletitle}{true}. This style also provides additional preamble options called \opt{pageref} and \opt{citepages}. See the style example for details.
该样式与 \texttt{verbose} 样式类似,
会在第一次引用某条目时打印出类似于参考文献那样的长格式,并且在之后打印出短格式。
与 \texttt{verbose} 样式不同的是,短格式会指向带有长格式的脚注。
如果文献包含了多个同一作者/编辑的作品,该短格式会带有标题。
如果可用的话,所有的短格式会使用 \bibfield{shorttitle} 域。
如果 \bibfield{shorthand} 域有定义,它会被 \texttt{verbose} 样式处理。
由于在第一次引用时提供了所有的文献数据,因此该样式的使用不需要参考文献和shorthand列表。
该样式仅仅用于脚注中给出的引用。
该样式在载入时设置如下的宏包选项:
\kvopt{autocite}{footnote}、\kvopt{citetracker}{context}、\kvopt{singletitle}{true}。
该样式还额外提供了导言区选项 \opt{pageref} 和 \opt{citepages}。
详见样式例子。

\item[verbose-inote]
%A variant of the \texttt{verbose"=note} style which replaces repeated citations by the abbreviation \emph{ibidem} unless the citation is the first one on the current page or double-page spread, or the \emph{ibidem} would be ambiguous in the sense of \kvopt{ibidtracker}{strict}. This style is exclusively intended for citations given in footnotes. It will set the following package options at load time: \kvopt{autocite}{footnote}, \kvopt{citetracker}{context}, \kvopt{ibidtracker}{constrict}, \kvopt{singletitle}{true}, \kvopt{pagetracker}{true}. This style also provides additional preamble options called \opt{ibidpage}, \opt{pageref}, and \opt{citepages}. See the style example for details.
\texttt{verbose-note} 样式的变种,会用缩略语 \emph{ibidem} 替代重复的引用,
除非该引用在当前页或跨页是第一次出现,
或者 \emph{ibidem} 在宏包选项 \kvopt{ibidtracker}{strict} 的意义下表意不清。
该样式仅仅用于脚注中给出的引用。
该样式在载入时设置如下的宏包选项:
\kvopt{autocite}{footnote}、\kvopt{citetracker}{context}、\kvopt{ibidtracker}{constrict}、\kvopt{singletitle}{true}、\kvopt{pagetracker}{true}。
该样式还额外提供了导言区选项 \opt{ibidpage}、\opt{pageref} 和 \opt{citepages}。
详见样式例子。

\item[verbose-trad1]
%This style implements a traditional citation scheme. It is similar to the \texttt{verbose} style in that it prints a full citation similar to a bibliography entry when an item is cited for the first time, and a short citation afterwards. Apart from that, it uses the scholarly abbreviations \emph{ibidem}, \emph{idem}, \emph{op.~cit.}, and \emph{loc.~cit.} to replace recurrent authors, titles, and page numbers in repeated citations in a special way. If the \bibfield{shorthand} field is defined, the shorthand is introduced on the first citation and used as the short citation thereafter. This style may be used without a list of references and shorthands since all bibliographic data is provided on the first citation. It is intended for citations given in footnotes. The style will set the following package options at load time: \kvopt{autocite}{footnote}, \kvopt{citetracker}{context}, \kvopt{ibidtracker}{constrict}, \kvopt{idemtracker}{constrict}, \kvopt{opcittracker}{context}, \kvopt{loccittracker}{context}. This style also provides additional preamble options called \opt{ibidpage}, \opt{strict}, and \opt{citepages}. See the style example for details.
该样式实现了传统的引用格式。
与 \texttt{verbose} 样式类似,
它会在第一次引用某条目时打印出类似于参考文献那样的长格式,并且在之后打印出短格式。
此外,它在重复的引用中使用学术性缩略语 \emph{ibidem}、\emph{idem}、\emph{op.~cit.} 和 \emph{loc.~cit.} 来代替重复的作者、标题、页码数。
如果 \bibfield{shorthand} 域有定义,那么会在第一次引用时被引入并在之后作为短格式被使用。
由于在第一次引用时提供了所有的文献数据,因此该样式的使用不需要参考文献和shorthand列表。
它用于脚注中给出的引用。
该样式在载入时设置如下的宏包选项:
\kvopt{autocite}{footnote}、\kvopt{citetracker}{context}、\kvopt{ibidtracker}{constrict}、\kvopt{idemtracker}{constrict}、\kvopt{opcittracker}{context}、\kvopt{loccittracker}{context}。
该样式还额外提供了导言区选项 \opt{ibidpage}、\opt{strict} 和 \opt{citepages}。
详见样式例子。

\item[verbose-trad2]
%Another traditional citation scheme. It is also similar to the \texttt{verbose} style but uses scholarly abbreviations like \emph{ibidem} and \emph{idem} in repeated citations. In contrast to the \texttt{verbose-trad1} style, the logic of the \emph{op.~cit.} abbreviations is different in this style and \emph{loc.~cit.} is not used at all. It is in fact more similar to \texttt{verbose-ibid} and \texttt{verbose-inote} than to \texttt{verbose-trad1}. The style will set the following package options at load time: \kvopt{autocite}{footnote}, \kvopt{citetracker}{context}, \kvopt{ibidtracker}{constrict}, \kvopt{idemtracker}{constrict}. This style also provides additional preamble options called \opt{ibidpage}, \opt{strict}, and \opt{citepages}. See the style example for details.
另外一种传统引用格式。
它同样类似于 \texttt{verbose} 样式但是在重复的引用中使用 \emph{ibidem} 和 \emph{idem} 等学术性缩略语。
与 \texttt{verbose-trad1} 样式不同的是,
\emph{op.~cit.} 缩略语的逻辑有所不同,并且不使用 \emph{loc.~cit.}。
事实上它更类似于 \texttt{verbose-ibid} 和 \texttt{verbose-inote} 而不是 \texttt{verbose-trad1}。
该样式在载入时设置如下的宏包选项:
\kvopt{autocite}{footnote}、\kvopt{citetracker}{context}、\kvopt{ibidtracker}{constrict}、\kvopt{idemtracker}{constrict}。
该样式还额外提供了导言区选项 \opt{ibidpage}、\opt{strict} 和 \opt{citepages}。
详见样式例子。

\item[verbose-trad3]
%Yet another traditional citation scheme. It is similar to the \texttt{verbose-trad2} style but uses the scholarly abbreviations \emph{ibidem} and \emph{op.~cit.} in a slightly different way. The style will set the following package options at load time: \kvopt{autocite}{footnote}, \kvopt{citetracker}{context}, \kvopt{ibidtracker}{constrict}, \kvopt{loccittracker}{constrict}. This style also provides additional preamble options called \opt{strict} and \opt{citepages}. See the style example for details.
仍然是一种传统的引用格式。
它类似于 \texttt{verbose-trad2} 样式,
但是使用缩略语 \emph{ibidem} 和 \emph{op.~cit.} 的方式稍有不同。
该样式在载入时设置如下的宏包选项:
\kvopt{autocite}{footnote}、\kvopt{citetracker}{context}、\kvopt{ibidtracker}{constrict}、\kvopt{loccittracker}{constrict}。
该样式也额外提供了导言区选项 \opt{strict} 和 \opt{citepages}。
详见样式例子。

\item[reading]
%A citation style which goes with the bibliography style by the same name. It simply loads the \texttt{authortitle} style.
一个同名的文献样式所带的引用样式,会载入 \texttt{authortitle} 样式。

\end{marglist}

%The following citation styles are special purpose styles. They are not intended for the final version of a document:
下列样式具有特殊目的,不用于文档的最终版本。

\begin{marglist}

\item[draft]
%A draft style which uses the entry keys in citations. The style will set the following package options at load time: \kvopt{autocite}{plain}.
在引用中使用条目键的草稿样式。
该样式在载入时设置如下的宏包选项:\kvopt{autocite}{plain}。

\item[debug]
%This style prints the entry key rather than some kind of label. It is intended for debugging only and will set the following package options at load time: \kvopt{autocite}{plain}.
该样式会打印出条目键而不是标签。
它只用于调试,在载入时设置如下的宏包选项:\kvopt{autocite}{plain}。

\end{marglist}

\subsubsection{著录样式}%\subsubsection{Bibliography Styles}
\label{use:xbx:bbx}

%All bibliography styles which come with this package use the same basic format for the individual bibliography entries. They only differ in the kind of label printed in the bibliography and the overall formatting of the list of references. There is a matching bibliography style for every citation style. Note that some bibliography styles are not mentioned below because they simply load a more generic style. For example, the bibliography style \texttt{authortitle-comp} will load the \texttt{authortitle} style.

本宏包所带的所有文献样式对于每一文献条目都使用相同的基本格式。
不同之处仅仅在于参考文献中打印的标签种类和文献列表的总体格式。
每一个引用样式都有一个对应的文献样式。
请注意,一些文献样式仅仅载入了另外更一般的样式,因此这里没有提及。
例如,文献样式 \texttt{authortitle-comp} 会载入 \texttt{authortitle} 样式。

\begin{marglist}

\item[numeric]
%This style prints a numeric label similar to the standard bibliographic facilities of \latex. It is intended for use in conjunction with a numeric citation style. Note that the \bibfield{shorthand} field overrides the default label. The style will set the following package options at load time: \kvopt{labelnumber}{true}. This style also provides an additional preamble option called \opt{subentry} which affects the formatting of entry sets. If this option is enabled, all members of a set are marked with a letter which may be used in citations referring to a set member rather than the entire set. See the style example for details.
该样式打印出类似于 \LaTeX 标准文献功能的数值标签。
它应与数值引用样式结合使用。
请注意,\bibfield{shorthand} 域会覆盖默认标签。
该样式在载入时设置如下的宏包选项:\kvopt{labelnumber}{true}。
该样式还额外提供了一个导言区选项 \opt{subentry},这会影响条目集的处理。
如果该选项被激活,条目集中的所有成员都会用一个字母标记,这可用于集成员的引用而不是整个条目集。
详见样式例子。

\item[alphabetic]
%This style prints an alphabetic label similar to the \path{alpha.bst} style of traditional \bibtex. It is intended for use in conjunction with an alphabetic citation style. Note that the \bibfield{shorthand} field overrides the default label. The style will set the following package options at load time: \kvopt{labelalpha}{true}, \kvopt{sorting}{anyt}.
该样式打印的字母顺序标签类似于传统 \BibTeX 的 \path{alpha.bst} 样式。
它应与字母顺序引用样式结合使用。
请注意,\bibfield{shorthand} 域会覆盖默认标签。
该样式在载入时设置如下的宏包选项:\kvopt{labelalpha}{true}、\kvopt{sorting}{anyt}。

\item[authoryear]
%This style differs from the other styles in that the publication date is not printed towards the end of the entry but rather after the author\slash editor. It is intended for use in conjunction with an author"=year citation style. Recurring author and editor names are replaced by a dash unless the entry is the first one on the current page or double-page spread. This style provides an additional preamble option called \opt{dashed} which controls this feature. It also provided a preamble option called \opt{mergedate}. See the style example for details. The style will set the following package options at load time: \kvopt{labeldate}{true}, \kvopt{sorting}{nyt}, \kvopt{pagetracker}{true}, \kvopt{mergedate}{true}.
该样式不同于其它样式之处在于,发表日期不是在条目的末尾而是在作者/编辑之后。
它应与一个作者---年份引用样式结合使用。
重复的作者和编辑名会用短横线代替,除非该条目是当前页或跨页的第一个。
该样式额外提供了导言区选项 \opt{dashed} 来控制该特征。
此外还额外提供了导言区选项 \opt{mergedate}。
详见样式例子。
该样式在载入时设置如下的宏包选项:
\kvopt{labeldate}{true}、\kvopt{sorting}{nyt}、\kvopt{pagetracker}{true}、 \kvopt{mergedate}{true}。

\item[authortitle]
%This style does not print any label at all. It is intended for use in conjunction with an author"=title citation style. Recurring author and editor names are replaced by a dash unless the entry is the first one on the current page or double-page spread. This style also provides an additional preamble option called \opt{dashed} which controls this feature. See the style example for details. The style will set the following package options at load time: \kvopt{pagetracker}{true}.
该样式不会打印出任何标签。
它应与一个作者---年份引用样式结合使用。
重复的作者和编辑名会用短横线代替,除非该条目是当前页或跨页的第一个。
该样式额外提供了一个导言区选项 \opt{dashed} 来控制该特征。
详见样式例子。
该样式在载入时设置如下的宏包选项:\kvopt{pagetracker}{true}。

\item[verbose]
%This style is similar to the \texttt{authortitle} style. It also provides an additional preamble option called \opt{dashed}. See the style example for details. The style will set the following package options at load time: \kvopt{pagetracker}{true}.
该样式类似于 \texttt{authortitle} 样式。
该样式额外提供了一个导言区选项 \opt{dashed}。
详见样式例子。
该样式在载入时设置如下的宏包选项:\kvopt{pagetracker}{true}。

\item[reading]
%This special bibliography style is designed for personal reading lists, annotated bibliographies, and similar applications. It optionally includes the fields \bibfield{annotation}, \bibfield{abstract}, \bibfield{library}, and \bibfield{file} in the bibliography. If desired, it also adds various kinds of short headers to the bibliography. This style also provides the additional preamble options \opt{entryhead}, \opt{entrykey}, \opt{annotation}, \opt{abstract}, \opt{library}, and \opt{file} which control whether or not the corresponding items are printed in the bibliography. See the style example for details. See also \secref{use:use:prf}. The style will set the following package options at load time: \kvopt{loadfiles}{true}, \kvopt{entryhead}{true}, \kvopt{entrykey}{true}, \kvopt{annotation}{true}, \kvopt{abstract}{true}, \kvopt{library}{true}, \kvopt{file}{true}.
这一特殊的文献样式是为个人阅读列表、带有注释的文献和类似应用而设计的。
它选择性地在参考文献中包含 \bibfield{annotation}、\bibfield{abstract}、\bibfield{library} 和 \bibfield{file} 等域。
如果需要的话,它还会在参考文献中添加不同种类的短标题。
该样式还额外提供了导言区选项 \opt{entryhead}、\opt{entrykey}、\opt{annotation}、\opt{abstract}、\opt{library} 和 \opt{file} 来控制是否在参考文献中打印相应的项目。
详见样式例子。见 \secref{use:use:prf} 节。
该样式在载入时设置如下的宏包选项:
\kvopt{loadfiles}{true}、\kvopt{entryhead}{true}、\kvopt{entrykey}{true}、\kvopt{annotation}{true}、\kvopt{abstract}{true}、\kvopt{library}{true}、\kvopt{file}{true}。

\end{marglist}

%The following bibliography styles are special purpose styles. They are not intended for the final version of a document:
下列样式具有特殊目的,不用于文档的最终版本。

\begin{marglist}

\item[draft]
%This draft style includes the entry keys in the bibliography. The bibliography will be sorted by entry key. The style will set the following package options at load time: \kvopt{sorting}{debug}.
草稿样式会在参考文献中包含条目键。
文献会按照条目键排序。
该样式在载入时设置如下的宏包选项:\kvopt{sorting}{debug}。

\item[debug]
%This style prints all bibliographic data in tabular format. It is intended for debugging only and will set the following package options at load time: \kvopt{sorting}{debug}.
该样式会以表格形式打印出所有的文献数据。
它只用于调试,在载入时设置如下的宏包选项:\kvopt{sorting}{debug}。

\end{marglist}

\subsection{关联条目}%\subsection{Related Entries}
\label{use:rel}

%Almost all bibliography styles require authors to specify certain types of relationship between entries such as «Reprint of», «Reprinted in» etc. It is impossible to provide data fields to cover all of these relationships and so \biblatex provides a general mechanism for this using the entry fields \bibfield{related}, \bibfield{relatedtype} and \bibfield{relatedstring}. A related entry does not need to be cited and does not appear in the bibliography itself (unless of course it is also cited itself independently) as a clone is taken of the related entry to be used as a data source. The \bibfield{relatedtype} field should specify a localization string which will be printed before the information from the related entries is printed, for example «Orig. Pub. as». The \bibfield{relatedstring} field can be used to override the string determined via \bibfield{relatedtype}. Some examples:

几乎所有的文献样式都需要作者去确定条目之间的某些关系类型,例如“Reprint of”、“Reprinted in”等等。
当然不可能通过提供数据域来覆盖所有的关系,
为此,\biblatex 通过使用条目域 \bibfield{related}、\bibfield{relatedtype} 和 \bibfield{relatedstring} 提供了一种一般性的机制。
被关联的条目不需要被引用,本身也不会出现在参考文献中(当然,除非它自己另外单独被引用),
而是作为数据源被拷贝一份副本。
\bibfield{relatedtype} 域需要确定在相关联条目的信息前打印的本地化字符串,例如“Orig. Pub. as”。
\bibfield{relatedstring} 域可以用于覆盖那些通过 \bibfield{relatedtype} 确定的字符串。
一些例子如下:

\begin{lstlisting}[style=bibtex]{}
@Book{key1,
  ...
  related     = {key2},
  relatedtype = {reprintof},
  ...
}

@Book{key2,
  ...
}
\end{lstlisting}
%
%Here we specify that entry \texttt{key1} is a reprint of entry \texttt{key2}. In the bibliography driver for \texttt{Book} entries, when \cmd{usebibmacro\{related\}} is called for entry \texttt{key1}:
这里我们指定条目 \texttt{key1} 是条目 \texttt{key2} 的重印本。
在 \texttt{Book} 条目的文献驱动里,
当为条目 \texttt{key1} 而调用 \cmd{usebibmacro\{related\}} 时:

\begin{itemize}
\item % If the localisation string «\texttt{reprintof}» is defined, it is printed in the \texttt{relatedstring:reprintof} format. If this formatting directive is undefined, the string is printed in the \texttt{relatedstring:default} format.
如果本地化字符串 “\texttt{reprintof}” 有定义,
那么将以 \texttt{relatedstring:reprintof} 格式打印出来。
如果该格式指令没有定义,这些字符串将以 \texttt{relatedstring:default} 格式打印。
\item %If the \texttt{related:reprintof} macro is defined, it is used to format the information contained in entry \texttt{key2}, otherwise the \texttt{related:default} macro is used
如果宏 \texttt{related:reprintof} 有定义,
那么将用于确定条目 \texttt{key2} 包含的信息的格式,否则将使用宏 \texttt{related:default}。
\item %If the \texttt{related:reprintof} format is defined, it is used to format both the localization string and data. If this format is not defined, then the \texttt{related} format is used instead.
如果 \texttt{related:reprintof} 格式有定义,
那么将用于格式化本地化字符串和数据;
如果该格式没有定义,将使用 \texttt{related} 格式。
\end{itemize}
%
%It is also supported to have cascading and/or circular relations:
也支持串联或者循环关系:

\begin{lstlisting}[style=bibtex]{}
@Book{key1,
  ...
  related     = {key2},
  relatedtype = {reprintof},
  ...
}

@Book{key2,
  ...
  related     = {key3},
  relatedtype = {translationof},
  ...
}

@Book{key3,
  ...
  related     = {key2},
  relatedtype = {translatedas},
  ...
}
\end{lstlisting}
%
%Multiple relations to the same entry are also possible:
也可以实现同一条目的多重关系:
\begin{lstlisting}[style=bibtex]{}
@MVBook{key1,
  ...
  related     = {key2,key3},
  relatedtype = {multivolume},
  ...
}

@Book{key2,
  ...
}

@Book{key3,
  ...
}
\end{lstlisting}
%
%Note the the order of the keys in lists of multiple related entries is important. The data from multiple related entries is printed in the order of the keys listed in this field. See \secref{aut:ctm:rel} for a more details on the mechanisms behind this feature. You can turn this feature off using the package option \opt{related} from \secref{use:opt:pre:gen}.
请注意,多重关联条目列表中的顺序是很重要的。
多重关联条目的数据将按照该域中所列的顺序打印。
关于该功能背后的机制请参考 \secref{aut:ctm:rel} 节。
可以通过 \secref{use:opt:pre:gen} 中的宏包选项 \opt{related} 来关闭该功能。

%You can use the \bibfield{relatedoptions} to set options on the related entry data clone. This is useful if you need to override the \opt{dataonly} option which is set by default on all related entry clones. For example, if you will expose some of the names in the related clone in your document, you may want to have them disambiguated from names in other entries but normally this won't happen as related clones have the per"=entry \opt{dataonly} option set and this in turn sets \kvopt{uniquename}{false} and \kvopt{uniquelist}{false}. In such a case, you can set \bibfield{relatedoptions} to just \opt{skiplab, skipbib, skipbiblist}.

可以使用 \bibfield{relatedoptions} 域为关联条目数据克隆设置选项。
如果你想覆盖\opt{dataonly} 选项,该选项是所有关联条目克隆的默认设置,那么该域会很有用。
例如,如果你想在文档中展示克隆的关联条目中的一些作者,同时希望它们与其它条目的作者区分清楚,但是正常情况下这是无法实现的,因为克隆的关联条目由基于每一条目的 \opt{dataonly} 选项设置,而它又设置了 \kvopt{uniquename}{false} 和 \kvopt{uniquelist}{false},这样 biblatex 不会对这一克隆的关联条目做任何姓名非歧义处理。这种情况下,你可以设置 \bibfield{relatedoptions} 为 \opt{skiplab, skipbib},来覆盖默认的\opt{dataonly} 选项。

\subsection{排序选项}
\label{use:srt}

%This package supports fully customisable sorting templates for the bibliography. The default global sorting template is selected with the \opt{sorting} package option from \secref{use:opt:pre:gen}. Apart from the regular data fields there are also some special fields which may be used to optimize the sorting of the bibliography. \Apxref{apx:srt:a1, apx:srt:a2} give an outline of the default alphabetic sorting templates supported by \biblatex. Chronological sorting templates are listed in \apxref{apx:srt:chr}. A few explanations concerning the default templates are in order.

本宏包支持多种文献排序格式。
排序格式由 \secref{use:opt:pre:gen} 节中的 \opt{sorting} 宏包选项确定。
除了常规的数据域之外,还有一些特殊域也可用于优化文献排序。
\Apxref{apx:srt:a1, apx:srt:a2} 大致概述了 \biblatex 支持的字母顺序排序格式。
而年代顺序排序格式则列在 \apxref{apx:srt:chr} 中。
以下依次是这些格式的一些解释。

%The first item considered in the sorting process is always the \bibfield{presort} field of the entry. If this field is undefined, \biblatex will use the default value <\texttt{mm}> as a presort string. The next item considered is the \bibfield{sortkey} field. If this field is defined, it serves as the master sort key. Apart from the \bibfield{presort} field, no further data is considered in this case. If the \bibfield{sortkey} field is undefined, sorting continues with the name. The package will try using the \bibfield{sortname}, \bibfield{author}, \bibfield{editor}, and \bibfield{translator} fields, in this order. Which fields are considered also depends on the setting of the \opt{use$<$name$>$} options. If all such options are disabled, the \bibfield{sortname} field is ignored as well. Note that all name fields are responsive to \opt{maxnames} and \opt{minnames}. If no name field is available, either because all of them are undefined or because all \opt{use$<$name$>$} options are disabled, \biblatex will fall back to the \bibfield{sorttitle} and \bibfield{title} fields as a last resort. The remaining items are, in various order: the \bibfield{sortyear} field, if defined, or the first four digits of the \bibfield{year} field otherwise; the \bibfield{sorttitle} field, if defined, or the \bibfield{title} field otherwise; the \bibfield{volume} field. Note that the sorting schemes shown in \apxref{apx:srt:a2} include an additional item: \bibfield{labelalpha} is the label used by <alphabetic> bibliography styles. Strictly speaking, the string used for sorting is \bibfield{labelalpha}~+ \bibfield{extraalpha}. The sorting schemes in \apxref{apx:srt:a2} are intended to be used in conjunction with alphabetic styles only.

在排序过程中首先要考虑的事项总是条目的 \bibfield{presort} 域。
如果该域没有定义,\biblatex 会使用缺省值“\texttt{mm}”作为预排序字符串。
其次考虑的是 \bibfield{sortkey} 域。
如果该域有定义,它将作为主要的排序关键字。
此时除了 \bibfield{presort} 域,将不考虑其它信息。
如果 \bibfield{sortkey} 域没有定义,排序将使用姓名信息。
本宏包将依次尝试使用 \bibfield{sortname}、\bibfield{author}、\bibfield{editor} 和 \bibfield{translator} 等域。
考虑哪些域也取决于 \opt{useauthor}、\opt{useeditor} 和 \opt{usetranslator} 选项的设置。
如果这三个选项都没有启用,那么 \bibfield{sortname} 也将被忽略。
请注意,所有的名称域都与 \opt{maxnames} 和 \opt{minnames} 有关。
如果没有名称域是合适的,或者由于它们没有定义、或者由于 \opt{use\prm{name}} 域都未启用,
那么 \biblatex 将采用  \bibfield{sorttitle} 和 \bibfield{title} 作为最后的备选。
余下考虑的诸项依次是:\bibfield{sortyear} 域(如果给出的话),否则考虑 \bibfield{year} 域的前四个数字;
\bibfield{sorttitle} 域(如果给出的话),否则考虑  \bibfield{title} 域;
\bibfield{volume} 域。
请注意,\apxref{apx:srt:a2} 展示的排序格式包括了额外一项:
\bibfield{labelalpha} 域是“alphabetic”文献样式所使用的标签。
严格地讲,用于排序的字符串是 \bibfield{labelalpha}~+ \bibfield{extraalpha}。
\apxref{apx:srt:a2} 中的排序格式只可以与字母顺序样式联合使用。

%The chronological sorting templates presented in \apxref{apx:srt:chr} also make use of the \bibfield{presort} and \bibfield{sortkey} fields, if defined. The next item considered is the \bibfield{sortyear} or the \bibfield{year} field, depending on availability. The \opt{ynt} scheme extracts the first four Arabic figures from the field. If both fields are undefined, the string \texttt{9999} is used as a fallback value. This means that all entries without a year will be moved to the end of the list. The \opt{ydnt} scheme is similar in concept but sorts the year in descending order. As with the \opt{ynt} scheme, the string \texttt{9999} is used as a fallback value. The remaining items are similar to the alphabetic sorting schemes discussed above. Note that the \opt{ydnt} sorting scheme will only sort the date in descending order. All other items are sorted in ascending order as usual.

\apxref{apx:srt:chr} 展示的年代排序格式同样使用域 \bibfield{presort} 和 \bibfield{sortkey} (如果有定义的话)。
其次考虑的是 \bibfield{sortyear} 或者 \bibfield{year} 域,这当然取决于是否可用。
\opt{ynt} 格式将从该域中提取前四个数字。
如果这两个域都没有定义,那么将使用后备值 \texttt{9999}。
这意味着没有年份的条目都会移动到列表末尾。
\opt{ydnt} 格式从概念上也是类似的,不过是用降序排列年份。
与 \opt{ynt} 格式一样,后备值是 \texttt{9999}。
余下考虑的项与上面讨论的字母排序格式类似。
请注意,\opt{ydnt} 排序格式只对日期按照降序排列。
其它项仍和平常一样按照升序排列。

%Using special fields such as \bibfield{sortkey}, \bibfield{sortname}, or \bibfield{sorttitle} is usually not required. The \biblatex package is quite capable of working out the desired sorting order by using the data found in the regular fields of an entry. You will only need them if you want to manually modify the sorting order of the bibliography or if any data required for sorting is missing. Please refer to the field descriptions in \secref{bib:fld:spc} for details on possible uses of the special fields.

通常来说不需要使用 \bibfield{sortkey}、\bibfield{sortname} 或 \bibfield{sorttitle} 等特殊域。
\biblatex 宏包通过使用条目常规域的数据就很容易得到所需的排列顺序。
只有当你想手动修改文献排序或者所需的数据缺失时,你才需要使用这些特殊域。
关于特殊域的可能用法请参考 \secref{bib:fld:spc} 节中的描述。


\subsection{数据注解}%\subsection{Data Annotations}
\label{use:annote}
%Ideally, there should be no formatting information in a bibliography data file, however, sometimes such questionable practice seems to the only way in which the desired results can be achieved. Data annotations are a way of addressing this by allowing users to attach semantic information (rather than typographical markup) to information in a bibliography data source so that the information can be used at markup time by a style. For example, if you wanted to highlight certain names in a work depending on whether they were a student author (indicated by a superscript asterisk in the references) or a corresponding author (indicated by bold face), then you might be tempted to try:
理想状态下,文献数据文件中不应当有格式信息。
然而,有时只有通过这种有争议的做法才能实现想要的结果。
数据注解(data annotations) 就是一种解决该问题的方法。
通过允许用户在文献数据源中添加某种语义信息(而不是排版标记),
使得文献样式可以在标记时使用该信息。
例如,如果想要按照如下规则高亮某些作品中的姓名:学生作者在文献中用上标星号表示,而通讯作者用粗体表示;
那么,可以尝试如下方法:

\begin{lstlisting}[style=bibtex]{}
@MISC{Article1,
  AUTHOR = {Last1\textsuperscript{*}, First1 and \textbf{Last2}, \textbf{First2} and Last3, First3}
}
\end{lstlisting}
%
%There are several problems with this. Firstly, it will break \bibtex's fragile name parsing routines and probably won't compile at all. Secondly, it is not only mixing up data with markup, it does so in a hard-coded way: this data can't easily be shared and used with other styles. While it is possible to achieve this formatting using \biblatex internals in a style or document, this is a complex and unreliable method which many users will not wish to use.
这一做法有一些问题。
首先,它会打断 \BibTeX 脆弱的姓名解析程序指令,可能根本不能编译。
其次,数据与标记的混合是硬编码的:其它样式不易共享和使用该数据。
当然,在样式或者文件中使用 \biblatex 内部指令可能实现该格式,
但是这一做法比较复杂而且不可靠,很多用户不愿意使用。

%In order to address these issues, \biblatex has a general data annotation facility which allows you to attach a comma"=separated list of annotations to data fields, items within data field lists (like names) and even parts of specific items such as parts of names (given name, family name etc.). There are macros provided to check for annotations which can be used in formatting directives.
为了处理这些问题,\biblatex 提供了一般性的数据注解功能,
使得可以向数据域、数据域列表中的项目(例如姓名),以及某些项目的一部分(例如姓、名等姓名部分)中附加逗号分隔列表作为注解。
此外还提供了一些宏来检查可以用于格式指令的注解。

%There are three «scopes» for data annotations, in order of increasing specificity:
数据注解有三种“尺度”,按照特性增加的顺序依次为:
\begin{itemize}
\item \opt{field}---%applied to top-level fields in a data source entry
用于数据源条目中的顶层域
\item \opt{item}---%applied to items within a list field in a data source entry
用于数据源条目中列表域中的项目
\item \opt{part}---%applied to parts within items within a list field in a data source entry
用于数据源条目中列表域中项目的一部分
\end{itemize}
%
%Data annotations are supported for \bibtex and \biblatexml data sources.
\BibTeX 和 \biblatexml 数据源都支持数据注解。

%Attaching annotations to data is easy in \biblatexml data sources as they are specified by simple XML attributes. Continuing with the example above, we would have:
在 \biblatexml 数据源中添加数据注解是很容易的,
因为可以通过简单的XML属性来指定。
继续上面的例子,我们有:

\begin{lstlisting}[language=xml]
<bltx:entries xmlns:bltx="http://biblatex-biber.sourceforge.net/biblatexml">
  <bltx:entry id="test" entrytype="misc">
    <bltx:names type="author">
      <bltx:name>
        <bltx:namepart type="given" initial="F">First1</bltx:namepart>
        <bltx:namepart type="family" initial="L" annotation="student">Last1</bltx:namepart>
      </bltx:name>
      <bltx:name annotation="corresponding">
        <bltx:namepart type="given" initial="F">First2</bltx:namepart>
        <bltx:namepart type="family" initial="L">Last2</bltx:namepart>
      </bltx:name>
      <bltx:name>
        <bltx:namepart type="given" initial="F">First3</bltx:namepart>
        <bltx:namepart type="family" initial="L">Last3</bltx:namepart>
      </bltx:name>
    </bltx:names>
  </bltx:entry>
</bltx:entries>
\end{lstlisting}
%
%Here the annotations are attached in an obvious way to the data items. With \bibtex data sources, the format for annotations is not quite as intuitive:
这里,向数据项目中添加注解的方式是很显然的。
而在 \BibTeX 数据源中,注解的格式就没有那么直观了:

\begin{lstlisting}[style=bibtex]{}
@MISC{ann1,
  AUTHOR    = {Last1, First1 and Last2, First2 and Last3, First3},
  AUTHOR+an = {1:family=student;2=corresponding},
}
\end{lstlisting}
%
%Here the field name suffix \texttt{+an} is a user-definable\footnote{See \biber's \opt{--annotation-marker} option.} suffix which marks a data field as an annotation of the unsuffixed field. The format of annotation fields in \bibtex data sources is is as follows:
这里域姓名后缀 \texttt{+an} 可以由用户定义\footnote{
	见 \biber 的 \opt{--annotation-marker} 选项。},
用于标记某个数据域为去掉后缀的域的注解。
\BibTeX 注解域的格式如下:

\begin{lstlisting}
<annotationspecs> ::= <annotationspec> [ ";" <annotationspec> ]
<annotationspec>  ::= [ <itemcount> [ ":" <part> ] ] "=" <annotations>
<annotations>     ::= <annotation> [ "," <annotation> ]
<annotation>      ::= (string)
\end{lstlisting}
%
%That is, one or more specifications separated by semi-colons. Each specification is an equals sign followed by a comma"=separated list of annotation keywords. To annotation a specific item in a list, put the number of the list item before the equals sign (lists start at 1). If you need to annotate a specific part of the list item, give its name after the list item number, preceded by a colon. Name part names are defined in the data model, see \secref{aut:bbx:drv}. Some examples:
也就是说,多个特性之间由分号分隔。
每一特性是一个等号后跟一个逗号分隔的注解关键字列表。
为了为列表中某一项作注解,需要将该列表项的编号放在等号前面(列表从1开始编号)。
如果需要为列表项的某一部分做注解,需要将该部分名放在编号之后,并且前接一个冒号。
姓名部分的名称在数据模型中有定义,见 \secref{aut:bbx:drv} 节。
以下是一些例子:

\begin{lstlisting}[style=bibtex]{}
AUTHOR      = {Last1, First1 and Last2, First2 and Last3, First3},
AUTHOR+an   = {3:given=annotation1, annotation2},
TITLE       = {A title},
TITLE+an    = {=a title annotation, another title annotation},
LANGUAGE    = {english and french},
LANGUAGE+an = {1=annotation3; 2=annotation4}
}
\end{lstlisting}
%
%To access the annotation information when formatting bibliography data, three macros are provided, corresponding to the three annotation scopes:
为了在文献格式中获取注解信息,
这里提供了三个宏,分别对应与相应的注解尺度:

\begin{ltxsyntax}

\cmditem{iffieldannotation}{annotation}{true}{false}

%Executes \prm{true} if the current data field has an annotation \prm{annotation} and false otherwise.

如果当前数据域有注解,那么执行 \prm{true},否则为 false。

\cmditem{ifitemannotation}{annotation}{true}{false}

%Executes \prm{true} if the current item in the current data field has an annotation \prm{annotation} and false otherwise.

如果当前数据域的当前项目有注解,那么执行 \prm{true},否则为 false。

\cmditem{ifpartannotation}{part}{annotation}{true}{false}

%Executes \prm{true} if the part named \prm{part} in current item in the current data field has an annotation \prm{annotation} and false otherwise.

如果当前数据域中当前项目中名为 \prm{part} 的部分有注解,那么执行 \prm{true},否则为 false。

\end{ltxsyntax}
%
%These macros are available in the same places as \cmd{currentfield}, \cmd{currentlist} and \cmd{currentname} (see \secref{aut:bib:fmt}), that is, inside formatting directives. They automatically determine the name of the current data field being processed and also the current \opt{listcount} value which determines the current item in list fields. Parts such as name parts need to be named explicitly. As an example of how to use the annotation information to solve the problem originally presented in this section, this could be used in the name formatting directives to put an asterisk after all family names annotated as «student»:
这些宏的使用场合与 \cmd{currentfield}, \cmd{currentlist} 和 \cmd{currentname} 等命令相同(见 \secref{aut:bib:fmt} 节),
即,在格式指令内部。
它们自动确定当前被处理的数据域的名称,以及能够确定列表域中当前项目的 \opt{listcount} 值。
姓名部分等项目部分需要显式地指明。
下面的例子可以用于姓名格式指令,说明如何使用注解信息来解决本节之前提出的问题:
在所有注解为“student”的姓之后加上星号:

\begin{lstlisting}[style=latex]{}
  \ifpartannotation{family}{student}
    {\textsuperscript{*}}
    {}%
\end{lstlisting}
%
%To put the given and family names of name list items annotated as «corresponding» in boldface:
将标记为 “corresponding”的姓名列表项中的姓和名加粗:

\begin{lstlisting}[style=latex]{}
\renewcommand*{\mkbibnamegiven}[1]{%
  \ifitemannotation{corresponding}
    {\textbf{#1}}
    {#1}}

\renewcommand*{\mkbibnamefamily}[1]{%
  \ifitemannotation{corresponding}
    {\textbf{#1}}
    {#1}}
\end{lstlisting}

\subsection{参考文献命令}%\subsection{Bibliography Commands}
\label{use:bib}

\subsubsection{数据源}%\subsubsection{Resources}
\label{use:bib:res}

\begin{ltxsyntax}

\cmditem{addbibresource}[options]{resource}

%Adds a \prm{resource}, such as a \file{.bib} file, to the default resource list. This command is only available in the preamble. It replaces the \cmd{bibliography} legacy command. Note that files must be specified with their full name, including the extension. Do not omit the \file{.bib} extension from the filename. Also note that the \prm{resource} is a single resource. Invoke \cmd{addbibresource} multiple times to add more resources, for example:

将 \prm{resource} 添加到默认资源列表中,例如 \file{.bib} 文件。
该命令只能在导言区中使用。
它取代了过时的 \cmd{bibliography} 命令。
请注意,文件名包括扩展名,所以不要省略文件名中的 \file{.bib} 扩展名。
另外要注意的是,\prm{resource} 只能是一个单独的数据源。
添加更多的资源需要多次调用 \cmd{addbibresource} 命令,例如:
\begin{ltxexample}
\addbibresource{bibfile1.bib}
\addbibresource{bibfile2.bib}
\addbibresource[location=remote]{http://www.citeulike.org/bibtex/group/9517}
\addbibresource[location=remote,label=lan]{ftp://192.168.1.57/~user/file.bib}
\end{ltxexample}
%
%Since the \prm{resource} string is read in a verbatim-like mode, it may contain arbitrary characters. The only restriction is that any curly braces must be balanced. The following \prm{options} are available:
由于 \prm{resource} 字符串的读取类似于抄录模式,因此它可以包含任意的字符。
唯一的限制是其中任何的花括号必须左右匹配。
可用的 \prm{options} 如下:

\begin{optionlist*}

\valitem{label}{identifier}

%Assigns a label to a resource. The \prm{identifier} may be used in place of the full resource name in the optional argument of \env{refsection} (see \secref{use:bib:sec}).

给该数据源分配一个标签。
\prm{identifier} 可以用于在 \env{refsection} 环境的可选参数中以取代该数据源的全名
(见 \secref{use:bib:sec} 节)。

\valitem[local]{location}{location}

%The location of the resource. The \prm{location} may be either \texttt{local} for local resources or \texttt{remote} for \acr{URL}s. Remote resources require \biber. The protocols \acr{HTTP} and \acr{FTP} are supported. The remote \acr{URL} must be a fully qualified path to a \file{bib} file or a \acr{URL} which returns a \file{bib} file.

数据源的地址。
\prm{location} 可以是 \texttt{local} 或者 \texttt{remote},
分别对应本地数据和在线 \acr{URL} 数据。
远程资源需要 \biber{} 程序。
支持 \acr{HTTP} 和 \acr{FTP} 协议。
远程的 \acr{URL} 必须是 \file{bib} 文件的合法路径全称或者是返回 \file{bib} 文件的 \acr{URL}。

\valitem[file]{type}{type}

%The type of resource. Currently, the only supported type is \texttt{file}.

资源的类型。目前唯一支持的类型是 \texttt{file}。

\valitem[bibtex]{datatype}{datatype}

%The data type (format) of the resource. The following formats are currently supported:

资源的数据类型(格式)。目前支持以下格式:

\begin{valuelist}[zoterordfxml]

\item[bibtex] %\bibtex format.
\BibTeX 格式。

\item[biblatexml] %Experimental XML format for \biblatex. See \secref{apx:biblatexml}.
针对 \biblatex 的实验性 XML 格式。见 \secref{apx:biblatexml}。

\end{valuelist}

\end{optionlist*}


\cmditem{addglobalbib}[options]{resource}

%This command differs from \cmd{addbibresource} in that the \prm{resource} is added to the global resource list. The difference between default resources and global resources is only relevant if there are reference sections in the document and the optional argument of \env{refsection} (\secref{use:bib:sec}) is used to specify alternative resources which replace the default resource list. Any global resources are added to all reference sections.

该命令不同于 \cmd{addbibresource} 之处在于将 \prm{resource} 添加到全局数据源列表中。
不过,只有当文档中有参考文献章节并且使用 \env{refsection} 环境的可选参数(见 \secref{use:bib:sec} 节)
作为确定代替默认资源列表的备选资源时,考虑默认数据源和全局数据源的不同才是有意义的。
任何全局资源将被添加到所有的参考文件章节中。

\cmditem{addsectionbib}[options]{resource}

%This command differs from \cmd{addbibresource} in that the resource \prm{options} are registered but the \prm{resource} not added to any resource list. This is only required for resources which 1) are given exclusively in the optional argument of \env{refsection} (\secref{use:bib:sec}) and 2) require options different from the default settings. In this case, \cmd{addsectionbib} is employed to qualify the \prm{resource} prior to using it by setting the appropriate \prm{options} in the preamble. The \opt{label} option may be useful to assign a short name to the resource.

该命令与 \cmd{addbibresource} 的不同之处在于,会记录数据源的 \prm{options} 但是 \prm{resource} 没有添加到任何数据源列表中。
有该需求的场合是 (1) 该数据源仅仅用于 \env{refsection} 环境的可选参数中(\secref{use:bib:sec} 节);
(2) 该数据源需要不同于默认设置的选项。
此时,\cmd{addsectionbib} 会在导言区中设置合适的 \prm{options},从而在其使用前声明 \prm{resource}。
\opt{label} 选项可以用于分配给该资源一个简短的名称。

\cmditem{bibliography}{bibfile, \dots}|\DeprecatedMark|

%The legacy command for adding bibliographic resources, supported for backwards compatibility. Like \cmd{addbibresource}, this command is only available in the preamble and adds resources to the default resource list. Its argument is a comma"=separated list of \file{bib} files. The \file{.bib} extension may be omitted from the filename. Invoking this command multiple times to add more files is permissible. This command is deprecated. Please consider using \cmd{addbibresource} instead.

添加文献资源的过时命令,仅处于向后兼容性而支持。
类似 \cmd{addbibresource},该命令只能在导言区中使用,并将资源添加到默认资源列表中。
它的选项是逗号分隔的 \file{bib} 文件列表。
文件名中的 \file{.bib} 扩展名可以省略。
也可以通过多次调用该命令来添加更多文件。
该命令已过时,请考虑使用 \cmd{addbibresource} 来取代。

\end{ltxsyntax}

\subsubsection{参考文献表}%\subsubsection{The Bibliography}
\label{use:bib:bib}

\begin{ltxsyntax}

\cmditem{printbibliography}[key=value, \dots]

%This command prints the bibliography. It takes one optional argument, which is a list of options given in \keyval notation. The following options are available:

该命令可以打印出参考文献。
它的可选参数是以 \keyval 形式给出的一列选项。
可用的选项如下:

\end{ltxsyntax}

\begin{optionlist*}

\valitem[bibliography/shorthands]{env}{name}

%The <high-level> layout of the bibliography and the list of shorthands is controlled by environments defined with \cmd{defbibenvironment}. This option selects an environment. The \prm{name} corresponds to the identifier used when defining the environment with \cmd{defbibenvironment}. By default, the \cmd{printbibliography} command uses the identifier \texttt{bibliography}; \cmd{printbiblist} uses \texttt{shorthands}. See also \secref{use:bib:biblist,use:bib:hdg}.

可以用 \cmd{defbibenvironment} 定义的环境来控制参考文献和shorthands列表的高层次布局。
该选项选择了一个环境。
\prm{name} 对应于用 \cmd{defbibenvironment} 定义环境时的标识符。
缺省状态下,\cmd{printbibliography} 命令使用标识符 \texttt{bibliography};
而 \cmd{printshorthands} 使用 \texttt{shorthands}。
另见 \secref{use:bib:biblist,use:bib:hdg} 节。


\valitem[bibliography/shorthands]{heading}{name}

%The bibliography and the list of shorthands typically have a chapter or section heading. This option selects the heading \prm{name}, as defined with \cmd{defbibheading}. By default, the \cmd{printbibliography} command uses the heading \texttt{bibliography}; \cmd{printbiblist} uses \texttt{shorthands}. See also \secref{use:bib:biblist,use:bib:hdg}.

参考文献和shorthand列表通常有一个章标题或者节标题。
该选项选择由 \cmd{defbibheading} 定义的标题名 \prm{name}。
缺省状态下,\cmd{printbibliography} 命令使用标题名 \texttt{bibliography};
而 \cmd{printshorthands} 使用 \texttt{shorthands}。
另见 \secref{use:bib:biblist,use:bib:hdg} 节。

\valitem{title}{text}

%This option overrides the default title provided by the heading selected with the \opt{heading} option, if supported by the heading definition. See \secref{use:bib:hdg} for details.

如果标题定义支持的话,该选项覆盖由 \opt{heading} 选项提供的缺省标题名。
详见 \secref{use:bib:hdg} 节。

\valitem{prenote}{name}

%The prenote is an arbitrary piece of text to be printed after the heading but before the list of references. This option selects the prenote \prm{name}, as defined with \cmd{defbibnote}. By default, no prenote is printed. The note is printed in the standard text font. It is not affected by \cmd{bibsetup} and \cmd{bibfont} but it may contain its own font declarations. See \secref{use:bib:nts} for details.

前注是打印在标题之后、文献列表之前的任意文本片段。
该选项选择由 \cmd{defbibnote} 所定义的前注 \prm{name}。
缺省状态下不打印任何前注。
该注记使用标准正文字体。
它不受 \cmd{bibsetup} 和 \cmd{bibfont} 的影响但可以包含自己的字体声明。
详见 \secref{use:bib:nts} 节。

\valitem{postnote}{name}

%The postnote is an arbitrary piece of text to be printed after the list of references. This option selects the postnote \prm{name}, as defined with \cmd{defbibnote}. By default, no postnote is printed. The note is printed in the standard text font. It is not affected by \cmd{bibsetup} and \cmd{bibfont} but it may contain its own font declarations. See \secref{use:bib:nts} for details.

后注是打印在参考文献列表之后的任意文本片段。
该选项选择由 \cmd{defbibnote} 所定义的后注 \prm{name}。
缺省状态下不打印任何后注。
该注记使用标准正文字体。
它不受 \cmd{bibsetup} 和 \cmd{bibfont} 的影响但可以包含自己的字体声明。
详见 \secref{use:bib:nts}。

\intitem[current section]{section}

%Print only entries cited in reference section \prm{integer}. The reference sections are numbered starting at~1. All citations given outside a \env{refsection} environment are assigned to section~0. See \secref{use:bib:sec} for details and \secref{use:use:mlt} for usage examples.

只打印在第 \prm{integer} 文节中引用的条目。
该参考文献节从~1 开始编号。
所有在 \env{refsection} 环境外给出的引用标记为第零节。
详见 \secref{use:bib:sec} 和 \secref{use:use:mlt} 节中的使用例子。

\intitem[0]{segment}

%Print only entries cited in reference segment \prm{integer}. The reference segments are numbered starting at~1. All citations given outside a \env{refsegment} environment are assigned to segment~0. See \secref{use:bib:seg} for details and \secref{use:use:mlt} for usage examples. Remember that segments within a section are numbered local to the section so the segment you request will be the nth segment in the requested (or currently active enclosing) section.

只打印在第 \prm{integer} 文献段中引用的条目。
参考文献段从~1 开始编号。
所有在 \env{refsection} 环境外给出的引用标记为第零段。
详见 \secref{use:bib:sec} 和 \secref{use:use:mlt} 节中的使用例子。
请注意,一节内部的片段是在该节中局部编号的,故而需要的片段是被查询(或者当前激活的)节的第 n 段。

\valitem{type}{entrytype}

%Print only entries whose entry type is \prm{entrytype}.

只打印类型为 \prm{entrytype} 的条目。

\valitem{nottype}{entrytype}

%Print only entries whose entry type is not \prm{entrytype}. This option may be used multiple times.

只打印类型不为 \prm{entrytype} 的条目。该选项可以使用多次。

\valitem{subtype}{subtype}

%Print only entries whose \bibfield{entrysubtype} is defined and \prm{subtype}.

只打印域 \bibfield{entrysubtype} 定义为 \prm{subtype} 的条目。

\valitem{notsubtype}{subtype}

%Print only entries whose \bibfield{entrysubtype} is undefined or not \prm{subtype}. This option may be used multiple times.

只打印域 \bibfield{entrysubtype} 没有定义或者不为 \prm{subtype} 的条目。该选项可以使用多次。

\valitem{keyword}{keyword}

%Print only entries whose \bibfield{keywords} field includes \prm{keyword}. This option may be used multiple times.

只打印域 \bibfield{keywords} 包括 \prm{keyword} 的条目。该选项可以使用多次。

\valitem{notkeyword}{keyword}

%Print only entries whose \bibfield{keywords} field does not include \prm{keyword}. This option may be used multiple times.

只打印域 \bibfield{keywords} 不包括 \prm{keyword} 的条目。该选项可以使用多次。

\valitem{category}{category}

%Print only entries assigned to category \prm{category}. This option may be used multiple times.

只打印属于 \prm{category} 类型的条目。该选项可以使用多次。

\valitem{notcategory}{category}

%Print only entries not assigned to category \prm{category}. This option may be used multiple times.

只打印不属于 \prm{category} 类型的条目。该选项可以使用多次。

\valitem{filter}{name}

%Filter the entries with filter \prm{name}, as defined with \cmd{defbibfilter}. See \secref{use:bib:flt} for details.

使用由 \cmd{defbibfilter} 定义的 filter \prm{name} 来过滤条目。详见 \secref{use:bib:flt} 节。

\valitem{check}{name}

%Filter the entries with check \prm{name}, as defined with \cmd{defbibcheck}. See \secref{use:bib:flt} for details.

使用由 \cmd{defbibcheck} 定义的 check \prm{name} 来过滤条目。详见  \secref{use:bib:flt} 节。

\valitem{resetnumbers}{true,false,number}

%This option applies to numerical citation\slash bibliography styles only and requires that the \opt{defernumbers} option from \secref{use:opt:pre:gen} be enabled globally. If enabled, it will reset the numerical labels assigned to the entries in the respective bibliography, \ie the numbering will restart at~1. You can also pass a number to this option, for example: \texttt{resetnumbers=10} to reset numbering to the specified number to aid numbering continuity across documents. Use this option with care as \biblatex can not guarantee unique labels globally if they are reset manually.

该选项只用于顺序数字编码制的标注/著录样式,
并且要求 \secref{use:opt:pre:gen} 中的 \opt{defernumbers} 选项全局启用。
如果启用的话,它将重新设置分配给相应文献中条目的数值标签,即,编号会重新从 1 开始。
此外还可以传递数值给该选项以重置编号为给定的数值,例如 \texttt{resetnumbers=10},
这样可以改进整个文档中编号的连续性。
请小心使用本选项,因为在手动重新设置下,\biblatex 不能保证标签是全局唯一的。

\boolitem{omitnumbers}

%This option applies to numerical citation\slash bibliography styles only and requires that the \opt{defernumbers} option from \secref{use:opt:pre:gen} be enabled globally. If enabled, \biblatex will not assign a numerical label to the entries in the respective bibliography. This is useful when mixing a numerical subbibliography with one or more subbibliographies using a different scheme (\eg author-title or author-year).

该选项只用于顺序数字编码制的标注/著录样式,
并且要求 \secref{use:opt:pre:gen} 中的 \opt{defernumbers} 选项全局启用。
如果启用的话,\biblatex 不会为相应文献中的条目分配数值标签。
当数值型子文献和其它不同格式(例如作者-标题或者作者-年份)的子文献相混合时,这是很有用的。

\end{optionlist*}

\begin{ltxsyntax}

\cmditem{bibbysection}[key=value, \dots]

%This command automatically loops over all reference sections. This is equivalent to giving one \cmd{printbibliography} command for every section but has the additional benefit of automatically skipping sections without references. Note that \cmd{bibbysection} starts looking for references in section \texttt{1}. It will ignore references given outside of \env{refsection} environments since they are assigned to section~0. See \secref{use:use:mlt} for usage examples. The options are a subset of those supported by \cmd{printbibliography}. Valid options are \opt{env}, \opt{heading}, \opt{prenote}, \opt{postnote}. The current bibliography context sorting scheme is used for all sections (see \secref{use:bib:context}).

该命令会自动遍历所有的参考文献节。
这等价于为每一节给出一个 \cmd{printbibliography} 命令,
不过会有额外好处:自动跳过不含参考文献的节。
请注意,\cmd{bibbysection} 一开始寻找第 \texttt{1} 节中的文献。
它会忽略 \env{refsection} 外给出的文献,因为它们被分配给第零节。
使用例子请参考 \secref{use:use:mlt} 节。
选项可以是由 \cmd{printbibliography} 支持的一个子集。
有效选项是 \opt{env}、\opt{heading}、\opt{prenote} 和 \opt{postnote}。
当前文献内容排序格式会应用在所有的节中(见 \secref{use:bib:context} 节)。

\cmditem{bibbysegment}[key=value, \dots]

%This command automatically loops over all reference segments. This is equivalent to giving one \cmd{printbibliography} command for every segment in the current \env{refsection} but has the additional benefit of automatically skipping segments without references. Note that \cmd{bibbysegment} starts looking for references in segment \texttt{1}. It will ignore references given outside of \env{refsegment} environments since they are assigned to segment~0. See \secref{use:use:mlt} for usage examples. The options are a subset of those supported by \cmd{printbibliography}. Valid options are \opt{env}, \opt{heading}, \opt{prenote}, \opt{postnote}. The current bibliography context sorting scheme is used for all segments (see \secref{use:bib:context}).

该命令会自动遍历所有的参考文献段。
这等价于为当前 \env{refsection} 的每一段给出一个 \cmd{printbibliography} 命令,不过会有额外好处:自动跳过不含参考文献的片段。
请注意,\cmd{bibbysection} 一开始寻找第 \texttt{1} 段中的文献。
它会忽略 \env{refsection} 外给出的文献,因为它们被分配给第 0 段。
使用例子请参考 \secref{use:use:mlt}。
选项可以是由 \cmd{printbibliography} 支持的一个子集。
有效选项是 \opt{env}、\opt{heading}、\opt{prenote} 和 \opt{postnote}。
当前文献内容排序格式会用于所有的段中(见 \secref{use:bib:context} 节)。

\cmditem{bibbycategory}[key=value, \dots]

%This command loops over all bibliography categories. This is equivalent to giving one \cmd{printbibliography} command for every category but has the additional benefit of automatically skipping empty categories. The categories are processed in the order in which they were declared. See \secref{use:use:mlt} for usage examples. The options are a subset of those supported by \cmd{printbibliography}. Valid options are \opt{env}, \opt{prenote}, \opt{postnote}, \opt{section}. Note that \opt{heading} is not available with this command. The name of the current category is automatically used as the heading name. This is equivalent to passing \texttt{heading=\prm{category}} to \cmd{printbibliography} and implies that there must be a matching heading definition for every category. The current bibliography context sorting scheme is used for all categories (see \secref{use:bib:context}).

该命令遍历所有的文献类型。
这等价于为每一类型给出一个 \cmd{printbibliography} 命令,
不过会有额外好处:自动跳过空类型。
类型按照声明的顺序处理。
示例见 \secref{use:use:mlt} 节。
选项可以是由 \cmd{printbibliography} 支持的一个子集。
有效选项是 \opt{env}、\opt{heading}、\opt{prenote} 和 \opt{postnote}。
请注意,\opt{heading} 对于该命令是无效的。
当前类型的名字会自动作为标题名。
这等价于传递 \texttt{heading=\prm{category}} 给 \cmd{printbibliography},
并且意味着对于每一类型都必须有一个匹配的标题定义。
当前文献内容排序格式会用于所有的类型中(见 \secref{use:bib:context} 节)。

\cmditem{printbibheading}[key=value, \dots]

%This command prints a bibliography heading defined with \cmd{defbibheading}. It takes one optional argument, which is a list of options given in \keyval notation. The options are a small subset of those supported by \cmd{printbibliography}. Valid options are \opt{heading} and \opt{title}. By default, this command uses the heading \texttt{bibliography}. See \secref{use:bib:hdg} for details. Also see \secref{use:use:mlt,use:use:div} for usage examples.

该命令打印出由 \cmd{defbibheading} 定义的参考文献标题。
它有一个可选项,是用 \keyval 记号给出的选项列表。
选项是 \cmd{printbibliography} 支持的一个小子集。
有效选项是 \opt{heading} 和 \opt{title}。
缺省情况下,该命令使用标题 \texttt{bibliography}。
详见 \secref{use:bib:hdg} 节。实例也可见 \secref{use:use:mlt,use:use:div} 节。

\end{ltxsyntax}
%
%To print a bibliography with a different sorting scheme than the global sorting scheme, use the bibliography context switching commands from \secref{use:bib:context}.
如果想要在参考文献中使用非全局排序格式的另外一种排序格式,
使用 \secref{use:bib:context} 节提供的文献内容切换命令。

%\subsubsection{缩略表 The List of Shorthands}
%\label{use:bib:los}
%
%\BibTeXOnlyMark This section applies only to \bibtex. When using \biber, the list of shorthands is just a special case of a bibliography list. See \secref{use:bib:biblist}.\footnote{本节需要重点关注一下,之前没有搞清楚}
%
%If any entry includes a \bibfield{shorthand} field, \sty{biblatex} automatically builds a list of shorthands which may be printed in addition to the regular bibliography. The following command prints the list of shorthands.
%
%\begin{ltxsyntax}
%
%\cmditem{printshorthands}[key=value, \dots]
%
%This command prints the list of shorthands. It takes one optional argument, which is a list of options given in \keyval notation. Valid options are all options supported by \cmd{printbibliography} (\secref{use:bib:bib}) except \opt{prefixnumbers}, \opt{resetnumbers}, and \opt{omitnumbers}. If there are any \env{refsection} environments in the document, the list of shorthands will be local to these environments; see \secref{use:bib:sec} for details. By default, this command uses the heading \texttt{shorthands}. See \secref{use:bib:hdg} for details.
%
%The \opt{sorting} option differs from \cmd{printbibliography} in that if omitted, the default is to sort by shorthand.
%
%\end{ltxsyntax}




\subsubsection{参考文献列表} %\subsubsection{Bibliography Lists}
\label{use:bib:biblist}

%\biblatex can, in addition to printing normal bibliographies, also print arbitrary lists of information derived from the bibliography data such as a list of shorthand abbreviations for particular entries or a list of abbreviations of journal titles.

\biblatex 除了可以打印常规参考文献之外,还能根据文献数据打印任意文献信息列表,
例如,与特定条目或者期刊标题缩写有关的速记缩写列表。

%A bibliography list differs from a normal bibliography in that the same bibliography driver is used to print all entries rather than a specific driver being used for each entry depending on the entry type.

文献列表与常规参考文献不同的是,
使用同一文献驱动打印所有条目,
而不是根据条目类型使用特定于条目的驱动。

\begin{ltxsyntax}

\cmditem{printbiblist}[key=value, \dots]{$<$biblistname$>$}

%This command prints a bibliography list. It takes an optional argument, which is a list of options given in \keyval notation. Valid options are all options supported by \cmd{printbibliography} (\secref{use:bib:bib}) except \opt{resetnumbers} and \opt{omitnumbers}. If there are any \env{refsection} environments in the document, the bibliography list will be local to these environments; see \secref{use:bib:sec} for details. By default, this command uses the heading \texttt{biblist}. See \secref{use:bib:hdg} for details.

该命令用于打印文献列表。
其可选项是 \keyval 形式的一列选项。
除了 \opt{resetnumbers} 和 \opt{omitnumbers},
\cmd{printbibliography} 命令(见 \secref{use:bib:bib} 节)支持的其它选项在这里都是有效的。
如果文档中有任何 \env{refsection} 环境,
那么文献列表只针对于这些环境,详见 \secref{use:bib:sec} 节。
默认情况下该命令使用标题 \texttt{biblist},详见 \secref{use:bib:hdg} 节。

%The \prm{biblistname} is a mandatory argument which names the bibliography list. This name is used to identify:
必选项 \prm{biblistname} 是文献列表的标题,用于确定如下项目:
\begin{itemize}
\item %The default bibliography driver used to print the list entries
用于打印列表条目的默认文献驱动。
\item %A default filter declared with \cmd{DeclareBiblistFilter} (see \secref{aut:ctm:bibfilt}) used to filter the entries returned from \biber
使用 \cmd{DeclareBiblistFilter} 声明的默认filter(见 \secref{aut:ctm:bibfilt} 节),
用于过滤 \biber 返回的条目。
\item %A default check declared with \cmd{defbibcheck} (see \secref{use:bib:flt}) used to post-process the list entries
使用 \cmd{defbibcheck} 命令声明的默认check(见 \secref{use:bib:flt} 节),
用于后置处理列表条目。
\item %The default bib environment to use
默认使用的bib 环境。
\item %The default sorting template to use
默认使用的排序格式名称。
\end{itemize}

%In terms of sorting the list, the default is to sort use the sorting scheme named after the bibliography list (if it exists) and only then to fall back to the current context sorting scheme is this is not defined (see \secref{use:bib:context}).

在列表的排序方面,默认使用与该文献列表同名的排序格式(如果存在的话)。
只有当未定义时才会切换到备选的当前内容排序格式(见 \secref{use:bib:context} 节)。

%The most common bibliography list is a list of shorthand abbreviations for certain entries and so this has a convenience alias \cmd{printshorthands[\dots]} for backwards compatibility which is defined as:

最常用的文献列表是关于某些条目的速记列表,
出于向后兼容性专门有一个别名 \cmd{printshorthands[\dots]},定义如下:

\begin{lstlisting}[style=latex]{}
\printbiblist[...]{shorthand}
\end{lstlisting}

%\biblatex provides automatic support for data source fields in the default data model marked as <Label fields> (See \secref{bib:fld:dat}). Such fields automatically have defined for them:

\biblatex 自动支持默认数据模型中标记为“Label fields”的数据域(见 \secref{bib:fld:dat} 节)。
这些域已经自动为其定义了如下项目:

\begin{itemize}
\item %A default bib environment (See \secref{use:bib:hdg})
默认的 bib 环境(见 \secref{use:bib:hdg} 节)。
\item %A bibliography list filter (See \secref{aut:ctm:bibfilt})
文献列表filter(见 \secref{aut:ctm:bibfilt} 节)
\item %Some supporting formats and lengths (See \secref{aut:fmt:ilc} and \secref{aut:fmt:ich})
一些支持的格式和长度(见 \secref{aut:fmt:ilc} 和 \secref{aut:fmt:ich} 节)。
\end{itemize}
%
%Therefore only a minimal setup is required to print bibliography lists with such fields. For example, to print a list of journal title abbreviations, you can minimally put this in your preamble:
因此,打印带有这些域的文献列表只需要很少的设置。
例如,想要打印出期刊标题缩写列表,
只需要将如下一小段代码放在导言区中:

\begin{ltxexample}
\DeclareBibliographyDriver{shortjournal}{%
  \printfield{journaltitle}}
\end{ltxexample}
%
%Then you can put this in your document where you want to print the list:
然后在正文中想要打印列表的地方使用如下代码:

\begin{ltxexample}
\printbiblist[title={Journal Shorthands}]{shortjournal}
\end{ltxexample}
%
%Since \bibfield{shortjournal} is defined in the default data model as a <Label field>, this example:
由于默认数据模型将 \bibfield{shortjournal} 定义为“标签域”,
因此在这个例子中:
\begin{itemize}
\item %Uses the automatically created <shortjournal> bib environment
使用自动创建的“shortjournal”bib环境。
\item %Uses the automatically created <shortjournal> bibliography list filter to return only entries with a \bibfield{shortjournal} field in the \file{.bbl}
使用自动创建的“shortjournal”文献列表filter,
返回 \file{.bbl} 文件中只带有 \bibfield{shortjournal} 域的条目。
\item %Uses the defined <shortjournal> bibliography driver to print the entries
使用定义的“shortjournal”文献驱动来打印条目。
\item %Uses the default <biblist> heading but overrides the title with <Journal Shorthands>
使用默认的“biblist”标题,但是这里用“Journal Shorthands”来代替。
\item %Uses the current bibliography context sorting scheme if no scheme exists with the name \bibfield{shortjournal}
如果没有名为 \bibfield{shortjournal} 的格式,那么使用当前文献内容排序格式。
\end{itemize}
%
%Often, you will want to sort on the label field of the list and since a sorting scheme is automatically picked up if it is named after the list, in this case you could simply do:
很多情况下想要根据列表中标签域进行排序。
由于根据列表名可以自动获取排序格式,因此此时可以简单地使用如下代码:

\begin{ltxexample}
\DeclareSortingTemplate{shortjournal}{
  \sort{
        \field{shortjournal}
  }
}
\end{ltxexample}

%Naturally all defaults can be overridden by options to \cmd{printbiblist} and definitions of the environments, filters etc. and in this way arbitrary types of bibliography lists can be printed containing a variety of information from the bibliography data.

自然地,\cmd{printbiblist} 命令的选项以及环境、filters等的定义可以覆盖所有的默认设置。
因此通过这种方法可以从文献数据中打印任意类型的文献列表,并且包含各式信息。
\end{ltxsyntax}

%Bibliography lists are often used to print lists of various kinds of shorthands and this can result in duplicate entries if more than one bibliography entry has the same shorthand. For example, several journal articles in the same journal would result in duplicate entries in a list of journal shorthands. You can use the fact that such lists automatically pick up a \cmd{bibcheck} with the same name as the list to define a check to remove duplicates. If you are defining a list to print all of the journal shorthands using the \bibfield{shortjournal} field, you could define a \cmd{bibcheck} like this:

文献列表通常用于打印各类shorthand列表。
如果多个条目有相同的shorthand就会导致重复的条目。
例如,如果有几篇论文在同一期刊上,那么期刊缩写列表中就会出现重复条目。
不过,这样的列表会自动获取与列表同名的 \cmd{bibcheck},进而定义相应的check来删除重复项目。
如果使用 \bibfield{shortjournal} 域来定义打印所有期刊缩写的列表,
那么需要定义如下的 \cmd{bibcheck}:

\begin{ltxexample}
\defbibcheck{shortjournal}{%
   \iffieldundef{shortjournal}
     {\skipentry}
     {\iffieldundef{journaltitle}
       {\skipentry}
       {\ifcsdef{\strfield{shortjournal}=\strfield{journaltitle}}
         {\skipentry}
         {\savefieldcs{journaltitle}{\strfield{shortjournal}=\strfield{journaltitle}}}}}}
\end{ltxexample}

\subsubsection{参考文献分节}% \subsubsection{Bibliography Sections}
\label{use:bib:sec}

%The \env{refsection} environment is used in the document body to mark a reference section. This environment is useful if you want separate, independent bibliographies and bibliography lists in each chapter, section, or any other part of a document. Within a reference section, all cited works are assigned labels which are local to the environment. Technically, reference sections are completely independent from document divisions such as \cmd{chapter} and \cmd{section} even though they will most likely be used per chapter or section. See the \opt{refsection} package option in \secref{use:opt:pre:gen} for a way to automate this. Also see \secref{use:use:mlt} for usage examples.

在文档中,\env{refsection} 环境用于标记参考文献分节。
该环境主要用于在文档的每一章、节或其它部分中实现各自独立的参考文献和shorthand列表。
在一个文献分节内部,所有引用文献分配的标签都局部在该环境中。
技术上,尽管文献分节通常在每一章或每一节中使用,
但它们与 \cmd{chapter} 和 \cmd{section} 等文档划分是完全独立的。
关于自动实现这一功能请参考 \secref{use:opt:pre:gen} 中的 \opt{refsection} 宏包选项。
使用例子也可以参见 \secref{use:use:mlt}。

\begin{ltxsyntax}

\envitem{refsection}[resource, \dots]

%The optional argument is a comma"=separated list of resources specific to the reference section. If the argument is omitted, the reference section will use the default resource list, as specified with \cmd{addbibresource} in the preamble. If the argument is provided, it replaces the default resource list. Global resources specified with \cmd{addglobalbib} are always considered. \env{refsection} environments may not be nested, but you may use \env{refsegment} environments within a \env{refsection} to subdivide it into segments. Use the \opt{section} option of \cmd{printbibliography} to select a section when printing the bibliography, and the corresponding option of \cmd{printbiblist} when printing bibliography lists. Bibliography sections are numbered starting at~\texttt{1}. The number of the current section is also written to the transcript file. All citations given outside a \env{refsection} environment are assigned to section~0. If \cmd{printbibliography} is used within a \env{refsection}, it will automatically select the current section. The \opt{section} option is not required in this case. This also applies to \cmd{printbiblist}.

可选项是特定于该参考文献分节的逗号分隔资源列表。
如果省略了该选项,参考文献节会使用缺省的数据源列表,由导言区的 \cmd{addbibresource} 指定。
如果提供了该选项,它会替代缺省的资源列表。
不过,由 \cmd{addglobalbib} 指定的全局文献资源总是包含在内的。
\env{refsection} 环境不可以相互嵌套,
但是可以在 \env{refsection} 环境内使用 \env{refsegment} 环境来进一步分段。
当打印参考文献时,使用 \cmd{printbibliography} 的 \opt{section} 选项来选择节;
同样地当打印文献列表时使用 \cmd{printbiblist} 对应的选项。
参考文献分节从~\texttt{1} 开始编号。当前节的编号也被写入副本文件中。
所有在 \env{refsection} 环境外给出的引用都归到第~0 节中。
如果在 \env{refsection} 内部使用 \cmd{printbibliography} 环境,它会自动选择当前节。
此时不需要 \opt{section} 选项。这也适用于 \cmd{printbiblist}。

\cmditem{newrefsection}[resource, \dots]

%This command is similar to the \env{refsection} environment except that it is a stand"=alone command rather than an environment. It automatically ends the previous reference section (if any) and immediately starts a new one. Note that the reference section started by the last \cmd{newrefsection} command in the document will extend to the very end of the document. Use \cmd{endrefsection} if you want to terminate it earlier.

该命令类似于 \env{refsection} 环境,不同之处在于它是单独命令而不是一个环境。
它会自动结束之前的文献分节(如果有的话)并立即开始新的一节。
请注意,文档中由最后一个 \cmd{newrefsection} 开始的文献节会延续到文档的最后。
如果你想提前终止的话可以使用 \cmd{endrefsection}。

\end{ltxsyntax}

\subsubsection{参考文献分段}%\subsubsection{Bibliography Segments}
\label{use:bib:seg}

%The \env{refsegment} environment is used in the document body to mark a reference segment. This environment is useful if you want one global bibliography which is subdivided by chapter, section, or any other part of the document. Technically, reference segments are completely independent from document divisions such as \cmd{chapter} and \cmd{section} even though they will typically be used per chapter or section. See the \opt{refsegment} package option in \secref{use:opt:pre:gen} for a way to automate this. Also see \secref{use:use:mlt} for usage examples.

在文档中,\env{refsegment} 环境用来标记参考文献片段。
该环境用于实现在文档的每一章、节或其它部分中将全局的参考文献分成片段。
技术上,尽管文献分段通常在每一章或每一节中使用,
但它们与 \cmd{chapter} 和 \cmd{section} 等文档划分是完全独立的。
关于自动实现这一功能请参考 \secref{use:opt:pre:gen} 中的 \opt{refsegment} 宏包选项。
使用例子也可以参见 \secref{use:use:mlt}。

\begin{ltxsyntax}

\envitem{refsegment}

%The difference between a \env{refsection} and a \env{refsegment} environment is that the former creates labels which are local to the environment whereas the latter provides a target for the \opt{segment} filter of \cmd{printbibliography} without affecting the labels. They will be unique across the entire document. \env{refsegment} environments may not be nested, but you may use them in conjunction with \env{refsection} to subdivide a reference section into segments. In this case, the segments are local to the enclosing \env{refsection} environment. Use the \env{segment} option of \cmd{printbibliography} to select a segment when printing the bibliography. Within a section, the reference segments are numbered starting at~\texttt{1} and the number of the current segment will be written to the transcript file. All citations given outside a \env{refsegment} environment are assigned to segment~0. In contrast to the \env{refsection} environment, the current segment is not selected automatically if \cmd{printbibliography} is used within a \env{refsegment} environment.

\env{regsection} 与 \env{refsegment} 环境的不同之处在于,
前者创建局部于该环境的标签而后者仅为 \cmd{printbibliography} 命令的 \opt{segment} filter 提供目标而不影响标签。
在整个文档中它们是唯一确定的。
\env{refsegment} 环境不可以嵌套,
但是你可以将其与 \env{refsection} 环境结合使用来将文献节细分为段。
此时,这些文献分段是局部于被包含的 \env{refsection} 环境的。
当打印参考文献时,使用 \cmd{printbibliography} 的 \opt{segment} 选项来选择文献分段。
在一节内,文献段从~\texttt{1} 开始编号,并且当前段的编号会被写入到一个副本文件中。
所有在 \env{refsegment} 环境之外的引用都归到第~0 段。
与 \env{refsection} 环境相反,
当 \cmd{printbibliography} 在一个 \env{refsegment} 环境内使用时,当前文献分段并不自动选定。

\csitem{newrefsegment}

%This command is similar to the \env{refsegment} environment except that it is a stand"=alone command rather than an environment. It automatically ends the previous reference segment (if any) and immediately starts a new one. Note that the reference segment started by the last \cmd{newrefsegment} command will extend to the end of the document. Use \cmd{endrefsegment} if you want to terminate it earlier.

该命令类似于 \env{refsegment} 环境,不同之处在于它是单独命令而不是一个环境。
它会自动结束之前的文献分段(如果有的话)并立即开始新的一段。
请注意,由最后一个 \cmd{newrefsegment} 开始的文献分段会延续到文档结束。
如果你想提前终止的话可以使用 \cmd{endrefsegment}。

\end{ltxsyntax}

\subsubsection{参考文献分类}%\subsubsection{Bibliography Categories}
\label{use:bib:cat}

%Bibliography categories allow you to split the bibliography into multiple parts dedicated to different topics or different types of references, for example primary and secondary sources. See \secref{use:use:div} for usage examples.

参考文献分类允许将参考文献针对不同主题或不同文献类型分成若干部分,例如分成主要文献和次要文献。
使用例子参见 \secref{use:use:div} 节。

\begin{ltxsyntax}

\cmditem{DeclareBibliographyCategory}{category}

%Declares a new \prm{category}, to be used in conjunction with \cmd{addtocategory} and the \opt{category} and \opt{notcategory} filters of \cmd{printbibliography}. This command is used in the document preamble.

声明一个新的 \prm{category},
可以和 \cmd{addtocategory} 以及 \cmd{printbibliography} 的 \opt{category}、\opt{notcategory} filter结合使用。
该命令在导言区中使用。

\cmditem{addtocategory}{category}{key}

%Assigns a \prm{key} to a \prm{category}, to be used in conjunction with the \opt{category} and \opt{notcategory} filters of \cmd{printbibliography}. This command may be used in the preamble and in the document body. The \prm{key} may be a single entry key or a comma"=separated list of keys. The assignment is global.

将 \prm{key} 关键字分配给 \prm{category} 类,
可以和 \cmd{addtocategory} 以及 \cmd{printbibliography} 的 \opt{category}、\opt{notcategory} filter结合使用。
该命令可以在导言区和正文中使用。
\prm{key} 可以是一个单独条目关键字或者逗号分隔的键值列表。该分配是全局的。

\end{ltxsyntax}

\subsubsection{参考文献标题与环境}%\subsubsection{Bibliography Headings and Environments}
\label{use:bib:hdg}

\begin{ltxsyntax}

\cmditem{defbibenvironment}{name}{begin code}{end code}{item code}

%This command defines bibliography environments. The \prm{name} is an identifier passed to the \opt{env} option of \cmd{printbibliography} and \cmd{printbiblist} when selecting the environment. The \prm{begin code} is \latex code to be executed at the beginning of the environment; the \prm{end code} is executed at the end of the environment; the \prm{item code} is code to be executed at the beginning of each entry in the bibliography or a bibliography list. Here is an example of a definition based on the standard \latex \env{list} environment:

该命令定义参考文献环境。
其中 \prm{name} 是标识符,
当选择该环境时会传递给 \cmd{printbibliography} 和 \cmd{printshorthands} 的 \opt{env} 选项。
\prm{begin code} 是该环境开始时执行的 \LaTeX 代码;
而 \prm{end code} 在该环境结束时执行;
\prm{item code} 是在参考文献或者shorthand列表的每一条目开始时执行的代码。
如下是基于 \LaTeX 标准 \env{list} 环境定义的例子。

\begin{ltxexample}
\defbibenvironment{bibliography}
  {\list{}
     {\setlength{\leftmargin}{\bibhang}%
      \setlength{\itemindent}{-\leftmargin}%
      \setlength{\itemsep}{\bibitemsep}%
      \setlength{\parsep}{\bibparsep}}}
  {\endlist}
  {\item}
\end{ltxexample}
%
%As seen in the above example, usage of \cmd{defbibenvironment} is roughly similar to \cmd{newenvironment} except that there is an additional mandatory argument for the \prm{item code}.
如上述例子所示,\cmd{defbibenvironment} 的使用大体类似于 \cmd{newenvironment},
不同之处在于有一个额外的必选项 \prm{item code}。

\cmditem{defbibheading}{name}[title]{code}

%This command defines bibliography headings. The \prm{name} is an identifier to be passed to the \opt{heading} option of \cmd{printbibliography} or \cmd{printbibheading} and \cmd{printbiblist} when selecting the heading. The \prm{code} should be \latex code generating a fully"=fledged heading, including page headers and an entry in the table of contents, if desired. If \cmd{printbibliography} or \cmd{printbiblist} are invoked with a \opt{title} option, the title will be passed to the heading definition as |#1|. If not, the default title specified by the optional \prm{title} argument is passed as |#1| instead. The \prm{title} argument will typically be \cmd{bibname}, \cmd{refname}, or \cmd{biblistname} (see \secref{aut:lng:key:bhd}). This command is often needed after changes to document headers in the preamble. Here is an example of a simple heading definition:

该命令定义参考文献标题。
其中 \prm{name} 是标识符,在\cmd{printbibliography} 和 \cmd{printshorthands} 命令中会被传递给 \opt{heading} 选项。
\prm{code} 是能生成完整标题的 \LaTeX 代码,包括页眉和目录中的条目(如果必要的话)。
如果 \cmd{printbibliography} 或 \cmd{printshorthands} 带有 \opt{title} 选项,
那么 \opt{title} 将作为 |#1| 传递给标题定义;
否则由可选的 \prm{title} 确定的标题将作为 |#1| 传递给标题定义。
\prm{title} 选项通常是 \cmd{bibname}、\cmd{refname} 或者 \cmd{biblistname}
(见 \secref{aut:lng:key:bhd} 节)。
如果在导言区中改变文档标题时,那么之后通常需要该命令。
如下是一个简单标题定义的例子:

\begin{ltxexample}
\defbibheading{bibliography}[\bibname]{%
  \chapter*{#1}%
  \markboth{#1}{#1}}
\end{ltxexample}

\end{ltxsyntax}

%The following headings, which are intended for use with \cmd{printbibliography} and \cmd{printbibheading}, are predefined:

以下预定义的标题与 \cmd{printbibliography} 和 \cmd{printbibheading} 结合使用:

\begin{valuelist*}

\item[bibliography]
%This is the default heading used by \cmd{printbibliography} if the \opt{heading} option is not given. Its default definition depends on the document class. If the class provides a \cmd{chapter} command, the heading is similar to the bibliography heading of the standard \latex \texttt{book} class, \ie it uses \cmd{chapter*} to create an unnumbered chapter heading which is not included in the table of contents. If there is no \cmd{chapter} command, it is similar to the bibliography heading of the standard \latex \texttt{article} class, \ie it uses \cmd{section*} to create an unnumbered section heading which is not included in the table of contents. The string used in the heading also depends on the document class. With \texttt{book}-like classes the localisation string \texttt{bibliography} is used, with other classes it is \texttt{references} (see \secref{aut:lng:key}). See also \secref{use:cav:scr, use:cav:mem} for class-specific hints.
如果没有给出 \opt{heading} 选项,那么这是 \cmd{printbibliography} 使用的默认标题。
缺省定义取决于文档类。
如果文类提供 \cmd{chapter} 命令,那么该标题就类似于标准 \LaTeX 的 \texttt{book} 文类的参考文献标题,
即使用 \cmd{chapter*} 来创建不带编号的章,并且不包含在目录中。
如果没有 \cmd{chapter} 命令,那么它将类似于标准 \LaTeX 的 \texttt{article} 文类的参考文献标题,
即使用 \cmd{section*} 来创建不带编号的节,并且不包含在目录中。
标题中使用的字符串也取决于文档类。
\texttt{book} 文档类使用本地化字符串 \texttt{bibliography},
在其它文档类中则是 \texttt{references}(见 \secref{aut:lng:key} 节)。
关于文档类的提示也可以见 \secref{use:cav:scr, use:cav:mem} 节。

\item[subbibliography]
%Similar to \texttt{bibliography} but one sectioning level lower. This heading definition uses \cmd{section*} instead of \cmd{chapter*} with a \texttt{book}-like class and \cmd{subsection*} instead of \cmd{section*} otherwise.
类似于 \texttt{bibliography},但是标题格式低一级。
即,在 \texttt{book} 文档类中使用 \cmd{section*} 而不是 \cmd{chapter*},
其它情况使用 \cmd{subsection*} 而不是 \cmd{section*}。

\item[bibintoc]
%Similar to \texttt{bibliography} above but adds an entry to the table of contents.
类似于 \texttt{bibliography} 但是在目录中添加条目。

\item[subbibintoc]
%Similar to \texttt{subbibliography} above but adds an entry to the table of contents.
类似于 \texttt{subbibliography} 但是在目录中添加条目。

\item[bibnumbered]
%Similar to \texttt{bibliography} above but uses \cmd{chapter} or \cmd{section} to create a numbered heading which is also added to the table of contents.
类似于 \texttt{bibliography} 但是使用 \cmd{chapter} 或 \cmd{section} 来创建带编号的条目,
同时也添加到目录中。

\item[subbibnumbered]
%Similar to \texttt{subbibliography} above but uses \cmd{section} or \cmd{subsection} to create a numbered heading which is also added to the table of contents.
类似于 \texttt{bibliography} 但是使用 \cmd{section} 或 \cmd{subsection} 来创建带编号的条目,
同时也添加到目录中。

\item[none]
%A blank heading definition. Use this to suppress the heading.
空白的标题定义,用来取消标题。

\end{valuelist*}

%The following headings intended for use with \cmd{printbiblist} are predefined:
以下预定义的标题与 \cmd{printshorthands} 结合使用:

\begin{valuelist*}

\item[biblist]
%This is the default heading used by \cmd{printbiblist} if the \opt{heading} option is not given. It is similar to \texttt{bibliography} above except that it uses the localisation string \texttt{shorthands} instead of \texttt{bibliography} or \texttt{references} (see \secref{aut:lng:key}). See also \secref{use:cav:scr, use:cav:mem} for class-specific hints.
如果没有给出 \opt{heading} 选项,那么这是 \cmd{printbiblist} 使用的缺省标题。
类似于上面的 \texttt{bibliography},
不过是使用本地化字符串 \texttt{shorthands} 而不是 \texttt{bibliography} 或 \texttt{references}
(见 \secref{aut:lng:key} 节)。
关于文档类的提示另见 \secref{use:cav:scr, use:cav:mem} 节。

\item[biblistintoc]
%Similar to \texttt{biblist} above but adds an entry to the table of contents.
类似于 \texttt{shorthands} 但是在目录中添加条目。

\item[biblistnumbered]
%Similar to \texttt{biblist} above but uses \cmd{chapter} or \cmd{section} to create a numbered heading which is also added to the table of contents.
类似于上面的 \texttt{biblist} 但是使用 \cmd{chapter} 或 \cmd{section} 来创建带编号的标题,
同时也添加到目录中。

\end{valuelist*}

\subsubsection{参考文献注记}%\subsubsection{Bibliography Notes}
\label{use:bib:nts}

\begin{ltxsyntax}

\cmditem{defbibnote}{name}{text}

%Defines the bibliography note \prm{name}, to be used via the \opt{prenote} and \opt{postnote} options of \cmd{printbibliography} and \cmd{printbiblist}. The \prm{text} may be any arbitrary piece of text, possibly spanning several paragraphs and containing font declarations. Also see \secref{use:cav:act}.

定义名为 \prm{name} 的参考文献注记 ,
通过 \cmd{printbibliography} 和 \cmd{printbiblist} 的 \opt{prenote} 和 \opt{postnote} 选项使用。
\prm{text} 可以是任意文本片段,通常包含若干段落和字体声明。
另见 \secref{use:cav:act} 节。

\end{ltxsyntax}

\subsubsection{参考文献筛选和过滤} %\subsubsection{Bibliography Filters and Checks}
\label{use:bib:flt}

\begin{ltxsyntax}

\cmditem{defbibfilter}{name}{expression}

%Defines the custom bibliography filter \prm{name}, to be used via the \opt{filter} option of \cmd{printbibliography}. The \prm{expression} is a complex test based on the logical operators \texttt{and}, \texttt{or}, \texttt{not}, the group separator \texttt{(...)}, and the following atomic tests:

定义一个可定制的文献过滤(筛选) \prm{name},
可以通过 \cmd{printbibliography} 的 \opt{filter} 选项使用。
\prm{expression} 是复合测试,
基于逻辑运算符 \texttt{and}、\texttt{or}、\texttt{not},组运算符 \texttt{(...)},以及以下的基本测试:

\end{ltxsyntax}

\begin{optionlist*}

\valitem{segment}{integer}

%Matches all entries cited in reference segment \prm{integer}.

匹配所有在参考文献分段 \prm{integer} 中引用的条目。

\valitem{type}{entrytype}

%Matches all entries whose entry type is \prm{entrytype}.

匹配所有类型为 \prm{entrytype} 的条目。

\valitem{subtype}{subtype}

%Matches all entries whose \bibfield{entrysubtype} is \prm{subtype}.

匹配所有 \bibfield{entrysubtype} 域为 \prm{subtype} 的条目。

\valitem{keyword}{keyword}

%Matches all entries whose \bibfield{keywords} field includes \prm{keyword}. If the \prm{keyword} contains spaces, it needs to be wrapped in braces.

匹配所有 \bibfield{keywords} 域包含 \prm{keyword} 的条目。
如果 \prm{keyword} 包含空格,那么需要用括号括起来。

\valitem{category}{category}

%Matches all entries assigned to \prm{category} with \cmd{addtocategory}.

匹配所有由 \cmd{addtocategory} 归入 \prm{category} 类的条目。

\end{optionlist*}

%Here is an example of a filter expression:
如下是一个过滤表达式的例子:

\begin{ltxexample}[style=latex,keywords={and,or,not,type,keyword}]{}
\defbibfilter{example}{%
  ( type=book or type=inbook )
  and keyword=abc
  and not keyword={x y z}
}
\end{ltxexample}
%
%This filter will match all entries whose entry type is either \bibtype{book} or \bibtype{inbook} and whose \bibfield{keywords} field includes the keyword <\texttt{abc}> but not <\texttt{x y z}>. As seen in the above example, all elements are separated by whitespace (spaces, tabs, or line endings). There is no spacing around the equal sign. The logical operators are evaluated with the \cmd{ifboolexpr} command from the \sty{etoolbox} package. See the \sty{etoolbox} manual for details about the syntax. The syntax of the \cmd{ifthenelse} command from the \sty{ifthen} package, which has been employed in older versions of \biblatex, is still supported. This is the same test using \sty{ifthen}-like syntax:
该 filter 匹配的条目规则是,条目类型是 \bibtype{book} 或 \bibtype{inbook},\bibfield{keywords} 域包含关键词 “\texttt{abc}”但不包含“\texttt{x y z}”。
从以上例子可以看出,所有的元素由空白分开(空格、制表符或者换行)。
等号周围没有空白。逻辑运算使用 \sty{etoolbox} 宏包的 \cmd{ifboolexpr} 执行。
关于该语法详见 \sty{etoolbox} 手册。
\biblatex 旧版本中使用的 \sty{ifthen} 宏包的 \cmd{ifthenelse} 语法这里仍然支持。
如下是相同的测试,使用 \sty{ifthen} 样式的语法:

\begin{ltxexample}[style=ifthen,morekeywords={\\type,\\keyword}]{}
\defbibfilter{example}{%
  \( \type{book} \or \type{inbook} \)
  \and \keyword{abc}
  \and \not \keyword{x y z}
}
\end{ltxexample}
%
%Note that custom filters are local to the reference section in which they are used. Use the \texttt{section} filter of \cmd{printbibliography} to select a different section. This is not possible from within a custom filter.
请注意,定制的过滤对于所在的参考文献分节是局部的。
使用 \cmd{printbibliography} 的 \texttt{section} 过滤器来选择不同的分节。
这在定制filter中是不可能的。

\begin{ltxsyntax}

\cmditem{defbibcheck}{name}{code}

%Defines the custom bibliography filter \prm{name}, to be used via the \opt{check} option of \cmd{printbibliography}. \cmd{defbibcheck} is similar in concept to \cmd{defbibfilter} but much more low-level. Rather than a high-level expression, the \prm{code} is \latex code, much like the code used in driver definitions, which may perform arbitrary tests to decide whether or not a given entry is to be printed. The bibliographic data of the respective entry is available when the \prm{code} is executed. Issuing the command \cmd{skipentry} in the \prm{code} will cause the current entry to be skipped. For example, the following filter will only output entries with an \bibfield{abstract} field:

定义了可定制的参考文献 check \prm{name},
可以通过 \cmd{printbibliography} 的 \opt{check} 选项使用。
\cmd{defbibcheck} 从概念上类似于 \cmd{defbibfilter} 不过更加低层。
与高层次表达式不同,\prm{code} 是 \LaTeX 代码,更像是驱动定义中使用的代码,
可以执行任意测试来决定是否打印某个给定的条目。
当执行 \prm{code} 时,相应条目的文献数据是可用的。
在 \prm{code} 中使用 \cmd{skipentry} 命令会跳过当前条目。
例如,下面的 check 只会输出带有 \bibfield{abstract} 域的条目:

\begin{ltxexample}
\defbibcheck{<<abstract>>}{%
  \iffieldundef{abstract}{<<\skipentry>>}{}}
...
\printbibliography[<<check=abstract>>]
\end{ltxexample}
%
%The following check will exclude all entries published before the year 2000:
下面的 check 会排除所有在2000年之前出版的条目:

\begin{ltxexample}
\defbibcheck{recent}{%
  \iffieldint{year}
    {\ifnumless{\thefield{year}}{2000}
       {\skipentry}
       {}}
    {\skipentry}}
\end{ltxexample}
%
%See the author guide, in particular \secref{aut:aux:tst,aut:aux:ife}, for further details.
更多细节请参见作者指南,特别是 \secref{aut:aux:tst,aut:aux:ife} 节。

\end{ltxsyntax}

\subsubsection{参考文献文境}%\subsubsection{Reference Contexts}
\label{use:bib:context}

%References in a bibliography are cited and printed in a <context>. The context determines the data which is actually used to cite or provide bibliographic data for an entry. A context consists of the following information (the <context> concept is designed for future extensibility):

参考文献列表中文献的引用和打印都处于某个\emph{文境(语境)}(context)内。
对于某一条目,文境决定了实际用于标注或者在文献表中打印的数据。
一个文境包括以下信息(设计“文境”这一概念的目的在于,使其在未来具有可扩展性。):

\begin{itemize}
 \item %A sorting template
 排序模板
 \item %A template for constructing the sorting keys for names
 构建姓名排序关键字的模板
 \item %A string prefix for citation schemes which use alphabetic or numeric labels
 使用字母或数值标签的标注格式的前缀字符串
 \item %A template for calculating name uniqueness information
 计算姓名唯一性信息的模板
 \item %A template for constructing alphabetic labels for names
 构造姓名的字母顺序标签的模板
\end{itemize}
%
%The purpose of bibliography contexts is twofold. Firstly, they are used to set options which influence a printed bibliography and secondly to influence the data printed by citation commands.
%The former use is the most common when one needs to print more than one bibliography list with different, for example, sorting.
著录文境具有双重意义。首先,会用于设置影响参考表打印的选项;其次,设置的选项还可以影响标注命令输出的数据。
前一应用场景是最常见的,例如,打印多个具有不同排序的参考文献表。

\begin{ltxexample}
\usepackage[sorting=nyt]{biblatex}
\begin{document}
\cite{one}
\cite{two}
\printbibliography
\newrefcontext[sorting=ydnt]
\printbibliography
\end{ltxexample}
%
%Here we print two bibliographies, one with the default <nyt> sorting scheme and one with the <ydnt> sorting scheme.
这里我们打印两个参考文献表。其中一个带有默认的“nyt”排序格式,另一个则使用“ydnt”排序格式。

%To demonstrate the second type of use of bibliography contexts, we have to understand that the actual data for an entry can vary depending on the context. This is most obvious in the case of the \opt{extra*} fields like \opt{extrayear} which are generated by the backend according to the order of entries \emph{after} sorting so that they come out in the expected <a, b, c> order. This clearly shows that the \emph{data} in an entry can be different between sorting schemes. If a document contains more than one bibliography list with different sorting schemes, it can happen then that the \file{.bbl} contains sorting lists with the same entry but containing different data (a different value for \bibfield{extrayear}, for example). The purpose of bibliography contexts is to encapsulate things inside a context so that \biblatex can use the correct entry data. An example is printing a bibliography list with a different sorting order to the global sorting order where the \opt{extra*} fields are different for the same entry between sorting lists:
为了说明著录文境的第二类应用,我们必须理解一点:条目的实际应用数据可以随著录文境的变化而变化。
\opt{extra*} 域(例如 \opt{extrayear})就是一个最明显的例子:\opt{extrayear}是由后端根据排序\emph{之后}的条目顺序生成的,所以会出现以“a, b, c”表示的预期的顺序。这就清楚地表明,条目的\emph{数据}在不同排序格式下可以不一样。
如果文档中包含多个具有不同排序格式的参考文献表,那么 \file{.bbl} 文件中就可能包含多个排序列表,列表中的条目相同但其数据可以不同(例如 \bibfield{extrayear} 的值可以不同)。著录文境的目的就在于将这些事项封装在一个语境内部,这样 \biblatex 就可以使用正确的条目数据。下面的例子打印一个使用与全局排序格式不同的排序格式的文献表,其中同一条目的 \opt{extra*} 域在不同排序的文献表中是不同的:

\begin{ltxexample}
\usepackage[sorting=nyt,style=authoryear]{biblatex}
\DeclareSortingTemplate{yntd}{
  \sort{
    \field[strside=left,strwidth=4]{sortyear}
    \field[strside=left,strwidth=4]{year}
    \literal{9999}
  }
  \sort{
    \field{sortname}
    \field{author}
    \field{editor}
  }
  \sort[direction=descending]{
    \field{sorttitle}
    \field{title}
  }
}
\begin{document}
\cite{one}
\cite{two}
\printbibliography
\newrefcontext[sorting=yntd]
\cite{one}
\cite{two}
\printbibliography
\end{ltxexample}
%
%Here, the second use of the citations, along with the \cmd{printbibliography} command will use data from the context of the custom <yntd> sorting scheme which may well be different from the data associated with the default <nyt> scheme. That is, the citation labels (in an authoryear style which uses \opt{extrayear}) may be different \emph{for the exact same entries} between different bibliography contexts and so the citations themselves may look different.
这里,第二次使用的标注(引用)和 \cmd{printbibliography} 命令,会使用自定义的“yntd”排序格式所在著录文境中的数据,
这些数据与默认“nyt”排序格式所在文境中的数据是不同的。也就是说,\emph{对于同一条目},不同著录文境中标注的标签域的值
(如\opt{authoryear} 样式中 \opt{extrayear} 域的值)可以不同,因此标注出来的标签自然就会不同。

%Reference contexts can be declared with \cmd{DeclareRefcontext} and referred to by name, see below.

著录文境可以使用 \cmd{DeclareRefcontext} 命令进行声明,然后通过文境名使用,见以下说明。

%By default, data for a citation is drawn from the reference context of the last bibliography in which it was printed. For example:
默认情况下,用于标注的数据来自于打印该条目的最后一个参考文献表所在的著录文境。例如:

\begin{ltxexample}[style=latex]{}
\DeclareRefcontext{ap}{labelprefix=A}
\begin{document}

\cite{book, article, misc}

\printbibliography[type=book]

\newrefcontext{ap}
\printbibliography[type=article]

\newrefcontext[sorting=ydnt]
\printbibliography[type=misc]

\end{document}
\end{ltxexample}
%
%This example also shows the declaration and use of a named reference context. Assuming the entrykeys are indicative of their entrytypes, this is the default situation for the citations which corresponds to what users normally expect:
这个例子也展示了文境的声明和使用。在该例中,假设条目类型就是条目的键名,
这是标注的默认情况,是符合用户的一般期待的:

\begin{itemize}
\item %The citation of entry \bibfield{book} would draw its data from the global reference context, because the last bibliography in which it was printed was the one in the global reference context.
条目 \bibfield{book} 的标注会从全局文境中提取数据,因为打印该条目的最后的文献表位于全局文境中。
\item %The citation of entry \bibfield{article} would draw its data from reference context with \opt{labelprefix=A} and would therefore have a <A> prefix when cited.
条目 \bibfield{article} 的标注会从带有 \opt{labelprefix=A} 的文境中提取数据,因此引用时会带有前缀“A”。
\item %The citation of entry \bibfield{misc} would draw its data from the reference context with \opt{sorting=ydnt}
条目 \bibfield{misc} 的标注会从带有 \opt{sorting=ydnt} 的文境中提取数据。
\end{itemize}
%
%In cases where the user has entries which occur in multiple bibliographies in different forms or with potentially different labels (in a numeric scheme with different \bibfield{labelprefix} values for example), it may be necessary to tell \biblatex from which reference context you wish to draw the citation information. As shown above this can be done by explicitly putting citations inside reference contexts. This can be onerous in a large document and so there is specific functionality for assigning citations to reference contexts programatically, see the \cmd{assignrefcontext*} macros below.
当有一些条目同时出现在具有不同形式或可能具有不同标签的文献表中时(例如,顺序编码带有不同的 \bibfield{labelprefix} 值),你有必要告诉 \biblatex 希望从哪个文境中提取标注信息。如上所示,这可以通过显式地将标注(引用)命令置于具体的文境中来实现。但若在一个大文档中这种方式就会很麻烦,因此 biblatex 提供了一个特殊功能以将标注用代码指定方式分配到需要的文境中,见下面的 \cmd{assignrefcontext*} 宏命令。

\begin{ltxsyntax}

\cmditem{DeclareRefcontext}{name}{key=value, \dots}

%Declares a named reference context with name \prm{name}. The \prm{key=value} options define the context attributes. All context attributes are optional and default to the global settings if absent. The valid options are:
声明一个名称为 \prm{name} 的文境。\prm{key=value} 选项定义了该文境的属性。所有的文境属性都是可选的,缺省时等价于使用全局文境的设置。有效的选项为:

\begin{optionlist*}

\valitem{sorting}{name}

%Specify a sorting template defined previously with \cmd{DeclareSortingtemplate}. This template is used to determine which data to retrieve and/or print for an entry in the commands inside the context.
指定由 \cmd{DeclareSortingScheme} 命令预先定义的排序模板(格式)。该模板确定了文境内条目相关命令所要获取或打印的数据。

\valitem{sortingnamekeytemplatename}{name}

%Specify a sorting name key template defined previously with \cmd{DeclareSortingNamekeyTemplate}. This scheme is used to construct sorting keys for names inside the context.
指定由 \cmd{DeclareSortingNamekeyTemplate} 命令预先定义的姓名排序关键字模板(格式)。该模板用于构建文境内的姓名排序关键字。

\valitem{uniquenametemplatename}{name}

%Specify a uniquename template defined previously with \cmd{DeclareUniquenameTemplate} (see \secref{aut:cav:amb}). This template is used to calculate uniqueness information for names inside the context. The template name can also be specified (in increasing order of preference) per"=entry, per"=name list and per"=name. See \secref{apx:opt} for information on setting per"=option, per"=namelist and per"=name options.
指定由 \cmd{DeclareUniquenameTemplate} 预先定义的唯一性姓名模板(见\secref{aut:cav:amb}),该模板用于计算文境内姓名的唯一性信息。模板名也可以指定给单个条目、单个姓名列表、单个姓名中。
设置单个条目、单个姓名列表、单个姓名的选项信息见\secref{apx:opt}。

\valitem{labelalphanametemplatename}{name}

%Specify a template defined previously with \cmd{DeclareLabelalphaNameTemplate} (see \secref{aut:ctm:lab}). This template is used to construct name parts of alphabetic labels for names inside the context. The template name can also be specified (in increasing order of preference) per"=entry, per"=name list and per"=name. See \secref{apx:opt} for information on setting per"=option, per"=namelist and per"=name options.
指定由 \cmd{DeclareLabelalphaNameTemplate} 预先定义的字母顺序姓名标签模板(见\secref{aut:ctm:lab}),该模板用于构造文境内姓名的字母顺序标签的姓名成分。模板名也可以指定给单个条目、单个姓名列表、单个姓名中。设置单个条目、单个姓名列表、单个姓名的选项信息见\secref{apx:opt}。

\valitem{nametemplates}{name}

%A convenience meta-option which sets \opt{sortingnamekeytemplate}, \opt{uniquenametemplate} and \opt{labelalphanametemplate} to the same template name. This option can also be specified (in increasing order of preference) per"=entry, per"=name list and per"=name. See \secref{apx:opt} for information on setting per"=option, per"=namelist and per"=name options.
用于对同一模板名设置\opt{sortingnamekeytemplate}, \opt{uniquenametemplate} 和 \opt{labelalphanametemplate}的一个方便选项。该选项也可指定给单个条目、单个姓名列表、单个姓名中。设置单个条目、单个姓名列表、单个姓名的选项信息见\secref{apx:opt}。

\valitem{labelprefix}{string}

%This option applies to numerical citation\slash bibliography styles only and requires that the \opt{defernumbers} option from \secref{use:opt:pre:gen} be enabled globally. Setting this option will implicitly enable \opt{resetnumbers} for the any \cmd{printbibliography} in the scope of the context (unless overridden by a user-specified value for \opt{resetnumbers}). The option assigns the \prm{string} as a prefix to all entries in the reference context. For example, if the \prm{string} is \texttt{A}, the numerical labels printed will be \texttt{[A1]}, \texttt{[A2]}, \texttt{[A3]}, etc. This is useful for subdivided numerical bibliographies where each subbibliography uses a different prefix. The \prm{string} is available to styles in the \bibfield{labelprefix} field of all affected entries. See \secref{aut:bbx:fld:lab} for details.
该选项只用于顺序编码的标注和著录样式样式,需要全局开启 \secref{use:opt:pre:gen} 节中的 \opt{defernumbers} 选项。设置该选项也会为文境范围内的任意 \cmd{printbibliography} 启用 \opt{resetnumbers} 选项(除非 \opt{resetnumbers} 被用户指定的值覆盖)。该选项将为文境内的所有条目分配一个 \prm{string} 的前缀。例如,如果 \prm{string} 是 \texttt{A},那么打印出来的编码标签将形如 \texttt{[A1]}, \texttt{[A2]}, \texttt{[A3]} 等。这适用于划分成不同部分的子文献表使用不同前缀的情况。
所有受影响条目的 \bibfield{labelprefix} 域的格式设置也能使用该\prm{string}。详见 \secref{aut:bbx:fld:lab} 节。

\end{optionlist*}
%

\envitem{refcontext}[key=value, \dots]{name}

%Wraps a reference context environment. The possible \prm{key=value} optional arguments are as for \cmd{DeclareRefcontext} and override options given for the named reference context \prm{name}. \prm{name} can also be omitted as \verb+{}+ or by omitting even the empty braces\footnote{This slightly odd syntax possibility is a result of backwards compatibility with \biblatex $<$3.5}.
将文境封装在一个环境内。可能的 \prm{key=value} 可选设置和 \cmd{DeclareRefcontext} 中的相同,
并且覆盖名为 \prm{name} 的文境的选项。\prm{name} 也可以省略成 \verb+{}+,甚至空的括号也可以省略
\footnote{这种有点怪异的句法是出于对 \biblatex $<$3.5 的向后兼容性。}。

%The \opt{refcontext} environment cannot be nested and \biblatex will generate an error if you try to do so.
\env{refcontext} 环境不能相互嵌套,如果有嵌套,\biblatex 会报错。

\cmditem{newrefcontext}[key=value, \dots]{name}

%This command is similar to the \env{refcontext} environment except that it is a stand"=alone command rather than an environment. It automatically ends any previous reference context section begun with \cmd{newrefcontext} (if any) and immediately starts a new one. Note that the context section started by the last \cmd{newrefcontext} command in the document will extend to the very end of the document. Use \cmd{endrefcontext} if you want to terminate it earlier.
该命令类似于 \env{refcontext} 环境,不同之处在于这是单独的命令而不是环境。
它会自动结束任何之前以 \cmd{newrefcontext} 开始的文境(如果有的话),并立即开启新的文境。
注意,文档中最后的 \cmd{newrefcontext} 命令开启的引用文境会一直持续到文档的最后。
如果想要提前终止,那么需要使用 \cmd{endrefcontext} 命令。

\end{ltxsyntax}
%
%At the beginning of the document, there is always a global context containing global settings for each of the reference context options. Here is an example summarising the reference contexts with various settings:

在文档的开始,总会有一个全局的文境,并会设置了全局的文境选项。如下例子总结了文境的不同设置:

\begin{ltxexample}[style=latex]{}
\usepackage[sorting=nty]{biblatex}

\DeclareRefcontext{testrc}{sorting=nyt}

% Global reference context:
%   sorting=nty
%   sortingnamekeytemplate=global
%   labelprefix=

\begin{document}

\begin{refcontext}{testrc}
% reference context:
%   sorting=nyt
%   sortingnamekeytemplate=global
%   labelprefix=
\end{refcontext}

\begin{refcontext}[labelprefix=A]{testrc}
% reference context:
%   sorting=nyt
%   sortingnamekeytemplate=global
%   labelprefix=A
\end{refcontext}

\begin{refcontext}[sorting=ydnt,labelprefix=A]
% reference context:
%   sorting=ydnt
%   sortingnamekeytemplate=global
%   labelprefix=A
\end{refcontext}

\newrefcontext}[labelprefix=B]
% reference context:
%   sorting=nty
%   sortingnamekeytemplate=global
%   labelprefix=B
\endrefcontext

\newrefcontext}[sorting=ynt,labelprefix=C]{testrc}
% reference context:
%   sorting=ynt
%   sortingnamekeytemplate=global
%   labelprefix=C
\endrefcontext
\end{ltxexample}

\begin{ltxsyntax}

\cmditem{assignrefcontextkeyws}[key=value, \dots]{keyword1,keyword2, ...}
\cmditem{assignrefcontextkeyws*}[key=value, \dots]{keyword1,keyword2, ...}
\cmditem{assignrefcontextcats}[key=value, \dots]{category1, category2, ...}
\cmditem{assignrefcontextcats*}[key=value, \dots]{category1, category2, ...}
\cmditem{assignrefcontextentries}[key=value, \dots]{entrykey1, entrykey2, ...}
\cmditem{assignrefcontextentries*}[key=value, \dots]{entrykey1, entrykey2, ...}
\cmditem{assignrefcontextentries}[key=value, \dots]{*}
\cmditem{assignrefcontextentries*}[key=value, \dots]{*}

\end{ltxsyntax}
%These commands automate putting citations into refcontexts when the default behaviour is not sufficient. The default behaviour is that the data for a citation is drawn from the refcontext of the last bibliography in which it was printed. For citations that are used in some way but not printed in a bibliography or bibliography list, they default to drawing their data from the global refcontext established at the beginning of the document. To override this behaviour, instead of manually wrapping citation commands in \env{refcontext} environments, which might be error-prone and tedious, you can register a comma"=separated list of \prm{keywords}, \prm{categories} or \prm{entrykeys} which, respectively, make the entries with any of the specified keywords, entries in any of the specified categories (see \secref{use:use:div}) or entries with any of the specified citation keys draw their data from a particular refcontext specified by the \prm{refcontext key/values} which are parsed as the per the corresponding \env{refcontext} options. Such refcontext auto-assignments are specific to the current refsection. You may specify the same citation key in any of these commands but be aware that assignment is done in the order \prm{keywords}, \prm{categories}, \prm{entrykeys} with the later specifications overriding the earlier. \cmd{assignrefcontextentries} accepts a single asterisk instead of a list of entrykeys which allows the assignment of all keys in a refsection to a refcontext with having to explicitly list them. An example:
标注命令自动从打印当前条目的最后一个文献表所在文境提取数据这种默认方式有时是无法满足需求的。
对于某些没有在文献表中打印条目但又以某种方式给出的标注命令,它们默认从文档开始时建立的全局文境获取数据。
要覆盖这一行为,可以手动将标注命令放在 \env{refcontext} 环境内,但是这样容易出错并且很繁琐。
而 \cmd{assignrefcontext*}这些命令可以用来设置一个由 \prm{keywords}、\prm{categories} 或 \prm{entrykeys} 构成的逗号分隔列表,使带有这些指定关键字、或属于任何指定类别(见 \secref{use:use:div} 节)、或指定引用关键字的条目可以从\prm{refcontext key/values}指出的特定文境获取数据,这些\prm{refcontext key/values}选项会被解析为对应的 \env{refcontext} 选项。这种文境自动分配是针对当前的参考文献分节的。当你在使用这些命令时可能会重复指定相同的条目,要注意的是 biblatex 通常会按照 \prm{keywords}, \prm{categories}, \prm{entrykeys} 的先后顺序由后面的设定覆盖前面的设定。
\cmd{assignrefcontextentries} 命令可以接受单个星号来代替条目键列表,这时会将参考文献分节中的所有条目键都分配给某个文境,而不必显式列出。例如:

\begin{ltxexample}[style=latex]{}
\assignrefcontextentries[labelprefix=A]{key2}
\cite{key1}
\begin{refcontext}[labelprefix=B]
\cite{key2}
\end{refcontext}
\end{ltxexample}
%
%Here, the data for the citation of \bibfield{key2} will be drawn from refcontext \opt{labelprefix=A} and not \opt{labelprefix=B} (resulting in a label with prefix <A> and not <B>).

这里 \bibfield{key2} 的标注用数据会从文境 \opt{labelprefix=A} 中提取,而不是 \opt{labelprefix=B}。
即,标签的前缀是“A”不是“B”。

%The starred versions do not override a local refcontext and so with:
而带星号的版本不会覆盖局部的文境,所以对于如下设定:

\begin{ltxexample}[style=latex]{}
\assignrefcontextentries*[labelprefix=A]{key2}
\cite{key1}
\begin{refcontext}[labelprefix=B]
\cite{key2}
\end{refcontext}
\end{ltxexample}
%
%the data for the citation of \bibfield{key2} will be drawn from refcontext \opt{labelprefix=B}. Note that these commands are rarely necessary unless you have multiple bibliographies in which the same citations occur and \biblatex\ cannot by default tell which bibliography list a citation should refer to. See the example file \file{94-labelprefix.tex} for more details.
\bibfield{key2} 的引用数据会从 \opt{labelprefix=B} 文境中提取。注意,这些命令大部分情况下都不必使用,
除非多个参考文献表中有相同文献的标注(引用)且 \biblatex 按照默认设置无法知道标注应该指向哪一个文献表。
更多细节详见文件 \file{94-labelprefix.tex} 中的例子。

\subsubsection{动态条目集}%\subsubsection{Dynamic Entry Sets}
\label{use:bib:set}

%In addition to the \bibtype{set} entry type, \biblatex also supports dynamic entry sets defined on a per-document\slash per-refsection basis. The following command, which may be used in the document preamble or the document body, defines the set \prm{key}:

除了 \bibtype{set} 条目类型之外,\biblatex 也支持基于文档或参考文献分节定义的动态条目集。
下面的命令定义了 \prm{key} 集合,可以用在导言区或正文中:

\begin{ltxsyntax}

\cmditem{defbibentryset}{key}{key1,key2,key3, \dots}

%The \prm{key} is the entry key of the set, which is used like any other entry key when referring to the set. The \prm{key} must be unique and it must not conflict with any other entry key. The second argument is a comma"=separated list of the entry keys which make up the set. \cmd{defbibentryset} implies the equivalent of a \cmd{nocite} command, \ie all sets which are declared are also added to the bibliography. When declaring the same set more than once, only the first invocation of \cmd{defbibentryset} will define the set. Subsequent definitions of the same \prm{key} are ignored and work like \cmd{nocite}\prm{key}. Dynamic entry sets defined in the document body are local to the enclosing \env{refsection} environment, if any. Otherwise, they are assigned to reference section~0. Those defined in the preamble are assigned to reference section~0. See \secref{use:use:set} for further details.

\prm{key} 是集合的条目键,像其它条目键一样用于指向该集合。
\prm{key} 必须是唯一的,并且不能与其它条目键名冲突。
第二个选项是组成该集合的逗号分隔条目键列表。
\cmd{defbibentryset} 也蕴含了与 \cmd{nocite} 命令的等价性,
即所有声明的集合也都添加到参考文献表中。
当多次声明相同集合时,只有第一次调用的 \cmd{defbibentryset} 会定义该集合。
接下来的对于相同 \prm{key} 的定义将被忽略并如同 \cmd{nocite}\prm{key} 一样处理。
在正文中定义的动态条目集如果包含在 \env{refsection} 环境中,那么是局部的。
否则它们会归到第~0 文献分节中。
在导言区中定义的动态条目集也归到第~0 文献分节中。
详见 \secref{use:use:set} 节。

\end{ltxsyntax}

\subsection{标注(引用)命令}%\subsection{Citation Commands}
\label{use:cit}

%All citation commands generally take one mandatory and two optional arguments. The \prm{prenote} is text to be printed at the beginning of the citation. This is usually a notice such as <see> or <compare>. The \prm{postnote} is text to be printed at the very end of the citation. This is usually a page number. If only one of these arguments is given, it is taken as a postnote. If you want to specify a prenote but no postnote, you need to leave the second optional argument empty, as in |\cite[see][]{key}|. The \prm{key} argument to all citation commands is mandatory. This is the entry key or a comma"=separated list of keys corresponding to the entry keys in the \sty{bib} file. In sum, all basic citations commands listed further down have the following syntax:

大体上,所有的标注(引用)命令都有一个必选参数和两个可选参数。
前注 \prm{prenote} 是标注开始时打印的文本,通常是“see”或“compare”等提示。
而后注 \prm{postnote} 是标注结束时打印的文本,通常是页码数。
如果只给出一个可选参数,那么将视作后缀。
如果想给出前注但不要后注,那么需要将第二个可选项设置为空,例如 |\cite[see][]{key}|。
所有的标注(引用)命令中选项 \prm{key} 都是必须的,
其内容是 \file{bib} 文件中的条目键或者对应于条目键的逗号分隔列表。
\emph{总的来说,以下所有的基本引用命令都具有如下的句法结构}:

\begin{ltxsyntax}

\cmditem*{command}[prenote][postnote]{keys}<punctuation>

%If the \opt{autopunct} package option from \secref{use:opt:pre:gen} is enabled, they will scan ahead for any \prm{punctuation} immediately following their last argument. This is useful to avoid spurious punctuation marks after citations. This feature is configured with \cmd{DeclareAutoPunctuation}, see \secref{aut:pct:cfg} for details.
如果启用了 \secref{use:opt:pre:gen} 节中的 \opt{autopunct} 宏包选项,
这些命令会继续扫描紧跟在最后一个参数后的标点符号(\prm{punctuation})。
这可用于避免在标注之后出现错误的标点符号。
该特性由 \cmd{DeclareAutoPunctuation} 命令配置,详见 \secref{aut:pct:cfg} 节。

\end{ltxsyntax}

\subsubsection{标准命令} %\subsubsection{Standard Commands}
\label{use:cit:std}

%The following commands are defined by the citation style. Citation styles may provide any arbitrary number of specialized commands, but these are the standard commands typically provided by general-purpose styles.

下列命令由标注样式定义。
标注样式可以提供任意特殊命令,但是这些是由通用样式提供的标准命令。

\begin{ltxsyntax}


\cmditem{cite}[prenote][postnote]{key}
\cmditem{Cite}[prenote][postnote]{key}

%These are the bare citation commands. They print the citation without any additions such as parentheses. The numeric and alphabetic styles still wrap the label in square brackets since the reference may be ambiguous otherwise. \cmd{Cite} is similar to \cmd{cite} but capitalizes the name prefix of the first name in the citation if the \opt{useprefix} option is enabled, provided that there is a name prefix and the citation style prints any name at all.

基本标注(引用)命令。打印出不带括号等任何附加物的标注标签。
不过顺序数字编码和字母编码样式仍然会将标签放在方括号里,因为不这样的话引文可能含糊不清。
\cmd{Cite} 与 \cmd{cite} 类似,不同之处仅仅在于如果开启了 \opt{useprefix} 选项它会大写名的前缀,假设姓名中存在前缀,且标注会打印姓名的所有成分。

\cmditem{parencite}[prenote][postnote]{key}
\cmditem{Parencite}[prenote][postnote]{key}

%These commands use a format similar to \cmd{cite} but enclose the entire citation in parentheses. The numeric and alphabetic styles use square brackets instead. \cmd{Parencite} is similar to \cmd{parencite} but capitalizes the name prefix of the first name in the citation if the \opt{useprefix} option is enabled, provided that there is a name prefix and the citation style prints any name at all.

这些命令的格式类似于 \cmd{cite},不过将引用全部放入圆括号内。
不过数值和字母样式仍然使用方括号。\cmd{Parencite} 与 \cmd{parencite} 类似,
不同之处仅仅在于如果开启了 \opt{useprefix} 选项它会大写名的前缀,假设姓名中存在前缀,且标注会打印姓名的所有成分。

\cmditem{footcite}[prenote][postnote]{key}
\cmditem{footcitetext}[prenote][postnote]{key}

%These command use a format similar to \cmd{cite} but put the entire citation in a footnote and add a period at the end. In the footnote, they automatically capitalize the name prefix of the first name if the \opt{useprefix} option is enabled, provided that there is a name prefix and the citation style prints any name at all. \cmd{footcitetext} differs from \cmd{footcite} in that it uses \cmd{footnotetext} instead of \cmd{footnote}.

这些命令的格式类似于 \cmd{cite},不过将引用的全部放入脚注内并在末尾加上句号。
在脚注中,如果有姓名前缀并且标注样式打印姓名的全部成分,且开启了 \opt{useprefix} 选项的话,那么该姓名前缀的会自动大写。
\cmd{footcitetext} 与 \cmd{footcite} 的不同之处在于它使用 \cmd{footnotetext} 而不是 \cmd{footnote} 命令。

\end{ltxsyntax}

\subsubsection{与样式相关的命令}%\subsubsection{Style-specific Commands}
\label{use:cit:cbx}

%The following additional citation commands are only provided by some of the citation styles which ship with this package.
下列附加的标注(引用)命令只由本宏包所带的某些标注样式提供。

\begin{ltxsyntax}

\cmditem{textcite}[prenote][postnote]{key}
\cmditem{Textcite}[prenote][postnote]{key}

%These citation commands are provided by all styles that come with this package. They are intended for use in the flow of text, replacing the subject of a sentence. They print the authors or editors followed by a citation label which is enclosed in parentheses. Depending on the citation style, the label may be a number, the year of publication, an abridged version of the title, or something else. The numeric and alphabetic styles use square brackets instead of parentheses. In the verbose styles, the label is provided in a footnote. Trailing punctuation is moved between the author or editor names and the footnote mark. \cmd{Textcite} is similar to \cmd{textcite} but capitalizes the name prefix of the first name in the citation if the \opt{useprefix} option is enabled, provided that there is a name prefix.
这些标注(引用)命令在本宏包所带的所有样式中都有提供。它们用于文本流中,用以代替句子的主语。
它们打印出作者或编辑,后面接着在括号中括起来的标注标签。
取决于不同的标注样式,标签可以是数字、出版年份、缩写的标题等等。
顺序编码和字母顺序样式会使用方括号来代替圆括号。
在完整信息(verbose)样式中,标签在脚注中给出。作者或编辑名与脚注标记的标点将被移除。
\cmd{Textcite} 与 \cmd{textcite} 类似,
不同之处在于,如果有姓名前缀且启用 \opt{useprefix} 选项的话,标注中的前缀名是大写的。

\cmditem{smartcite}[prenote][postnote]{key}
\cmditem{Smartcite}[prenote][postnote]{key}

%Like \cmd{parencite} in a footnote and like \cmd{footcite} in the body.
在脚注中与 \cmd{parencite} 相似而在正文中与 \cmd{footcite} 类似。

\cmditem{cite*}[prenote][postnote]{key}

%This command is provided by all author-year and author-title styles. It is similar to the regular \cmd{cite} command but merely prints the year or the title, respectively.
该命令由所有的\texttt{作者--年份}和\texttt{作者--标题}样式提供。
它类似于常规的 \cmd{cite} 命令,但是只打印出年份或者标题。

\cmditem{parencite*}[prenote][postnote]{key}

%This command is provided by all author-year and author-title styles. It is similar to the regular \cmd{parencite} command but merely prints the year or the title, respectively.
该命令由所有的\texttt{作者--年份}和\texttt{作者--标题}样式提供。
它类似于常规的 \cmd{parencite} 命令,但是只打印出年份或者标题。

\cmditem{supercite}{key}

%This command, which is only provided by the numeric styles, prints numeric citations as superscripts without brackets. It uses \cmd{supercitedelim} instead of \cmd{multicitedelim} as citation delimiter. Note that any \prm{prenote} and \prm{postnote} arguments are ignored. If they are given, \cmd{supercite} will discard them and issue a warning message.
该命令只在顺序编码样式中提供,会以不带括号的上标形式打印出数字标签。
它使用  \cmd{supercitedelim} 而不是 \cmd{multicitedelim} 作为标注定界符。
请注意,任何的  \prm{prenote} 和 \prm{postnote} 选项都会被忽略。
如果给出的话也会弃掉它们并且显示警告信息。

\end{ltxsyntax}

\subsubsection{合格的标注列表}%\subsubsection{Qualified Citation Lists}
\label{use:cit:mlt}

%This package supports a class of special citation commands called <multicite> commands. The point of these commands is that their argument is a list of citations where each item forms a fully qualified citation with a pre- and\slash or postnote. This is particularly useful with parenthetical citations and citations given in footnotes. It is also possible to assign a pre- and\slash or postnote to the entire list. The multicite commands are built on top of backend commands like \cmd{parencite} and \cmd{footcite}. The citation style provides a multicite definition with \cmd{DeclareMultiCiteCommand} (see \secref{aut:cbx:cbx}). The following example illustrates the syntax of multicite commands:
本宏包支持一类特殊的称为 <multicite> 的标注命令。这些命令的特点是,它们的参数是一个标注列表,其中每一项都具有完整的前注/后注形式。这对于带括号或脚注中给出的标注特别有用。也可以将一个前注或后注分配给整个列表。这种多重标注命令构建在 \cmd{parencite} 和 \cmd{footcite} 等后端命令基础之上。标注样式可以通过 \cmd{DeclareMultiCiteCommand}命令(见 \secref{aut:cbx:cbx} 节)对多重标注进行定义。下例展示了多重标注命令的语法:

\begin{ltxexample}
\parencites[35]{key1}[88--120]{key2}[23]{key3}
\end{ltxexample}
%
%The format of the arguments is similar to that of the regular citation commands, except that only one citation command is given. If only one optional argument is given for an item in the list, it is taken as a postnote. If you want to specify a prenote but no postnote, you need to leave the second optional argument of the respective item empty:
参数的格式与常规标注命令类似,不过只需给出一个标注命令。如果列表中的某一项只给出一个可选参数,那么该可选参数将被视为后注。
如果只想要前注而不要后注,那么需要将相应项的第二个可选参数设置为空:

\begin{ltxexample}
\parencites[35]{key1}[chapter 2 in][]{key2}[23]{key3}
\end{ltxexample}
%
%In addition to that, the entire citation list may also have a pre- and\slash or postnote. The syntax of these global notes differs from other optional arguments in that they are given in parentheses rather than the usual brackets:
此外,整个的标注列表也可以有一个共同的前注或后注。与其他可选参数的语法不同,这些全局注记在圆括号而不是通常的方括号中:

\begin{ltxexample}
\parencites<<(>>and chapter 3<<)>>[35]{key1}[78]{key2}[23]{key3}
\parencites<<(>>Compare<<)()>>[35]{key1}[78]{key2}[23]{key3}
\parencites<<(>>See<<)(>>and the introduction<<)>>[35]{key1}[78]{key2}[23]{key3}
\end{ltxexample}
%
%Note that the multicite commands keep on scanning for arguments until they encounter a token that is not the start of an optional or mandatory argument. If a left brace or bracket follows a multicite command, you need to mask it by adding \cmd{relax} or a control space (a backslash followed by a space) after the last valid argument. This will cause the scanner to stop.
请注意,多重标注命令会一直扫描选项,直到遇到一个不是可选或必选参数的记号为止。
如果文本中紧接着多重标注命令的是一个左圆括号或方括号,那么需要手动在最后一个有效参数后添加 \cmd{relax} 命令或者控制空格(跟在反斜线后的空格)作为标记,这样才会停止该命令的扫描。

\begin{ltxexample}[style=latex,showspaces]{}
\parencites[35]{key1}[78]{key2}<<\relax>>[...]
\parencites[35]{key1}[78]{key2}<<\ >>{...}
\end{ltxexample}
%
%By default, this package provides the following multicite commands which correspond to regular commands from \secref{use:cit:std, use:cit:cbx}:
默认情况下,本宏包提供了如下一些多重标注命令,分别对应于 \secref{use:cit:std, use:cit:cbx} 节中常规命令:

\begin{ltxsyntax}

\cmditem{cites}(multiprenote)(multipostnote)[prenote][postnote]{key}|...|[prenote][postnote]{key}
\cmditem{Cites}(multiprenote)(multipostnote)[prenote][postnote]{key}|...|[prenote][postnote]{key}

%The multicite version of \cmd{cite} and \cmd{Cite}, respectively.
\cmd{cite} 和 \cmd{Cite} 的多重标注版本。

\cmditem{parencites}(multiprenote)(multipostnote)[prenote][postnote]{key}|...|[prenote][postnote]{key}
\cmditem{Parencites}(multiprenote)(multipostnote)[prenote][postnote]{key}|...|[prenote][postnote]{key}

%The multicite version of \cmd{parencite} and \cmd{Parencite}, respectively.
\cmd{parencite} 和 \cmd{Parencite} 的多重标注版本。

\cmditem{footcites}(multiprenote)(multipostnote)[prenote][postnote]{key}|...|[prenote][postnote]{key}
\cmditem{footcitetexts}(multiprenote)(multipostnote)[prenote][postnote]{key}|...|[prenote][postnote]{key}

%The multicite version of \cmd{footcite} and \cmd{footcitetext}, respectively.
\cmd{footcite} 和 \cmd{footcitetext} 的多重标注版本。

\cmditem{smartcites}(multiprenote)(multipostnote)[prenote][postnote]{key}|...|[prenote][postnote]{key}
\cmditem{Smartcites}(multiprenote)(multipostnote)[prenote][postnote]{key}|...|[prenote][postnote]{key}

%The multicite version of \cmd{smartcite} and \cmd{Smartcite}, respectively.
\cmd{smartcite} 和 \cmd{Smartcite} 的多重标注版本。

\cmditem{textcites}(multiprenote)(multipostnote)[prenote][postnote]{key}|...|[prenote][postnote]{key}
\cmditem{Textcites}(multiprenote)(multipostnote)[prenote][postnote]{key}|...|[prenote][postnote]{key}

%The multicite version of \cmd{textcite} and \cmd{Textcite}, respectively.
\cmd{textcite} 和 \cmd{Textcite} 的多重标注版本。

\cmditem{supercites}(multiprenote)(multipostnote)[prenote][postnote]{key}|...|[prenote][postnote]{key}

%The multicite version of \cmd{supercite}. This command is only provided by the numeric styles.
\cmd{supercite} 的多重标注版本。该命令只由顺序编码样式提供。

\end{ltxsyntax}

\subsubsection{与样式无关的命令} %\subsubsection{Style-independent Commands}
\label{use:cit:aut}

%Sometimes it is desirable to give the citations in the source file in a format that is not tied to a specific citation style and can be modified globally in the preamble. The format of the citations is easily changed by loading a different citation style. However, when using commands such as \cmd{parencite} or \cmd{footcite}, the way the citations are integrated with the text is still effectively hard"=coded. The idea behind the \cmd{autocite} command is to provide higher"=level citation markup which makes global switching from inline citations to citations given in footnotes (or as superscripts) possible. The \cmd{autocite} command is built on top of backend commands like \cmd{parencite} and \cmd{footcite}. The citation style provides an \cmd{autocite} definition with \cmd{DeclareAutoCiteCommand} (see \secref{aut:cbx:cbx}). This definition may be activated with the \opt{autocite} package option from \secref{use:opt:pre:gen}. The citation style will usually initialize this package option to a value which is suitable for the style, see \secref{use:xbx:cbx} for details. Note that there are certain limits to high"=level citation markup. For example, inline author-year citation schemes often integrate citations so tightly with the text that it is virtually impossible to automatically convert them to footnotes. The \cmd{autocite} command is only applicable in cases in which you would normally use \cmd{parencite} or \cmd{footcite} (or \cmd{supercite}, with a numeric style). The citations should be given at the end of a sentence or a partial sentence, immediately preceding the terminal punctuation mark, and they should not be a part of the sentence in a grammatical sense (like \cmd{textcite}, for example).

有时我们需要源文件中的引用格式不依赖于某种特定的引用样式,而是可以在导言区中全局修改。
通过导入不同的引用样式我们可以很容易地改变引用格式。
但是,当使用 \cmd{parencite} 或 \cmd{footcite} 等命令时,引用和正文结合的方式仍然是硬编码的(hard-coded)。
\cmd{autocite} 命令的想法是提供高层次上的引用标记,使得可以全局地切换行内引用和脚注引用(或上标引用)。
\cmd{autocite} 命令构建在 \cmd{parencite} 和 \cmd{footcite} 等后端命令之上。
由引用样式通过 \cmd{DeclareAutoCiteCommand} (见 \secref{aut:cbx:cbx} 节)来定义。
该定义可以通过 \secref{use:opt:pre:gen} 节中的 \opt{autocite} 宏包选项启用。
引用样式通常会将该宏包选项初始化为适合该样式的值,详见 \secref{use:xbx:cbx}。
请注意,这种高层次引用标记有一些限制。
例如,行内的\texttt{作者-年份}引用格式通常和正文结合得很紧密,事实上不可能自动转化为脚注。
只有当正常地使用 \cmd{parencite} 或 \cmd{footcite} (或者数值样式中的 \cmd{supercite})时,\cmd{autocite} 命令才是可用的。
引用应当在句子或其部分的末尾、标点符号之前给出,并且在语法上不应当是句子的一部分(例如 \cmd{textcite})。

\begin{ltxsyntax}

\cmditem{autocite}[prenote][postnote]{key}
\cmditem{Autocite}[prenote][postnote]{key}

%In contrast to other citation commands, the \cmd{autocite} command does not only scan ahead for punctuation marks following its last argument to avoid double punctuation marks, it actually moves them around if required. For example, with \kvopt{autocite}{footnote}, a trailing punctuation mark will be moved such that the footnote mark is printed after the punctuation. \cmd{Autocite} is similar to \cmd{autocite} but capitalizes the name prefix of the first name in the citation if the \opt{useprefix} option is enabled, provided that there is a name prefix and the citation style prints any name at all.

与其它引用命令不同,\cmd{autocite} 命令不仅扫描跟在最后一个选项的标点符号来避免双重标点,还在必要时挪动标点。
例如,后面的标点在 \kvopt{autocite}{footnote} 选项下会被移动使得脚注标记打印在标点之后。
\cmd{Autocite} 类似于 \cmd{autocite},不同之处在于,
如果引用样式打印全名并且有姓名前缀,并且启用了 \opt{useprefix} 选项,那么引用中的姓名前缀会大写。

\cmditem*{autocite*}[prenote][postnote]{key}
\cmditem*{Autocite*}[prenote][postnote]{key}

%The starred variants of \cmd{autocite} do not behave differently from the regular ones. The asterisk is simply passed on to the backend command. For example, if \cmd{autocite} is configured to use \cmd{parencite}, then \cmd{autocite*} will execute \cmd{parencite*}.

\cmd{autocite} 带星号的变种没有什么不同。
星号只是传递给后端命令。
例如,如果 \cmd{autocite} 配置使用 \cmd{parencite},那么 \cmd{autocite*} 就会执行 \cmd{parencite*}。

\cmditem{autocites}(multiprenote)(multipostnote)[prenote][postnote]{key}|...|[prenote][postnote]{key}
\cmditem{Autocites}(multiprenote)(multipostnote)[prenote][postnote]{key}|...|[prenote][postnote]{key}

%This is the multicite version of \cmd{autocite}. It also detects and moves punctuation if required. Note that there is no starred variant. \cmd{Autocites} is similar to \cmd{autocites} but capitalizes the name prefix of the first name in the citation if the \opt{useprefix} option is enabled, provided that there is a name prefix and the citation style prints any name at all.

\cmd{autocite} 的多重引用版本。
它同样会在必要时检测和移动标点。
请注意它没有带星号的变种。
\cmd{Autocites} 类似于 \cmd{autocites},不同之处在于,
如果引用样式打印全名并且有姓名前缀,并且启用了 \opt{useprefix} 选项,那么引用中的姓名前缀会大写。

\end{ltxsyntax}

\subsubsection{文本命令}%\subsubsection{Text Commands}
\label{use:cit:txt}

%The following commands are provided by the core of \biblatex. They are intended for use in the flow of text. Note that all text commands are excluded from citation tracking.

下列命令由 \biblatex 内核提供,在文本流中使用。
请注意,这里所有的文本命令都与引用追踪无关。

\begin{ltxsyntax}

\cmditem{citeauthor}[prenote][postnote]{key}
\cmditem*{citeauthor*}[prenote][postnote]{key}
\cmditem{Citeauthor}[prenote][postnote]{key}
\cmditem*{Citeauthor*}[prenote][postnote]{key}

%These commands print the authors. Strictly speaking, it prints the \bibfield{labelname} list, which may be the \bibfield{author}, the \bibfield{editor}, or the \bibfield{translator}. \cmd{Citeauthor} is similar to \cmd{citeauthor} but capitalizes the name prefix of the first name in the citation if the \opt{useprefix} option is enabled, provided that there is a name prefix. The starred variants effectively force maxcitenames to 1 for just this command on so only print the first name in the labelname list (potentially followed by the «et al» string if there are more names). This allows more natural textual flow when refering to a paper in the singular when otherwise \cmd{citeauthor} would generate a (naturally plural) list of names.

这些命令打印出作者。
严格地讲,打印出的是 \bibfield{labelnames} 列表,
这可以是 \bibfield{author}、\bibfield{editor} 或者 \bibfield{translator} 等域。
\cmd{Citeauthor} 类似于 \cmd{citeauthor},不同之处在于,
如果有姓名前缀并且启用  \opt{useprefix} 选项的话,引用中的名字前缀采用大写形式。
带星号的版本会强制设置 \texttt{maxcitenames} 为 1,
这样只会打印标签名称列表中第一个姓名(如果有更多名字的话会后接“et al”字符串)。
当涉及单一作者的文章时,这会使行文更加自然;
否则的话(多位作者时),使用 \cmd{citeauthor} 会产生名称列表(当然是复数的)。


\cmditem{citetitle}[prenote][postnote]{key}
\cmditem*{citetitle*}[prenote][postnote]{key}

%This command prints the title. It will use the abridged title in the \bibfield{shorttitle} field, if available. Otherwise it falls back to the full title found in the \bibfield{title} field. The starred variant always prints the full title.

该命令会打印出标题。
它会在可用时使用 \bibfield{shorttitle} 域中的短标题,否则的话会使用备用的 \bibfield{title} 域中的标题全称。
带星号的版本总会打印出标题全称。

\cmditem{citeyear}[prenote][postnote]{key}
\cmditem*{citeyear*}[prenote][postnote]{key}

%This command prints the year (\bibfield{year} field or year component of \bibfield{date}). The starred variant includes the \bibfield{extrayear} information, if any.

该命令会打印出年份(\bibfield{year} 域或者 \bibfield{date} 域中的年份成分)。
带星号的版本会包括可能有的 \bibfield{extrayear} 信息。

\cmditem{citedate}[prenote][postnote]{key}
\cmditem*{citedate*}[prenote][postnote]{key}

%This command prints the full date (\bibfield{date} or \bibfield{year}). The starred variant includes the \bibfield{extrayear} information, if any.

该命令会打印出日期(\bibfield{date} 或 \bibfield{year} 域)。
带星号的版本会包括可能有的 \bibfield{extrayear} 信息。

\cmditem{citeurl}[prenote][postnote]{key}

%This command prints the \bibfield{url} field.

该命令打印 \bibfield{url} 域。

\cmditem{parentext}{text}

%This command wraps the \prm{text} in context sensitive parentheses.

该命令将 \prm{text} 装入上下文相关的圆括号中。

\cmditem{brackettext}{text}

%This command wraps the \prm{text} in context sensitive brackets.

该命令将 \prm{text} 装入上下文相关的方括号中。

\end{ltxsyntax}

\subsubsection{特殊命令}%\subsubsection{Special Commands}
\label{use:cit:spc}

%The following special commands are also provided by the core of \biblatex.

以下特殊命令同样由 \biblatex 内核提供。

\begin{ltxsyntax}

\cmditem{nocite}{key}
\cmditem*{nocite}|\{*\}|

%This command is similar to the standard \latex \cmd{nocite} command. It adds the \prm{key} to the bibliography without printing a citation. If the \prm{key} is an asterisk, all entries available in the in-scope bibliography datasource(s) are added to the bibliography. Like all other citation commands, \cmd{nocite} commands in the document body are local to the enclosing \env{refsection} environment, if any. In contrast to standard \latex, \cmd{nocite} may also be used in the document preamble. In this case, the references are assigned to reference section~0.

该命令类似于 \LaTeX\ 中的 \cmd{nocite} 命令。
它将没有引用的条目 \prm{key} 添加到参考文献中。
如果 \prm{key} 是星号,\file{bib} 文件中的所有可用条目都会被添加到参考文献中。
与其它引用命令一样,正文中的 \cmd{nocite} 是在可能的 \env{refsection} 环境中的局部命令。
与标准的\LaTeX\ 不同的是,\cmd{nocite} 也可以在导言区中使用,此时参考文献将归到第零参考文献分节中。

\cmditem{fullcite}[prenote][postnote]{key}

%This command uses the bibliography driver for the respective entry type to create a full citation similar to the bibliography entry. It is thus related to the bibliography style rather than the citation style.

该命令针对相应的条目类型使用参考文献驱动,从而创建类似于文献条目的完整标注(引用)格式。
当然,这样的话就与参考文献著录样式相关而不是与标注(引用)样式相关。

\cmditem{footfullcite}[prenote][postnote]{key}

%Similar to \cmd{fullcite} but puts the entire citation in a footnote and adds a period at the end.

类似于 \cmd{fullcite},但是会将整个引用放在脚注中并在末尾添加句号。

\cmditem{volcite}[prenote]{volume}[page]{key}
\cmditem{Volcite}[prenote]{volume}[page]{key}

%These commands are similar to \cmd{cite} and \cmd{Cite} but intended for references to multi"=volume works which are cited by volume and page number. Instead of the \prm{postnote}, they take a mandatory \prm{volume} and an optional \prm{page} argument. Since they merely compose the postnote and pass it to the \cmd{cite} command provided by the citation style as a \prm{postnote} argument, these commands are style independent. The format of the volume portion is controlled by the field formatting directive \opt{volcitevolume}, the format of the page/text portion is controlled by the field formatting directive \opt{volcitepages} (\secref{aut:fmt:ich}). The delimiter printed between the volume portion and the page/text portion may be modified by redefining the macro \cmd{volcitedelim} (\secref{aut:fmt:fmt}).

这些命令类似于 \cmd{cite} 和 \cmd{Cite},不过用于多卷作品以卷数和页码样式的引用。
与 \prm{postnote} 不同,命令中有一个必选的 \prm{volume} 和一个可选的 \prm{page} 选项。
这些命令实际上是样式无关的,
因为 \prm{volume} 和 \prm{page} 选项构成后注并且将其以 \prm{postnote} 选项传递给由引用样式提供的 \cmd{cite} 命令。
卷数部分的格式由域格式指令 \opt{volcitevolume} 直接控制,
而页码或文本部分的格式由域格式指令 \opt{volcitepages} (\secref{aut:fmt:ich})直接控制。
卷数部分与页码部分之间的定界符可用通过重新定义 \cmd{volcitedelim} 宏(见 \secref{aut:fmt:fmt} 节)来修改。

\cmditem{volcites}(multiprenote)(multipostnote)[prenote]{volume}[page]{key}|\\...|[prenote]{volume}[page]{key}
\cmditem{Volcites}(multiprenote)(multipostnote)[prenote]{volume}[page]{key}|\\...|[prenote]{volume}[page]{key}

%The multicite version of \cmd{volcite} and \cmd{Volcite}, respectively.
\cmd{volcite} 和 \cmd{Volcite} 命令的多重引用版本。

\cmditem{pvolcite}[prenote]{volume}[page]{key}
\cmditem{Pvolcite}[prenote]{volume}[page]{key}

%Similar to \cmd{volcite} but based on \cmd{parencite}.
类似于 \cmd{volcite},不过基于 \cmd{parencite} 命令。

\cmditem{pvolcites}(multiprenote)(multipostnote)[prenote]{volume}[page]{key}|\\...|[prenote]{volume}[page]{key}
\cmditem{Pvolcites}(multiprenote)(multipostnote)[prenote]{volume}[page]{key}|\\...|[prenote]{volume}[page]{key}

%The multicite version of \cmd{pvolcite} and \cmd{Pvolcite}, respectively.
\cmd{pvolcite} 和 \cmd{Pvolcite} 相应的多重引用版本。

\cmditem{fvolcite}[prenote]{volume}[page]{key}
\cmditem{ftvolcite}[prenote]{volume}[page]{key}

%Similar to \cmd{volcite} but based on \cmd{footcite} and \cmd{footcitetext}, respectively.
类似于 \cmd{volcite} 但是分别基于 \cmd{footcite} 和 \cmd{footcitetext}。

\cmditem{fvolcites}(multiprenote)(multipostnote)[prenote]{volume}[page]{key}|\\...|[prenote]{volume}[page]{key}
\cmditem{Fvolcites}(multiprenote)(multipostnote)[prenote]{volume}[page]{key}|\\...|[prenote]{volume}[page]{key}

%The multicite version of \cmd{fvolcite} and \cmd{Fvolcite}, respectively.
\cmd{fvolcite} 和 \cmd{Fvolcite} 相应的多重引用版本。

\cmditem{svolcite}[prenote]{volume}[page]{key}
\cmditem{Svolcite}[prenote]{volume}[page]{key}

%Similar to \cmd{volcite} but based on \cmd{smartcite}.
类似于 \cmd{volcite} 但是基于 \cmd{smartcite}。

\cmditem{svolcites}(multiprenote)(multipostnote)[prenote]{volume}[page]{key}|\\...|[prenote]{volume}[page]{key}
\cmditem{Svolcites}(multiprenote)(multipostnote)[prenote]{volume}[page]{key}|\\...|[prenote]{volume}[page]{key}

%The multicite version of \cmd{svolcite} and \cmd{Svolcite}, respectively.
\cmd{svolcite} 和 \cmd{Svolcite} 相应的多重引用版本。

\cmditem{tvolcite}[prenote]{volume}[page]{key}
\cmditem{Tvolcite}[prenote]{volume}[page]{key}

%Similar to \cmd{volcite} but based on \cmd{textcite}.
类似于 \cmd{volcite} 但是基于 \cmd{textcite}。

\cmditem{tvolcites}(multiprenote)(multipostnote)[prenote]{volume}[page]{key}|\\...|[prenote]{volume}[page]{key}
\cmditem{Tvolcites}(multiprenote)(multipostnote)[prenote]{volume}[page]{key}|\\...|[prenote]{volume}[page]{key}

%The multicite version of \cmd{tvolcite} and \cmd{Tvolcite}, respectively.
\cmd{tvolcite} 和 \cmd{Tvolcite} 相应的多重引用版本。

\cmditem{avolcite}[prenote]{volume}[page]{key}
\cmditem{Avolcite}[prenote]{volume}[page]{key}

%Similar to \cmd{volcite} but based on \cmd{autocite}.
类似于 \cmd{volcite} 但是基于 \cmd{autocite}。

\cmditem{avolcites}(multiprenote)(multipostnote)[prenote]{volume}[page]{key}|\\...|[prenote]{volume}[page]{key}
\cmditem{Avolcites}(multiprenote)(multipostnote)[prenote]{volume}[page]{key}|\\...|[prenote]{volume}[page]{key}

%The multicite version of \cmd{avolcite} and \cmd{Avolcite}, respectively.
\cmd{avolcite} 和 \cmd{Avolcite} 相应的多重引用版本。

\cmditem{notecite}[prenote][postnote]{key}
\cmditem{Notecite}[prenote][postnote]{key}

%These commands print the \prm{prenote} and \prm{postnote} arguments but no citation. Instead, a \cmd{nocite} command is issued for every \prm{key}. This may be useful for authors who incorporate implicit citations in their writing, only giving information not mentioned before in the running text, but who still want to take advantage of the automatic \prm{postnote} formatting and the implicit \cmd{nocite} function. This is a generic, style"=independent citation command. Special citation styles may provide smarter facilities for the same purpose. The capitalized version forces capitalization (note that this is only applicable if the note starts with a command which is sensitive to \biblatex's punctuation tracker).

这些命令打印出 \prm{prenote} 和 \prm{postnote} 选项但是没有引用部分。
与之不同的是,\cmd{nocite} 命令用于每个 \prm{key}。
主要的应用场合是,
用户想要隐式地包含引用条目,但只给出没有在之前行文中提到的信息,
同时仍想利用自动的 \prm{postnote} 格式以及隐式的 \cmd{nocite} 功能。
这是一般的、样式无关的引用命令。
一些特殊的引用样式可以针对相同目的提供更加智能的工具。
其中,\cmd{Notecite} 命令会强制大写
(请注意,只有当注记以 \biblatex 标点追踪器相关的命令开始时,该命令才是可用的)。

\cmditem{pnotecite}[prenote][postnote]{key}
\cmditem{Pnotecite}[prenote][postnote]{key}

%Similar to \cmd{notecite} but the notes are printed in parentheses.
类似于 \cmd{notecite} 但是注记打印在括号中。

\cmditem{fnotecite}[prenote][postnote]{key}

%Similar to \cmd{notecite} but the notes are printed in a footnote.
类似于 \cmd{notecite} 但是注记打印在脚注中。

\end{ltxsyntax}

\subsubsection{底层命令}%\subsubsection{Low-level Commands}
\label{use:cit:low}

%The following commands are also provided by the core of \biblatex. They grant access to all lists and fields at a lower level.
以下命令同样由 \biblatex 内核提供,向所有列表和域提供底层接口。

\begin{ltxsyntax}

\cmditem{citename}[prenote][postnote]{key}[format]{name list}

%The \prm{format} is a formatting directive defined with \cmd{DeclareNameFormat}. Formatting directives are discussed in \secref{aut:bib:fmt}. If this optional argument is omitted, this command falls back to the format \texttt{citename}. The last argument is the name of a \prm{name list}, in the sense explained in \secref{bib:fld}.

\prm{format} 是由 \cmd{DeclareNameFormat} 定义的格式指令。
格式指令的讨论在 \secref{aut:bib:fmt} 节。
如果忽略可选参数,该命令调用备选的 \texttt{citename} 格式。
最后一个选项是 \prm{name list} 的名称(其意义在 \secref{bib:fld} 节中解释)。

\cmditem{citelist}[prenote][postnote]{key}[format]{literal list}

%The \prm{format} is a formatting directive defined with \cmd{DeclareListFormat}. Formatting directives are discussed in \secref{aut:bib:fmt}. If this optional argument is omitted, this command falls back to the format \texttt{citelist}. The last argument is the name of a \prm{literal list}, in the sense explained in \secref{bib:fld}.

\prm{format} 是由 \cmd{DeclareListFormat} 定义的格式指令。
格式指令的讨论在 \secref{aut:bib:fmt} 节。
如果忽略可选参数,该命令调用备选的 \texttt{citelist} 格式。
最后一个选项是 \prm{literal list} 的名称(其意义在 \secref{bib:fld} 节中解释)。

\cmditem{citefield}[prenote][postnote]{key}[format]{field}

%The \prm{format} is a formatting directive defined with \cmd{DeclareFieldFormat}. Formatting directives are discussed in \secref{aut:bib:fmt}. If this optional argument is omitted, this command falls back to the format \texttt{citefield}. The last argument is the name of a \prm{field}, in the sense explained in \secref{bib:fld}.

\prm{format} 是由 \cmd{DeclareFieldFormat} 定义的格式指令。
格式指令的讨论在 \secref{aut:bib:fmt} 一节。
如果忽略可选参数,该命令调用备选的 \texttt{citefield} 格式。
最后一个选项是 \prm{field} 的名称(其意义在 \secref{bib:fld} 节中解释)。

\end{ltxsyntax}

\subsubsection{其它命令}%\subsubsection{Miscellaneous Commands}
\label{use:cit:msc}

%The commands in this section are little helpers related to citations.

本节中的命令是一些与引用相关的小工具。

\begin{ltxsyntax}

\csitem{citereset}

%This command resets the citation style. This may be useful if the style replaces repeated citations with abbreviations like \emph{ibidem}, \emph{idem}, \emph{op. cit.}, etc. and you want to force a full citation at the beginning of a new chapter, section, or some other location. The command executes a style specific initialization hook defined with the \cmd{InitializeCitationStyle} command from \secref{aut:cbx:cbx}. It also resets the internal citation trackers of this package. The reset will affect the \cmd{ifciteseen}, \cmd{ifentryseen}, \cmd{ifciteibid}, and \cmd{ifciteidem} tests discussed in \secref{aut:aux:tst}. When used inside a \env{refsection} environment, the reset of the citation tracker is local to the current \env{refsection} environment. Also see the \opt{citereset} package option in \secref{use:opt:pre:gen}.

该命令重新设置引用样式。
如果样式使用 \emph{ibidem}、\emph{idem}、\emph{op. cit.} 等缩略语代替重复引用,
而你想在新的章节开始或别的地方强制改为引用全称,那么该命令很有用。
该命令执行由 \secref{aut:cbx:cbx} 节的 \cmd{InitializeCitationStyle} 命令所定义的样式相关初始化钩子(hook)。
它同样重新设置了本宏包的内部引用追踪器。
该重置会影响 \secref{aut:aux:tst} 节讨论的 \cmd{ifciteseen}、\cmd{ifentryseen}、\cmd{ifciteibid} 和 \cmd{ifciteidem} 等测试。
当在 \env{refsection} 环境内部使用时,引用追踪器的重置仅局部在当前 \env{refsection} 环境中。
另见 \secref{use:opt:pre:gen} 节的 \opt{citereset} 宏包选项。

\csitem{citereset*}

%Similar to \cmd{citereset} but only executes the style's initialization hook, without resetting the internal citation trackers.

类似于 \cmd{citereset} 但只执行样式初始化钩子,而不重置内部引用追踪器。

\csitem{mancite}

%Use this command to mark manually inserted citations if you mix automatically generated and manual citations. This is particularly useful if the citation style replaces repeated citations by an abbreviation like \emph{ibidem} which may get ambiguous or misleading otherwise. Always use \cmd{mancite} in the same context as the manual citation, \eg if the citation is given in a footnote, include \cmd{mancite} in the footnote. The \cmd{mancite} command executes a style specific reset hook defined with the \cmd{OnManualCitation} command from \secref{aut:cbx:cbx}. It also resets the internal <ibidem> and <idem> trackers of this package. The reset will affect the \cmd{ifciteibid} and \cmd{ifciteidem} tests discussed in \secref{aut:aux:tst}.

如果你想混合自动生成引用和手工引用,可以使用该命令来标记手工插入的引用。
如果引用样式使用 \emph{ibidem} 等可能含糊不清或引起歧义的缩略语代替重复引用,那么该命令特别有用。
应当总是在相同的场合使用 \cmd{mancite} 作为手工引用,
例如,如果引用在脚注中给出,那么在脚注中加入 \cmd{mancite}。
\cmd{mancite} 命令会执行 \secref{aut:cbx:cbx} 节的 \cmd{OnManualCitation} 命令定义的样式相关重置钩子。
它也会重置本宏包的内部“ibidem”和“idem”追踪器。
该重置会影响 \secref{aut:aux:tst} 节讨论的 \cmd{ifciteibid} 和 \cmd{ifciteidem} 等测试。

\csitem{pno}

%This command forces a single page prefix in the \prm{postnote} argument to a citation command. See \secref{use:cav:pag} for further details and usage instructions. Note that this command is only available locally in citations and the bibliography.

该命令在引用命令的 \prm{postnote} 选项中强制开始单页前缀。
更多细节和使用说明见 \secref{use:cav:pag} 节。
请注意,该命令只能用于引用和参考文献中局部。

\csitem{ppno}

%Similar to \cmd{pno} but forces a range prefix. See \secref{use:cav:pag} for further details and usage instructions. Note that this command is only available locally in citations and the bibliography.

类似于 \cmd{pno} 但是强制为区间前缀。
更多细节和使用说明见 \secref{use:cav:pag} 节。
请注意,该命令只能用于引用和参考文献中局部。

\csitem{nopp}

%Similar to \cmd{pno} but suppresses all prefixes. See \secref{use:cav:pag} for further details and usage instructions. Note that this command is only available locally in citations and the bibliography.

类似于 \cmd{pno} 但是抑制所有前缀。
更多细节和使用说明见 \secref{use:cav:pag} 节。
请注意,该命令只能用于引用和参考文献中局部。

\csitem{psq}

%In the \prm{postnote} argument to a citation command, this command indicates a range of two pages where only the starting page is given. See \secref{use:cav:pag} for further details and usage instructions. The suffix printed is the localisation string \texttt{sequens}, see \secref{aut:lng:key}. The spacing inserted between the suffix and the page number may be modified by redefining the macro \cmd{sqspace}. The default is an unbreakable interword space. Note that this command is only available locally in citations and the bibliography.

在引用命令的 \prm{postnote} 选项中,该命令在只给出开始页之处表示两页的范围。
更多细节和使用说明见 \secref{use:cav:pag} 节。
打印出的后缀是本地化字符串 \texttt{sequens}(见 \secref{aut:lng:key} 节)。
在后缀和页码之间的空格可以通过重定义 \cmd{sqspace} 修改。
默认值是一个不可打断的词间空格。
请注意,该命令只能用于引用和参考文献中局部。

\csitem{psqq}

%Similar to \cmd{psq} but indicates an open-ended page range. See \secref{use:cav:pag} for further details and usage instructions. The suffix printed is the localisation string \texttt{sequentes}, see \secref{aut:lng:key}. This command is only available locally in citations and the bibliography.

类似于 \cmd{psq} 但表示一个不包括结束页的页码范围。
更多细节和使用说明见 \secref{use:cav:pag} 节。
打印出的后缀是本地化字符串 \texttt{sequens}(见 \secref{aut:lng:key} 节)。
请注意,该命令只能用于引用和参考文献中局部。

\cmditem{RN}{integer}

%This command prints an integer as an uppercase Roman numeral. The formatting applied to the numeral may be modified by redefining the macro \cmd{RNfont}.

该命令将一个整数打印为大写罗马数字形式。
可以通过重定义 \cmd{RNfont} 宏来修改用于该数值的格式。

\cmditem{Rn}{integer}

%Similar to \cmd{RN} but prints a lowercase Roman numeral. The formatting applied to the numeral may be modified by redefining the macro \cmd{Rnfont}.

类似于 \cmd{RN} 但是打印出小写罗马数字。
可以通过重定义 \cmd{Rnfont} 宏来修改用于该数值的格式。

\end{ltxsyntax}

\subsubsection{\sty{natbib} 兼容命令} %\subsubsection{\sty{natbib} Compatibility Commands}
\label{use:cit:nat}

%The \opt{natbib} package option loads a \sty{natbib} compatibility module. The module defines aliases for the citation commands provided by the \sty{natbib} package. This includes aliases for the core citation commands \cmd{citet} and \cmd{citep} as well as the variants \cmd{citealt} and \cmd{citealp}. The starred variants of these commands, which print the full author list, are also supported. The \cmd{cite} command, which is handled in a particular way by \sty{natbib}, is not treated in a special way. The text commands (\cmd{citeauthor}, \cmd{citeyear}, etc.) are also supported, as are all commands which capitalize the name prefix (\cmd{Citet}, \cmd{Citep}, \cmd{Citeauthor}, etc.). Aliasing with \cmd{defcitealias}, \cmd{citetalias}, and \cmd{citepalias} is possible as well. Note that the compatibility commands will not emulate the citation format of the \sty{natbib} package. They merely alias \sty{natbib}'s commands to functionally equivalent facilities of the \biblatex package. The citation format depends on the main citation style. However, the compatibility style will adapt \cmd{nameyeardelim} to match the default style of the \sty{natbib} package.

本宏包的 \opt{natbib} 选项会导入 \sty{natbib} 兼容性模块。
该模块定义了由 \sty{natbib} 宏包提供的引用命令的别名。
其中包括核心引用命令 \cmd{citet} 和 \cmd{citep} 以及 \cmd{citealt} 和 \cmd{citealp} 等变形命令。
另外也支持这些命令带星号的变种,从而可以打印出完整的作者列表。
\cmd{cite} 命令在 \sty{natbib} 宏包中经过了特殊处理,但在这里不会专门处理。
同时,也支持文本命令(\cmd{citeauthor}、\cmd{citeyear}等)
以及所有首字母大写的命令(\cmd{Citet}、\cmd{Citep}、\cmd{Citeauthor} 等)。
另外,\cmd{defcitealias}、\cmd{citetalias} 和 \cmd{citepalias} 也是可以使用的。
请注意,这些兼容性命令不会模仿 \sty{natbib} 宏包的引用格式,
而仅仅是 \sty{biblatex} 宏包中功能上等价的 \sty{natbib} 命令别称。
引用格式取决于主引用样式。
不过,兼容性样式会调整 \cmd{nameyeardelim} 命令以匹配 \sty{natbib} 宏包的默认样式。

\subsubsection[\sty{mcite} 引用命令]{\sty{mcite} 引用命令}
%\subsubsection[\sty{mcite}-like Citation Commands]{\sty{mcite}-like Citation Commands}
\label{use:cit:mct}

%The \opt{mcite} package option loads a special citation module which provides \sty{mcite}\slash \sty{mciteplus}-like citation commands. Strictly speaking, what the module provides are wrappers for the commands of the main citation style. For example, the following command:

宏包的 \opt{mcite} 选项导入了一个特殊的引用模块,
提供了类似于 \sty{mcite}\slash\sty{mciteplus} 的引用命令。
严格地讲,该模块提供的是主引用样式命令的封装。
例如,以下命令:

\begin{ltxexample}
\mcite{key1,setA,*keyA1,*keyA2,*keyA3,key2,setB,*keyB1,*keyB2,*keyB3}
\end{ltxexample}
%
%is essentially equivalent to this:
本质上等价于

\begin{ltxexample}
\defbibentryset{setA}{keyA1,keyA2,keyA3}%
\defbibentryset{setB}{keyB1,keyB2,keyB3}%
\cite{key1,setA,key2,setB}
\end{ltxexample}
%
%The \cmd{mcite} command will work with any style since the \cmd{cite} backend command is controlled by the main citation style as usual. The \texttt{mcite} module provides wrappers for the standard commands in \secref{use:cit:std,use:cit:cbx}. See \tabref{use:cit:mct:tab2} for an overview. Pre and postnotes as well as starred variants of all commands are also supported. The parameters will be passed to the backend command. For example:
由于 \cmd{cite} 后端命令和通常一样由主引用样式控制,因此 \cmd{mcite} 命令可以和任何样式兼容。
\texttt{mcite} 模块提供了 \secref{use:cit:std,use:cit:cbx} 中标准命令的封装,
见\tabref{use:cit:mct:tab2}。
它也支持前注和后注以及所有命令带星号的版本,
其参数会传递到后端命令中。例如:

\begin{ltxexample}
\mcite*[pre][post]{setA,*keyA1,*keyA2,*keyA3}
\end{ltxexample}
%
%will execute:
会执行:

\begin{ltxexample}
\defbibentryset{setA}{keyA1,keyA2,keyA3}%
\cite*[pre][post]{setA}
\end{ltxexample}
%
%Note that the \texttt{mcite} module is not a compatibility module. It provides commands which are very similar but not identical in syntax and function to \sty{mcite}'s commands. When migrating from \sty{mcite}\slash\sty{mciteplus} to \biblatex, legacy files must be updated. With \sty{mcite}, the first member of the citation group is also the identifier of the group as a whole. Borrowing an example from the \sty{mcite} manual, this group:
请注意,\texttt{mcite} 模块不是完全兼容的。
它提供的命令与 \sty{mcite} 宏包的命令在语法和功能上十分相似但并不完全等价。
当从 \sty{mcite}\slash\sty{mciteplus} 迁移到 \biblatex 时必须更新旧文件。
在 \sty{mcite} 中,引用组的第一个成员也是该组整个的标识符。
举个 \sty{mcite} 手册中的例子,以下的组:

\begin{table}
\tablesetup
\begin{tabular}{@{}V{0.5\textwidth}@{}V{0.5\textwidth}@{}}
\toprule
%\multicolumn{1}{@{}H}{Standard Command} &
%\multicolumn{1}{@{}H}{\sty{mcite}-like Command} \\
\multicolumn{1}{@{}H}{标准命令} &
\multicolumn{1}{@{}H}{\sty{mcite} 命令} \\
\cmidrule(r){1-1}\cmidrule{2-2}
|\cite|		& |\mcite| \\
|\Cite|		& |\Mcite| \\
|\parencite|	& |\mparencite| \\
|\Parencite|	& |\Mparencite| \\
|\footcite|	& |\mfootcite| \\
|\footcitetext|	& |\mfootcitetext| \\
|\textcite|	& |\mtextcite| \\
|\Textcite|	& |\Mtextcite| \\
|\supercite|	& |\msupercite| \\
\bottomrule
\end{tabular}
%\caption{\sty{mcite}-like commands}
\caption{类 \sty{mcite} 命令}
\label{use:cit:mct:tab1}
\end{table}

\begin{ltxexample}
\cite{<<glashow>>,*salam,*weinberg}
\end{ltxexample}
%
%consists of three entries and the entry key of the first one also serves as identifier of the entire group. In contrast to that, a \biblatex entry set is an entity in its own right. Therefore, it requires a unique entry key which is assigned to the set as it is defined:
包括三个条目,且第一个条目键也同样是整个组的标识符。
与此相反,\biblatex 条目集是一个实体。因此,它需要分配给该集的唯一的条目键值:

\begin{ltxexample}
\mcite{<<set1>>,*glashow,*salam,*weinberg}
\end{ltxexample}
%
%Once defined, an entry set is handled like any regular entry in a \file{bib} file. When using one of the \texttt{numeric} styles which ship with \texttt{biblatex} and activating its \opt{subentry} option, it is even possible to refer to set members. See \tabref{use:cit:mct:tab2} for some examples. Restating the original definition of the set is redundant, but permissible. In contrast to \sty{mciteplus}, however, restating a part of the original definition is invalid. Use the entry key of the set instead.
一旦定义之后,条目集的处理与其它 \file{bib} 文件中的常规条目相同。
当在 \sty{biblatex} 中使用 \texttt{numeric} 样式并且启动了 \opt{subentry} 选项时,甚至可以指向集成员。
参考\tabref{use:cit:mct:tab2} 中的一些例子。
也可以重启条目集的原始定义,但这没有必要。
不过与 \sty{mciteplus} 不同的是,重启原始定义的一部分是无效的,需要使用集合的条目键。

\begin{table}
\tablesetup
\begin{tabular}{@{}V{0.5\textwidth}@{}V{0.1\textwidth}@{}p{0.4\textwidth}@{}}
\toprule
%\multicolumn{1}{@{}H}{Input} &
%\multicolumn{1}{@{}H}{Output} &
%\multicolumn{1}{@{}H}{Comment} \\
\multicolumn{1}{@{}H}{输入} &
\multicolumn{1}{@{}H}{输出} &
\multicolumn{1}{@{}H}{评注} \\
\cmidrule(r){1-1}\cmidrule(r){2-2}\cmidrule{3-3}
%|\mcite{set1,*glashow,*salam,*weinberg}|& [1]	& Defining and citing the set \\
%|\mcite{set1}|				& [1]	& Subsequent citation of the set \\
%|\cite{set1}|				& [1]	& Regular |\cite| works as usual \\
%|\mcite{set1,*glashow,*salam,*weinberg}|& [1]	& Redundant, but permissible \\
%|\mcite{glashow}|			& [1a]	& Citing a set member \\
%|\cite{weinberg}|			& [1c]	& Regular |\cite| works as well \\
|\mcite{set1,*glashow,*salam,*weinberg}|& [1]	& 定义并引用该集合 \\
|\mcite{set1}|				& [1]	& 该集合随后的引用 \\
|\cite{set1}|				& [1]	& 常规的 |\cite| \\
|\mcite{set1,*glashow,*salam,*weinberg}|& [1]	& 冗余,但是允许的 \\
|\mcite{glashow}|			& [1a]	& 引用集成员 \\
|\cite{weinberg}|			& [1c]	& 又一次常规的 |\cite| \\
\bottomrule
\end{tabular}
%\caption[\sty{mcite}-like syntax]
%{\sty{mcite}-like syntax (sample output with \kvopt{style}{numeric} and \opt{subentry} option)}
\caption[类\sty{mcite} 语法]{\sty{mcite} 语法(使用 \kvopt{style}{numeric} 和 \opt{subentry} 选项时的样例输出)}
\label{use:cit:mct:tab2}
\end{table}

\subsection{本地化命令}%\subsection{Localization Commands}
\label{use:lng}

%The \biblatex package provides translations for key terms such as <edition> or <volume> as well as definitions for language specific features such as the date format and ordinals. These definitions, which are loaded automatically, may be modified or extended in the document preamble or the configuration file with the commands introduced in this section.

\biblatex 宏包提供了“edition”和“volume”等关键词的翻译,
并定义日期格式和序数等语言相关的特征。
这些定义是自动导入的,可以通过本节所介绍的命令在导言区或配置文件中修改或扩展。

\begin{ltxsyntax}

\cmditem{DefineBibliographyStrings}{language}{definitions}

%This command is used to define localisation strings. The \prm{language} must be a language name known to the \sty{babel}/\sty{polyglossia} packages, \ie one of the identifiers listed in \tabref{bib:fld:tab1} on page \pageref{bib:fld:tab1}. The \prm{definitions} are \keyval pairs which assign an expression to an identifier:

该命令用于定义本地化字符串。
\prm{language} 选项必须是 \sty{babel}/\sty{polyglossia} 可知的语言名,
即 \pageref{bib:fld:tab1} 页的\tabref{bib:fld:tab1} 所列出的标识符。
\prm{definitions} 是 \keyval 对,将表达式分配给标识符:

\begin{ltxexample}
\DefineBibliographyStrings{american}{%
  bibliography  = {Bibliography},
  shorthands    = {Abbreviations},
  editor        = {editor},
  editors       = {editors},
}
\end{ltxexample}
%
%A complete list of all keys supported by default is given is \secref{aut:lng:key}. Note that all expressions should be capitalized as they usually are when used in the middle of a sentence. The \biblatex package will automatically capitalize the first word when required at the beginning of a sentence. Expressions intended for use in headings should be capitalized in a way that is suitable for titling. In contrast to \cmd{DeclareBibliographyStrings}, \cmd{DefineBibliographyStrings} overrides both the full and the abbreviated version of the string. See \secref{aut:lng:cmd} for further details.
默认支持的所有关键字列表在 \secref{aut:lng:key} 节中给出。
请注意,所有的表达式在句子中间时也要像本来一样首字母大写。
\biblatex 宏包会在必要时自动在句首处将首字母大写。
用于标题的表达式的大写方式要与标题匹配。
与 \cmd{DeclareBibliographyStrings} 相反,\cmd{DefineBibliographyStrings} 覆盖了字符串及其缩写两个版本。
详见 \secref{aut:lng:cmd} 节。

\cmditem{DefineBibliographyExtras}{language}{code}

%This command is used to adapt language specific features such as the date format and ordinals. The \prm{language} must be a language name known to the \sty{babel}/\sty{polyglossia} packages. The \prm{code}, which may be arbitrary \latex code, will usually consist of redefinitions of the formatting commands from \secref{use:fmt:lng}.

该命令用于调整日期格式和序数等语言相关的特征。
\prm{language} 必须是 \sty{babel}/\sty{polyglossia} 可知的语言名。
\prm{code} 可以是任意代码,通常包括 \secref{use:fmt:lng} 中格式命令的重定义。

\cmditem{UndefineBibliographyExtras}{language}{code}

%This command is used to restore the original definition of any commands modified with \cmd{DefineBibliographyExtras}. If a redefined command is included in \secref{use:fmt:lng}, there is no need to restore its previous definition since these commands are adapted by all language modules anyway.

该命令用于存储被 \cmd{DefineBibliographyExtras} 修改的命令的原始定义。
如果重定义的命令在 \secref{use:fmt:lng} 中,
那么没有必要存储之前的定义,因为这些命令总归会被所有的语言模块调整的。

\cmditem{DefineHyphenationExceptions}{language}{text}

%This is a \latex frontend to \tex's \cmd{hyphenation} command which defines hyphenation exceptions.
The \prm{language} must be a language name known to the \sty{babel}/\sty{polyglossia} packages. The \prm{text} is a whitespace"=separated list of words. Hyphenation points are marked with a dash:

这是 \TeX{} 的 \cmd{hyphenation} 命令的 \LaTeX 前端,用于定义了断词例外。
\prm{language} 必须是 \sty{babel}/\sty{polyglossia} 可知的语言名。
\prm{text} 是空格分开的单词列表,断词点用短横线标记:

\begin{ltxexample}
\DefineHyphenationExceptions{american}{%
  hy-phen-ation ex-cep-tion
}
\end{ltxexample}

\cmditem{NewBibliographyString}{key}

%This command declares new localisation strings, \ie it initializes a new \prm{key} to be used in the
\prm{definitions} of \cmd{DefineBibliographyStrings}. The \prm{key} argument may also be a comma"=separated list of key names. The keys listed in \secref{aut:lng:key} are defined by default.
该命令声明了新的本地化字符串,即初始化新的 \prm{key},用在命令 \cmd{DefineBibliographyStrings} 的 \prm{definitions} 中。选项 \prm{key} 也可以是逗号分隔的键值名列表。默认的键名在 \secref{aut:lng:key} 节中列出。
\end{ltxsyntax}

\subsection{条目查询命令}%\subsection{Entry Querying Commands}
\label{use:eq}

%The commands in this section are user-facing equivalents of the identically-named commands in section \secref{aut:aux:tst}. They can be used to test for the presence and attributes of specific bibliography entries. See section \secref{aut:aux:tst} for usage.
本节中的命令是面向用户,它们等价于\secref{aut:aux:tst}节中同名命令。可以用于测试具体文献条目的存在性及其它属性。用法见\secref{aut:aux:tst}节。

\begin{ltxsyntax}
\cmditem{ifentryseen}{entrykey}{true}{false}
\cmditem{ifentryinbib}{entrykey}{true}{false}
\cmditem{ifentrycategory}{entrykey}{category}{true}{false}
\cmditem{ifentrykeyword}{entrykey}{keyword}{true}{false}
\end{ltxsyntax}


\subsection{格式命令}%\subsection{Formatting Commands}
\label{use:fmt}

%The commands and facilities presented in this section may be used to adapt the format of citations and the bibliography.

本节介绍的命令和工具可以用于调整引用和参考文献的格式。

\subsubsection{一般命令和钩子} %\subsubsection{Generic Commands and Hooks}
\label{use:fmt:fmt}

%The commands in this section may be redefined with \cmd{renewcommand} in the document preamble. Those marked as <Context Sensitive> in the margin can also be customised with \cmd{DeclareDelimFormat} and are printed with \cmd{printdelim} (\secref{use:fmt:csd}). Note that all commands starting with \cmd{mk\dots} take one argument. All of these commands are defined in \path{biblatex.def}.

本小节的命令可以在导言区用 \cmd{renewcommand} 重定义。
页边标记为“Context Sensitive”的命令还可以用 \cmd{DeclareDelimFormat} 进行定制,
并使用 \secref{use:fmt:csd} 节的 \cmd{printdelim} 打印。
请注意,所有以 \cmd{mk\dots} 开头的命令都带有一个选项。
所有这些命令的定义在 \path{biblatex.def} 中。

\begin{ltxsyntax}

\csitem{bibsetup}
%Arbitrary code to be executed at the beginning of the bibliography, intended for commands which affect the layout of the bibliography.
在参考文献开始执行的任意代码,用于影响参考文献的页面布局的命令。

\csitem{bibfont}
%Arbitrary code setting the font used in the bibliography. This is very similar to \cmd{bibsetup} but intended for switching fonts.
在参考文献中用于设置字体的任意代码。
类似于 \cmd{bibsetup} 但用于切换字体。

\csitem{citesetup}
%Arbitrary code to be executed at the beginning of each citation command.
在引用命令开始执行的任意代码。

\csitem{newblockpunct}
%The separator inserted between <blocks> in the sense explained in \secref{aut:pct:new}. The default definition is controlled by the package option \opt{block} (see \secref{use:opt:pre:gen}).
插入在 <blocks>(其意义在 \secref{aut:pct:new} 节中解释)之间的分隔符。
缺省定义由宏包选项 \opt{block}(见 \secref{use:opt:pre:gen})所控制。

\csitem{newunitpunct}
%The separator inserted between <units> in the sense explained in \secref{aut:pct:new}. This will usually be a period or a comma plus an interword space. The default definition is a period and a space.
插入在 <units>(其意义在\secref{aut:pct:new} 节中解释)之间的分隔符。
这通常是句号或者逗号加上一个词间距。
缺省定义是句号加一个空格。

\csitem{finentrypunct}
%The punctuation printed at the very end of every bibliography entry, usually a period. The default definition is a period.
在每条文献条目最后打印的标点,通常是句号。
缺省定义是句号。

\csitem{entrysetpunct}
%The punctuation printed between bibliography subentries of an entry set. The default definition is a semicolon and a space.
一个条目集中文献子条目之间的标点。
默认定义是分号和一个空格。

\csitem{bibnamedelima}
%This delimiter controls the spacing between the elements which make up a name part. It is inserted automatically after the first name element if the element is less than three characters long and before the last element. The default definition is an interword space penalized by the value of the \cnt{highnamepenalty} counter (\secref{use:fmt:len}). Please refer to \secref{use:cav:nam} for further details.
该定界符控制组成姓名成分之间的空白。
它将自动插入在 first name 之后(如果该成分少于三个字符长度)和最后的姓名成分之前。
缺省值是一个词间距加上由计数器 \cnt{highnamepenalty}(\secref{use:fmt:len})的值控制的惩罚项。
详见 \secref{use:cav:nam} 节。

\csitem{bibnamedelimb}
%This delimiter is inserted between the elements which make up a name part where \cmd{bibnamedelima} does not apply. The default definition is an interword space penalized by the value of the \cnt{lownamepenalty} counter (\secref{use:fmt:len}). Please refer to \secref{use:cav:nam} for further details.
该定界符插入在 \cmd{bibnamedelima} 不可用的姓名成分之间。
缺省值是一个词间距加上由计数器 \cnt{lownamepenalty}(\secref{use:fmt:len})的值控制的惩罚项。
详见 \secref{use:cav:nam} 节。

\csitem{bibnamedelimc}
%This delimiter controls the spacing between name parts. It is inserted between the name prefix and the last name if \kvopt{useprefix}{true}. The default definition is an interword space penalized by the value of the \cnt{highnamepenalty} counter (\secref{use:fmt:len}). Please refer to \secref{use:cav:nam} for further details.
该定界符控制姓名成分之间的空白。
如果 \kvopt{useprefix}{true},那么它插入在名前缀(name prefix)和姓(last name)之间。
缺省值是一个词间距加上由计数器 \cnt{highnamepenalty}(\secref{use:fmt:len})的值控制的惩罚项。
详见 \secref{use:cav:nam} 节。

\csitem{bibnamedelimd}
%This delimiter is inserted between all name parts where \cmd{bibnamedelimc} does not apply. The default definition is an interword space penalized by the value of the \cnt{lownamepenalty} counter (\secref{use:fmt:len}). Please refer to \secref{use:cav:nam} for further details.
该定界符插入在所有的 \cmd{bibnamedelimc} 不可用的姓名成分之间。
缺省值是一个词间距加上由计数器 \cnt{lownamepenalty}(\secref{use:fmt:len})的值控制的惩罚项。
详见 \secref{use:cav:nam} 节。

\csitem{bibnamedelimi}
%This delimiter replaces \cmd{bibnamedelima/b} after initials. Note that this only applies to initials given as such in the \file{bib} file, not to the initials automatically generated by \biblatex which use their own set of delimiters.
该定界符在首字符缩写中取代 \cmd{bibnamedelima/b}。
请注意该情况只用于既定的首字母缩写(例如在 \file{bib} 文件中给出的),
而不是由 \biblatex 自动生成的首字母缩写,后者会使用自己的定界符集。

\csitem{bibinitperiod}
%The punctuation inserted after initials unless \cmd{bibinithyphendelim} applies. The default definition is a period (\cmd{adddot}). Please refer to \secref{use:cav:nam} for further details.
当没有使用 \cmd{bibinithyphendelim} 时插入在首字母缩写之后的标点。
缺省值是句点(\cmd{adddot})。
详见 \secref{use:cav:nam} 节。

\csitem{bibinitdelim}
%The spacing inserted between multiple initials unless \cmd{bibinithyphendelim} applies. The default definition is an unbreakable interword space. Please refer to \secref{use:cav:nam} for further details.
当没有使用 \cmd{bibinithyphendelim} 时多重首字母缩写之间的空白。
缺省值是一个不可打断的词间距。
详见 \secref{use:cav:nam} 节。

\csitem{bibinithyphendelim}
%The punctuation inserted between the initials of hyphenated name parts, replacing \cmd{bibinitperiod} and \cmd{bibinitdelim}. The default definition is a period followed by an unbreakable hyphen. Please refer to \secref{use:cav:nam} for further details.
带连字符的姓名成分首字母缩写之间插入的标点,用以替代 \cmd{bibinitperiod} 和 \cmd{bibinitdelim}。
缺省定义是一个句点后接一个不可打断的连字符。
详见 \secref{use:cav:nam} 节。

\csitem{bibindexnamedelima}
%Replaces \cmd{bibnamedelima} in the index.
在索引中代替 \cmd{bibnamedelima}。

\csitem{bibindexnamedelimb}
%Replaces \cmd{bibnamedelimb} in the index.
在索引中代替 \cmd{bibnamedelimb}。

\csitem{bibindexnamedelimc}
%Replaces \cmd{bibnamedelimc} in the index.
在索引中代替 \cmd{bibnamedelimc}。

\csitem{bibindexnamedelimd}
%Replaces \cmd{bibnamedelimd} in the index.
在索引中代替 \cmd{bibnamedelimd}。

\csitem{bibindexnamedelimi}
%Replaces \cmd{bibnamedelimi} in the index.
在索引中代替 \cmd{bibnamedelimi}。

\csitem{bibindexinitperiod}
%Replaces \cmd{bibinitperiod} in the index.
在索引中代替 \cmd{bibinitperiod}。

\csitem{bibindexinitdelim}
%Replaces \cmd{bibinitdelim} in the index.
在索引中代替 \cmd{bibinitdelim}。

\csitem{bibindexinithyphendelim}
%Replaces \cmd{bibinithyphendelim} in the index.
在索引中代替 \cmd{bibinithyphendelim}。

\csitem{revsdnamepunct}
%The punctuation to be printed between the first and last name parts when a name is reversed. Here is an example showing a name with the default comma as \cmd{revsdnamedelim}:
当名字反写(姓在前)时姓和名之间的标点。
如下是一个使用缺省的逗号作为 \cmd{revsdnamedelim} 的例子:

\begin{ltxexample}
Jones<<,>> Edward
\end{ltxexample}

%This command should be used with \cmd{bibnamedelimd} as a reversed-name separator in formatting directives for name lists. Please refer to \secref{use:cav:nam} for further details.
对于姓名列表,该命令应当与格式指令中的姓名反写分隔符 \cmd{bibnamedelimd} 一起使用。
详见 \secref{use:cav:nam} 节。

\csitem{bibnamedash}
%The dash to be used as a replacement for recurrent authors or editors in the bibliography. The default is an <em> or an <en> dash, depending on the indentation of the list of references.
用于替代文献中连续重复的作者或编辑的破折号。
缺省值是一个 <em> 或 <en> 长度横线,取决于文献列表的缩进。

\csitem{labelnamepunct}
%The separator printed after the name used for alphabetizing in the bibliography (\bibfield{author} or \bibfield{editor}, if the \bibfield{author} field is undefined). With the default styles, this separator replaces \cmd{newunitpunct} at this location. The default definition is \cmd{newunitpunct}, \ie it is not handled differently from regular unit punctuation.
在参考文献按字母排序时用在名字之后的分隔符(\bibfield{author},在其未定义时是 \bibfield{editor})。
对于缺省样式,该分隔符在此位置代替 \cmd{newunitpunct}。
缺省定义是 \cmd{newunitpunct},即对其处理与常规单元标点没有不同之处。

\csitem{subtitlepunct}
%The separator printed between the fields \bibfield{title} and \bibfield{subtitle}, \bibfield{booktitle} and \bibfield{booksubtitle}, as well as \bibfield{maintitle} and \bibfield{mainsubtitle}. With the default styles, this separator replaces \cmd{newunitpunct} at this location. The default definition is \cmd{newunitpunct}, \ie it is not handled differently from regular unit punctuation.
域 \bibfield{title} 和 \bibfield{subtitle}、\bibfield{booktitle} 和 \bibfield{booksubtitle},以及 \bibfield{maintitle} 和 \bibfield{mainsubtitle} 之间的分隔符。
对于缺省样式,该分隔符在此位置代替 \cmd{newunitpunct}。
缺省定义是 \cmd{newunitpunct},即对其处理与常规单元标点没有不同之处。

\csitem{intitlepunct}
%The separator between the word «in» and the following title in entry types such as \bibtype{article}, \bibtype{inbook}, \bibtype{incollection}, etc. The default definition is a colon plus an interword space (\eg «Article, in: \emph{Journal}» or «Title, in: \emph{Book}»). Note that this is the separator string, not only the punctuation mark. If you don't want a colon after «in», \cmd{intitlepunct} should still insert a space.
在 \bibtype{article}, \bibtype{inbook}, \bibtype{incollection} 等条目类型中单词“in”与之后的标题之间的分隔符。
缺省值是一个冒号加上一个词间距(例如,“Article, in: \emph{Journal}” 或者“Title, in: \emph{Book}”)。
请注意,这是分隔字符串而不仅是标点。
如果你不想在“in”后加冒号,\cmd{intitlepunct} 仍然应当插入一个空格。

\csitem{bibpagespunct}
%The separator printed before the \bibfield{pages} field. The default is a comma plus an interword space.
域 \bibfield{pages} 之前的分隔符。
缺省值是逗号加上一个词间距。

\csitem{bibpagerefpunct}
%The separator printed before the \bibfield{pageref} field. The default is an interword space.
域 \bibfield{pageref} 之前的分隔符。
缺省值是一个词间距。

\csitem{multinamedelim}\CSdelimMark
%The delimiter printed between multiple items in a name list like \bibfield{author} or \bibfield{editor} if there are more than two names in the list. The default is a comma plus an interword space. See \cmd{finalnamedelim} for an example.\footnote{Note that \cmd{multinamedelim} is not used at all if there are only two names in the list. In this case, the default styles use the \cmd{finalnamedelim}.}
在 \bibfield{author} 或 \bibfield{editor} 等姓名列表中各项之间的定界符(如果有两个以上姓名)。
缺省值是一个逗号加上一个词间距。
例子请参考 \cmd{finalnamedelim}。\footnote{%
	请注意,如果列表中只有两个姓名,那么不会使用 \cmd{multinamedelim},此时缺省样式使用 \cmd{finalnamedelim}。
}

\csitem{finalnamedelim}\CSdelimMark
%The delimiter printed instead of \cmd{multinamedelim} before the final name in a name list. The default is the localised term <and>, separated by interword spaces. Here is an example:
在姓名列表的最后一个姓名之前用以代替 \cmd{multinamedelim} 的定界符。
缺省值是用词间距分隔的本地化项“and”。这里是一个例子:

\begin{ltxexample}
Michel Goossens<<,>> Frank Mittelbach <<and>> Alexander Samarin
Edward Jones <<and>> Joe Williams
\end{ltxexample}
%
%The comma in the first example is the \cmd{multinamedelim} whereas the string <and> in both examples is the \cmd{finalnamedelim}. See also \cmd{finalandcomma} in \secref{use:fmt:lng}.
第一个例子中的逗号是 \cmd{multinamedelim},而这两个例子中的字符串“and”是 \cmd{finalnamedelim}。
另见 \secref{use:fmt:lng} 节中的 \cmd{finalandcomma}。

\csitem{revsdnamedelim}\CSdelimMark
%An extra delimiter printed after the first name in a name list if the first name is reversed. The default is an empty string, \ie no extra delimiter will be printed. Here is an example showing a name list with a comma as \cmd{revsdnamedelim}:
当姓名反序时打印在名(first name)后面的额外定界符。
缺省值是空字符串,即没有额外的定界符。
这里是一个设置 \cmd{revsdnamedelim} 为逗号的姓名列表例子:

\begin{ltxexample}
Jones, Edward<<, and>> Joe Williams
\end{ltxexample}
%
%In this example, the comma after <Edward> is the \cmd{revsdnamedelim} whereas the string <and> is the \cmd{finalnamedelim}, printed in addition to the former.
在本例中,“Edward”之后的逗号是 \cmd{revsdnamedelim} 而字符串“and”是前者之后的 \cmd{finalnamedelim}。

\csitem{andothersdelim}\CSdelimMark
%The delimiter printed before the localisation string <\texttt{andothers}> if a name list like \bibfield{author} or \bibfield{editor} is truncated. The default is an interword space.
当 \bibfield{author} 或 \bibfield{editor} 等姓名列表被截断时本地化字符串“\texttt{andothers}” 之前的定界符。
缺省值是一个词间距。

\csitem{multilistdelim}\CSdelimMark
%The delimiter printed between multiple items in a literal list like \bibfield{publisher} or \bibfield{location} if there are more than two items in the list. The default is a comma plus an interword space. See \cmd{multinamedelim} for further explanation.
在 \bibfield{publisher} 或 \bibfield{location} 等文本列表中诸项之间的定界符(如果列表中有两个以上项)。
缺省值是一个逗号加上一个词间距。
进一步解释参见 \cmd{multinamedelim}。

\csitem{finallistdelim}\CSdelimMark
%The delimiter printed instead of \cmd{multilistdelim} before the final item in a literal list. The default is the localised term <and>, separated by interword spaces. See \cmd{finalnamedelim} for further explanation.
在文本列表的最后一项之前代替 \cmd{multilistdelim} 的定界符。
缺省值是用词间距分隔的本地化字符串“and”。
进一步解释参见 \cmd{finalnamedelim}。

\csitem{andmoredelim}\CSdelimMark
%The delimiter printed before the localisation string <\texttt{andmore}> if a literal list like \bibfield{publisher} or \bibfield{location} is truncated. The default is an interword space.
当 \bibfield{publisher} 或 \bibfield{location} 等文本列表被截断时打印在本地化字符串“\texttt{andmore}”之前的定界符。
缺省值是一个词间距。

\csitem{multicitedelim}
%The delimiter printed between citations if multiple entry keys are passed to a single citation command. The default is a semicolon plus an interword space.
当多个条目键传递给单个引用命令时打印在引用之间的定界符。
缺省值是一个分号加上一个词间距。

\csitem{supercitedelim}
%Similar to \cmd{multicitedelim}, but used by the \cmd{supercite} command only. The default is a comma.
类似于 \cmd{multicitedelim} 但只用在 \cmd{supercite} 中。缺省值是一个逗号。

\csitem{compcitedelim}
%Similar to \cmd{multicitedelim}, but used by certain citation styles when <compressing> multiple citations. The default definition is a comma plus an interword space.
类似于 \cmd{multicitedelim} 但只用于某些“压缩”的多重引用样式。
缺省值是一个逗号加上一个词间距。

\csitem{datecircadelim}\CSdelimMark
%When formatting dates with the global option \opt{datecirca} enabled, the delimiter printed after any localised <circa> term. Defaults to interword space.
开启全局选项 \opt{datecirca} 时,在日期格式中本地化“circa”项之后的定界符。
缺省值是一个词间距。

\csitem{dateeradelim}\CSdelimMark
%When formatting dates with the global option \opt{dateera} set, the delimiter printed before the localisation era term. Defaults to interword space.
设置全局选项 \opt{dateera} 时,在日期格式中本地化纪年项之前的定界符。
缺省值是一个词间距。

\csitem{dateuncertainprint}
%Prints date uncertainty information when the global option \opt{dateuncertain} is enabled and the \cmd{ifdateuncertain} test is true. By default, prints the language specific \cmd{bibdateuncertain} string (\secref{use:fmt:lng}).
开启全局选项 \opt{dateuncertain} 并且 \cmd{ifdateuncertain} 测试为真时打印的日期不确定信息。
缺省打印语言相关的 \cmd{bibdateuncertain} 字符串(\secref{use:fmt:lng} 节)。

\csitem{enddateuncertainprint}
%Prints date uncertainty information when the global option \opt{dateuncertain} is enabled and the \cmd{ifenddateuncertain} test is true. By default, prints the language specific \cmd{bibdateuncertain} string (\secref{use:fmt:lng}).
开启全局选项 \opt{dateuncertain} 并且 \cmd{ifenddateuncertain} 测试为真时打印的日期不确定信息。
缺省打印语言相关的 \cmd{bibdateuncertain} 字符串(\secref{use:fmt:lng} 节)。

\csitem{datecircaprint}
%Prints date circa information when the global option \opt{datecirca} is enabled and the \cmd{ifdatecirca} test is true. By default, prints the <circa> localised term (\secref{aut:lng:key:dt}) and the \opt{datecircadelim} delimiter.
开启全局选项 \opt{datecirca} 并且 \cmd{ifdatecirca} 测试为真时打印日期约数信息。
缺省打印本地化的“circa”项(\secref{aut:lng:key:dt} 节)和 \opt{datecircadelim} 定界符。

\csitem{enddatecircaprint}
%Prints date circa information when the global option \opt{datecirca} is enabled and the \cmd{ifenddatecirca} test is true. By default, prints the <circa> localised term (\secref{aut:lng:key:dt}) and the \opt{datecircadelim} delimiter.
开启全局选项 \opt{datecirca} 并且 \cmd{ifenddatecirca} 测试为真时打印日期约数信息。
缺省打印本地化的“circa”项(\secref{aut:lng:key:dt} 节)和 \opt{datecircadelim} 定界符。

\csitem{datecircaprintiso}
%Prints \acr{ISO8601-2} format date circa information when the global option \opt{datecirca} is enabled and the \cmd{ifdatecirca} test is true. Prints \cmd{textasciitilde}.
开启全局选项 \opt{datecirca} 并且 \cmd{ifdatecirca} 测试为真时打印
\acr{ISO8601-2} 格式的日期约数信息。缺省打印 \cmd{textasciitilde}。

\csitem{enddatecircaprintiso}
%Prints \acr{ISO8601-2} format date circa information when the global option \opt{datecirca} is enabled and the \cmd{ifenddatecirca} test is true. Prints \cmd{textasciitilde}.
开启全局选项 \opt{datecirca} 并且 \cmd{ifenddatecirca} 测试为真时打印
\acr{ISO8601-2} 格式的日期约数信息。
缺省打印 \cmd{textasciitilde}。

\csitem{dateeraprint}{yearfield}
%Prints date era information when the global option \opt{dateera} is set to <secular> or <christian>. By default, prints the \opt{dateeradelim} delimiter and the appropriate localised era term (\secref{aut:lng:key:dt}). If the \opt{dateeraauto} option is set, then the passed \prm{yearfield} (which is the name of a year field such as <year>, <origyear>, <endeventyear> etc.) is tested to see if its value is earlier than the \opt{dateeraauto} threshold and if so, then the BCE/CE localisation will be output too. The default setting for \opt{dateeraauto} is 0 and so only BCE/BC localisation strings are candidates for output. Detects whether the start or end year era information is to be printed by looking at the \prm{yearfield} name passed to it.
当全局选项 \opt{dateera} 设置为 \opt{secular} 或 \opt{christian} 时打印的日期纪年信息。
缺省情况下打印 \opt{dateeradelim} 定界符和合适的本地化纪年词语(\secref{aut:lng:key:dt} 节)。
如果设置了 \opt{dateeraauto},那么将测试传递过来的 \prm{yearfield}
(\opt{year}、\opt{origyear}、\opt{endeventyear} 等年份域的名称)
以确定对应的值是否早于 \opt{dateeraauto} 阈值。
如果更早的话还会输出本地化的 BCE/CE 词语。
\opt{dateeraauto} 的缺省设置为0,这样只会输出 BCE/BC 本地化字符串。
通过探测传递来的 \prm{yearfield} 名称确定是否打印开始和结束年份的纪年信息。

\csitem{dateeraprintpre}
%Prints date era information when the global option \opt{dateera} is set to <astronomical>. By default, prints \opt{bibdataeraprefix}. Detects whether the start or end year era information is to be printed by looking at the \prm{yearfield} name passed to it.
当全局选项 \opt{dateera} 设置为 \opt{astronomical} 时打印的日期约数信息。
缺省值是 \opt{bibdataeraprefix}。
通过探测传递来的 \prm{yearfield} 名称确定是否打印开始和结束年份的纪年信息。

\csitem{textcitedelim}
%Similar to \cmd{multicitedelim}, but used by \cmd{textcite} and related commands (\secref{use:cit:cbx}). The default is a comma plus an interword space. The standard styles modify this provisional definition to ensure that the delimiter before the final citation is the localised term <and>, separated by interword spaces. See also \cmd{finalandcomma} and \cmd{finalandsemicolon} in \secref{use:fmt:lng}.
类似于 \cmd{multicitedelim} 但用于 \cmd{textcite} 和相关命令(\secref{use:cit:cbx} 节)。
缺省值是一个逗号加上一个词间距。
标准样式会修改这一临时定义以确保最后一个引用之前的定界符是词间距分隔的本地化字符串“and”。
另见 \secref{use:fmt:lng} 节的 \cmd{finalandcomma} 和 \cmd{finalandsemicolon}。

\csitem{nametitledelim}\CSdelimMark
%The delimiter printed between the author\slash editor and the title by author-title and some verbose citation styles. The default definition inside bibliographies is \cmd{labelnamepunct} and is a comma plus an interword space otherwise.
作者---标题和其它一些详细引用样式中作者/编辑和标题之间的定界符。
缺省定义是一个逗号加上一个词间距。

\csitem{nameyeardelim}\CSdelimMark
%The delimiter printed between the author\slash editor and the year by author-year citation styles. The default definition is an interword space.
作者---年份引用样式中作者/编辑和年份之间的定界符。缺省定义是一个词间距。

\csitem{namelabeldelim}\CSdelimMark
%The delimiter printed between the name\slash title and the label by alphabetic and numeric citation styles. The default definition is an interword space.
字母样式和数值样式中姓名\slash 标题和标签之间的定界符。
缺省定义是一个词间距。

\csitem{nonameyeardelim}\CSdelimMark
%The delimiter printed between the substitute for the labelname when it does not exist (usually the label or title in standard styles) and the year in author-year citation styles. This is only used when there is no labelname since when the labelname exists, \cmd{nameyeardelim} is used. The default definition is an interword space.
在作者---年份引用样式中,当某一标签名不存在(标准样式中通常是标签或者标题)时其替代者与年份之间的定界符。
仅当没有标签名时使用。这是因为标签名存在时会使用 \cmd{nameyeardelim}。缺省值是一个词间距。

\csitem{authortypedelim}\CSdelimMark
%The delimiter printed between the author and the \texttt{authortype}.
作者和作者类型之间打印的分隔符。

\csitem{editortypedelim}\CSdelimMark
%The delimiter printed between the editor and the \texttt{editor} or \texttt{editortype} string.
编者和编者(类型)本地化字符串之间打印的分隔符。

\csitem{translatortypedelim}\CSdelimMark
%The delimiter printed between the translator and the \texttt{translator} string.
译者和译者本地化字符串之间打印的分隔符。

\csitem{labelalphaothers}
%A string to be appended to the non"=numeric portion of the \bibfield{labelalpha} field (\ie the field holding the citation label used by alphabetic citation styles) if the number of authors\slash editors exceeds the \opt{maxalphanames} threshold or the \bibfield{author}\slash \bibfield{editor} list was truncated in the \file{bib} file with the keyword <\texttt{and others}>. This will typically be a single character such as a plus sign or an asterisk. The default is a plus sign. This command may also be redefined to an empty string to disable this feature. In any case, it must be redefined in the preamble.
当作者/编辑数目超过了 \opt{maxalphanames} 阈值或者 \bibfield{author}\slash \bibfield{editor} 列表在 \file{bib} 文件中由关键词“\texttt{and others}”截断时,
接在 \bibfield{labelalpha} 域的非数值部分之后的字符串(即,该域包含了由字母引用样式使用的引用标签)。
这通常是加号或者星号等单字符。缺省值是加号。
该命令可以通过重定义为空字符串来关闭该特性。
任何情况下必须在导言区中重定义。

\csitem{sortalphaothers}
%Similar to \cmd{labelalphaothers} but used in the sorting process. Setting it to a different value is advisable if the latter contains formatting commands, for example:
类似于 \cmd{labelalphaothers} 但使用在排序过程中。
如果 \cmd{labelalphaothers} 中包含了格式命令,建议将本命令设置为另外一个值,例如:

\begin{ltxexample}
\renewcommand*{\labelalphaothers}{\textbf{+}}
\renewcommand*{\sortalphaothers}{+}
\end{ltxexample}
%
%If \cmd{sortalphaothers} is not redefined, it defaults to \cmd{labelalphaothers}.
如果 \cmd{sortalphaothers} 没有定义,其缺省值为 \cmd{labelalphaothers}。

\csitem{prenotedelim}
%The delimiter printed after the \prm{prenote} argument of a citation command. See \secref{use:cit} for details. The default is an interword space.
引用命令的 \prm{prenote} 选项之后打印的定界符。
详见 \secref{use:cit} 节。缺省值是一个词间距。

\csitem{postnotedelim}
%The delimiter printed before the \prm{postnote} argument of a citation command. See \secref{use:cit} for details. The default is a comma plus an interword space.
引用命令的 \prm{postnote} 选项之前打印的定界符。
详见 \secref{use:cit} 节。缺省值是一个词间距。

\csitem{extpostnotedelim}
%The delimiter printed between the citation and the parenthetical \prm{postnote} argument of a citation command when the postnote occurs outside of the citation parentheses. In the standard styles, this occurs when the citation uses the shorthand field of the entry. See \secref{use:cit} for details. The default is an interword space.
在引用命令中,当后注出现在引用括号外时,引用和带括号的 \prm{postnote} 选项之间的定界符。
在标准样式中,如果引用使用条目的shorthand域就会出现这一现象。
详见 \secref{use:cit} 节。
缺省值是一个词间距。

\cmditem{mkbibname\prm{namepart}}{text}
%This command, which takes one argument, is used to format the name part <namepart> of name list fields. The default datamodel defines the name parts <family>, <given>, <prefix> and <suffix> and therefore the following macros are automatically defined:
该命令带有一个选项,用于姓名列表域的姓名成分 \prm{namepart} 的格式。
默认数据模型定义了姓名成分 <family>, <given>, <prefix> 和 <suffix>,
因此自动定义了以下宏命令:

\begin{ltxexample}
\mkbibnamefamily
\mkbibnamegiven
\mkbibnameprefix
\mkbibnamesuffix
\end{ltxexample}
%
%For backwards compatibility with the legacy \bibtex name parts, the following are also defined, will generate warnings and will set the correct macro:
出于对传统 \BibTeX 姓名成分的向后兼容性也定义以下宏命令。
这些宏会生成警告并设置正确的宏:

\begin{ltxexample}
\mkbibnamelast
\mkbibnamefirst
\mkbibnameaffix
\end{ltxexample}

\csitem{relatedpunct}
%The separator between the \bibfield{relatedtype} bibliography localisation string and the data from the first related entry. Here is an example with \cmd{relatedpunct} set to a dash:
在 \bibfield{relatedtype} 文献的本地化字符串和第一个相应条目数据之间的分隔符。
如下是将 \cmd{relatedpunct} 设置为短横线的例子:

\begin{ltxexample}
A. Smith. Title. 2000, (Orig. pub. as<<->>Origtitle)
\end{ltxexample}

\csitem{relateddelim}
%The generic separator between the data of multiple related entries. The default definition is an optional dot plus linebreak. Here is an example where volumes A-E are related entries of the 5 volume main work:
多重相关条目数据之间的分隔符。
缺省定义是一个可选的点号加上一个断行。
如下是一个五卷作品的例子,其中卷A-E是相关条目:

\begin{ltxexample}
Donald E. Knuth. Computers & Typesetting. 5 vols. Reading, Mass.: Addison-
Wesley, 1984-1986.
Vol. A: The TEXbook. 1984.
Vol. B: TEX: The Program. 1986.
Vol. C: The METAFONTbook. By. 1986.
Vol. D: METAFONT: The Program. 1986.
Vol. E: Computer Modern Typefaces. 1986.
\end{ltxexample}

\csitem{relateddelim\prm{relatedtype}}
%The separator between the data of multiple related entries inside related entries of type <relatedtype>. There is no default, if such a type-specific delimiter does not exist, \cmd{relateddelim} is used.
\prm{relatedtype} 类型的相关联条目中,多重关联条目数据之间的分隔符。
没有缺省值。
如果这样的类型相关的定界符不存在,那么将使用 \cmd{relateddelim}。

\end{ltxsyntax}

\subsubsection{环境敏感的分隔符}% \subsubsection{Context-sensitive Delimiters}
\label{use:fmt:csd}
%The delimiters described in \secref{use:fmt:fmt} are globally defined. That is, no matter where you use them, they print the same thing. This is not necessarily desirable for delimiters which you might want to print different things in different contexts. Here <context> means things like <inside a text citation> or <inside a bibliography item>. For this reason, \biblatex\ provides a more sophisticated delimiter specification and user interface alongside the standard one based on normal macros defined with \cmd{newcommand}.
\secref{use:fmt:fmt} 节介绍的分隔符(定界符)是全局定义的。
也就是说,无论在哪里使用,打印的分隔符都是相同的。
然而,如果想要在不同环境
\footnote{注意:分隔符环境(delimcontext定义的context)要与参考文献文境(newrefcontext使用的context)区分清楚,delimcontext仅针对分隔符----译注}
下中打印不同的符号,这些全局的分隔符就不可取了。
这里“context”可以是“标注文本内部”或者“参考文献项”等。
为此,\biblatex 在利用\cmd{newcommand} 定义分隔符的常规方式基础上,还提供了更为智能的分隔符规范和用户接口。

\begin{ltxsyntax}
\cmditem{DeclareDelimFormat}[context, \dots]{name, \dots}{code}
\cmditem*{DeclareDelimFormat}*[context, \dots]{name, \dots}{code}

%Declares the delimiter macros in the comma"=separated list \prm{names} with the replacement test \prm{code}. If the optional comma"=separated list of \prm{contexts} is given, declare the \prm{names} only for those contexts. \prm{names} defined without any \prm{contexts} behave just like the global delimiter definitions which \cmd{newcommand} gives---just a plain macro with a replacement which can be used as \cmd{name}. However, you can also call delimiter macros defined in this way by using \cmd{printdelim}, which is context-aware. The starred version clears all \prm{context} specific declarations for all \prm{names} first.

在逗号分隔列表 \prm{names} 中使用替代测试 \prm{code} 声明分隔符宏。
如果给出了可选的逗号分隔列表 \prm{context},那么只为相应的 context 声明 \prm{names}。
如果没有可选的 \prm{context},那么被定义的 \prm{names} 与由 \cmd{newcommand} 给出的全局分隔符的定义是一致的——
仅仅是一个简单的宏,用于替代 \cmd{name}。
不过,通过使用能探测 context 的 \cmd{printdelim} 命令,仍然可以调用以这种方式定义的分隔符宏。
带星号的命令会首先清楚所有 \prm{names} 中的 \prm{context} 声明。

\cmditem{DeclareDelimAlias}{alias}{delim}
\cmditem*{DeclareDelimAlias}*[context, \dots]{alias}{delim}


%Declares \prm{alias} to be an alias for the delimiter \prm{delim}. The assigment is valid for all existing contexts of \prm{alias}.The starred version assigns the alias for the given \opt{contexts} only---if the optional argument is empty, assigment is for the global/empty context.
将\prm{alias}声明为分隔符\prm{delim}的别名。该声明对于现有的全部文境都有效。带星号的版本则将声明限定在给出的文境内——当可选参数为空时,声明时针对全局/空文境的。


\cmditem{printdelim}[context]{name}

%Prints a delimiter with name \prm{name}, locally establishing a optional \prm{context} first. Without the optional \prm{context}, \cmd{printdelim} uses the currently active delimiter context.

打印名为 \prm{name} 的分隔符,
首先建立可选的局部 \prm{context}。
如果没有可选的 \prm{context},那么 \cmd{printdelim} 会使用当前活动的分隔符 context。

%Delimiter contexts are simply a string, the value of the internal macro \cmd{blx@delimcontext} which can be set manually by the command \cmd{delimcontext}

分隔符文境是一个简单的字符串,内部宏 \cmd{blx@delimcontext} 的值。
后者可以通过 \cmd{delimcontext} 命令手动设置。

\cmditem{delimcontext}{context}

%Set the delimiter context to \prm{context}. This setting is not global so that delimiter contexts can be nested using the usual \latex group method.

设置分隔符文境为 \prm{context}。
该设置不是全局的,因此分隔符 context 可以使用通常的 \LaTeX 组方法进行嵌套。

\cmditem{DeclareDelimcontextAlias}{alias}{name}

%The context-sensitive delimiter system creates delimiter contexts based on
%the name of citation commands (<parencite>, <textcite> etc.) passed to
%\cmd{DeclareCiteCommand}. In certain cases where there are nested
%definitions of citation commands where \cmd{DeclareCiteCommand} calls
%itself (see the definition of \cmd{textcite} in \sty{authoryear-icomp}
%for example). The delimiter context is then usually incorrect and the
%delimiter specifications do not work. For example, the definition of
%\cmd{textcite} in fact defines and uses \cmd{cbx@textcite} and so the
%context is automatically set to \opt{cbx@textcite} when printing the
%citation. Delimiter definitions expecting to see the context \opt{textcite}
%therefore do not work. Therefore this command is provided for style authors
%which aliases the context \prm{alias} to the context \prm{name}. For
%example:

基于传递给 \cmd{DeclareCiteCommand} 的引用命令(“parencite”、“textcite”等)的名称,
分隔符系统会创建分隔符 context。
在某些情况下引用命令存在嵌套定义,此时 \cmd{DeclareCiteCommand} 会调用自己
(相关示例见 \sty{authoryear-icomp} 中 \cmd{textcite} 的定义)。
此时的 context 一般不准确,从而无法使用分隔符规范。
例如,\cmd{textcite} 的定义实际上定义并使用了 \cmd{cbx@textcite},
因此当打印引用时 context 会自动设置为 \opt{cbx@textcite}。
因此期望 context 为 \opt{textcite} 的分隔符定义就无法运行。
为此,该命令可以将 \prm{alias} 作为 \prm{name} 的别称,例如:

\begin{ltxexample}[style=latex]{}
\DeclareDelimcontextAlias{cbx@textcite}{textcite}
\end{ltxexample}
%
%This (which is a default setting), makes sure that when inside the
%\cmd{cbx@textcite} citation command, the context is in fact \opt{textcite}
%as expected.
这会确保 context 在 \cmd{cbx@textcite} 引用命令内部时就是预期的 \opt{textcite}。

\end{ltxsyntax}
%
%\biblatex\ has several default contexts which are established automatically in various places:
\biblatex 在不同位置自动创建了若干个默认的 context:

\begin{description}
	\item[none] %At begin document
	文档开始处。
	\item[bib] %Inside a bibliography begun with \cmd{printbibliography} or inside a \cmd{usedriver}
	以 \cmd{printbibliography} 开始的参考文献内部或者 \cmd{usedriver} 内部。
	\item[biblist] %Inside a bibliography list begun with \cmd{printbiblist}
	以 \cmd{printbiblist} 开始的参考文献列表内部
	\item[<citecommand>] %Inside a citation command \cmd{citecommand} defined with \cmd{DeclareCiteCommand}
	使用 \cmd{DeclareCiteCommand} 定义的 \cmd{citecommand} 引用命令内部
\end{description}

%For example, the defaults for \cmd{nametitledelim} are:
例如,\cmd{nametitledelim} 的默认设置为:

\begin{ltxexample}[style=latex]{}
\DeclareDelimFormat{nametitledelim}{\addcomma\space}
\DeclareDelimFormat[textcite]{nametitledelim}{\addspace}
\end{ltxexample}
%
%This means that \cmd{nametitledelim} is defined globally as <\cmd{addcomma}\cmd{space}> as per the standard delimiter interface. However, in addition, the delimiter can be printed using \cmd{printdelim} which would print the same as \cmd{nametitledelim} apart from inside a \cmd{textcite}, in which it would print \cmd{addspace} which is more suitable for running text. If desired, a context can be forced with the optional argument to \cmd{printdelim}, so
这意味着对于每个标准分隔符接口,\cmd{nametitledelim} 全局定义为 \cmd{addcomma}\cmd{space}。
此外,总体上使用 \cmd{printdelim} 可以打印该分隔符为 \cmd{nametitledelim},
不过在 \cmd{textcite} 内部会打印 \cmd{addspace},这对于行文更适合。
如果需要的话,可以强制添加 context 作为\cmd{printdelim} 的可选项,例如:

\begin{ltxexample}[style=latex]{}
\printdelim[textcite]{nametitledelim}
\end{ltxexample}
%
%Would print \cmd{addspace} regardless of the surrounding context of the \cmd{printdelim}. Contexts are just arbitrary strings and so you can establish them at any time, using \cmd{delimcontext}. If \cmd{printdelim} finds no special value for the delimiter \prm{name} in the current context, it simply prints \cmd{name}. This means that style authors can use \cmd{printdelim} and users expecting to be able to use \cmd{renewcommand} to redefine delimiters can do so with one caveat---such a definition won't change any context-specific delimiters which are defined:
无论 \cmd{printdelim} 周围的内容怎样,总会打印出 \cmd{addspace}。
由于context 是任意字符串,因此可以在任何时刻使用 \cmd{delimcontext} 构建。
如果 \cmd{printdelim} 在当前 context 没有找到分隔符 \prm{name} 的特定值,那么就直接打印出 \cmd{name}。
这意味着样式作者可以使用 \cmd{printdelim}。
同时希望使用 \cmd{renewcommand} 重定义分隔符的用户也可以这样做。
不过前提是,该定义不能改变任何 context 相关的分隔符,如下所示:

\begin{ltxexample}[style=latex]{}
\DeclareDelimFormat{delima}{X}
\DeclareDelimFormat[textcite]{delima}{Y}
\renewcommand*{\delima}{Z}
\end{ltxexample}
%
%Here, \cmd{delima} always prints <Z>. \verb+\printdelim{delima}+ in any context other than <textcite> also prints \cmd{delima} and hence <Z> but in a <textcite> context prints <Y>. See the \file{04-delimiters.tex} example file that comes with \biblatex\ for more information.
这里,\cmd{delima} 总会打印“Z”。
在任何“textcite”之外的 context 中,\verb+\printdelim{delima}+ 也会打印 \cmd{delima} 也就是“Z”,
而在“textcite”中则打印“Y”。
更多信息参考 \biblatex 附带的示例文件 \file{04-delimiters.tex}。

\subsubsection{语言相关命令}%\subsubsection{Language-specific Commands}
\label{use:fmt:lng}

%The commands in this section are language specific. When redefining them, you need to wrap the new definition in a \cmd{DeclareBibliographyExtras} command (in an \file{.lbx} file) or a \cmd{DefineBibliographyExtras} command (user documents), see \secref{use:lng} for details. Note that all commands starting with \cmd{mk\dots} take one or more arguments.

本节中的命令是与语言相关的。
因此重定义时需要将新的定义包裹在 \cmd{DeclareBibliographyExtras} 命令(在 \file{.lbx} 文件中)
或者 \cmd{DefineBibliographyExtras} 命令中(用户文件中),详见 \secref{use:lng} 节。
请注意所有以 \cmd{mk\dots} 开头的命令都带有一个或更多选项。

\begin{ltxsyntax}

\csitem{bibrangedash}

%The language specific dash to be used for ranges of numbers. Defaults to \cmd{textendash}.
用于数字范围的横线。
默认值为 \cmd{textendash}。

\csitem{bibrangessep}

%The language specific separator to be used between multiple ranges. Defaults to a comma followed by a space.
用于多重范围之间的分隔符。
默认为一个逗号加一个空格。

\csitem{bibdatesep}

%The language specific separator used between date components in terse date formats. Defaults to \cmd{hyphen}.
在短日期格式用于日期成分之间的分隔符。
默认为 \cmd{hyphen}。

\csitem{bibdaterangesep}

%The language specific separator to be used for date ranges. Defaults to \cmd{textendash} for all date formats apart from \opt{ymd} which defaults to a \cmd{slash}. The date format option \opt{edtf} is hard-coded to \cmd{slash} since this is a standards compliant format.
用于日期范围的分隔符。
对于 \opt{ymd} 格式默认为 \cmd{slash},
对于其它日期格式默认为 \cmd{textendash}。
日期格式选项 \opt{edtf} 则硬编码为 \cmd{slash},因为这是标准兼容格式。

\csitem{mkbibdatelong}

%Takes the names of three field as arguments which correspond to three date components (in the order year\slash month\slash day) and uses their values to print the date in the language specific long date format.

以对应于三个日期成分(以年/月/日的顺序)的三个域名作为选项,
并在长日期格式中使用相应值来打印日期。

\csitem{mkbibdateshort}

%Similar to \cmd{mkbibdatelong} but using the language specific short date format.
类似于 \cmd{mkbibdatelong} 但是使用短日期格式。

\csitem{mkbibtimezone}

%Modifies a timezone string passed in as the only argument. By default this changes <Z> to the value of \cmd{bibtimezone}.
修改传入的时区字符串作为唯一选项。
默认情况会将“Z”改为 \cmd{bibtimezone} 的值。

\csitem{bibdateuncertain}

%The language specific marker to be used after uncertain dates when the global option \opt{dateuncertain} is enabled. Defaults to a space followed by a question mark.
当启用全局选项 \opt{dateuncertain} 时用于不确定日期之后的标识符。
默认为空格加一个问号。

\csitem{bibdateeraprefix}

%The language specific marker which is printed as a prefix to beginning BCE/BC dates in a date range when the option \opt{dateera} is set to <astronomical>. Defaults to \cmd{textminus}, if defined and \cmd{textendash} otherwise.
当设置 \opt{dateera} 选项为 \opt{astronomical} 时,在日期范围中公元前日期的前缀标识符。
命令 \cmd{textminus} 如果有定义则为默认值,否则 \cmd{textendash} 为默认值。

\csitem{bibdateeraendprefix}

%The language specific marker which is printed as a prefix to end BCE/BC dates in a date range when the option \opt{dateera} is set to <astronomical>. Defaults to a thin space followed by \cmd{bibdateeraprefix} when \cmd{bibdaterangesep} is set to a dash and to \cmd{bibdateeraprefix} otherwise.  This is a separate macro so that you may add extra space before a negative date marker which, for example follows a dash date range marker as this can look a little odd.
当设置 \opt{dateera} 选项为 \opt{astronomical} 时,在日期范围中公元前日期结束的标识符。
当 \cmd{bibdaterangesep} 选项设置为短横线时默认值为窄间距(thin space)加上 \cmd{bibdateeraprefix},
否则默认值为 \cmd{bibdateeraprefix}。

\csitem{bibtimesep}

%The language specific marker which separates time components. Defaults to a colon.
分隔时间成分的标识符,默认为冒号。

\csitem{bibtimezonesep}

%The language specific marker which separates an optional time zone component from a time. Empty by default.
分隔时间与可选的时区信息的标识符。默认为无。

\csitem{bibtzminsep}

%The language specific marker which separates hour and minute component of offset timezones. Defaults to a \cmd{bibtimesep}.
偏移时区的小时和分钟信息直接的分隔符。
默认值为 \cmd{bibtimesep}。

\csitem{bibdatetimesep}

%The language specific separator printed between date and time components when printing time components along with date components (see the \opt{$<$datetype$>$dateusetime} option in \secref{use:opt:pre:gen}). Defaults to a space for non-\acr{ISO8601-2} output formats, and 'T' for \acr{ISO8601-2} output format.

当打印时间成分与日期成分时,打印二者之间与具体语言相关的分隔符
(见 \secref{use:opt:pre:gen} 节的 \opt{\prm{datetype}dateusetime} 选项)。
对于非\acr{ISO8601-2}输出格式,默认值为空格;
对于\acr{ISO8601-2}输出格式,默认值为“T”。

\csitem{finalandcomma}

%Prints the comma to be inserted before the final <and> in a list, if applicable in the respective language. Here is an example:
当在对应语言中可用时在列表的最后一个“and” 之前打印逗号。这里是一个例子:

\begin{ltxexample}
Michel Goossens, Frank Mittelbach<<,>> and Alexander Samarin
\end{ltxexample}
%
%\cmd{finalandcomma} is the comma before the word <and>. See also \cmd{multinamedelim}, \cmd{finalnamedelim}, \cmd{textcitedelim}, and \cmd{revsdnamedelim} in \secref{use:fmt:fmt}.
\cmd{finalandcomma} 是单词“and”之前的逗号。
另见 \secref{use:fmt:fmt} 节的 \cmd{multinamedelim}、\cmd{finalnamedelim}、\cmd{textcitedelim} 和 \cmd{revsdnamedelim}。

\csitem{finalandsemicolon}

%Prints the semicolon to be inserted before the final <and> in a list of lists, if applicable in the respective language. Here is an example:
当在对应语言中可用时在列表的最后一个“and”之前打印分号。例如:

\begin{ltxexample}
Goossens, Mittelbach, and Samarin; Bertram and Wenworth<<;>> and Knuth
\end{ltxexample}
%
%\cmd{finalandsemicolon} is the semicolon before the word <and>. See also \cmd{textcitedelim} in \secref{use:fmt:fmt}.
\cmd{finalandsemicolon} 是单词“and”之前的分号。
另见 \secref{use:fmt:fmt} 节的 \cmd{textcitedelim}。

\cmditem{mkbibordinal}{integer}

%This command, which takes an integer as its argument, prints an ordinal number.
该命令将一个整数作为选项,打印出一个序数。

\cmditem{mkbibmascord}{integer}

%Similar to \cmd{mkbibordinal}, but prints a masculine ordinal, if applicable in the respective language.
类似于 \cmd{mkbibordinal},但在对应语言中可用时打印出阳性序数。

\cmditem{mkbibfemord}{integer}

%Similar to \cmd{mkbibordinal}, but prints a feminine ordinal, if applicable in the respective language.
类似于 \cmd{mkbibordinal},但在对应语言中可用时打印出阴性序数。

\cmditem{mkbibneutord}{integer}

%Similar to \cmd{mkbibordinal}, but prints a neuter ordinal, if applicable in the respective language.
类似于 \cmd{mkbibordinal},但在对应语言中可用时打印出中性序数。

\cmditem{mkbibordedition}{integer}

%Similar to \cmd{mkbibordinal}, but intended for use with the term <edition>.
类似于 \cmd{mkbibordinal},但与单词“edition”一起使用。

\cmditem{mkbibordseries}{integer}

%Similar to \cmd{mkbibordinal}, but intended for use with the term <series>.
类似于 \cmd{mkbibordinal},但与单词“series”一起使用。

\end{ltxsyntax}

\subsubsection{长度和计数器}%\subsubsection{Lengths and Counters}
\label{use:fmt:len}

%The length registers and counters in this section may be changed in the document preamble with \cmd{setlength} and \cmd{setcounter}, respectively.

本节中的长度寄存器和计数器可以在导言区中分别用 \cmd{setlength} 和 \cmd{setcounter} 来修改。

\begin{ltxsyntax}

\lenitem{bibhang}

%The hanging indentation of the bibliography, if applicable. This length is initialized to \cmd{parindent} at load-time.

参考文献的悬挂缩进(当使用时)。
该长度在导入时初始化为 \cmd{parindent}。

\lenitem{biblabelsep}

%The horizontal space between entries and their corresponding labels in the bibliography. This only applies to bibliography styles which print labels, such as the \texttt{numeric} and \texttt{alphabetic} styles. This length is initialized to twice the value of \cmd{labelsep} at load-time.

参考文献中条目和相应标签之间的水平距离。
这只应用于 \texttt{numeric} 和 \texttt{alphabetic} 等打印标签的参考文献样式。
该长度在导入时初始化为 \cmd{labelsep} 值的两倍。

\lenitem{bibitemsep}

%The vertical space between the individual entries in the bibliography. This length is initialized to \cmd{itemsep} at load-time. Note that \len{bibitemsep}, \len{bibnamesep}, and \len{bibinitsep} obey the rules for \cmd{addvspace}, that is, when vertical space introduced by any of these commands immediately follows on from space introduced by another of them, the resulting total space is equal to the largest of them.

参考文献中每一条目之间的垂直间距。
该长度在导入时初始化为 \cmd{itemsep}。
请注意 \len{bibitemsep}、\len{bibnamesep} 和 \len{bibinitsep} 服从 \cmd{addvspace} 的规则,
也就是,当这些命令中任何一个直接在另外一个之后引入垂直间距时,所得到的总间距是其中的最大值。

\lenitem{bibnamesep}

%Vertical space to be inserted between two entries in the bibliography whenever an entry starts with a name which is different from the initial name of the previous entry. The default value is zero. Setting this length to a positive value greater than \len{bibitemsep} will group the bibliography by author\slash editor name. Note that \len{bibitemsep}, \len{bibnamesep}, and \len{bibinitsep} obey the rules for \cmd{addvspace}, that is, when vertical space introduced by any of these commands immediately follows on from space introduced by another of them, the resulting total space is equal to the largest of them.

参考文献中插入在两条姓名不同的条目之间的垂直间距。缺省值是零。
将该值设为大于 \len{bibitemsep} 会使得参考文献按照作者/编辑名分组。
请注意 \len{bibitemsep}、\len{bibnamesep} 和 \len{bibinitsep} 服从 \cmd{addvspace} 的规则,
也就是,当这些命令中任何一个直接在另外一个之后引入垂直间距时,所得到的总间距是其中的最大值。

\lenitem{bibinitsep}

%Vertical space to be inserted between two entries in the bibliography whenever an entry starts with a letter which is different from the initial letter of the previous entry. The default value is zero. Setting this length to a positive value greater than \len{bibitemsep} will group the bibliography alphabetically. Note that \len{bibitemsep}, \len{bibnamesep}, and \len{bibinitsep} obey the rules for \cmd{addvspace}, that is, when vertical space introduced by any of these commands immediately follows on from space introduced by another of them, the resulting total space is equal to the largest of them.

参考文献中插入在两条首字母不同的条目之间的垂直间距。缺省值是零。
将该值设置为大于 \len{bibitemsep} 会使得参考文献按字母分组。
请注意 \len{bibitemsep}、\len{bibnamesep} 和 \len{bibinitsep} 服从 \cmd{addvspace} 的规则,
也就是,当这些命令中任何一个直接在另外一个之后引入垂直间距时,所得到的总间距是其中的最大值。

\lenitem{bibparsep}

%The vertical space between paragraphs within an entry in the bibliography. The default value is zero.

参考文献中条目内部的段间距。缺省值为零。

\cntitem{abbrvpenalty}

%This counter, which is used by the localisation modules, holds the penalty used in short or abbreviated localisation strings. For example, a linebreak in expressions such as «et al.» or «ed. by» is unfortunate, but should still be possible to prevent overfull boxes. This counter is initialized to \cmd{hyphenpenalty} at load-time. The idea is making \tex treat the whole expression as if it were a single, hyphenatable word as far as line"=breaking is concerned. If you dislike such linebreaks, use a higher value. If you do not mind them at all, set this counter to zero. If you want to suppress them unconditionally, set it to <infinite> (10\,000 or higher).\footnote{The default values assigned to \cnt{abbrvpenalty}, \cnt{lownamepenalty}, and \cnt{highnamepenalty} are deliberately very low to prevent overfull boxes. This implies that you will hardly notice any effect on line-breaking if the text is set justified. If you set these counters to 10\,000 to suppress the respective breakpoints, you will notice their effect but you may also be confronted with overfull boxes. Keep in mind that line-breaking in the bibliography is often more difficult than in the body text and that you can not resort to rephrasing a sentence. In some cases it may be preferable to set the entire bibliography \cmd{raggedright} to prevent suboptimal linebreaks. In this case, even the fairly low default penalties will make a visible difference.}

该计数器在本地化模块中使用,用于设定本地化字符串中缩写和短语中使用的惩罚值。
例如,“et al”或“ed. by” 等短语中的断行是不美观的,但为了防止盒子溢出仍然应当可以使用。
该计数器在导入时初始化为 \cmd{hyphenpenalty}。
断行考虑的原则是,使得 \TeX 将整个语句看做是单个可以用连字符断行的单词。
如果你不喜欢相应的断行,可以设置为更高的值。
如果你不介意这些效果,可以设置为零。
如果你想无条件地取消这种效果,
可以设置为“无穷”(10\,000 或更高)。\footnote{%
这里很慎重地将 \cnt{abbrvpenalty}、\cnt{lownamepenalty} 和 \cnt{highnamepenalty} 的缺省值设定得非常低以防止盒子溢出。
这意味着,如果文本设置合理,那么你几乎不会注意到断行的影响。
如果你将这些值设置为 10\,000 以取消相应断点,那么你就会注意到它们的影响,不过同时你也许要面对盒子溢出现象。
需要注意的是,参考文献中的断行往往比正文中更困难,而且你不能通过换一种表达方式来解决。
在某些情况下,在整个参考文献中设置 \cmd{raggedright} 来阻止非最佳的断行往往更好。
此时,即使是相对低的缺省惩罚项也会造成不同效果。}

\cntitem{highnamepenalty}

%This counter holds a penalty affecting line"=breaking in names. Please refer to \secref{use:cav:nam,use:fmt:fmt} for explanation. The counter is initialized to \cmd{hyphenpenalty} at load-time. Use a higher value if you dislike the respective linebreaks. If you do not mind them at all, set this counter to zero. If you prefer the traditional \bibtex behavior (no linebreaks at \cnt{highnamepenalty} breakpoints), set it to <infinite> (10\,000 or higher).

该计数器设定了影响姓名中断行的惩罚值。其解释见 \secref{use:cav:nam,use:fmt:fmt} 节。
该计数器在导入时初始化为 \cmd{hyphenpenalty}。
如果你不喜欢相应的断行,可以设置为更高的值。
如果你不介意这些效果,可以设置为零。
如果你更喜欢传统的 \BibTeX 样式(在 \cnt{highnamepenalty} 处没有断行),可以设置为“无穷”(10\,000或更高)。

\cntitem{lownamepenalty}

%Similar to \cnt{highnamepenalty}. Please refer to \secref{use:cav:nam,use:fmt:fmt} for explanation. The counter is initialized to half the \cmd{hyphenpenalty} at load-time. Use a higher value if you dislike the respective linebreaks. If you do not mind them at all, set this counter to zero.

类似于 \cnt{highnamepenalty}。其解释见 \secref{use:cav:nam,use:fmt:fmt} 节。
该计数器在导入时初始化为 \cmd{hyphenpenalty} 的一半。
如果你不喜欢相应的断行,可以设置为更高的值。
如果你不介意这些效果,可以设置为零。

\end{ltxsyntax}

\subsubsection{通用命令}%\subsubsection{All-purpose Commands}
\label{use:fmt:aux}

%The commands in this section are all-purpose text commands which are generally available, not only in citations and the bibliography.
本节中的命令是通用文本命令,除了引用和参考文献之外在一般情况下都可以使用。

\begin{ltxsyntax}

\csitem{bibellipsis}

%An ellipsis symbol with brackets: <[\dots\unkern]>.
带有方括号的省略号:“[\dots\unkern]”。

\csitem{noligature}

%Disables ligatures at this position and adds some space. Use this command to break up standard ligatures like <fi> and <fl>. It is similar to the \verb+"|+ shorthand provided by some language modules of the \sty{babel}/\sty{polyglossia} packages.

在该位置取消连字并增加一些空格。
使用该命令来打断标准中“fi”、“fl”等连字。
类似于 \sty{babel}/\sty{polyglossia} 宏包中一些语言模块提供的 \verb+"|+ 缩写。

\csitem{hyphenate}

%A conditional hyphen. In contrast to the standard \cmd{-} command, this one allows hyphenation in the rest of the word. It is similar to the \verb|"-| shorthand provided by some language modules of the \sty{babel}/\sty{polyglossia} packages.

条件连字号。与标准的 \cmd{-} 命令不同,该命令允许在单词的剩余部分使用连字号。
类似于 \sty{babel}/\sty{polyglossia} 宏包中一些语言模块提供的 \verb+"-+ 缩写。

\csitem{hyphen}

%An explicit, breakable hyphen intended for compound words. In contrast to a literal <\texttt{-}>, this command allows hyphenation in the rest of the word. It is similar to the \verb|"=| shorthand provided by some language modules of the \sty{babel}/\sty{polyglossia} packages.

用于复合词的显式可断连字号。
与文本“\texttt{-}”不同,该命令允许在单词的剩余部分使用连字号。
类似于 \sty{babel}/\sty{polyglossia} 宏包中一些语言模块提供的 \verb+"=+ 缩写。

\csitem{nbhyphen}

%An explicit, non-breakable hyphen intended for compound words. In contrast to a literal <\texttt{-}>, this command does not permit line breaks at the hyphen but still allows hyphenation in the rest of the word. It is similar to the \verb|"~| shorthand provided by some language modules of the \sty{babel}/\sty{polyglossia} packages.

用于复合词的显式不可断连字号。
与文本“\texttt{-}”不同,该命令不允许在连字号处断行但仍然允许在单词的剩余部分使用连字号。
类似于 \sty{babel}/\sty{polyglossia} 宏包中一些语言模块提供的 \verb+"~+ 缩写。

\csitem{nohyphenation}

%A generic switch which suppresses hyphenation locally. Its scope should normally be confined to a group.

局部抑制连字号的一般切换命令。
正常情况下其作用范围应被限定在一个组内。

\cmditem{textnohyphenation}{text}

%Similar to \cmd{nohyphenation} but restricted to the \prm{text} argument.

类似于 \cmd{nohyphenation} 但是限制在 \prm{text} 选项中。

\cmditem{mknumalph}{integer}

%Takes an integer in the range 1--702 as its argument and converts it to a string as follows: 1=a, \textellipsis, 26=z, 27=aa, \textellipsis, 702=zz. This is intended for use in formatting directives for the \bibfield{extrayear} and \bibfield{extraalpha} fields.

将 1--702 间的整数作为选项并将其转化为如下的字符串: 1=a, \textellipsis, 26=z, 27=aa, \textellipsis, 702=zz。
这用于 \bibfield{extrayear} 和 \bibfield{extraalpha} 等域的格式设置。

\cmditem{mkbibacro}{text}

%Generic command which typesets an acronym using the small caps variant of the current font, if available, and as-is otherwise. The acronym should be given in uppercase letters.

在可用时使用当前字体的小型大写变体排版首字母缩写的一般性命令,否则依原样排版。
首字母缩写应当以大写字母形式给出。

\cmditem{autocap}{character}

%Automatically converts the \prm{character} to its uppercase form if \biblatex's punctuation tracker would capitalize a localisation string at the current location. This command is robust. It is useful for conditional capitalization of certain strings in an entry. Note that the \prm{character} argument is a single character given in lowercase. For example:

如果 \biblatex 的标点追踪能够在当前位置将本地化字符串大写,
该命令自动将 \prm{character} 转化为大写形式。
该命令是鲁棒的。在条目的某些字符串需要给定条件下的大写时该命令是很有用的。
请注意,\prm{character} 选项是以小写形式给出的单字符。例如:

\begin{ltxexample}
\autocap{s}pecial issue
\end{ltxexample}
%
%will yield <Special issue> or <special issue>, as appropriate. If the string to be capitalized starts with an inflected character given in Ascii notation, include the accent command in the \prm{character} argument as follows:
将产生合适的“Special issue”或者“special issue”。
如果被大写的字符串以Ascii记号给出的变体字符开始,包括如下 \prm{character} 选项中的重音命令:

\begin{ltxexample}
\autocap{\'e}dition sp\'eciale
\end{ltxexample}
%
%This will yield <Édition spéciale> or <édition spéciale>. If the string to be capitalized starts with a command which prints a character, such as \cmd{ae} or \cmd{oe}, simply put the command in the \prm{character} argument:
这会生成“Édition spéciale”或者“édition spéciale”。
如果大写的字符串以能打印出字符的命令开始,例如 \cmd{ae} 或 \cmd{oe},只要该命令放入 \prm{character} 选项即可:

\begin{ltxexample}
\autocap{\oe}uvres
\end{ltxexample}
%
%This will yield <Œuvres> or <œuvres>.
这会生成“Œuvres”或“œuvres”。

\end{ltxsyntax}

\subsection[关于语言的注意事项]{特定语言的注意事项}%\subsection[Language notes]{Language-specific Notes}
\label{use:loc}

%The facilities discussed in this section are specific to certain localisation modules.
本节讨论的功能特定于某些本地化模块。

\subsubsection{保加利亚文}%\subsubsection{Bulgarian}
\label{use:loc:bul}

%Like the Greek localisation module, the Bulgarian module also requires \utf support. It will not work with any other encoding.
类似于希腊文本地化模块,保加利亚语模块也需要\utf 支持,采用其他编码均无效。

\subsubsection{美式英文}%\subsubsection{American}
\label{use:loc:us}

%The American localisation module uses \cmd{uspunctuation} from \secref{aut:pct:cfg} to enable <American-style> punctuation. If this feature is enabled, all trailing commas and periods after \cmd{mkbibquote} will be moved inside the quotes. If you want to disable this feature, use \cmd{stdpunctuation} as follows:

美式英文本地化模块使用 \secref{aut:pct:cfg} 节的 \cmd{uspunctuation} 来激活“美式”标点。
如果启用该特性,所有在 \cmd{mkbibquote} 之后的逗号和句号会前移到引号内。
如果想要关闭该特性,使用如下的 \cmd{stdpunctuation}:

\begin{ltxexample}
\DefineBibliographyExtras{american}{%
  \stdpunctuation
}
\end{ltxexample}
%
%By default, the <American punctuation> feature is enabled by the \texttt{american} localisation module only. The above code is only required if you want American localisation without American punctuation. Since standard punctuation is the package default, it would be redundant with any other language.
缺省情况下,“美式标点”特性只由 \texttt{american} 本地化模块启用。
以上代码只在你想要不带美式标点的美式英文本地化时需要。
由于标准的标点是宏包缺省的,对于其它语言这会是多余的。

%It is highly advisable to always specify \texttt{american}, \texttt{british}, \texttt{australian}, etc. rather than \texttt{english} when loading the \sty{babel}/\sty{polyglossia} packages to avoid any possible confusion. Older versions of the \sty{babel} package used to treat \opt{english} as an alias for \opt{british}; more recent ones treat it as an alias for \opt{american}. The \biblatex package essentially treats \texttt{english} as an alias for \opt{american}, except for the above feature which is only enabled if \texttt{american} is requested explicitly.

为了避免可能的混淆,在导入 \sty{babel}/\sty{polyglossia} 宏包时,
强烈建议总是指明 \texttt{american}、\texttt{british}、\texttt{australian} 等而不是 \texttt{english}。
老版本的 \sty{babel} 宏包过去将 \opt{english} 作为 \opt{british} 的别名;
而更近的版本中将其作为 \opt{american} 的别名。
除了以上只在需要显式指明 \texttt{american} 时才启用的特性,
\biblatex 宏包本质上将 \texttt{english} 作为 \opt{american} 的别名。

\subsubsection{西班牙文}%\subsubsection{Spanish}
\label{use:loc:esp}

%Handling the word <and> is more difficult in Spanish than in the other languages supported by this package because it may be <y> or <e>, depending on the initial sound of the following word. Therefore, the Spanish localisation module does not use the localisation string <\texttt{and}> but a special internal <smart and> command. The behavior of this command is controlled by the \cnt{smartand} counter.

在本宏包中,处理西班牙文的单词“and”比其它语言更困难,
因为可以是“y”或“e”,这取决于下个单词的第一个音节。
因此,西班牙文的本地化模块不使用本地化字符串“\texttt{and}”,
而是使用特殊的内部“智能and”命令。
该命令的行为由 \cnt{smartand} 计数器控制。

\begin{ltxsyntax}

\cntitem{smartand}

%This counter controls the behavior of the internal <smart and> command. When set to 1, it prints <y> or <e>, depending on the context. When set to 2, it always prints <y>. When set to 3, it always prints <e>. When set to 0, the <smart and> feature is disabled. This counter is initialized to 1 at load-time and may be changed in the preamble. Note that setting this counter to a positive value implies that the Spanish localisation module ignores \cmd{finalnamedelim} and \cmd{finallistdelim}.

该计数器控制内部“智能and”命令的行为。
当设置为1时,取决于语境会打印“y”或“e”。
当设置为2时,总是打印“y”。
当设置为3时,总是打印“e”。
当设置为0时则禁用“智能and”特性。
该计数器在导入时初始化为1,可以在导言区中修改。
请注意,将该计数器设置为一个正整数则说明西班牙文的本地化模块会忽略 \cmd{finalnamedelim} 和 \cmd{finallistdelim}。

\csitem{forceE}

%Use this command in \file{bib} files if \biblatex gets the <and> before a certain name wrong. As its name suggests, it will enforce <e>. This command must be used in a special way to be correct \bibtex datafile format. Here is an example:

如果 \biblatex 在某个错误的名字前得到“and”,那么可以在 \file{bib} 文件中使用该命令。
如同该命令的名字,该命令会强制输出“e”。
为了获得正确的 \BibTeX 数据文件格式,必须以特定方式使用该命令。例如:

\begin{lstlisting}[style=bibtex]{}
author = {Edward Jones and Eoin Maguire},
author = {Edward Jones and <<{\forceE{E}}>>oin Maguire},
\end{lstlisting}
%
%Note that the initial letter of the respective name component is given as an argument to \cmd{forceE} and that the entire construct is wrapped in an additional pair of curly braces.
请注意,相应姓名成分的首字母作为 \cmd{forceE} 的选项,
然后整个地放在额外一对花括号中。

\csitem{forceY}

%Similar to \cmd{forceE} but enforces <y>.
类似于 \cmd{forceE} 但是强制输出“y”。

\end{ltxsyntax}

\subsubsection{希腊文}%\subsubsection{Greek}
\label{use:loc:grk}

%The Greek localisation module requires \utf support. It will not work with any other encoding. Generally speaking, the \biblatex package is compatible with the \sty{inputenc} package and with \xelatex. The \sty{ucs} package will not work. Since \sty{inputenc}'s standard \file{utf8} module is missing glyph mappings for Greek, this leaves Greek users with \xelatex. Note that you may need to load additional packages which set up Greek fonts. As a rule of thumb, a setup which works for regular Greek documents should also work with \biblatex. However, there is one fundamental limitation. As of this writing, \biblatex has no support for switching scripts. Greek titles in the bibliography should work fine, but English and other titles in the bibliography may be rendered in Greek letters. If you need multi-script bibliographies, using \xelatex is the only sensible choice.

希腊文本地化模块需要 \utf 支持,与其它编码不兼容。
一般来说,\biblatex 宏包与 \sty{inputenc} 宏包以及 \XeLaTeX 都兼容。
而 \sty{ucs} 宏包不可用。
由于 \sty{inputenc} 的标准 \opt{utf8} 模块缺失一部分希腊语字形映射,
因此希腊文用户可以选择 \XeLaTeX 。
请注意,用户仍需要载入额外宏包来设置希腊文字体。
根据经验,常规希腊文的文档设置一般也应当可以使用 \biblatex 。
然而有一个根本性限制:\biblatex 不支持切换语言。
参考文献中可以出现希腊文标题,但英文和其它标题可能会渲染为希腊字母。
如果需要多语言的参考文献,使用 \XeLaTeX 是明智的选择。

\subsubsection{俄文}%\subsubsection{Russian}
\label{use:loc:rus}

%Like the Greek localisation module, the Russian module also requires \utf support. It will not work with any other encoding.

与希腊文模块类似,俄文模块同样需要 \utf 支持。
与其它编码不兼容。

\subsection{关于用法的注意事项}%\subsection{Usage Notes}
\label{use:use}

%The following sections give a basic overview of the \biblatex package and discuss some typical usage scenarios.

以下几节讨论了 \biblatex 宏包的基本概述以及一些典型的使用场合。

\subsubsection{概述}%\subsubsection{Overview}
\label{use:use:int}

%Using the \biblatex package is slightly different from using traditional \bibtex styles and related packages. Before we get to specific usage scenarios, we will therefore have a look at the structure of a typical document first:

使用 \biblatex 宏包与传统的 \BibTeX 样式和相关宏包稍有不同。
因此,在讨论具体的使用场景之前,我们首先要看一下典型的文件结构:

\begin{ltxexample}
\documentclass{...}
\usepackage[...]{biblatex}
<<\addbibresource>>{<<bibfile.bib>>}
\begin{document}
<<\cite>>{...}
...
<<\printbibliography>>
\end{document}
\end{ltxexample}
%
%With traditional \bibtex, the \cmd{bibliography} command serves two purposes. It marks the location of the bibliography and it also specifies the \file{bib} file(s). The file extension is omitted. With \biblatex, resources are specified in the preamble with  \cmd{addbibresource} using the full name with \file{.bib} suffix. The bibliography is printed using the \cmd{printbibliography} command which may be used multiple times (see \secref{use:bib} for details). The document body may contain any number of citation commands (\secref{use:cit}). Processing this example file requires that a certain procedure be followed. Suppose our example file is called \path{example.tex} and our bibliographic data is in \path{bibfile.bib}. The procedure, then, is as follows:
在传统的 \BibTeX 下,\cmd{bibliography} 命令提供了两个目的:
标记文献的位置并且确定 \file{bib} 文件。
文件扩展名是省略的。
而在 \biblatex 下则在导言区通过 \cmd{addbibresource} 使用文件名全称(带有 \file{.bib} 后缀)来确定文献资源。
文献的打印则使用 \cmd{printbibliography} 命令,而且该命令可以使用多次(详见 \secref{use:bib} 节)。
文档正文可以包含任意多个引用命令(\secref{use:cit} 节)。
处理示例文件需要以下若干步骤。
假设我们的示例文件叫 \path{example.tex},参考文献数据在 \path{bibfile.bib} 中,那么过程如下:

\begin{enumerate}

\item %Run \bin{latex} on \path{example.tex}. If the file contains any citations, \biblatex will request the respective data from \biber by writing commands to the auxiliary file \path{example.bcf}.
对 \file{example.tex} 运行 \bin{latex} 命令。
如果该文件包含引用,\biblatex 会将有关命令写入辅助文件 \file{example.bcf},进而从 \biber 调用相关数据。
\item %Run \bin{biber} on \path{example.bcf}. \biber will retrieve the data from \path{bibfile.bib} and write it to the auxiliary file \path{example.bbl} in a format which can be processed by \biblatex.
对 \file{example.bcf} 运行 \bin{biber} 命令。
\biber 会从 \file{bibfile.bib} 中检索数据,并将其写入辅助文件 \file{example.bib} 中,写入的格式可以被 \biblatex 处理。
\item %Run \bin{latex} on \path{example.tex}. \biblatex will read the data from \path{example.bbl} and print all citations as well as the bibliography.
对 \file{example.tex} 运行 \bin{latex} 命令。
\biblatex 会从 \file{example.bbl} 中读取数据并打印所有的引用及参考文献。

\end{enumerate}

\subsubsection{辅助文件}%\subsubsection{Auxiliary Files}
\label{use:use:aux}

%The \biblatex package uses one auxiliary \file{bcf} file only. Even if there are citation commands in a file included via \cmd{include}, you only need to run \biber on the main \file{bcf} file. All information \biber needs is in the \file{bcf} file, including information about all refsections if using multiple \env{refsection} environments (see \secref{use:use:mlt}).

\biblatex 宏包只使用一个 \file{bcf} 辅助文件。
即便文件中通过 \cmd{include} 包含引用命令,你也只需在主 \file{bcf} 文件上运行 \biber 。
\biber 需要的全部信息都在 \file{bcf} 文件中,包括当使用多重 \env{refsection} 环境
(见 \secref{use:use:mlt} 节)时关于所有参考文献分节的信息。

\subsubsection{多重文献}%\subsubsection{Multiple Bibliographies}
\label{use:use:mlt}

%In a collection of articles by different authors, such as a conference proceedings volume for example, it is very common to have one bibliography for each article rather than a global one for the entire book. In the example below, each article would be presented as a separate \cmd{chapter} with its own bibliography.

在由多位作者所写文章的合集中,例如会议文集的一卷,非常常见的做法是对每篇文章而不是对整本书分别做文献索引。
在以下的例子中,每篇文章是不同的一章 \cmd{chapter},并带有自己的文献索引。

\begin{ltxexample}
\documentclass{...}
\usepackage{biblatex}
\addbibresource{...}
\begin{document}
\chapter{...}
<<\begin{refsection}>>
...
<<\printbibliography[heading=subbibliography]>>
<<\end{refsection}>>
\chapter{...}
<<\begin{refsection}>>
...
<<\printbibliography[heading=subbibliography]>>
<<\end{refsection}>>
\end{document}
\end{ltxexample}
%
%If \cmd{printbibliography} is used inside a \env{refsection} environment, it automatically restricts the scope of the list of references to the enclosing \env{refsection} environment. For a cumulative bibliography which is subdivided by chapter but printed at the end of the book, use the \opt{section} option of \cmd{printbibliography} to select a reference section, as shown in the next example.
如果 \cmd{printbibliography} 在 \env{refsection} 环境内部使用,
它会自动将文献列表范围限制在 \env{refsection} 环境内。
对于在一本书末尾列出但是按照每一章划分的累积参考文献,
使用 \cmd{printbibliography} 的 \opt{section} 选项来选择参考文献分节,如下面的例子所示。

\begin{ltxexample}
\documentclass{...}
\usepackage{biblatex}
<<\defbibheading>>{<<subbibliography>>}{%
  \section*{References for Chapter \ref{<<refsection:\therefsection>>}}}
\addbibresource{...}
\begin{document}
\chapter{...}
<<\begin{refsection}>>
...
<<\end{refsection}>>
\chapter{...}
<<\begin{refsection}>>
...
<<\end{refsection}>>
\printbibheading
<<\printbibliography>>[<<section=1>>,<<heading=subbibliography>>]
<<\printbibliography>>[<<section=2>>,<<heading=subbibliography>>]
\end{document}
\end{ltxexample}
%
%Note the definition of the bibliography heading in the above example. This is the definition taking care of the subheadings in the bibliography. The main heading is generated with a plain \cmd{chapter} command in this case. The \biblatex package automatically sets a label at the beginning of every \env{refsection} environment, using the standard \cmd{label} command. The identifier used is the string \texttt{refsection:} followed by the number of the respective \env{refsection} environment. The number of the current section is accessible via the \cnt{refsection} counter. When using the \opt{section} option of \cmd{printbibliography}, this counter is also set locally. This means that you may use the counter in heading definitions to print subheadings like «References for Chapter 3», as shown above. You could also use the title of the respective chapter as a subheading by loading the \sty{nameref} package and using \cmd{nameref} instead of \cmd{ref}:
请注意上面例子中文献标题的定义。
该定义考虑到了参考文献中的子标题。
主标题由普通的 \cmd{chapter} 生成。
\biblatex 宏包会自动在每个相应的 \env{refsection} 环境开始处用标准 \cmd{label} 命令分别设置标签。
其标识符是字符串 \texttt{refsection:} 接上 \env{refsection} 环境的序数。
当前节的序号可以通过 \cnt{refsection} 计数器获得。
当使用 \cmd{printbibliography} 的 \opt{section} 选项时,该计数器也被设置为局部的。
这意味着你可以在标题定义中使用该计数器来打印类似于上面例子中“References for Chapter 3”这样的子标题。
你也可以通过载入 \sty{nameref} 宏包和使用 \cmd{nameref} 代替 \cmd{ref} 来使用相应的章名作为子标题:

\begin{ltxexample}
\usepackage{<<nameref>>}
\defbibheading{subbibliography}{%
  \section*{<<\nameref{refsection:\therefsection}>>}}
\end{ltxexample}
%
%Since giving one \cmd{printbibliography} command for each part of a subdivided bibliography is tedious, \biblatex provides a shorthand. The \cmd{bibbysection} command automatically loops over all reference sections. This is equivalent to giving one \cmd{printbibliography} command for every section but has the additional benefit of automatically skipping sections without references. In the example above, the bibliography would then be generated as follows:
为参考文献的每一个子部分都给出 \cmd{printbibliography} 是很繁琐的,所以 \biblatex 提供了一个缩写语。
\cmd{bibbysection} 命令会自动遍历所有的参考文献分节。
这等价于为每节给出一个 \cmd{printbibliography} 命令,此外还会自动跳过没有文献的节。
在上面的例子中,参考文献可以按如下方式生成:

\begin{ltxexample}
\printbibheading
<<\bibbysection[heading=subbibliography]>>
\end{ltxexample}
%
%When using a format with one cumulative bibliography subdivided by chapter (or any other document division) it may be more appropriate to use \env{refsegment} rather than \env{refsection} environments. The difference is that the \env{refsection} environment generates labels local to the environment while \env{refsegment} does not affect the generation of labels, hence they will be unique across the entire document. The next example could also be given in \secref{use:use:div} because, visually, it creates one global bibliography subdivided into multiple segments.
当使用按章(或其它文档单元)划分的累积参考文献格式时,使用 \env{refsegment} 比 \env{refsection} 环境更合适一些。
不同之处在于 \env{refsection} 环境生成的标签是环境局部的,
而 \env{refsegment} 环境不影响标签生成,因此在整个文档中是唯一的。
下面的例子也可以在 \secref{use:use:div} 节中给出,因为它看起来创建了一个划分为多重片段的全局参考文献。

\begin{ltxexample}
\documentclass{...}
\usepackage{biblatex}
<<\defbibheading>>{<<subbibliography>>}{%
  \section*{References for Chapter \ref{<<refsegment:\therefsection\therefsegment>>}}}
\addbibresource{...}
\begin{document}
\chapter{...}
<<\begin{refsegment}>>
...
<<\end{refsegment}>>
\chapter{...}
<<\begin{refsegment}>>
...
<<\end{refsegment}>>
\printbibheading
<<\printbibliography>>[<<segment=1>>,<<heading=subbibliography>>]
<<\printbibliography>>[<<segment=2>>,<<heading=subbibliography>>]
\end{document}
\end{ltxexample}
%
%The use of \env{refsegment} is similar to \env{refsection} and there is also a corresponding \opt{segment} option for \cmd{printbibliography}. The \biblatex package automatically sets a label at the beginning of every \env{refsegment} environment using the string \texttt{refsegment:} followed by the number of the respective \env{refsegment} environment as an identifier. There is a matching \cnt{refsegment} counter which may be used in heading definitions, as shown above. As with reference sections, there is also a shorthand command which automatically loops over all reference segments:
\env{refsegment} 的使用类似于 \env{refsection},也有对应于 \cmd{printbibliography} 的 \opt{segment} 选项。
\biblatex 宏包自动在每个 \env{refsegment} 环境开始用字符串 \texttt{refsegment:}
后接相应 \env{regsegment} 环境的序号来设置标签作为标识符。
如前所述,有一个匹配的 \cnt{refsegment} 计数器可以用在标题定义中。
对于文献节,也有缩写名可以自动遍历所有的文献片段:

\begin{ltxexample}
\printbibheading
<<\bibbysegment[heading=subbibliography]>>
\end{ltxexample}
%
%This is equivalent to giving one \cmd{printbibliography} command for every segment in the current \env{refsection}.
这等价于为当前 \env{refsection} 的每个片段分别给出 \cmd{printbibliography} 命令。

\subsubsection{文献表划分}%\subsubsection{Subdivided Bibliographies}
\label{use:use:div}

%It is very common to subdivide a bibliography by certain criteria. For example, you may want to list printed and online resources separately or divide a bibliography into primary and secondary sources. The former case is straightforward because you can use the entry type as a criterion for the \opt{type} and \opt{nottype} filters of \cmd{printbibliography}. The next example also demonstrates how to generate matching subheadings for the two parts of the bibliography.

依照某一标准进行文献划分是非常普遍的。
例如,你也许需要分别列出印刷和网络资源,或者将参考文献分为主要和次要类型。
前一种情况比较简单,
因为可以使用条目类型作为 \cmd{printbibliography} 的 \opt{type} 和 \opt{nottype} 过滤的标准。
下面的例子也演示了如何为参考文献的两部分生成匹配的子标题。

\begin{ltxexample}
\documentclass{...}
\usepackage{biblatex}
\addbibresource{...}
\begin{document}
...
\printbibheading
\printbibliography[<<nottype=online>>,heading=subbibliography,
                   <<title={Printed Sources}>>]
\printbibliography[<<type=online>>,heading=subbibliography,
                   <<title={Online Sources}>>]

\end{document}
\end{ltxexample}
%
%You may also use more than two subdivisions:
也可以使用两个以上的划分:

\begin{ltxexample}
\printbibliography[<<type=article>>,...]
\printbibliography[<<type=book>>,...]
\printbibliography[<<nottype=article>>,<<nottype=book>>,...]
\end{ltxexample}
%
%It is even possible to give a chain of different types of filters:
甚至可以给出一组的不同类型的 filter:

\begin{ltxexample}
\printbibliography[<<section=2>>,<<type=book>>,<<keyword=abc>>,<<notkeyword=xyz>>]
\end{ltxexample}
%
%This would print all works cited in reference section~2 whose entry type is \bibtype{book} and whose \bibfield{keywords} field includes the keyword <abc> but not <xyz>. When using bibliography filters in conjunction with a numeric style, see \secref{use:cav:lab}. If you need complex filters with conditional expressions, use the \opt{filter} option in conjunction with a custom filter defined with \cmd{defbibfilter}. See \secref{use:bib:flt} for details on custom filters.
这会打印出所有在第二参考分节中条目类型为 \bibtype{book}
并且 \bibfield{keywords} 域包括关键词“abc”但是不包括“xyz”的作品。
关于结合数值样式使用文献过滤见 \secref{use:cav:lab} 节。
如果你需要带有条件表达式的复杂过滤器,
可以使用 \opt{filter} 选项结合由 \cmd{defbibfilter} 定义的定制过滤器。
关于定制过滤器详见 \secref{use:bib:flt} 节。

\begin{ltxexample}
\documentclass{...}
\usepackage{biblatex}
\addbibresource{...}
\begin{document}
...
\printbibheading
\printbibliography[<<keyword=primary>>,heading=subbibliography,%
                   <<title={Primary Sources}>>]
\printbibliography[<<keyword=secondary>>,heading=subbibliography,%
                   <<title={Secondary Sources}>>]
\end{document}
\end{ltxexample}
%
%Dividing a bibliography into primary and secondary sources is possible with a \bibfield{keyword} filter, as shown in the above example. In this case, with only two subdivisions, it would be sufficient to use one keyword as filter criterion:
如上例所示,将参考文献分为主要和次要部分可以通过 \bibfield{keyword} 过滤器实现。
在该情况下(只分成两部分),使用一个关键词作为过滤器标准就足够了:

\begin{ltxexample}
\printbibliography[<<keyword=primary>>,...]
\printbibliography[<<notkeyword=primary>>,...]
\end{ltxexample}
%
%Since \biblatex has no way of knowing if an item in the bibliography is considered to be primary or secondary literature, we need to supply the bibliography filter with the required data by adding a \bibfield{keywords} field to each entry in the \file{bib} file. These keywords may then be used as targets for the \opt{keyword} and \opt{notkeyword} filters, as shown above. It may be a good idea to add such keywords right away while building a \file{bib} file.
由于 \biblatex 无法知道文献中的某一条是否被认为是主要或者次要文献,
我们需要在 \file{bib} 文件中为每一条目增加 \bibfield{keywords} 域来提供文献 过滤器所需的数据。
如上例所示,这些关键词可以用于 \opt{keyword} 和 \opt{notkeyword} 过滤器的目标。
在建立 \file{bib} 文件时就添加这样的关键词是一个不错的办法。

\begin{lstlisting}[style=bibtex]{}
@Book{key,
  <<keywords>>      = {<<primary>>,some,other,keywords},
  ...
\end{lstlisting}
%
%An alternative way of subdividing the list of references are bibliography categories. They differ from the keywords"=based approach shown in the example above in that they work on the document level and do not require any changes to the \file{bib} file.
另外一种划分文献列表的方法是使用参考文献类别。
这与上述例子中使用的基于关键词的方法的不同之处在于,
它们在文档水平处理而并不需要修改 \file{bib} 文件。

\begin{ltxexample}
\documentclass{...}
\usepackage{biblatex}
<<\DeclareBibliographyCategory>>{<<primary>>}
<<\DeclareBibliographyCategory>>{<<secondary>>}
<<\addtocategory>>{<<primary>>}{key1,key3,key6}
<<\addtocategory>>{<<secondary>>}{key2,key4,key5}
\addbibresource{...}
\begin{document}
...
\printbibheading
\printbibliography[<<category=primary>>,heading=subbibliography,%
                   <<title={Primary Sources}>>]
\printbibliography[<<category=secondary>>,heading=subbibliography,%
                   <<title={Secondary Sources}>>]
\end{document}
\end{ltxexample}
%
%In this case it would also be sufficient to use one category only:
在这个例子中,只使用一个类别也是可以的:

\begin{ltxexample}
\printbibliography[<<category=primary>>,...]
\printbibliography[<<notcategory=primary>>,...]
\end{ltxexample}
%
%It is still a good idea to declare all categories used in the bibliography explicitly because there is a \cmd{bibbycategory} command which automatically loops over all categories. This is equivalent to giving one \cmd{printbibliography} command for every category, in the order in which they were declared.
不过,显式地声明参考文献中使用的所有类别仍然是个不错的主意,
因为有一个 \cmd{bibbycategory} 命令能自动遍历所有的类别。
这等价于为每一类别按照所声明的顺序依次给出 \cmd{printbibliography} 命令。

\begin{ltxexample}
\documentclass{...}
\usepackage{biblatex}
<<\DeclareBibliographyCategory>>{<<primary>>}
<<\DeclareBibliographyCategory>>{<<secondary>>}
\addtocategory{primary}{key1,key3,key6}
\addtocategory{secondary}{key2,key4,key5}
<<\defbibheading>>{<<primary>>}{\section*{Primary Sources}}
<<\defbibheading>>{<<secondary>>}{\section*{Secondary Sources}}
\addbibresource{...}
\begin{document}
...
\printbibheading
<<\bibbycategory>>
\end{document}
\end{ltxexample}
%
%The handling of the headings is different from \cmd{bibbysection} and \cmd{bibbysegment} in this case. \cmd{bibbycategory} uses the name of the current category as a heading name. This is equivalent to passing \texttt{heading=\prm{category}} to \cmd{printbibliography} and implies that you need to provide a matching heading for every category.
在这个例子中,标题的处理与 \cmd{bibbysection} 和 \cmd{bibbysegment} 是不同的。
\cmd{bibbycategory} 使用当前类别的名字作为标题名。
这等价于将 \texttt{heading=\prm{category}} 传递给 \cmd{printbibliography},
从而意味着你需要为每一类别提供相匹配的标题。

\subsubsection{条目集}%\subsubsection{Entry Sets}
\label{use:use:set}

%An entry set is a group of entries which are cited as a single reference and listed as a single item in the bibliography. The individual entries in the set are separated by \cmd{entrysetpunct} (\secref{aut:fmt:fmt}). The \biblatex package supports two types of entry sets. Static entry sets are defined in the \file{bib} file like any other entry. Dynamic entry sets are defined with \cmd{defbibentryset} (\secref{use:bib:set}) on a per-document\slash per-refsection basis in the document preamble or the document body. This section deals with the definition of entry sets; style authors should also see \secref{aut:cav:set} for further information.

条目集是用单个引用并在参考文献中作为一项列出的一组条目。
条目集中每一项用 \cmd{entrysetpunct} 分隔(\secref{aut:fmt:fmt} 节)。
\biblatex 宏包支持两种类型的条目集。
静态条目集在 \file{bib} 文件中定义,这与其它条目类似。
而动态条目集在文档导言区或者正文中用 \cmd{defbibentryset} (\secref{use:bib:set})定义,
并且基于文档或参考文献分节。
本节讨论条目集的定义问题;对于样式作者更多信息可参考 \secref{aut:cav:set} 节。

\paragraph{静态条目集}%\paragraph{Static entry sets}

%Static entry sets are defined in the \file{bib} file like any other entry. Defining an entry set is as simple as adding an entry of type \bibtype{set}. The entry has an \bibfield{entryset} field defining the members of the set as a separated list of entry keys:

静态条目集如同其它条目一样在 \file{bib} 文件中定义。
定义这样的条目集只需添加一个类型为 \bibtype{set} 的条目。
该条目有一个 \bibfield{entryset} 域,其中使用条目键值的逗号分隔列表定义了条目集的元素:

\begin{lstlisting}[style=bibtex]{}
<<@Set>>{<<set1>>,
  <<entryset>> = {<<key1,key2,key3>>},
}
\end{lstlisting}
%
%Entries may be part of a set in one document\slash refsection and stand-alone references in another one, depending on the presence of the \bibtype{set} entry. If the \bibtype{set} entry is cited, the set members are grouped automatically. If not, they will work like any regular entry.
条目可以是文档或参考文件分节中一个集合的一部分,或者是另外一个条目集中的孤立文献,这取决于 \bibtype{set} 条目。
如果 \bibtype{set} 条目被引用,其中的成员自动分成一组。否则它们就像其它的常规条目一样。

%\paragraph[Dynamic entry sets]{Dynamic entry sets}
\paragraph[动态条目集]{动态条目集}

%Dynamic entry sets are set up and work much like static ones. The main difference is that they are defined in the document preamble or on the fly in the document body using the \cmd{defbibentryset} command from \secref{use:bib:set}:

动态条目集的设置和运行和静态条目集很相似。
主要的区别是,它们是在导言区或者实时地在文档中使用 \secref{use:bib:set} 节的 \cmd{defbibentryset} 命令来定义的:

\begin{lstlisting}[style=bibtex]{}
\defbibentryset{set1}{key1,key2,key3}
\end{lstlisting}
%
%Dynamic entry sets in the document body are local to the enclosing \env{refsection} environment, if any. Otherwise, they are assigned to reference section~0. Those defined in the preamble are assigned to reference section~0.
正文中的动态条目集在其所在的 \env{refsection} 环境中是局部的(如果有的话)。
否则它们被分配给第零文献分节。
定义在导言区的动态条目集也被分在第零文献节。

\subsubsection[数据容器]{数据容器}%\subsubsection[Data Containers]{Data Containers}
\label{use:use:xdat}

%The \bibtype{xdata} entry type serves as a data container holding one or more fields. These fields may be inherited by other entries using the \bibfield{xdata} field. \bibtype{xdata} entries may not be cited or added to the bibliography, they only serve as a data source for other entries. This data inheritance mechanism is useful for fixed field combinations such as \bibfield{publisher}\slash \bibfield{location} and for other frequently used data:

作为数据容器,\bibtype{xdata} 条目类型可以包含一个或更多域。
这些域可以被其它条目使用 \bibfield{xdata} 来继承。
\bibtype{xdata} 条目可以不被引用或者打印在参考文献中,它们只为其它条目提供数据源。
这种数据继承机制常用于 \bibfield{publisher}\slash \bibfield{location} 这样的固定域组合或者其它常用数据:

\begin{lstlisting}[style=bibtex]{}
<<@XData>>{<<hup>>,
  publisher  = {Harvard University Press},
  location   = {Cambridge, Mass.},
}
@Book{...,
  author     = {...},
  title	     = {...},
  date	     = {...},
  <<xdata>>      = {<<hup>>},
}
\end{lstlisting}
%
%Using a separated list of keys in its \bibfield{xdata} field, an entry may inherit data from several \bibtype{xdata} entries. Cascading \bibtype{xdata} entries are supported as well, \ie an \bibtype{xdata} entry may reference one or more other \bibtype{xdata} entries:
一个条目通过在 \bibfield{xdata} 域中使用分隔键列表,可以继承若干个 \bibtype{xdata} 条目的数据。
\bibtype{xdata} 条目的串联也是支持的,
即,一个 \bibtype{xdata} 条目可以涉及到一个或更多其它 \bibtype{xdata} 条目:

\begin{lstlisting}[style=bibtex]{}
@XData{macmillan:name,
  publisher  = {Macmillan},
}
@XData{macmillan:place,
  location   = {New York and London},
}
@XData{macmillan,
  xdata      = {macmillan:name,macmillan:place},
}
@Book{...,
  author     = {...},
  title	     = {...},
  date	     = {...},
  xdata	     = {macmillan},
}
\end{lstlisting}
%
%See also \secref{bib:typ:blx,bib:fld:spc}.
另见 \secref{bib:typ:blx,bib:fld:spc} 节。

\subsubsection{电子出版信息}%\subsubsection{Electronic Publishing Information}
\label{use:use:epr}

%The \biblatex package provides three fields for electronic publishing information: \bibfield{eprint}, \bibfield{eprinttype}, and \bibfield{eprintclass}. The \bibfield{eprint} field is a verbatim field similar to \bibfield{doi} which holds the identifier of the item. The \bibfield{eprinttype} field holds the resource name, \ie the name of the site or electronic archive. The optional \bibfield{eprintclass} field is intended for additional information specific to the resource indicated by the \bibfield{eprinttype} field. This could be a section, a path, classification information, etc. If the \bibfield{eprinttype} field is available, the standard styles will use it as a literal label. In the following example, they would print «Resource: identifier» rather than the generic «eprint: identifier»:

\biblatex 宏包为电子出版信息提供了三种域:\bibfield{eprint}、\bibfield{eprinttype} 和 \bibfield{eprintclass}。
\bibfield{eprint} 域类似于 \bibfield{doi},是一种保持项目标识符的抄录模式域。
\bibfield{eprinttype} 域保存资源名称,即网址或电子档案的名称。
可选的 \bibfield{eprintclass} 域用于标明特定于 \bibfield{eprinttype} 域所指资源的额外信息。
这可以是章节、路径、分类信息等。
如果 \bibfield{eprinttype} 可用,标准样式会将其当做文本标签使用。
在以下例子中,它们会打印“Resource: identifier”而不是一般的“eprint: identifier”:

\begin{lstlisting}[style=bibtex]{}
<<eprint>>     = {<<identifier>>},
<<eprinttype>> = {<<Resource>>},
\end{lstlisting}
%
%The standard styles feature dedicated support for a few online archives. For arXiv references, put the identifier in the \bibfield{eprint} field and the string \texttt{arxiv} in the \bibfield{eprinttype} field:
标准样式对一些在线资源提供了专门支持。
对于 arXiv 文献,将标识符放在 \bibfield{eprint} 域中,将字符串 \texttt{arxiv} 放在 \bibfield{eprinttype} 域中:

\begin{lstlisting}[style=bibtex]{}
<<eprint>>     = {<<math/0307200v3>>},
<<eprinttype>> = {<<arxiv>>},
\end{lstlisting}
%
%For papers which use the new identifier scheme (April 2007 and later) add the primary classification in the \bibfield{eprintclass} field:
对于使用新标识格式的文章(2007年四月之后),将主分类放在 \bibfield{eprintclass} 域中:

\begin{lstlisting}[style=bibtex]{}
eprint      = {1008.2849v1},
eprinttype  = {arxiv},
<<eprintclass>> = {<<cs.DS>>},
\end{lstlisting}
%
%There are two aliases which ease the integration of arXiv entries. \bibfield{archiveprefix} is treated as an alias for \bibfield{eprinttype}; \bibfield{primaryclass} is an alias for \bibfield{eprintclass}. If hyperlinks are enabled, the eprint identifier will be transformed into a link to \nolinkurl{arxiv.org}. See the package option \opt{arxiv} in \secref{use:opt:pre:gen} for further details.
为了方便 arXiv 条目的整合专门设置了两个别称。
\bibfield{archiveprefix} 是 \bibfield{eprinttype} 的别称;
而 \bibfield{primaryclass} 是 \bibfield{eprintclass} 的别称。
如果启用超链接,\bibfield{eprint} 标识符将转换为指向 \nolinkurl{arxiv.org} 的链接。
更多信息可参见 \secref{use:opt:pre:gen} 节中的宏包选项 \opt{arxiv}。

%For \acr{JSTOR} references, put the stable \acr{JSTOR} number in the \bibfield{eprint} field and the string \texttt{jstor} in the \bibfield{eprinttype} field:
对于 \acr{JSTOR} 资源,将稳定的 \acr{JSTOR} 号放在 \bibfield{eprint} 域中,将字符串 \texttt{jstor} 放在 \bibfield{eprinttype} 域中:

\begin{lstlisting}[style=bibtex]{}
<<eprint>>     = {<<number>>},
<<eprinttype>> = {<<jstor>>},
\end{lstlisting}
%
%When using \acr{JSTOR}'s export feature to export citations in \bibtex format, \acr{JSTOR} uses the \bibfield{url} field by default (where the \prm{number} is a unique and stable identifier):
当使用 \acr{JSTOR} 的导出功能来导出 \BibTeX 格式引用时,
\acr{JSTOR} 缺省使用 \bibfield{url} 域(当 \prm{number} 是唯一稳定标识符时):

\begin{lstlisting}[style=bibtex]{}
url = {http://www.jstor.org/stable/<<number>>},
\end{lstlisting}
%
%While this will work as expected, full \acr{URL}s tend to clutter the bibliography. With the \bibfield{eprint} fields, the standard styles will use the more readable «\acr{JSTOR}: \prm{number}» format which also supports hyperlinks. The \prm{number} becomes a clickable link if \sty{hyperref} support is enabled.
尽管这样可以运行,但整个的 \acr{URL} 会使参考文献变得杂乱无章。
而使用 \bibfield{eprint} 域,标准样式会使用更加可读的“\acr{JSTOR}: \prm{number}” 格式而且同样支持超链接。
当启用 \sty{hyperref} 支持时,\prm{number} 会变成可以点击的链接。

%For PubMed references, put the stable PubMed identifier in the \bibfield{eprint} field and the string \texttt{pubmed} in the \bibfield{eprinttype} field. This means that:
对于 PubMed 资源,将稳定的 PubMed 标识符放在 \bibfield{eprint} 域中,将字符串 \texttt{pubmed} 放在 \bibfield{eprinttype} 域中。
也就是

\begin{lstlisting}[style=bibtex]{}
url = {http://www.ncbi.nlm.nih.gov/pubmed/<<pmid>>},
\end{lstlisting}
%
%becomes:
会变成:

\begin{lstlisting}[style=bibtex]{}
<<eprint>>     = {<<pmid>>},
<<eprinttype>> = {<<pubmed>>},
\end{lstlisting}
%
%and the standard styles will print «\acr{PMID}: \prm{pmid}» instead of the lengthy \acr{URL}. If hyperref support is enabled, the \prm{pmid} will be a clickable link to PubMed.
并且标准样式会打印出“\acr{PMID}: \prm{pmid}” 来取代冗长的 \acr{URL}。
如果启用 \sty{hyperref} 支持,\prm{pmid} 会变成指向 PubMed 的可点击的链接。

%For handles (\acr{HDL}s), put the handle in the \bibfield{eprint} field and the string \texttt{hdl} in the \bibfield{eprinttype} field:
对于句柄系统\footnote{%
参考 \url{http://www.handle.net/}——译注}
(\acr{HDL}),将句柄放在 \bibfield{eprint} 域中,
将字符串 \texttt{hdl} 放在 \bibfield{eprinttype} 域中:

\begin{lstlisting}[style=bibtex]{}
<<eprint>>     = {<<handle>>},
<<eprinttype>> = {<<hdl>>},
\end{lstlisting}
%
%For Google Books references, put Google's identifier in the \bibfield{eprint} field and the string \texttt{googlebooks} in the \bibfield{eprinttype} field. This means that, for example:
对于 Google Books 资源,将Google标识符放在 \bibfield{eprint} 域中,
将字符串 \texttt{googlebooks} 放在 \bibfield{eprinttype} 域中。
如下例。

\begin{lstlisting}[style=bibtex]{}
url = {http://books.google.com/books?id=<<XXu4AkRVBBoC>>},
\end{lstlisting}
%
%would become:
会变成:

\begin{lstlisting}[style=bibtex]{}
<<eprint>>     = {<<XXu4AkRVBBoC>>},
<<eprinttype>> = {<<googlebooks>>},
\end{lstlisting}
%
%and the standard styles would print «Google Books: |XXu4AkRVBBoC|» instead of the full \acr{URL}. If hyperref support is enabled, the identifier will be a clickable link to Google Books.\footnote{Note that the Google Books \acr{id} seems to be a bit of an <internal> value. As of this writing, there does not seem to be any way to search for an \acr{id} on Google Books. You may prefer to use the \bibfield{url} in this case.}
并且标准样式会打印出“Google Books: |XXu4AkRVBBoC|”代替整个 \acr{URL}。
如果启用了 \sty{hyperref} 支持,该标识符会变成指向 Google Books 的可点击的链接。\footnote{ %
	请注意,Google Books \acr{id} 似乎是一个“内部”值。
	从这份手册开始,似乎没有办法在Google Books 上搜索 \acr{id}。
	此时也许最好使用 \bibfield{url} 域。%
}

%Note that \bibfield{eprint} is a verbatim field. Always give the identifier in its unmodified form. For example, there is no need to replace |_| with |\_|. Also see \secref{aut:cav:epr} on how to add dedicated support for other eprint resources.

请注意 \bibfield{eprint} 是一个抄录模式域,故而总是以未修改的形式给出标识符。
例如没有必要将 |_| 改成 |\_|。
对于如何为其它电子出版资源增加细致的支持,也可以参考 \secref{aut:cav:epr} 节。

\subsubsection{外部摘要和注释}%\subsubsection{External Abstracts and Annotations}
\label{use:use:prf}

%Styles which print the fields \bibfield{abstract} and\slash or \bibfield{annotation} may support an alternative way of adding abstracts or annotations to the bibliography. Instead of including the text in the \file{bib} file, it may also be stored in an external \latex file. For example, instead of saying

打印 \bibfield{abstract} 和/或  \bibfield{annotation} 域的样式可以支持另一种将摘要或注释添加到参考文献的方法。
与将文本包含在 \file{bib} 文件中不同,它也可以保存在一个外部的 \LaTeX 文件中。
例如,除了在 \file{bib} 文件中写入如下内容之外,

\begin{ltxexample}[style=bibtex]
@Article{key1,
  ...
  abstract	  = {This is an abstract of entry `key1'.}
}
\end{ltxexample}
%
%in the \file{bib} file, you may create a file named \path{bibabstract-key1.tex} and put the abstract in this file:
你也可以创建一个名为 \path{bibabstract-key1.tex} 的文件并将摘要放在该文件中:

\begin{ltxexample}
This is an abstract of entry `key1'.
\endinput
\end{ltxexample}
%
%The name of the external file must be the entry key prefixed with \texttt{bibabstract-} or \texttt{bibannotation-}, respectively. You can change these prefixes by redefining \cmd{bibabstractprefix} and \cmd{bibannotationprefix}. Note that this feature needs to be enabled explicitly by setting the package option \opt{loadfiles} from \secref{use:opt:pre:gen}. The option is disabled by default for performance reasons. Also note that any \bibfield{abstract} and \bibfield{annotation} fields in the \file{bib} file take precedence over the external files. Using external files is strongly recommended if you have long abstracts or a lot of annotations since this may increase memory requirements significantly. It is also more convenient to edit the text in a dedicated \latex file. Style authors should see \secref{aut:cav:prf} for further information.
外部文件名必须是条目键分别加上前缀\texttt{bibabstract-} 或 \texttt{bibannotation-}。
你可以通过重定义 \cmd{bibabstractprefix} 和 \cmd{bibannotationprefix} 来改变这些前缀。
请注意,该特性需要通过显式地设置 \secref{use:opt:pre:gen} 中的宏包选项 \opt{loadfiles} 来启用。
缺省情况下出于性能原因该选项是关闭的。
同样要注意的是,\file{bib} 文件中的任何 \bibfield{abstract} 和 \bibfield{annotation} 域都优先于外部文件。
如果你的摘要或注释很长(这会显著增加内存需求),那么强烈推荐使用外部文件。
此外,在专门的 \LaTeX 文件中编辑文本也是很方便的。
样式作者就更多信息应该参考 \secref{aut:cav:prf} 节。

%\subsection{Hints and Caveats}
\subsection{提示和注意事项}
\label{use:cav}

%This section provides additional usage hints and addresses some common problems and potential misconceptions.
本节提供了其它一些使用提示,并介绍了一些常见问题和可能的错误概念。

%\subsubsection{Usage with \acr{KOMA}-Script Classes}
\subsubsection{与 \acr{KOMA}-Script 文档类共用}
\label{use:cav:scr}

%When using \biblatex in conjunction with one of the \sty{scrbook}, \sty{scrreprt}, or \sty{scrartcl} classes, the headings \texttt{bibliography} and \texttt{biblist} from \secref{use:bib:hdg} are responsive to the bibliography"=related options of these classes.\footnote{This applies to the traditional syntax of these options (\opt{bibtotoc} and \opt{bibtotocnumbered}) as well as to the \keyval syntax introduced in \acr{KOMA}-Script 3.x, \ie to \kvopt{bibliography}{nottotoc}, \kvopt{bibliography}{totoc}, and \kvopt{bibliography}{totocnumbered}. The global \kvopt{toc}{bibliography} and \kvopt{toc}{bibliographynumbered} options as well as their aliases are detected as well. In any case, the options must be set globally in the optional argument to \cmd{documentclass}.} You can override the default headings by using the \opt{heading} option of \cmd{printbibliography}, \cmd{printbibheading} and \cmd{printbiblist}. See \secref{use:bib:bib, use:bib:biblist, use:bib:hdg} for details. All default headings are adapted at load-time such that they blend with the behavior of these classes. If one of the above classes is detected, \biblatex will also provide the following additional tests which may be useful in custom heading definitions:

当在 \sty{scrbook}、\sty{scrreprt} 或 \sty{scrartcl} 文档类中使用 \biblatex 时,
\secref{use:bib:hdg} 节中的标题 \texttt{bibliography} 和 \texttt{shorthands} 会与这些文档类的文献相关选项有关。\footnote{%
	这既可以应用到传统的选项语法(\opt{bibtotoc} 和 \opt{bibtotocnumbered}),
	也可以应用到 \acr{KOMA}-Script 3.x 引入的 \keyval 语法,
	即 \kvopt{bibliography}{nottotoc}、\kvopt{bibliography}{totoc} 和 \kvopt{bibliography}{totocnumbered}。
	全局的 \kvopt{toc}{bibliography} 和 \kvopt{toc}{bibliographynumbered} 以及它们的别称也会检测到。
	在任何一种情况下,选项必须在 \cmd{documentclass} 的可选项中全局地设置。
}
可以通过使用 \cmd{printbibliography}、\cmd{printbibheading} 和 \cmd{printshorthands} 的 \opt{heading} 选项来覆盖缺省标题。
详见 \secref{use:bib:bib, use:bib:hdg} 节。
所有的缺省标题都在导入时进行调整以使其与这些文档类相称。
如果 \biblatex 探测到这些文档类中的某一个,它还会提供以下额外的测试,这对定制标题定义很有用:

\begin{ltxsyntax}

\cmditem{ifkomabibtotoc}{true}{false}

%Expands to \prm{true} if the class would add the bibliography to the table of contents, and to \prm{false} otherwise.

如果该文档类将参考文献加入目录中则展开为 \prm{true},否则为 \prm{false}。

\cmditem{ifkomabibtotocnumbered}{true}{false}

%Expands to \prm{true} if the class would add the bibliography to the table of contents as a numbered section, and to \prm{false} otherwise. If this test yields \prm{true}, \cmd{ifkomabibtotoc} will always yield \prm{true} as well, but not vice versa.

如果该文档类将参考文献加入目录作为带编号的一节,则展开为 \prm{true},否则为 \prm{false}。
如果该测试结果为 \prm{true},那么 \cmd{ifkomabibtotoc} 也总是为 \prm{true},但反过来不一定。

\end{ltxsyntax}

\subsubsection{与 Memoir 文档类共用}%\subsubsection{Usage with the Memoir Class}
\label{use:cav:mem}

%When using \biblatex with the \sty{memoir} class, most class facilities for adapting the bibliography have no effect. Use the corresponding facilities of this package instead (\secref{use:bib:bib, use:bib:hdg, use:bib:nts}). Instead of redefining \sty{memoir}'s \cmd{bibsection}, use the \opt{heading} option of \cmd{printbibliography} and \cmd{defbibheading} (\secref{use:bib:bib, use:bib:hdg}). Instead of \cmd{prebibhook} and \cmd{postbibhook}, use the \opt{prenote} and \opt{postnote} options of \cmd{printbibliography} and \cmd{defbibnote} (\secref{use:bib:bib, use:bib:nts}). All default headings are adapted at load-time such that they blend well with the default layout of this class. The default headings \texttt{bibliography} and \texttt{biblist} (\secref{use:bib:hdg}) are also responsive to \sty{memoir}'s \cmd{bibintoc} and \cmd{nobibintoc} switches. The length register \len{bibitemsep} is used by \biblatex in a way similar to \sty{memoir} (\secref{use:fmt:len}). This section also introduces some additional length registers which correspond to \sty{memoir}'s \cmd{biblistextra}. Lastly, \cmd{setbiblabel} does not map to a single facility of the \biblatex package since the style of all labels in the bibliography is controlled by the bibliography style. See \secref{aut:bbx:env} in the author section of this manual for details. If the \sty{memoir} class is detected, \biblatex will also provide the following additional test which may be useful in custom heading definitions:

当在 \sty{memoir} 文档类中使用 \biblatex 时,大部分调整参考文献的文档类工具都没有效果。
相反,请使用本宏包的相应工具(\secref{use:bib:bib, use:bib:hdg, use:bib:nts} 节)。
使用 \cmd{printbibliography} 和 \cmd{defbibheading}(\secref{use:bib:bib, use:bib:hdg} 节)的 \opt{heading} 选项,
而不要重定义 \sty{memoir} 的 \cmd{bibsection}。
使用 \cmd{printbibliography} 和 \cmd{defbibnote}(\secref{use:bib:bib, use:bib:nts} 节)的 \opt{prenote} 和 \opt{postnote} 来取代 \cmd{prebibhook} 和 \cmd{postbibhook}。
所有缺省标题都在导入时调整以与该文档类的缺省布局相称。
缺省的标题 \texttt{bibliography} 和 \texttt{shorthands} (\secref{use:bib:hdg} 节)也与 \sty{memoir} 的 \cmd{bibintoc} 和 \cmd{nobibintoc} 开关相对应。
长度计数器 \len{bibitemsep} 在 \biblatex 中的使用方法与在 \sty{memoir} 类似(\secref{use:fmt:len} 节)。
本节还解释额外一些对应于 \sty{memoir} 中 \cmd{biblistextra} 的长度计数器。
最后,\cmd{setbiblabel} 并不能对应于 \biblatex 宏包的某一单独工具,因为参考文献中所有标签的样式由参考文献样式控制。
详见本手册的 \secref{aut:bbx:env}。
如果 \biblatex 探测到 \sty{memoir} 文档类的使用,它还会提供以下额外的测试,这对定制标题定义很有用:

\begin{ltxsyntax}

\cmditem{ifmemoirbibintoc}{true}{false}

%Expands to \prm{true} or \prm{false}, depending on \sty{memoir}'s \cmd{bibintoc} and \cmd{nobibintoc} switches. This is a \latex frontend to \sty{memoir}'s \cmd{ifnobibintoc} test. Note that the logic of the test is reversed.

取决于 \sty{memoir} 的 \cmd{bibintoc} 和 \cmd{nobibintoc} 开关,可以展开为 \prm{true} 或 \prm{false}。
这是对应于 \sty{memoir} 的 \cmd{ifnobibintoc} 测试的 \LaTeX 前端。
请注意该测试的逻辑可以反过来。

\end{ltxsyntax}

\subsubsection{引用中的页码数}%\subsubsection{Page Numbers in Citations}
\label{use:cav:pag}

%If the \prm{postnote} argument to a citation command is a page number or page range, \biblatex will automatically prefix it with <p.> or <pp.> by default. This works reliably in typical cases, but sometimes manual intervention may be required. In this case, it is important to understand how this argument is handled in detail. First, \biblatex checks if the postnote is an Arabic or Roman numeral (case insensitive). If this test succeeds, the postnote is considered as a single page or other number which will be prefixed with <p.> or some other string which depends on the \sty{pagination} field (see \secref{bib:use:pag}). If it fails, a second test is performed to find out if the postnote is a range or a list of Arabic or Roman numerals. If this test succeeds, the postnote will be prefixed with <pp.> or some other string in the plural form. If it fails as well, the postnote is printed as is. Note that both tests expand the \prm{postnote}. All commands used in this argument must therefore be robust or prefixed with \cmd{protect}. Here are a few examples of \prm{postnote} arguments which will be correctly recognized as a single number, a range of numbers, or a list of numbers, respectively:

如果一个引用命令的 \prm{postnote} 选项是页码数或页码范围,
那么 \biblatex 会自动给其增加缺省前缀“p.”或“pp.”。
在通常情况下这很可靠,但有时也需要手动调整。
这时理解该选项怎样处理的细节就很重要。
首先 \biblatex 检查后注是否是阿拉伯或罗马数字(大小写不敏感)。
如果该测试成功,那么该后注会被认为是一个单独页码或者其它数字,
这时会被加上前缀“p.”或其它取决于 \bibfield{pagination} 域(见 \secref{bib:use:pag} 节)的字符串。
如果测试没有成功,那么会执行第二项测试来检测该后注是否是一个区间或者一列阿拉伯或罗马数字。
如果该测试成功,那么该后注会被加上前缀“pp.”或其它复数形式的字符串。
如果该测试也没有成功,该后注会依原样打印。
请注意这两项测试都会展开 \prm{postnote}。
因此所有在该选项中使用的命令都必须是鲁棒的或者用 \cmd{protect} 加以保护。
这里分别是一些 \prm{postnote} 选项会被正确识别为单独数字、数字范围或者一列数字的一些例子:

\begin{ltxexample}
\cite[25]{key}
\cite[vii]{key}
\cite[XIV]{key}
\cite[34--38]{key}
\cite[iv--x]{key}
\cite[185/86]{key}
\cite[XI \& XV]{key}
\cite[3, 5, 7]{key}
\cite[vii--x; 5, 7]{key}
\end{ltxexample}
%
%In some other cases, however, the tests may get it wrong and you need to resort to the auxiliary commands \cmd{pno}, \cmd{ppno}, and \cmd{nopp} from \secref{use:cit:msc}. For example, suppose a work is cited by a special pagination scheme consisting of numbers and letters. In this scheme, the string <|27a|> would mean <page~27, part~a>. Since this string does not look like a number or a range to \biblatex, you need to force the prefix for a single number manually:
然而在其它一些情况该测试会失败,
此时需要考虑 \secref{use:cit:msc} 节的一些辅助命令 \cmd{pno}、\cmd{ppno} 和 \cmd{nopp}。
例如,假设一部作品由一种包含数字和字母的特殊页码标记格式所引用。
在这种格式中,字符串“|27a|”的意思是“page~27, part~a”。
因为对于 \biblatex 来说该字符串并不像数字或者数字范围,因此你需要手动强制加上单独页码的前缀:

\begin{ltxexample}
\cite[\pno~27a]{key}
\end{ltxexample}
%
%There is also a \cmd{ppno} command which forces a range prefix as well as a \cmd{nopp} command which suppresses all prefixes:
同样地,\cmd{ppno} 命令会强制为范围前缀,而 \cmd{nopp} 命令会取消所有的前缀:

\begin{ltxexample}
\cite[\ppno~27a--28c]{key}
\cite[\nopp 25]{key}
\end{ltxexample}
%
%These commands may be used anywhere in the \prm{postnote} argument. They may also be used multiple times. For example, when citing by volume and page number, you may want to suppress the prefix at the beginning of the postnote and add it in the middle of the string:
这些命令可以用于 \prm{postnote} 选项的任何地方,也可以被多次使用。
例如,当以卷数和页码数引用时,你或许希望在后注的开始取消前缀,而在字符串的中间加上:

\begin{ltxexample}
\cite[VII, \pno~5]{key}
\cite[VII, \pno~3, \ppno~40--45]{key}
\cite[see][\ppno~37--46, in particular \pno~40]{key}
\end{ltxexample}
%
%There are also two auxiliary command for suffixes like <the following page(s)>. Instead of inserting such suffixes literally (which would require \cmd{ppno} to force a prefix):
还有两个用于形如“the following page(s)”的后缀的辅助命令。
使用辅助命令 \cmd{psq} 和 \cmd{psqq} 来代替用文本插入这样的后缀(这要求 \cmd{ppno} 来强加一个前缀):

\begin{ltxexample}
\cite[\ppno~27~sq.]{key}
\cite[\ppno~55~sqq.]{key}
\end{ltxexample}
%
%use the auxiliary commands \cmd{psq} and \cmd{psqq}. Note that there is no space between the number and the command. This space will be inserted automatically and may be modified by redefining the macro \cmd{sqspace}.
请注意数字和命令之前没有空格。该空格会自动插入并可以通过重定义 \cmd{sqspace} 宏来修改。

\begin{ltxexample}
\cite[27\psq]{key}
\cite[55\psqq]{key}
\end{ltxexample}
%
%Since the postnote is printed without any prefix if it includes any character which is not an Arabic or Roman numeral, you may also type the prefix manually:
由于当后注包括任何非阿拉伯或罗马数字时将会以不带任何前缀的方式打印,
也可以手动输入前缀:

\begin{ltxexample}
\cite[p.~5]{key}
\end{ltxexample}
%
%It is possible to suppress the prefix on a per"=entry basis by setting the \bibfield{pagination} field of an entry to <\texttt{none}>, see \secref{bib:use:pag} for details. If you do not want any prefixes at all or prefer to type them manually, you can also disable the entire mechanism in the document preamble or the configuration file as follows:
可以通过设置条目的 \bibfield{pagination} 域为 “\texttt{none}”来基于每一条目取消前缀,详见 \secref{bib:use:pag} 节。
如果你不需要任何前缀或者更想要手动输入,也可以在导言区或者配置文件中整个地关闭该机制,如下所示:

\begin{ltxexample}
\DeclareFieldFormat{postnote}{#1}
\end{ltxexample}
%
%The \prm{postnote} argument is handled as a field and the formatting of this field is controlled by a field formatting directive which may be freely redefined. The above definition will simply print the postnote as is. See \secref{aut:cbx:fld, aut:bib:fmt} in the author guide for further details.
\prm{postnote} 选项会像条目域一样处理,该域的格式由域格式指令来控制,而该指令可以自由地重定义。
以上的定义会简单地将后注依原样打印。
更多细节可以参见作者指南部分的 \secref{aut:cbx:fld, aut:bib:fmt} 节。

%\subsubsection{Name Parts and Name Spacing}
\subsubsection{姓名组成部分及其间距}
\label{use:cav:nam}

%The \biblatex package gives users and style authors very fine"=grained control of name spacing and the line"=breaking behavior of names. The commands discussed in the following are documented in \secref{use:fmt:fmt,aut:fmt:fmt}. This section is meant to give an overview of how they are put together. A note on terminology: a name \emph{part} is a basic part of the name, for example the given or the family name. Each part of a name may be a single name or it may be composed of multiple names. For example, the name part \enquote*{given name} may be composed of a given and a middle name. The latter are referred to as name \emph{elements} in this section. Let's consider a simple name first: \enquote{John Edward Doe}. This name is composed of the following parts:

\biblatex 宏包可以使用户和样式作者对姓名空格和换行进行精细的控制。
下面讨论的命令在本文档的 \secref{use:fmt:fmt,aut:fmt:fmt} 节。
本节的目的在于大致介绍如何将这些命令结合起来。
术语说明:
名成分(name \emph{part})是姓名中的各个基本部分,例如名(given name/first name) 或姓(family name/last name)。
姓名的每一成分可以是单个部分或者包含多个部分。
例如,名成分 \enquote*{given name/first name} 可以包含 given/first和 middle name。后者可以认为是本节所说的名元素(name \emph{elements})。
我们首先考虑一个简单的名字 \enquote{John Edward Doe},它包含了如下成分:

\begin{nameparts}
Given	& John Edward \\
Prefix	& --- \\
Family	& Doe \\
Suffix	& --- \\
\end{nameparts}
%
%The spacing, punctuation and line"=breaking behavior of names is controlled by six macros:
姓名中的空格、标点和断行行为由六个宏所控制:

\begin{namedelims}
a & \cmd{bibnamedelima} &
%Inserted by the backend after the first element of every name part if that element is less than three characters long and before the last element of every name part.
由后端程序插入在每一名部分的第一个元素后(如果该元素少于三个字符长度),在每一名部分的最后元素之前。\\
b & \cmd{bibnamedelimb} &
%Inserted by the backend between all elements of a name part where \cmd{bibnamedelima} does not apply.
由后端程序插入在名部分的元素之间且 \cmd{bibnamedelima} 没有使用之处。\\
c & \cmd{bibnamedelimc} &
%Inserted by a formatting directive between the name prefix and the family name if \kvopt{useprefix}{true}. If \kvopt{useprefix}{false}, \cmd{bibnamedelimd} is used instead.
当 \kvopt{useprefix}{true} 时,由格式指令插入在 name prefix 和 last name 之间。
如果 \kvopt{useprefix}{false},将使用 \cmd{bibnamedelimd}。\\
d & \cmd{bibnamedelimd} &
%Inserted by a formatting directive between name parts where \cmd{bibnamedelimc} does not apply.
由格式指令插入在名部分之间且 \cmd{bibnamedelimc} 没有使用之处。\\
i & \cmd{bibnamedelimi} &
%Replaces \cmd{bibnamedelima/b} after initials
在首字符缩写之后代替 \cmd{bibnamedelima/b} 的命令。\\
p & \cmd{revsdnamepunct} &
%Inserted by a formatting directive after the family name when the name parts are reversed.
当名部分顺序反过来时,由格式指令插入在 last name 之后。
\end{namedelims}
%
%This is how the delimiters are employed:
以下展示了如何使用这些分隔符:

\begin{namesample}
\item John\delim{~}{a}Edward\delim{ }{d}Doe
\item Doe\delim{,}{p}\delim{ }{d}John\delim{~}{a}Edward
\end{namesample}
%
%Initials in the \file{bib} file get a special delimiter:
\file{bib} 文件中的首字符缩写会有一个特别的分隔符:

\begin{namesample}
\item J.\delim{~}{i}Edward\delim{ }{d}Doe
\end{namesample}
%
%Let's consider a more complex name: \enquote{Charles-Jean Étienne Gustave Nicolas de La Vallée Poussin}. This name is composed of the following parts:
考虑一个更复杂的名字:\enquote{Charles-Jean Étienne Gustave Nicolas de La Vallée Poussin}。
它包含了如下几部分:

\begin{nameparts}
Given	& Charles-Jean Étienne Gustave Nicolas \\
Prefix	& de \\
Family	& La Vallée Poussin \\
Suffix	& --- \\
\end{nameparts}
%
%The delimiters:
分隔符为:

\begin{namesample}
\item Charles-Jean\delim{~}{b}Étienne\delim{~}{b}Gustave\delim{~}{a}Nicolas\delim{ }{d}%
      de\delim{ }{c}%
      La\delim{~}{a}Vallée\delim{~}{a}Poussin
\end{namesample}
%
%Note that \cmd{bibnamedelima/b/i} are inserted by the backend. The backend processes the name parts and takes care of the delimiters between the elements that make up a name part, processing each part individually. In contrast to that, the delimiters between the parts of the complete name (\cmd{bibnamedelimc/d}) are added by name formatting directives at a later point in the processing chain. The spacing and punctuation of initials is also handled by the backend and may be customized by redefining the following three macros:
请注意 \cmd{bibnamedelima/b/i} 由后端程序插入。
后端程序处理名部分并考虑组成名部分的元素之间的分隔符,从而分别处理每一部分。
与此相反,全名的名部分之间的分隔符(\cmd{bibnamedelimc/d})由名称格式指令在处理过程的之后时间点添加。
首字符缩写的空格和标点同样由后端程序处理,并且可以通过重定义以下三个宏来定制:

\begin{namedelims}
a & \cmd{bibinitperiod} &
%Inserted by the backend after initials.
由后端程序插入在首字母缩写之后。\\
b & \cmd{bibinitdelim} &
%Inserted by the backend between multiple initials.
由后端程序插入在多个首字母缩写之间。\\
c & \cmd{bibinithyphendelim} &
%Inserted by the backend between the initials of hyphenated name parts, replacing \cmd{bibinitperiod} and \cmd{bibinitdelim}.
由后端程序插入在带有连字符的名部分中首字母缩写之间,用以代替 \cmd{bibinitperiod} 和 \cmd{bibinitdelim}。\\
\end{namedelims}
%
%This is how they are employed:
以下是使用方式:

\begin{namesample}
\item J\delim{.}{a}\delim{~}{b}E\delim{.}{a} Doe
\item K\delim{.-}{c}H\delim{.}{a} Mustermann
\end{namesample}

\subsubsection{文献筛选和标注标签}%\subsubsection{Bibliography Filters and Citation Labels}
\label{use:cav:lab}

%The citation labels generated by this package are assigned to the full list of references before it is split up by any bibliography filters. They are guaranteed to be unique across the entire document (or a \env{refsection} environment), no matter how many bibliography filters you are using. When using a numeric citation scheme, however, this will most likely lead to discontinuous numbering in split bibliographies. Use the \opt{defernumbers} package option to avoid this problem. If this option is enabled, numeric labels are assigned the first time an entry is printed in any bibliography.

本宏包生成的标注(引用)标签在被文献筛选(过滤)器分开之前就被分配给整个文献列表。
因此能够确保在整个文档(或者一个 \env{refsection} 环境中)是唯一的,无论使用多少文献过滤器。
然而,当使用数值型引用格式时,这很可能会导致在各个划分出来的参考文献表中编号不连续。
使用 \opt{defernumber} 宏包选项可以避免这一问题。
如果启用该选项,数值标签会在任一文献条目中第一次打印时才被分配。

\subsubsection{参考文献标题中的活动字符}%\subsubsection{Active Characters in Bibliography Headings}
\label{use:cav:act}

%Packages using active characters, such as \sty{babel}, \sty{polyglossia}, \sty{csquotes}, or \sty{underscore}, usually do not make them active until the beginning of the document body to avoid interference with other packages. A typical example of such an active character is the Ascii quote |"|, which is used by various language modules of the \sty{babel}/\sty{polyglossia} packages. If shorthands such as |"<| and |"a| are used in the argument to \cmd{defbibheading} and the headings are defined in the document preamble, the non-active form of the characters is saved in the heading definition. When the heading is typeset, they do not function as a command but are simply printed literally. The most straightforward solution consists in moving \cmd{defbibheading} after |\begin{document}|. Alternatively, you may use \sty{babel}'s \cmd{shorthandon} and \cmd{shorthandoff} commands to temporarily make the shorthands active in the preamble. The above also applies to bibliography notes and the \cmd{defbibnote} command.

\sty{babel}、\sty{polyglossia}、\sty{csquotes} 和 \sty{underscore} 等使用活动字符的宏包通常直到正文开始才将这些字符激活,这样可以避免与其它宏包的冲突。
一个典型的活动字符例子是 Ascii 引号 |"|,
它用于 \sty{babel}/\sty{polyglossia} 宏包的不同语言模块。
如果在 \cmd{defbibheading} 的选项中使用 |"<| 和 |"a| 等速记方式,并且标题定义在导言区中,
那么标题定义中保存的是字符的非激活形式。
因此当标题打印出来时,它们不具有命令的功能而仅会原样打印。
最直接的解决方法是将 \cmd{defbibheading} 放在  |\begin{document}| 之后。
此外,你也可以使用 \sty{babel} 的 \cmd{shorthandon} 和 \cmd{shorthandoff} 命令来临时在导言区中激活这些简记方式。
这同样应用于文献注记和 \cmd{defbibnote} 命令。

\subsubsection{参考文献分节和分段中的编组}%\subsubsection{Grouping in Reference Sections and Segments}
\label{use:cav:grp}

%All \latex environments enclosed in \cmd{begin} and \cmd{end} form a group. This may have undesirable side effects if the environment contains anything that does not expect to be used within a group. This issue is not specific to \env{refsection} and \env{refsegment} environments, but it obviously applies to them as well. Since these environments will usually enclose much larger portions of the document than a typical \env{itemize} or similar environment, they are simply more likely to trigger problems related to grouping. If you observe any malfunctions after adding \env{refsection} environments to a document (for example, if anything seems to be <trapped> inside the environment), try the following syntax instead:

所有在 \cmd{begin} 和 \cmd{end} 中的 \LaTeX 环境形成了一个分组。
如果该环境包含一些不希望在组内使用的东西时,那么可能会引起一些不良反应。
这个问题并不仅仅针对 \env{refsection} 和 \env{refsegment} 环境,但显然也包括它们。
由于这些环境通常比典型的 \env{itemize} 或其它环境包含更多文本,因此它们自然更有可能引起涉及到分组的问题。
如果你在添加 \env{refsection} 环境之后观察到任何不正常的现象(例如,如果环境内有任何“受限”情况),
请尝试使用以下语句来代替:

\begin{ltxexample}
\chapter{...}
<<\refsection>>
...
<<\endrefsection>>
\end{ltxexample}
%
%This will not from a group, but otherwise works as usual. As far as \biblatex is concerned, it does not matter which syntax you use. The alternative syntax is also supported by the \env{refsegment} environment. Note that the commands \cmd{newrefsection} and \cmd{newrefsegment} do not form a group. See \secref{use:bib:sec, use:bib:seg} for details.
这不会形成一个分组,但是像正常一样工作。
就 \biblatex 而言,它并不影响你使用哪种语句。
\env{refsegment} 环境也支持这种语句。
请注意,命令 \cmd{newrefsection} 和 \cmd{newrefsegment} 不会形成分组。
详见 \secref{use:bib:sec, use:bib:seg} 节。

\subsection{使用备选的 \BibTeX 后端}%\subsection{Using the fallback \bibtex\ backend}
\label{use:bibtex}

%To utilise all of the features described here, \biblatex must be used with the
%\biber program as a backend. Indeed, the documentation in general assumes this. However, for a \emph{limited} subset of use cases it is possible to use the
%long-established \bibtex program (either the 7-bit \texttt{bibtex} or
%8-bit \texttt{bibtex8}) as the supporting backend. This works in much the
%same way as for \biber with the only proviso being that \bibtex is much more
%limited as a backend.

\biblatex 必须使用 \biber 程序作为后端才能使用这里描述的所有特性。实际上,本文档正是默认这一假定。不过,如果只是使用\emph{受限制}的一部分功能,也可以使用历史悠久的 \BibTeX 程序(7-bit 的  \bin{bibtex} 或者 8-bit 的 \bin{bibtex8})作为后端程序。


%Using \bibtex as the backend requires that the option \opt{backend=bibtex}
%or \opt{backend=bibtex8} is given at load time. The \biblatex package will
%then write appropriate data for use by \bibtex into the auxiliary file(s)
%and a special data file (automatically included in those to be read by
%\bibtex).  The \bibtex(8) program should then be run on each auxiliary file:
%\biblatex will list all of the required files in the log.

使用 \BibTeX 作为后端程序需要在载入时开启选项 \opt{backend=bibtex} 或 \opt{backend=bibtex8}.
\biblatex 宏包随后会将合适的数据写入辅助文件以及特殊的数据文件中以供 \BibTeX 使用
(自动包含那些被 \BibTeX 读取的文件)。
然后 \bin{bibtex(8)} 程序应当运行在每一辅助文件上:
\biblatex 会在日志文件中列出所有所需的文件。

%Key limitations of the \bibtex backend are:
\BibTeX 后端的主要局限有:

\begin{itemize}

\item %Sorting is global and is limited to Ascii ordering
排序是全局的,而且只限于按照 Ascii 顺序。


\item %No re-encoding is possible and thus database entries must be in
%LICR form to work reliably
不可以重编码,因此数据库中的条目必须按照 \LaTeX 内部字符表示\footnote{\LaTeX{} Internal Character Representation, LICR——译注}的形式,才可以确保程序可靠。

\item %The data model is fixed
数据模型是固定的。

\item %Cross-referencing is more limited and entry sets must be written into the \path{.bib} file
交叉引用有限制,条目集必须写入 \path{.bib} 文件。
\item %Fixed memory capacity (using the \verb|--wolfgang| option with
%\verb|bibtex8| is strongly recommended to minimize the likelihood of
%an issue here)
内存容量有限。在 \verb|bibtex8| 中,强烈建议使用 \verb|--wolfgang| 选项以尽量减少这一问题的可能性。

\end{itemize}

\endinput
