\section{样式作者指南}
\label{aut}

%This part of the manual documents the author interface of the \biblatex package. The author guide covers everything you need to know in order to write new citation and bibliography styles or localisation modules. You should read the user guide first before continuing with this part of the manual.

本节内容是样式作者指南,主要介绍\biblatex 包的接口。该指南囊括了设计参考文献著录和标注样式或者本地化模型所需知晓的所有内容。在阅读本部分内容前最好先阅读上一节的用户手册。

\subsection{概述}%Overview
\label{aut:int}
在讨论\biblatex 提供的命令和工具之前,我们首先介绍一些基本概念。\biblatex 包以一种特殊的方式使用辅助文件。最值得注意的是当使用\bibtex 后端程序时,\file{bbl}文件的使用方式存在差别,即只有一个\file{bst}文件可用来实现结构化的数据接口,该文件并非用来输出可打印数据。

使用\latex 的标准参考文献工具,一个文档通常包含任意数量的文献引用命令,以及常放在文档最后的\cmd{bibliographystyle}和\cmd{bibliography}命令。文献引用命令在文档中的位置是任意的,而\cmd{bibliographystyle}和\cmd{bibliography}命令则标记了打印参考文献表的位置,比如:
%Before we get to the commands and facilities provided by \biblatex, we will have a look at some of its fundamental concepts. The \biblatex package uses auxiliary files in a special way. Most notably, the \file{bbl} file is used differently and when using \bibtex as the backend, there is only one \file{bst} file which implements a structured data interface rather than exporting printable data. With \latex's standard bibliographic facilities, a document includes any number of citation commands in the document body plus \cmd{bibliographystyle} and \cmd{bibliography}, usually towards the end of the document. The location of the former is arbitrary, the latter marks the spot where the list of references is to be printed:

\begin{ltxexample}
\documentclass{...}
\begin{document}
\cite{...}
...
\bibliographystyle{...}
\bibliography{...}
\end{document}
\end{ltxexample}
%
%Processing this files requires that a certain procedure be followed. This procedure is as follows:

处理这些文件遵循一定的流程,其过程如下:

\begin{enumerate}

\item 运行 \bin{latex}: 第一次运行\bin{latex}, 在file{aux}文件中写入\cmd{bibstyle}和 \cmd{bibdata}命令,以及所有标注的\cmd{citation}命令。这时,各引文标注\footnote{这里的references译为引文标注,指在引用命令导致在正文中出现的标注,这个标注由标签label构成。}是未定义的,因为 \latex 等待\bibtex 提供需要的数据,当然参考文献表也没生成。
    %Run \bin{latex}: On the first run, \cmd{bibstyle} and \cmd{bibdata} commands are written to the \file{aux} file, along with \cmd{citation} commands for all citations. At this point, the references are undefined because \latex is waiting for \bibtex to supply the required data. There is also no bibliography yet.


\item 运行 \bin{bibtex}:\bibtex 在\file{bbl}文件中写入一个\env{thebibliography}环境,用以提供\file{aux}文件中\cmd{citation}命令所需求的所有条目,这些条目的数据来自\file{bib}文件。
    %Run \bin{bibtex}: \bibtex writes a \env{thebibliography} environment to the \file{bbl} file, supplying all entries from the \file{bib} file which were requested by the \cmd{citation} commands in the \file{aux} file.


\item 运行\bin{latex},第二次运行\bin{latex},\env{thebibliography}环境中的\cmd{bibitem}命令在\file{aux}文件中为各参考文献条目写入\cmd{bibcite}命令。这些\cmd{bibcite}命令定义的标签将用于\cmd{cite}命令。然而,各引文标注仍然未定义,因为这些标签在最后一次运行\bin{latex}前仍未知。
    %Run \bin{latex}: Starting with the second run, the \cmd{bibitem} commands in the \env{thebibliography} environment write one \cmd{bibcite} command for each bibliography entry to the \file{aux} file. These \cmd{bibcite} commands define the citation labels used by \cmd{cite}. However, the references are still undefined because the labels are not available until the end of this run.


\item 运行\bin{latex}:第三次运行,随着导言区最后读入了\file{aux}文件,引文标注的标签定义完成。这样所有的标注可以正确打印。
    %Run \bin{latex}: Starting with the third run, the citation labels are defined as the \file{aux} file is read in at the end of the preamble. All citations can now be printed.

\end{enumerate}

注意到所有的参考文献数据都以最终格式(指最后打印出的格式)写入\file{bbl}文件。该文件的读取和处理如同任何文档中的可打印章节。例如,考虑在一个\file{bib}文件中有如下条目:
%Note that all bibliographic data is written to the \file{bbl} file in the final format. The \file{bbl} file is read in and processed like any printable section of the document. For example, consider the following entry in a \file{bib} file:

\begin{lstlisting}[style=bibtex]{}
@Book{companion,
  author    = {Michel Goossens and Frank Mittelbach and Alexander Samarin},
  title     = {The LaTeX Companion},
  publisher = {Addison-Wesley},
  address   = {Reading, Mass.},
  year      = {1994},
}
\end{lstlisting}
%
根据\path{plain.bst}样式,\bibtex 在\file{bbl}文件中输出该条目如下:
%With the \path{plain.bst} style, \bibtex exports this entry to the \file{bbl} file as follows:

\begin{ltxexample}
\bibitem{companion}
Michel Goossens, Frank Mittelbach, and Alexander Samarin.
\newblock {\em The LaTeX Companion}.
\newblock Addison-Wesley, Reading, Mass., 1994.
\end{ltxexample}
%
默认情况下,\latex 生成顺序编码制标注标签,因此\cmd{bibitem}命令在\file{aux}文件中写入的行如下所示:
%By default, \latex generates numeric citation labels, hence \cmd{bibitem} writes lines such as the following to the \file{aux} file:

\begin{ltxexample}
\bibcite{companion}{1}
\end{ltxexample}
%
要实现一个不同的标注标签样式,意味着需要通过\file{aux}文件传递更多的数据。比如,当使用\sty{natbib}包时,\file{aux}文件包含的标注(或引用)信息行,如下:
%Implementing a different citation style implies that more data has to be transferred via the \file{aux} file. With the \sty{natbib} package, for example, the \file{aux} file contains lines like this one:

\begin{ltxexample}
\bibcite{companion}{{1}{1994}{{Goossens et~al.}}{{Goossens, Mittelbach,
and Samarin}}}
\end{ltxexample}
%

\biblatex 包支持任何格式的标注标签,因此标注命令需要访问所有的参考文献数据。看一看同样需要在标注中提供所有参考文献数据的\sty{jurabib}包的输出,我们将更清楚地理解这对上述处理过程的意义。
%The \biblatex package supports citations in any arbitrary format, hence citation commands need access to all bibliographic data. What this would mean within the scope of the procedure outlined above becomes obvious when looking at the output of the \sty{jurabib} package which also makes all bibliographic data available in citations:

\begin{ltxexample}
\bibcite{companion}{{Goossens\jbbfsasep Mittelbach\jbbstasep Samarin}%
  {}{{0}{}{book}{1994}{}{}{}{}{Reading, Mass.\bpubaddr{}Addison-Wesley%
  \bibbdsep{} 1994}}{{The LaTeX Companion}{}{}{2}{}{}{}{}{}}{\bibnf
  {Goossens}{Michel}{M.}{}{}\Bibbfsasep\bibnf{Mittelbach}{Frank}{F.}%
  {}{}\Bibbstasep\bibnf{Samarin}{Alexander}{A.}{}{}}{\bibtfont{The
  LaTeX Companion}.\ \apyformat{Reading, Mass.\bpubaddr{}
  Addison-Wesley\bibbdsep{} 1994}}}
\end{ltxexample}
%

在这种情况下,整个\env{thebibliography}环境的内容能通过\file{aux}文件有效传递。数据首先从\file{bbl}文件中读取出来,写入到\file{aux}中,然后再从\file{aux}读出保存到内存中。只有读入\file{bbl}文件,参考文献表才能生成。而\biblatex 包将被迫通过\file{aux}文件回收所有的数据。这意味着处理过度且多余,因为不管怎么样数据都必须保存在内存中。
%In this case, the contents of the entire \env{thebibliography} environment are effectively transferred via the \file{aux} file. The data is read from the \file{bbl} file, written to the \file{aux} file, read back from the \file{aux} file and then kept in memory. The bibliography itself is still generated as the \file{bbl} file is read in. The \biblatex package would also be forced to cycle all data through the \file{aux} file. This implies processing overhead and is also redundant because the data has to be kept in memory anyway.

这种传统的处理过程都基于一个假设,即条目的完整数据只是参考文献表需要而所有的标注都使用短标签。这对于有内存限制的情况是非常有效的,但也意味着很难扩展。这就是\biblatex 采取另一种方式的原因。首先,文档结构略有变化。取消在文档内使用\cmd{bibliography}命令,数据库文件由导言区的\cmd{addbibresource}命令指定,完全忽略\cmd{bibliographystyle}命令(所有的功能都将由包选项控制),参考文献表使用\cmd{printbibliography}命令打印:
%The traditional procedure is based on the assumption that the full bibliographic data of an entry is only required in the bibliography and that all citations use short labels. This makes it very effective in terms of memory requirements, but it also implies that it does not scale well. That is why \biblatex takes a different approach. First of all, the document structure is slightly different. Instead of using \cmd{bibliography} in the document body, database files are specified in the preamble with \cmd{addbibresource}, \cmd{bibliographystyle} is omitted entirely (all features are controlled by package options), and the bibliography is printed using \cmd{printbibliography}:

\begin{ltxexample}
\documentclass{...}
\usepackage[...]{biblatex}
\addbibresource{...}
\begin{document}
\cite{...}
...
\printbibliography
\end{document}
\end{ltxexample}
%

为了简化整个流程,\biblatex 基本上以应用\file{aux}文件的方式应用\file{bbl}文件,并舍弃了\cmd{bibcite}命令。于是,我们得到如下流程:
%In order to streamline the whole procedure, \biblatex essentially employs the \file{bbl} file like an \file{aux} file, rendering \cmd{bibcite} obsolete. We then get the following procedure:

\begin{enumerate}

\item 运行\bin{latex}:第一步类似于上述的传统方式:\cmd{bibstyle} 和 \cmd{bibdata}以及所有引用的\cmd{citation}命令写入到\file{aux}文件中(以\bibtex 为后端程序)或者写到\file{bcf}文件中(以\biber 为后端程序)。然后等待后端程序提供需要的数据。当以\bibtex 为后端程序时,\biblatex 使用一个特殊 \file{bst}的文件,该文件用于实现\bibtex 后端程序的数据接口,因此\cmd{bibstyle} 命令则必须是|\bibstyle{biblatex}|。
    %Run \bin{latex}: The first step is similar to the traditional procedure described above: \cmd{bibstyle} and \cmd{bibdata} commands are written to the \file{aux} file (\bibtex backend) or \file{bcf} file (\biber backend), along with \cmd{citation} commands for all citations. We then wait for the backend to supply the required data. With \bibtex as a backend, since \biblatex uses a special \file{bst} file which implements its data interface on the \bibtex end, the \cmd{bibstyle} command is always |\bibstyle{biblatex}|.


\item 运行\bin{biber} 或 \bin{bibtex}:后端程序提供了辅助文件中所有\cmd{citation}命令所需的条目,这些条目的数据来自\file{bib}文件。然而,它并不在\file{bbl}文件中写出一个可打印的参考文献表,而是一个参考文献的结构化表达数据。类似于\file{aux}文件,读入该\file{bbl}文件时不打印任何东西,仅是将数据存入内存中。
    %Run \bin{biber} or \bin{bibtex}: The backend supplies those entries from the \file{bib} file which were requested by the \cmd{citation} commands in the auxiliary file. However, it does not write a printable bibliography to the \file{bbl} file, but rather a structured representation of the bibliographic data. Just like an \file{aux} file, this \file{bbl} file does not print anything when read in. It merely puts data in memory.


\item 运行\bin{latex}: 第二次运行,\file{bbl}文件在文档正文开始的时候处理,类似于\file{aux}文件。从这开始,所有参考文献数据都已在内存中,所以所有的引用都可以正确打印。\footnote{如果\opt{defernumbers} 包选项打开, \biblatex 以类似于传统过程的一种算法来生成顺序制标签。这种情况下,这些数字在参考文献表打印的时候指定且需从后端程序辅助文件中回收。因此需要额外运行一次\latex 以在标注中获得它们。 } 引用命令不仅可以访问预定义的标签,还可以访问完整的参考文献数据。参考文献表由内存中的相同数据生成,可以根据需要进行筛选和划分。
    %Run \bin{latex}: Starting with the second run, the \file{bbl} file is processed right at the beginning of the document body, just like an \file{aux} file. From this point on, all bibliographic data is available in memory so that all citations can be printed right away.\footnote{If the \opt{defernumbers} package option is enabled \biblatex uses an algorithm similar to the traditional procedure to generate numeric labels. In this case, the numbers are assigned as the bibliography is printed and then cycled through the backend auxiliary file. It will take an additional \latex run for them to be picked up in citations.} The citation commands have access to the complete bibliographic data, not only to a predefined label. The bibliography is generated from memory using the same data and may be filtered or split as required.
\end{enumerate}

我们再次考虑上面给出的条目样例:
%Let's consider the sample entry given above once more:

\begin{lstlisting}[style=bibtex]{}
@Book{companion,
  author    = {Michel Goossens and Frank Mittelbach and Alexander Samarin},
  title     = {The LaTeX Companion},
  publisher = {Addison-Wesley},
  address   = {Reading, Mass.},
  year      = {1994},
}
\end{lstlisting}
%
使用\biblatex 及\biber 后端程序,这一条目实际上以如下格式输出:
%With \biblatex and the \biber backend, this entry is essentially exported in the following format:

\begin{ltxexample}
\entry{companion}{book}{}
  \labelname{author}{3}{}{%
    {{uniquename=0,hash=...}{Goossens}{G.}{Michel}{M.}{}{}{}{}}%
    {{uniquename=0,hash=...}{Mittelbach}{M.}{Frank}{F.}{}{}{}{}}%
    {{uniquename=0,hash=...}{Samarin}{S.}{Alexander}{A.}{}{}{}{}}%
  }
  \name{author}{3}{}{%
    {{uniquename=0,hash=...}{Goossens}{G.}{Michel}{M.}{}{}{}{}}%
    {{uniquename=0,hash=...}{Mittelbach}{M.}{Frank}{F.}{}{}{}{}}%
    {{uniquename=0,hash=...}{Samarin}{S.}{Alexander}{A.}{}{}{}{}}%
  }
  \list{publisher}{1}{%
    {Addison-Wesley}%
  }
  \list{location}{1}{%
    {Reading, Mass.}%
  }
  \field{title}{The LaTeX Companion}
  \field{year}{1994}
\endentry
\end{ltxexample}
%

由这一例子可见,某种程度上说结构化的数据构成了\file{bbl}文件内容\footnote{这里应该是bbl文件而不是原文的bib文件}。从这点上说,没有任何关于参考文献条目最终格式的决定。而参考文献表和引用标注的格式化由 \latex 宏控制,这些宏定义在参考文献和引用样式文件中。
%As seen in this example, the data is presented in a structured format that resembles the structure of a \file{bib} file to some extent. At this point, no decision concerning the final format of the bibliography entry has been made. The formatting of the bibliography and all citations is controlled by \latex macros, which are defined in bibliography and citation style files.

\subsection{参考文献著录样式}%Bibliography Styles
\label{aut:bbx}

一个参考文献著录样式是用于控制打印参考文献表中条目的宏的集合,定义在扩展名为\file{bbx}的文件中。\biblatex 包在其结尾加载所选择的参考文献样式文件。需要注意:一些由多个标准样式文件共享的常用宏定义在\path{biblatex.def}文件中。该文件同样在包结尾加载,但先于参考文献样式文件。
%A bibliography style is a set of macros which print the entries in the bibliography. Such styles are defined in files with the suffix \file{bbx}. The \biblatex package loads the selected bibliography style file at the end of the package. Note that a small repertory of frequently used macros shared by several of the standard bibliography styles is included in \path{biblatex.def}. This file is loaded at the end of the package as well, prior to the selected bibliography style.

\subsubsection{参考文献著录样式文件}% Bibliography Style Files
\label{aut:bbx:bbx}

在我们讨论参考文献著录样式的各部分之前,考虑一个典型的\file{bbx}文件总体结构,如下:
%Before we go over the individual components of a bibliography style, consider this example of the overall structure of a typical \file{bbx} file:

\begin{ltxexample}
\ProvidesFile{example.bbx}[2006/03/15 v1.0 biblatex bibliography style]

\defbibenvironment{bibliography}
  {...}
  {...}
  {...}
\defbibenvironment{shorthand}
  {...}
  {...}
  {...}
\InitializeBibliographyStyle{...}
\DeclareBibliographyDriver{article}{...}
\DeclareBibliographyDriver{book}{...}
\DeclareBibliographyDriver{inbook}{...}
...
\DeclareBibliographyDriver{shorthand}{...}
\endinput
\end{ltxexample}
%
参考文献著录样式文件的主要结构包含如下命令:
%The main structure of a bibliography style file consists of the following commands:

\begin{ltxsyntax}

\cmditem{RequireBibliographyStyle}{style}

该命令是可选的,用于引入一些建立在更一般的参考文献样式上的特殊样式。该命令加载样式文件\path{style.bbx}。
%This command is optional and intended for specialized bibliography styles built on top of a more generic style. It loads the bibliography style \path{style.bbx}.

\cmditem{InitializeBibliographyStyle}{code}

该命令在参考文献表开始之前插入任意给定的\prm{code},但在参考文献表所形成的组内。该命令是可选的。它对于不同的参考文献驱动共享一些定义是有用的,但不能用于参考文献组外。记住,文档中可以有多个参考文献表,如果参考文献驱动进行了任何全局设置,应在下一个参考文献开始前重设\footnote{这里不是很理解}。
%Specifies arbitrary \prm{code} to be inserted at the beginning of the bibliography, but inside the group formed by the bibliography. This command is optional. It may be useful for definitions which are shared by several bibliography drivers but not used outside the bibliography. Keep in mind that there may be several bibliographies in a document. If the bibliography drivers make any global assignments, they should be reset at the beginning of the next bibliography.

\cmditem{DeclareBibliographyDriver}{entrytype}{code}

定义一个参考文献驱动。一个驱动 <driver> 是一个宏用于控制某一具体的参考文献条目(当打印参考文献表的时候)或者某一具体命名了的参考文献表(当打印多个参考文献表的时候)。\prm{entrytype}与\file{bib}文件中使用的条目类型对应,以小写字母给出(见\secref{bib:typ})。\prm{entrytype}变量可以是一个星号。这种情况下,该驱动退化为没有具体驱动的条目类型。\prm{code}是任意代码用于打印各自\prm{entrytype}的参考文献条目。该命令是必须的。每个参考文献样式都应提供所用到的每类条目的驱动。
%Defines a bibliography driver. A <driver> is a macro which handles a specific entry type (when printing bibliography lists) or a specific named bibliography list (when printing bibliography lists). The \prm{entrytype} corresponds to the entry type used in \file{bib} files, specified in lowercase letters (see \secref{bib:typ}). The \prm{entrytype} argument may also be an asterisk. In this case, the driver serves as a fallback which is used if no specific driver for the entry type has been defined. The \prm{code} is arbitrary code which typesets all bibliography entries of the respective \prm{entrytype}. This command is mandatory. Every bibliography style should provide a driver for each entry type.

\cmditem{DeclareBibliographyAlias}{alias}{entrytype}

如果一个参考文献驱动用于处理多个参考文献条目类型,该命令可以用来定义某类已经定义驱动的\prm{entrytype}别名。\prm{alias}选项可以是一个星号,这种情况下,该驱动用于那些没有指定驱动的参考文献条目。
%If a bibliography driver covers more than one entry type, this command may be used to define an alias where \prm{entrytype} is the name of a defined driver. This command is optional. The \prm{alias} argument may also be an asterisk. In this case, the \prm{entrytype} driver serves as a fallback which is used if no specific driver for an entry has been defined.

\cmditem{DeclareBibliographyOption}[datatype]{key}[value]{code}

该命令以\keyval 格式定义额外的导言区选项。\prm{key}是选项键。\prm{code}是当使用该选项时执行的任意\tex 代码。键值作为|#1|传递给\prm{code}。可选的\prm{value}是当该选项仅有键名而无键值给出时的默认键值。这对于布尔选项非常有用。\prm{datatype}是选项的数据类型(datatype),如果缺省,那么默认为 <boolean>(布尔类型),比如一个定义如下:

%This command defines additional preamble options in \keyval format. The \prm{key} is the option key. The \prm{code} is arbitrary \tex code to be executed whenever the option is used. The value passed to the option is passed on to the \prm{code} as |#1|. The optional \prm{value} is a default value to be used if the bare key is given without any value. This is useful for boolean switches.
%The \prm{datatype} is a the datatype for the option. If omitted, it defaults to <boolean>. For example, with a definition like the following:

\begin{ltxexample}
\DeclareBibliographyOption[boolean]{somekey}[true]{...}
\end{ltxexample}
%
给出<\texttt{somekey}>而没有键值等价于<\kvopt{somekey}{true}>。有效的\prm{datatype}值定义默认的\biber 数据模型中,比如:
%giving <\texttt{somekey}> without a value is equivalent to <\kvopt{somekey}{true}>. Valid \prm{datatype} values are defined in the default \biber Datamodel as:

\begin{ltxexample}
\DeclareDatamodelConstant[type=list]{optiondatatypes}{boolean,integer,string,xml}
\end{ltxexample}

\cmditem{DeclareEntryOption}[datatype]{key}[value]{code}

类似于\cmd{DeclareBibliographyOption},但用于定义\secref{bib:fld:spc}节的\bibfield{options}域中的选项,且仅基于per"=entry(条目)进行设置。当\biblatex 为标注命令和参考文献驱动准备数据时,执行\prm{code}。
%Similar to \cmd{DeclareBibliographyOption} but defines options which are settable on a per"=entry basis in the \bibfield{options} field from \secref{bib:fld:spc}. The \prm{code} is executed whenever \biblatex prepares the data of the entry for use by a citation command or a bibliography driver.

\end{ltxsyntax}

\subsubsection{参考文献表环境}%Bibliography Environments
\label{aut:bbx:env}

除了定义参考文献驱动,参考文献著录样式也要定义参考文献表环境用于控制参考文献表的输出。这些环境由命令\cmd{defbibenvironment}名义。默认情况下,\cmd{printbibliography}使用\opt{bibliography}环境。下面是一个适用于不打印标签的参考文献表的环境定义:
%Apart from defining bibliography drivers, the bibliography style is also responsible for the environments which control the layout of the bibliography and bibliography lists. These environments are defined with \cmd{defbibenvironment}. By default, \cmd{printbibliography} uses the environment \opt{bibliography}. Here is a definition suitable for a bibliography style which does not print any labels in the bibliography:

\begin{ltxexample}
\defbibenvironment{bibliography}
  {\list
     {}
     {\setlength{\leftmargin}{\bibhang}%
      \setlength{\itemindent}{-\leftmargin}%
      \setlength{\itemsep}{\bibitemsep}%
      \setlength{\parsep}{\bibparsep}}}
  {\endlist}
  {\item}
\end{ltxexample}
%
该定义使用\biblatex 提供的\cmd{bibhang}尺寸,应用了一个带悬挂缩进的\env{list}环境。它允许使用\cmd{bibitemsep} 和 \cmd{bibparsep}来实现一定程度的布局调整,\biblatex 提供的这两个尺寸就是为了该目的(见 \secref{aut:fmt:len})。作者年制(\texttt{authoryear})和作者题名制(\texttt{authortitle})的参考文献样式使用类似于该例的定义。
%This definition employs a \env{list} environment with hanging indentation, using the \cmd{bibhang} length register provided by \biblatex. It allows for a certain degree of configurability by using \cmd{bibitemsep} and \cmd{bibparsep}, two length registers provided by \biblatex for this very purpose (see \secref{aut:fmt:len}). The \texttt{authoryear} and \texttt{authortitle} bibliography styles use a definition similar to this example.

\begin{ltxexample}
\defbibenvironment{bibliography}
  {\list
     {\printfield[labelnumberwidth]{labelnumber}}
     {\setlength{\labelwidth}{\labelnumberwidth}%
      \setlength{\leftmargin}{\labelwidth}%
      \setlength{\labelsep}{\biblabelsep}%
      \addtolength{\leftmargin}{\labelsep}%
      \setlength{\itemsep}{\bibitemsep}%
      \setlength{\parsep}{\bibparsep}}%
      \renewcommand*{\makelabel}[1]{\hss##1}}
  {\endlist}
  {\item}
\end{ltxexample}
%
一些参考文献样式在参考文献列表中打印标签。比如,设计一个顺序引用格式的参考文献样式需要在参考文献表的每个条目前面打印顺序数字,这样参考文献看起来就像一个顺序列表。在第一个例子中,\cmd{list}命令的第一个参数是空的。在这个例子中,我们需要在其中插入数字,这些数字由\biblatex 的\bibfield{labelnumber}域中的数字提供。我们也应用\biblatex 提供的几个尺寸和工具,详见 \secref{aut:fmt:ich, aut:fmt:ilc}。顺序制(\texttt{numeric})参考文献样式使用如上的定义。除\textsf{\texttt{labelnumber}}由\texttt{labelalpha}代替和\texttt{labelnumberwidth}由\texttt{labelalphawidth}代替外,顺序字母制(\texttt{alphabetic})的样式也是类似的。
%Some bibliography styles print labels in the bibliography. For example, a bibliography style designed for a numeric citation scheme will print the number of every entry such that the bibliography looks like a numbered list. In the first example, the first argument to \cmd{list} was empty. In this example, we need it to insert the number, which is provided by \biblatex in the \bibfield{labelnumber} field. We also employ several length registers and other facilities provided by \biblatex, see \secref{aut:fmt:ich, aut:fmt:ilc} for details. The \texttt{numeric} bibliography style uses the definition given above. The \texttt{alphabetic} style is similar, except that \textsf{\texttt{labelnumber}} is replaced by \texttt{labelalpha} and \texttt{labelnumberwidth} by \texttt{labelalphawidth}.

各参考文献表以类似方式处理。\cmd{printbiblist}命令默认使用以bibliography list命名的环境(当使用\bibtex 时,\cmd{printshorthands}总是使用\texttt{shorthand}环境)。一个典型的例子如下,其中的尺寸和工具定义详见第\secref{aut:fmt:ich, aut:fmt:ilc}节。
%Bibliography lists are handled in a similar way. \cmd{printbiblist} uses the environment named after the bibliography list by default. A typical example is given below. See \secref{aut:fmt:ich, aut:fmt:ilc} for details on the length registers and facilities used in this example.

\begin{ltxexample}
\defbibenvironment{shorthand}
  {\list
     {\printfield[shorthandwidth]{shorthand}}
     {\setlength{\labelwidth}{\shorthandwidth}%
      \setlength{\leftmargin}{\labelwidth}%
      \setlength{\labelsep}{\biblabelsep}%
      \addtolength{\leftmargin}{\labelsep}%
      \setlength{\itemsep}{\bibitemsep}%
      \setlength{\parsep}{\bibparsep}%
      \renewcommand*{\makelabel}[1]{##1\hss}}}
  {\endlist}
  {\item}
\end{ltxexample}

\subsubsection{参考文献驱动} %Bibliography Drivers
\label{aut:bbx:drv}

在我们讨论\biblatex 包的数据接口命令前,了解一下参考文献驱动的结构是有益的。注意,虽然下面给出的例子是大为简化的,但仍具有说明价值。为可读性考虑,我们忽略了一些可能是\bibtype{book}条目的域,并且简化处理没有忽略的域。主要是为了说明驱动的结构。关于\bibtex 文件的格式域与\biblatex 包的数据类型的映射信息,见\secref{bib:fld}。
%Before we go over the commands which form the data interface of the \biblatex package, it may be instructive to have a look at the structure of a bibliography driver. Note that the example given below is greatly simplified, but still functional. For the sake of readability, we omit several fields which may be part of a \bibtype{book} entry and also simplify the handling of those which are considered. The main point is to give you an idea of how a driver is structured. For information about the mapping of the \bibtex file format fields to \biblatex's data types, see \secref{bib:fld}.

\begin{ltxexample}
\DeclareBibliographyDriver{book}{%
  \printnames{author}%
  \newunit\newblock
  \printfield{title}%
  \newunit\newblock
  \printlist{publisher}%
  \newunit
  \printlist{location}%
  \newunit
  \printfield{year}%
  \finentry}
\end{ltxexample}
%
标准的参考文献样式应用两个参考文献宏\texttt{begentry}和\texttt{finentry}。
%The standard bibliography styles employ two bibliography macros \texttt{begentry} and \texttt{finentry}:

\begin{ltxexample}
\DeclareBibliographyDriver{<<entrytype>>}{%
  \usebibmacro{begentry}
  ...
  \usebibmacro{finentry}}
\end{ltxexample}
%
作为默认的定义。
%with the default definitions
\begin{ltxexample}
\newbibmacro*{begentry}{}
\newbibmacro*{finentry}{\finentry}
\end{ltxexample}
%
推荐使用这两个宏,因为方便在驱动开始或结束时使用钩子。
%Use of these macros is recommended for easy hooks into the beginning and end of the driver.

回到上述给出\texttt{book}条目类型的驱动,我们发现有一些缺省:即\cmd{printnames}, \cmd{printlist}, 和 \cmd{printfield}命令所使用的格式命令。为了说明一个格式话指令是什么,这里给出上述驱动举例中所使用虚构指令。域的格式是直接的,域的值直接作为参数传递给格式命令,并根据需要格式化。下面的指令简单地将输入参数用一个\cmd{emph}命令包裹:
%Returning to the driver for the \texttt{book} entrytype above, there is still one piece missing: the formatting directives used by \cmd{printnames}, \cmd{printlist}, and \cmd{printfield}. To give you an idea of what a formatting directive looks like, here are some fictional ones used by our sample driver. Field formats are straightforward, the value of the field is passed to the formatting directive as an argument which may be formatted as desired. The following directive will simply wrap its argument in an \cmd{emph} command:

\begin{ltxexample}
\DeclareFieldFormat{title}{\emph{#1}}
\end{ltxexample}
%
列表格式则要复杂一些。在将列表划分为独立的项后,\biblatex 将对列表中的每一项执行格式化命令。各项作为参数传递给格式化命令。列表中各项间的分隔符由相应的命令控制,因此我们必须在插入分隔符前要检查是否在列表中或者是列表末尾。
%List formats are slightly more complex. After splitting up the list into individual items, \biblatex will execute the formatting directive once for every item in the list. The item is passed to the directive as an argument. The separator to be inserted between the individual items in the list is also handled by the corresponding directive, hence we have to check whether we are in the middle of the list or at the end when inserting it.

\begin{ltxexample}
\DeclareListFormat{location}{%
  #1%
  \ifthenelse{\value{listcount}<\value{liststop}}
    {\addcomma\space}
    {}}
\end{ltxexample}
%
%姓名(name)的格式化命令类似于这种抄录列表,但列表中的单个项是姓名,因此需要自动的解析为姓名的不同组成部分。列表格式化命令对列表中每个姓名都执行一次,信吗的各个部分以分开的参数传递给该命令。比如,|#1|是姓(last name)和|#3|是名(first name)。下面给出一个简化的格式化命令例子:
%上述各格式化命令调换了第一个作者的姓名前后顺序«Last, First»),而其余姓名则是常规顺序(«First Last»)。注意:必须要保证提供的姓名部分是姓(last name),因此我们必须要检查实际数据中姓名的哪些部分是存在的。如果姓名的一些部分不存在,则相关的变量就为空。如同抄录列表的命令,在各独立项之间插入的分隔符也由格式化命令控制,因为我们也要检查是否处于列表中还是在其末尾,这也是第二个\cmd{ifthenelse}命令做的事情。

姓名(name)的格式化指令类似于抄录列表。
%Formatting directives for names are similar to those for literal lists.

依赖于数据模型常量<nameparts>的姓名有如下默认定义:
%Names depend on the datamodel constant <nameparts> which has the default definition:

\begin{ltxexample}
\DeclareDatamodelConstant[type=list]{nameparts}
                                    {prefix,family,suffix,given}
\end{ltxexample}
%
这可以通过定制或者添加更多的姓名成分来处理比如来自父系姓的问题(见文件\file{93-nameparts.tex})。自然的,数据源需要一个扩展的姓名格式。\biblatexml (\secref{apx:biblatexml})用来处理该问题,其中有一个扩展的姓名格式,可以处理自定义的姓名成分,当使用\biber 后端的时候(见\biber 文档)。
%This can be customised to add more name parts to deal with things like patronymics (see the example file \file{93-nameparts.tex}). Naturally this needs an extended name format for data sources. \biblatexml (\secref{apx:biblatexml}) handles this natively and there is an extended name format which can handle custom nameparts available when using \biber (see \biber documentation).

在姓名格式中,姓名成分常量声明将为每个姓名成分提供数据模型定义的宏:
%Inside name formats, the nameparts constant declaration makes available two macros for each name part defined in the datamodel:

\begin{ltxexample}
\namepart<namepart>
\namepart<namepart>i
\end{ltxexample}
%
姓名的格式化执行对姓名列表中的每一个姓名进行处理,看下面的例子:
%The name formatting directive is executed once for each name in the name list. Here is an example:

\begin{ltxexample}
\DeclareNameFormat{author}{%
  \ifthenelse{\value{listcount}=1}
    {\namepartfamily%
     \ifblank{\namepartgiven}{}{\addcomma\space\namepartgiven}}
    {\ifblank{\namepartgiven}{}{\namepartgiven\space}%
     \namepartfamily}%
  \ifthenelse{\value{listcount}<\value{liststop}}
    {\addcomma\space}
    {}}
\end{ltxexample}
%
上述各格式化命令调换了第一个作者的姓名前后顺序«Last, First»),而其余姓名则是常规顺序(«First Last»)。注意:必须要保证提供的姓名部分是姓(last name),因此我们必须要检查实际数据中姓名的哪些成分是存在的。如果姓名的一些成分不存在,则相关的变量就为空。如同抄录列表的命令,在各独立项之间插入的分隔符也由格式化命令控制,因为我们也要检查是否处于列表中还是在其末尾,这也是第二个\cmd{ifthenelse}命令做的事情。
%The above directive reverses the name of the first author («Last, First») and prints the remaining names in their regular sequence («First Last»). Note that the only component which is guaranteed to be available is the last name, hence we have to check which parts of the name are actually present. If a certain name part is not available, the corresponding macro will be empty. As with directives for literal lists, the separator to be inserted between the individual items in the name list is also handled by the formatting directive, hence we have to check whether we are in the middle of the list or at the end when inserting it. This is what the second \cmd{ifthenelse} test does.

\subsubsection{特殊域}%Special Fields
\label{aut:bbx:fld}
下面的列表和域用于\biblatex 给参考文献驱动和引用命令传递数据。它们由宏包自动定义,并不在\file{bib}文件中使用。从参考文献著录和标注样式的角度看,它们与\file{bib}文件中的域并没有什么不同。

%The following lists and fields are used by \biblatex to pass data to bibliography drivers and citation commands. They are not used in \file{bib} files but defined automatically by the package. From the perspective of a bibliography or citation style, they are not different from the fields in a \file{bib} file.

\paragraph{一般域} %Generic Fields
\label{aut:bbx:fld:gen}

\begin{fieldlist}

\fielditem{$<$datetype$>$dateunspecified}{string}

如果$<$datetype$>$date具有一个EDTF 5.2.2 <unspecified>,该域将被设置为\opt{yearindecade}, \opt{yearincentury}, \opt{monthinyear}, \opt{dayinmonth}或\opt{dayinyear}之一,这些字符串指定了unspecified 信息的粒度。这些字符串可用于日期范围的判断,该日期范围自动为这些<unspecified>日期创建,一个样式可能选择一种特殊方式来格式化日期。参见\secref{bib:use:dat}。例如:一个条目的日期为:
%If $<$datetype$>$date held an EDTF 5.2.2 <unspecified>, this field will be set to one of \opt{yearindecade}, \opt{yearincentury}, \opt{monthinyear}, \opt{dayinmonth} or \opt{dayinyear} which specifies the granularity of the unspecified information. These strings can be tested for and along with the date ranges which are automatically created for such <unspecified> dates, a style may choose to format the date in a special way. See \secref{bib:use:dat}. For example, an entry with dates such as:

\begin{lstlisting}[style=bibtex]{}
@book{key,
  date     = {19uu},
  origdate = {199u}
}
\end{lstlisting}
%
将在\file{.bbl}产生如下信息:
%would result in the same information in the \file{.bbl} as:
\begin{lstlisting}[style=bibtex]{}
@book{key,
  date     = {1900/1999},
  origdate = {1990/1999}
}
\end{lstlisting}
%
但也会额外的将域\bibfield{dateunspecified}设置为<yearincentury>,将\bibfield{origdateunspecified}设置为<yearindecade>。这一信息可以用来给\bibfield{date}提供可能的信息<20th century>,给\bibfield{origdate}提供<The 1990s>,这一信息无法单独从日期范围推算。因为这种自动生成的范围具有一个已知值,给出<unspecified>元信息,因此使用该范围值来进行特殊的格式化相对容易。而标准样式不做此处理,\file{96-dates.tex}给出了一些例子。
%but would additionally have the field \bibfield{dateunspecified} set to <yearincentury> and \bibfield{origdateunspecified} set to <yearindecade>. This information could be used to render the \bibfield{date} as perhaps <20th century> and \bibfield{origdate} as <The 1990s>, information which cannot be derived from the date ranges alone. Since such auto-generated ranges have a know values, given the <unspecified> meta-information, it is relatively easy to use the range values to format special cases. While the standard styles not do this, examples are given in the file \file{96-dates.tex}.

\fielditem{entrykey}{string}

\file{bib}文件中某一项的条目关键词(entry key)。这是一个字符串,用于\biblatex 及其后端程序确定\file{bib}文件中的某一条目。
%The entry key of an item in the \file{bib} file. This is the string used by \biblatex and the backend to identify an entry in the \file{bib} file.

\fielditem{childentrykey}{string}

当引用一个条目集的子条目时,\biblatex 给引用数据提供了父\bibtype{set}条目的数据。这意味着\bibfield{entrykey}表示的是父条目的关键词。而子条目的关键词在\bibfield{childentrykey}域中提供。该域仅在引用一个条目集的某一子条目时使用。
%When citing a subentry of an entry set, \biblatex provides the data of the parent \bibtype{set} entry to citation commands. This implies that the \bibfield{entrykey} field holds the entry key of the parent. The entry key of the child entry being cited is provided in the \bibfield{childentrykey} field. This field is only available when citing a subentry of an entry set.

\fielditem{labelnamesource}{literal}

保存给\bibfield{labelname}提供信息的域的域名,由\cmd{DeclareLabelname}确定。
%Holds the name of the field used to populate \bibfield{labelname},determined by \cmd{DeclareLabelname}.

\fielditem{labeltitlesource}{literal}

保存给\bibfield{labeltitle}提供信息的域的域名,由\cmd{DeclareLabeltitle}确定。
%Holds the name of the field used to populate \bibfield{labeltitle},determined by \cmd{DeclareLabeltitle}.

\fielditem{labeldatesource}{literal}

保存如下之一:
%Holds one of:

\begin{itemize}
\item 由\cmd{DeclareLabeldate}选择的日期域域名的<date>前的前缀。
%The prefix coming before <date> of the date field name chosen by   \cmd{DeclareLabeldate}
\item 一个域的域名。
%The name of a field
\item 一个抄录或本地化字符串。\footnote{literal 译为抄录}
%A literal or localisation string
\end{itemize}
%
一般情况下保存由\cmd{DeclareLabeldate}选择的日期域域名的<date>前的前缀。例如,如果labeldate域是\bibfield{eventdate},那么\bibfield{labeldatesource}就是<event>。如果\cmd{DeclareLabeldate}命令选择了\bibfield{date}域,\bibfield{labeldatesource}将会定义为一个空字符串作为<date>的前缀,因为date label名中<date>前为空。这就是说\bibfield{labeldatesource}的内容可以用于构建对\cmd{DeclareLabeldate}选择的域的指针。因为\cmd{DeclareLabeldate}也可以选择抄录字符串作为备选,\bibfield{labeldatesource}可以指向一个域或者不进行定义。记住:\cmd{DeclareLabeldate}命令可以用于选择非日期域作为备选,所以\bibfield{labeldatesource}可能包含一个域名。所以,总结起来,规则如下:
%Normally holds the prefix coming before <date> of the date field name chosen by \cmd{DeclareLabeldate}. For example, if the labeldate field is \bibfield{eventdate}, then \bibfield{labeldatesource} will be <event>. In case \cmd{DeclareLabeldate} selects the \bibfield{date} field, then \bibfield{labeldatesource} will be defined but will be an empty string as the prefix coming before <date> in the date label name is empty. This is so that the contents of \bibfield{labeldatesource} can be used in constructing references to the field which \cmd{DeclareLabeldate} chooses. Since \cmd{DeclareLabeldate} can also select literal strings for fallbacks, \bibfield{labeldatesource} may not refer to a field or may be undefined. Bear in mind that \cmd{DeclareLabeldate} can also be used to select non-date fields as a fallback and so \bibfield{labeldatesource} might contain a field name. So, in summary, the rules are

\begin{ltxexample}
\iffieldundef{labeldatesource}
  {}% labeldate package option is not set
  {\iffieldundef{\thefield{labeldatesource}year}
    % \DeclareLabeldate resolved to either a literal/localisation
    % string or a non-date field since
    % if a date is defined by a date field, there is
    % at least a year
    {\iffieldundef{\thefield{labeldatesource}}
       {}% \DeclareLabeldate resolved to a literal/localisation string
       {}% \DeclareLabeldate resolved to a non-date field
    }
    {} % \DeclareLabeldate resolved a date field name prefix like "" or "orig"
  }
\end{ltxexample}

\fielditem{entrytype}{string}

	条目类型(\bibtype{book}, \bibtype{inbook},等),以小写字母给出。
%The entry type (\bibtype{book}, \bibtype{inbook}, etc.), given in lowercase letters.

\fielditem{childentrytype}{string}

	当引用一个条目集的子条目时,\biblatex 给引用命令提供父集条目的数据。这意味着\bibfield{entrytype}保存父条目的类型。子条目的类型则由\bibfield{childentrytype}域提供。该域仅在引用一个条目集的子条目时使用。
%When citing a subentry of an entry set, \biblatex provides the data of the parent \bibtype{set} entry to citation commands. This implies that the \bibfield{entrytype} field holds the entry type of the parent. The entry type of the child entry being cited is provided in the \bibfield{childentrytype} field. This field is only available when citing a subentry of an entry set.

\fielditem{entrysetcount}{integer}

	该域保存的整数用于指明一个集中某个集成员的位置(起始值是\texttt{1})。该域仅对一个条目集的子条目有用。
%This field holds an integer indicating the position of a set member in the entry set (starting at \texttt{1}). This field is only available in the subentries of an entry set.

\fielditem{hash}{string}

	该域非常特殊,仅在姓名格式化命令中使用。它保存一个hash字符串,用于唯一的确定姓名列表中的单个姓名。该信息对于姓名列表中的所有姓名都有提供。另外参见\bibfield{namehash}和\bibfield{fullhash}。
%This field is special in that it is only available locally in name formatting directives. It holds a hash string which uniquely identifies individual names in a name list. This information is available for all names in all name lists. See also \bibfield{namehash} and \bibfield{fullhash}.

\fielditem{namehash}{string}

	一个hash字符串用于唯一确定\bibfield{labelname}列表。这对再现检查很有用。比如,一个将再次出现的作者和编者用一个类似<idem>的字符串代替的引用样式,可以用\cmd{savefield}命令保存\bibfield{namehash}域,并将其用于后面\cmd{iffieldequals}(见\secref{aut:aux:dat, aut:aux:tst})命令的比较中。\bibfield{namehash}域通过\bibfield{labelname}列表的截短得到,即它的结果与\opt{maxnames}和\opt{minnames}选项相关。另外参见\bibfield{hash}和\bibfield{fullhash}。
%A hash string which uniquely identifies the \bibfield{labelname} list. This is useful for recurrence checks. For example, a citation style which replaces recurrent authors or editors with a string like <idem> could save the \bibfield{namehash} field with \cmd{savefield} and use it in a comparison with \cmd{iffieldequals} later (see \secref{aut:aux:dat, aut:aux:tst}). The \bibfield{namehash} is derived from the truncated \bibfield{labelname} list, \ie it is responsive to \opt{maxnames} and \opt{minnames}. See also \bibfield{hash} and \bibfield{fullhash}.

\fielditem{$<$namelist$>$namehash}{string}

类似于\bibfield{namehash},但用于 <namelist>姓名列表。
%As \bibfield{namehash} for the name list called <namelist>.

\fielditem{fullhash}{string}

	一个hash字符串用于唯一确定\bibfield{labelname}列表。该域域\bibfield{namehash}有两点不同:1.产生hash时忽略\bibfield{shortauthor}和\bibfield{shorteditor}列表。2.该hash指的是完整的列表,忽略\opt{maxnames}和\opt{minnames}选项。另外见\bibfield{hash}和\bibfield{namehash}。
%A hash string which uniquely identifies the \bibfield{labelname} list. This fields differs from \bibfield{namehash} in two details: 1) The \bibfield{shortauthor} and \bibfield{shorteditor} lists are ignored when generating the hash. 2) The hash always refers to the full list, ignoring \opt{maxnames} and \opt{minnames}. See also \bibfield{hash} and \bibfield{namehash}.

\fielditem{$<$namelist$>$fullhash}{string}

类似于\bibfield{fullhash},,但用于 <namelist>姓名列表。
%As \bibfield{fullhash} for the name list called <namelist>.

\listitem{pageref}{literal}

	如果\opt{backref}包选项打开,该域保存各被引用条目所在页的页码。如果文档中有\env{refsection}环境,反向引用是针对当前参考文献节的。
%If the \opt{backref} package option is enabled, this list holds the page numbers of the pages on which the respective bibliography entry is cited. If there are \env{refsection} environments in the document, the back references are local to the reference sections.

\fielditem{sortinit}{literal}

	该域保存用于排序的信息首字符。使用\bibtex 时,该域也用来代替\bibfield{sortinithash}域。
%This field holds the initial character of the information used during sorting.

\fielditem{sortinithash}{string}

	使用\biber 时,该域保存排序字符串的第一个扩展字素集群(基本上是第一个字符)的Unicode排序规则算法主要权重的hash值。当按照字母表顺序划分参考文献列表时很有用,该域有内部\cmd{bibinitsep}所使用。(见\secref{use:fmt:len})
%This field holds a hash of the (locale-specific) Unicode Collation Algorithm primary weight of the first extended grapheme cluster (essentially the first character) of the string used during sorting. This is useful when subdividing the bibliography alphabetically and is used internally by \cmd{bibinitsep} (see \secref{use:fmt:len}).

\fielditem{clonesourcekey}{string}

	该域保存复制条目源条目的关键词。复制条目常用于处理相关条目和\bibfield{related}域。
%This field holds the entry key of the entry from which an entry was cloned. Clones are created for entries which are mentioned in \bibfield{related} fields as part of related entry processing, for example.

\end{fieldlist}

\paragraph{标注(引用)标签中使用的域}%Fields for Use in Citation Labels
\label{aut:bbx:fld:lab}

\begin{fieldlist}

\fielditem{labelalpha}{literal}

	当使用\bibtex 为后端程序时,生成一个类似于传统\bibtex 的\path{alpha.bst}样式的标签。这一默认标签由抽取\bibfield{labelname}列表的首字母加上出版年的最后两个数字构成。\bibfield{label}域可用来重写它的非数值部分(non"=numeric portion)。如果定义了\bibfield{label}域,\biblatex 将使用它的值加上出版年的后两个数字生成\bibfield{labelalpha}。\bibfield{shorthand}域也可用来重写整个标签。如果定义了该域,\bibfield{labelalpha}就是\bibfield{shorthand}域,而不是一个自动生成的标签。\BiberOnlyMark 使用\biber 时,用户可以定义用来构建字母顺序标签的模板(见\secref{aut:ctm:lab}),而默认的模板域上面bibtex后端程序使用的格式相同。一个完整的字母顺序(<alphabetic>)标签由以下域构成:
%A label similar to the labels generated by the \path{alpha.bst} style of traditional \bibtex. This default label consists of initials drawn from the \bibfield{labelname} list plus the last two digits of the publication year. The \bibfield{label} field may be used to override its non"=numeric portion. If the \bibfield{label} field is defined, \biblatex will use its value and append the last two digits of the publication year when generating \bibfield{labelalpha}. The \bibfield{shorthand} field may be used to override the entire label. If defined, \bibfield{labelalpha} is the \bibfield{shorthand} rather than an automatically generated label. Users can specify a template used to construct the alphabetic label (see \secref{aut:ctm:lab}) and the default template mirrors the format mentioned for bibtex above. A complete <alphabetic> label consists of the fields \bibfield{labelalpha} plus \bibfield{extraalpha}. Note that the \bibfield{labelalpha} and \bibfield{extraalpha} fields need to be requested with the package option \opt{labelalpha} (\secref{use:opt:pre:int}). See also \bibfield{extraalpha} as well as \cmd{labelalphaothers} in \secref{use:fmt:fmt}.

\fielditem{extraalpha}{integer}

	当参考文献中包含同一作者同一年出版的多个引文时,<alphabetic>引用格式常需要一个额外的字母加入标签来区分。这种情况下\bibfield{extraalpha}域保存一个整数可用命令 \cmd{mknumalph} 转换成字母或以其他方式格式化。该域类似于在作者年(author"=year)格式中\bibfield{extrayear}的作用。完整的 <alphabetic>的标签由\bibfield{labelalpha} 加 \bibfield{extraalpha}构成。注意包选项 \opt{labelalpha}要求使用\bibfield{labelalpha}和 \bibfield{extraalpha}域(详见\secref{use:opt:pre:int})。另外参见 \bibfield{labelalpha}和 \secref{use:fmt:fmt}的\cmd{labelalphaothers}。表\ref{use:opt:tab1}总结了不同的\opt{extra*}非歧义计数器和他们追踪的信息。
%The <alphabetic> citation scheme usually requires a letter to be appended to the label if the bibliography contains two or more works by the same author which were all published in the same year. In this case, the \bibfield{extraalpha} field holds an integer which may be converted to a letter with \cmd{mknumalph} or formatted in some other way. This field is similar to the role of \bibfield{extrayear} in the author"=year scheme. A complete <alphabetic> label consists of the fields \bibfield{labelalpha} plus \bibfield{extraalpha}. Note that the \bibfield{labelalpha} and \bibfield{extraalpha} fields need to be requested with the package option \opt{labelalpha}, see \secref{use:opt:pre:int} for details. See also \bibfield{labelalpha} as well as \cmd{labelalphaothers} in \secref{use:fmt:fmt}. Table \ref{use:opt:tab1} summarises the various \opt{extra*} disambiguation counters and what they track.

\listitem{labelname}{name}

	引用中打印的姓名。该列表可以是\bibfield{shortauthor}, \bibfield{author},  \bibfield{shorteditor}, \bibfield{editor}, 或\bibfield{translator}域的复制值,正常情况以该顺序检测。如果没有作者(authors)和编者(authors),该列表时未定义的。注意该列表也与\opt{use$<$name$>$}相关,见\secref{use:opt:bib}。引用样式打印引用中的姓名时使用这一列表。提供该列表仅为方便起见,没有附加的意义。\BiberOnlyMark 使用\biber 时,该域可以定制,详见\secref{aut:ctm:fld}。
%The name to be printed in citations. This list is a copy of either the \bibfield{shortauthor}, the \bibfield{author}, the \bibfield{shorteditor}, the \bibfield{editor}, or the \bibfield{translator} list, which are normally checked for in this order. If no authors and editors are available, this list is undefined. Note that this list is also responsive to the \opt{use$<$name$>$}, options, see \secref{use:opt:bib}. Citation styles should use this list when printing the name in a citation. This list is provided for convenience only and does not carry any additional meaning.This field may be customized. See \secref{aut:ctm:fld} for details.

\fielditem{labelnumber}{literal}

参考文献条目的序号,用于顺序编码类的样式。如果定义了\bibfield{shorthand}域,\biblatex 不再给各条目赋予一个数值。这种情况下,\bibfield{labelnumber}就是shorthand而不是一个数字。顺序编码类的样式必须使用该域的值而不是一个计数器值。注意: 包选项\opt{labelnumber}要求使用该域,详见\secref{use:opt:pre:int}。另可参见\secref{use:opt:pre:gen}节的\opt{defernumbers}选项。
%The number of the bibliography entry, as required by numeric citation schemes. If the \bibfield{shorthand} field is defined, \biblatex does not assign a number to the respective entry. In this case \bibfield{labelnumber} is the shorthand rather than a number. Numeric styles must use the value of this field instead of a counter. Note that this field needs to be requested with the package option \opt{labelnumber}, see \secref{use:opt:pre:int} for details. Also see the package option \opt{defernumbers} in \secref{use:opt:pre:gen}.

\fielditem{labelprefix}{literal}

如果为了在一个subbibliography文献表的所有条目前都添加一个固定的字符串,设置了\cmd{newrefcontext}命令的\opt{labelprefix}选项,那么所有受影响的\bibfield{labelprefix}域将提供该字符串。如果未设置前缀,相应条目的\bibfield{labelprefix}域是未定义的。详见\secref{use:bib:context}节\cmd{newrefcontext}命令的\opt{labelprefix}选项。如果定义了\bibfield{shorthand}域,\biblatex 不会给相应条目的\bibfield{labelprefix}域设置前缀。这种情况下\bibfield{labelprefix}是未定义的。
%If the \opt{labelprefix} option of \cmd{newrefcontext} has been set in order to prefix all entries in a subbibliography with a fixed string, this string is available in the \bibfield{labelprefix} field of all affected entries. If no prefix has been set, the \bibfield{labelprefix} field of the respective entry is undefined. See the \opt{labelprefix} option of \cmd{newrefcontext} in \secref{use:bib:context} for details. If the \bibfield{shorthand} field is defined, \biblatex does not assign the prefix to the \bibfield{labelprefix} field of the respective entry. In this case, the \bibfield{labelprefix} field is undefined.

\fielditem{labeltitle}{literal}
一篇文献可打印题名(或标题)。在一些环境中,一个样式可能需要在一些可能的标题域中选择一个标题。例如,标注样式打印短标题可能需要打印\bibfield{shorttitle}域,如果它存在的话,否则将打印\bibfield{title}域。构建\bibfield{labeltitle}时考虑的域的列表可以自定义。详见 \secref{aut:ctm:fld}。注意:\opt{labeltitle}包选项要求使用\bibfield{extratitle}域,详见\secref{use:opt:pre:int}。另可参见\bibfield{extratitle}。也要注意,\opt{labeltitleyear}需要包选项需要\bibfield{extratitleyear}域,另可参见\bibfield{extratitleyear}。
%The printable title of a work. In some circumstances, a style might need to choose a title from a list of a possible title fields. For example, citation styles printing short titles may want to print the \bibfield{shorttitle} field if it exists but otherwise print the \bibfield{title} field. The list of fields to be considered when constructing \bibfield{labeltitle} may be customized. See \secref{aut:ctm:fld} for details. Note that the \bibfield{extratitle} field needs to be requested with the package option \opt{labeltitle}, see \secref{use:opt:pre:int} for details. See also \bibfield{extratitle}. Note also that the \bibfield{extratitleyear} field needs to be requested with the package option \opt{labeltitleyear}. See also \bibfield{extratitleyear}.

\fielditem{extratitle}{integer}

该命令有时很有用,比如在author"=title标注样式中,用于区别标题相同的文献。当有文献具有相同的\bibfield{labelname}和\bibfield{labeltitle},\bibfield{extratitle}域保存一个整数,可以利用\cmd{mknumalph}转换为一个字母或者以其它方式格式化(或者可以仅仅作为一个标志,用于表示将一些其它域比如日期与\bibfield{labeltitle}域合并)。当文献表中具有相同\bibfield{labeltitle}和\bibfield{labelname}的文献只有一篇时,该域不定义。\footnote{there is only one work with the same \bibfield{labeltitle} by the same \bibfield{labelname} in the bibliography?}注意:\bibfield{extratitle}域是\opt{labeltitle}包选项所要求使用,详见\secref{use:opt:pre:int}。另可参见\bibfield{labeltitle}。\ref{use:opt:tab1} 总结了各种\opt{extra*}计数器及其作用。
%It is sometimes useful, for example in author"=title citation schemes, to be able to disambiguate works with the same title. For works by the same \bibfield{labelname} with the same \bibfield{labeltitle}, the \bibfield{extratitle} field holds an integer which may be converted to a letter with \cmd{mknumalph} or formatted in some other way (or it can be merely used as a flag to say that some other field such as a date should be used in conjunction with the \bibfield{labeltitle} field). This field is undefined if there is only one work with the same \bibfield{labeltitle} by the same \bibfield{labelname} in the bibliography. Note that the \bibfield{extratitle} field needs to be requested with the package option \opt{labeltitle}, see \secref{use:opt:pre:int} for details. See also \bibfield{labeltitle}. Table \ref{use:opt:tab1} summarises the various \opt{extra*} disambiguation counters and what they track.

\fielditem{extratitleyear}{integer}

该命令有时很有用,比如在author"=title标注样式中,用于区别标题相同年份相同但没有责任者的文献。当有文献具有相同的\bibfield{labeltitle}和\bibfield{labelyear},\bibfield{extratitleyear}域保存一个整数,可以利用\cmd{mknumalph}转换为一个字母或者以其它方式格式化(或者可以仅仅作为一个标志,用于表示将一些其它域比如出版者与\bibfield{labelyear}域合并)。当文献表中具有相同\bibfield{labeltitle}和\bibfield{labelyear}的文献只有一篇时,该域不定义。注意:\ bibfield{extratitle}域是\opt{labeltitleyear}包选项所要求使用,详见\secref{use:opt:pre:int}。另可参见\bibfield{labeltitleyear}。\ref{use:opt:tab1} 总结了各种\opt{extra*}计数器及其作用。
%It is sometimes useful, for example in author"=title citation schemes, to be able to disambiguate works with the same title in the same year but with no author. For works with the same \bibfield{labeltitle} and with the same \bibfield{labelyear}, the \bibfield{extratitleyear} field holds an integer which may be converted to a letter with \cmd{mknumalph} or formatted in some other way (or it can be merely used as a flag to say that some other field such as a publisher should be used in conjunction with the \bibfield{labelyear} field). This field is undefined if there is only one work with the same \bibfield{labeltitle} and \bibfield{labelyear} in the bibliography. Note that the \bibfield{extratitleyear} field needs to be requested with the package option \opt{labeltitleyear}, see \secref{use:opt:pre:int} for details. See also \bibfield{labeltitleyear}. Table \ref{use:opt:tab1} summarises the various \opt{extra*} disambiguation counters and what they track.

\fielditem{labelyear}{literal}

由\cmd{DeclareLabeldate}(\secref{aut:ctm:fld})命令选择的日期域的年或者\bibfield{year}域用于作者年制标签。一个完整的作者年标签由\bibfield{labelyear}加\bibfield{extrayear}域构成。注意\bibfield{labelyear}和\bibfield{extrayear}域是 \opt{labeldateparts}包选项要求使用的,详见\secref{use:opt:pre:int}。另可参见\bibfield{extrayear}。
%The year of the date field selected by \cmd{DeclareLabeldate} (\secref{aut:ctm:fld}) or the \bibfield{year} field, for use in author-year labels. A complete author-year label consists of the fields \bibfield{labelyear} plus \bibfield{extrayear}. Note that the \bibfield{labelyear} and \bibfield{extrayear} fields need to be requested with the package option \opt{labeldateparts}, see \secref{use:opt:pre:int} for details. See also \bibfield{extrayear}.

\fielditem{labelendyear}{literal}

\cmd{DeclareLabeldate} (\secref{aut:ctm:fld})命令选择的日期域的终止年,如果选择的日期是一个范围。
%The end year of the date field selected by \cmd{DeclareLabeldate} (\secref{aut:ctm:fld}) if the selected date is a range.

\fielditem{labelmonth}{datepart}

由\cmd{DeclareLabeldate}(\secref{aut:ctm:fld})命令选择的日期域的月或者\bibfield{month}域用于作者年制标签。注意\bibfield{labelmonth}域是 \opt{labeldateparts}包选项要求使用的,详见\secref{use:opt:pre:int}。
%The month of the date field selected by \cmd{DeclareLabeldate} (\secref{aut:ctm:fld}), or the \bibfield{month} field for use in author-year labels. Note that the \bibfield{labelmonth} field needs to be requested with the package option \opt{labeldateparts}, see \secref{use:opt:pre:int} for details.

\fielditem{labelendmonth}{datepart}

\cmd{DeclareLabeldate} (\secref{aut:ctm:fld})命令选择的日期域的终止月,如果选择的日期是一个范围。
%The end month of the date field selected by \cmd{DeclareLabeldate} (\secref{aut:ctm:fld}) if the selected date is a range.

\fielditem{labelday}{datepart}

由\cmd{DeclareLabeldate}(\secref{aut:ctm:fld})命令选择的日期域的日或者\bibfield{month}域用于作者年制标签。注意 \bibfield{labelday}域是 \opt{labeldateparts}包选项要求使用的,详见\secref{use:opt:pre:int}。
%The month of the date field selected by \cmd{DeclareLabeldate} (\secref{aut:ctm:fld}) for use in author-year labels. Note that the \bibfield{labelday} field needs to be requested with the package option \opt{labeldateparts}, see \secref{use:opt:pre:int} for details.

\fielditem{labelendday}{datepart}

\cmd{DeclareLabeldate} (\secref{aut:ctm:fld})命令选择的日期域的终止日,如果选择的日期是一个范围。
%The end day of the date field selected by \cmd{DeclareLabeldate} (\secref{aut:ctm:fld}) if the selected date is a range.

\fielditem{extrayear}{integer}

当参考文献表中包含两个或更多的具有相同作者的文献且出版年份也相同时,author"=year标注样式常需要在年后面附加一个字母以示区别。这种情况下,\bibfield{extrayear}域保存一个整数可以利用\cmd{mknumalph}转换为一个字母或者以其它方式格式化。当文献表中某作者的文献只有一篇或者所有该作者的文献的出版年不同时,该域不定义。完整的作者年标签由\bibfield{labelyear}加\bibfield{extrayear}域构成。注意\bibfield{labelyear} 和\bibfield{extrayear}域是\opt{labeldateparts}包选项所需要使用的,详见\secref{use:opt:pre:int}。另可参见\bibfield{labelyear}。\ref{use:opt:tab1} 总结了各种\opt{extra*}计数器及其作用。
%The author"=year citation scheme usually requires a letter to be appended to the year if the bibliography contains two or more works by the same author which were all published in the same year. In this case, the \bibfield{extrayear} field holds an integer which may be converted to a letter with \cmd{mknumalph} or formatted in some other way. This field is undefined if there is only one work by the author in the bibliography or if all works by the author have different publication years. A complete author-year label consists of the fields \bibfield{labelyear} plus \bibfield{extrayear}. Note that the \bibfield{labelyear} and \bibfield{extrayear} fields need to be requested with the package option \opt{labeldateparts}, see \secref{use:opt:pre:int} for details. See also \bibfield{labelyear}. Table \ref{use:opt:tab1} summarises the various \opt{extra*} disambiguation counters and what they track.

\end{fieldlist}

\paragraph{Date的成分域}%Date Component Fields
\label{aut:bbx:fld:dat}

注意,可以在数据模型中定义新的日期域,这些新定义的日期域的使用方式与本节将介绍的默认的数据模型类似。
%Note that it is possible to define new date fields in the datamodel which behave exactly like the default data model date fields described in this section.

\file{bib}文件中的日期域与央视接口提供的日期域如何关联详见\tabref{aut:bbx:fld:tab1}。当对样式中像\bibfield{origdate}这样的域进行判断时,使用如下代码:
%See \tabref{aut:bbx:fld:tab1} for an overview of how the date fields in \file{bib} files are related to the date fields provided by the style interface. When testing for a field like \bibfield{origdate} in a style, use code like:

\begin{ltxcode}
<<\iffieldundef>>{orig<<year>>}{...}{...}
\end{ltxcode}
%
它将告诉你相应的日期是否已定义。下面的判断:
%This will tell you if the corresponding date is defined at all. This test:

\begin{ltxcode}
<<\iffieldundef>>{orig<<endyear>>}{...}{...}
\end{ltxcode}
%
将告诉你相应的日期和一个(完全确定的)范围是否已定义。下面的判断
%will tell you if the corresponding date is defined and a (fully specified) range. This test:

\begin{ltxcode}
<<\iffieldequalstr>>{orig<<endyear>>}{}{...}{...}
\end{ltxcode}
%
将告诉你相应的日期和一个无终点的(open"=ended)范围已经定义。 Open"=ended范围由一个空的\texttt{endyear}成分表示(而不是一个未定义的\texttt{endyear}成分)。更多例子详见\secref{bib:use:dat}节和\pageref{bib:use:tab1}页的\tabref{bib:use:tab1}。
%will tell you if the corresponding date is defined and an open"=ended range. Open"=ended ranges are indicated by an empty \texttt{endyear} component (as opposed to an undefined \texttt{endyear} component). See \secref{bib:use:dat} and \tabref{bib:use:tab1} on page~\pageref{bib:use:tab1} for further examples.

{
\tablesetup
\begin{longtable}[l]{%
	@{}V{0.15\textwidth}%
	@{}V{0.4\textwidth}%
	@{}V{0.3\textwidth}%
	@{}V{0.2\textwidth}@{}}
\toprule
\multicolumn{2}{@{}H}{\file{bib} File} &
\multicolumn{2}{H}{Data Interface} \\
\cmidrule(r){1-2}\cmidrule(l){3-4}
\multicolumn{1}{@{}H}{Field} &
\multicolumn{1}{H}{Value (Example)} &
\multicolumn{1}{H}{Field} &
\multicolumn{1}{H}{Value (Example)} \\
\cmidrule{1-1}\cmidrule(lr){2-2}\cmidrule(l){3-3}\cmidrule(l){4-4}
date		& 1988			& day		& undefined \\
		&			& month		& undefined \\
		&			& year		& 1988 \\
		&			& season  & undefined \\
		&			& endday	& undefined \\
		&			& endmonth	& undefined \\
		&			& endyear	& undefined \\
		&			& endseason  & undefined \\
		&			& hour	& undefined \\
		&			& minute	& undefined \\
		&			& second	& undefined \\
		&			& timezone & undefined \\
		&			& endhour	& undefined \\
		&			& endminute	& undefined \\
		&			& endsecond	& undefined \\
		&			& endtimezone & undefined \\
date		& 1997/			& day		& undefined \\
		&			& month		& undefined \\
		&			& year		& 1997 \\
		&			& season  & undefined \\
		&			& endday	& undefined \\
		&			& endmonth	& undefined \\
		&			& endyear	& empty \\
		&			& endseason  & undefined \\
		&			& hour	& undefined \\
		&			& minute	& undefined \\
		&			& second	& undefined \\
		&			& timezone & undefined \\
		&			& endhour	& undefined \\
		&			& endminute	& undefined \\
		&			& endsecond	& undefined \\
		&			& endtimezone & undefined \\
urldate		& 2009-01-31		& urlday	& 31 \\
		&			& urlmonth	& 01 \\
		&			& urlyear	& 2009 \\
		&			& urlseason  & undefined \\
		&			& urlendday	& undefined \\
		&			& urlendmonth	& undefined \\
		&			& urlendyear	& undefined \\
		&			& urlendseason  & undefined \\
		&			& urlhour	& undefined \\
		&			& urlminute	& undefined \\
		&			& urlsecond	& undefined \\
		&			& urltimezone & undefined \\
		&			& urlendhour	& undefined \\
		&			& urlendminute	& undefined \\
		&			& urlendsecond	& undefined \\
		&			& urlendtimezone & undefined \\
urldate		& 2009-01-31T15:34:04Z		& urlday	& 31 \\
		&			& urlmonth	& 01 \\
		&			& urlyear	& 2009 \\
		&			& urlseason  & undefined \\
		&			& urlendday	& undefined \\
		&			& urlendmonth	& undefined \\
		&			& urlendyear	& undefined \\
		&			& urlendseason  & undefined \\
		&			& urlhour	& 15 \\
		&			& urlminute	& 34 \\
		&			& urlsecond	& 04 \\
		&			& urltimezone & Z \\
		&			& urlendhour	& undefined \\
		&			& urlendminute	& undefined \\
		&			& urlendsecond	& undefined \\
		&			& urlendtimezone & undefined \\
urldate		& 2009-01-31T15:34:04+05:00		& urlday	& 31 \\
		&			& urlmonth	& 01 \\
		&			& urlyear	& 2009 \\
		&			& urlseason  & undefined \\
		&			& urlendday	& undefined \\
		&			& urlendmonth	& undefined \\
		&			& urlendyear	& undefined \\
		&			& urlendseason  & undefined \\
		&			& urlhour	& 15 \\
		&			& urlminute	& 34 \\
		&			& urlsecond	& 04 \\
		&			& urltimezone & +0500 \\
		&			& urlendhour	& undefined \\
		&			& urlendminute	& undefined \\
		&			& urlendsecond	& undefined \\
		&			& urlendtimezone & undefined \\
urldate		& \parbox[t]{0.4\textwidth}{2009-01-31T15:34:04/\\2009-01-31T16:04:34}& urlday	& 31 \\
		&			& urlmonth	& 1 \\
		&			& urlyear	& 2009 \\
		&			& urlseason  & undefined \\
		&			& urlendday	& 31 \\
		&			& urlendmonth	& 1 \\
		&			& urlendyear	& 2009 \\
		&			& urlendseason  & undefined \\
		&			& urlhour	& 15 \\
		&			& urlminute	& 34 \\
		&			& urlsecond	& 4 \\
		&			& urltimezone & floating \\
		&			& urlendhour	& 16 \\
		&			& urlendminute	& 4 \\
		&			& urlendsecond	& 34 \\
		&			& urlendtimezone & floating \\
origdate	& 2002-21/2002-23	& origday	& undefined \\
		&			& origmonth	& 01 \\
		&			& origyear	& 2002 \\
		&			& origseason  & spring \\
		&			& origendday	& undefined \\
		&			& origendmonth	& 02 \\
		&			& origendyear	& 2002 \\
		&			& origendseason  & autumn \\
		&			& orighour	& undefined \\
		&			& origminute	& undefined \\
		&			& origsecond	& undefined \\
		&			& origtimezone & undefined \\
		&			& origendhour	& undefined \\
		&			& origendminute	& undefined \\
		&			& origendsecond	& undefined \\
		&			& origendtimezone & undefined \\
eventdate	& 1995-01-31/1995-02-05	& eventday	& 31 \\
		&			& eventmonth	& 01 \\
		&			& eventyear	& 1995 \\
		&			& eventseason  & undefined \\
		&			& eventendday	& 05 \\
		&			& eventendmonth	& 02 \\
		&			& eventendyear	& 1995 \\
		&			& eventendseason  & undefined \\
		&			& eventhour	& undefined \\
		&			& eventminute	& undefined \\
		&			& eventsecond	& undefined \\
		&			& eventtimezone & undefined \\
		&			& eventendhour	& undefined \\
		&			& eventendminute	& undefined \\
		&			& eventendsecond	& undefined \\
		&			& eventendtimezone & undefined \\
\bottomrule
%\end{tabularx}
\caption{日期接口(注意:biblatex3.7版提供的四个可解析日期接口,分别是date,origdate,eventdate,urldate,在多数场合已经够用)}%Date Interface
\label{aut:bbx:fld:tab1}
\end{longtable}
}

\begin{fieldlist}

\fielditem{hour}{datepart}

该域保存\bibfield{date}域的小时(hour)成分,当日期是一个范围时,它保存开始日期的小时成分。
%This field holds the hour component of the \bibfield{date} field. If the date is a range, it holds the hour component of the start date.

\fielditem{minute}{datepart}

该域保存\bibfield{date}域的分钟成分,当日期是一个范围时,它保存开始日期的分钟成分。
%This field holds the minute component of the \bibfield{date} field. If the date is a range, it holds the minute component of the start date.

\fielditem{second}{datepart}

该域保存\bibfield{date}域的秒钟成分,当日期是一个范围时,它保存开始日期的秒钟成分。
%This field holds the second component of the \bibfield{date} field. If the date is a range, it holds the second component of the start date.

\fielditem{timezone}{datepart}

该域保存\bibfield{date}域的时区成分,当日期是一个范围时,它保存开始日期的时区成分。
%This field holds the timezone component of the \bibfield{date} field. If the date is a range, it holds the timezone component of the start date.

\fielditem{day}{datepart}

该域保存\bibfield{date}域的日成分,当日期是一个范围时,它保存开始日期的日成分。
%This field holds the day component of the \bibfield{date} field. If the date is a range, it holds the day component of the start date.

\fielditem{month}{datepart}

该域保存数据源文件中的\bibfield{month}域或者\bibfield{date}域的月成分,当日期是一个范围时,它保存开始日期的月成分。
%This field is the \bibfield{month} as given in the database file or it holds the month component of the \bibfield{date} field. If the date is a range, it holds the month component of the start date.

\fielditem{year}{datepart}

该域保存数据源文件中的\bibfield{year}域或者\bibfield{date}域的年成分,当日期是一个范围时,它保存开始日期的年成分。
%This field is the \bibfield{year} as given in the database file or it holds the year component of the \bibfield{date} field. If the date is a range, it holds the year component of the start date.

\fielditem{season}{datepart}

该域保存由\acr{EDTF} 5.2.5(见\secref{bib:use:dat})规定的\bibfield{date}域的季节成分,它包含一个季节本地化字符串。当日期是一个范围时,它保存开始日期的季节成分。
%This field holds the season component of the \bibfield{date} field as specified by \acr{EDTF} 5.2.5 (\secref{bib:use:dat}). It contains a season localisation string (\secref{aut:lng:key:dt}). If the date is a range, it holds the season component of the start date.

\fielditem{endhour}{datepart}

如果\bibfield{date}域中给出的日期是一个范围,该域保存结束日期的小时成分。
%If the date specification in the \bibfield{date} field is a range, this field holds the hour component of the end date.

\fielditem{endminute}{datepart}

如果\bibfield{date}域中给出的日期是一个范围,该域保存结束日期的分钟成分。
%If the date specification in the \bibfield{date} field is a range, this field holds the minute component of the end date.

\fielditem{endsecond}{datepart}

如果\bibfield{date}域中给出的日期是一个范围,该域保存结束日期的秒钟成分。
%If the date specification in the \bibfield{date} field is a range, this field holds the second component of the end date.

\fielditem{endtimezone}{datepart}

如果\bibfield{date}域中给出的日期是一个范围,该域保存结束日期的时区成分。
%If the date specification in the \bibfield{date} field is a range, this field holds the timezone component of the end date.

\fielditem{endday}{datepart}

如果\bibfield{date}域中给出的日期是一个范围,该域保存结束日期的日成分。
%If the date specification in the \bibfield{date} field is a range, this field holds the day component of the end date.

\fielditem{endmonth}{datepart}

如果\bibfield{date}域中给出的日期是一个范围,该域保存结束日期的月成分。
%If the date specification in the \bibfield{date} field is a range, this field holds the month component of the end date.

\fielditem{endyear}{datepart}

如果\bibfield{date}域中给出的日期是一个范围,该域保存结束日期的年成分。空的(但已定义)的\bibfield{endyear}成分表示无终点的日期范围。
%If the date specification in the \bibfield{date} field is a range, this field holds the year component of the end date. A blank (but defined) \bibfield{endyear} component indicates an open ended \bibfield{date} range.

\fielditem{endseason}{datepart}

如果\bibfield{date}域中给出的日期是一个范围,该域保存\acr{EDTF} 5.2.5 (\secref{bib:use:dat})规定的结束日期的季节成分。它包含一个季节本地化字符串(见\secref{aut:lng:key:dt}),空的(但已定义)的\bibfield{endseason}成分表示无终点的日期范围。
%If the date specification in the \bibfield{date} field is a range, this field holds the season component of the end date as specified by \acr{EDTF} 5.2.5 (\secref{bib:use:dat}). It contains a season localisation string (\secref{aut:lng:key:dt}). A blank (but defined) \bibfield{endseason} component indicates an open ended \bibfield{date} range.

\fielditem{orighour}{datepart}

该域保存\bibfield{origdate}域的小时(hour)成分,当日期是一个范围时,它保存开始日期的小时成分。
%This field holds the hour component of the \bibfield{origdate} field. If the date is a range, it holds the hour component of the start date.

\fielditem{origminute}{datepart}

该域保存\bibfield{origdate}域的分钟成分,当日期是一个范围时,它保存开始日期的分钟成分。
%This field holds the minute component of the \bibfield{origdate} field. If the date is a range, it holds the minute component of the start date.

\fielditem{origsecond}{datepart}

该域保存\bibfield{origdate}域的秒钟成分,当日期是一个范围时,它保存开始日期的秒钟成分。
%This field holds the second component of the \bibfield{origdate} field. If the date is a range, it holds the second component of the start date.

\fielditem{origtimezone}{datepart}

该域保存\bibfield{origdate}域的时区成分,当日期是一个范围时,它保存开始日期的时区成分。
%This field holds the timezone component of the \bibfield{origdate} field. If the date is a range, it holds the timezone component of the start date.

\fielditem{origday}{datepart}

该域保存\bibfield{origdate}域的日成分,当日期是一个范围时,它保存开始日期的日成分。
%This field holds the day component of the \bibfield{origdate} field. If the date is a range, it holds the day component of the start date.

\fielditem{origmonth}{datepart}

该域保存\bibfield{origdate}域的月成分,当日期是一个范围时,它保存开始日期的月成分。
%This field holds the month component of the \bibfield{origdate} field. If the date is a range, it holds the month component of the start date.

\fielditem{origyear}{datepart}

该域保存\bibfield{origdate}域的年成分,当日期是一个范围时,它保存开始日期的年成分。
%This field holds the year component of the \bibfield{origdate} field. If the date is a range, it holds the year component of the start date.

\fielditem{origseason}{datepart}

该域保存由\acr{EDTF} 5.2.5(见\secref{bib:use:dat})规定的\bibfield{origdate}域的季节成分,它包含一个季节本地化字符串。当日期是一个范围时,它保存开始日期的季节成分。
%This field holds the season component of the \bibfield{origdate} field as specified by \acr{EDTF} 5.2.5 (\secref{bib:use:dat}). It contains a season localisation string (\secref{aut:lng:key:dt}). If the date is a range, it holds the season component of the start date.

\fielditem{origendhour}{datepart}

如果\bibfield{origdate}域中给出的日期是一个范围,该域保存结束日期的小时成分。
%If the date specification in the \bibfield{origdate} field is a range, this field holds the hour component of the end date.

\fielditem{origendminute}{datepart}

如果\bibfield{origdate}域中给出的日期是一个范围,该域保存结束日期的分钟成分。
%If the date specification in the \bibfield{origdate} field is a range, this field holds the minute component of the end date.

\fielditem{origendsecond}{datepart}

如果\bibfield{origdate}域中给出的日期是一个范围,该域保存结束日期的秒钟成分。
%If the date specification in the \bibfield{origdate} field is a range, this field holds the second component of the end date.

\fielditem{origendtimezone}{datepart}

如果\bibfield{origdate}域中给出的日期是一个范围,该域保存结束日期的时区成分。
%If the date specification in the \bibfield{origdate} field is a range, this field holds the timezone component of the end date.

\fielditem{origendday}{datepart}

如果\bibfield{origdate}域中给出的日期是一个范围,该域保存结束日期的日成分。
%If the date specification in the \bibfield{origdate} field is a range, this field holds the day component of the end date.

\fielditem{origendmonth}{datepart}

如果\bibfield{origdate}域中给出的日期是一个范围,该域保存结束日期的月成分。
%If the date specification in the \bibfield{origdate} field is a range, this field holds the month component of the end date.

\fielditem{origendyear}{datepart}

如果\bibfield{origdate} 域中给出的日期是一个范围,该域保存结束日期的年成分。空的(但已定义)的\bibfield{origendyear}成分表示无终点的日期范围。
%If the date specification in the \bibfield{origdate} field is a range, this field holds the year component of the end date. A blank (but defined) \bibfield{origendyear} component indicates an open ended \bibfield{origdate} range.

\fielditem{origendseason}{datepart}

如果\bibfield{origdate}域中给出的日期是一个范围,该域保存\acr{EDTF} 5.2.5 (\secref{bib:use:dat})规定的结束日期的季节成分。它包含一个季节本地化字符串(见\secref{aut:lng:key:dt}),空的(但已定义)的\bibfield{origendseason}成分表示无终点的\bibfield{origdate}范围。
%If the date specification in the \bibfield{origdate} field is a range, this field holds the season component of the end date as specified by \acr{EDTF} 5.2.5 (\secref{bib:use:dat}). It contains a season localisation string (\secref{aut:lng:key:dt}). A blank (but defined) \bibfield{origendseason} component indicates an open ended \bibfield{origdate} range.

\fielditem{eventhour}{datepart}

该域保存\bibfield{eventdate}域的小时(hour)成分,当日期是一个范围时,它保存开始日期的小时成分。
%This field holds the hour component of the \bibfield{eventdate} field. If the date is a range, it holds the hour component of the start date.

\fielditem{eventminute}{datepart}

该域保存\bibfield{eventdate}域的分钟成分,当日期是一个范围时,它保存开始日期的分钟成分。
%This field holds the minute component of the \bibfield{eventdate} field. If the date is a range, it holds the minute component of the start date.

\fielditem{eventsecond}{datepart}

该域保存\bibfield{eventdate}域的秒钟成分,当日期是一个范围时,它保存开始日期的秒钟成分。
%This field holds the second component of the \bibfield{eventdate} field. If the date is a range, it holds the second component of the start date.

\fielditem{eventtimezone}{datepart}

该域保存\bibfield{eventdate}域的时区成分,当日期是一个范围时,它保存开始日期的时区成分。
%This field holds the timezone component of the \bibfield{eventdate} field. If the date is a range, it holds the timezone component of the start date.

\fielditem{eventday}{datepart}

该域保存\bibfield{eventdate}域的日成分,当日期是一个范围时,它保存开始日期的日成分。
%This field holds the day component of the \bibfield{eventdate} field. If the date is a range, it holds the day component of the start date.

\fielditem{eventmonth}{datepart}

该域保存\bibfield{eventdate}域的月成分,当日期是一个范围时,它保存开始日期的月成分。
%This field holds the month component of the \bibfield{eventdate} field. If the date is a range, it holds the month component of the start date.

\fielditem{eventyear}{datepart}

该域保存\bibfield{eventdate}域的年成分,当日期是一个范围时,它保存开始日期的年成分
%This field holds the year component of the \bibfield{eventdate} field. If the date is a range, it holds the year component of the start date.

\fielditem{eventseason}{datepart}

该域保存由\acr{EDTF} 5.2.5(见\secref{bib:use:dat})规定的\bibfield{eventdate}域的季节成分,它包含一个季节本地化字符串。当日期是一个范围时,它保存开始日期的季节成分。
%This field holds the season component of the \bibfield{eventdate} field as specified by \acr{EDTF} 5.2.5 (\secref{bib:use:dat}). It contains a season localisation string (\secref{aut:lng:key:dt}). If the date is a range, it holds the season component of the start date.

\fielditem{eventendhour}{datepart}

如果\bibfield{eventdate}域中给出的日期是一个范围,该域保存结束日期的小时成分。
%If the date specification in the \bibfield{eventdate} field is a range, this field holds the hour component of the end date.

\fielditem{eventendminute}{datepart}

如果\bibfield{eventdate}域中给出的日期是一个范围,该域保存结束日期的分钟成分。
%If the date specification in the \bibfield{eventdate} field is a range, this field holds the minute component of the end date.

\fielditem{eventendsecond}{datepart}

如果\bibfield{eventdate}域中给出的日期是一个范围,该域保存结束日期的秒钟成分。
%If the date specification in the \bibfield{eventdate} field is a range, this field holds the second component of the end date.

\fielditem{eventendtimezone}{datepart}

如果\bibfield{eventdate}域中给出的日期是一个范围,该域保存结束日期的时区成分。
%If the date specification in the \bibfield{eventdate} field is a range, this field holds the timezone component of the end date.

\fielditem{eventendday}{datepart}

如果\bibfield{eventdate}域中给出的日期是一个范围,该域保存结束日期的日成分。
%If the date specification in the \bibfield{eventdate} field is a range, this field holds the day component of the end date.

\fielditem{eventendmonth}{datepart}

如果\bibfield{eventdate}域中给出的日期是一个范围,该域保存结束日期的月成分。
%If the date specification in the \bibfield{eventdate} field is a range, this field holds the month component of the end date.

\fielditem{eventendyear}{datepart}

如果\bibfield{eventdate} 域中给出的日期是一个范围,该域保存结束日期的年成分。空的(但已定义)的\bibfield{eventendyear}成分表示无终点的日期范围。
%If the date specification in the \bibfield{eventdate} field is a range, this field holds the year component of the end date. A blank (but defined) \bibfield{eventendyear} component indicates an open ended \bibfield{eventdate} range.

\fielditem{eventendseason}{datepart}

如果\bibfield{eventdate}域中给出的日期是一个范围,该域保存\acr{EDTF} 5.2.5 (\secref{bib:use:dat})规定的结束日期的季节成分。它包含一个季节本地化字符串(见\secref{aut:lng:key:dt}),空的(但已定义)的\bibfield{eventendseason}成分表示无终点的\bibfield{eventdate}范围。
%If the date specification in the \bibfield{eventdate} field is a range, this field holds the season component of the end date as specified by \acr{EDTF} 5.2.5 (\secref{bib:use:dat}). It contains a season localisation string (\secref{aut:lng:key:dt}). A blank (but defined) \bibfield{eventendseason} component indicates an open ended \bibfield{eventdate} range.

\fielditem{urlhour}{datepart}

该域保存\bibfield{urldate}域的小时(hour)成分,当日期是一个范围时,它保存开始日期的小时成分。
%This field holds the hour component of the \bibfield{urldate} field. If the date is a range, it holds the hour component of the start date.

\fielditem{urlminute}{datepart}

该域保存\bibfield{urldate}域的分钟成分,当日期是一个范围时,它保存开始日期的分钟成分。
%This field holds the minute component of the \bibfield{urldate} field. If the date is a range, it holds the minute component of the start date.

\fielditem{urlsecond}{datepart}

该域保存\bibfield{urldate}域的秒钟成分,当日期是一个范围时,它保存开始日期的秒钟成分。
%This field holds the second component of the \bibfield{urldate} field. If the date is a range, it holds the second component of the start date.

\fielditem{timezone}{urldatepart}

该域保存\bibfield{urldate}域的时区成分,当日期是一个范围时,它保存开始日期的时区成分。
%This field holds the timezone component of the \bibfield{urldate} field. If the date is a range, it holds the timezone component of the start date.

\fielditem{urlday}{datepart}

该域保存\bibfield{urldate}域的日成分。
%This field holds the day component of the \bibfield{urldate} field.

\fielditem{urlmonth}{datepart}

该域保存\bibfield{urldate}域的月成分。
%This field holds the month component of the \bibfield{urldate} field.

\fielditem{urlyear}{datepart}

该域保存\bibfield{urldate}域的年成分。
%This field holds the year component of the \bibfield{urldate} field.

\fielditem{urlseason}{datepart}

该域保存由\acr{EDTF} 5.2.5(见\secref{bib:use:dat})规定的 \bibfield{urldate} 域的季节成分,它包含一个季节本地化字符串。当日期是一个范围时,它保存开始日期的季节成分。
%This field holds the season component of the \bibfield{urldate} field as specified by \acr{EDTF} 5.2.5 (\secref{bib:use:dat}). It contains a season localisation string (\secref{aut:lng:key:dt}). If the date is a range, it holds the season component of the start date.

\fielditem{urlendhour}{datepart}

如果\bibfield{urldate}域中给出的日期是一个范围,该域保存结束日期的小时成分
%If the date specification in the \bibfield{urldate} field is a range, this field holds the hour component of the end date.

\fielditem{urlendminute}{datepart}

如果\bibfield{urldate}域中给出的日期是一个范围,该域保存结束日期的分钟成分
%If the date specification in the \bibfield{urldate} field is a range, this field holds the minute component of the end date.

\fielditem{urlendsecond}{datepart}

如果\bibfield{urldate}域中给出的日期是一个范围,该域保存结束日期的秒钟成分
%If the date specification in the \bibfield{urldate} field is a range, this field holds the second component of the end date.

\fielditem{urlendtimezone}{datepart}

如果\bibfield{urldate}域中给出的日期是一个范围,该域保存结束日期的时区成分
%If the date specification in the \bibfield{urldate} field is a range, this field holds the timezone component of the end date.

\fielditem{urlendday}{datepart}

如果\bibfield{urldate}域中给出的日期是一个范围,该域保存结束日期的日成分
%If the date specification in the \bibfield{urldate} field is a range, this field holds the day component of the end date.

\fielditem{urlendmonth}{datepart}

如果\bibfield{urldate}域中给出的日期是一个范围,该域保存结束日期的月成分
%If the date specification in the \bibfield{urldate} field is a range, this field holds the month component of the end date.

\fielditem{urlendyear}{datepart}

如果\bibfield{urldate} 域中给出的日期是一个范围,该域保存结束日期的年成分。空的(但已定义)的\bibfield{urlendyear}成分表示无终点的日期范围。
%If the date specification in the \bibfield{urldate} field is a range, this field holds the year component of the end date. A blank (but defined) \bibfield{urlendyear} component indicates an open ended \bibfield{urldate} range.

\fielditem{urlendseason}{datepart}

如果\bibfield{urldate}域中给出的日期是一个范围,该域保存\acr{EDTF} 5.2.5 (\secref{bib:use:dat})规定的结束日期的季节成分。它包含一个季节本地化字符串(见\secref{aut:lng:key:dt}),空的(但已定义)的\bibfield{urlendseason}成分表示无终点的\bibfield{eventdate}范围。
%If the date specification in the \bibfield{urldate} field is a range, this field holds the season component of the end date as specified by \acr{EDTF} 5.2.5 (\secref{bib:use:dat}). It contains a season localisation string (\secref{aut:lng:key:dt}). A blank (but defined) \bibfield{urlendseason} component indicates an open ended \bibfield{urldate} range.

\end{fieldlist}

\subsection{标注样式}%Citation Styles
\label{aut:cbx}
标注样式是诸如\cmd{cite}等用于打印不同类型标注的命令集。这些样式定义在后缀为\file{cbx}的文件中。\biblatex 在包末尾加载它们。注意:一些标准标注样式的常用共享宏集放在\path{biblatex.def}文件中。这一文件也在包末尾加载,先于选择的标注样式。它也包含由来自\secref{use:cit:txt}节的命令的定义。
%A citation style is a set of commands such as \cmd{cite} which print different types of citations. Such styles are defined in files with the suffix \file{cbx}. The \biblatex package loads the selected citation style file at the end of the package. Note that a small repertory of frequently used macros shared by several of the standard citation styles is also included in \path{biblatex.def}. This file is loaded at the end of the package as well, prior to the selected citation style. It also contains the definitions of the commands from \secref{use:cit:txt}.

\subsubsection{标注样式文件}% Citation Style Files
\label{aut:cbx:cbx}
在讨论标注样式文件提供的各个命令前,考虑如下一个典型\file{cbx}文件的整体结构:
%Before we go over the individual commands available in citation style files, consider this example of the overall structure of a typical \file{cbx} file:

\begin{ltxexample}
\ProvidesFile{example.cbx}[2006/03/15 v1.0 biblatex citation style]

\DeclareCiteCommand{\cite}{...}{...}{...}{...}
\DeclareCiteCommand{\parencite}[\mkbibparens]{...}{...}{...}{...}
\DeclareCiteCommand{\footcite}[\mkbibfootnote]{...}{...}{...}{...}
\DeclareCiteCommand{\textcite}{...}{...}{...}{...}
\endinput
\end{ltxexample}

\begin{ltxsyntax}

\cmditem{RequireCitationStyle}{style}

这个命令是可选的,用于在一些更一般的样式基础上构建特殊的标注样式。它加载标注样式\path{style.cbx}。
%This command is optional and intended for specialized citation styles built on top of a more generic style. It loads the citation style \path{style.cbx}.

\cmditem{InitializeCitationStyle}{code}

指定初始化或重设标注样式需要的任意\prm{code}。这个钩子将在包加载的时候执行一次,并且每次都使用\secref{use:cit:msc}节的\cmd{citereset}命令。\cmd{citereset}命令也重设本宏包的内部标注追踪器。它会影响\secref{aut:aux:tst}节中列出的\ cmd{ifciteseen}, \cmd{ifentryseen}, \cmd{ifciteibid}, 和\cmd{ifciteidem}等判断。当使用\env{refsection}环境时,标注追踪器重设当前的\env{refsection}局部环境。
%Specifies arbitrary \prm{code} required to initialize or reset the citation style. This hook will be executed once at package load-time and every time the \cmd{citereset} command from \secref{use:cit:msc} is used. The \cmd{citereset} command also resets the internal citation trackers of this package. The reset will affect the \cmd{ifciteseen}, \cmd{ifentryseen}, \cmd{ifciteibid}, and \cmd{ifciteidem} tests discussed in \secref{aut:aux:tst}. When used in a \env{refsection} environment, the reset of the citation tracker is local to the current \env{refsection} environment.

\cmditem{OnManualCitation}{code}
指定重设部分标注样式需要的任意\prm{code}。这一钩子将在\secref{use:cit:msc}中的\cmd{mancite}命令使用时调用。它有时特别有用,可以代替像 <ibidem>或<op. cit.>等缩写表示的重复标注,因为当自动生成和人工产生的标注混合使用的时候这些缩写可能会有歧义。\cmd{mancite}命令也会重设宏包的内部<ibidem>和<idem>追踪器,进而影响\secref{aut:aux:tst}节讨论的\cmd{ifciteibid}和\cmd{ifciteidem}判断。
%Specifies arbitrary \prm{code} required for a partial reset of the citation style. This hook will be executed every time the \cmd{mancite} command from \secref{use:cit:msc} is used. It is particularly useful in citation styles which replace repeated citations by abbreviations like <ibidem> or <op. cit.> which may get ambiguous if automatically generated and manual citations are mixed. The \cmd{mancite} command also resets the internal <ibidem> and <idem> trackers of this package. The reset will affect the \cmd{ifciteibid} and \cmd{ifciteidem} tests discussed in \secref{aut:aux:tst}.

\cmditem{DeclareCiteCommand}{command}[wrapper]{precode}{loopcode}{sepcode}{postcode} \cmditem*{DeclareCiteCommand*}{command}[wrapper]{precode}{loopcode}{sepcode}{postcode}

这是用于定义所有标注(引用)命令的核心命令。它有1个可选参数和5个必选参数。\prm{command}是要定义的命令,比如\cmd{cite}。如果给出可选的\prm{wrapper}参数,整个标注将会作为一个参数传递给\prm{wrapper},即包围(wrapper)命令必须要取得一个必选参数。\footnote{典型的包围命令是\cmd{mkbibparens}和\cmd{mkbibfootnote}。}
%This is the core command used to define all citation commands. It takes one optional and five mandatory arguments. The \prm{command} is the command to be defined, for example \cmd{cite}. If the optional \prm{wrapper} argument is given, the entire citation will be passed to the \prm{wrapper} as an argument, \ie the wrapper command must take one mandatory argument.\footnote{Typical examples of wrapper commands are \cmd{mkbibparens} and \cmd{mkbibfootnote}.}
\prm{precode}是在标注开始时执行的任意代码。典型地,它将处理由\bibfield{prenote}域提供的\prm{prenote}参数。它可以可用来对\prm{loopcode}所需的宏进行初始化。\prm{loopcode}是每个条目关键词传递给\prm{command}命令时执行的任意代码。它是打印标注标签或其它任意数据的核心代码。\prm{sepcode}是每次执行\prm{loopcode}完成后执行的代码。它仅在条目关键词列表传递给\prm{command}时起作用。\prm{sepcode}常用于插入一些分隔符,比如逗号或分号等。
%The \prm{precode} is arbitrary code to be executed at the beginning of the citation. It will typically handle the \prm{prenote} argument which is available in the \bibfield{prenote} field. It may also be used to initialize macros required by the \prm{loopcode}. The \prm{loopcode} is arbitrary code to be executed for each entry key passed to the \prm{command}. This is the core code which prints the citation labels or any other data. The \prm{sepcode} is arbitrary code to be executed after each iteration of the \prm{loopcode}. It will only be executed if a list of entry keys is passed to the \prm{command}. The \prm{sepcode} will usually insert some kind of separator, such as a comma or a semicolon.
\prm{postcode}是在标注结束时执行的代码。典型地,它将处理由\bibfield{postnote}域提供的\prm{postnote}参数。\footnote{能给\prm{loopcode}提供的参考文献数据是正在处理的条目的数据。此外,第一个(<First>)条目的数据可以用于\prm{precode},最后一个(<last>)条目的数据可以用于\prm{postcode}。<First> and <last>指的是标注的打印顺序。如果\opt{sortcites}包选项打开,这是经过排序处理后的顺序。注意: 没有任何参考文献数据可用于\prm{sepcode}。}
%The \prm{postcode} is arbitrary code to be executed at the end of the citation. The \prm{postcode} will typically handle the \prm{postnote} argument which is available in the \bibfield{postnote} field.\footnote{The bibliographic data available to the \prm{loopcode} is the data of the entry currently being processed. In addition to that, the data of the first entry is available to the \prm{precode} and the data of the last one is available to the \prm{postcode}. <First> and <last> refer to the order in which the citations are printed. If the \opt{sortcites} package option is active, this is the order of the list after sorting. Note that no bibliographic data is available to the \prm{sepcode}.}
带星号的\cmd{DeclareCiteCommand}命令定义了一个带星号的\prm{command}。例如|\DeclareCiteCommand*{cite}|命令将定义|\cite*|。
\footnote{注意:无星号的\cmd{DeclareCiteCommand}命令也将定义隐式的定义一个带星号的标注命令,除非该标注命令前面已经定义。这只是用于提供备选。这种隐式方式定义的命令将等同于不带星号的命令。}
%The starred variant of \cmd{DeclareCiteCommand} defines a starred \prm{command}. For example, |\DeclareCiteCommand*{cite}| would define |\cite*|.\footnote{Note that the regular variant of \cmd{DeclareCiteCommand} defines a starred version of the \prm{command} implicitly, unless the starred version has been defined before. This is intended as a fallback. The implicit definition is an alias for the regular variant.}

\cmditem{DeclareMultiCiteCommand}{command}[wrapper]{cite}{delimiter}

该命令定义<multicite>类命令(见\secref{use:cit:mlt})。\prm{command}是要定义的multicite命令,比如\cmd{cites}。它自动在由\cmd{DeclareCiteCommand}定义的后端命令基础上构建鲁棒的命令,其中\prm{cite}参数用于指定使用的后端命令名。注意后端命令的包围命令(封套)(即传递给\cmd{DeclareCiteCommand}命令的\prm{wrapper}参数)自动忽略。使用可选\prm{wrapper}参数作为其替换。\prm{delimiter}是列表中单个标注之间的分隔字符串。下面给出的例子是典型的\cmd{multicitedelim}命令,取自\path{biblatex.def}中的实际定义:
%This command defines <multicite> commands (\secref{use:cit:mlt}). The \prm{command} is the multicite command to be defined, for example \cmd{cites}. It is automatically made robust. Multicite commands are built on top of backend commands defined with \cmd{DeclareCiteCommand} and the \prm{cite} argument specifies the name of the backend command to be used. Note that the wrapper of the backend command (\ie the \prm{wrapper} argument passed to \cmd{DeclareCiteCommand}) is ignored. Use the optional \prm{wrapper} argument to specify an alternative wrapper. The \prm{delimiter} is the string to be printed as a separator between the individual citations in the list. This will typically be \cmd{multicitedelim}. The following examples are real definitions taken from \path{biblatex.def}:

\begin{ltxexample}
\DeclareMultiCiteCommand{\cites}%
	{\cite}{\multicitedelim}
\DeclareMultiCiteCommand{\parencites}[\mkbibparens]%
	{\parencite}{\multicitedelim}
\DeclareMultiCiteCommand{\footcites}[\mkbibfootnote]%
	{\footcite}{\multicitedelim}
\end{ltxexample}

\cmditem{DeclareAutoCiteCommand}{name}[position]{cite}{multicite}

该命令为\cmd{autocite}和\cmd{autocites}类命令提供定义(见\secref{use:cit:aut})。要使定义生效需要打开 \secref{use:opt:pre:gen}节的\opt{autocite}包选项。\prm{name}是一个标识向包选项传递一个值。autocite类命令是在\cmd{parencite}和\cmd{parencites}等后端命令基础上构建的。\prm{cite}和\prm{multicite}参数指定了使用的后端命令。\prm{cite}参数用于\cmd{autocite},而\prm{multicite}用于\cmd{autocites}。
%This command provides definitions for the \cmd{autocite} and \cmd{autocites} commands from \secref{use:cit:aut}. The definitions are enabled with the \opt{autocite} package option from \secref{use:opt:pre:gen}. The \prm{name} is an identifier which serves as the value passed to the package option. The autocite commands are built on top of backend commands like \cmd{parencite} and \cmd{parencites}. The arguments \prm{cite} and \prm{multicite} specify the backend commands to use. The \prm{cite} argument refers to \cmd{autocite} and \prm{multicite} refers to \cmd{autocites}.
\prm{position}参数控制标注后的任何标点符号的处理。可能的值是\texttt{l}, \texttt{r}, \texttt{f}。\texttt{r}表示标点置于标注的右侧,即它不会移动。 \texttt{l}表示将标点移动到标注的左侧。\texttt{f}在脚注中的作用类似于\texttt{r},在其它情况下则类似于\texttt{l}。该参数是可选的默认是\texttt{r}。另可参见\secref{aut:pct:cfg}节的\cmd{DeclareAutoPunctuation}命令和\secref{use:opt:pre:gen}节的\opt{autopunct}包选项。下面的例子取自\path{biblatex.def}中的实际定义:
%The \prm{position} argument controls the handling of any punctuation marks after the citation. Possible values are \texttt{l}, \texttt{r}, \texttt{f}. \texttt{r} means that the punctuation is placed to the right of the citation, \ie it will not be moved around. \texttt{l} means that any punctuation after the citation is moved to the left of the citation. \texttt{f} is like \texttt{r} in a footnote and like \texttt{l} otherwise. This argument is optional and defaults to \texttt{r}. See also \cmd{DeclareAutoPunctuation} in \secref{aut:pct:cfg} and the \opt{autopunct} package option in \secref{use:opt:pre:gen}. The following examples are real definitions taken from \path{biblatex.def}:

\begin{ltxexample}
\DeclareAutoCiteCommand{plain}{\cite}{\cites}
\DeclareAutoCiteCommand{inline}{\parencite}{\parencites}
\DeclareAutoCiteCommand{footnote}[l]{\footcite}{\footcites}
\DeclareAutoCiteCommand{footnote}[f]{\smartcite}{\smartcites}
\end{ltxexample}
%
文档导言区提供的定义可以利用如下方式随后采用(见\secref{use:cfg:opt}):
%A definition provided in the document preamble can be subsequently adopted with the following: (see \secref{use:cfg:opt}).

\begin{ltxexample}
\ExecuteBibliographyOptions{autocite=<<name>>}
\end{ltxexample}

\end{ltxsyntax}

\subsubsection{特殊域}%Special Fields
\label{aut:cbx:fld}

下面的域用于向标注命令传递数据。它们不用于\file{bib}文件中而由宏包自动定义。从标注样式的角度看,它们与\file{bib}中的域并无区别。另可参见\secref{aut:bbx:fld}。
%The following fields are used by \biblatex to pass data to citation commands. They are not used in \file{bib} files but defined automatically by the package. From the perspective of a citation style, they are not different from the fields in a \file{bib} file. See also \secref{aut:bbx:fld}.

\begin{fieldlist}

\fielditem{prenote}{literal}

作为\prm{prenote}参数向标注命令传递。该域仅用于标注而不能用在参考文献表中。如果\prm{prenote}参数缺省或为空,该域不定义。
%The \prm{prenote} argument passed to a citation command. This field is specific to citations and not available in the bibliography. If the \prm{prenote} argument is missing or empty, this field is undefined.

\fielditem{postnote}{literal}

作为\prm{postnote}参数向标注命令传递。该域仅用于标注而不能用在参考文献表中。如果\prm{postnote}参数缺省或为空,该域不定义。
%The \prm{postnote} argument passed to a citation command. This field is specific to citations and not available in the bibliography. If the \prm{postnote} argument is missing or empty, this field is undefined.

\fielditem{multiprenote}{literal}

作为\prm{multiprenote}参数向multicite类标注命令传递。该域仅用于标注而不能用在参考文献表中。如果\prm{multiprenote}参数缺省或为空,该域不定义。
%The \prm{multiprenote} argument passed to a multicite command. This field is specific to citations and not available in the bibliography. If the \prm{multiprenote} argument is missing or empty, this field is undefined.

\fielditem{multipostnote}{literal}

作为\prm{multipostnote}参数向ulticite类标注命令传递。该域仅用于标注而不能用在参考文献表中。如果\prm{multipostnote}参数缺省或为空,该域不定义。
%The \prm{multipostnote} argument passed to a multicite command. This field is specific to citations and not available in the bibliography. If the \prm{multipostnote} argument is missing or empty, this field is undefined.

\fielditem{postpunct}{punctuation command}

作为拖尾的标点参数隐式地向标注命令传递。该域仅用于标注而不能用在参考文献表中。如果一个标注命令后面跟着的字符不在\cmd{DeclareAutoPunctuation} (\secref{aut:pct:cfg})命令的定义中,该域不定义。
%The trailing punctuation argument implicitly passed to a citation command. This field is specific to citations and not available in the bibliography. If the character following a given citation command is not specified in \cmd{DeclareAutoPunctuation} (\secref{aut:pct:cfg}), this field is undefined.

\end{fieldlist}

\subsection{数据接口}%Data Interface
\label{aut:bib}

数据接口是用于格式化和打印全部参考文献数据的工具。这些工具均可在著录和标注样式中使用。
%The data interface are the facilities used to format and print all bibliographic data. These facilities are available in both bibliography and citation styles.

\subsubsection{数据命令}%Data Commands
\label{aut:bib:dat}
本节介绍\biblatex 包的主要数据接口。这些命令处理了绝大部分工作,即实际上由它们来对列表和域提供的数据进行打印。
%This section introduces the main data interface of the \biblatex package. These are the commands doing most of the work, \ie they actually print the data provided in lists and fields.

\begin{ltxsyntax}

\cmditem{DeprecateField}{field}{message}
\cmditem{DeprecateList}{list}{message}
\cmditem{DeprecateName}{name}{message}


用于在打印\prm{field}, \prm{list}, \prm{name}时给出表示不允许的警告信息\prm{message}。它帮助那些需要在样式中修改域名的样式作者。注意: 不允许的项只能是未在当前工作的数据模型中定义的项,\prm{field}, \prm{list}或\prm{name}不能出现在\cmd{DeclareDatamodelFields}命令的参数中。
%When an attempt is made to print \prm{field}, \prm{list}, \prm{name}, a
%deprecation warning issued with the additional \prm{message}.  This aids
%style authors who are changing field names in their style. Note that the
%deprecated item must no longer be defined in the datamodel for this work;
%\prm{field}, \prm{list} or \prm{name} cannot be listed anywhere as an
%argument to \cmd{DeclareDatamodelFields}.

\cmditem{printfield}[format]{field}

该命令使用由\cmd{DeclareFieldFormat}定义的格式化指令\prm{format}打印\prm{field}。如果声明了type"=specific的格式指令,其将优先于设置的通用格式指令。如果\prm{field}未定义则不作打印。如果\prm{format}缺省,\cmd{printfield}将尝试域名作为格式化指令名进行格式化。例如:要打印\bibfield{title}域,且\prm{format}未指定,它将尝试用于域格式\texttt{title}。\footnote{换句话说,\texttt{\textbackslash printfield\{title\}}等价于\texttt{\textbackslash printfield[title]\{title\}}。}
%This command prints a \prm{field} using the formatting directive \prm{format}, as defined with \cmd{DeclareFieldFormat}. If a type"=specific \prm{format} has been declared, the type"=specific formatting directive takes precedence over the generic one. If the \prm{field} is undefined, nothing is printed. If the \prm{format} is omitted, \cmd{printfield} tries using the name of the field as a format name. For example, if the \bibfield{title} field is to be printed and the \prm{format} is not specified, it will try to use the field format \texttt{title}.\footnote{In other words, \texttt{\textbackslash printfield\{title\}} is equivalent to \texttt{\textbackslash printfield[title]\{title\}}.}
这种情况下,任何type"=specific(具体类型的)格式化指令将优先于通用指令采用。如果所有的这类格式都未定义,它将返回到\texttt{default}作为最后的方式。注意:\cmd{printfield}为格式化指令提供当前正在\cmd{currentfield}中处理的域名。
%In this case, any type"=specific formatting directive will also take precedence over the generic one. If all of these formats are undefined, it falls back to \texttt{default} as a last resort. Note that \cmd{printfield} provides the name of the field currently being processed in \cmd{currentfield} for use in field formatting directives.

\cmditem{printlist}[format][start\ensuremath\rangle--\ensuremath\langle stop]{literal list}
该命令对所有在\prm{literal list}中的项进行循环处理,从项数\prm{start}开始,到项数\prm{stop}结束,包括\prm{start}和\prm{stop}(所有的列表中项以1开始计数)。每一项都用由\cmd{DeclareListFormat}定义的格式化指令\prm{format}打印。如果声明了type"=specific的格式指令,其将优先于设置的通用格式指令。如果\prm{literal list}未定义则不作打印。如果\prm{format}缺省,\cmd{printlist}将尝试列表名作为格式化指令名进行格式化。这种情况下,任何type"=specific(具体类型的)格式化指令将优先于通用指令采用。如果所有的这类格式都未定义,它将返回到\texttt{default}作为最后的方式。\prm{start}参数默认是1,\prm{stop}默认是列表中项的总数。如果项的总数大于\prm{maxitems},\prm{stop}默认为\prm{minitems}(见\secref{use:opt:pre:gen})。更多细节参见\cmd{printnames}。注意:\cmd{printlist}为格式化指令提供当前正在\cmd{currentlist}中处理的域名。
%This command loops over all items in a \prm{literal list}, starting at item number \prm{start} and stopping at item number \prm{stop}, including \prm{start} and \prm{stop} (all lists are numbered starting at~1). Each item is printed using the formatting directive \prm{format}, as defined with \cmd{DeclareListFormat}. If a type"=specific \prm{format} has been declared, the type"=specific formatting directive takes precedence over the generic one. If the \prm{literal list} is undefined, nothing is printed. If the \prm{format} is omitted, \cmd{printlist} tries using the name of the list as a format name. In this case, any type"=specific formatting directive will also take precedence over the generic one. If all of these formats are undefined, it falls back to \texttt{default} as a last resort. The \prm{start} argument defaults to 1; \prm{stop} defaults to the total number of items in the list. If the total number is greater than \prm{maxitems}, \prm{stop} defaults to \prm{minitems} (see \secref{use:opt:pre:gen}). See \cmd{printnames} for further details. Note that \cmd{printlist} provides the name of the literal list currently being processed in \cmd{currentlist} for use in list formatting directives.

\cmditem{printnames}[format][start\ensuremath\rangle--\ensuremath\langle stop]{name list}
该命令对所有在\prm{name list}中的项进行循环处理,从项数\prm{start}开始,到项数\prm{stop}结束,包括\prm{start}和\prm{stop}(所有的列表中项以1开始计数)。每一项都用由\cmd{DeclareNameFormat}定义的格式化指令\prm{format}打印。如果声明了type"=specific的格式指令,其将优先于设置的通用格式指令。如果\prm{name list}未定义则不作打印。如果\prm{format}缺省, \cmd{printnames}将尝试列表名作为格式化指令名进行格式化。这种情况下,任何type"=specific(具体类型的)格式化指令将优先于通用指令采用。如果所有的这类格式都未定义,它将返回到\texttt{default}作为最后的方式。\prm{start}参数默认是1,\prm{stop}默认是列表中项的总数。如果项的总数大于\prm{maxnames},\prm{stop}默认为\prm{minnames}(见\secref{use:opt:pre:gen})。如果你要自己制定一个范围而又要使用默认的列表格式,第一个可选参数必须给出但要留空:
%This command loops over all items in a \prm{name list}, starting at item number \prm{start} and stopping at item number \prm{stop}, including \prm{start} and \prm{stop} (all lists are numbered starting at~1). Each item is printed using the formatting directive \prm{format}, as defined with \cmd{DeclareNameFormat}. If a type"=specific \prm{format} has been declared, the type"=specific formatting directive takes precedence over the generic one. If the \prm{name list} is undefined, nothing is printed. If the \prm{format} is omitted, \cmd{printnames} tries using the name of the list as a format name. In this case, any type"=specific formatting directive will also take precedence over the generic one. If all of these formats are undefined, it falls back to \texttt{default} as a last resort. The \prm{start} argument defaults to 1; \prm{stop} defaults to the total number of items in the list. If the total number is greater than \prm{maxnames}, \prm{stop} defaults to \prm{minnames} (see \secref{use:opt:pre:gen}). If you want to select a range but use the default list format, the first optional argument must still be given, but is left empty:

\begin{ltxexample}
\printnames[][1-3]{...}
\end{ltxexample}

\prm{start}和\prm{stop}之一可以缺省,因此下面的参数都是有效的:
%One of \prm{start} and \prm{stop} may be omitted, hence the following arguments are all valid:

\begin{ltxexample}
\printnames[...][-1]{...}
\printnames[...][2-]{...}
\printnames[...][1-3]{...}
\end{ltxexample}

如果你要覆盖\prm{maxnames}和\prm{minnames}并打印整个列表,你可以在第二个可选参数中以如下方式设置\cnt{listtotal}计数器。
%If you want to override \prm{maxnames} and \prm{minnames} and force printing of the entire list, you may refer to the \cnt{listtotal} counter in the second optional argument:

\begin{ltxexample}
\printnames[...][-\value{listtotal}]{...}
\end{ltxexample}

当\cmd{printnames}和\cmd{printlist}处理一个列表时,当前状态的信息可以通过4个计数器获知: \cnt{listtotal} 计数器保存当前列表中项的总数,\cnt{listcount}保存当前正在处理的项的序号,\cnt{liststart}是传递给\cmd{printnames} 或\cmd{printlist}命令的 \prm{start}参数,\cnt{liststop} 则是 \prm{stop}参数。这些计数器用于列表的格式化指令。\cnt{listtotal}也可以在第二个可选参数中使用。注意,这些计数器仅在列表格式化指令中有意义在其它任何地方都无效。对于每类列表,都有一个具有相同名字的计数器保存该类列表的项的总数。例如,\cnt{author}计数器保存\bibfield{author}列表中的项的总数。这些计数器类似于\cnt{listtotal},但可用于列表格式化指令之外。还有\cnt{maxnames} ,\cnt{minnames}, \cnt{maxitems}和\cnt{minitems}计数器,用于保存相应的包选项的值。这些内部计数器的完整列表详见\secref{aut:fmt:ilc}。注意: \cmd{printnames}为格式化指令提供当前正在\cmd{currentname}中处理的域名。
%Whenever \cmd{printnames} and \cmd{printlist} process a list, information concerning the current state is accessible by way of four counters: the \cnt{listtotal} counter holds the total number of items in the current list, \cnt{listcount} holds the number of the item currently being processed, \cnt{liststart} is the \prm{start} argument passed to \cmd{printnames} or \cmd{printlist}, \cnt{liststop} is the \prm{stop} argument. These counters are intended for use in list formatting directives. \cnt{listtotal} may also be used in the second optional argument to \cmd{printnames} and \cmd{printlist}. Note that these counters are local to list formatting directives and do not hold meaningful values when used anywhere else. For every list, there is also a counter by the same name which holds the total number of items in the corresponding list. For example, the \cnt{author} counter holds the total number of items in the \bibfield{author} list. These counters are similar to \cnt{listtotal} except that they may also be used independently of list formatting directives. There are also \cnt{maxnames} and \cnt{minnames} as well as \cnt{maxitems} and \cnt{minitems} counters which hold the values of the corresponding package options. See \secref{aut:fmt:ilc} for a complete list of such internal counters. Note that \cmd{printnames} provides the name of the name list currently being processed in \cmd{currentname} for use in name formatting directives.

\cmditem{printtext}[format]{text}
该命令用于打印\prm{text},可以是可打印的文本或者产生可打印文本的任意代码。它清除插入\prm{text}之前的标点缓存并且通知\biblatex 打印文本已经插入。这保证了所有之前和之后的\cmd{newblock}和\cmd{newunit}命令能产生预期的作用。\cmd{printfield}、\cmd{printnames} 、\cmd{bibstring}及其相关命令都这般自动处理(见\secref{aut:str})。如果一个参考文献样式需要插入抄录文本(包括来自\secref{aut:pct:pct, aut:pct:spc}的命令),需要使用该命令来确保block 和unit标点在\secref{aut:pct:new}节中所述正常功能。可选参数\prm{format}指定一个域格式指令用于格式化\prm{text}。当需要把若干个域打印为某一格式的集合块,这就会很有用,比如把集合块用括号或引号包起来。如果声明了type"=specific的格式指令,其将优先于设置的通用格式指令。如果\prm{format}缺省,那么\prm{text}如实输出(原样打印)。更多实用细节见第\secref{aut:cav:pct}节。
%This command prints \prm{text}, which may be printable text or arbitrary code generating printable text. It clears the punctuation buffer before inserting \prm{text} and informs \biblatex that printable text has been inserted. This ensures that all preceding and following \cmd{newblock} and \cmd{newunit} commands have the desired effect. \cmd{printfield} and \cmd{printnames} as well as \cmd{bibstring} and its companion commands (see \secref{aut:str}) do that automatically. Using this command is required if a bibliography styles inserts literal text (including the commands from \secref{aut:pct:pct, aut:pct:spc}) to ensure that block and unit punctuation works as advertised in \secref{aut:pct:new}. The optional \prm{format} argument specifies a field formatting directive to be used to format \prm{text}. This may also be useful when several fields are to be printed as one chunk, for example, by enclosing the entire chunk in parentheses or quotation marks. If a type"=specific \prm{format} has been declared, the type"=specific formatting directive takes precedence over the generic one. If the \prm{format} is omitted, the \prm{text} is printed as is. See also \secref{aut:cav:pct} for some practical hints.

\cmditem{printfile}[format]{file}
该命令类似于\cmd{printtext},差别在于第二个参数是一个文件名而不是抄录文本。\prm{file}参数必须是一个能在\tex 搜索路径找到的有效的\latex 文件。\cmd{printfile}将使用\cmd{input}来加载该\prm{file}。如果指定文件不存在,\cmd{printfile}不做任何操作。可选的\prm{format}参数指定了一个域格式化指令应用于该\prm{file}。如果声明了type"=specific的格式指令,其将优先于设置的通用格式指令。如果\prm{format}缺省,那么\prm{file}如实输出(原样打印)。注意该功能需要显式的打开\secref{use:opt:pre:gen}节的包选项\opt{loadfiles}。默认情况下,\cmd{printfile}不加载任何文件。
%This command is similar to \cmd{printtext} except that the second argument is a file name rather than literal text. The \prm{file} argument must be the name of a valid \latex file found in \tex's search path. \cmd{printfile} will use \cmd{input} to load this \prm{file}. If there is no such file, \cmd{printfile} does nothing. The optional \prm{format} argument specifies a field formatting directive to be applied to the \prm{file}. If a type"=specific \prm{format} has been declared, the type"=specific formatting directive takes precedence over the generic one. If the \prm{format} is omitted, the \prm{file} is printed as is. Note that this feature needs to be enabled explicitly by setting the package option \opt{loadfiles} from \secref{use:opt:pre:gen}. By default, \cmd{printfile} will not input any files.

\csitem{printdate}

该命令打印条目定义在\bibfield{date}或\bibfield{month}\slash\bibfield{year}域中的日期。日期格式由\secref{use:opt:pre:gen}节中的\opt{date}包选项控制。另外也可以通过调整域格式\texttt{date} (见\secref{aut:fmt:ich})来进一步格式化(比如设置字体等)。注意:该命令与标点追踪器自动交互,不必使用\cmd{printtext}命令将其包围起来。
%This command prints the date of the entry, as specified in the fields \bibfield{date} or \bibfield{month}\slash \bibfield{year}. The date format is controlled by the package option \opt{date} from \secref{use:opt:pre:gen}. Additional formatting (fonts etc.) may be applied by adjusting the field format \texttt{date} (\secref{aut:fmt:ich}). Note that this command interfaces with the punctuation tracker. There is no need to wrap it in a \cmd{printtext} command.

\csitem{printdateextra}

类似于\cmd{printdate},但指定的日期域是\bibfield{extrayear}域。用于设计作者年制的参考文献样式。
%Similar to \cmd{printdate} but incorporates the \bibfield{extrayear} field in the date specification. This is useful for bibliography styles designed for author-year citations.

\csitem{printlabeldate}

类似于\cmd{printdate},但打印的是日期域由\cmd{DeclareLabeldate}决定。日期格式由\secref{use:opt:pre:gen}节中的\opt{labeldate}包选项控制。另外也可以通过调整域格式\texttt{labeldate} (见\secref{aut:fmt:ich})来进一步格式化。
%Similar to \cmd{printdate} but prints the date field determined by \cmd{DeclareLabeldate}. The date format is controlled by the package option \opt{labeldate} from \secref{use:opt:pre:gen}. Additional formatting may be applied by adjusting the field format \texttt{labeldate} (\secref{aut:fmt:ich}).

\csitem{printlabeldateextra}

类似于\cmd{printlabeldate},但指定的日期域是\bibfield{extrayear}域,用于设计作者年制的参考文献样式。
%Similar to \cmd{printlabeldate} but incorporates the \bibfield{extrayear} field in the date specification. This is useful for bibliography styles designed for author-year citations.

\csitem{print$<$datetype$>$date}

类似于\cmd{printdate},但打印的是日期域是条目的由\bibfield{$<$datetype$>$date}域。日期格式由\secref{use:opt:pre:gen}节中的\opt{$<$datetype$>$date}包选项控制。另外也可以通过调整域格式\texttt{$<$datetype$>$date} (见\secref{aut:fmt:ich})来进一步格式化。$<$datetype$>$在默认的数据模型中有:<> (用于\bibfield{date}域), <orig>, <event> 和<url>。
%As \cmd{printdate} but prints the \bibfield{$<$datetype$>$date} of the entry. The date format is controlled by the package option \opt{$<$datetype$>$date} from \secref{use:opt:pre:gen}. Additional formatting may be applied by adjusting the field format \texttt{$<$datetype$>$date} (\secref{aut:fmt:ich}). The $<$datetype$>$s in the default data model are <> (for the main \bibfield{date} field), <orig>, <event> and <url>.

\csitem{printtime}

该命令打印条目定义在\bibfield{date}域 (见\secref{bib:use:dat})中的时间范围,时间格式由\secref{use:opt:pre:gen}节中的\opt{time}包选项控制。另外也可以通过调整域格式\texttt{time} (见\secref{aut:fmt:ich})来进一步格式化(比如设置字体等)。时间格式化相关内容还包括\opt{timezeros}选项,\cmd{bibtimesep}和\cmd{bibtimezonesep}宏(\secref{use:fmt:lng})。注意:该命令与标点追踪器自动交互,不必使用\cmd{printtext}命令将其包围起来。注意该命令打印的是独立于日期元素的时间范围。当\opt{$<$datepart$>$dateusetime}选项打开时,也可以与日期范围一起打印,而不是各自分别打印。
%This command prints the time range of the entry, as specified in the \bibfield{date} field (see \secref{bib:use:dat}). The time format is controlled by the package option \opt{time} from \secref{use:opt:pre:gen}. Additional formatting (fonts etc.) may be applied by adjusting the field format \texttt{time} (\secref{aut:fmt:ich}). Relevant to time formatting are the \opt{timezeros} option and the \cmd{bibtimesep} and \cmd{bibtimezonesep} macros (\secref{use:fmt:lng}). Note that this command interfaces with the punctuation tracker. There is no need to wrap it in a \cmd{printtext} command. Note that this command prints a stand-alone time range apart from the date elements. With the \opt{$<$datepart$>$dateusetime} option, you can have the printed along with a date when printing a date range instead of printing the time range completely separately, which is what this command allows for.

\csitem{print$<$datetype$>$time}

类似于\cmd{printtime},但打印的条目的\bibfield{$<$datetype$>$time}域。时间格式由\secref{use:opt:pre:gen}节中的\opt{$<$datetype$>$time}包选项控制。另外也可以通过调整域格式\texttt{$<$datetype$>$time}  (见\secref{aut:fmt:ich})来进一步格式化。$<$datetype$>$在默认的数据模型中有:<> (用于\bibfield{date}域), <orig>, <event> 和<url>。
%As \cmd{printtime} but prints the \bibfield{$<$datetype$>$time} of the entry. The time format is controlled by the package option \opt{$<$datetype$>$time} from \secref{use:opt:pre:gen}. Additional formatting may be applied by adjusting the field format \texttt{$<$datetype$>$time} (\secref{aut:fmt:ich}). The $<$datetype$>$s in the default data model are <> (for the main \bibfield{date} field), <orig>, <event> and <url>.

\cmditem{indexfield}[format]{field}

该命令类似于\cmd{printfield},差别在于不是打印\prm{field}而是将其添加到索引中,其格式化指令\prm{format}由\cmd{DeclareIndexFieldFormat}命令定义。如果声明了type"=specific的格式指令,其将优先于设置的通用格式指令。如果\prm{field}域未定义,该命令不做任何操作。如果\prm{format}缺省,那么\cmd{indexfield}将采用与域名相同的格式名。这种情况下任何type"=specific的格式指令都将优先于通用的格式指令。若所有的这些格式都未定义,那么将采用\texttt{default}格式作为最后的选择。
%This command is similar to \cmd{printfield} except that the \prm{field} is not printed but added to the index using the formatting directive \prm{format}, as defined with \cmd{DeclareIndexFieldFormat}. If a type"=specific \prm{format} has been declared, it takes precedence over the generic one. If the \prm{field} is undefined, this command does nothing. If the \prm{format} is omitted, \cmd{indexfield} tries using the name of the field as a format name. In this case, any type"=specific formatting directive will also take precedence over the generic one. If all of these formats are undefined, it falls back to \texttt{default} as a last resort.

\cmditem{indexlist}[format][start\ensuremath\rangle--\ensuremath\langle stop]{literal list}

该命令类似于\cmd{printlist},差别在于不是打印列表的项而是将其添加到索引中,其格式化指令\prm{format}由\cmd{DeclareIndexListFormat}命令定义。如果声明了type"=specific的格式指令,其将优先于设置的通用格式指令。如果\prm{literal list}未定义,该命令不做任何操作。如果\prm{format}缺省,那么\cmd{indexlist}将采用与列表名相同的格式名。这种情况下任何type"=specific的格式指令都将优先于通用的格式指令。若所有的这些格式都未定义,那么将采用\texttt{default}格式作为最后的选择。
%This command is similar to \cmd{printlist} except that the items in the list are not printed but added to the index using the formatting directive \prm{format}, as defined with \cmd{DeclareIndexListFormat}. If a type"=specific \prm{format} has been declared, the type"=specific formatting directive takes precedence over the generic one. If the \prm{literal list} is undefined, this command does nothing. If the \prm{format} is omitted, \cmd{indexlist} tries using the name of the list as a format name. In this case, any type"=specific formatting directive will also take precedence over the generic one. If all of these formats are undefined, it falls back to \texttt{default} as a last resort.

\cmditem{indexnames}[format][start\ensuremath\rangle--\ensuremath\langle stop]{name list}

该命令类似于\cmd{printnames},差别在于不是打印姓名列表的项而是将其添加到索引中,其格式化指令\prm{format}由\cmd{DeclareIndexNameFormat}命令定义。如果声明了type"=specific的格式指令,其将优先于设置的通用格式指令。如果\prm{name list}未定义,该命令不做任何操作。如果\prm{format}缺省,那么\cmd{indexnames}将采用与列表名相同的格式名。这种情况下任何type"=specific的格式指令都将优先于通用的格式指令。若所有的这些格式都未定义,那么将采用\texttt{default}格式作为最后的选择。
%This command is similar to \cmd{printnames} except that the items in the list are not printed but added to the index using the formatting directive \prm{format}, as defined with \cmd{DeclareIndexNameFormat}. If a type"=specific \prm{format} has been declared, the type"=specific formatting directive takes precedence over the generic one. If the \prm{name list} is undefined, this command does nothing. If the \prm{format} is omitted, \cmd{indexnames} tries using the name of the list as a format name. In this case, any type"=specific formatting directive will also take precedence over the generic one. If all of these formats are undefined, it falls back to \texttt{default} as a last resort.

\cmditem{entrydata}{key}{code}
\cmditem*{entrydata*}{key}{code}

像\cmd{printfield}之类的数据命令正常情况下应用当前正在处理的条目的数据。可以使用\cmd{entrydata}在局部环境中转换应用数据。\prm{key}是要局部使用条目的关键词。\prm{code}是在当前局部环境执行的任意代码。这一地面将在一个编组中执行。举例见\secref{aut:cav:mif}节。注意该命令自动转换语言,如果\opt{autolang}包选项打开的话。带星号的命令\cmd{entrydata*}将复制封装条目(the enclosing entry)的所有域,使用域,计数器,和其它以字符串<\texttt{saved}>为前缀命名的资源。这在比较两个数据集是很有用。例如,在\prm{code}的参数中,\bibfield{author}域保存了条目\prm{key}的作者,而封装条目的作者保存在\bibfield{savedauthor}域中。\cnt{author}计数器保存了\prm{key}条目的\bibfield{author}域的姓名数量,而封装条目的作者数量由\bibfield{savedauthor}计数器计数。
%Data commands like \cmd{printfield} normally use the data of the entry currently being processed. You may use \cmd{entrydata} to switch contexts locally. The \prm{key} is the entry key of the entry to use locally. The \prm{code} is arbitrary code to be executed in this context. This code will be executed in a group. See \secref{aut:cav:mif} for an example. Note that this command will automatically switch languages if the \opt{autolang} package option is enabled. The starred version \cmd{entrydata*} will clone all fields of the enclosing entry, using field, counter, and other resource names prefixed with the string <\texttt{saved}>. This is useful when comparing two data sets. For example, inside the \prm{code} argument, the \bibfield{author} field holds the author of entry \prm{key} and the author of the enclosing entry is available as \bibfield{savedauthor}. The \cnt{author} counter holds the number of names in the \bibfield{author} field of \prm{key}; the \bibfield{savedauthor} counter refers to the author count of the enclosing entry.

\cmditem{entryset}{precode}{postcode}

该命令用在处理\bibtype{set}条目集的参考文献驱动中。它将对由\bibfield{entryset}域指出的集的所有成员进行循环处理,对集的各个成员执行相应的驱动。这相当于对每个集成员执行\secref{aut:aux:msc}节的\cmd{usedriver}命令。\prm{precode}是在集的每项处理之前执行的任意代码。\prm{postcode}是在集的每项处理之后执行的任意代码。两个参数语法上必须的,但可以留空。用法举例见\secref{aut:cav:set}节。
%This command is intended for use in bibliography drivers handling \bibtype{set} entries. It will loop over all members of the set, as indicated by the \bibfield{entryset} field, and execute the appropriate driver for the respective set member. This is similar to executing the \cmd{usedriver} command from \secref{aut:aux:msc} for each set member. The \prm{precode} is arbitrary code to be executed prior to processing each item in the set. The \prm{postcode} is arbitrary code to be executed immediately after processing each item. Both arguments are mandatory in terms of the syntax but may be left empty. See \secref{aut:cav:set} for usage examples.

\cmditem{DeclareFieldInputHandler}{field}{code}

给命令用于定义从\file{.bbl}读取数据所采用的域的数据输入处理器。在\prm{code}内,宏\cmd{NewValue}包含了域的值。比如,要忽略出现的\bibfield{volumes}域,可以作:
%This command can be used to define a data input handler for \prm{field} when it is read from the \file{.bbl}. Within the \prm{code}, the macro \cmd{NewValue} contains the value of the field. For example, to ignore the \bibfield{volumes} field when it appears, you could do

\begin{ltxexample}
\DeclareFieldInputHandler{volumes}{\def\NewValue{}}
\end{ltxexample}
%
一般情况下,要删除和修改域需要使用\cmd{DeclareSourcemap}(见\secref{aut:ctm:map}节),而这一替代方法在一些情形下会很有用,例如当强调的是数据的外观而不是数据本身时,因为\prm{code}可以是任意的\tex 代码。
%Generally, you would want to use \cmd{DeclareSourcemap} (see \secref{aut:ctm:map}) to remove and modify fields but this alternative method may be useful in some circumstances when the emphasis is on appearance rather than data since the \prm{code} can be arbitraty \tex.

\cmditem{DeclareListInputHandler}{list}{code}

类似于\cmd{DeclareFieldInputHandler},但用于列表。在\prm{code}内,宏\cmd{NewValue}包含了列表的值,而\cmd{NewCount}保存列表中项的序号。
%As \cmd{DeclareFieldInputHandler} but for lists. Within the \prm{code}, the macro \cmd{NewValue} contains the value of the list and \cmd{NewCount} contains the number of items in the list.

\cmditem{DeclareNameInputHandler}{name}{code}

类似于\cmd{DeclareFieldInputHandler},但用于姓名列表。在\prm{code}内,宏\cmd{NewValue}包含了姓名列表的值,而\cmd{NewCount}保存列表中姓名的序号,\cmd{NewOption}保存了\file{.bbl}文件传递的任意姓名所属的选项。
%As \cmd{DeclareFieldInputHandler} but for names. Within the \prm{code}, the macro \cmd{NewValue} contains the value of the name, \cmd{NewCount} contains the number of individual names in the name and \cmd{NewOption} contains any per-name options passed in the \file{.bbl}.

\end{ltxsyntax}

\subsubsection{格式化指令}%Formatting Directives
\label{aut:bib:fmt}

本节介绍用于定义\secref{aut:bib:dat}节的数据命令所需的格式化指令的命令。注意:所有标注的格式定义在\path{biblatex_.def}文件中。
%This section introduces the commands used to define the formatting directives required by the data commands from \secref{aut:bib:dat}. Note that all standard formats are defined in \path{biblatex_.def}.

\begin{ltxsyntax}

\cmditem{DeclareFieldFormat}[entrytype, \dots]{format}{code}
\cmditem*{DeclareFieldFormat}*{format}{code}

定义域格式\prm{format}。该格式化指令是由\cmd{printfield}命令执行的任意\prm{code}。域的值作为仅有的第一个参数传递给\prm{code}。正在处理的域名在\prm{code}中以\cmd{currentfield}表示。如果指定一种条目类型(\prm{entrytype}),那么格式是该类型专属的。\prm{entrytype}可以是一个逗号分隔(comma"=separated)的值列表。带星的命令类似于不带星的命令,区别在于它还将清除所有对条目类型做的格式定义。
%Defines the field format \prm{format}. This formatting directive is arbitrary \prm{code} to be executed by \cmd{printfield}. The value of the field will be passed to the \prm{code} as its first and only argument. The name of the field currently being processed is available to the \prm{code} as \cmd{currentfield}. If an \prm{entrytype} is specified, the format is specific to that type. The \prm{entrytype} argument may be a comma"=separated list of values. The starred variant of this command is similar to the regular version, except that all type-specific formats are cleared.

\cmditem{DeclareListFormat}[entrytype, \dots]{format}{code}
\cmditem*{DeclareListFormat}*{format}{code}

定义文本列表\footnote{literal 译为文本}的格式\prm{format}。格式化指令是\cmd{printlist}命令处理列表中每一项时执行的任意\prm{code}。当前正在处理的项作为第一个和唯一的参数传递给\prm{code}。正在处理的文本列表名在\prm{code}中以\cmd{currentlist}表示。如果指定了\prm{entrytype},那么格式是该类型专属的。\prm{entrytype}参数可以是一个逗号分隔(comma"=separated)的值列表。注意格式化指令也会处理在列表各项间插入的标点。需要对当前项是在列表中间或者末尾进行检测,即检查\cnt{listcount}是否小于或等于\cnt{liststop}。带星的命令类似于不带星的命令,区别在于它还将清除所有对条目类型做的格式定义。
%Defines the literal list format \prm{format}. This formatting directive is arbitrary \prm{code} to be executed for every item in a list processed by \cmd{printlist}. The current item will be passed to the \prm{code} as its first and only argument. The name of the literal list currently being processed is available to the \prm{code} as \cmd{currentlist}. If an \prm{entrytype} is specified, the format is specific to that type. The \prm{entrytype} argument may be a comma"=separated list of values. Note that the formatting directive also handles the punctuation to be inserted between the individual items in the list. You need to check whether you are in the middle of or at the end of the list, \ie whether \cnt{listcount} is smaller than or equal to \cnt{liststop}. The starred variant of this command is similar to the regular version, except that all type-specific formats are cleared.

\cmditem{DeclareNameFormat}[entrytype, \dots]{format}{code}
\cmditem*{DeclareNameFormat}*{format}{code}

定义姓名列表的格式\prm{format}。格式化指令是\cmd{printnames}命令处理列表中每一项时执行的任意\prm{code}。如果指定了\prm{entrytype},那么格式是该类型专属的。\prm{entrytype}参数可以是一个逗号分隔(comma"=separated)的值列表。单个姓名的各个组成部分由自动创建的宏表示(见下)。默认的数据模型定义了四个组成部分对应于标准的\bibtex 姓名成分参数。
%Defines the name list format \prm{format}. This formatting directive is arbitrary \prm{code} to be executed for every name in a list processed by \cmd{printnames}. If an \prm{entrytype} is specified, the format is specific to that type. The \prm{entrytype} argument may be a comma"=separated list of values. The individual parts of a name will be available in automatically created macros (see below). The default data mode defines four name part which correspond to the standard \bibtex name parts arguments:

\begin{argumentlist}{00}
\item[family] 姓,\bibtex 中为<last> name部分。当一个姓名只有一个部分组成时(比如 <Aristotle>),这一部分将被处理为姓。
%The family name(s), know as <last> in \bibtex.  If a name consists of a single part only (for example, <Aristotle>), this part will be treated as the family name.
\item[given] 名。注意名在\bibtex 中为<first> name部分。
%The given name(s). Note that given names are referred to as the <first> names in the \bibtex file format documentation.
\item[prefix] 前缀,比如von, van, of, da, de, del, della等。注意前缀在\bibtex 格式文件中为<von>部分。
%Any name prefices, for example von, van, of, da, de, del, della, etc. Note that name prefices are referred to as the <von> part of the name in the \bibtex file format documentation.
\item[suffix] 后缀,比如Jr, Sr等。注意后缀在\bibtex 格式文件中为<Jr>部分。
%Any name suffices, for example Jr, Sr. Note that name suffices are referred to as the <Jr> part of the name in the \bibtex file format documentation.
\end{argumentlist}
%
数据模型<nameparts>常量的值(见\secref{aut:bbx:drv})在姓名的数据模型中为每个姓名组成部分创建了两个宏。比如,在默认的数据模型中,姓名格式由如下宏定义:
%The value of the datamodel <nameparts> constant (see \secref{aut:bbx:drv}) creates two macros for each name part in the datamodel for the name. So, for example, in the default data model, name formats will have defined the following macros:

\begin{ltxexample}
\namepartprefix %表示前缀部分
\namepartprefixi %表示前缀首字母
\namepartfamily %表示姓
\namepartfamilyi %表示姓首字母
\namepartsuffix %表示后缀
\namepartsuffixi %表示后缀首字母
\namepartgiven %表示名
\namepartgiveni %表示名首字母
\end{ltxexample}
%
如果一个姓名的某些部分没有给出,相应的宏将为空,因此可以使用,比如\sty{etoolbox}中\cmd{ifdefvoid}这类的判断来检查姓名的各个组成部分。正在处理的姓名列表名在\prm{code}中以\cmd{currentname}表示。注意格式化指令也会处理在列表各项间插入的标点。需要对当前项是在列表中间或者末尾进行检测,即检查\cnt{listcount}是否小于或等于\cnt{liststop}(见\secref{use:cav:nam}节)。带星的命令类似于不带星的命令,区别在于它还将清除所有对条目类型做的格式定义。
%If a certain part of a name is not available, the corresponding macro will be empty, hence you may use, for example, the \sty{etoolbox} tests like \cmd{ifdefvoid} to check for the individual parts of a name. The name of the name list currently being processed is available to the \prm{code} as \cmd{currentname}. Note that the formatting directive also handles the punctuation to be inserted between separate names and between the individual parts of a name. You need to check whether you are in the middle of or at the end of the list, \ie whether \cnt{listcount} is smaller than or equal to \cnt{liststop}. See also \secref{use:cav:nam}. The starred variant of this command is similar to the regular version, except that all type-specific formats are cleared.

\cmditem{DeclareIndexFieldFormat}[entrytype, \dots]{format}{code}
\cmditem*{DeclareIndexFieldFormat}*{format}{code}

定义域格式\prm{format}。该格式化指令是由\cmd{indexfield}命令执行的任意\prm{code}。域的值作为仅有的第一个参数传递给\prm{code}。正在处理的域名在\prm{code}中以\cmd{currentfield}表示。如果指定一种条目类型(\prm{entrytype}),那么格式是该类型专属的。\prm{entrytype}可以是一个逗号分隔(comma"=separated)的值列表。该命令类似于\cmd{DeclareFieldFormat},差别在于\prm{code}处理的数据不是用于打印而是用于索引。注意\cmd{indexfield}将执行\prm{code}本身,即\prm{code}必须包含\cmd{index}或类似命令。带星的命令类似于不带星的命令,区别在于它还将清除所有对条目类型做的格式定义。
%Defines the field format \prm{format}. This formatting directive is arbitrary \prm{code} to be executed by \cmd{indexfield}. The value of the field will be passed to the \prm{code} as its first and only argument. The name of the field currently being processed is available to the \prm{code} as \cmd{currentfield}. If an \prm{entrytype} is specified, the format is specific to that type. The \prm{entrytype} argument may be a comma"=separated list of values. This command is similar to \cmd{DeclareFieldFormat} except that the data handled by the \prm{code} is not intended to be printed but written to the index. Note that \cmd{indexfield} will execute the \prm{code} as is, \ie the \prm{code} must include \cmd{index} or a similar command. The starred variant of this command is similar to the regular version, except that all type-specific formats are cleared.

\cmditem{DeclareIndexListFormat}[entrytype, \dots]{format}{code}
\cmditem*{DeclareIndexListFormat}*{format}{code}

定义文本列表格式\prm{format}。该格式化指令是由\cmd{indexlist}命令执行的任意\prm{code}。列表中当前值作为唯一参数传递给\prm{code}。正在处理的列表名在\prm{code}中以\cmd{currentlist}表示。如果指定一种条目类型(\prm{entrytype}),那么格式是该类型专属的。\prm{entrytype}可以是一个逗号分隔(comma"=separated)的值列表。该命令类似于\cmd{DeclareListFormat},差别在于\prm{code}处理的数据不是用于打印而是用于索引。注意\cmd{indexlist}将执行\prm{code}本身,即\prm{code}必须包含\cmd{index}或类似命令。带星的命令类似于不带星的命令,区别在于它还将清除所有对条目类型做的格式定义。
%Defines the literal list format \prm{format}. This formatting directive is arbitrary \prm{code} to be executed for every item in a list processed by \cmd{indexlist}. The current item will be passed to the \prm{code} as its only argument. The name of the literal list currently being processed is available to the \prm{code} as \cmd{currentlist}. If an \prm{entrytype} is specified, the format is specific to that type. The \prm{entrytype} argument may be a comma"=separated list of values. This command is similar to \cmd{DeclareListFormat} except that the data handled by the \prm{code} is not intended to be printed but written to the index. Note that \cmd{indexlist} will execute the \prm{code} as is, \ie the \prm{code} must include \cmd{index} or a similar command. The starred variant of this command is similar to the regular version, except that all type-specific formats are cleared.

\cmditem{DeclareIndexNameFormat}[entrytype, \dots]{format}{code}
\cmditem*{DeclareIndexNameFormat}*{format}{code}

定义姓名列表格式\prm{format}。该格式化指令是由\cmd{indexnames}命令执行的任意\prm{code}。列表中当前值作为唯一参数传递给\prm{code}。正在处理的列表名在\prm{code}中以\cmd{currentname}表示。如果指定一种条目类型(\prm{entrytype}),那么格式是该类型专属的。\prm{entrytype}可以是一个逗号分隔(comma"=separated)的值列表。该命令类似于\cmd{DeclareNameFormat},差别在于\prm{code}处理的数据不是用于打印而是用于索引。注意\cmd{indexnames}将执行\prm{code}本身,即\prm{code}必须包含\cmd{index}或类似命令。带星的命令类似于不带星的命令,区别在于它还将清除所有对条目类型做的格式定义。
%Defines the name list format \prm{format}. This formatting directive is arbitrary \prm{code} to be executed for every name in a list processed by \cmd{indexnames}. The name of the name list currently being processed is available to the \prm{code} as \cmd{currentname}. If an \prm{entrytype} is specified, the format is specific to that type. The \prm{entrytype} argument may be a comma"=separated list of values. The parts of the name will be passed to the \prm{code} as separate arguments. This command is very similar to \cmd{DeclareNameFormat} except that the data handled by the \prm{code} is not intended to be printed but written to the index. Note that \cmd{indexnames} will execute the \prm{code} as is, \ie the \prm{code} must include \cmd{index} or a similar command. The starred variant of this command is similar to the regular version, except that all type-specific formats are cleared.

\cmditem{DeclareFieldAlias}[entry type]{alias}[format entry type]{format}

声明\prm{alias}作为域格式\prm{format}的别名。如果指定一种条目类型(\prm{entrytype}),别名是该类型专属的。\prm{format entry type}是后端格式的条目类型。这仅在声明某一具体条目类型的格式化指令的别名时需要。
%Declares \prm{alias} to be an alias for the field format \prm{format}. If an \prm{entrytype} is specified, the alias is specific to that type. The \prm{format entry type} is the entry type of the backend format. This is only required when declaring an alias for a type"=specific formatting directive.

\cmditem{DeclareListAlias}[entry type]{alias}[format entry type]{format}

声明\prm{alias}作为文本列表格式\prm{format}的别名。如果指定一种条目类型(\prm{entrytype}),别名是该类型专属的。\prm{format entry type}是后端格式的条目类型。这仅在声明某一具体条目类型的格式化指令的别名时需要。
%Declares \prm{alias} to be an alias for the literal list format \prm{format}. If an \prm{entrytype} is specified, the alias is specific to that type. The \prm{format entry type} is the entry type of the backend format. This is only required when declaring an alias for a type"=specific formatting directive.

\cmditem{DeclareNameAlias}[entry type]{alias}[format entry type]{format}

声明\prm{alias}作为姓名列表格式\prm{format}的别名。如果指定一种条目类型(\prm{entrytype}),别名是该类型专属的。\prm{format entry type}是后端格式的条目类型。这仅在声明某一具体条目类型的格式化指令的别名时需要。
%Declares \prm{alias} to be an alias for the name list format \prm{format}. If an \prm{entrytype} is specified, the alias is specific to that type. The \prm{format entry type} is the entry type of the backend format. This is only required when declaring an alias for a type"=specific formatting directive.

\cmditem{DeclareIndexFieldAlias}[entry type]{alias}[format entry type]{format}

声明\prm{alias}作为域格式\prm{format}的别名。如果指定一种条目类型(\prm{entrytype}),别名是该类型专属的。\prm{format entry type}是后端格式的条目类型。这仅在声明某一具体条目类型的格式化指令的别名时需要。
%Declares \prm{alias} to be an alias for the field format \prm{format}. If an \prm{entrytype} is specified, the alias is specific to that type. The \prm{format entry type} is the entry type of the backend format. This is only required when declaring an alias for a type"=specific formatting directive.

\cmditem{DeclareIndexListAlias}[entry type]{alias}[format entry type]{format}

声明\prm{alias}作为文本列表格式\prm{format}的别名。如果指定一种条目类型(\prm{entrytype}),别名是该类型专属的。\prm{format entry type}是后端格式的条目类型。这仅在声明某一具体条目类型的格式化指令的别名时需要。
%Declares \prm{alias} to be an alias for the literal list format \prm{format}. If an \prm{entrytype} is specified, the alias is specific to that type. The \prm{format entry type} is the entry type of the backend format. This is only required when declaring an alias for a type"=specific formatting directive.

\cmditem{DeclareIndexNameAlias}[entry type]{alias}[format entry type]{format}

声明\prm{alias}作为姓名列表格式\prm{format}的别名。如果指定一种条目类型(\prm{entrytype}),别名是该类型专属的。\prm{format entry type}是后端格式的条目类型。这仅在声明某一具体条目类型的格式化指令的别名时需要。
%Declares \prm{alias} to be an alias for the name list format \prm{format}. If an \prm{entrytype} is specified, the alias is specific to that type. The \prm{format entry type} is the entry type of the backend format. This is only required when declaring an alias for a type"=specific formatting directive.

\end{ltxsyntax}

\subsection{定制}%Customization
\label{aut:ctm}

\subsubsection{关联条目}%Related Entries
\label{aut:ctm:rel}
%The related entries feature comprises the following components:
关联条目功能由如下部分构成:\footnote{这里related data用关联而不是相关,其它地方用到相关的应一并改过来 }
\begin{itemize}
\item 条目中的特殊域用于建立和描述关系
%\item Special fields in an entry to set up and describe relationships
\item 本地化字符串作为关联数据的前缀(可选)
%\item Optionally, localisation strings to prefix the related data
\item 抽取和打印关联数据的宏
%\item Macros to extract and print the related data
\item 用于本地化字符串和关联数据格式化的格式
%\item Formats to format the localisation string and related data
\end{itemize}
%
特殊域是\bibfield{related}, \bibfield{relatedtype}, \bibfield{relatedstring}和 \bibfield{relatedoptions}:
%The special fields are \bibfield{related}, \bibfield{relatedtype}, \bibfield{relatedstring} and \bibfield{relatedoptions}:
\begin{keymarglist}
\item[related] 与当前条目某种程度关联的条目的条目关键词列表\footnote{这里separated list用分离列表有没有更好的说法}。注意:条目关键词\footnote{这里的key关键词应该是引用关键词,即bibtex键}的顺序很重要。来自多个关联条目的数据是按该域中关键词的顺序打印。
%\item[related] A separated list of keys of entries which are related to this entry in some way. Note the the order of the keys is important. The data from multiple related entries is printed in the order of the keys listed in this field.
\item[relatedtype] The type of relationship. This serves three purposes. If the value of this field resolves to a localisation string identifier, then the resulting localised string is printed before the data from the related entries. Secondly, if there is a macro called \texttt{related:\prm{relatedtype}}, this is used to format the data from the related entries. If no such macro exists, then the macro \texttt{related:default} is used. Lastly, if there is a format named \texttt{related:\prm{relatedtype}}, then it is used to format both the localised string and related entry data. If there is no related type specific format, the \texttt{related} format is used.
\item[relatedstring] If an entry contains this field, then if value of the field resolves to a localisation string identifier, the localisation key value specified is printed before data from the related entries. If the field does not specify a localisation key, its value is printed literally. If both \bibfield{relatedtype} and \bibfield{relatedstring} are present in an entry, \bibfield{relatedstring} is used for the pre-data string (but \bibfield{relatedtype} is still used to determine the macro and format to use when printing the data).
\item[relatedoptions] A list of per"=entry options to set on the related entry (actually on the clone of the related entry which is used as a data source---the actual related entry is not modified because it might be cited directly itself).
\end{keymarglist}

The related entry feature is enabled by default by the package option \opt{related} from \secref{use:opt:pre:gen}. The related information entry data from the related entries is included via a \cmd{usebibmacro\{related\}} call. Standard styles call this macro towards the end of each driver. Style authors should ensure the existence of (or take note of existing) localisation strings which are useful as values for the \bibfield{relatedtype} field, such as \texttt{translationof} or perhaps \texttt{translatedas}. A plural variant can be identified with the localisation key \prm{relatedtype}\texttt{s}. This key's corresponding string is printed whenever more than one entry is specified in \bibfield{related}. Bibliography macros and formatting directives for printing entries related by \prm{relatedtype} should be defined using the name \texttt{related:\prm{relatedtype}}. The file \path{biblatex.def} contains macros and formats for some common relation types which can be used as templates. In particular, the \cmd{entrydata*} command is essential in such macros in order to make the data of the related entries available. Examples of entries using this feature can be found in the \biblatex distribution examples file \path{biblatex-examples.bib}. There are some specific formatting macros for this feature which control delimiters and separators in related entry information, see \secref{aut:fmt:fmt}.

\subsubsection{数据源集}%Datasource Sets
\label{aut:ctm:dsets}

能够定义在循环等中的使用的数据源(datasource)域名的集是有用的。而且,\biber 可以利用这种集来应用选项或者对某些数据源域的特定集进行操作。下面的宏允许用来定义数据源域的任意集合,这些数据源域以\sty{etoolbox}列表的形式提供给\biblatex 并通过\file{.bcf}文件中提供给\biber 。
%It is useful to be able to define named sets of datasource field names for use in loops etc. In addition, \biber can use such sets in order to apply options and perform operations on particular sets of datasource fields. The following macros allow the user to define arbitrary sets of datasource fields, exposed to \biblatex as \sty{etoolbox} lists and to \biber in the \file{.bcf}.


\begin{ltxsyntax}

\cmditem{DeclareDatafieldSet}{name}{specification}

声明一个数据源域的集,其名为\prm{name}。
%Declare a set of datasource fields with name \prm{name}.

\begin{optionlist*}

\valitem{name}{set name}

集的名。
%The name of the set.

\end{optionlist*}

\prm{specification}是1个或更多的\cmd{member}(成员)项:
%The \prm{specification} is one or more \cmd{member} items:

\cmditem{member}

\begin{optionlist*}

\valitem{fieldtype}{fieldtype}
\valitem{datatype}{datatype}
\valitem{field}{fieldname}

\end{optionlist*}

一个\cmd{member}说明将域添加到集中。域可以由数据模型\prm{fieldtype} 和/或 \prm{datatype}指定 (见 \secref{aut:ctm:dm})。或者,域也可以通过使用\prm{field}选项显式的以名字添加。一旦完成定义,集就以\sty{etoolbox}列表的形式存在,命名为 \cmd{datafieldset<setname>}并通过\file{.bcf}文件传递给\biber。
%A \cmd{member} specification appends fields to the set. Fields can be specified by datamodel \prm{fieldtype} and/or \prm{datatype} (see \secref{aut:ctm:dm}). Alternatively, fields can be explicitly added by name using the \prm{field} option. Once defined, the set is available as an \sty{etoolbox} list called \cmd{datafieldset<setname>} and is also passed via the \file{.bcf} to \biber.

下面的例子就是\biblatex 为姓名域和标题域定义的默认集:
%For example, here are the default sets defined by \biblatex for name fields and title fields:

\end{ltxsyntax}

\begin{ltxexample}[style=latex]{}
\DeclareDatafieldSet{setnames}{
  \member[datatype=name, fieldtype=list]
}

\DeclareDatafieldSet{settitles}{
  \member[field=title]
  \member[field=booktitle]
  \member[field=eventtitle]
  \member[field=issuetitle]
  \member[field=journaltitle]
  \member[field=maintitle]
  \member[field=origtitle]
}
\end{ltxexample}
%
这将定义\cmd{datafieldsetsetnames}和\cmd{datafieldsetsettitles}宏以\sty{etoolbox}列表形式包含指定的成员数据源域的名。
%This defines the macros \cmd{datafieldsetsetnames} and \cmd{datafieldsetsettitles} as \sty{etoolbox} lists containing the names of the member datasource fields specified.

\subsubsection{数据动态修改}%Dynamic Modification of Data
\label{aut:ctm:map}

对自动生成或者你无法控制的参考文献数据源进行修改在某种程度上会是一个问题。因此,\biber 提供了对它所读取的数据进行修改的能力,这样你可以对源数据流进行修改二不必实际改变它。这种改变可以在\biber 的配置文件(见\biber 文档)定义,或者通过\biblatex 宏进行定义,通过宏定义的方法你可以在样式中或者以全局定义的方式,将修改应用在具体的文档中。
%Bibliographic data sources which are automatically generated or which you have no control over can be a problem if you need to edit them in some way. For this reason, \biber has the ability to modify data as it is read so that you can apply modifications to the source data stream without actually changing it. The modification can be defined in \biber's config file (see \biber docs), or via \biblatex macros in which case you can apply the modification only for specific documents, styles or globally.

源映射发生在数据解析过程中,因此也在诸如继承和排序等任何其它操作之前。
%Source mapping happens during data parsing and therefore before any other operation such as inheritance and sorting.

源映射可以在不同的层( «levels» )进行定义,各层以某一定义的顺序进行处理。见\biblatex\ 手册,考虑这些宏:\\[2ex]
%Source mappings can be defined at different «levels» which are applied
%in a defined order. See the \biblatex\ manual regarding these macros:\\[2ex]

\noindent \cmd{DeclareSourcemap}命令定义的\texttt{user}-层映射$\rightarrow$\\
\hspace*{1em}在\biber 配置文件定义的\texttt{user}-层映射(见 \biber 文档)$\rightarrow$\\
\hspace*{2em}\cmd{DeclareStyleSourcemap}定义的\texttt{style}-层映射$\rightarrow$\\
\hspace*{3em}\cmd{DeclareDriverSourcemap}定义的\texttt{driver}-层映射\\[2ex]
%\noindent \texttt{user}-level maps defined with \cmd{DeclareSourcemap}$\rightarrow$\\
%\hspace*{1em}\texttt{user}-level maps defined in the \biber config file (see \biber docs)$\rightarrow$\\
%\hspace*{2em}\texttt{style}-level maps defined with \cmd{DeclareStyleSourcemap}$\rightarrow$\\
%\hspace*{3em}\texttt{driver}-level maps defined with \cmd{DeclareDriverSourcemap}\\[2ex]

\begin{ltxsyntax}

\cmditem{DeclareSourcemap}{specification}

定义源数据修改(映射)规则,可以用于执行如下任务或其任意组合:
%Defines source data modification (mapping) rules which can be used to perform any combination of the following tasks:

\begin{itemize}
\item 将数据源条目类型映射为其它类型
%Map data source entrytypes to different entrytypes
\item 将数据源域映射为其它其它域
%Map datasource fields to different fields
\item 给条目添加新域
%Add new fields to an entry
\item 从条目移除域
%Remove fields from an entry
\item 用标准的Perl 正则表达式匹配和替换修改域的内容。
%Modify the contents of a field using standard Perl regular expression match and replace
\item 将上述操作限制在来自特定数据源的条目,这些特定数据源可以在\cmd{addresource}宏中定义。
%Restrict any of the above operations to entries coming from particular datasources which you defined in \cmd{addresource} macros
\item 将上述操作限制在某些条目类型。
%Restrict any of the above operations to entries only of a certain entrytype
\item 将上述操作限制在某一特定的参考文献节。
%Restrict any of the above operations to entries in a particular reference section
\end{itemize}

\prm{specification}是一个不限范围的\cmd{maps}指令列表,这些指令说明了应用于某一特定数据源类型的映射规则的容器(\secref{use:bib:res})。可以自由地使用空格、制表符和行末符号来整理\prm{specification}内容达到视觉效果。\footnote{visually arrange the \prm{specification}} 但空行是不允许的。这一命令仅能用于导言区并且只能使用一次---后面的命令将覆盖前面的定义。
%The \prm{specification} is an undelimited list of \cmd{maps} directives which specify containers for mappings rules applying to a particular data source type (\secref{use:bib:res}). Spaces, tabs, and line endings may be used freely to visually arrange the \prm{specification}. Blank lines are not permissible. This command may only be used in the preamble and may only be used once---subsequent uses will overwrite earlier definitions.

\cmditem{maps}[options]{elements}

包含\cmd{map}元素的有序集,每个\cmd{map}都是应用于数据源的映射步的逻辑相关集。\prm{options}包括:
%Contains an ordered set of \cmd{map} elements each of which is a logically related set of mapping steps to apply to the data source. The \prm{options} are:

\begin{optionlist*}

\choitem[bibtex]{datatype}{bibtex, biblatexml}

包含的\cmd{map}应用的数据源的类型(见\secref{use:bib:res})
%Data source type to which the contained \cmd{map} directives apply (\secref{use:bib:res}).

\boolitem[false]{overwrite}

具体说明一个映射规则是否允许覆盖条目中已经存在数据。如果该选项未指定,默认是\texttt{false}。简易形式\opt{overwrite}等价于\kvopt{overwrite}{true}。
%Specify whether a mapping rule is allowed to overwrite already existing data in an entry. If this option is not specified, the default is \texttt{false}. The short form \opt{overwrite} is equivalent to \kvopt{overwrite}{true}.

\end{optionlist*}

\cmditem{map}[options]{restrictions,steps}

A container for an ordered set of map \cmd{step}s, optionally restricted to particular entrytypes or data sources. This is a grouping element to allow a set of mapping steps to apply only to specific entrytypes or data sources. Mapping steps must always be contained within a \cmd{map} element. The \prm{options} are:

\begin{optionlist*}

\boolitem{overwrite}

As the same option on the parent \cmd{maps} element. This option allows an override on a per-map group basis. If this option is not specified, the default is the parent \cmd{maps} element option value. The short form \opt{overwrite} is equivalent to \kvopt{overwrite}{true}.

\valitem{foreach}{loopval}

Loop over all \cmd{step}s in this \cmd{map}, setting the special variable |$MAPLOOP| %$
to each of the comma-separated values contained in \prm{loopval}. \prm{loopval} can either be the name of a datafield set defined with \cmd{DeclareDatafieldSet} (see \secref{aut:ctm:dsets}), a datasource field which is fetched and parsed as a comma"=separated values list or an explicit comma"=separated values list. \prm{loopval} is determined in this order. This allows the user to repeat a group of \cmd{step}s for each value \prm{loopval}. Using regexp maps, it is possible to create a CSV field for use with this functionality. The special variable |$MAPUNIQ| %$
may also be used the \cmd{step}s to generate a random unique string. This can be useful when creating keys for new entries. An example:

\begin{ltxexample}[style=latex]{}
\DeclareSourcemap{
  \maps[datatype=bibtex]{
    \map[overwrite, foreach={author,editor, translator}]{
      \step[fieldsource=\regexp{$MAPLOOP}, match={Smith}, replace={Jones}]
    }
  }
}
\end{ltxexample}
%$<- to stop emacs highlighting breaking

\intitem{refsection}

Only apply the contained \cmd{step} commands to entries in the reference section with number \prm{refsection}.

\end{optionlist*}

\cmditem{perdatasource}{datasource}

Restricts all \cmd{step}s in this \cmd{map} element to entries from the named \prm{datasource}. The \prm{datasource} name should be exactly as given in a \cmd{addresource} macro defining a data source for the document. Multiple \cmd{perdatasource} restrictions are allowed within a \cmd{map} element.

\cmditem{pertype}{entrytype}

Restricts all \cmd{step}s in this \cmd{map} element to entries of the named \prm{entrytype}. Multiple \cmd{pertype} restrictions are allowed within a \cmd{map} element.

\cmditem{pernottype}{entrytype}

Restricts all \cmd{step}s in this \cmd{map} element to entries not of the named \prm{entrytype}. Multiple \cmd{pernottype} restrictions are allowed within a \cmd{map} element.

\cmditem{step}[options]

A mapping step. Each step is applied sequentially to every relevant entry where <relevant> means those entries which correspond to the data source type, entrytype and data source name restrictions mentioned above. Each step is applied to the entry as it appears after the application of all previous steps. The mapping performed by the step is determined by the following \prm{option}s:

\begin{optionlist*}

\valitem{typesource}{entrytype}
\valitem{typetarget}{entrytype}
\valitem{fieldsource}{entryfield}
\valitem{notfield}{entryfield}
\valitem{fieldtarget}{entryfield}
\valitem{match}{regexp}
\valitem{notmatch}{regexp}
\valitem{replace}{regexp}
\valitem{fieldset}{entryfield}
\valitem{fieldvalue}{string}
\valitem{entryclone}{clonekey}
\valitem{entrynew}{entrynewkey}
\valitem{entrynewtype}{string}
\valitem{entrytarget}{string}
\boolitem[false]{entrynull}
\boolitem[false]{append}
\boolitem[false]{final}
\boolitem[false]{null}
\boolitem[false]{origfield}
\boolitem[false]{origfieldval}
\boolitem[false]{origentrytype}
%
For all boolean \cmd{step} options, the short form \opt{option} is equivalent to \kvopt{option}{true}. The following rules for a mapping step apply:

\renewcommand{\labelitemii}{$\circ$}

\begin{itemize}
\item If \texttt{entrynew} is set, a new entry is created with the entry key \texttt{entrynewkey} and the entry type given in the option \texttt{entrynewtype}. This
  entry is only in-scope during the processing of the current entry and can be referenced by
  \texttt{entrytarget}.  In \texttt{entrynewkey}, you may use standard Perl regular expression
  backreferences to captures from a  previous \texttt{match} step.
\item When a \texttt{fieldset} step has \texttt{entrytarget} set to the entrykey of an entry
  created by \texttt{entrynew}, the target for the field set will be the \texttt{entrytarget} entry
  rather than the entry being currently processed. This allows users to create new entries and set
  fields in them.
\item If \texttt{entrynull} is set, processing of the \cmd{map}
  immediately terminates and the current entry is not created. It is
  as if it did not exist in the datasource. Obviously, you should
  select the entries which you want to apply this to using prior
  mapping steps.
\item If \texttt{entryclone} is set, a clone of the entry is created with an entry key
  \texttt{clonekey}. Obviously this may cause labelling problems in author/year styles etc.
  and should be used with care. The cloned entry is in-scope during the processing of the
  current entry and can be modified by passing its key as the value to \texttt{entrytarget}.
  In \texttt{clonekey}, you may use standard Perl regular expression backreferences to
  captures from a previous \texttt{match} step.
\item Change the \texttt{typesource} \prm{entrytype} to the
  \texttt{typetarget} \prm{entrytype}, if defined. If
  \texttt{final} is \texttt{true} then if the \prm{entrytype} of the entry is not \texttt{typesource}, processing of the parent \cmd{map} immediately terminates.
\item Change the \texttt{fieldsource} \prm{entryfield} to
  \texttt{fieldtarget}, if defined. If
  \texttt{final} is \texttt{true} then if there is no \texttt{fieldsource} \prm{entryfield} in the entry, processing of the parent \cmd{map} immediately terminates.
\item If \texttt{notfield} is used then only apply the step if the \prm{entryfield} does not exist.
\item If \texttt{match} is defined but
  \texttt{replace} is not, only apply the step if the \texttt{fieldsource} \prm{entryfield} matches the
  \texttt{match} regular expression (logic is reversed if you use \texttt{notmatch} instead)\footnote{Regular expressions are full Perl 5.16 regular expressions. This means you may need to deal with special characters, see examples.}. You may use capture parenthesis as usual and refer to these (\$1\ldots\$9) in later \texttt{fieldvalue} specifications. This allows you to pull out parts of some fields and put these parts in other fields.
\item Perform a regular expression match and replace on the value of the \texttt{fieldsource} \prm{entryfield} if \texttt{match} and \texttt{replace} are defined.
\item If \texttt{fieldset} is defined, then its value is \prm{entryfield}
  which will be set to a value specified by further options. If
  \texttt{overwrite} is false for this step and the field to set already
  exists then the map step is ignored. If \texttt{final} is also true for
  this step, then processing of the parent map stops at this point. If
  \texttt{append} is true, then the value to set is appended to the current
  value of \prm{entryfield}. The value to set is specified by a mandatory
  one and only one of the following options:
  \begin{itemize}
    \item\ \texttt{fieldvalue} --- The \texttt{fieldset} \prm{entryfield} is set to the \texttt{fieldvalue} \prm{string}
    \item\ \texttt{null} --- The \texttt{fieldset} \prm{entryfield} is ignored, as if it did not exist in the datasource
    \item\ \texttt{origentrytype} --- The \texttt{fieldset} \prm{entryfield} is set to the most recently mentioned \texttt{typesource} \prm{entrytype} name
    \item\ \texttt{origfield} --- The \texttt{fieldset} \prm{entryfield} is set to the most recently mentioned \texttt{fieldsource} \prm{entryfield} name
    \item\ \texttt{origfieldval} --- The \texttt{fieldset} \prm{entryfield} is set to the most recently mentioned \texttt{fieldsource} value
  \end{itemize}
\end{itemize}

\end{optionlist*}

\end{ltxsyntax}

\noindent With \bibtex\ datasources, you may specify the
pseudo-field \bibfield{entrykey} for \texttt{fieldsource}
which is the citation key of the entry. With \biblatexml\ the \bibfield{entrykey} is a normal attribute and can be reference like any other attribute. Naturally, this <field> cannot
be changed (used as \texttt{fieldset}, \texttt{fieldtarget} or changed using \texttt{replace}).

\begin{ltxsyntax}

\cmditem{DeclareStyleSourcemap}{specification}

This command sets the source mappings used by a style. Such mappings are conceptually separate from user mappings defined with \cmd{DeclareSourcemap} and are applied directly after user maps. The syntax is identical to \cmd{DeclareSourcemap}. This command is provided for style authors so that any maps defined for the style do not interfere with user maps or the default driver maps defined with \cmd{DeclareDriverSourcemap}. This command is for use in style files and can be used multiple times, the maps being run in order of definition.

\end{ltxsyntax}

\begin{ltxsyntax}

\cmditem{DeclareDriverSourcemap}[datatype=driver]{specification}

This command sets the driver default source mappings for the specified \prm{driver}. Such mappings are conceptually separate from user mappings defined with \cmd{DeclareSourcemap} and style mapping defined with \cmd{DeclareStyleSourcemap}. They consist of mappings which are part of the driver setup. Users should not normally need to change these. Driver default mapping are applied after user mappings (\cmd{DeclareSourcemap}) and style mappings (\cmd{DeclareStyleSourcemap}). These defaults are described in Appendix \secref{apx:maps}. The \prm{specification} is identical to that for \cmd{DeclareSourcemap} but without the \cmd{maps} elements: the \prm{specification} is just a list of \cmd{map} elements since each \cmd{DeclareDriverSourcemap} only applies to one datatype driver. See the default definitions in Appendix \secref{apx:maps} for examples.

\end{ltxsyntax}

Here are some data source mapping examples:

\begin{ltxexample}
\DeclareSourcemap{
  \maps[datatype=bibtex]{
    \map{
      \perdatasource{<<example1.bib>>}
      \perdatasource{<<example2.bib>>}
      \step[fieldset=<<keywords>>, fieldvalue={<<keyw1, keyw2>>}]
      \step[fieldsource=<<entrykey>>]
      \step[fieldset=<<note>>, origfieldval]
    }
  }
}
\end{ltxexample}
%
This would add a \bibfield{keywords} field with value <keyw1, keyw2> and set the \bibfield{note} field to the entry key to all entries which are found in either the
\texttt{examples1.bib} or \texttt{examples2.bib} files.
%
\begin{ltxexample}
\DeclareSourcemap{
  \maps[datatype=bibtex]{
    \map{
      \step[fieldsource=<<title>>]
      \step[fieldset=<<note>>, origfieldval]
    }
  }
}
\end{ltxexample}
%
Copy the \bibfield{title} field to the \bibfield{note} field unless the
\bibfield{note} field already exists.
%
\begin{ltxexample}
\DeclareSourcemap{
  \maps[datatype=bibtex]{
    \map{
      \step[typesource=<<chat>>, typetarget=<<customa>>, final]
      \step[fieldset=<<type>>, origentrytype]
    }
  }
}
\end{ltxexample}
%
Any \bibfield{chat} entrytypes would become \bibfield{customa} entrytypes and
would automatically have a \bibfield{type} field set to
<chat> unless the \bibfield{type} field already exists in the entry (because
\texttt{overwrite} is false by default). This mapping applies only to entries of type
\bibtype{chat} since the first step has \texttt{final} set and so if the
\texttt{typesource} does not match the entry entrytype, processing of this
\cmd{map} immediately terminates.
%
\begin{ltxexample}
\DeclareSourcemap{
  \maps[datatype=bibtex]{
    \map{
      \perdatasource{<<examples.bib>>}
      \pertype{<<article>>}
      \pertype{<<book>>}
       \step[fieldset=<<abstract>>, null]
       \step[fieldset=<<note>>, fieldvalue={<<Auto-created this field>>}]
    }
  }
}
\end{ltxexample}
%
Any entries of entrytype \bibtype{article} or \bibtype{book} from the
\texttt{examples.bib} datasource would have their \bibfield{abstract}
fields removed and a \bibfield{note} field added with value <Auto-created this field>.
%
\begin{ltxexample}
\DeclareSourcemap{
  \maps[datatype=bibtex]{
    \map{
       \step[fieldset=<<abstract>>, null]
       \step[fieldsource=<<conductor>>, fieldtarget=<<namea>>]
       \step[fieldsource=<<gps>>, fieldtarget=<<usera>>]
    }
  }
}
\end{ltxexample}
%
This removes \bibfield{abstract} fields from any entry, changes
\bibfield{conductor} fields to \bibfield{namea} fields and changes \bibfield{gps}
fields to \bibfield{usera} fields.
%
\begin{ltxexample}
\DeclareSourcemap{
  \maps[datatype=bibtex]{
    \map{
       \step[fieldsource=<<pubmedid>>, fieldtarget=<<eprint>>, final]
       \step[fieldset=<<eprinttype>>, origfield]
       \step[fieldset=<<userd>>, fieldvalue={<<Some string of things>>}]
    }
  }
}
\end{ltxexample}
%
Applies only to entries with \bibfield{pubmed} fields and maps
\bibfield{pubmedid} fields to \bibfield{eprint} fields, sets the \bibfield{eprinttype}
field to <pubmedid> and also sets the \bibfield{userd} field to the string
<Some string of things>.
%
\begin{ltxexample}
\DeclareSourcemap{
  \maps[datatype=bibtex]{
    \map{
       \step[fieldsource=<<series>>,
             match=\regexp{<<\A\d*(.+)>>},
             replace=\regexp{<<\L$1>>}]
    }
  }
}
\end{ltxexample}
%$<- to stop emacs highlighting breaking
Here, the contents of the \bibfield{series} field have leading numbers stripped and the remainder of the contents lowercased. Since regular expressions usually contain all sort of special characters, it is best to enclose them in the provided \cmd{regexp} macro as shown---this will pass the expression through to \biber\ correctly.
%
\begin{ltxexample}
\DeclareSourcemap{
  \maps[datatype=bibtex]{
    \map{
       \step[fieldsource=<<maintitle>>,
             match=\regexp{<<Collected\s+Works.+Freud>>},
             final]
       \step[fieldset=<<keywords>>, fieldvalue=<<freud>>]
    }
  }
}
\end{ltxexample}
%$<- to stop emacs highlighting breaking
Here, if for an entry, the \bibfield{maintitle} field matches a particular regular expression, we set a special keyword so we can, for example, make a references section just for certain items.
%
\begin{ltxexample}
\DeclareSourcemap{
  \maps[datatype=bibtex]{
    \map{
       \step[fieldsource=<<lista>>, match=\regexp{<<regexp>>}, final]
       \step[fieldset=<<lista>>, null]
    }
  }
}
\end{ltxexample}
%
If an entry has a \bibfield{lista} field which matches regular expression <regexp>, then it is removed.
%
\begin{ltxexample}
\DeclareSourcemap{
  \maps[datatype=bibtex]{
    \map[overwrite=false]{
       \step[fieldsource=<<author>>]
       \step[fieldset=<<editor>>, origfieldval, final]
       \step[fieldsource=<<editor>>, match=\regexp{<<\A(.+?)\s+and.*>>}, replace={<<$1>>}]
    }
  }
}
\end{ltxexample}
%$<- to stop emacs highlighting breaking
For any entry with an \bibfield{author} field, try to set
\bibfield{editor} to the same as \bibfield{author}. If this fails because
\bibfield{editor} already exists, stop, otherwise truncate
\bibfield{editor} to just the first name in the name list.
%
\begin{ltxexample}
\DeclareSourcemap{
  \maps[datatype=bibtex]{
    \map{
       \step[fieldsource=<<author>>,
             match={<<Smith, Bill>>},
             replace={<<Smith, William>>}]
       \step[fieldsource=<<author>>,
             match={<<Jones, Baz>>},
             replace={<<Jones, Barry>>}]
    }
  }
}
\end{ltxexample}
%
Here, we use multiple match/replace for the same field to regularise some inconstant name variants. Bear in mind that \cmd{step} processing within a \opt{map} element is sequential and so the changes from a previous \cmd{step}s are already committed. Note that we don't need the \cmd{regexp} macro to protect the regular expressions in this example as they contain no characters which need special escaping. Please note that due to the difficulty of protecting regular expressions in \LaTeX, there should be no literal spaces in the argument to \cmd{regexp}. Please use escape code equivalents if spaces are needed. For example, this example, if using \cmd{regexp}, should be:
%
\begin{ltxexample}
\DeclareSourcemap{
  \maps[datatype=bibtex]{
    \map{
       \step[fieldsource=<<author>>,
             match=\regexp{<<Smith,\s+Bill>>},
             replace=\regexp{<<Smith,\x20William>>}]
       \step[fieldsource=<<author>>,
             match=\regexp{<<Jones,\s+Baz>>},
             replace=\regexp{<<Jones,\x20Barry>>}]
    }
  }
}
\end{ltxexample}
%
Here, we have used the hexadecimal escape sequence <|\x20|> in place of literal spaces in the replacement strings.
%
\begin{ltxexample}
\DeclareSourcemap{
  \maps[datatype=bibtex]{
    \map[overwrite]{
       \step[fieldsource=<<author>>, match={<<Doe,>>}, final]
       \step[fieldset=<<shortauthor>>, origfieldval]
       \step[fieldset=<<sortname>>, origfieldval]
       \step[fieldsource=<<shortauthor>>,
             match=\regexp{<<Doe,\s*(?:\.|ohn)(?:[-]*)(?:P\.|Paul)*>>},
             replace={<<Doe, John Paul>>}]
       \step[fieldsource=<<sortname>>,
             match=\regexp{<<Doe,\s*(?:\.|ohn)(?:[-]*)(?:P\.|Paul)*>>},
             replace={<<Doe, John Paul>>}]
    }
  }
}
\end{ltxexample}
%
Only applies to entries with an \bibfield{author} field matching <Doe,>. First the \bibfield{author} field is copied to both the \bibfield{shortauthor} and \bibfield{sortname} fields, overwriting them if they already exist. Then, these two new fields are modified to canonicalise a particular name, which presumably has some variants in the data source.
%
\begin{ltxexample}
\DeclareSourcemap{
  \maps[datatype=bibtex]{
    \map[overwrite]{
      \step[fieldsource=<<verba>>, final]
      \step[fieldset=<<verbb>>, fieldvalue=<</>>, append]
      \step[fieldset=<<verbb>>, origfieldval, append]
      \step[fieldsource=<<verbb>>, final]
      \step[fieldset=<<verbc>>, fieldvalue=<</>>, append]
      \step[fieldset=<<verbc>>, origfieldval, append]
    }
  }
}
\end{ltxexample}
%
This example demonstrates the sequential nature of the step processing and the \opt{append} option. If an entry has a \bibfield{verba} field then first, a forward slash is appended to the \bibfield{verbb} field. Then, the contents of \bibfield{verba} are appended to the \bibfield{verbb} field. A slash is then appended to the \bibfield{verbc} field and the contents of \bibfield{verbb} are appended to the \bibfield{verbc} field.
%
\begin{ltxexample}
\DeclareSourcemap{
  \maps[datatype=bibtex]{
    \map[overwrite]{
      \step[fieldset=<<autourl>>, fieldvalue={<<http://scholar.google.com/scholar?q=">>}]
      \step[fieldsource=<<title>>]
      \step[fieldset=<<autourl>>, origfieldval, append]
      \step[fieldset=<<autourl>>, fieldvalue={<<"+author:>>}, append]
      \step[fieldsource=<<author>>, match=\regexp{<<\A([^,]+)\s*,>>}]
      \step[fieldset=<<autourl>>, fieldvalue={<<$1>>}, append]
      \step[fieldset=<<autourl>>, fieldvalue={<<&as_ylo=>>}, append]
      \step[fieldsource=<<year>>]
      \step[fieldset=<<autourl>>, origfieldval, append]
      \step[fieldset=<<autourl>>, fieldvalue={<<&as_yhi=>>}, append]
      \step[fieldset=<<autourl>>, origfieldval, append]
    }
  }
}
\end{ltxexample}%$ <- keep AucTeX highlighting happy
%
This example assumes you have created a field called \bibfield{autourl} using the datamodel macros from \secref{aut:ctm:dm} in order to hold, for example, a Google Scholar query URL auto-created from elements of the entry. The example progressively extracts information from the entry, constructing the URL as it goes. It demonstrates that it is possible to refer to parenthetical matches from the most recent \texttt{match} in any following \texttt{fieldvalue} which allows extracting the family name from the \bibfield{author}, assuming a <family, given> format. The resulting field could then be used as a hyperlink from, for example, the title of the work in the bibliography.
%
\begin{ltxexample}
\DeclareSourcemap{
  \maps[datatype=bibtex]{
    \map{
      \step[fieldsource=<<title>>, match={A Title}, final]
      \step[entrynull]
    }
  }
}
\end{ltxexample}
%
Any entry with a \bibfield{title} field matching <A Title> will be completely ignored.
%
\begin{ltxexample}
\DeclareSourcemap{
  \maps[datatype=bibtex]{
    \map{
      \pernottype{book}
      \pernottype{article}
      \step[entrynull]
    }
  }
}
\end{ltxexample}
%
Any entry which is not a \bibtype{book} or \bibtype{article} will be ignored.
%
\begin{ltxexample}
\DeclareSourcemap{
  \maps[datatype=bibtex]{
    \map{
      \perdatasource{<<biblatex-examples.bib>>}
      \step[entryclone={rel-}]
    }
  }
}
\end{ltxexample}
%
Here, a clone of an entry from the specified data source will be created. The entry key of the clone will be the same as the original but prefixed by the value of the \texttt{entryclone} parameter. The cloned entry would still need to be cited in the document using its new entry key. This type of mapping step should be used with care as it may produce labelling problems in authoryear styles which use, for example, \opt{extrayear}. One use case is for numeric styles which contain multiple bibliographies containing the same entry. In this case, you may need different bibliography number labeld for the same entry and this is very tricky when there is only one entry which needs different labels. Creating clones with different entry keys solves this problem.

\biblatexml\ datasources are more structured than \bibtex\ since they are XML. Sourcemapping is possible with \biblatexml\ too but the specifications of source and target fields etc. also support XPath 1.0 paths in order to be able to work with the structured data. Fields can be specified as per the \bibtex\ examples above and these are converted into XPath 1.0 queries internally as necessary. For example:

\begin{ltxexample}
\DeclareSourcemap{
  \maps[datatype=biblatexml]{
    \map{
   \step[fieldsource=\regexp{./bltx:names[@type='author']/bltx:name[2]/bltx:namepart[@type='family']},
      match=\regexp{\ASmith},
      replace={Jones}]
    }
    \map{
      \step[fieldsource=editor, fieldtarget=translator]
    }
    \map{
      \step[fieldsource=\regexp{./bltx:names[@type='editor']},
            fieldtarget=\regexp{./bltx:names[@type='translator']}]
    }
    \map{
      \step[fieldset=\regexp{./bltx:names[@type='author']/bltx:name[2]/@useprefix},
            fieldvalue={false}]
    }
  }
}
\end{ltxexample}
%
These maps, respectively,

\begin{itemize}
\item Replace the family name <Smith> of the second \bibfield{author} name with <Jones>
\item Move the \bibfield{editor} to \bibfield{translator}
\item Move the \bibfield{editor} to \bibfield{translator} but with explicit XPaths
\item Set the per-namelist \opt{useprefix} option on the \bibfield{author} name list to <false>
\end{itemize}

\subsubsection{数据模型规范}%Data Model Specification
\label{aut:ctm:dm}
\biblatex 使用的数据模型包括4个主要元素:
%The data model which \biblatex uses consists of four main elements:

\begin{itemize}
\item Specification of constant strings and lists of strings
\item Specification of valid Entrytypes
\item Specification of valid Fields along with their type, datatype and any special flags
\item Specification of which Fields are valid in which Entrytypes
\item Specification of constraints which can be used to validate data against the data model
\end{itemize}

The default data model is defined in the core \biblatex file \file{blx-dm.def} using the macros described in this section. The default data model is described in detail in \secref{bib}. The data model is used internally by \biblatex and also by the backend. In practice, changing the data model means that you can define the entrytypes and fields for your datasources and validate your data against the data model. Naturally, this is not much use unless your style supports any new entrytypes or fields and it raises issues of portability between styles (although this can be mitigated by using the dynamic data modification functionality described in \secref{aut:ctm:map}).

Validation against the data model means that after mapping your data sources into the data model, \biber (using its \path{--validate_datamodel} option) can check:

\begin{itemize}
\item Whether all entrytypes are valid entrytypes
\item Whether all fields are valid fields for their entrytype
\item Whether the fields obey various constraints on their format which you specify
\end{itemize}
%
Redefining the data model can be done in several places. Style authors can create a \file{.dbx} file which contains the data model macros required and this will be loaded automatically when using the \biblatex package \opt{style} option by looking for a file named after the style with a \file{.dbx} extension (just like the \file{.cbx} and \file{.bbx} files for a style). If the \opt{style} option is not used but rather the \opt{citestyle} and \opt{bibstyle} options, then the package will try to load \file{.dbx} files called \file{$<$citestyle$>$.dbx} and \file{$<$bibstyle$>$.dbx}.
Alternatively, the name of the data model file can be different from any of the style option names by specifying the name (without \file{.dbx} extension) to the package \opt{datamodel} option. After loading the style data model file, \biblatex then loads, if present, a users \file{biblatex-dm.cfg} which should be put somewhere \biblatex can find it, just like the main configuration file \sty{biblatex.cfg}. To summarise, the data model is determined by adding to the data model from each of these locations, in order:\\

\noindent\file{blx-dm.def}$\rightarrow$\\
\hspace*{1em}\file{$<$datamodel option$>$.dbx} $\rightarrow$\\
\hspace*{2em}\file{$<$style option$>$.dbx} $\rightarrow$\\
\hspace*{3em}\file{$<$citestyle option$>$.dbx} and \file{$<$bibstyle option$>$.dbx} $\rightarrow$\\
\hspace*{4em}\file{biblatex-dm.cfg}\\

\noindent It is not possible to add to a loaded data model by using the macros below in your preamble as the preamble is read after \biblatex has defined critical internal macros based on the data model. If any data model macro is used in a document, it will be ignored and a warning will be generated. The data model is defined using the following macros:

\begin{ltxsyntax}

\cmditem{DeclareDatamodelConstant}[options]{name}{constantdef}

Declares the \prm{name} as a datamodel constant with definition \prm{constantdef}. Such constants are typically used internally by \biber.

\begin{optionlist*}

\choitem[string]{type}{string, list}

A constant can be a simple string (default if the \prm{type} option is omitted) or a comma"=separated list of strings.

\end{optionlist*}

\cmditem{DeclareDatamodelEntrytypes}[options]{entrytypes}

Declares the comma"=separated list of \prm{entrytypes} to be valid entrytypes in the data model. As usual in \tex csv lists, make sure each element is immediately followed by a comma or the closing brace---no extraneous whitespace.

\begin{optionlist*}

\boolitem[false]{skipout}

This entrytype is not output to the \file{.bbl}. Typically used for special entrytypes which are processed and consumed by the backend such as \bibtype{xdata}.

\end{optionlist*}

\cmditem{DeclareDatamodelFields}[options]{fields}

Declares the comma"=separated list of \prm{fields} to be valid fields in the data model with associated comma"=separated \prm{options}. The \prm{type} and \prm{datatype} options are mandatory. All valid \prm{options} are:

\begin{optionlist*}

\valitem{type}{field type}

Set the type of the field as described in \secref{bib:fld:typ}, typically <field> or <list>.

\valitem{format}{field format}

Any special format of the field. Normally unspecified but can take the value <xsv> which tells \biber that this field has a separated values format. The exact separator can be controlled with the \biber option \opt{xsvsep} and defaults to the expected comma surrounded by optional whitespace.

\valitem{datatype}{field datatype}

Set the datatype of the field as described in \secref{bib:fld:typ}. For example, <name> or <literal>.

\boolitem[false]{nullok}

The field is allowed to be defined but empty.

\boolitem[false]{skipout}

The field is not output to the \file{.bbl} and is therefore not present during \biblatex style processing. As usual in \tex csv lists, make sure each element is immediately followed by a comma or the closing brace---no extraneous whitespace.

\boolitem[false]{label}

The field can be used as a label in a bibliography or bibliography list. Specifying this causes \biblatex to create several helper macros for the field so that there are some internal lengths and headings etc. defined.

\end{optionlist*}

\cmditem{DeclareDatamodelEntryfields}[entrytypes]{fields}

Declares that the comma"=separated list of \prm{fields} is valid for the comma"=separated list of \prm{entrytypes}. If \prm{entrytypes} is not given, the fields are valid for all entrytypes. As usual in \tex csv lists, make sure each element is immediately followed by a comma or the closing brace---no extraneous whitespace.

\cmditem{DeclareDatamodelConstraints}[entrytypes]{specification}

If a comma"=separated list of \prm{entrytypes} is given, the constraints apply only to those entrytypes. The \prm{specification} is an undelimited list of \cmd{constraint} directives which specify a constraint. Spaces, tabs, and line endings may be used freely to visually arrange the \prm{specification}. Blank lines are not permissible.

\cmditem{constraint}[type=constrainttype]{elements}

Specifies a constraint of type \prm{constrainttype}. Valid constraint types are:

\begin{optionlist*}

\choitem{type}{data, mandatory, conditional}

Constraints of type <data> put restrictions on the value of a field. Constraints of type <mandatory> specify which fields or combinations of fields an entrytype should have. Constraints of type <conditional> allow more sophisticated conditional and quantified field constraints.

\choitem{datatype}{integer, isbn, issn, ismn, date, pattern}

For constraints of type \prm{data}, constrain field values to be the given datatype.

\valitem{rangemin}{num}

For constraints of \prm{type} <data> and \prm{datatype} <integer>, constrain field values to be at least \prm{num}.

\valitem{rangemax}{num}

For constraints of \prm{type} <data> and \prm{datatype} <integer>, constrain field values to be at most \prm{num}.

\valitem{pattern}{patt}

For constraints of \prm{type} <data> and \prm{datatype} <pattern>, constrain field values to match regular expression pattern \prm{patt}. It is best to wrap any regular expression in the macro \cmd{regexp}, see \secref{aut:ctm:map}.

\end{optionlist*}

A \cmd{constraint} macro may contain any of the following:

\cmditem{constraintfieldsor}{fields}

For constraints of \prm{type} <mandatory>, specifies that an entry must contain a boolean OR of the \cmd{constraintfield}s.

\cmditem{constraintfieldsxor}{fields}

For constraints of \prm{type} <mandatory>, specifies that an entry must contain a boolean XOR of the \cmd{constraintfield}s.

\cmditem{antecedent}[quantifier=quantspec]{fields}

For constraints of \prm{type} <conditional>, specifies a quantified set of \cmd{constraintfield}s which must be satisfied before the \cmd{consequent} of the constraint is checked. \prm{quantspec} should have one of the following values:

\begin{optionlist*}

\choitem{quantifier}{all, one, none}

Specifies how many of the \cmd{constrainfield}'s inside the \cmd{antecedent} have to be present to satisfy the antecedent of the conditional constraint.

\end{optionlist*}

\cmditem{consequent}[quantifier=quantspec]{fields}

For constraints of \prm{type} <conditional>, specifies a quantified set of \cmd{constraintfield}s which must be satisfied if the preceding \cmd{antecedent} of the constraint was satisfied. \prm{quantspec} should have one of the following values:

\begin{optionlist*}

\choitem{quantifier}{all, one, none}

Specifies how many of the \cmd{constraintfield}'s inside the \cmd{consequent} have to be present to satisfy the consequent of the conditional constraint.

\end{optionlist*}

\cmditem{constraintfield}{field}

For constraints of \prm{type} <data>, the constraint applies to this \prm{field}. For constraints of \prm{type} <mandatory>, the entry must contain this \prm{field}.

The data model declaration macros may be used multiple times as they append to the previous definitions. In order to replace, change or remove existing definitions (such as the default model which is loaded with \biblatex), you should reset (clear) the current definition and then set what you want using the following macros. Typically, these macros will be the first things in any \file{biblatex-dm.cfg} file:

\cmditem{ResetDatamodelEntrytypes}

Clear all data model entrytype information.

\cmditem{ResetDatamodelFields}

Clear all data model field information.

\cmditem{ResetDatamodelEntryfields}

Clear all data model fields for entrytypes information.

\cmditem{ResetDatamodelConstraints}

Clear all data model fields Constraints information.

\end{ltxsyntax}

Here is an example of a simple data model. Refer to the core \biblatex file \file{blx-dm.def} for the default data model specification.

\begin{ltxexample}
\ResetDatamodelEntrytypes
\ResetDatamodelFields
\ResetDatamodelEntryfields
\ResetDatamodelConstraints

\DeclareDatamodelEntrytypes{<<entrytype1, entrytype2>>}

\DeclareDatamodelFields[type=field, datatype=literal]{<<field1,field2,field3,field4>>}

\DeclareDatamodelEntryfields{<<field1>>}
\DeclareDatamodelEntryfields[entrytype1]{<<field2,field3>>}
\DeclareDatamodelEntryfields[entrytype2]{<<field2,field3,field4>>}

\DeclareDatamodelConstraints[<<entrytype1>>]{
  \constraint[type=data, datatype=integer, rangemin=3, rangemax=10]{
    \constraintfield{<<field1>>}
  }
  \constraint[type=mandatory]{
    \constraintfield{<<field1>>}
    \constraintfieldsxor{
      \constraintfield{<<field2>>}
      \constraintfield{<<field3>>}
    }
  }
}
\DeclareDatamodelConstraints{
  \constraint[type=conditional]{
    \antecedent[quantifier=none]{
      \constraintfield{<<field2>>}
    }
    \consequent[quantifier=all]{
      \constraintfield{<<field3>>}
      \constraintfield{<<field4>>}
    }
  }
}
\end{ltxexample}
%
This model specifies:

\begin{itemize}
\item Clear the default data model completely
\item Two valid entry types \bibtype{entrytype1} and \bibtype{entrytype2}
\item Four valid literal field fields
\item \bibfield{field1} is valid for all entrytypes
\item \bibfield{field2} and \bibfield{field3} are valid for \bibfield{entrytype1}
\item \bibfield{field2}, \bibfield{field3} and \bibfield{field4} are valid for \bibtype{entrytype2}
\item For \bibtype{entrytype1}:
  \begin{itemize}
  \item \bibfield{field1} must be an integer between 3 and 10
  \item \bibfield{field1} must be present
  \item One and only one of \bibfield{field2} or \bibfield{field3} must be present
  \end{itemize}
\item For any entrytype, if \bibfield{field2} is not present, \bibfield{field3} and \bibfield{field4} must be present
\end{itemize}

\subsubsection{标签}%Labels
\label{aut:ctm:lab}
字母顺序制样式使用一个标签来区分参考文献条目。这个标签由条目的内容使用一个描述怎么构建标签的模板构建。该模板可以全局自定义或者分条目类型定义。如何抽取姓名域的部分作为标签使用一个独立的模板,因为姓名域是相当复杂的域。标签的自定义需要用\biber 后端程序而不能用其它后端程序。
%Alphabetic styles use a label to identify bibliography entries. This label is constructed from components of the entry using a template which describes how to build the label. The template can be customised on a global or per-type basis. A separate template is used to specify how to extract parts of name fields for labels, since names can be quite complex fields.

\begin{ltxsyntax}

\cmditem{DeclareLabelalphaTemplate}[entrytype, \dots]{specification}

Defines the alphabetic label template for the given entrytypes. If no entrytypes are specified in the first argument, then the global label template is defined. The \prm{specification} is an undelimited list of \cmd{labelelement} directives which specify the elements used to build the label. Spaces, tabs, and line endings may be used freely to visually arrange the \prm{specification}. Blank lines are not permissible. This command may only be used in the preamble.

\cmditem{labelelement}{elements}

Specifies the elements used to build the label. The \prm{elements} are an undelimited list of \cmd{field} or \cmd{literal} commands which are evaluated in the order in which they are given. The first \cmd{field} or \cmd{literal} which expands to a non-empty string is used as the \cmd{labelelement} expansion and the next \cmd{labelelement}, if any, is then processed.

\cmditem{field}[options]{field}

If \prm{field} is non-empty, use it as the current label \cmd{labelelement}, subject to the options below. Useful values for \prm{field} are typically the name list type fields, date fields, and title fields. You may also use the `citekey' pseudo-field to specify the citation key as part of the label. Name list fields are treated specially and when a name list field is specified, the template defined with \cmd{DeclareLabelalphaNameTemplate} is used to extract parts from the name which then returns the string that the \cmd{field} option uses.

\begin{optionlist*}

\boolitem[false]{final}

This option marks a \cmd{field} directive as the final one in the \prm{specification}. If the \prm{field} is non-empty, then this field is used for the label and the remainder of the \prm{specification} will be ignored. The short form \opt{final} is equivalent to \kvopt{final}{true}.

\boolitem[false]{lowercase}

Forces the label part derived from the field to lowercase. By default, the case is taken from the field source and not modified.

\intitem[1]{strwidth}

The number of characters of the \prm{field} to use. This setting may be overridden by an individual name part when extracting characters from a name. See \cmd{DeclareLabelalphaNameTemplate} below.

\choitem[left]{strside}{left, right}

The side of the string from which to take the \texttt{strwidth} number of characters. This setting may be overridden by an individual name part when extracting characters from a name. See \cmd{DeclareLabelalphaNameTemplate} below.

\choitem[right]{padside}{left, right}

Side to pad the label part when using the \texttt{padchar} option. Only for use with fixed-width label strings (\texttt{strwidth}).

\valitem{padchar}{character}

If present, pads the label part on the \texttt{padside} side with the specified character to the length of \texttt{strwidth}. Only for use with fixed-width label strings (\texttt{strwidth}).

\boolitem[false]{uppercase}

Forces the label part derived from the field to uppercase. By default, the case is taken from the field source and not modified.

\boolitem[false]{varwidth}

Use a variable width, left-side substring of characters from the string returned for \prm{field}. The length of the string is determined by the minimum length needed to disambiguate the substring from all other \prm{field} elements in the same position in the label. For name list fields, this means that each name substring is disambiguated from all other name substrings which occur in the same position in the name list (see examples below). This option overrides \texttt{strwidth} if both are used. The short form \opt{varwidth} is equivalent to \kvopt{varwidth}{true}. For name list fields, the \cmd{namepart}s with the \opt{pre} option set are prepended to the string returned from this disambiguation.

\boolitem[false]{varwidthnorm}

As \texttt{varwidth} but will force the disambiguated substrings for the \prm{field} to be the same length as the longest disambiguated substring. This can be used to regularise the format of the labels if desired. This option overrides \texttt{strwidth} if both are used. The short form \opt{varwidthnorm} is equivalent to \kvopt{varwidthnorm}{true}.

\boolitem[false]{varwidthlist}

Alternative method of automatic label disambiguation where the field as a whole is disambiguated from all other fields in the same label position. For non-name list fields, this is equivalent to \texttt{varwidth}. For name list fields, names in a name list are not disambiguated from other names in the same position in their name lists but instead the entire name list is disambiguated as a whole from other name lists (see examples below). This option overrides \texttt{strwidth} if both are used. The short form \opt{varwidthlist} is equivalent to \kvopt{varwidthlist}{true}.  For name list fields, the \cmd{namepart}s with the \opt{pre} option set are prepended to the string returned from this disambiguation.

\intitem{strwidthmax}

When using \texttt{varwidth}, this option sets a limit (in number of characters) on the length of variable width substrings. This option can be used to regularise the label.

\intitem[1]{strfixedcount}

When using \texttt{varwidthnorm}, there must be at least \texttt{strfixedcount} disambiguated substrings with the same, maximal length to trigger the forcing of all disambiguated substrings to this same maximal length.

\valitem{ifnames}{range}

Only use this \cmd{field} specification if it is a name list field with a number of names matching the \texttt{ifnames} range value. This allows a \cmd{labelelement} to be conditionalised on name length (see below). The range can specified as in the following examples:

\begin{lstlisting}[language=xml]
ifnames=3     -> Only apply to name lists containing exactly 3 names
ifnames={2-4} -> Only apply to name lists containing minimum 2 and maximum 4 names
ifnames={-3}  -> Only apply to name lists containing at most 3 names
ifnames={2-}  -> Only apply to name lists containing at least 2 names
\end{lstlisting}

\valitem{names}{range}

By default, for name list fields, the names used range from the first name to the \cnt{maxalphanames}\slash \cnt{minalphanames} truncation. This option can be used to override this with an explicit range of names to consider. The plus <+> sign is a special end of range marker denoting the truncation point of max/minalphanames. The range separator can be any number of characters with the Unicode Dash property. For example:

\begin{lstlisting}[language=xml]
name=3     -> Use first 3 names in the name list
name={2-3} -> Use second and thirds names only
name={-3}  -> Same as 1-3
name={2-}  -> Use all names starting with the second name (ignoring max/minalphanames truncation)
name={2-+} -> Use all names starting with the second name (respecting max/minalphanames truncation)
\end{lstlisting}

\valitem[empty]{namessep}{string}

An arbitrary string separator to put between names in a namelist.

\boolitem[false]{noalphaothers}

By default, \cmd{labelalphaothers} is appended to label parts derived from name lists if there are more names in the list than are shown in the label part. This option can be used to disable the default behaviour.

\end{optionlist*}

\cmditem{literal}{characters}

Insert the literal \prm{characters} into the label at this point.

\end{ltxsyntax}
%
When a name list \cmd{field} is specified, the method of extracting the string is specified by a separate template specified by the following command:

\begin{ltxsyntax}

\cmditem{DeclareLabelalphaNameTemplate}[entrytype, \dots]{specification}

Specifies the template to use to extract a label string from a name list when a \cmd{field} specification in \cmd{DeclareLabelalphaTemplate} contains a name list. The template can be specified per"=entrytype.

\cmditem{namepart}[options]{namepart}

\prm{namepart} is one of the datamodel nameparts defined with the \cmd{DeclareDatamodelConstant} command (see \secref{aut:bbx:drv}). The \opt{options} are:

\begin{optionlist*}

\boolitem[false]{use}

Only use the \prm{namepart} in constructing the label information if there is a corresponding option \opt{use<namepart>} and that option is true.

\boolitem[false]{pre}

When constructing label strings from names, the \cmd{namepart} \emph{without} a \opt{pre} option will be used to construct label string, passing through disambiguation, substring etc. operations as specified by the \cmd{field} options in \cmd{DeclareLabelalpaTemplate}. Then the \cmd{namepart} options \emph{with} the \opt{pre} option set will be prepended to the result, (in the order given, if there are more than one such \cmd{namepart}s). This allows to unconditionally prepend certain namepart information to name label strings, like name prefices. Note that the \opt{uppercase} and \opt{lowercase} options of \cmd{field} in \cmd{DeclareLabelalphaTemplate} are applied to the entire label returned from \cmd{DeclareLabelalphaTemplate}, both \opt{pre} parts and non \opt{pre}.

\boolitem[false]{compound}

For static (non-varwidth) disambiguation in \cmd{DeclareLabelalphaTemplate}, nameparts separated by whitespace or hyphens (compound names) as separate names for label generation. This means that when forming a label out of, for example the surname <Ballam Forsyth> with a 1 character, left-side substring, this name would give <BF> with \kvopt{compound}{true} and <B> with \kvopt{compound}{false}. The short form \opt{compound} is equivalent to \kvopt{compound}{true}.

\intitem[1]{strwidth}

The number of characters of the \prm{namepart} to use.

\choitem[left]{strside}{left, right}

The side of the string from which to take the \texttt{strwidth} number of characters.

\end{optionlist*}

\end{ltxsyntax}

Note that the templates for labels can be defined per-type and you should be aware of this when using the automatically disambiguated label functionality. Disambiguation is not per-type as this might lead to ambiguity due to different label formats for different types being isolated from each others disambiguation process. Normally, you will want to use very different label formats for different types to make the type obvious by the label.

Here are some examples. The default global \biblatex alphabetic label template is defined below. Firstly, \bibfield{shorthand} has \kvopt{final}{true} and so if there is a \bibfield{shorthand} field, it is used as the label and nothing more of the template is considered. Next, the \bibfield{label} field is used as the first label element if it exists. Otherwise, if there is only one name (\kvopt{ifnames}{1}) in the \bibfield{labelname} list, then three characters from the left side of the family name in the \bibfield{labelname} are used as the first label element. If the \bibfield{labelname} has more than one name in it, one character from the left side of each family name is used as the first label element. The second label element consists of 2 characters from the right side of the \bibfield{year} field.

The default template for constructing labels from names is also shown. This prepends the first character from the left side of any prefix (if the \opt{useprefix} option is true) to a label extracted from the family name (according to the options on the calling \cmd{field} option from \cmd{DeclareLabelalphaTemplate}), allowing for compound family names.

\begin{ltxexample}
\DeclareLabelalphaTemplate{
  \labelelement{
    \field[final]{<<shorthand>>}
    \field{<<label>>}
    \field[strwidth=3,strside=left,ifnames=1]{<<labelname>>}
    \field[strwidth=1,strside=left]{<<labelname>>}
  }
  \labelelement{
    \field[strwidth=2,strside=right]{<<year>>}
  }
}

\DeclareLabelalphaNameTemplate{
  \namepart[use=true, pre=true, strwidth=1, compound=true]{prefix}
  \namepart{family}
}

\end{ltxexample}
%
To get an idea of how the label automatic disambiguation works, consider the following author lists:

\begin{lstlisting}{}
Agassi, Chang, Laver   (2000)
Agassi, Connors, Lendl (2001)
Agassi, Courier, Laver (2002)
Borg, Connors, Edberg  (2003)
Borg, Connors, Emerson (2004)
\end{lstlisting}
%
Assuming a template declaration such as:

\begin{ltxexample}
\DeclareLabelalphaTemplate{
  \labelelement{
    \field[varwidth]{<<labelname>>}
  }
}
\end{ltxexample}
%
Then the labels would be:

\begin{lstlisting}{}
Agassi, Chang, Laver    [AChLa]
Agassi, Connors, Lendl  [AConLe]
Agassi, Courier, Laver  [ACouLa]
Borg, Connors, Edberg   [BConEd]
Borg, Connors, Emerson  [BConEm]
\end{lstlisting}
%
With normalised variable width labels defined:

\begin{ltxexample}
\DeclareLabelalphaTemplate{
  \labelelement{
    \field[varwidthnorm]{<<labelname>>}
  }
}
\end{ltxexample}
%
You would get the following as the substrings of names in each position are extended to the length of the longest substring in that same position:

\begin{lstlisting}{}
Agassi, Chang, Laver    [AChaLa]
Agassi, Connors, Lendl  [AConLe]
Agassi, Courier, Laver  [ACouLa]
Borg, Connors, Edberg   [BConEd]
Borg, Connors, Emerson  [BConEm]
\end{lstlisting}
%
With a restriction to two characters for the name components of the label element defined like this:

\begin{ltxexample}
\DeclareLabelalphaTemplate{
  \labelelement{
    \field[varwidthnorm,strwidthmax=2]{<<labelname>>}
  }
}
\end{ltxexample}
%
This would be the result (note that the individual family name label parts are no longer unambiguous):

\begin{lstlisting}{}
Agassi, Chang, Laver    [AChLa]
Agassi, Connors, Lendl  [ACoLe]
Agassi, Courier, Laver  [ACoLa]
Borg, Connors, Edberg   [BCoEd]
Borg, Connors, Emerson  [BCoEm]
\end{lstlisting}
%
Alternatively, you could choose to disambiguate the name lists as a whole with:

\begin{ltxexample}
\DeclareLabelalphaTemplate{
  \labelelement{
    \field[varwidthlist]{<<labelname>>}
  }
}
\end{ltxexample}
%
Which would result in:

\begin{lstlisting}{}
Agassi, Chang, Laver    [AChL]
Agassi, Connors, Lendl  [ACoL]
Agassi, Courier, Laver  [ACL]
Borg, Connors, Edberg   [BCEd]
Borg, Connors, Emerson  [BCE]
\end{lstlisting}
%
Perhaps you only want to consider at most two names for label generation but disambiguate at the whole name list level:
\begin{ltxexample}
\DeclareLabelalphaTemplate{
  \labelelement{
    \field[varwidthlist,names=2]{<<labelname>>}
  }
}
\end{ltxexample}
%
Which would result in:
\begin{lstlisting}{}
Agassi, Chang, Laver    [ACh+]
Agassi, Connors, Lendl  [ACo+]
Agassi, Courier, Laver  [AC+]
Borg, Connors, Edberg   [BC+a]
Borg, Connors, Emerson  [BC+b]
\end{lstlisting}
%
In this last example, you can see \cmd{labelalphaothers} has been appended to show that there are more names. The last two labels now require disambiguating with \cmd{extraalpha} as there is no way of disambiguating this label name list with only two names.

Finally, here is an example using multiple label elements:

\begin{ltxexample}
\DeclareLabelalphaTemplate{
  \labelelement{
    \field[varwidthlist]{<<labelname>>}
  }
  \labelelement{
    \literal{-}
  }
  \labelelement{
    \field[strwidth=3,strside=right]{<<labelyear>>}
  }
}
\end{ltxexample}
%
Which would result in:
\begin{lstlisting}{}
Agassi, Chang, Laver    [AChL-000]
Agassi, Connors, Lendl  [AConL-001]
Agassi, Courier, Laver  [ACouL-002]
Borg, Connors, Edberg   [BCEd-003]
Borg, Connors, Emerson  [BCEm-004]
\end{lstlisting}
%
Here is another rather contrived example showing that you don't need to specially quote \latex special characters (apart from <\%>, obviously) when specifying padding characters and literals:

\begin{ltxexample}
\DeclareLabelalphaTemplate{
  \labelelement{
    \literal{>}
  }
  \labelelement{
    \literal{\%}
  }
  \labelelement{
    \field[namessep={/}, strwidth=4, padchar=_]{<<labelname>>}
  }
  \labelelement{
    \field[strwidth=3, padchar=&, padside=left]{title}
  }
  \labelelement{
    \field[strwidth=2,strside=right]{<<year>>}
  }
}
\end{ltxexample}
%
which given:
\begin{lstlisting}[style=bibtex]{}
@Book{test,
  author    = {XXX YY and WWW ZZ},
  title     = {T},
  year      = {2007},
}
\end{lstlisting}
would resulting a label looking like this:
\begin{verbatim}
[>%YY/ZZ__&&T07]
\end{verbatim}

Generating labels from fields may involve some difficulties when you have fields containing diacritics, hyphens, spaces etc. Often, you want to ignore things like separator characters or spaces when generating labels. An option is provided to customise the regular expression(s) to strip from a field before it is passed to the label generation system.

\begin{ltxsyntax}

\cmditem{DeclareNolabel}{specification}

Defines regular expressions to strip from any field before generating a label part for the field. The \prm{specification} is an undelimited list of \cmd{nolabel} directives which specify the regular expressions to remove from fields. Spaces, tabs and line endings may be used freely to visually arrange the \prm{specification}. Blank lines are not permissible. This command may only be used in the preamble.

\cmditem{nolabel}{regexp}

Any number of \cmd{nolabel} commands can be given each of which specifies to remove the \prm{regexp} from the copy of the field which the label generation system sees. Since regular expressions usually contain special characters, it is best to enclose them in the provided \cmd{regexp} macro as shown---this will pass the expression through to \biber correctly.

\end{ltxsyntax}

If there is no \cmd{DeclareNolabel} specification, \biber will default to:

\begin{ltxexample}
\DeclareNolabel{
  % strip punctuation, symbols, separator and control characters
  \nolabel{\regexp{<<[\p{P}\p{S}\p{C}]+>>}}
}
\end{ltxexample}
%
This \biber default strips punctuation, symbol, separator and control characters from fields before passing the field string to the label generation system.

\begin{ltxsyntax}

\cmditem{DeclareNolabelwidthcount}{specification}

Defines regular expressions to ignore from any field when counting characters in fixed-width substrings. The \prm{specification} is an undelimited list of \cmd{nolabelwidthcount} directives which specify the regular expressions to ignore when counting characters for fixed-width substrings. Spaces, tabs and line endings may be used freely to visually arrange the \prm{specification}. Blank lines are not permissible. This command may only be used in the preamble.

\cmditem{nolabelwidthcount}{regexp}

Any number of \cmd{nolabelwidthcount} commands can be given each of which specifies to ignore the \prm{regexp} when generating fixed-width substrings during label generation. Since regular expressions usually contain special characters, it is best to enclose them in the provided \cmd{regexp} macro as shown---this will pass the expression through to \biber correctly.

\end{ltxsyntax}

There is no default \cmd{DeclareNolabelwidthcount} specification. Note that
this setting is only taken into account when using fixed-width substrings
(non-varwidth) during label part generation. See \secref{aut:ctm:lab}.

\subsubsection{排序}%Sorting
\label{aut:ctm:srt}

In addition to the predefined sorting schemes discussed in \secref{use:srt}, it is possible to define new ones or modify the default definitions. The sorting process may be customized further by excluding certain fields from sorting on a per-type basis and by automatically populating the \bibfield{presort} field on a per-type basis.

\begin{ltxsyntax}

\cmditem{DeclareSortingScheme}[options]{name}{specification}

Defines the sorting scheme \prm{name}. The \prm{name} is the identifier passed to the \opt{sorting} option (\secref{use:opt:pre:gen}) when selecting the sorting scheme. The \cmd{DeclareSortingScheme} command supports the following optional arguments:

\begin{optionlist*}

\choitem{locale}{\prm{locale}}

The locale for the sorting scheme which then overrides the global sorting locale in the \opt{sortlocale} option discussed in \secref{use:opt:pre:gen}.

\end{optionlist*}

The \prm{specification} is an undelimited list of \cmd{sort} directives which specify the elements to be considered in the sorting process. Spaces, tabs, and line endings may be used freely to visually arrange the \prm{specification}. Blank lines are not permissible. This command may only be used in the preamble.

\cmditem{sort}{elements}

Specifies the elements considered in the sorting process. The \prm{elements} are an undelimited list of \cmd{field}, \cmd{literal}, and \cmd{citeorder} commands which are evaluated in the order in which they are given. If an element is defined, it is added to the sort key and the sorting routine skips to the next \cmd{sort} directive. If it is undefined, the next element is evaluated. Since literal strings are always defined, any \cmd{literal} commands should be the sole or the last element in a \cmd{sort} directive. All \prm{elements} should be the same datatype as described in \secref{bib:fld:dat} since they will be potentially compared to any of the other \prm{elements} in other entries.. The \cmd{sort} command supports the following optional arguments:

\begin{optionlist*}

\choitem{locale}{\prm{locale}}

Override the locale used for sorting at the level of a particular set of sorting elements. If specified, the locale overrides the locale set at the level of \cmd{DeclareSortingScheme} and also the global setting. See also the discussion of the global sorting locale option \opt{sortlocale} in \secref{use:opt:pre:gen}.

\choitem[ascending]{direction}{ascending, descending}

The sort direction, which may be either \texttt{ascending} or \texttt{descending}. The default is ascending order.

\boolitem[false]{final}

This option marks a \cmd{sort} directive as the final one in the \prm{specification}. If one of the \prm{elements} is available, the remainder of the \prm{specification} will be ignored. The short form \opt{final} is equivalent to \kvopt{final}{true}.

\boolitem{sortcase}

Whether or not to sort case"=sensitively. The default setting depends on the global \opt{sortcase} option.

\boolitem{sortupper}

Whether or not to sort in <uppercase before lowercase> (\texttt{true}) or <lowercase before uppercase> order (\texttt{false}). The default setting depends on the global \opt{sortupper} option.

\end{optionlist*}

\cmditem{field}[key=value, \dots]{field}

The \cmd{field} element adds a \prm{field} to the sorting specification. If the \prm{field} is undefined, the element is skipped. The \cmd{field} command supports the following optional arguments:

\begin{optionlist*}

\choitem[left]{padside}{left, right}

Pads a field on the \texttt{left} or \texttt{right} side using \opt{padchar} so that its width is \opt{padwidth}. If no padding option is set, no padding is done at all. If any padding option is specified, then padding is performed and the missing options are assigned built-in default values. If padding and substring matching are both specified, the substring match is performed first.

\intitem[4]{padwidth}

The target width in characters.

\valitem[0]{padchar}{character}

The character to be used when padding the field.

\choitem[left]{strside}{left, right}

Performs a substring match on the \texttt{left} or \texttt{right} side of the field. The number of characters to match is specified by the corresponding \texttt{strwidth} option. If no substring option is set, no substring matching is performed at all. If any substring option is specified, then substring matching is performed and the missing options are assigned built-in default values. If padding and substring matching are both specified, the substring match is performed first.

\intitem[4]{strwidth}

The number of characters to match.

\end{optionlist*}

\cmditem{literal}{string}

The \cmd{literal} element adds a literal \prm{string} to the sorting specification. This is useful as a fallback if some fields are not available.

\csitem{citeorder}

The \cmd{citeorder} element has a special meaning. It requests a sort based on the lexical order of the actual citations. For entries cited within the same citation command like:

\begin{ltxexample}
\cite{one,two}
\end{ltxexample}
%
there is a distinction between the lexical order and the semantic order. Here «one» and «two» have the same semantic order but a unique lexical order. The semantic order only matters if you specify further sorting to disambiguate entries with the same semantic order. For example, this is the definition of the \opt{none} sorting scheme:

\begin{ltxexample}
\DeclareSortingScheme{none}{
  \sort{\citeorder}
}
\end{ltxexample}
%
This sorts the bibliography purely lexically by the order of the keys in the citation commands. In the example above, it sorts «one» before «two». However, suppose that you consider «one» and «two» to have the same order (semantic order) since they are cited at the same time and want to further sort these by year. Suppose «two» has an earlier \bibfield{year} than «one»:

\begin{ltxexample}
\DeclareSortingScheme{noneyear}{
  \sort{\citeorder}
  \sort{<<year>>}
}
\end{ltxexample}
%
This sorts «two» before «one», even though lexically, «one» would sort before «two». This is possible because the semantic order can be disambiguated by the further sorting on year. With the standard \opt{none} sorting scheme, the lexical order and semantic order are identical because there is nothing further to disambiguate them. This means that you can use \cmd{citeorder} just like any other sorting specification element, choosing how to further sort entries cited at the same time (in the same citation command).

\cmditem{DeclareSortingNamekeyScheme}[schemename]{specification}

Defines how the sorting keys for names are constructed. This can change the sorting order of names arbitrarily because you can choose how to put together the name parts when constructing the string to compare when sorting. The sorting key construction scheme so defined is called \prm{schemename} which defaults to «global» if this optional parameter is absent. When constructing the sorting key for a name, a sorting key for each name part is constructed and the key for each name is formed into an ordered key list with a special internal separator. The point of this option is to accommodate languages or situations where sorting of names needs to be customised (for example, Icelandic names are sometimes sorted by given names rather than by family names). This macro may be used multiple times to define schemes with different names which can then be referred to later. Sorting name key schemes can have the following scopes, in order of increasing precedence:

\begin{itemize}
\item The default scheme defined without the optional name argument
\item Given as the \opt{sortingnamekeyscheme} option to a reference context (see \secref{use:bib:context})
\item Given as a per-entry option \opt{sortnamekeyscheme} in a bibliography data source entry
\item Given as a per-namelist option \opt{sortnamekeyscheme}
\item Given as a per-name option \opt{sortnamekeyscheme}
\end{itemize}

By default there is only a global scheme which has the following \prm{specification}:

\begin{ltxexample}
\DeclareSortingNamekeyScheme{
  \keypart{
    \namepart[use=true]{<<prefix>>}
  }
  \keypart{
    \namepart{<<family>>}
  }
  \keypart{
    \namepart{<<given>>}
  }
  \keypart{
    \namepart{<<suffix>>}
  }
  \keypart{
    \namepart[use=false]{<<prefix>>}
  }
}
\end{ltxexample}
%
This means that the key is constructed by concatenating, in order, the name prefix (only if the \opt{useprefix} option is true), the family name(s), the given names(s), the name suffix and then the name prefix (only if the \opt{useprefix} option is false).

\cmditem{keypart}{part}

\prm{part} is an ordered list of of \cmd{namepart} and \cmd{literal} specifications which are concatenated together when constructing a part of the name sorting key.

\cmditem{literal}{string}

A literal string to insert into the name sorting key.

\cmditem{namepart}{name}

Specifies the \prm{name} of a namepart to use in constructing the name sorting key.

\begin{optionlist*}

\boolitem[true]{use}

Indicates that the namepart \prm{name} is only to be used in this concatenation position if the corresponding \opt{use<name>} option is set to the specified boolean value.

\boolitem[true]{inits}

Indicates that only the initials of namepart \prm{name} are to be used in
constructing the sorting specification.

\end{optionlist*}

\end{ltxsyntax}

As an example, suppose you wanted to be able to sort names by given name rather than family name, you could define a sorting name key scheme like this:

\begin{ltxexample}
\DeclareSortingNamekeyScheme[givenfirst]{
  \keypart{
    \namepart{<<given>>}
  }
  \keypart{
    \namepart[use=true]{<<prefix>>}
  }
  \keypart{
    \namepart{<<family>>}
  }
  \keypart{
    \namepart[use=false]{<<prefix>>}
  }
}
\end{ltxexample}
%
You can then use the name \opt{givenfirst} at the appropriate scope in order to make \biber use this scheme when constructing sorting name keys. For example, you could enable this for one bibliography list like this:

\begin{ltxexample}
\begin{refcontext}[sortnamekeyscheme=givenfirst]
\printbibliography
\end{refcontext}
\end{ltxexample}
%
or perhaps you only want to do this for a particular entry:

\begin{lstlisting}[style=bibtex]{}
@BOOK{key,
  OPTIONS = {sortnamekeyscheme=givenfirst},
  AUTHOR = {Arnar Vigfusson}
}
\end{lstlisting}
%
or just a name list by using the option as a pseudo-name which will be ignored:

\begin{lstlisting}[style=bibtex]{}
@BOOK{key,
  AUTHOR = {sortnamekeyscheme=givenfirst and Arnar Vigfusson}
}
\end{lstlisting}
%
or just a single name by passing the option as part of the extended name information
format which \biber supports (see \biber doc):

\begin{lstlisting}[style=bibtex]{}
@BOOK{key,
  AUTHOR = {given=Arnar, family=Vigfusson, sortnamekeyscheme=givenfirst}
}
\end{lstlisting}
%
Now we give some examples of sorting schemes. In the first example, we define a simple name\slash title\slash year scheme. The name element may be either the \bibfield{author}, the \bibfield{editor}, or the \bibfield{translator}. Given this specification, the sorting routine will use the first element which is available and continue with the \bibfield{title}. Note that the options \opt{use$<$name$>$} options are considered automatically in the sorting process:

\begin{ltxexample}
\DeclareSortingScheme{sample}{
  \sort{
    \field{<<author>>}
    \field{<<editor>>}
    \field{<<translator>>}
  }
  \sort{
    \field{<<title>>}
  }
  \sort{
    \field{<<year>>}
  }
}
\end{ltxexample}
%
In the next example, we define the same scheme in a more elaborate way, considering special fields such as \bibfield{presort}, \bibfield{sortkey}, \bibfield{sortname}, etc. Since the \bibfield{sortkey} field specifies the master sort key, it needs to override all other elements except for \bibfield{presort}. This is indicated by the \opt{final} option. If the \bibfield{sortkey} field is available, processing will stop at this point. If not, the sorting routine continues with the next \cmd{sort} directive. This setup corresponds to the default definition of the \texttt{nty} scheme:

\begin{ltxexample}
\DeclareSortingScheme{nty}{
  \sort{
    \field{<<presort>>}
  }
  \sort[final]{
    \field{<<sortkey>>}
  }
  \sort{
    \field{<<sortname>>}
    \field{<<author>>}
    \field{<<editor>>}
    \field{<<translator>>}
    \field{<<sorttitle>>}
    \field{<<title>>}
  }
  \sort{
    \field{<<sorttitle>>}
    \field{<<title>>}
  }
  \sort{
    \field{<<sortyear>>}
    \field{<<year>>}
  }
}
\end{ltxexample}
%
Finally, here is an example of a sorting scheme which overrides the global sorting locale and additionally overrides again when sorting by the \bibfield{origtitle} field. Note the use in the scheme-level override of a babel/polyglossia language name instead of a real locale identifier. \biber will map this to a suitable, real locale identifier (in this case, \texttt{sv\_SE}):

\begin{ltxexample}
\DeclareSortingScheme[locale=swedish]{custom}{
  \sort{
    \field{<<sortname>>}
    \field{<<author>>}
    \field{<<editor>>}
    \field{<<translator>>}
    \field{<<sorttitle>>}
    \field{<<title>>}
  }
  \sort[locale=de_DE_phonebook]{
    \field{<<origtitle>>}
  }
}
\end{ltxexample}

\begin{ltxsyntax}

\cmditem{DeclareSortExclusion}{entrytype, \dots}{field, \dots}

Specifies fields to be excluded from sorting on a per-type basis. The \prm{entrytype} argument and the \prm{field} argument may be a comma"=separated list of values. A blank \prm{field} argument will clear all exclusions for this \prm{entrytype}. A value of <*> for \prm{entrytype} will exclude \prm{field,\dots} for every entrytype. This is equivalent to simply deleting the field from the sorting specification and is only normally used in combination with \cmd{DeclareSortInclusion} when one wishes to exclude a field for all but explicitly included entrytypes. See example in \cmd{DeclareSortInclusion} below. This command may only be used in the preamble.

\cmditem{DeclareSortInclusion}{entrytype, \dots}{field, \dots}

Only used along with \cmd{DeclareSortExclusion}. Specifies fields to be included in sorting on a per-type basis. This allows the user to exclude a field from sorting for all entrytypes and then to override this for certain entrytypes. This is easier sometimes than using \cmd{DeclareSortExclusion} to list exclusions for many entrytypes. The \prm{entrytype} argument and the \prm{field} argument may be a comma"=separated list of values. This command may only be used in the preamble. For example, this would use \bibfield{title} during sorting only for \bibtype{article}s:

\begin{ltxexample}
\DeclareSortExclusion{*}{title}
\DeclareSortInclusion{article}{title}
\end{ltxexample}

\cmditem{DeclarePresort}[entrytype, \dots]{string}

Specifies a string to be used to automatically populate the \bibfield{presort} field of entries without a \bibfield{presort} field. The \bibfield{presort} may be defined globally or on a per-type basis. If the optional \prm{entrytype} argument is given, the \prm{string} applies to the respective entry type. If not, it serves as the global default value. Specifying an \prm{entrytype} in conjunction with a blank \prm{string} will clear the type-specific setting. The \prm{entrytype} argument may be a comma"=separated list of values. This command may only be used in the preamble.

\cmditem{DeclareSortTranslit}[entrytype]{specification}

Languages which can be written in different scripts or alphabets often only have CLDR sorting tailoring for one script and it is expected that you transliterate into the supported script for sorting purposes. A common example is Sanskrit which is often written in academic contexts in IAST
romanised script but which needs to be sorted in the <sa> locale which expects the Devanāgarī script. This means that it is necessary to transliterate into the sorting script internally. \cmd{DeclareSortTranslit} declares which parts of an entry you would like to transliterate for sorting purposes. Without the \prm{entrytype} parameter, the \prm{specification} applies to all entrytypes. The \prm{specification} is one or more \cmd{translit} commands:

\cmditem{translit}{field or fieldset}{from}{to}

Specifies that the data field \bibfield{field} or all fields in a fieldset \prm{fieldset} declared with \cmd{DeclareDatafieldSet} (see \secref{aut:ctm:dsets}) should be transliterated from script \prm{from} to script \prm{to} for sorting purposes. The field/set argument can also be <*> to apply transliteration to all fields. The valid \prm{from} and \prm{to} values are given in table \ref{tab:translit}. Note that \biblatex does not aim to support general transliteration, only those which are useful for sorting purposes. Please open a GitHub ticket for \biblatex\ if you think you need additional transliterations.

An example of transliterating titles so that they sort correctly in Sanskrit:

\begin{ltxexample}
\DeclareDatafieldSet{settitles}{
  \member[field=title]
  \member[field=booktitle]
  \member[field=eventtitle]
  \member[field=issuetitle]
  \member[field=journaltitle]
  \member[field=maintitle]
  \member[field=origtitle]
}

\DeclareSortTranslit{
  \translit[settitles]{iast}{devanagari}
}
\end{ltxexample}

\end{ltxsyntax}

\begin{table}
\tablesetup\centering
\begin{tabular}{lll}
\toprule
\sffamily\bfseries\spotcolor From
  & \sffamily\bfseries\spotcolor To
  & Description\\
\midrule
iast & devanagari & Sanskrit IAST\footnote{International Alphabet of Sanskrit Transliteration} to Devanāgarī\\
\bottomrule
\end{tabular}
\caption{Valid transliteration pairs}
\label{tab:translit}
\end{table}

\subsubsection[Bibliography List Filters]{Bibliography List Filters}
\label{aut:ctm:bibfilt}

When using customisable bibliography lists (See \secref{use:bib:biblist}), usually one wants to return in the \file{.bbl} only those entries which have the particular fields which the bibliography list is summarising. For example, when printing a normal list of shorthands, you want the list returned by \biber in the \file{.bbl} to contain only those entries which have a \bibfield{shorthand} field. This is accomplished by defining a bibliography list filter using the \cmd{DeclareBiblistFilter} command. This differs from the filters defined using \cmd{defbibfilter} (see \secref{use:bib:flt}) since the filters defined by \cmd{defbibfilter} run inside \biblatex after the \file{.bbl} has been generated. In addition, bibliography lists in the \file{.bbl} do not contain entry data, only the citation keys for the entries and so no filtering by \biblatex using \cmd{defbibfilter} is possible for bibliography lists.

\begin{ltxsyntax}
\cmditem{DeclareBiblistFilter}{name}{specification}

Defines a bibliography list filter with \prm{name}. The \prm{specification} consists of one or more \cmd{filter} or \cmd{filteror} macros, all of which must be satisfied for the entry to pass the filter:

\cmditem{filter}[filterspec]{filter}

Filter entries according to the \prm{filterspec} and \prm{filter}. \prm{filterspec} can be one of:

\end{ltxsyntax}

\begin{description}
\item[type/nottype] Entry is/is not of \bibfield{entrytype} \prm{filter}
\item[subtype/notsubtype] Entry is/is not of \bibfield{subtype} \prm{filter}
\item[keyword/notkeyword] Entry has/does not have \bibfield{keyword} \prm{filter}
\item[field/notfield] Entry has/does not have a field called \prm{filter}
\end{description}

\begin{ltxsyntax}
\cmditem{filteror}{type}{filters}

A wrapper around one or more \cmd{filter} commands specifying that they form a disjunctive set, i.e. any one of the \prm{filters} must be satisfied.

\end{ltxsyntax}

Fields in the datamodel which are marked as <Label fields> (see \secref{aut:ctm:dm}) automatically have a filter defined for them with the same name and which filters out any entries which do no contain the field. For example, \biblatex automatically generates a filter for the \bibfield{shorthand} field:

\begin{ltxexample}
\DeclareBiblistFilter{<<shorthand>>}{
  \filter[type=field,filter=shorthand]
}
\end{ltxexample}

\subsubsection{Controlling Name Initials Generation}
\label{aut:ctm:noinit}

Generating initials for name parts from a given name involves some difficulties when you have names with prefixes, diacritics, hyphens etc. Often, you want to ignore things like prefixes when generating initials so that the initials for «al-Hasan» is just «H» instead of «a-H». This is tricky when you also have names like «Ho-Pun» where you want the initials to be «H-P», for example.

\begin{ltxsyntax}

\cmditem{DeclareNoinit}{specification}

Defines regular expressions to strip from names before generating initials. The \prm{specification} is an undelimited list of \cmd{noinit} directives which specify the regular expressions to remove from the name. Spaces, tabs and line endings may be used freely to visually arrange the \prm{specification}. Blank lines are not permissible. This command may only be used in the preamble.

\cmditem{noinit}{regexp}

Any number of \cmd{noinit} commands can be given each of which specifies to remove the \prm{regexp} from the copy of the name which the initials generation system sees. Since regular expressions usually contain special characters, it is best to enclose them in the provided \cmd{regexp} macro as shown---this will pass the expression through to \biber correctly.

\end{ltxsyntax}

If there is no \cmd{DeclareNoinit} specification, \biber will default to:

\begin{ltxexample}
\DeclareNoinit{
  % strip lowercase prefixes like 'al-' when generating initials from names
  \noinit{\regexp{<<\b\p{Ll}{2}\p{Pd}>>}}
  % strip some common diacritics when generating initials from names
  \noinit{\regexp{<<[\x{2bf}\x{2018}]>>}}
}
\end{ltxexample}
%
This \biber default strips a couple of diacritics and also strips lowercase prefixes from names before generating initials.

\subsubsection{排序微调 Fine Tuning Sorting}
\label{aut:ctm:nosort}
对排序微调是很有用的,它可以忽略一些特殊域的某些部分。
It can be useful to fine tune sorting so that it ignores certain parts of particular fields.

\begin{ltxsyntax}

\cmditem{DeclareNosort}{specification}

Defines regular expressions to strip from particular fields or types of fields when sorting. The \prm{specification} is an undelimited list of \cmd{nosort} directives which specify the regular expressions to remove from particular fields or type of field. Spaces, tabs and line endings may be used freely to visually arrange the \prm{specification}. Blank lines are not permissible. This command may only be used in the preamble.

\cmditem{nosort}{field or field type}{regexp}

Any number of \cmd{nosort} commands can be given each of which specifies to remove the \prm{regexp} from the \prm{field} or \prm{field type}. A \prm{field type} is simple a convenience grouping of semantically similar fields from which you might want to remove a regexp. Table \ref{aut:nosort} shows the available field types. Since regular expressions usually contain special characters, it is best to enclose them in the provided \cmd{regexp} macro as shown---this will pass the expression through to \biber correctly.

\end{ltxsyntax}

The default is:

\begin{ltxexample}
\DeclareNosort{
  % strip prefixes like 'al-' when sorting names
  \nosort{type_names}{\regexp{<<\A\p{L}{2}\p{Pd}>>}}
  % strip some diacritics when sorting names
  \nosort{type_names}{\regexp{<<[\x{2bf}\x{2018}]>>}}
}
\end{ltxexample}
%
This \biber default strips a couple of diacritics and also strips prefixes from names when sorting. Suppose you wanted to ignore «The» at the beginning of a \bibfield{title} field when sorting:

\begin{ltxexample}
\DeclareNosort{
  \nosort{<<title>>}{\regexp{<<\AThe\s+>>}}
}
\end{ltxexample}
%
Or if you wanted to ignore «The» at the beginning of any title field:

\begin{ltxexample}
\DeclareNosort{
  \nosort{<<type_title>>}{\regexp{<<\AThe\s+>>}}
}
\end{ltxexample}

\begin{table}[h]
\tablesetup
\begin{tabular}{@{}V{0.5\textwidth}@{}V{0.5\textwidth}@{}}
\toprule
\multicolumn{1}{@{}H}{Field Type} &
\multicolumn{1}{@{}H}{Fields} \\
\cmidrule(r){1-1}\cmidrule{2-2}
|type_name| & |author| \\
             & |afterword| \\
             & |annotator| \\
             & |bookauthor| \\
             & |commentator| \\
             & |editor| \\
             & |editora| \\
             & |editorb| \\
             & |editorc| \\
             & |foreword| \\
             & |holder| \\
             & |introduction| \\
             & |namea| \\
             & |nameb| \\
             & |namec| \\
             & |shortauthor| \\
             & |shorteditor| \\
             & |translator| \\
|type_title| & |booktitle| \\
              & |eventtitle| \\
              & |issuetitle| \\
              & |journaltitle| \\
              & |maintitle| \\
              & |origtitle| \\
              & |title| \\
\bottomrule
\end{tabular}
\caption{Field types for \cmd{nosort}}
\label{aut:nosort}
\end{table}

\subsubsection{特殊域 Special Fields}
\label{aut:ctm:fld}

Some of the automatically generated fields from \secref{aut:bbx:fld:lab} may be customized.

\begin{ltxsyntax}

\cmditem{DeclareLabelname}[entrytype, \dots]{specification}

Defines the fields to consider when generating the \bibfield{labelname} field (see \secref{aut:bbx:fld:lab}). The \prm{specification} is an ordered list of \cmd{field} commands. The fields are checked in the order listed and the first field which is available will be used as \bibfield{labelname}. This is the default definition:

\begin{ltxexample}
\DeclareLabelname{%
  \field{shortauthor}
  \field{author}
  \field{shorteditor}
  \field{editor}
  \field{translator}
}
\end{ltxexample}
%
The \bibfield{labelname} field may be customized globally or on a per-type basis. If the optional \prm{entrytype} argument is given, the specification applies to the respective entry type. If not, it is applied globally. The \prm{entrytype} argument may be a comma"=separated list of values. This command may only be used in the preamble.

\cmditem{DeclareLabeldate}[entrytype, \dots]{specification}

Defines the date components to consider when generating \bibfield{labelyear}, \bibfield{labelmonth}, \bibfield{labelday}, \bibfield{labelendyear}, \bibfield{labelendmonth} and \bibfield{labelendday} fields (see \secref{aut:bbx:fld:lab}). The \prm{specification} is an ordered list of \cmd{field} or \cmd{literal} commands. The items are checked in the order listed and the first item which is available will be used to popluate the mentioned fields. Note that the \cmd{field} items do not have to be datetype <date> in the data model so that you can create pseudo-year labels by, for example, using a \bibfield{pubstate} field contents, if available, as the year label by defining \cmd{DeclareLabeldate} suitably. Note also that a \cmd{literal} command will always be used when found and so this should always be the last thing in the list. If the value of a \cmd{literal} command is a valid localisation string, then this will be resolved in the current language, otherwise the value is used as a literal string as-is. This is the default definition:

\begin{ltxexample}
\DeclareLabeldate{%
  \field{date}
  \field{year}
  \field{eventdate}
  \field{origdate}
  \field{urldate}
  \literal{nodate}
}
\end{ltxexample}
%
Note that the \bibfield{date} field is split by the backend into \bibfield{year}, \bibfield{month} which are also valid fields in the default data model. In order to support legacy data which directly sets \bibfield{year} and/or \bibfield{month}, the specification <\bibfield{date}> in \cmd{DeclareLabeldate} will also match \bibfield{year} and \bibfield{month} fields, if present.
The \bibfield{label*} fields may be customized globally or on a per-type basis. If the optional \prm{entrytype} argument is given, the specification applies to the respective entry type. If not, it is applied globally. The \prm{entrytype} argument may be a comma"=separated list of values. This command may only be used in the preamble. See also \secref{aut:bbx:fld:dat}.

\cmditem{DeclareLabeltitle}[entrytype, \dots]{specification}

Defines the fields to consider when generating the \bibfield{labeltitle} field (see \secref{aut:bbx:fld:lab}). The \prm{specification} is an ordered list of \cmd{field} commands. The fields are checked in the order listed and the first field which is available will be used as \bibfield{labeltitle}. This is the default definition:

\begin{ltxexample}
\DeclareLabeltitle{%
  \field{shorttitle}
  \field{title}
}
\end{ltxexample}
%
The \bibfield{labeltitle} field may be customized globally or on a per-type basis. If the optional \prm{entrytype} argument is given, the specification applies to the respective entry type. If not, it is applied globally. The \prm{entrytype} argument may be a comma"=separated list of values. This command may only be used in the preamble.

\end{ltxsyntax}

\subsubsection{数据继承 Data Inheritance (\bibfield{crossref})}
\label{aut:ctm:ref}

\biber features a highly customizable cross-referencing mechanism with flexible data inheritance rules. This sections deals with the configuration interface. See \apxref{apx:ref} for the default configuration. A note on terminology: the \emph{child} or \emph{target} is the entry with the \bibfield{crossref} field, the \emph{parent} or \emph{source} is the entry the \bibfield{crossref} field points to. The child inherits data from the parent.

\begin{ltxsyntax}

\cmditem{DefaultInheritance}[exceptions]{options}

Configures the default inheritance behavior. This command may only be used in the preamble. The default behavior may be customized be setting the following \prm{options}:

\begin{optionlist*}

\boolitem[true]{all} Whether or not to inherit all fields from the parent by default.

\kvopt{all}{true} means that the child entry inherits all fields from the parent, unless a more specific inheritance rule has been set up with \cmd{DeclareDataInheritance}. If an inheritance rule is defined for a field, data inheritance is controlled by that rule. \kvopt{all}{false} means that no data is inherited from the parent by default and each field to be inherited requires an explicit inheritance rule set up with \cmd{DeclareDataInheritance}. The package default is \kvopt{all}{true}.

\boolitem[false]{override} Whether or not to overwrite target fields with source fields if both are defined. This applies both to automatic inheritance and to explicit inheritance rules. The package default is \kvopt{override}{false}, \ie existing fields of the child entry are not overwritten.

\valitem{ignore}{csv list of uniqueness options}

This option takes a comma-separated list of one of more of <singletitle>, <uniquetitle>, <uniquebaretitle> and/or <uniquework>. The purpose of this option is to ignore tracking information for these three options when the field which would trigger the tracking (\tabref{use:opt:wu}) is inherited. An example---Suppose that you have several \bibtype{book} entries which all crossref a \bibtype{mvbook} from which they get their \bibfield{author} field. You might reasonably want the \cmd{ifsingletitle} test to return <true> for this author as their only <work> is the \bibtype{mvbook}. Similar comments would apply to situations involving the \cmd{ifuniquetitle}, \cmd{ifuniquebaretitle} and \cmd{ifuniquework} tests. The \opt{ignore} option lists which of these should have their tracking information ignored when the fields which would trigger them are inherited. The idea is that the presence of an inherited field does not contribute towards the determination of whether some combination of name/title is unique in the bibliographic data. For example, this modified default setting would ignore \opt{singletitle} and \opt{uniquetitle} tracking:

\begin{ltxexample}
\DefaultInheritance{ignore={singletitle,uniquetitle}, all=true, override=false}
\end{ltxexample}
%
Of course, the ignoring of tracking does nothing if the fields inherited do not play a role in tracking. Only the fields listed in \tabref{use:opt:wu} are relevant to this option.

\end{optionlist*}

The optional \prm{exceptions} are an undelimited list of \cmd{except} directives. Spaces, tabs, and line endings may be used freely to visually arrange the \prm{exceptions}. Blank lines are not permissible.

\cmditem{except}{source}{target}{options}

Defines an exception to the default inheritance rules.

\cmd{DeclareDataInheritance} sets the inheritance \prm{options} for a specific \prm{source} and \prm{target} combination. The \prm{source} and \prm{target} arguments specify the parent and the child entry type. The asterisk matches all types and is permissible in either argument.

\cmditem{DeclareDataInheritance}[options]{source, \dots}{target, \dots}{rules}

Declares inheritance rules. The \prm{source} and \prm{target} arguments specify the parent and the child entry type. Either argument may be a single entry type, a comma"=separated list of types, or an asterisk. The asterisk matches all entry types. The \prm{rules} are an undelimited list of \cmd{inherit} and\slash or \cmd{noinherit} directives. Spaces, tabs, and line endings may be used freely to visually arrange the \prm{rules}. Blank lines are not permissible. This command may only be used in the preamble. The options are:

\begin{optionlist*}

\valitem{ignore}{csv list of uniqueness options}

As the \opt{ignore} option on \cmd{DefaultInheritance} explained above. When set here, it takes precedence over any global options set with \cmd{DefaultInheritance}. For example, this would ignore \opt{singletitle} and \opt{uniquetitle} tracking for a \bibtype{book} inheriting from a \bibtype{mvbook}.

\begin{ltxexample}
\DeclareDataInheritance[ignore={singletitle,uniquetitle}]{mvbook}{book}{<<...>>}
\end{ltxexample}

\end{optionlist*}

\cmditem{inherit}[option]{source}{target}

Defines an inheritance rule by mapping a \prm{source} field to a \prm{target} field. \prm{option} can be one of

\begin{optionlist*}

\boolitem[false]{override}

As the \opt{override} option for \cmd{DefaultInheritance} explained above. When set here, it takes precedence over any global options set with \cmd{DefaultInheritance}.

\end{optionlist*}

\cmditem{noinherit}{source}

Unconditionally prevents inheritance of the \prm{source} field.

\csitem{ResetDataInheritance}

Clears all inheritance rules defined with \cmd{DeclareDataInheritance}. This command may only be used in the preamble.

\end{ltxsyntax}

Here are some practical examples:

\begin{ltxexample}
\DefaultInheritance{<<all=true>>,<<override=false>>}
\end{ltxexample}
%
This example shows how to configure the default inheritance behavior. The above settings are the package defaults.

\begin{ltxexample}
\DefaultInheritance[
  \except{<<*>>}{<<online>>}{<<all=false>>}
]{all=true,override=false}
\end{ltxexample}
%
This example is similar to the one above but adds one exception: entries of type \bibtype{online} will, by default, not inherit any data from any parent.

\begin{ltxexample}
\DeclareDataInheritance{<<collection>>}{<<incollection>>}{
  \inherit{<<title>>}{<<booktitle>>}
  \inherit{<<subtitle>>}{<<booksubtitle>>}
  \inherit{<<titleaddon>>}{<<booktitleaddon>>}
}
\end{ltxexample}
%
So far we have looked at setting up standard inheritance. For example, \kvopt{all}{true} means that the \bibfield{publisher} field of a source entry is copied to the \bibfield{publisher} field of the target entry. In some cases, however, asymmetric mappings are required. They are defined with \cmd{DeclareDataInheritance}. The above example sets up three typical rules for \bibtype{incollection} entries referencing a \bibtype{collection}. We map the \bibfield{title} and related fields of the source to the corresponding \bibfield{booktitle} fields of the target.

\begin{ltxexample}
\DeclareDataInheritance{<<mvbook,book>>}{<<inbook,bookinbook>>}{
  \inherit{<<author>>}{<<author>>}
  \inherit{<<author>>}{<<bookauthor>>}
}
\end{ltxexample}
%
This rule is an example of one-to-many mapping: it maps the \bibfield{author} field of the source to both the \bibfield{author} and the \bibfield{bookauthor} fields of the target in order to allow for compact \bibfield{inbook}\slash \bibfield{bookinbook} entries. The source may be either a \bibtype{mvbook} or a \bibtype{book} entry, the target either an \bibtype{inbook} or a \bibtype{bookinbook} entry.

\begin{ltxexample}
\DeclareDataInheritance{<<*>>}{<<inbook,incollection>>}{
  \noinherit{<<introduction>>}
}
\end{ltxexample}
%
This rule prevents inheritance of the \bibfield{introduction} field. It applies to all targets of type
\bibtype{inbook} or \bibtype{incollection}, regardless of the source entry type.

\begin{ltxexample}
\DeclareDataInheritance{<<*>>}{<<*>>}{
  \noinherit{<<abstract>>}
}
\end{ltxexample}
%
This rule, which applies to all entries, regardless of the source and target entry types, prevents inheritance of the \bibfield{abstract} field.

\begin{ltxexample}
\DefaultInheritance{all=true,override=false}
\ResetDataInheritance
\end{ltxexample}
%
This example demonstrates how to emulate traditional \bibtex's cross"=referencing mechanism. It enables inheritance by default, disables overwriting, and clears all other inheritance rules and mappings.

In a bibliography entry, you can give an option <noinherit> where the value
is a datafield set defined with \cmd{DeclareDatafieldSet}
(\secref{aut:ctm:dsets}). This will block inheritance of the fields in the
set on a per-entry basis. For example:

\begin{ltxexample}
\DeclareDatafieldSet{nobtitle}{
  \member[field=booktitle]
}
\end{ltxexample}

\begin{lstlisting}[style=bibtex]{}
@INBOOK{s1,
  OPTIONS  = {noinherit=nobtitle},
  TITLE    = {Subtitle},
  CROSSREF = {s2}
}

@BOOK{s2,
  TITLE = {Title}
}
\end{lstlisting}
%
Here, \bibfield{s1} will not inherit the \bibfield{TITLE} of \bibfield{s2}
as \bibfield{BOOKTITLE} as this is blocked by the datafield set given as
the value to the \opt{noinherit} option.
%
One important thing to note is that children will never inherit any dateparts of a given type if they already contain a datepart of that type. So, for example:

\begin{lstlisting}[style=bibtex]{}
@INBOOK{b1,
  DATE     = {2004-03-03},
  ORIGDATE = {2004-03},
  CROSSREF = {b2}
}

@BOOK{b2,
  DATE      = {2004-03-03/2005-08-09},
  ORIGDATE  = {2004-03/2005-08},
  EVENTDATE = {2004-03/2005-08},
}
\end{lstlisting}
%
Here, \bibfield{b1} will not inherit any of \bibfield{endyear}, \bibfield{endmonth}, \bibfield{endday}, \bibfield{origendyear} or \bibfield{origendmonth} as this would make a mess of its own dates. It will, given the inheritance defaults, inherit all of the \bibfield{event*} date parts.

\subsection{辅助命令}%Auxiliary Commands
\label{aut:aux}
本节的工具用来分析和保存参考文献数据而不是对其进行格式化或者打印。
%The facilities in this section are intended for analyzing and saving bibliographic data rather than formatting and printing it.

\subsubsection{数据命令}%Data Commands
\label{aut:aux:dat}
本节的命令允许以 low"=level 方式访问未格式化的参考文献数据。这些命令不是用来输出,而是用来将数据保存到临时宏中,可以用于下一步的比较。
%The commands in this section grant low"=level access to the unformatted bibliographic data. They are not intended for typesetting but rather for things like saving data to a temporary macro so that it may be used in a comparison later.

\begin{ltxsyntax}

\cmditem{thefield}{field}

展开为未格式化的\prm{field}。如果\prm{field}未定义那么展开为一个空字符串。
%Expands to the unformatted \prm{field}. If the \prm{field} is undefined, this command expands to an empty string.

\cmditem{strfield}{field}

类似于\cmd{thefield}命令,但其值经自动净化,以便安全的用于构成控制序列名。
%Similar to \cmd{thefield}, except that the field is automatically sanitized such that its value may safely be used in the formation of a control sequence name.

\cmditem{csfield}{field}

类似于\cmd{thefield}命令,但禁止展开
%Similar to \cmd{thefield}, but prevents expansion.

\cmditem{usefield}{command}{field}

执行\prm{command}命令使用未格式化的\prm{field}作为其参数
%Executes \prm{command} using the unformatted \prm{field} as its argument.

\cmditem{thelist}{literal list}

%Expands to the unformatted \prm{literal list}. If the list is undefined, this command expands to an empty string. Note that this command will dump the \prm{literal list} in the internal format used by this package. This format is not suitable for printing.
展开为未格式化的\prm{literal list}。如果list未定义那么展开为一个空字符串。注意该命令中将\prm{literal list}转存为本宏包使用的内部格式。这一格式不适合打印。

\cmditem{strlist}{literal list}

类似于\cmd{thelist},差别在于该命令能自动处理列表的内部表示,因此列表的值可以安全地用于控制序列名的构建。
%Similar to \cmd{thelist}, except that the list internal representation is automatically sanitized such that its value may safely be used in the formation of a control sequence name.

\cmditem{thename}{name list}

%Expands to the unformatted \prm{name list}. If the list is undefined, this command expands to an empty string. Note that this command will dump the \prm{name list} in the internal format used by this package. This format is not suitable for printing.
展开为未格式化的\prm{name list}。如果list未定义那么展开为一个空字符串。注意该命令中将\prm{name list}转存为本宏包使用的内部格式。这一格式不适合打印。

\cmditem{strname}{name list}

类似于\cmd{thename},差别在于该命令能自动处理列表的内部表示,因此列表的值可以安全地用于控制序列名的构建。
%Similar to \cmd{thename}, except that the name internal representation is automatically sanitized such that its value may safely be used in the formation of a control sequence name.

\cmditem{savefield}{field}{macro}
\cmditem*{savefield*}{field}{macro}

将未格式化的\prm{field}拷贝到一个\prm{macro}中。不带星的命令全局的定义\prm{macro},而带星的命令是局部定义。
%Copies an unformatted \prm{field} to a \prm{macro}. The regular variant of this command defines the \prm{macro} globally, the starred one works locally.

\cmditem{savelist}{literal list}{macro}
\cmditem*{savelist*}{literal list}{macro}

将未格式化的\prm{literal list}拷贝到一个\prm{macro}中。不带星的命令全局的定义\prm{macro},而带星的命令是局部定义。
%Copies an unformatted \prm{literal list} to a \prm{macro}. The regular variant of this command defines the \prm{macro} globally, the starred one works locally.

\cmditem{savename}{name list}{macro}
\cmditem*{savename*}{name list}{macro}

将未格式化的\prm{name list}拷贝到一个\prm{macro}中。不带星的命令全局的定义\prm{macro},而带星的命令是局部定义。
%Copies an unformatted \prm{name list} to a \prm{macro}. The regular variant of this command defines the \prm{macro} globally, the starred one works locally.

\cmditem{savefieldcs}{field}{csname}
\cmditem*{savefieldcs*}{field}{csname}

类似于\cmd{savefield}命令,当将控制序列名\prm{csname}(即没有斜杠)作为参数,而不是宏。
%Similar to \cmd{savefield}, but takes the control sequence name \prm{csname} (without a leading backslash) as an argument, rather than a macro name.

\cmditem{savelistcs}{literal list}{csname}
\cmditem*{savelistcs*}{literal list}{csname}

类似于\cmd{savelist}命令,当将控制序列名\prm{csname}(即没有斜杠)作为参数,而不是宏。
%Similar to \cmd{savelist}, but takes the control sequence name \prm{csname} (without a leading backslash) as an argument, rather than a macro name.

\cmditem{savenamecs}{name list}{csname}
\cmditem*{savenamecs*}{name list}{csname}

类似于\cmd{savename}命令,当将控制序列名\prm{csname}(即没有斜杠)作为参数,而不是宏。
%Similar to \cmd{savename}, but takes the control sequence name \prm{csname} (without a leading backslash) as an argument, rather than a macro name.

\cmditem{restorefield}{field}{macro}

从之前用\cmd{savefield}命令定义的\prm{macro}中将\prm{field}恢复回来。该域是在局部范围内恢复。
%Restores a \prm{field} from a \prm{macro} defined with \cmd{savefield} before. The field is restored within a local scope.

\cmditem{restorelist}{literal list}{macro}

从之前用\cmd{savelist}命令定义的\prm{macro}中将\prm{literal list}恢复回来。该list是在局部范围内恢复。
%Restores a \prm{literal list} from a \prm{macro} defined with \cmd{savelist} before. The list is restored within a local scope.

\cmditem{restorename}{name list}{macro}

从之前用\cmd{savename}命令定义的\prm{macro}中将\prm{name list}恢复回来。该list是在局部范围内恢复。
%Restores a \prm{name list} from a \prm{macro} defined with \cmd{savename} before. The list is restored within a local scope.

\cmditem{clearfield}{field}

在局部范围内清除\prm{field}。以这种方式清除的域对于后续的数据命令来说相当于没有定义。
%Clears the \prm{field} within a local scope. A field cleared this way is treated as undefined by subsequent data commands.

\cmditem{clearlist}{literal list}

在局部范围内清除\prm{literal list}。以这种方式清除的list对于后续的数据命令来说相当于没有定义。
%Clears the \prm{literal list} within a local scope. A list cleared this way is treated as undefined by subsequent data commands.

\cmditem{clearname}{name list}

在局部范围内清除\prm{name list}。以这种方式清除的list对于后续的数据命令来说相当于没有定义。
%Clears the \prm{name list} within a local scope. A list cleared this way is treated as undefined by subsequent data commands.

\end{ltxsyntax}

\subsubsection{独立判断命令}%Stand-alone Tests
\label{aut:aux:tst}
本节的命令是不同类型的stand"=alone判断命令,用于参考文献著录和标注样式中。
%The commands in this section are various kinds of stand"=alone tests for use in bibliography and citation styles.

\begin{ltxsyntax}

\cmditem{if$<$datetype$>$julian}{true}{false}

当日期<datetype>date因为\opt{julian}和\opt{gregorianstart}选项的设置转换为儒略历(Julian Calendar)时,展开为\prm{true}。
%Expands to \prm{true} if the date <datetype>date (\opt{date}, \opt{urldate}, \opt{eventdate} etc.) Was converted to the Julian Calendar due to the settings of the \opt{julian}and \opt{gregorianstart}  options.

\cmditem{ifdatejulian}{true}{false}

类似于\cmd{if$<$datetype$>$julian}但用于\cmd{mkbibdate*}格式化命令中(\secref{aut:fmt:lng}),在这些格式化命令中恰当使用的\cmd{if$<$datetype$>$julian}命令等价于该命令。
%As \cmd{if$<$datetype$>$julian} but for use in \cmd{mkbibdate*} formatting commands (\secref{aut:fmt:lng}) inside which the appropriate \cmd{if$<$datetype$>$julian} command is aliased to this command.

\cmditem{if$<$datetype$>$dateera}{era}{true}{false}

当日期<datetype>date(\opt{date}, \opt{urldate}, \opt{eventdate}等)指定了一个时区等于\prm{era},则展开为\prm{true},否则展开为\prm{false}。\biber 确认并在\file{.bbl}文件中传递的可用\prm{era}字符串是:
%Expands to \prm{true} if the date <datetype>date (\opt{date}, \opt{urldate}, \opt{eventdate} etc.) has an era specification equal to \prm{era} and \prm{false} otherwise.  The supported \prm{era} strings which \biber determines and passes in the \file{.bbl} are:

\begin{description}
\item[bce] BCE/BC era
\item[ce] CE/AD era
\end{description}

该命令用于确定是否打印\secref{aut:lng:key:dt}节的地址字符串。
%This command is useful for determining whether to print the location strings in \secref{aut:lng:key:dt}.

\cmditem{ifdateera}{era}{true}{false}

类似于\cmd{if$<$datetype$>$dateera},但用于\cmd{mkbibdate*}格式化命令(\secref{aut:fmt:lng}),在这些格式化命令中恰当使用的\cmd{if$<$datetype$>$dateera}命令等价于该命令。
%As \cmd{if$<$datetype$>$dateera} but for use in \cmd{mkbibdate*} formatting commands (\secref{aut:fmt:lng}) inside which the appropriate \cmd{if$<$datetype$>$dateera} command is aliased to this command.

\cmditem{if$<$datetype$>$datecirca}{true}{false}

当日期<datetype>date(\opt{date}, \opt{urldate}, \opt{eventdate}等)在数据源中具有一个<circa>标记时,则展开为\prm{true},否则展开为\prm{false}。参见\secref{bib:use:dat}。该命令用于确定是否打印\secref{aut:lng:key:dt}节中的字符串。
%Expands to \prm{true} if the date <datetype>date (\opt{date}, \opt{urldate}, \opt{eventdate} etc.) had a <circa> marker in the source and \prm{false} otherwise.  See \secref{bib:use:dat}. This command is useful for determining whether to print the location strings in \secref{aut:lng:key:dt}.

\cmditem{ifdatecirca}{true}{false}

类似于\cmd{if$<$datetype$>$datecirca},但用于\cmd{mkbibdate*}格式化命令(\secref{aut:fmt:lng}),在这些格式化命令中恰当使用的\cmd{if$<$datetype$>$datecirca} 命令等价于该命令。
%As \cmd{if$<$datetype$>$datecirca} but for use in \cmd{mkbibdate*} formatting commands (\secref{aut:fmt:lng}) inside which the appropriate \cmd{if$<$datetype$>$datecirca} command is aliased to this command.

\cmditem{if$<$datetype$>$dateuncertain}{true}{false}

当日期<datetype>date(\opt{date}, \opt{urldate}, \opt{eventdate}等)在数据源中具有一个不确定标记时,则展开为\prm{true},否则展开为\prm{false}。参见\secref{bib:use:dat}。该命令用于确定是否打印例如年份后的一个问号。
%Expands to \prm{true} if the date <datetype>date (\opt{date}, \opt{urldate}, \opt{eventdate} etc.) had an uncertainty marker in the source and \prm{false} otherwise.  See \secref{bib:use:dat}. This command is useful for determining whether to print, for example, a question mark after a year.

\cmditem{ifdateuncertain}{true}{false}

类似于\cmd{if$<$datetype$>$dateuncertain},但用于\cmd{mkbibdate*}格式化命令(\secref{aut:fmt:lng}),在这些格式化命令中恰当使用的\cmd{if$<$datetype$>$dateuncertain}命令等价于该命令。
%As \cmd{if$<$datetype$>$dateuncertain} but for use in \cmd{mkbibdate*} formatting commands (\secref{aut:fmt:lng}) inside which the appropriate \cmd{if$<$datetype$>$dateuncertain} command is aliased to this command.

\cmditem{ifenddateuncertain}{true}{false}

类似于\cmd{ifend$<$datetype$>$dateuncertain},但用于\cmd{mkbibdate*}格式化命令(\secref{aut:fmt:lng}),在这些格式化命令中恰当使用的\cmd{ifend$<$datetype$>$dateuncertain}命令等价于该命令。
%As \cmd{ifend$<$datetype$>$dateuncertain} but for use in \cmd{mkbibdate*} formatting commands (\secref{aut:fmt:lng}) inside which the appropriate \cmd{ifend$<$datetype$>$dateuncertain} command is aliased to this command.

\cmditem{ifcaselang}[language]{true}{false}

如果可选的\prm{language}是\cmd{DeclareCaseLangs}(见\secref{aut:aux:msc})声明的语言之一,展开为\prm{true},否则展开为\prm{false}。但可选参数不给出时,对\cmd{currentlang}值进行判断。
%Expands to \prm{true} if the the optional \prm{language} is one of those
%declared by \cmd{DeclareCaseLangs} (see \secref{aut:aux:msc}) and to
%\prm{false} otherwise. Without the optional argument, checks the current
%value of \cmd{currentlang}.

\cmditem{ifsortnamekeyscheme}{string}{true}{false}

如果\prm{string}等于范围排序名关键词格式名\footnote{the current in scope sorting name key scheme name待议}(\ref{aut:ctm:srt}),否则展开为\prm{false}。
%Expands to \prm{true} if the \prm{string} is equal to the current in scope sorting name key scheme name (see \ref{aut:ctm:srt}), and to \prm{false} otherwise.

\cmditem{iffieldundef}{field}{true}{false}

展开为\prm{true},如果\prm{field}未定义,否则展开为\prm{false}
%Expands to \prm{true} if the \prm{field} is undefined, and to \prm{false} otherwise.

\cmditem{iflistundef}{literal list}{true}{false}

展开为\prm{true},如果\prm{literal list}未定义,否则展开为\prm{false}
%Expands to \prm{true} if the \prm{literal list} is undefined, and to \prm{false} otherwise.

\cmditem{ifnameundef}{name list}{true}{false}

展开为\prm{true},如果\prm{name list}未定义,否则展开为\prm{false}
%Expands to \prm{true} if the \prm{name list} is undefined, and to \prm{false} otherwise.

\cmditem{iffieldsequal}{field 1}{field 2}{true}{false}

展开为\prm{true},如果\prm{field 1}和\prm{field 2}相等,否则展开为\prm{false}
%Expands to \prm{true} if the values of \prm{field 1} and \prm{field 2} are equal, and to \prm{false} otherwise.

\cmditem{iflistsequal}{literal list 1}{literal list 2}{true}{false}

展开为\prm{true},如果\prm{literal list 1}和\prm{literal list 2}相等,否则展开为\prm{false}
%Expands to \prm{true} if the values of \prm{literal list 1} and \prm{literal list 2} are equal, and to \prm{false} otherwise.

\cmditem{ifnamesequal}{name list 1}{name list 2}{true}{false}

展开为\prm{true},如果\prm{name list 1}和\prm{name list 2}相等,否则展开为\prm{false}
%Expands to \prm{true} if the values of \prm{name list 1} and \prm{name list 2} are equal, and to \prm{false} otherwise.

\cmditem{iffieldequals}{field}{macro}{true}{false}

展开为\prm{true},如果\prm{field}的值和\prm{macro}的定义相等,否则展开为\prm{false}。\footnote{可以用于改进gb7714-2015中的新闻和标准的判断}
%Expands to \prm{true} if the value of the \prm{field} is equal to the definition of \prm{macro}, and to \prm{false} otherwise.

\cmditem{iflistequals}{literal list}{macro}{true}{false}

展开为\prm{true},如果\prm{literal list}的值和\prm{macro}的定义相等,否则展开为\prm{false}。
%Expands to \prm{true} if the value of the \prm{literal list} is equal to the definition of \prm{macro}, and to \prm{false} otherwise.

\cmditem{ifnameequals}{name list}{macro}{true}{false}

展开为\prm{true},如果\prm{name list}的值和\prm{macro}的定义相等,否则展开为\prm{false}。
%Expands to \prm{true} if the value of the \prm{name list} is equal to the definition of \prm{macro}, and to \prm{false} otherwise.

\cmditem{iffieldequalcs}{field}{csname}{true}{false}

类似于\cmd{iffieldequals},但将控制序列名\prm{csname}(不带斜杠)作为参数,而不是一个宏名。
%Similar to \cmd{iffieldequals} but takes the control sequence name \prm{csname} (without a leading backslash) as an argument, rather than a macro name.

\cmditem{iflistequalcs}{literal list}{csname}{true}{false}

类似于\cmd{iflistequals},但将控制序列名\prm{csname}(不带斜杠)作为参数,而不是一个宏名。
%Similar to \cmd{iflistequals} but takes the control sequence name \prm{csname} (without a leading backslash) as an argument, rather than a macro name.

\cmditem{ifnameequalcs}{name list}{csname}{true}{false}

类似于\cmd{ifnameequals},但将控制序列名\prm{csname}(不带斜杠)作为参数,而不是一个宏名。
%Similar to \cmd{ifnameequals} but takes the control sequence name \prm{csname} (without a leading backslash) as an argument, rather than a macro name.

\cmditem{iffieldequalstr}{field}{string}{true}{false}

展开为\prm{true},如果\prm{field}的值和字符串\prm{string}的定义相等,否则展开为\prm{false}。该命令是鲁棒的。
%Executes \prm{true} if the value of the \prm{field} is equal to \prm{string}, and \prm{false} otherwise. This command is robust.

\cmditem{iffieldxref}{field}{true}{false}

如果一个条目定义了\bibfield{crossref}\slash \bibfield{xref},该命令检测\prm{field}是否与cross"=referenced父条目相关联。如果子条目的\prm{field}与父条目对应的\prm{field}相等,那么执行\prm{true},否则执行\prm{false}。如果\bibfield{crossref}\slash \bibfield{xref}未定义,总是执行\prm{false}。该命令是鲁棒的。\bibfield{crossref}和 \bibfield{xref}域的描述见\secref{bib:fld:spc},更多关于cross"=referencing的信息见\secref{bib:cav:ref}。
%If the \bibfield{crossref}\slash \bibfield{xref} field of an entry is defined, this command checks if the \prm{field} is related to the cross"=referenced parent entry. It executes \prm{true} if the \prm{field} of the child entry is equal to the corresponding \prm{field} of the parent entry, and \prm{false} otherwise. If the \bibfield{crossref}\slash \bibfield{xref} field is undefined, it always executes \prm{false}. This command is robust. See the description of the \bibfield{crossref} and \bibfield{xref} fields in \secref{bib:fld:spc} as well as \secref{bib:cav:ref} for further information concerning cross"=referencing.

\cmditem{iflistxref}{literal list}{true}{false}

类似于\cmd{iffieldxref}命令,但检测\prm{literal list}是否与cross"=referenced父条目相关联。
\bibfield{crossref}和 \bibfield{xref}域的描述见\secref{bib:fld:spc},更多关于cross"=referencing的信息见\secref{bib:cav:ref}。
%Similar to \cmd{iffieldxref} but checks if a \prm{literal list} is related to the cross"=referenced parent entry. See the description of the \bibfield{crossref} and \bibfield{xref} fields in \secref{bib:fld:spc} as well as \secref{bib:cav:ref} for further information concerning cross"=referencing.

\cmditem{ifnamexref}{name list}{true}{false}

类似于\cmd{iffieldxref}命令,但检测\prm{name list}是否与cross"=referenced父条目相关联。
\bibfield{crossref}和 \bibfield{xref}域的描述见\secref{bib:fld:spc},更多关于cross"=referencing的信息见\secref{bib:cav:ref}。
%Similar to \cmd{iffieldxref} but checks if a \prm{name list} is related to the cross"=referenced parent entry. See the description of the \bibfield{crossref} and \bibfield{xref} fields in \secref{bib:fld:spc} as well as \secref{bib:cav:ref} for further information concerning cross"=referencing.

\cmditem{ifcurrentfield}{field}{true}{false}

执行\prm{true},如果当前域为\prm{field},否则执行\prm{false}。该命令是鲁棒的。它主要用于域格式指令中,如果在其它环境中总是执行\prm{false}。
%Executes \prm{true} if the current field is \prm{field}, and \prm{false} otherwise. This command is robust. It is intended for use in field formatting directives and always executes \prm{false} when used in any other context.

\cmditem{ifcurrentlist}{literal list}{true}{false}

执行\prm{true},如果当前list为\prm{literal list},否则执行\prm{false}。该命令是鲁棒的。它主要用于域格式指令中,如果在其它环境中总是执行\prm{false}。
%Executes \prm{true} if the current list is \prm{literal list}, and \prm{false} otherwise. This command is robust. It is intended for use in list formatting directives and always executes \prm{false} when used in any other context.

\cmditem{ifcurrentname}{name list}{true}{false}

执行\prm{true},如果当前list为\prm{name list},否则执行\prm{false}。该命令是鲁棒的。它主要用于域格式指令中,如果在其它环境中总是执行\prm{false}。
%Executes \prm{true} if the current list is \prm{name list}, and \prm{false} otherwise. This command is robust. It is intended for use in list formatting directives and always executes \prm{false} when used in any other context.

\cmditem{ifuseprefix}{true}{false}

执行\prm{true},如果\opt{useprefix}选项打开(无论是全局的还是针对当前条目),否则执行\prm{false}。该选项的细节见\secref{use:opt:bib}。
%Expands to \prm{true} if the \opt{useprefix} option is enabled (either globally or for the current entry), and \prm{false} otherwise. See \secref{use:opt:bib} for details on this option.

\cmditem{ifuseauthor}{true}{false}

这只是下面的\cmd{ifuse$<$name$>$}宏的一个特例,因为\bibfield{author}是默认数据模型的一部分所以放到这里来说。执行\prm{true},如果\opt{useauthor}选项打开(无论是全局的还是针对当前条目),否则执行\prm{false}。该选项的细节见\secref{use:opt:bib}。
%This is just a particular case of the \cmd{ifuse$<$name$>$} macro below but is mentioned here as \bibfield{author} is part of the default data model. Expands to \prm{true} if the \opt{useauthor} option is enabled (either globally or for the current entry), and \prm{false} otherwise. See \secref{use:opt:bib} for details on this option.

\cmditem{ifuseeditor}{true}{false}

这只是下面的\cmd{ifuse$<$name$>$}宏的一个特例,因为\bibfield{editor}是默认数据模型的一部分所以放到这里来说。执行\prm{true},如果\opt{useeditor}选项打开(无论是全局的还是针对当前条目),否则执行\prm{false}。该选项的细节见\secref{use:opt:bib}。
%This is just a particular case of the \cmd{ifuse$<$name$>$} macro below but is mentioned here as \bibfield{editor} is part of the default data model. Expands to \prm{true} if the \opt{useeditor} option is enabled (either globally or for the current entry), and \prm{false} otherwise. See \secref{use:opt:bib} for details on this option.

\cmditem{ifusetranslator}{true}{false}

这只是下面的\cmd{ifuse$<$name$>$}宏的一个特例,因为\bibfield{translator}是默认数据模型的一部分所以放到这里来说。执行\prm{true},如果\opt{usetranslator}选项打开(无论是全局的还是针对当前条目),否则执行\prm{false}。该选项的细节见\secref{use:opt:bib}。
%This is just a particular case of the \cmd{ifuse$<$name$>$} macro below but is mentioned here as \bibfield{translator} is part of the default data model. Expands to \prm{true} if the \opt{usetranslator} option is enabled (either globally or for the current entry), and \prm{false} otherwise. See \secref{use:opt:bib} for details on this option.

\cmditem{ifuse$<$name$>$}{true}{false}

%Expands to \prm{true} if the \opt{use$<$name$>$} option is enabled (either globally or for the current entry), and \prm{false} otherwise. See \secref{use:opt:bib} for details on this option.
展开为\prm{true},如果选项\opt{use$<$name$>$}打开(无论全局还是当前条目的选项),否则展开为\prm{false}。这一选项的细节详见第\secref{use:opt:bib}节。

\cmditem{ifcrossrefsource}{true}{false}

展开为\prm{true},如果包含在\file{.bbl}中的条目的间接引用(referenced)\footnote{应该是交叉引用次数}次数大于\opt{mincrossrefs},否则展开为\prm{false}。见\secref{use:opt:pre:gen}。如果条目被直接引用则展开为\prm{false}。
%Expands to \prm{true} if the entry was inclued in the \file{.bbl} due to being referenced more than \opt{mincrossrefs} times and false otherwise. See \secref{use:opt:pre:gen}. Also expands to false if the entry was directly cited.

\cmditem{ifxrefsource}{true}{false}

展开为\prm{true},如果包含在\file{.bbl}中的条目的间接引用(referenced)\footnote{应该是交叉引用次数}次数大于\\opt{minxrefs},否则展开为\prm{false}。见\secref{use:opt:pre:gen}。如果条目被直接引用则展开为\prm{false}。
%Expands to \prm{true} if the entry was inclued in the \file{.bbl} due to being referenced more than \opt{minxrefs} times and false otherwise. See \secref{use:opt:pre:gen}. Also expands to false if the entry was directly cited.

\cmditem{ifsingletitle}{true}{false}

%Expands to \prm{true} if there is only one work by the \opt{labelname} name in the bibliography, and to \prm{false} otherwise. If \opt{labelname} is not set for an entry, this will always expand to \prm{false}. Note that this feature needs to be enabled explicitly with the package option \opt{singletitle}.
展开为\prm{true},如果文献表中只有以\opt{labelname}为名的一片文献,否则展开为\prm{false}。如果没有\opt{labelname}为名的条目,当文献表中有以\opt{labeltitle}为题的文献则展开为\prm{true},否则展开为\prm{false}。如果条目既没设置\opt{labelname}也没设置\opt{labeltitle},总是展开为\prm{false}。注意该功能需要显式的打开宏包选项\opt{singletitle}才行。

\cmditem{ifuniquetitle}{true}{false}

展开为\prm{true},如果只有一篇文献的题名是\opt{labeltitle},否则展开为\prm{false}。如果条目的\opt{labeltitle}未设置也展开为\prm{false}。注意:要使用这一功能需要显式地打开包选项\opt{uniquetitle}。
%Expands to \prm{true} if there is only one work with the title \opt{labeltitle} and to \prm{false} otherwise. If \opt{labeltitle} is not set for an entry, this will always expand to \prm{false}. Note that this feature needs to be enabled explicitly with the package option \opt{uniquetitle}.

\cmditem{ifuniquebaretitle}{true}{false}

展开为\prm{true},如果\bibfield{labelname}域为空且只有一篇文献的题名是\opt{labeltitle},否则展开为\prm{false}。如果条目的\opt{labeltitle}未设置也展开为\prm{false}。注意:要使用这一功能需要显式地打开包选项\opt{uniquebaretitle}。
%Expands to \prm{true} if \bibfield{labelname} is empty and there is only one work with the title \opt{labeltitle} and to \prm{false} otherwise. If \opt{labeltitle} is not set for an entry, this will always expand to \prm{false}. Note that this feature needs to be enabled explicitly with the package option \opt{uniquebaretitle}.

\cmditem{ifuniquework}{true}{false}

展开为\prm{true},如果文献表中只有一篇文献的标签名是\opt{labelname}且题名是\opt{labeltitle},否则展开为\prm{false}。如果条目的\opt{labelname}和\opt{labeltitle}均未设置也展开为\prm{false}。注意:要使用这一功能需要显式地打开包选项\opt{uniquework}。如果同一条目的\bibfield{singletitle}和\bibfield{uniquetitle}都是false,可能是因为其他条目也有相同的\bibfield{labelname}或者\bibfield{labeltitle}。\bibfield{uniquework}可以让我们知道有另一条目具有相同的\bibfield{labelname}和\bibfield{labeltitle}。这对于一种多人合作的情况很有用,当多个同时维护参考文献数据源时,有可能会添加内容相同但引用关键词不同的文献。这一判断能帮助找到这中存在副本情况。
%Expands to \prm{true} if there is only one work by the \opt{labelname} name with the \opt{labeltitle} title in the bibliography, and to \prm{false} otherwise. If neither \opt{labelname} nor \opt{labeltitle} are set for an entry, this will always expand to \prm{false}. Note that this feature needs to be enabled explicitly with the package option \opt{uniquework}. If both \bibfield{singletitle} and \bibfield{uniquetitle} are false for the same entry, this could be because another entry has the same \bibfield{labdlname} and yet another, different, entry has the same \bibfield{labeltitle}. \bibfield{uniquework} would let you know that there is another entry that has \emph{both} the same \bibfield{labelname} \emph{and} the same \bibfield{labeltitle}. This could be helpful in cases where multiple people maintain bibliography datasources and there is a risk of adding the same work with different keys without other parties realising this. This test could help to find such duplicates.

\cmditem{ifuniqueprimaryauthor}{true}{false}

展开为\prm{true},如果一篇文献的对于其\opt{labelname}的第一作者的姓是唯一的,否则展开为\prm{false}。如果条目的\opt{labelname}未设置,将展开为\prm{false}。注意使用该功能需要显式的打开包选项\opt{uniqueprimaryauthor}。
%Expands to \prm{true} if there is only one work by the primary (first) author family
%name of \opt{labelname} and to \prm{false} otherwise. If \opt{labelname} is not set for an entry, this will always expand to \prm{false}. Note that this feature needs to be enabled explicitly with the package option \opt{uniqueprimaryauthor}.

\cmditem{ifandothers}{list}{true}{false}

展开为\prm{true},如果\prm{list}已定义并且在\file{bib}文件中以关键词<\texttt{and others}>截短了,否则展开为\prm{false}。\prm{list}可以是literal或name列表。
%Expands to \prm{true} if the \prm{list} is defined and has been truncated in the \file{bib} file with the keyword <\texttt{and others}>, and to \prm{false} otherwise. The \prm{list} may be a literal list or a name list.

\cmditem{ifmorenames}{true}{false}

展开为\prm{true},如果当前姓名列表已经截短或将截短,否则展开为\prm{false}。该命令用于姓名列表的格式化指令中,在其它地方使用将展开为\prm{false}。该命令对当前列表执行与\cmd{ifandothers}判断一样的操作。如果判断结果为否,它将检测\cnt{listtotal}是否大于\cnt{liststop}。该命令用于格式化命令中用以决定是否需要在列表默认打印«and others» or «et al.»这样的标注。注意: 当需要检测实在列表中间或者末尾时,即\cnt{listcount}是否小于或等于\cnt{liststop},详见第\secref{aut:bib:dat}节。
%Expands to \prm{true} if the current name list has been or will be truncated, and to \prm{false} otherwise. This command is intended for use in formatting directives for name lists. It will always expand to \prm{false} when used elsewhere. This command performs the equivalent of an \cmd{ifandothers} test for the current list. If this test is negative, it also checks if the \cnt{listtotal} counter is larger than \cnt{liststop}. This command may be used in a formatting directive to decide if a note such as «and others» or «et al.» is to be printed at the end of the list. Note that you still need to check whether you are in the middle or at the end of the list, \ie whether \cnt{listcount} is smaller than or equal to \cnt{liststop}, see \secref{aut:bib:dat} for details.

\cmditem{ifmoreitems}{true}{false}

类似于\cmd{ifmorenames},但检测literal列表。用于literal列表的格式化指令,其它地方用总是展开为\prm{false}。
%This command is similar to \cmd{ifmorenames} but checks the current literal list. It is intended for use in formatting directives for literal lists. It will always expand to \prm{false} when used elsewhere.

\cmditem{if$<$namepart$>$inits}{true}{false}

根据\opt{firstinits}包选项的状态,展开为\prm{true}或\prm{false}(见第\secref{use:opt:pre:int}节)。该命令用于姓名列表的格式化指令。
%Expands to \prm{true} or \prm{false}, depending on the state of the \opt{$<$namepart$>$inits} package option (see \secref{use:opt:pre:int}). This command is intended for use in formatting directives for name lists.

\cmditem{ifterseinits}{true}{false}

根据\opt{terseinits}包选项的状态,展开为\prm{true}或\prm{false}(见第\secref{use:opt:pre:int}节)。该命令用于姓名列表的格式化指令。
%Expands to \prm{true} or \prm{false}, depending on the state of the \opt{terseinits} package option (see \secref{use:opt:pre:int}). This command is intended for use in formatting directives for name lists.

\cmditem{ifentrytype}{type}{true}{false}

如果当前处理条目类型是\prm{type},则展开为\prm{true},否则展开为\prm{false}。
%Executes \prm{true} if the entry type of the entry currently being processed is \prm{type}, and \prm{false} otherwise.

\cmditem{ifkeyword}{keyword}{true}{false}

如果\prm{keyword}能在当前处理的条目的\bibfield{keywords}域中找到,展开为\prm{true},否则展开为\prm{false}。
%Executes \prm{true} if the \prm{keyword} is found in the \bibfield{keywords} field of the entry currently being processed, and \prm{false} otherwise.

\cmditem{ifentrykeyword}{entrykey}{keyword}{true}{false}

当条目关键词作为\cmd{ifkeyword}命令参数的变化形式,在判断当前处理条目是否是某一条目时很有用。
%A variant of \cmd{ifkeyword} which takes an entry key as its first argument. This is useful for testing an entry other than the one currently processed.

\cmditem{ifcategory}{category}{true}{false}

执行\prm{true},如果当前正在处理条目被指派为由\cmd{addtocategory}命令定义的\prm{category}中,否则执行\prm{false}。
%Executes \prm{true} if the entry currently being processed has been assigned to a \prm{category} with \cmd{addtocategory}, and \prm{false} otherwise.

\cmditem{ifentrycategory}{entrykey}{category}{true}{false}

当条目关键词作为\cmd{ifcategory}命令参数时的变化形式,在判断当前处理条目是否是某一条目时很有用。
%A variant of \cmd{ifcategory} which takes an entry key as its first argument. This is useful for testing an entry other than the one currently processed.

\cmditem{ifciteseen}{true}{false}

展开为\prm{true},如果当前条目之前已经被引用过,否则展开为\prm{false}。该命令是鲁棒的,用于标注样式中。如果文档中有\env{refsection}环境,引用追踪是基于这些环境的。注意:引用追踪器需要显式的以包选项\opt{citetracker}打开,如果追踪器未打开,该命令总是展开为\prm{false}。另可参见第\secref{aut:aux:msc}节的\cmd{citetrackertrue}和\cmd{citetrackerfalse}开关。
%Executes \prm{true} if the entry currently being processed has been cited before, and \prm{false} otherwise. This command is robust and intended for use in citation styles. If there are any \env{refsection} environments in the document, the citation tracking is local to these environments. Note that the citation tracker needs to be enabled explicitly with the package option \opt{citetracker}. The behavior of this test depends on the mode the citation tracker is operating in, see \secref{use:opt:pre:int} for details. If the citation tracker is disabled, the test always yields \prm{false}. Also see the \cmd{citetrackertrue} and \cmd{citetrackerfalse} switches in \secref{aut:aux:msc}.

\cmditem{ifentryseen}{entrykey}{true}{false}

当条目关键词作为\cmditem{ifciteseen}命令参数时的变化形式。因为\prm{entrykey}先于判断展开,它也可以用来测试在\bibfield{xref}等域中的条目关键词。
%A variant of \cmd{ifciteseen} which takes an entry key as its first argument. Since the \prm{entrykey} is expanded prior to performing the test, it is possible to test for entry keys in a field such as \bibfield{xref}:

\begin{ltxexample}
\ifentryseen{<<\thefield{xref}>>}{true}{false}
\end{ltxexample}
%
除了一个额外参数,\cmd{ifentryseen}的操作类似于\cmd{ifciteseen}。
%Apart from the additional argument, \cmd{ifentryseen} behaves like \cmd{ifciteseen}.

\cmditem{ifentryinbib}{entrykey}{true}{false}

如果\prm{entrykey}出现当前文献表中,执行\prm{true},否则执行\prm{false}。该命令用于参考文献著录样式。
%Executes \prm{true} if the entry \prm{entrykey} appears in the current bibliography, and \prm{false} otherwise. This command is intended for use in bibliography styles.

\cmditem{iffirstcitekey}{true}{false}

如果当前处理条目是引用列表中的第一个,执行\prm{true},否则执行\prm{false}。该命令依赖于\cnt{citecount}, \cnt{citetotal}, \cnt{multicitecount} 和 \cnt{multicitetotal}计数器(见\secref{aut:fmt:ilc}),因此只能用于\cmd{DeclareCiteCommand}命令定义的标注命令的循环执行代码\prm{loopcode}中。
%Executes \prm{true} if the entry currently being processed is the first one in the citation list, and \prm{false} otherwise. This command relies on the \cnt{citecount}, \cnt{citetotal}, \cnt{multicitecount} and \cnt{multicitetotal} counters (\secref{aut:fmt:ilc}) and thus is intended for use only in the \prm{loopcode} of a citation command defined with \cmd{DeclareCiteCommand}.

\cmditem{iflastcitekey}{true}{false}

类似于\cmd{iffirstcitekey},但判断的是是否为引用列表中的最后一个。
%Similar \cmd{iffirstcitekey}, but executes \prm{true} if the entry currently being processed is the last one in the citation list, and \prm{false} otherwise.

\cmditem{ifciteibid}{true}{false}

如果当前处理条目于前一条相同,展开为\prm{true},否则展开为\prm{false}。该命令用于标注样式。如果有\env{refsection}环境,追踪器是基于这些环境的。注意:<ibidem>追踪器需要由\opt{ibidtracker}包选项显式的打开。该判断命令的运行方式与追踪器运行的模式相关,详见\secref{use:opt:pre:int}。如果追踪器未打开,总是展开为\prm{false}。另可参见\secref{aut:aux:msc}节的\cmd{citetrackertrue}和\cmd{citetrackerfalse}开关。
%Expands to \prm{true} if the entry currently being processed is the same as the last one, and to \prm{false} otherwise. This command is intended for use in citation styles. If there are any \env{refsection} environments in the document, the tracking is local to these environments. Note that the <ibidem> tracker needs to be enabled explicitly with the package option \opt{ibidtracker}. The behavior of this test depends on the mode the tracker is operating in, see \secref{use:opt:pre:int} for details. If the tracker is disabled, the test always yields \prm{false}. Also see the \cmd{citetrackertrue} and \cmd{citetrackerfalse} switches in \secref{aut:aux:msc}.

\cmditem{ifciteidem}{true}{false}

如果当前处理条目的责任者(即作者或编者)于前一条目的相同,展开为\prm{true},否则展开为\prm{false}。该命令用于标注样式。如果有\env{refsection}环境,追踪器是基于这些环境的。注意: <idem> 追踪器需要由\opt{idemtracker}包选项显式的打开。该判断命令的运行方式与追踪器运行的模式相关,详见\secref{use:opt:pre:int}。如果追踪器未打开,总是展开为\prm{false}。另可参见\secref{aut:aux:msc}节的\cmd{citetrackertrue}和\cmd{citetrackerfalse}开关。
%Expands to \prm{true} if the primary name (\ie the author or editor) in the entry currently being processed is the same as the last one, and to \prm{false} otherwise. This command is intended for use in citation styles. If there are any \env{refsection} environments in the document, the tracking is local to these environments. Note that the <idem> tracker needs to be enabled explicitly with the package option \opt{idemtracker}. The behavior of this test depends on the mode the tracker is operating in, see \secref{use:opt:pre:int} for details. If the tracker is disabled, the test always yields \prm{false}. Also see \cmd{citetrackertrue} and \cmd{citetrackerfalse} in \secref{aut:aux:msc}.

\cmditem{ifopcit}{true}{false}

该命令类似于\cmd{ifciteibid},但只要当前处理等条目的\emph{作者 或 编者 }与前一条目相同,则展开为\prm{true}。注意: <opcit> 追踪器需要由\opt{opcittracker}包选项显式的打开。该判断命令的运行方式与追踪器运行的模式相关,详见\secref{use:opt:pre:int}。如果追踪器未打开,总是展开为\prm{false}。另可参见\secref{aut:aux:msc}节的\cmd{citetrackertrue}和\cmd{citetrackerfalse}开关。
%This command is similar to \cmd{ifciteibid} except that it expands to \prm{true} if the entry currently being processed is the same as the last one \emph{by this author or editor}. Note that the <opcit> tracker needs to be enabled explicitly with the package option \opt{opcittracker}. The behavior of this test depends on the mode the tracker is operating in, see \secref{use:opt:pre:int} for details. If the tracker is disabled, the test always yields \prm{false}. Also see the \cmd{citetrackertrue} and \cmd{citetrackerfalse} switches in \secref{aut:aux:msc}.

\cmditem{ifloccit}{true}{false}

该命令类似于\cmd{ifopcit},但还要比较\prm{postnote}的参数,如果他们相同且是数值(\secref{aut:aux:tst}节的\cmd{ifnumerals}命令判断),则展开为\prm{true}。即:如果引文的页码与前一文献相同则展开为\texttt{true}。 注意: <loccit> 追踪器需要由\opt{loccittracker}包选项显式的打开。该判断命令的运行方式与追踪器运行的模式相关,详见\secref{use:opt:pre:int}。如果追踪器未打开,总是展开为\prm{false}。另可参见\secref{aut:aux:msc}节的\cmd{citetrackertrue}和\cmd{citetrackerfalse}开关。
%This command is similar to \cmd{ifopcit} except that it also compares the \prm{postnote} arguments and expands to \prm{true} only if they match and are numerical (in the sense of \cmd{ifnumerals} from \secref{aut:aux:tst}), \ie \cmd{ifloccit} will yield \texttt{true} if the citation refers to the same page cited before. Note that the <loccit> tracker needs to be enabled explicitly with the package option \opt{loccittracker}. The behavior of this test depends on the mode the tracker is operating in, see \secref{use:opt:pre:int} for details. If the tracker is disabled, the test always yields \prm{false}. Also see the \cmd{citetrackertrue} and \cmd{citetrackerfalse} switches in \secref{aut:aux:msc}.

\cmditem{iffirstonpage}{true}{false}

该命令的运行与\opt{pagetracker}包选项相关,如果选项设置成\texttt{page},当当前项是页中的第一项,展开为\prm{true},否则展开为\prm{false}。如果选项设置成\texttt{spread},当当前项是合页中的第一项,展开为\prm{true},否则展开为\prm{false}。如果选项未打开,总是展开为\prm{false}。根据所处环境不同,<item>可以是一个标注,或者参考文献表中的条目。注意该命令区分正文文本和脚注,例如,当在某页的第一个脚注中使用,即便是文中有一个标注且先于该脚注。另可参见\secref{aut:aux:msc}节的\cmd{pagetrackertrue}和\cmd{pagetrackerfalse}开关。
%The behavior of this command is responsive to the package option \opt{pagetracker}. If the option is set to \texttt{page}, it expands to \prm{true} if the current item is the first one on the page, and to \prm{false} otherwise. If the option is set to \texttt{spread}, it expands to \prm{true} if the current item is the first one on the double-page spread, and to \prm{false} otherwise. If the page tracker is disabled, this test always yields \prm{false}. Depending on the context, the <item> may be a citation or an entry in the bibliography or a bibliography list. Note that this test distinguishes between body text and footnotes. For example, if used in the first footnote on a page, it will expand to \prm{true} even if there is a citation in the body text prior to the footnote. Also see the \cmd{pagetrackertrue} and \cmd{pagetrackerfalse} switches in \secref{aut:aux:msc}.

\cmditem{ifsamepage}{instance 1}{instance 2}{true}{false}

如果两个引用实例位于同于页或者同一合页中,展开为\prm{true},否则为\prm{false}。一个引用实例可以是一个标注也可以是文献表中的条目。这些实例用\cnt{instcount}计数区分,见\secref{aut:fmt:ilc}。该命令的运行与\opt{pagetracker}包选项相关,如果选项设置成\texttt{spread},其本质是<if same spread>(是否同意合页)的判断。如果选项未打开,总是展开为\prm{false}。参数\prm{instance 1}和\prm{instance 2}以\etex's \cmd{numexpr}方式当成整数表达式处理。这意味着可以在参数中计算。比如:
%This command expands to \prm{true} if two instances of a reference are located on the same page or double-page spread, and to \prm{false} otherwise. An instance of a reference may be a citation or an entry in the bibliography or a bibliography list. These instances are identified by the value of the \cnt{instcount} counter, see \secref{aut:fmt:ilc}. The behavior of this command is responsive to the package option \opt{pagetracker}. If this option is set to \texttt{spread}, \cmd{ifsamepage} is in fact an <if same spread> test. If the page tracker is disabled, this test always yields \prm{false}. The arguments \prm{instance 1} and \prm{instance 2} are treated as integer expressions in the sense of \etex's \cmd{numexpr}. This implies that it is possible to make calculations within these arguments, for example:

\begin{ltxexample}
\ifsamepage{<<\value>>{instcount}}{<<\value>>{instcount}<<-1>>}{true}{false}
\end{ltxexample}
注意:\cmd{value}命令不是以\cmd{the}为前缀,在第二个参数中做了减法运算。如果\prm{instance 1} 或 \prm{instance 2}是无效数字(比如一个负值),总是展开为\prm{false}。也要注意该命令不区分正文和脚注。另可参见\secref{aut:aux:msc}节的\cmd{pagetrackertrue}和\cmd{pagetrackerfalse}开关。
%Note that \cmd{value} is not prefixed by \cmd{the} and that the subtraction is included in the second argument in the above example. If \prm{instance 1} or \prm{instance 2} is an invalid number (for example, a negative one), the test yields \prm{false}. Also note that this test does not distinguish between body text and footnotes. Also see the \cmd{pagetrackertrue} and \cmd{pagetrackerfalse} switches in \secref{aut:aux:msc}.

\cmditem{ifinteger}{string}{true}{false}

如果\prm{string}是一个正整数,展开为\prm{true},否则为\prm{false},该命令鲁棒。
%Executes \prm{true} if the \prm{string} is a positive integer, and \prm{false} otherwise. This command is robust.

\cmditem{ifnumeral}{string}{true}{false}

如果\prm{string}是一个阿拉伯或者罗马数字,展开为\prm{true},否则为\prm{false},该命令鲁棒。另可参见\secref{aut:aux:msc}节的\cmd{DeclareNumChars}和\cmd{NumCheckSetup}命令。
%Executes \prm{true} if the \prm{string} is an Arabic or Roman numeral, and \prm{false} otherwise. This command is robust. See also \cmd{DeclareNumChars} and \cmd{NumCheckSetup} in \secref{aut:aux:msc}.

\cmditem{ifnumerals}{string}{true}{false}

如果\prm{string}是一个阿拉伯或者罗马数字的范围或列表,展开为\prm{true},否则为\prm{false},该命令鲁棒。相比于\cmd{ifnumeral}命令,当参数像 «52--58», «14/15», «1,~3,~5»等时,该命令会执行\prm{true}。
另可参见\secref{aut:aux:msc}节的\cmd{DeclareNumChars},\cmd{NumCheckSetup},\cmd{DeclareRangeCommands}和 \cmd{NumCheckSetup}命令。
%Executes \prm{true} if the \prm{string} is a range or a list of Arabic or Roman numerals, and \prm{false} otherwise. This command is robust. In contrast to \cmd{ifnumeral}, it will also execute \prm{true} with arguments like «52--58», «14/15», «1,~3,~5», and so on. See also \cmd{DeclareNumChars}, \cmd{DeclareRangeChars}, \cmd{DeclareRangeCommands}, and \cmd{NumCheckSetup} in \secref{aut:aux:msc}.

\cmditem{ifpages}{string}{true}{false}

类似于\cmd{ifnumerals},但也考虑\secref{aut:aux:msc}节的\cmd{DeclarePageCommands}命令。
%Similar to \cmd{ifnumerals}, but also considers \cmd{DeclarePageCommands} from \secref{aut:aux:msc}.

\cmditem{iffieldint}{field}{true}{false}

类似于\cmd{ifinteger}命令,但使用\prm{field}的值而不是一个字符串,如果域未定义,执行\prm{false}。
%Similar to \cmd{ifinteger}, but uses the value of a \prm{field} rather than a literal string in the test. If the \prm{field} is undefined, it executes \prm{false}.

\cmditem{iffieldnum}{field}{true}{false}

类似于\cmd{ifnumeral}命令,但使用\prm{field}的值而不是一个字符串,如果域未定义,执行\prm{false}。
%Similar to \cmd{ifnumeral}, but uses the value of a \prm{field} rather than a literal string in the test. If the \prm{field} is undefined, it executes \prm{false}.

\cmditem{iffieldnums}{field}{true}{false}

类似于\cmd{ifnumerals}命令,但使用\prm{field}的值而不是一个字符串,如果域未定义,执行\prm{false}。\footnote{是否可以用来解析卷期的范围?}
%Similar to \cmd{ifnumerals}, but uses the value of a \prm{field} rather than a literal string in the test. If the \prm{field} is undefined, it executes \prm{false}.

\cmditem{iffieldpages}{field}{true}{false}

类似于\cmd{ifpages}命令,但使用\prm{field}的值而不是一个字符串,如果域未定义,执行\prm{false}。
%Similar to \cmd{ifpages}, but uses the value of a \prm{field} rather than a literal string in the test. If the \prm{field} is undefined, it executes \prm{false}.

\cmditem{ifbibstring}{string}{true}{false}

如果\prm{string}是已知的本地化关键词,展开为\prm{true},否则\prm{false}。默认定义的本地化字符串见\secref{aut:lng:key}。新的字符串可以用命令\cmd{NewBibliographyString}定义。
%Expands to \prm{true} if the \prm{string} is a known localisation key, and to \prm{false} otherwise. The localisation keys defined by default are listed in \secref{aut:lng:key}. New ones may be defined with \cmd{NewBibliographyString}.

\cmditem{ifbibxstring}{string}{true}{false}

类似于\cmd{ifbibstring},但\prm{string}是展开的。
%Similar to \cmd{ifbibstring}, but the \prm{string} is expanded.

\cmditem{iffieldbibstring}{field}{true}{false}

类似于\cmd{ifbibstring},但使用\prm{field}域的值而不是一个字符串,如果域未定义,执行\prm{false}。
%Similar to \cmd{ifbibstring}, but uses the value of a \prm{field} rather than a literal string in the test. If the \prm{field}  is undefined, it expands to \prm{false}.

\cmditem{ifdriver}{entrytype}{true}{false}

展开为\prm{true}如果\prm{entrytype}的驱动存在,否则为\prm{false}。
%Expands to \prm{true} if a driver for the \prm{entrytype} is available, and to \prm{false} otherwise.

\cmditem{ifcapital}{true}{false}

如果\biblatex 的标点追踪器将当前位置的本地化字符串大写,则执行\prm{true},否则执行\prm{false}。给命令在格式化指令中对于姓名的某一部分做有条件的大写处理时有用。
%Executes \prm{true} if \biblatex's punctuation tracker would capitalize a localisation string at the current location, and \prm{false} otherwise. This command is robust. It may be useful for conditional capitalization of certain parts of a name in a formatting directive.

\cmditem{ifcitation}{true}{false}

当处于标注中则展开为\prm{true},否则为\prm{false}。注意这一命令与其所在的最外层环境有关。比如,当由\cmd{DeclareCiteCommand}命令定义的标注命令执行一个由\cmd{DeclareBibliographyDriver}定义的驱动,则任何在该驱动中的\cmd{ifcitation}都会展开为\prm{true}。在\secref{aut:cav:mif}可以看到一个实例。
%Expands to \prm{true} when located in a citation, and to \prm{false} otherwise. Note that this command is responsive to the outermost context in which it is used. For example, if a citation command defined with \cmd{DeclareCiteCommand} executes a driver defined with \cmd{DeclareBibliographyDriver}, any \cmd{ifcitation} tests in the driver code will yield \prm{true}. See \secref{aut:cav:mif} for a practical example.

\cmditem{ifbibliography}{true}{false}

当处于文献表中则展开为\prm{true},否则为\prm{false}。注意这一命令与其所在的最外层环境有关。比如,当由\cmd{DeclareBibliographyDriver}命令定义的驱动执行一个由\cmd{DeclareCiteCommand}定义的标注,则任何在该标注中的\cmd{ifbibliography}都会展开为\prm{true}。在\secref{aut:cav:mif}可以看到一个实例。
%Expands to \prm{true} when located in a bibliography, and to \prm{false} otherwise. Note that this command is responsive to the outermost context in which it is used. For example, if a driver defined with \cmd{DeclareBibliographyDriver} executes a citation command defined with \cmd{DeclareCiteCommand}, any \cmd{ifbibliography} tests in the citation code will yield \prm{true}. See \secref{aut:cav:mif} for a practical example.

\cmditem{ifnatbibmode}{true}{false}

根据\secref{use:opt:ldt}的\opt{natbib}选项展开为\prm{true}或\prm{false}。
%Expands to \prm{true} or \prm{false} depending on the \opt{natbib} option from \secref{use:opt:ldt}.

\cmditem{ifciteindex}{true}{false}

根据\secref{use:opt:pre:gen}的\opt{indexing}选项展开为\prm{true}或\prm{false}。
%Expands to \prm{true} or \prm{false} depending on the \opt{indexing} option from \secref{use:opt:pre:gen}.

\cmditem{ifbibindex}{true}{false}

根据\secref{use:opt:pre:gen}的\opt{indexing}选项展开为\prm{true}或\prm{false}。
%Expands to \prm{true} or \prm{false} depending on the \opt{indexing} option from \secref{use:opt:pre:gen}.

\cmditem{iffootnote}{true}{false}

当处于脚注中时,展开为\prm{true},否则为\prm{false}。注意:在\env{minipage}中的脚注被认为正文的一部分。当处于页面底部的脚注中或者由\sty{endnotes}提供的endnotes中时,只会展开为\prm{true}。
%Expands to \prm{true} when located in a footnote, and to \prm{false} otherwise. Note that footnotes in \env{minipage} environments are considered to be part of the body text. This command will only expand to \prm{true} in footnotes a the bottom of the page and in endnotes as provided by the \sty{endnotes} package.

\cntitem{citecounter}

这一计数器表示当前处理条目在当前reference section中的引用次数。注意该功能需要以包选项\opt{citecounter}显式的打开。如果选项设置为\texttt{context},正文和脚注中的引用分别计数。这种情况下,\cnt{citecounter}记录其所在环境中的值。
%This counter indicates how many times the entry currently being processed is cited in the current reference section. Note that this feature needs to be enabled explicitly with the package option \opt{citecounter}. If the option is set to \texttt{context}, citations in the body text and in footnotes are counted separately. In this case, \cnt{citecounter} will hold the value of the context it is used in.

\cntitem{uniquename}
这一计数器用于\bibfield{labelname}列表。它以每个名字为基础进行设置。如果姓不同,它的值设置为0,当增加姓名的其它部分的首字母使得姓名能区分,则设置为1,如果需要完整的姓名才能区分,则设置为2。作者年值和作者标题值得标注格式需要这一信息来增加姓名的其它部分以对姓相同的作者进行引用。比如:当引用列表中有一个<John Doe>和一个<Edward Doe>,该计数器将设置为1。如果有有一个<John Doe>和一个<John Doe>,该计数器将设置为2。如果选项设置成\texttt{init}\slash \texttt{allinit}\slash \texttt{mininit},那么计数器将限制值最大为\texttt{1}。这对于标注样式不打印全名而使用首字母来区分姓名很有用。如果添加首字母还无法区分姓名,\cnt{uniquename}将设置为\texttt{0}。该功能需要以包选项\opt{uniquename}显式的打开。注意\cnt{uniquename} 是对\cmd{printnames}局部的,仅根据\bibfield{labelname}列表或其来源姓名列表(典型如\bibfield{author} 或\bibfield{editor})设置。它的值在任何正文中都是0,即它仅在处理姓名的格式化指令中计算,更多细节和实例见\secref{aut:cav:amb}。
%This counter refers to the \bibfield{labelname} list. It is set on a per-name basis. Its value is \texttt{0} if the base name (by default the <family> part of the name) is unique, \texttt{1} if adding the other parts of the name (as specified in the uniquename template defined by \cmd{DeclareUniquenameTemplate})as initials will make it unique, and \texttt{2} if the full name is required to disambiguate the name. This information is required by author-year and author-title citation schemes which add additional parts of the name when citing different authors with the same last name. For example, (given the default \cmd{DeclareUniquenameTemplate} definition) if there is one <John Doe> and one <Edward Doe> in the list of references, this counter will be set to \texttt{1}. If there is one <John Doe> and one <Jane Doe>, the value of the counter will be \texttt{2}. If the option is set to \texttt{init}\slash \texttt{allinit}\slash \texttt{mininit}, the counter will be limited to \texttt{1}. This is useful for citations styles which use initials to disambiguate names but never print the full name in citations. If adding the initials is not sufficient to disambiguate the name, \cnt{uniquename} will also be set to \texttt{0} for that name. This feature needs to be enabled explicitly with the package option \opt{uniquename}. Note that the \cnt{uniquename} counter is local to \cmd{printnames} and that it is only set for the \bibfield{labelname} list and to the name list \bibfield{labelname} has been derived from (typically \bibfield{author} or \bibfield{editor}). Its value is zero in any other context, i.e., it must be evaluated in the name formatting directives handling name lists. See \secref{aut:cav:amb} for further details and practical examples.

\cntitem{uniquelist}
该计数器用于\bibfield{labelname}列表。它以每个域为基础进行设置。它的值表示当使用\cnt{maxnames}\slash \cnt{minnames}自动将姓名列表截短后导致标注歧义时,消除歧义需要的最小姓名数。比如,有一篇作者是<Doe\slash Smith\slash Johnson>的文献和另一篇作者是<Doe\slash Edwards\slash Williams>的文献,设置\kvopt{maxnames}{1}将导致两篇的作者都是<Doe et al.>。这种情况下,两个条目的\bibfield{labelname}列表的\cnt{uniquelist}将设置成\texttt{2},因为至少需要两个名字来区分。注意\cnt{uniquelist}是对\cmd{printnames}命令局部的,仅根据\bibfield{labelname}列表或其来源姓名列表(典型如\bibfield{author} 或\bibfield{editor})设置。它的值在任何正文中都是0,即它仅在处理姓名的格式化指令中计算,如果该值存在,则\cmd{printnames}命令在处理姓名列表时将自动应用,即自动覆盖\cnt{maxnames}\slash \cnt{minnames}。该功能需要以包选项\opt{uniquelist}显式的打开。更多细节和实例见\secref{aut:cav:amb}。
%This counter refers to the \bibfield{labelname} list. It is set on a per-field basis. Its value indicates the number of names required to disambiguate the name list if automatic \cnt{maxnames}\slash \cnt{minnames} truncation would lead to ambiguous citations. For example, if there is one work by <Doe\slash Smith\slash Johnson> and another one by <Doe\slash Edwards\slash Williams>, setting \kvopt{maxnames}{1} would lead to <Doe et al.> in both cases. In this case, \cnt{uniquelist} would be set to \texttt{2} on the \bibfield{labelname} lists of both entries because at least the first two names are required to disambiguate them. Note that the \cnt{uniquelist} counter is local to \cmd{printnames} and that it is only set for the \bibfield{labelname} list and to the name list \bibfield{labelname} has been derived from (typically \bibfield{author} or \bibfield{editor}). Its value is zero in any other context. If available, the \cnt{uniquelist} value will be used automatically by \cmd{printnames} when processing the name list, \ie it will automatically override \cnt{maxnames}\slash \cnt{minnames}. This feature needs to be enabled explicitly with the package option \opt{uniquelist}. See \secref{aut:cav:amb} for further details and practical examples.

\cntitem{parenlevel}

圆括号和/或方括号的嵌套层级。该信息仅在\secref{use:opt:pre:int}的\opt{parentracker}选项打开的情况下提供。
%The current nesting level of parentheses and\slash or brackets. This information is only available if the \opt{parentracker} from \secref{use:opt:pre:int} is enabled.

\end{ltxsyntax}

\subsubsection{使用\cmd{ifboolexpr}和\cmd{ifthenelse}的判断}%Tests with \cmd{ifboolexpr} and \cmd{ifthenelse}
\label{aut:aux:ife}

第\secref{aut:aux:tst}节介绍的判断可以与\sty{etoolbox}宏包提供的\cmd{ifboolexpr}命令和\sty{ifthen}宏包提供的\cmd{ifthenelse}命令一同使用。这种情况下,其语法略有差异,判断命令的\prm{true}和\prm{false}参数自动省略,而直接传递给\cmd{ifboolexpr} 或 \cmd{ifthenelse}。注意,使用这些命令需要一些计算代价。如果不需要一些布尔操作,使用\secref{aut:aux:tst}节的stand"=alone判断命令更高效。
%The tests introduced in \secref{aut:aux:tst} may also be used with the \cmd{ifboolexpr} command provided by the \sty{etoolbox} package and the \cmd{ifthenelse} command provided by the \sty{ifthen} package. The syntax of the tests is slightly different in this case: the \prm{true} and \prm{false} arguments are omitted from the test itself and passed to the \cmd{ifboolexpr} or \cmd{ifthenelse} command instead. Note that the use of these commands implies some processing overhead. If you do not need any boolean operators, it is more efficient to use the stand"=alone tests from \secref{aut:aux:tst}.

\begin{ltxsyntax}

\cmditem{ifboolexpr}{expression}{true}{false}

该\sty{etoolbox}包命令允许进行包括布尔运算和编组的复杂判断。
%\sty{etoolbox} command which allows for complex tests with boolean operators and grouping:

\begin{lstlisting}[style=ifthen]{}
\ifboolexpr{ (
	       test {\ifnameundef{editor}}
	       and
	       not test {\iflistundef{location}}
	     )
	     or test {\iffieldundef{year}}
  }
  {...}
  {...}
\end{lstlisting}

\cmditem{ifthenelse}{tests}{true}{false}

该\sty{ifthen}包命令允许进行包括布尔运算和编组的复杂判断。
%\sty{ifthen} command which allows for complex tests with boolean operators and grouping:

\begin{lstlisting}[style=ifthen]{}
\ifthenelse{ \(
		\ifnameundef{editor}
		\and
		\not \iflistundef{location}
	     \)
	     \or \iffieldundef{year}
  }
  {...}
  {...}
\end{lstlisting}
%
\biblatex 提供的附加判断命令仅在标注命令和文献表中使用\cmd{ifboolexpr}或\cmd{ifthenelse}命令时可用。
%The additional tests provided by \biblatex are only available when \cmd{ifboolexpr} or \cmd{ifthenelse} are used in citation commands and in the bibliography.

\end{ltxsyntax}

\subsubsection{综合命令}%Miscellaneous Commands
\label{aut:aux:msc}

本节介绍参考文献著录和标注样式中使用的一些综合命令和小巧工具。
%The section introduced miscellaneous commands and little helpers for use in bibliography and citation styles.

\begin{ltxsyntax}

\cmditem{newbibmacro}{name}[arguments][optional]{definition}
\cmditem*{newbibmacro*}{name}[arguments][optional]{definition}

定义一个用于后面\cmd{usebibmacro}调用的宏。该命令的语法类似于\cmd{newcommand},除了\prm{name}可以包含一些数字或标点,但不以斜杠开头。可选参数\prm{arguments}是一个整数用于指定宏需要处理的参数数量。如果\prm{optional}给出,它指定了该宏的第一个参数的默认值,这第一个参数自动变成为可选参数。相比于\cmd{newcommand},当宏已经定义时,\cmd{newbibmacro}命令会给出一个警告信息,并自动转换为\cmd{renewbibmacro}命令。类似于\cmd{newcommand},该命令的常规形式在定义中使用\cmd{long}前缀,而带星的命令则没有。如果一个宏声明为long,它的参数可以包含\cmd{par}记号。提供\cmd{newbibmacro}和\cmd{renewbibmacro}命令是为了方便使用,样式作者也可以使用\cmd{newcommand} 或\cmd{def}。然而,需要注意,共享文件 \path{biblatex.def}中的绝大多数定义都是用\cmd{newbibmacro}定义的,因此,要使用和修改它们要用相应的方式处理。
%Defines a macro to be executed via \cmd{usebibmacro} later. The syntax of this command is very similar to \cmd{newcommand} except that \prm{name} may contain characters such as numbers and punctuation marks and does not start with a backslash. The optional argument \prm{arguments} is an integer specifying the number of arguments taken by the macro. If \prm{optional} is given, it specifies a default value for the first argument of the macro, which automatically becomes an optional argument. In contrast to \cmd{newcommand}, \cmd{newbibmacro} issues a warning message if the macro is already defined, and automatically falls back to \cmd{renewbibmacro}. As with \cmd{newcommand}, the regular variant of this command uses the \cmd{long} prefix in the definition while the starred one does not. If a macro has been declared to be long, it may take arguments containing \cmd{par} tokens. \cmd{newbibmacro} and \cmd{renewbibmacro} are provided for convenience. Style authors are free to use \cmd{newcommand} or \cmd{def} instead. However, note that most shared definitions found in \path{biblatex.def} are defined with \cmd{newbibmacro}, hence they must be used and modified accordingly.

\cmditem{renewbibmacro}{name}[arguments][optional]{definition}
\cmditem*{renewbibmacro*}{name}[arguments][optional]{definition}

类似于\cmd{newbibmacro},但用于重定义\prm{name}。相比于\cmd{newcommand},当宏未定义时,\cmd{renewbibmacro}命令给出一个警告信息,并自动转换为\cmd{newbibmacro}命令。
%Similar to \cmd{newbibmacro} but redefines \prm{name}. In contrast to \cmd{renewcommand}, \cmd{renewbibmacro} issues a warning message if the macro is undefined, and automatically falls back to \cmd{newbibmacro}.

\cmditem{providebibmacro}{name}[arguments][optional]{definition}
\cmditem*{providebibmacro*}{name}[arguments][optional]{definition}

类似于\cmd{newbibmacro},但仅在\prm{name}未定义时定义宏。该命令概念上类似于\cmd{providecommand}。
%Similar to \cmd{newbibmacro} but only defines \prm{name} if it is undefined. This command is similar in concept to \cmd{providecommand}.

\cmditem{usebibmacro}{name}
\cmditem*{usebibmacro*}{name}

该命令执行由\cmd{newbibmacro}定义的宏\prm{name}。如果宏带参数,只要简单的跟在\prm{name}后面即可。该命令的常规形式会处理\prm{name},而带星的命令不会。
%This command executes the macro \prm{name}, as defined with \cmd{newbibmacro}. If the macro takes any arguments, they are simply appended after \prm{name}. The regular variant of this command sanitizes
\prm{name} while the starred variant does not.

\cmditem{savecommand}{command}
\cmditem{restorecommand}{command}

这两个命令用来保存和恢复\prm{command},其中\prm{command}必须是以斜杠开头的命令。两个命令都在局部范围内起作用。它们主要用于本地化文件中。
%These commands save and restore any \prm{command}, which must be a command name starting with a backslash. Both commands work within a local scope. They are mainly provided for use in localisation files.

\cmditem{savebibmacro}{name}
\cmditem{restorebibmacro}{name}

这两个命令用来保存和恢复宏\prm{name},其中\prm{name}由\cmd{newbibmacro}定义的宏的标识。两个命令都在局部范围内起作用。它们主要用于本地化文件中。
%These commands save and restore the macro \prm{name}, where \prm{name} is the identifier of a macro defined with \cmd{newbibmacro}. Both commands work within a local scope. They are mainly provided for use in localisation files.

\cmditem{savefieldformat}[entry type]{format}
\cmditem{restorefieldformat}[entry type]{format}

这两个命令用来保存和恢复格式化指令\prm{format},其中\prm{format}由\cmd{DeclareFieldFormat}定义。两个命令都在局部范围内起作用。它们主要用于本地化文件中。
%These commands save and restore the formatting directive \prm{format}, as defined with \cmd{DeclareFieldFormat}. Both commands work within a local scope. They are mainly provided for use in localisation files.

\cmditem{savelistformat}[entry type]{format}
\cmditem{restorelistformat}[entry type]{format}

这两个命令用来保存和恢复格式化指令\prm{format},其中\prm{format}由\cmd{DeclareListFormat}定义。两个命令都在局部范围内起作用。它们主要用于本地化文件中。
%These commands save and restore the formatting directive \prm{format}, as defined with \cmd{DeclareListFormat}. Both commands work within a local scope. They are mainly provided for use in localisation files.

\cmditem{savenameformat}[entry type]{format}
\cmditem{restorenameformat}[entry type]{format}

这两个命令用来保存和恢复格式化指令\prm{format},其中\prm{format}由\cmd{DeclareNameFormat}定义。两个命令都在局部范围内起作用。它们主要用于本地化文件中。
%These commands save and restore the formatting directive \prm{format}, as defined with \cmd{DeclareNameFormat}. Both commands work within a local scope. They are mainly provided for use in localisation files.

\cmditem{ifbibmacroundef}{name}{true}{false}

如果参考文献宏\prm{name}未定义,展开为\prm{true}否则为\prm{false}。
%Expands to \prm{true} if the bibliography macro \prm{name} is undefined, and to \prm{false} otherwise.

\cmditem{iffieldformatundef}[entry type]{name}{true}{false}
\cmditem{iflistformatundef}[entry type]{name}{true}{false}
\cmditem{ifnameformatundef}[entry type]{name}{true}{false}

如果参考文献格式化指令\prm{format}未定义,展开为\prm{true}否则为\prm{false}。
%Expands to \prm{true} if the formatting directive \prm{format} is undefined, and to \prm{false}
otherwise.

\cmditem{usedriver}{code}{entrytype}

执行\prm{entrytype}类条目的参考文献驱动。在由\cmd{DeclareCiteCommand}定义的标注命令的\prm{loopcode}中调用该命令是打印类似于一个参考文献条目的完整标注的简单方法。诸如\cmd{newblock}等命令无法用于标注,自动省略。附加的初始化命令可以通过\prm{code}参数传递。该参数在一个编组内执行,这一编组用于运行相应驱动。注意: 该参数语法上是必须的,但可以留空。也要注意如果\opt{autolang}包选项打开的话,该命令会自动切换语言。
%Executes the bibliography driver for an \prm{entrytype}. Calling this command in the \prm{loopcode} of a citation command defined with \cmd{DeclareCiteCommand} is a simple way to print full citations similar to a bibliography entry. Commands such as \cmd{newblock}, which are not applicable in a citation, are disabled automatically. Additional initialization commands may be passed as the \prm{code} argument. This argument is executed inside the group in which \cmd{usedriver} runs the respective driver. Note that it is mandatory in terms of the syntax but may be left empty. Also note that this command will automatically switch languages if the \opt{autolang} package option is enabled.

\cmditem{bibhypertarget}{name}{text}

\sty{hyperref}的\cmd{hypertarget}命令的封套\footnote{wrapper译为包围器,封套,包套?}。\prm{name}是超链接锚的名字,\prm{text}的内容作为超链接锚,可以是任意可打印文字或代码。如果文档中存在\env{refsection}环境,\prm{name}是基于当前refsection环境。如果\opt{hyperref}包选项未打开或者\sty{hyperref}包未加载,该命令简单的传递\prm{text}变量。另可参见\secref{aut:fmt:ich}节的格式化指令\texttt{bibhypertarget}。
%A wrapper for \sty{hyperref}'s \cmd{hypertarget} command. The \prm{name} is the name of the anchor, the \prm{text} is arbitrary printable text or code which serves as an anchor. If there are any \env{refsection} environments in the document, the \prm{name} is local to the current environment. If the \opt{hyperref} package option is disabled or the \sty{hyperref} package has not been loaded, this command will simply pass on its \prm{text} argument. See also the formatting directive \texttt{bibhypertarget} in \secref{aut:fmt:ich}.

\cmditem{bibhyperlink}{name}{text}

\sty{hyperref}的\cmd{hyperlink}命令的包套。\prm{name}是由\cmd{bibhypertarget}定义的超链接锚的名字,\prm{text}的内容将转变成超链接,可以是任意可打印文字或代码。如果文档中存在\env{refsection}环境,\prm{name}是基于当前refsection环境。如果\opt{hyperref}包选项未打开或者\sty{hyperref}包未加载,该命令简单的传递\prm{text}变量。另可参见\secref{aut:fmt:ich}节的格式化指令\texttt{bibhyperlink}。
%A wrapper for \sty{hyperref}'s \cmd{hyperlink} command. The \prm{name} is the name of an anchor defined with \cmd{bibhypertarget}, the \prm{text} is arbitrary printable text or code to be transformed into a link. If there are any \env{refsection} environments in the document, the \prm{name} is local to the current environment. If the \opt{hyperref} package option is disabled or the \sty{hyperref} package has not been loaded, this command will simply pass on its \prm{text} argument. See also the formatting directive \texttt{bibhyperlink} in \secref{aut:fmt:ich}.

\cmditem{bibhyperref}[entrykey]{text}

将\prm{text}转变为指向参考文献表中的\prm{entrykey}(即某一条目)的内部链接。如果\prm{entrykey}省略,该命令使用当前正在处理的条目的引用关键词。该命令用于将标注转换为可点击的超链接,可以链接到参考文献表中的相应条目。链接目标由\biblatex 自动标记。如果文档中有多个文献表,链接目标将是所有文献表中第一个出现的\prm{entrykey}条目。如果文档中存在\env{refsection}环境,则超链接基于当前refsection环境。另可参见\secref{aut:fmt:ich}节的格式化指令\texttt{bibhyperref}。
%Transforms \prm{text} into an internal link pointing to \prm{entrykey} in the bibliography. If \prm{entrykey} is omitted, this command uses the key of the entry currently being processed. This command is employed to transform citations into clickable links pointing to the corresponding entry in the bibliography. The link target is marked automatically by \biblatex. If there are multiple bibliographies in a document, the target will be the first occurence of \prm{entrykey} in one of the bibliographies. If there are \env{refsection} environments, the links are local to the environment. See also the formatting directive \texttt{bibhyperref} in \secref{aut:fmt:ich}.

\cmditem{ifhyperref}{true}{false}

展开为\prm{true},如果\opt{hyperref}包选项已打开(意味着\sty{hyperref}包已加载),否则展开为\prm{false}。
%Expands to \prm{true} if the \opt{hyperref} package option is enabled (which implies that the \sty{hyperref} package has been loaded), and to \prm{false} otherwise.

\cmditem{docsvfield}{field}

类似于\sty{etoolbox}包的\cmd{docsvlist}命令,差别在于它的参数是一个域名。域的值将以一个comma"=separated(英文逗号分隔)的列表进行解析。如果\prm{field}为定义,该命令展开为空字符串。
%Similar to the \cmd{docsvlist} command from the \sty{etoolbox} package, except that it takes a field name as its argument. The value of this field is parsed as a comma"=separated list. If the \prm{field} is undefined, this command expands to an empty string.

\cmditem{forcsvfield}{handler}{field}

类似于\sty{etoolbox}包的\cmd{forcsvlist}命令,差别在于它的参数是一个域名。域的值将以一个comma"=separated(英文逗号分隔)的列表进行解析。如果\prm{field}为定义,该命令展开为空字符串
%Similar to the \cmd{forcsvlist} command from the \sty{etoolbox} package, except that it takes a field name as its argument. The value of this field is parsed as a comma"=separated list. If the \prm{field} is undefined, this command expands to an empty string.

\cmditem{MakeCapital}{text}

类似于\cmd{MakeUppercase},但仅将\prm{text}的第一个可打印字符转换为大写。注意:\cmd{MakeUppercase}命令的限制也适用于这一命令。即: \prm{text}中的所有命令必须是鲁棒的或者以\cmd{protect}为前缀,因为在大写操作中\prm{text}需要展开。除了Ascii字符和标准重音命令外,该命令也处理\sty{inputenc}包的活动字符和\sty{babel}包的缩略词。如果\prm{text}以一个控制序列开头,不做任何大写操作。该命令是鲁棒的。
%Similar to \cmd{MakeUppercase} but only converts the first printable character in \prm{text} to uppercase. Note that the restrictions that apply to \cmd{MakeUppercase} also apply to this command. Namely, all commands in \prm{text} must either be robust or prefixed with \cmd{protect} since the \prm{text} is expanded during capitalization. Apart from Ascii characters and the standard accent commands, this command also handles the active characters of the \sty{inputenc} package as well as the shorthands of the \sty{babel} package. If the \prm{text} starts with a control sequence, nothing is capitalized. This command is robust.

\cmditem{MakeSentenceCase}{text}
\cmditem*{MakeSentenceCase*}{text}

将\prm{text}参数转换为sentence case(句子模式),即字符串中的第一个单词首字母大写而剩下其他部分转换为小写。该命令是鲁棒的。带星号的命令与常规命令(不带星号)的差别在于它能考虑条目的语言,根据\bibfield{langid}域指定。只有当\bibfield{langid}未定义或者定为由\cmd{DeclareCaseLangs}命令(见后面)声明的某种语言时,它才将\prm{text}转换为句子模式。\footnote{默认情况下,如下语言支持转换: \texttt{american}, \texttt{british}, \texttt{canadian}, \texttt{english}, \texttt{australian}, \texttt{newzealand} as well as the aliases \texttt{USenglish} and \texttt{UKenglish}. 要扩展或修改该列表请使用\cmd{DeclareCaseLangs}命令。} 否则\prm{text}不做任何改变。推荐使用\cmd{MakeSentenceCase*}而不是常规命令。两个命令都支持\file{bib}文件的传统\bibtex 规范,即: 遇到任何以花括号包围的内容大小写都不作变化,例如:
%Converts its \prm{text} argument to sentence case, \ie the first word is capitalized and the remainder of the string is converted to lowercase. This command is robust. The starred variant differs from the regular version in that it considers the language of the entry, as specified in the \bibfield{langid} field. If the \bibfield{langid} field is defined and holds a language declared with \cmd{DeclareCaseLangs} (see below)\footnote{By default, converting to sentence case is enabled for the following language identifiers: \texttt{american}, \texttt{british}, \texttt{canadian}, \texttt{english}, \texttt{australian}, \texttt{newzealand} as well as the aliases \texttt{USenglish} and \texttt{UKenglish}. Use \cmd{DeclareCaseLangs} to extend or change this list.}, then the sentence case conversion is performed. If the \bibfield{langid} field is undefined, then the language list declared with \cmd{DeclareCaseLangs} is checked for the presence of the main document language derived from the \opt{language} option. If found, sentence case conversion is performed, if not, the \prm{text} is not altered in any way. It is recommended to use \cmd{MakeSentenceCase*} rather than the regular variant in formatting directives. Both variants support the traditional \bibtex convention for \file{bib} files that anything wrapped in a pair of curly braces is not modified when changing the case. For example:

\begin{ltxexample}
\MakeSentenceCase{an Introduction to LaTeX}
\MakeSentenceCase{an Introduction to {LaTeX}}
\end{ltxexample}
%
将得到:
%would yield:

\begin{lstlisting}[style=plain]{}
An introduction to latex
An introduction to LaTeX
\end{lstlisting}
%
在以传统\bibtex 方式设计的\file{bib}文件中,为阻止字母的case"=changing(大小写变化),将单个字母用花括号包围是一种相当常见的方法。
%In \file{bib} files designed with traditional \bibtex in mind, it has been fairly common to only wrap single letters in braces to prevent case"=changing:

\begin{lstlisting}[style=bibtex]{}
title = {An Introduction to {L}a{T}e{X}}
\end{lstlisting}
%
这种方式存在一个问题是括号会压缩被包围字母两侧的字距。最好的方式是如第一个例子所示的那样,将整个单词都包围起来。
%The problem with this convention is that the braces will suppress the kerning on both sides of the enclosed letter. It is preferable to wrap the entire word in braces as shown in the first example.

\cmditem{mkpageprefix}[pagination][postpro]{text}

该命令用于域格式化指令中,包括标注命令的\prm{postnote}参数和文献条目的\bibfield{pages}域的格式化。默认情况下,它将会解析\prm{text}参数,并且以<p.> or <pp.>做为前缀。可选参数\prm{pagination}保存指示pagination类型的域名,可以是\bibfield{pagination}或\bibfield{bookpagination},默认是\bibfield{pagination}。前缀与\prm{text}之间的间距可以通过重定义\cmd{ppspace}命令来调整。默认是一个不可断行的词内空格。详见\secref{bib:use:pag, use:cav:pag}。另可参见\cmd{DeclareNumChars}, \cmd{DeclareRangeChars}, \cmd{DeclareRangeCommands}, 和\cmd{NumCheckSetup}。可选参数\prm{postpro}指定了用于对\prm{text}后处理的宏。如果只给出一个可选参数,将作为\prm{pagination},下面是两个典型例子:
%This command is intended for use in field formatting directives which format the page numbers in the \prm{postnote} argument of citation commands and the \bibfield{pages} field of bibliography entries. It will parse its \prm{text} argument and prefix it with <p.> or <pp.> by default. The optional \prm{pagination} argument holds the name of a field indicating the pagination type. This may be either \bibfield{pagination} or \bibfield{bookpagination}, with \bibfield{pagination} being the default. The spacing between the prefix and the \prm{text} may be modified by redefining \cmd{ppspace}. The default is an unbreakable interword space. See \secref{bib:use:pag, use:cav:pag} for further details. See also \cmd{DeclareNumChars}, \cmd{DeclareRangeChars}, \cmd{DeclareRangeCommands}, and \cmd{NumCheckSetup}. The optional \prm{postpro} argument specifies a macro to be used for post-processing the \prm{text}. If only one optional argument is given, it is taken as \prm{pagination}. Here are two typical examples:

\begin{ltxexample}
\DeclareFieldFormat{postnote}{<<\mkpageprefix[pagination]{#1}>>}
\DeclareFieldFormat{pages}{<<\mkpageprefix[bookpagination]{#1}>>}
\end{ltxexample}
%
第一个例子中的可选参数\bibfield{pagination}可以省略。
%The optional argument \bibfield{pagination} in the first example is omissible.

\cmditem{mkpagetotal}[pagination][postpro]{text}

该命令类似于\cmd{mkpageprefix},差别在于它用于条目的\bibfield{pagetotal}域,即它将打印«123 pages»而不是«page 123»。可选参数\prm{pagination}默认是\bibfield{bookpagination}。在\prm{text}和后缀之间的间距可由对\cmd{ppspace}重定义进行调整。可选参数\prm{postpro}指定了用于对\prm{text}后处理的宏。如果只给出一个可选参数,将作为\prm{pagination},下面是一个典型例子:
%This command is similar to \cmd{mkpageprefix} except that it is intended for the \bibfield{pagetotal} field of bibliography entries, \ie it will print «123 pages» rather than «page 123». The optional \prm{pagination} argument defaults to \bibfield{bookpagination}. The spacing inserted between the pagination suffix and the \prm{text} may be modified by redefining the macro \cmd{ppspace}. The optional \prm{postpro} argument specifies a macro to be used for post-processing the \prm{text}. If only one optional argument is given, it is taken as \prm{pagination}. Here is a typical example:

\begin{ltxexample}
\DeclareFieldFormat{pagetotal}{<<\mkpagetotal[bookpagination]{#1}>>}
\end{ltxexample}
%
在本例中可选参数\bibfield{bookpagination}可省略。
%The optional argument \bibfield{bookpagination} is omissible in this case.

\begin{table}
\tablesetup\lnstyle
\begin{tabularx}{\textwidth}{@{}>{\ttfamily}X@{}p{0.25\textwidth}@{}p{0.25\textwidth}@{}p{0.25\textwidth}@{}}
\toprule
\multicolumn{1}{@{}H}{Input} &
\multicolumn{3}{@{}H}{Output} \\
\cmidrule(r){1-1}\cmidrule{2-4}
& \multicolumn{1}{@{}H}{\ttfamily mincomprange=10}
& \multicolumn{1}{@{}H}{\ttfamily mincomprange=100}
& \multicolumn{1}{@{}H}{\ttfamily mincomprange=1000} \\
\cmidrule(r){2-2}\cmidrule(r){3-3}\cmidrule{4-4}
11--15		& 11--5		& 11--15	& 11--15	\\
111--115	& 111--5	& 111--5	& 111--115	\\
1111--1115	& 1111--5	& 1111--5	& 1111--5	\\
\cmidrule{2-4}
& \multicolumn{1}{@{}H}{\ttfamily maxcomprange=1000}
& \multicolumn{1}{@{}H}{\ttfamily maxcomprange=100}
& \multicolumn{1}{@{}H}{\ttfamily maxcomprange=10} \\
\cmidrule(r){2-2}\cmidrule(r){3-3}\cmidrule{4-4}
1111--1115	& 1111--5	& 1111--5	& 1111--5	\\
1111--1155	& 1111--55	& 1111--55	& 1111--1155	\\
1111--1555	& 1111--555	& 1111--1555	& 1111--1555	\\
\cmidrule{2-4}
& \multicolumn{1}{@{}H}{\ttfamily mincompwidth=1}
& \multicolumn{1}{@{}H}{\ttfamily mincompwidth=10}
& \multicolumn{1}{@{}H}{\ttfamily mincompwidth=100} \\
\cmidrule(r){2-2}\cmidrule(r){3-3}\cmidrule{4-4}
1111--1115	& 1111--5	& 1111--15	& 1111--115	\\
1111--1155	& 1111--55	& 1111--55	& 1111--155	\\
1111--1555	& 1111--555	& 1111--555	& 1111--555	\\
\bottomrule
\end{tabularx}
\caption{\cmd{mkcomprange} setup}
\label{aut:aux:tab1}
\end{table}

\cmditem{mkcomprange}[postpro]{text}
\cmditem*{mkcomprange*}[postpro]{text}

该命令,用于域格式化指令,将\prm{text}参数解析为页码范围并且压缩这些范围。
扫描程序将\cmd{bibrangedash}和hyphens作为范围间隔符。支持范围列表以\cmd{bibrangessep}(\biber\footnote{\biber 总会将commas/semicolon(逗号或冒号)的多范围分隔符转换为 \cmd{bibrangessep} ,因此可以在样式中控制。})或commas/semicolon(BibTeX)分隔。如果因为某些原因需要隐藏来自list/range扫描程序的一个字符,那么可以将该字符或者整个字符串用花括号包围起来。可选参数\prm{postpro}指定了一个用于对\prm{text}进行后处理的宏。怎么使用该参数见\cmd{mkcomprange}命令。带星的命令的差别在于\prm{postpro}参数应用于列表的各项。例如:
%This command, which is intended for use in field formatting directives, will parse its \prm{text} argument for page ranges and compress them. For example, «125--129» may be formatted as «125--9». You may configure the behavior of \cmd{mkcomprange} by adjusting the \latex counters \cnt{mincomprange}, \cnt{maxcomprange}, and \cnt{mincompwidth}, as illustrated in \tabref{aut:aux:tab1}. The default settings are \texttt{10}, \texttt{100000}, and \texttt{1}, respectively. This means that the command tries to compress as much as possible by default. Use \cmd{setcounter} to adjust the parameters. The scanner recognizes \cmd{bibrangedash} and hyphens as range dashes. It will normalize the dash by replacing any number of consecutive hyphens with \cmd{bibrangedash}. Lists of ranges delimited with \cmd{bibrangessep} are also supported. The backend will normalise any comma or semi-colons surrounded by optional space by replacing them with \cmd{bibrangessep}. If you want to hide a character from the list/range scanner for some reason, wrap the character or the entire string in curly braces. The optional \prm{postpro} argument specifies a macro to be used for post-processing the \prm{text}. This is important if you want to combine \cmd{mkcomprange} with other formatting macros which also need to parse their \prm{text} argument, such as \cmd{mkpageprefix}. Simply nesting these commands will not work as expected. Use the \prm{postpro} argument to set up the processing chain as follows:

\begin{ltxexample}
\DeclareFieldFormat{postnote}{\mkcomprange[<<{>>\mkpageprefix[pagination]<<}>>]{#1}}
\end{ltxexample}
%
注意:\cmd{mkcomprange}命令首先处理,\cmd{mkpageprefix}则作为后处理器。也要注意\prm{postpro}被额外的一对花括号包围。这仅在特殊情况下需要,为阻止\latex 的可选参数扫描器与嵌套的方括号混淆。带星的命令与不带星命令的差别是它应用于值得列表,例如:
%Note that \cmd{mkcomprange} is executed first, using \cmd{mkpageprefix} as post-processor. Also note that the \prm{postpro} argument is wrapped in an additional pair of braces. This is only required in this particular case to prevent \latex's optional argument scanner from getting confused by the nested brackets. The starred version of this command differs from the regular one in the way the \prm{postpro} argument is applied to a list of values. For example:

\begin{ltxexample}
\mkcomprange[\mkpageprefix]{5, 123-129, 423-439}
\mkcomprange*[\mkpageprefix]{5, 123-129, 423-439}
\end{ltxexample}
%
将输出:
%will output:

\begin{ltxexample}
pp. 5, 123-9, 423-39
p. 5, pp. 123-9, pp. 423-39
\end{ltxexample}

\cmditem{mkfirstpage}[postpro]{text}
\cmditem*{mkfirstpage*}[postpro]{text}

该命令,用于域格式化指令,将\prm{text}参数解析为页码范围并且仅打印这些范围的起始页码。扫描程序将\cmd{bibrangedash}和hyphens作为范围间隔符。支持范围列表以\cmd{bibrangessep}(\biber\footnote{\biber 总会将commas/semicolon(逗号或冒号)的多范围分隔符转换为 \cmd{bibrangessep} ,因此可以在样式中控制。})或commas/semicolon(BibTeX)分隔。如果因为某些原因需要隐藏来自list/range扫描程序的一个字符,那么可以将该字符或者整个字符串用花括号包围起来。可选参数\prm{postpro}指定了一个用于对\prm{text}进行后处理的宏。怎么使用该参数见\cmd{mkcomprange}命令。带星的命令的差别在于\prm{postpro}参数应用于列表的各项。例如:
%This command, which is intended for use in field formatting directives, will parse its \prm{text} argument for page ranges and print the start page of the range only. The scanner recognizes \cmd{bibrangedash} and hyphens as range dashes. Lists of ranges delimited with \cmd{bibrangessep} are also supported. If you want to hide a character from the list/range scanner for some reason, wrap the character or the entire string in curly braces. The optional \prm{postpro} argument specifies a macro to be used for post-processing the \prm{text}. See \cmd{mkcomprange} on how to use this argument. The starred version of this command differs from the regular one in the way the \prm{postpro} argument is applied to a list of values. For example:

\begin{ltxexample}
\mkfirstpage[\mkpageprefix]{5, 123-129, 423-439}
\mkfirstpage*[\mkpageprefix]{5, 123-129, 423-439}
\end{ltxexample}
%
将输出:
%will output:

\begin{ltxexample}
pp. 5, 123, 423
p. 5, p. 123, p. 423
\end{ltxexample}

\cmditem{rangelen}{rangefield}
该命令将其参数解析为一个范围,并返回范围的长度。对于开口的范围将返回-1。
这可以作为样式中一些判断的一部分,例如将<f>作为只有两页的范围的后缀,比如范围<36-37>将打印<36f>。这可以通过命令\cmd{ifnumcomp}实现:

%Takes the name of a bibfield declared as a range field in the data model and returns the length of the range. This is calculated by \biber and can handle many special cases. It will return $-1$ for open ended ranges. Specifically \cmd{rangelen} can:

\begin{itemize}
\item Calculate the total of multiple ranges in the same field such as <1-10, 20-30>
\item Handle implicit ranges such as <22-4> and <130-33>
\item Handle roman numeral ranges in upper and lower case and consisting of both ASCII and Unicode roman numeral representations.
\end{itemize}
%
下面是一些例子:
%Here are some examples:

\begin{tabular}{ll}
pages = <10> & |\rangelen{pages}| returns '1'\\
pages = <10-15> & |\rangelen{pages}| returns '6'\\
pages = <10-15,47-53> & |\rangelen{pages}| returns '13'\\
pages = <10-> & |\rangelen{pages}| returns '-1'\\
pages = <-10> & |\rangelen{pages}| returns '-1'\\
pages = <48-9> & |\rangelen{pages}| returns '2'\\
pages = <172-77> & |\rangelen{pages}| returns '6'\\
pages = <i-vi> & |\rangelen{pages}| returns '6'\\
pages = <X-XX> & |\rangelen{pages}| returns '11'\\
pages = <ⅥⅠ-ⅻ> & |\rangelen{pages}| returns '6'\\
pages = <ⅥⅠ-ⅻ, 145-7, 135-39> & |\rangelen{pages}| returns '14'
\end{tabular}

\cmd{rangelen}命令可以用于判断中:
%The \cmd{rangelen} command can be used in tests:

\begin{ltxexample}
\ifnumcomp{\rangelen{pages}}{=}{1}{add 'f'}{do nothing}
\end{ltxexample}

\cmditem{DeclareNumChars}{characters}
\cmditem*{DeclareNumChars*}{characters}

该命令设置\secref{aut:aux:tst}节的\cmd{ifnumeral}, \cmd{ifnumerals}, 和\cmd{ifpages}命令。该设置也将影响\cmd{iffieldnum}, \cmd{iffieldnums}, \cmd{iffieldpages}, \cmd{mkpageprefix} 和 \cmd{mkpagetotal}命令。\prm{characters}参数是一个无分隔符的符号列表,将作为数字的一部分进行处理。不带星命令将替换当前设置,带星命令则将其参数附加到当前列表中。默认设置为:
%This command configures the \cmd{ifnumeral}, \cmd{ifnumerals}, and \cmd{ifpages} tests from \secref{aut:aux:tst}. The setup will also affect \cmd{iffieldnum}, \cmd{iffieldnums}, \cmd{iffieldpages} as well as \cmd{mkpageprefix} and \cmd{mkpagetotal}. The \prm{characters} argument is an undelimited list of characters which are to be considered as being part of a number. The regular version of this command replaces the current setting, the starred version appends its argument to the current list. The default setting is:

\begin{ltxexample}
\DeclareNumChars{.}
\end{ltxexample}
%
这意味着(节或者其他)数值比如 <3.4.5> 将被认为是一个数字。注意,默认检测的是阿拉伯和罗马数字,没有必要对此做显式声明。
%This means that a (section or other) number like <3.4.5> will be considered as a number. Note that Arabic and Roman numerals are detected by default, there is no need to declare them explicitly.

\cmditem{DeclareRangeChars}{characters}
\cmditem*{DeclareRangeChars*}{characters}

该命令设置\secref{aut:aux:tst}的\cmd{ifnumerals}和\cmd{ifpages}命令。其设置还将影响\cmd{iffieldnums}, \cmd{iffieldpages} ,\cmd{mkpageprefix}和\cmd{mkpagetotal}。\prm{characters}参数是一个无分隔符的符号列表,将作为范围指示符进行处理。不带星命令将替换当前设置,带星命令则将其参数附加到当前列表中。默认设置为:
%This command configures the \cmd{ifnumerals} and \cmd{ifpages} tests from \secref{aut:aux:tst}. The setup will also affect \cmd{iffieldnums} and \cmd{iffieldpages} as well as \cmd{mkpageprefix} and \cmd{mkpagetotal}. The \prm{characters} argument is an undelimited list of characters which are to be considered as range indicators. The regular version of this command replaces the current setting, the starred version appends its argument to the current list. The default setting is:

\begin{ltxexample}
\DeclareRangeChars{~,;-+/}
\end{ltxexample}
%
这意味着比如<3--5>, <35+>, <8/9>等字符串会被\cmd{ifnumerals}和\cmd{ifpages}认为是一个范围。这些字符串中的非范围字符将被认为是数字。因此,类似于<3a--5a>和<35b+>之类的字符串默认情况下不被认为是范围。更多细节详见\secref{bib:use:pag, use:cav:pag}。
%This means that strings like <3--5>, <35+>, <8/9> and so on will be considered as a range by \cmd{ifnumerals} and \cmd{ifpages}. Non-range characters in such strings are recognized as numbers. So strings like <3a--5a> and <35b+> are not deemed to be ranges by default. See also \secref{bib:use:pag, use:cav:pag} for further details.

\cmditem{DeclareRangeCommands}{commands}
\cmditem*{DeclareRangeCommands*}{commands}

该命令类似于\cmd{DeclareRangeChars},差别在于\prm{commands}参数是一个无分隔符的命令列表,将被视为范围指示符。不带星命令将替换当前设置,带星命令则将其参数附加到当前列表中。默认列表相当长,应该覆盖所有一般情况。下面是一个简单例子:
%This command is similar to \cmd{DeclareRangeChars}, except that the \prm{commands} argument is an undelimited list of commands which are to be considered as range indicators. The regular version of this command replaces the current setting, the starred version appends its argument to the current list. The default list is rather long and should cover all common cases; here is a shorter example:

\begin{ltxexample}
\DeclareRangeCommands{\&\bibrangedash\textendash\textemdash\psq\psqq}
\end{ltxexample}
%
更多细节参见\secref{bib:use:pag, use:cav:pag}。
%See also \secref{bib:use:pag, use:cav:pag} for further details.

\cmditem{DeclarePageCommands}{commands}
\cmditem*{DeclarePageCommands*}{commands}

该命令类似于\cmd{DeclareRangeCommands},差别在于它仅影响\cmd{ifpages}和\cmd{iffieldpages}判断,而不影响\cmd{ifnumerals} 和\cmd{iffieldnums}。默认设置为:
%This command is similar to \cmd{DeclareRangeCommands}, except that it only affects the \cmd{ifpages} and \cmd{iffieldpages} tests but not \cmd{ifnumerals} and \cmd{iffieldnums}. The default setting is:

\begin{ltxexample}
\DeclarePageCommands{\pno\ppno}
\end{ltxexample}

\cmditem{NumCheckSetup}{code}

该命令用于临时重定义一些命令,若不重定义,这些命令将与\secref{aut:aux:tst}节的\cmd{ifnumeral}, \cmd{ifnumerals}, \cmd{ifpages}命令执行的判断冲突。该设置也将影响\cmd{iffieldnum}, \cmd{iffieldnums}, \cmd{iffieldpages}, \cmd{mkpageprefix}和\cmd{mkpagetotal}。这些命令将在组内执行\prm{code}。因为上述命令将展开为字符串用于分析,可以利用将冲突命令展开为空字符串(将被判断命令忽略)的方式来移除这些命令。更多细节参见\secref{bib:use:pag, use:cav:pag}。
%Use this command to temporarily redefine any commands which interfere with the tests performed by \cmd{ifnumeral}, \cmd{ifnumerals}, and \cmd{ifpages} from \secref{aut:aux:tst}. The setup will also affect \cmd{iffieldnum}, \cmd{iffieldnums}, \cmd{iffieldpages} as well as \cmd{mkpageprefix} and \cmd{mkpagetotal}. The \prm{code} will be executed in a group by these commands. Since the above mentioned commands will expand the string to be analyzed, it is possible to remove commands to be ignored by the tests by making them expand to an empty string. See also \secref{bib:use:pag, use:cav:pag} for further details.

\cmditem{DeclareCaseLangs}{languages}
\cmditem*{DeclareCaseLangs*}{languages}

定义语言列表,该列表在\cmd{MakeSentenceCase*}命令将一个字符串转换成句子时考虑。\prm{languages}参数是一个由\sty{babel}/\sty{polyglossia}语言标识构成的comma"=separated (逗号分隔)列表。不带星命令用于替换当前设置,而带星的命令用于附加当前列表。默认的设置为:
%Defines the list of languages which are considered by the \cmd{MakeSentenceCase*} command as it converts a string to sentence case. The \prm{languages} argument is a comma"=separated list of \sty{babel}/\sty{polyglossia} language identifiers. The regular version of this command replaces the current setting, the starred version appends its argument to the current list. The default setting is:

\begin{ltxexample}
\DeclareCaseLangs{%
  american,british,canadian,english,australian,newzealand,USenglish,UKenglish}
\end{ltxexample}
%
语言标识的列表见\sty{babel}/\sty{polyglossia}手册和\tabref{bib:fld:tab1}。
%See the \sty{babel}/\sty{polyglossia} manuals and \tabref{bib:fld:tab1} for a list of languages identifiers.

\cmditem{BibliographyWarning}{message}

该命令类似于\cmd{PackageWarning},但打印内容除了输入行号外还有当前处理条目的引用关键词。如果\prm{message}相当长,可以使用\cmd{MessageBreak}命令来断行。注意: 标准的\cmd{PackageWarning}命令在参考文献中使用时无法提供一个有意义的提示,因为其打印的输入行号只是\cmd{printbibliography}命令所在的行号。
%This command is similar to \cmd{PackageWarning} but prints the entry key of the entry currently being processed in addition to the input line number. It may be used in the bibliography as well as in citation commands. If the \prm{message} is fairly long, use \cmd{MessageBreak} to include line breaks. Note that the standard \cmd{PackageWarning} command does not provide a meaningful clue when used in the bibliography since the input line number is the line on which the \cmd{printbibliography} command was given.

%该命令用于\file{cbx}\slash\file{bbx}文件和\file{bib}文件的\texttt{@preamble}(导言)中。它检测选择的后端程序,如果后端不是\biber 则发出警告。可选参数\prm{severity}是一个整数用于指定警告的严重程度。当值等于1时将触发一个消息陈述推荐使用\biber;当值等于2触发一个警告陈述需要使用\biber ,否则样式或\slash\file{bib}文件可能无法正常工作;当值等于3触发一个错误信息陈述\biber 是严格必须的,样式或\slash\file{bib}文件在其它后端下将无法正常工作。如果\cmd{RequireBiber}命令多次使用,\prm{severity}应取最大值。\file{cbx}\slash\file{bbx}文件和所有\file{bib}文件的\texttt{@preamble}(导言)部分作为两个方面,其追踪是分别进行的。如果可选参数\prm{severity}缺省,那么默认值是2(触发一个警告信息)。

\boolitem{pagetracker}

这些命令将打开或关闭局部引用追踪器(这将影响来自\secref{aut:aux:tst}节的\cmd{iffirstonpage}和\cmd{ifsamepage}判断)。可在标注命令定义或者正文中的任意位置使用。要使标注命令完全排除页码追踪,可以在\cmd{DeclareCiteCommand}命令的\prm{precode}参数中使用\cmd{pagetrackerfalse}。详见\secref{aut:cbx:cbx}。注意:当全局页码追踪关闭时,这些命令无效。
%These commands activate or deactivate the citation tracker locally (this will affect the \cmd{iffirstonpage} and \cmd{ifsamepage} test from \secref{aut:aux:tst}). They are intended for use in the definition of citation commands or anywhere in the document body. If a citation command is to be excluded from page tracking, use \cmd{pagetrackerfalse} in the \prm{precode} argument of \cmd{DeclareCiteCommand}. See \secref{aut:cbx:cbx} for details. Note that these commands have no effect if page tracking has been disabled globally.

\boolitem{citetracker}

这些命令将打开或关闭所有的局部引用追踪器(这将影响来自\secref{aut:aux:tst}节的 \cmd{ifciteseen}, \cmd{ifentryseen}, \cmd{ifciteibid}, 和\cmd{ifciteidem}判断)。可在标注命令定义或者正文中的任意位置使用。要使标注命令完全排除页码追踪,可以在\cmd{DeclareCiteCommand}命令的\prm{precode}参数中使用\cmd{citetrackerfalse}。详见\secref{aut:cbx:cbx}。注意:当全局追踪关闭时,这些命令无效。
%These commands activate or deactivate all citation trackers locally (this will affect the \cmd{ifciteseen}, \cmd{ifentryseen}, \cmd{ifciteibid}, and \cmd{ifciteidem} tests from \secref{aut:aux:tst}). They are intended for use in the definition of citation commands or anywhere in the document body. If a citation command is to be excluded from tracking, use \cmd{citetrackerfalse} in the \prm{precode} argument of \cmd{DeclareCiteCommand}. See \secref{aut:cbx:cbx} for details. Note that these commands have no effect if tracking has been disabled globally.

\boolitem{backtracker}

这些命令将打开或关闭所有的局部\texttt{backref}追踪器。可在标注命令定义或者正文中的任意位置使用。要使标注命令完全排除反向链接追踪,可以在\cmd{DeclareCiteCommand}命令的\prm{precode}参数中使用\cmd{backtrackerfalse}。注意:当\texttt{backref}选项未进行全局设置,这些命令无效。
%These commands activate or deactivate the \texttt{backref} tracker locally. They are intended for use in the definition of citation commands or anywhere in the document body. If a citation command is to be excluded from backtracking, use \cmd{backtrackerfalse} in the \prm{precode} argument of \cmd{DeclareCiteCommand}. Note that these commands have no effect if the \texttt{backref} option has been not been set globally.

\end{ltxsyntax}

\subsection[标点和间距]{标点和间距}%Punctuation and Spacing
\label{aut:pct}

The \biblatex package provides elaborate facilities designed to manage and track punctuation and spacing in the bibliography and in citations. These facilities work on two levels. The high"=level commands discussed in \secref{aut:pct:new} deal with punctuation and whitespace inserted by the bibliography style between the individual segments of a bibliography entry. The commands in \secref{aut:pct:chk, aut:pct:pct, aut:pct:spc} work at a lower level. They use \tex's space factor and modified space factor codes to track punctuation in a robust and efficient way. This way it is possible to detect trailing punctuation marks within fields, not only those explicitly inserted between fields. The same technique is also used for automatic capitalization of localisation strings, see \cmd{DeclareCapitalPunctuation} in \secref{aut:pct:cfg} as well as \secref{aut:str} for details. Note that these facilities are only made available locally in citations and bibliographies. They will not affect any other part of a document.

\subsubsection{块和单元标点 Block and Unit Punctuation}
\label{aut:pct:new}

The major segments of a bibliography entry are <blocks> and <units>. A block is the larger segment of the two, a unit is shorter or at most equal in length. For example, the values of fields such as \bibfield{title} or \bibfield{note} usually form a unit which is separated from subsequent data by a period or a comma. A block may comprise several fields which are treated as separate units, for example \bibfield{publisher}, \bibfield{location}, and \bibfield{year}. The segmentation of an entry into blocks and units is at the discretion of the bibliography style. An entry is segmented by inserting \cmd{newblock} and \cmd{newunit} commands at suitable places and \cmd{finentry} at the very end (see \secref{aut:bbx:drv} for an example). See also \secref{aut:cav:pct} for some practical hints.

\begin{ltxsyntax}

\csitem{newblock}

Records the end of a block. This command does not print anything, it merely marks the end of the block. The block delimiter \cmd{newblockpunct} will be inserted by a subsequent \cmd{printtext}, \cmd{printfield}, \cmd{printlist}, \cmd{printnames}, or \cmd{bibstring} command. You may use \cmd{newblock} at suitable places without having to worry about spurious blocks. A new block will only be started by the next \cmd{printfield} (or similar) command if this command prints anything. See \secref{aut:cav:pct} for further details.

\csitem{newunit}

Records the end of a unit and puts the default delimiter \cmd{newunitpunct} in the punctuation buffer. This command does not print anything, it merely marks the end of the unit. The punctuation buffer will be inserted by the next \cmd{printtext}, \cmd{printfield}, \cmd{printlist}, \cmd{printnames}, or \cmd{bibstring} command. You may use \cmd{newunit} after commands like \cmd{printfield} without having to worry about spurious punctuation and whitespace. The buffer will only be inserted by the next \cmd{printfield} or similar command if \emph{both} fields are non"=empty. This also applies to \cmd{printtext}, \cmd{printlist}, \cmd{printnames}, and \cmd{bibstring}. See \secref{aut:cav:pct} for further details.

\csitem{finentry}

Inserts \cmd{finentrypunct}. This command should be used at the very end of every bibliography entry.

\cmditem{setunit}{punctuation}
\cmditem*{setunit*}{punctuation}

The \cmd{setunit} command is similar to \cmd{newunit} except that it uses \prm{punctuation} instead of \cmd{newunitpunct}. The starred variant differs from the regular version in that it checks if the last \cmd{printtext}, \cmd{printfield}, \cmd{printlist}, \cmd{printnames}, or \cmd{bibstring} command did actually print anything. If not, it does nothing.

\cmditem{printunit}{punctuation}
\cmditem*{printunit*}{punctuation}

The \cmd{printunit} command is similar to \cmd{setunit} except that \prm{punctuation} persists in the buffer. This ensures that \prm{punctuation} is inserted before the next non"=empty field printed by the \cmd{printtext}, \cmd{printfield}, \cmd{printlist}, \cmd{printnames}, or \cmd{bibstring} commands---regardless of any intermediate calls to \cmd{newunit} or \cmd{setunit}.

\cmditem{setpunctfont}{command}

This command, which is intended for use in field formatting directives, provides an alternative way of dealing with unit punctuation after a field printed in a different font (for example, a title printed in italics). The standard \latex way of dealing with this is adding a small amount of space, the so-called italic correction. This command allows adapting the punctuation to the font of the preceding field. The \prm{command} should be a text font command which takes one argument, such as \cmd{emph} or \cmd{textbf}. This command will only affect punctuation marks inserted by one of the commands from \secref{aut:pct:pct}. The font adaption is applied to the next punctuation mark only and will be reset automatically thereafter. If you want to reset it manually before it takes effect, issue \cmd{resetpunctfont}. If the \opt{punctfont} package option is disabled, this command does nothing. Note that the \cmd{mkbibemph}, \cmd{mkbibitalic} and \cmd{mkbibbold}  wrappers from \secref{aut:fmt:ich} incorporate this feature by default.

\csitem{resetpunctfont}

This command resets the unit punctuation font defined with \cmd{setpunctfont} before it takes effect. If the \opt{punctfont} package option is disabled, this command does nothing.

\end{ltxsyntax}

\subsubsection{标点判断 Punctuation Tests}
\label{aut:pct:chk}

The following commands may be used to test for preceding punctuation marks at any point in citations and the bibliography.

\begin{ltxsyntax}

\cmditem{ifpunct}{true}{false}

Executes \prm{true} if preceded by any punctuation mark except for an abbreviation dot, and \prm{false} otherwise.

\cmditem{ifterm}{true}{false}

Executes \prm{true} if preceded by a terminal punctuation mark, and \prm{false} otherwise. A terminal punctuation mark is any punctuation mark which has been registered for automatic capitalization, either with \cmd{DeclareCapitalPunctuation} or by default, see \secref{aut:pct:cfg} for details. By default, this applies to periods, exclamation marks, and question marks.

\cmditem{ifpunctmark}{character}{true}{false}

Executes \prm{true} if preceded by the punctuation mark \prm{character}, and \prm{false} otherwise. The \prm{character} may be a comma, a semicolon, a colon, a period, an exclamation mark, a question mark, or an asterisk. Note that a period denotes an end-of"=sentence period. Use the asterisk to test for the dot after an abbreviation. If this command is used in a formatting directive for name lists, \ie in the argument to \cmd{DeclareNameFormat}, the \prm{character} may also be an apostrophe.

\cmditem{ifprefchar}{true}{false}

Executes \prm{true} if preceded by any prefix character declared by \cmd{DeclarePrefChars}.

\end{ltxsyntax}

\subsubsection{添加标点  Adding Punctuation}
\label{aut:pct:pct}
下面的命令设计用来重复标点。参考文献和标注样式总需要使用这些命令来代替原样输出标点符号。本节中所有的\cmd{add...}命令自动利用\cmd{unspace}命令移除前面的空白(见\secref{aut:pct:spc})。注意:下面讨论的所有的\cmd{add...}命令的作用是宏包默认的,无论\biblatex 换哪种语言都会重新恢复。其作用可以通过\cmd{DeclarePunctuationPairs}命令进行调整,见\secref{aut:pct:cfg}。节

The following commands are designed to prevent double punctuation marks. Bibliography and citation styles should always use these commands instead of literal punctuation marks. All \cmd{add...} commands in this section automatically remove preceding whitespace with \cmd{unspace} (see \secref{aut:pct:spc}). Note that the behavior of all \cmd{add...} commands discussed below is the package default, which is restored whenever \biblatex switches languages. This behavior may be adjusted with \cmd{DeclarePunctuationPairs} from \secref{aut:pct:cfg}.

\begin{ltxsyntax}

\csitem{adddot}

如果前面输出的不是任何一种标点符号,那么添加一个句点(period)。该命令的目的是在一个缩写后面插入点(dot)。以这种方式插入的点被认为与其它标点命令插入的标点性质相同。该命令也用来将前面如实输出的句点(原样输出的句点,literal period) 转换成一个缩写的点。
Adds a period unless it is preceded by any punctuation mark. The purpose of this command is inserting the dot after an abbreviation. Any dot inserted this way is recognized as such by the other punctuation commands. This command may also be used to turn a previously inserted literal period into an abbreviation dot.

\csitem{addcomma}

如果前面输出不是一个逗号(comma)、分号(semicolon)、冒号(colon)和句点(period),那么添加一个逗号。
Adds a comma unless it is preceded by another comma, a semicolon, a colon, or a period.

\csitem{addsemicolon}

Adds a semicolon unless it is preceded by a comma, another semicolon, a colon, or a period.

\csitem{addcolon}

Adds a colon unless it is preceded by a comma, a semicolon, another colon, or a period.

\csitem{addperiod}

如果前面输出不是一个缩写点或其他任何标点符号,那么添加一个句号。该命令也可以用来将前面插入的缩写点转换为句号,比如在句子的末尾\footnote{不是很理解,前面如果是缩写的点,那么不加入句号,那么缩写的点就转变为句号了?}。
Adds a period unless it is preceded by an abbreviation dot or any other punctuation mark. This command may also be used to turn a previously inserted abbreviation dot into a period, for example at the end of a sentence.

\csitem{addexclam}
Adds an exclamation mark unless it is preceded by any punctuation mark except for an abbreviation dot.

\csitem{addquestion}

Adds a question mark unless it is preceded by any punctuation mark except for an abbreviation dot.

\csitem{isdot}

当前面输出的是句号的时候,将其转换为缩写的点,如果前面是其它符号那么不添加任何符号。
Turns a previously inserted literal period into an abbreviation dot. In contrast to \cmd{adddot}, nothing is inserted if this command is not preceded by a period.

\csitem{nopunct}

Adds an internal marker which will cause the next punctuation command to print nothing.

\end{ltxsyntax}

\subsubsection{添加空格 Adding Whitespace}
\label{aut:pct:spc}

The following commands are designed to prevent spurious whitespace. Bibliography and citation styles should always use these commands instead of literal whitespace. In contrast to the commands in \secref{aut:pct:chk, aut:pct:pct}, they are not restricted to citations and the bibliography but available globally.

\begin{ltxsyntax}

\csitem{unspace}

Removes preceding whitespace, \ie removes all skips and penalties from the end of the current horizontal list. This command is implicitly executed by all of the following commands.

\csitem{addspace}

Adds a breakable interword space.

\csitem{addnbspace}

Adds a non"=breakable interword space.

\csitem{addthinspace}

Adds a \emph{breakable} thin space.

\csitem{addnbthinspace}

Adds a non"=breakable thin space. This is similar to \cmd{,} and \cmd{thinspace}.

\csitem{addlowpenspace}

Adds a space penalized by the value of the \cnt{lownamepenalty} counter, see \secref{use:fmt:len, aut:fmt:len} for details.

\csitem{addhighpenspace}

Adds a space penalized by the value of the \cnt{highnamepenalty} counter, see \secref{use:fmt:len, aut:fmt:len} for details.

\csitem{addlpthinspace}

Similar to \cmd{addlowpenspace} but adds a breakable thin space.

\csitem{addhpthinspace}

Similar to \cmd{addhighpenspace} but adds a breakable thin space.

\csitem{addabbrvspace}

Adds a space penalized by the value of the \cnt{abbrvpenalty} counter, see \secref{use:fmt:len, aut:fmt:len} for details.

\csitem{addabthinspace}

Similar to \cmd{addabbrvspace} but using a thin space.

\csitem{adddotspace}

Executes \cmd{adddot} and adds a space penalized by the value of the \cnt{abbrvpenalty} counter, see \secref{use:fmt:len, aut:fmt:len} for details.

\csitem{addslash}

Adds a breakable slash. This command differs from the \cmd{slash} command in the \latex kernel in that a linebreak after the slash is not penalized at all.

\end{ltxsyntax}

Note that the commands in this section implicitly execute \cmd{unspace} to remove spurious whitespace, hence they may be used to override each other. For example, you may use \cmd{addnbspace} to transform a previously inserted interword space into a non"=breakable one and \cmd{addspace} to turn a non"=breakable space into a breakable one.

\footnote{注意有的时候\cmd{unspace}看似能够起到作用,但其实并不能随意使用的。在beamer中printtext老是有些问题,可能是实现printtext命令的依赖命令,在beamer中重定义了,跟aritcle文档类中的情况差别很大。}
\subsubsection{标点设置和大写 Configuring Punctuation and Capitalization}
\label{aut:pct:cfg}

The following commands configure various features related to punctuation and automatic capitalization.
\footnote{这里的capitalization是大写的意思么?}
\begin{ltxsyntax}

\cmditem{DeclarePrefChars}{characters}

This command declares characters that are to be treated specially when testing to see if \cmd{bibnamedelimc} is to be inserted between a name prefix and a family name. If a character is in the list of \prm{characters}, \cmd{bibnamedelimc} is not inserted. It is used to allow abbreviated name prefices like <d'Argent> where no space should be inserted after the apostrophe. The default setting is:

\begin{ltxexample}
\DeclarePrefChars{'}
\end{ltxexample}

\cmditem{DeclareAutoPunctuation}{characters}

This command defines the punctuation marks to be considered by the citation commands as they scan ahead for punctuation. Note that \prm{characters} is an undelimited list of characters. Valid \prm{characters} are period, comma, semicolon, colon, exclamation and question mark. The default setting is:

\begin{ltxexample}
\DeclareAutoPunctuation{.,;:!?}
\end{ltxexample}
%
This definition is restored automatically whenever the \opt{autopunct} package option is set to \texttt{true}. Executing |\DeclareAutoPunctuation{}| is equivalent to setting \kvopt{autopunct}{false}, \ie it disables this feature.

\cmditem{DeclareCapitalPunctuation}{characters}

When \biblatex inserts localisation strings, \ie key terms such as <edition> or <volume>, it automatically capitalizes them after terminal punctuation marks. This command defines the punctuation marks which will cause localisation strings to be capitalized if one of them precedes a string. Note that \prm{characters} is an undelimited list of characters. Valid \prm{characters} are period, comma, semicolon, colon, exclamation and question mark. The package default is:

\begin{ltxexample}
\DeclareCapitalPunctuation{.!?}
\end{ltxexample}
%
Using \cmd{DeclareCapitalPunctuation} with an empty argument is equivalent to disabling automatic capitalization. Since this feature is language specific, this command must be used in the argument to \cmd{DefineBibliographyExtras} (when used in the preamble) or \cmd{DeclareBibliographyExtras} (when used in a localisation module). See \secref{use:lng, aut:lng} for details. By default, strings are capitalized after periods, exclamation marks, and question marks. All strings are generally capitalized at the beginning of a paragraph (in fact whenever \tex is in vertical mode).

\cmditem{DeclarePunctuationPairs}{identifier}{characters}

Use this command to declare valid pairs of punctuation marks. This will affect the punctuation commands discussed in \secref{aut:pct:pct}. For example, the description of \cmd{addcomma} states that this command adds a comma unless it is preceded by another comma, a semicolon, a colon, or a period. In other words, commas after abbreviation dots, exclamation marks, and question marks are permitted. These valid pairs are declared as follows:

\begin{ltxexample}
\DeclarePunctuationPairs{comma}{*!?}
\end{ltxexample}
%
The \prm{identifier} selects the command to be configured. The identifiers correspond to the names of the punctuation commands from \secref{aut:pct:pct} without the \cmd{add} prefix, \ie valid \prm{identifier} strings are \texttt{dot}, \texttt{comma}, \texttt{semicolon}, \texttt{colon}, \texttt{period}, \texttt{exclam}, \texttt{question}. The \prm{characters} argument is an undelimited list of punctuation marks. Valid \prm{characters} are comma, semicolon, colon, period, exclamation mark, question mark, and asterisk. A period in the \prm{characters} argument denotes an end-of"=sentence period, an asterisk the dot after an abbreviation. This is the default setup, which is automatically restored whenever \biblatex switches languages and corresponds to the behavior described in \secref{aut:pct:pct}:

\begin{ltxexample}
\DeclarePunctuationPairs{dot}{}
\DeclarePunctuationPairs{comma}{*!?}
\DeclarePunctuationPairs{semicolon}{*!?}
\DeclarePunctuationPairs{colon}{*!?}
\DeclarePunctuationPairs{period}{}
\DeclarePunctuationPairs{exclam}{*}
\DeclarePunctuationPairs{question}{*}
\end{ltxexample}
%
Since this feature is language specific, \cmd{DeclarePunctuationPairs} must be used in the argument to \cmd{DefineBibliographyExtras} (when used in the preamble) or \cmd{DeclareBibliographyExtras} (when used in a localisation module). See \secref{use:lng, aut:lng} for details. Note that some localisation modules may use a setup which is different from the package default.\footnote{As of this writing, the \texttt{american} module uses different settings for <American-style> punctuation.}

\cmditem{DeclareQuotePunctuation}{characters}

This command controls <American-style> punctuation. The \cmd{mkbibquote} wrapper from \secref{aut:fmt:ich} can interact with the punctuation facilities discussed in \secref{aut:pct:new, aut:pct:pct, aut:pct:spc}. Punctuation marks after \cmd{mkbibquote} will be moved inside the quotes if they have been registered with \cmd{DeclareQuotePunctuation}. Note that \prm{characters} is an undelimited list of characters. Valid \prm{characters} are period, comma, semicolon, colon, exclamation and question mark. Here is an example:

\begin{ltxexample}
\DeclareQuotePunctuation{.,}
\end{ltxexample}
%
Executing |\DeclareQuotePunctuation{}| is equivalent to disabling this feature. This is the package default. Since this feature is language specific, this command must be used in the argument to \cmd{DefineBibliographyExtras} (when used in the preamble) or \cmd{DeclareBibliographyExtras} (when used in a localisation module). See \secref{use:lng, aut:lng} for details. See also \secref{use:loc:us}.

\csitem{uspunctuation}

A shorthand using the lower-level commands \cmd{DeclareQuotePunctuation} and \cmd{DeclarePunctuationPairs} to activate <American-style> punctuation. See \secref{use:loc:us} for details. This shorthand is provided for convenience only. The effective settings are applied by the lower-level commands.

\csitem{stdpunctuation}

Undoes the settings applied by \cmd{uspunctuation}, restoring standard punctuation. As standard punctuation is the default setting, you only need this command to override a previously executed \cmd{uspunctuation} command. See \secref{use:loc:us} for details.

\end{ltxsyntax}

\subsubsection{Correcting Punctuation Tracking}
\label{aut:pct:ctr}

The facilities for punctuation tracking and automatic capitalization are very reliable under normal circumstances, but there are always marginal cases which may require manual intervention. Typical cases are localisation strings printed as the first word in a footnote (which is usually treated as the beginning of a paragraph as far as capitalization is concerned, but \tex is not in vertical mode at this point) or punctuation after periods which are not really end"=of"=sentence periods (for example, after an ellipsis like «[\dots\unkern]» a command such as \cmd{addperiod} would do nothing since parentheses and brackets are transparent to the punctuation tracker). In such cases, use the following commands in bibliography and citation styles to mark the beginning or middle of a sentence if and where required:

\begin{ltxsyntax}

\csitem{bibsentence}

This command marks the beginning of a sentence. A localisation string immediately after this command will be capitalized and the punctuation tracker is reset, \ie this command hides all preceding punctuation marks from the punctuation tracker and enforces capitalization.

\csitem{midsentence}

This command marks the middle of a sentence. A localisation string immediately after this command will not be capitalized and the punctuation tracker is reset, \ie this command hides all preceding punctuation marks from the punctuation tracker and suppresses capitalization.

\csitem*{midsentence*}

The starred variant of \cmd{midsentence} differs from the regular one in that a preceding abbreviation dot is not hidden from the the punctuation tracker, \ie any code after \cmd{midsentence*} will see a preceding abbreviation dot. All other punctuation marks are hidden from the punctuation tracker and capitalization is suppressed.

\end{ltxsyntax}

\subsection{本地化字符串 Localization Strings}
\label{aut:str}

Localization strings are key terms such as <edition> or <volume> which are automatically translated by \biblatex's localisation modules. See \secref{aut:lng} for an overview and \secref{aut:lng:key} for a list of all strings supported by default. The commands in this section are used to print the localised term.

\begin{ltxsyntax}

\cmditem{bibstring}[wrapper]{key}

Prints the localisation string \prm{key}, where \prm{key} is an identifier in lowercase letters (see \secref{aut:lng:key}). The string will be capitalized as required, see \secref{aut:pct:cfg} for details.
Depending on the \opt{abbreviate} package option from \secref{use:opt:pre:gen}, \cmd{bibstring} prints the short or the long version of the string. If localisation strings are nested, \ie if \cmd{bibstring} is used in another string, it will behave like \cmd{bibxstring}.
If the \prm{wrapper} argument is given, the string is passed to the \prm{wrapper} for formatting. This is intended for font commands such as \cmd{emph}.

\cmditem{biblstring}[wrapper]{key}

Similar to \cmd{bibstring} but always prints the long string, ignoring the \opt{abbreviate} option.

\cmditem{bibsstring}[wrapper]{key}

Similar to \cmd{bibstring} but always prints the short string, ignoring the \opt{abbreviate} option.

\cmditem{bibcpstring}[wrapper]{key}

Similar to \cmd{bibstring} but the term is always capitalized.

\cmditem{bibcplstring}[wrapper]{key}

Similar to \cmd{biblstring} but the term is always capitalized.

\cmditem{bibcpsstring}[wrapper]{key}

Similar to \cmd{bibsstring} but the term is always capitalized.

\cmditem{bibucstring}[wrapper]{key}

Similar to \cmd{bibstring} but the whole term is uppercased.

\cmditem{bibuclstring}[wrapper]{key}

Similar to \cmd{biblstring} but the whole term is uppercased.

\cmditem{bibucsstring}[wrapper]{key}

Similar to \cmd{bibsstring} but the whole term is uppercased.

\cmditem{biblcstring}[wrapper]{key}

Similar to \cmd{bibstring} but the whole term is lowercased.

\cmditem{biblclstring}[wrapper]{key}

Similar to \cmd{biblstring} but the whole term is lowercased.

\cmditem{biblcsstring}[wrapper]{key}

Similar to \cmd{bibsstring} but the whole term is lowercased.

\cmditem{bibxstring}{key}

A simplified but expandable version of \cmd{bibstring}. Note that this variant does not capitalize automatically, nor does it hook into the punctuation tracker. It is intended for special cases in which strings are nested or an expanded localisation string is required in a test.

\cmditem{bibxlstring}[wrapper]{key}

Similar to \cmd{bibxstring} but always uses the long string, ignoring the \opt{abbreviate} option.

\cmditem{bibxsstring}[wrapper]{key}

Similar to \cmd{bibxstring} but always uses the short string, ignoring the \opt{abbreviate} option.

\cmditem{mainlang}

Switches from the current language to the main document language. This can be used the \prm{wrapper} argument in the localisation string commands above.

\end{ltxsyntax}

\subsection{本地化模块 Localization Modules}
\label{aut:lng}

A localisation module provides translations for key terms such as <edition> or <volume> as well as definitions for language specific features such as the date format and ordinals. These definitions are provided in files with the suffix \file{lbx}. The base name of the file must be a language name known to the \sty{babel}/\sty{polyglossia} packages. The \file{lbx} files may also be used to map \sty{babel}/\sty{polyglossia} language names to the backend modules of the \biblatex package. All localisation modules are loaded on demand in the document body. Note that the contents of the file are processed in a group and that the category code of the character \texttt{@} is temporarily set to <letter>.

\subsubsection{本地化命令 Localization Commands}
\label{aut:lng:cmd}

The user"=level versions of the localisation commands were already introduced in \secref{use:lng}. When used in \file{lbx} files, however, the syntax of localisation commands is different from the user syntax in the preamble and the configuration file. When used in localisation files, there is no need to specify the \prm{language} because the mapping of strings to a language is already provided by the name of the \file{lbx} file.

\begin{ltxsyntax}

\cmditem{DeclareBibliographyStrings}{definitions}

This command is only available in \file{lbx} files. It is used to define localisation strings. The \prm{definitions} consist of \keyval pairs which assign an expression to an identifier. A complete list of all keys supported by default is given is \secref{aut:lng:key}. Note that the syntax of the value is different in \file{lbx} files. The value assigned to a key consists of two expressions, each of which is wrapped in an additional pair of brackets. This is best shown by example:

\begin{ltxexample}
\DeclareBibliographyStrings{%
  bibliography  = {{Bibliography}{Bibliography}},
  shorthands    = {{List of Abbreviations}{Abbreviations}},
  editor        = {{editor}{ed.}},
  editors       = {{editors}{eds.}},
}
\end{ltxexample}
%
The first value is the long, written out expression, the second one is an abbreviated or short form. Both strings must always be given even though they may be identical if an expression is always (or never) abbreviated. Depending on the setting of the \opt{abbreviate} package option (see \secref{use:opt:pre:gen}), \biblatex selects one expression when loading the \file{lbx} file. There is also a special key named \texttt{inherit} which copies the strings from a different language. This is intended for languages which only differ in a few expressions, such as German and Austrian or American and British English. For example, here are the complete definitions for Austrian:

\begin{ltxexample}
\DeclareBibliographyStrings{%
  inherit       = {german},
  january       = {{J\"anner}{J\"an.}},
}
\end{ltxexample}

The above examples are slightly simplified. Real localisation files should use the punctuation and formatting commands discussed in \secref{aut:pct:pct, use:fmt} instead of literal punctuation. Here is an excerpt from a real localisation file:

\begin{ltxexample}
  bibliography     = {{Bibliography}{Bibliography}},
  shorthands       = {{List of Abbreviations}{Abbreviations}},
  editor           = {{editor}{ed\adddot}},
  editors          = {{editors}{eds\adddot}},
  byeditor         = {{edited by}{ed\adddotspace by}},
  mathesis         = {{Master's thesis}{MA\addabbrvspace thesis}},
\end{ltxexample}
%
Note the handling of abbreviation dots, the spacing in abbreviated expressions, and the capitalization in the example above. All expressions should be capitalized as they usually are when used in the middle of a sentence. The \biblatex package will automatically capitalize the first word when required at the beginning of a sentence, see \cmd{DeclareCapitalPunctuation} in \secref{aut:pct:cfg} for details. Expressions intended for use in headings are special. They should be capitalized in a way that is suitable for titling and should not be abbreviated (but they may have a short form).

\cmditem{InheritBibliographyStrings}{language}

This command is only available in \file{lbx} files. It copies the localisation strings for \prm{language} to the current language, as specified by the name of the \file{lbx} file.

\cmditem{DeclareBibliographyExtras}{code}

This command is only available in \file{lbx} files. It is used to adapt language specific features such as the date format and ordinals. The \prm{code}, which may be arbitrary \latex code, will usually consist of redefinitions of the formatting commands from \secref{aut:fmt:lng}.

\cmditem{UndeclareBibliographyExtras}{code}

This command is only available in \file{lbx} files. It is used to restore any formatting commands modified with \cmd{DeclareBibliographyExtras}. If a redefined command is included in \secref{aut:fmt:lng}, there is no need to restore its previous definition since these commands are localised by all language modules anyway.

\cmditem{InheritBibliographyExtras}{language}

This command is only available in \file{lbx} files. It copies the bibliography extras for \prm{language} to the current language, as specified by the name of the \file{lbx} file.

\cmditem{DeclareHyphenationExceptions}{text}

This command corresponds to \cmd{DefineHyphenationExceptions} from \secref{use:lng}. The difference is that it is only available in \file{lbx} files and that the \prm{language} argument is omitted. The hyphenation exceptions will affect the language of the \file{lbx} file currently being processed.

\cmditem{DeclareRedundantLanguages}{language, language, ...}{langid, langid, ...}

This command provides the language mappings required by the \opt{clearlang} option from \secref{use:opt:pre:gen}.
The \prm{language} is the string given in the \bibfield{language} field (without the optional \texttt{lang} prefix); \prm{langid} is \sty{babel}/\sty{polyglossia}'s language identifier, as given in the optional argument of \cmd{usepackage} when loading \sty{babel} or the argument of \cmd{setdefaultlanguage} or \cmd{setotherlanguages} when using \sty{polyglossia}. This command may be used in \file{lbx} files or in the document preamble. Here are some examples:

\begin{ltxexample}
\DeclareRedundantLanguages{french}{french}
\DeclareRedundantLanguages{german}{german,ngerman,austrian,naustrian,
        nswissgerman,swissgerman}
\DeclareRedundantLanguages{english,american}{english,american,british,
	canadian,australian,newzealand,USenglish,UKenglish}
\end{ltxexample}
%
Note that this feature needs to be enabled globally with the \opt{clearlang} option from \secref{use:opt:pre:gen}. If it is disabled, all mappings will be ignored. If the \prm{langid} parameter is blank, \biblatex will clear the mappings for the corresponding \prm{language}, \ie the feature will be disabled for this \prm{language} only.

\cmditem{DeclareLanguageMapping}{language}{file}

This command maps a \sty{babel}/\sty{polyglossia} language identifier to an \file{lbx} file. The \prm{language} must be a language name known to the \sty{babel}/\sty{polyglossia} package, \ie one of the identifiers listed in \tabref{bib:fld:tab1}. The \prm{file} argument is the name of an alternative \file{lbx} file without the \texttt{.lbx} suffix. Declaring the same mapping more than once is possible. Subsequent declarations will simply overwrite any previous ones. This command may only be used in the preamble. See \secref{aut:cav:lng} for further details.

\cmditem{NewBibliographyString}{key}

This command, which may be used in the preamble (including \file{cbx} and \file{bbx} files) as well as in \file{lbx} files, declares new localisation strings, \ie it initializes a new \prm{key} to be used in the \prm{definitions} of \cmd{DefineBibliographyStrings} or \cmd{DeclareBibliographyStrings}. The \prm{key} argument may also be a comma"=separated list of key names. When used in an \file{lbx}, the \prm{key} is initialized only for the language specified by the name of the \file{lbx} file. The keys listed in \secref{aut:lng:key} are defined by default.

\end{ltxsyntax}

\subsubsection{Localization Keys}
\label{aut:lng:key}

The localisation keys in this section are defined by default and covered by the localisation files which come with \biblatex. Note that these strings are only available in citations, the bibliography and bibliography lists. All expressions should be capitalized as they usually are when used in the middle of a sentence. \biblatex will capitalize them automatically at the beginning of a sentence. The only exceptions to these rules are the three strings intended for use in headings.

\paragraph{Headings}
\label{aut:lng:key:bhd}

The following strings are special because they are intended for use in headings and made available globally via macros. For this reason, they should be capitalized for use in headings and they must not include any local commands which are part of \biblatex's author interface.

\begin{keymarglist}
\item[bibliography] The term <bibliography>, also available as \cmd{bibname}.
\item[references] The term <references>, also available as \cmd{refname}.
\item[shorthands] The term <list of shorthands> or <list of abbreviations>, also available as \cmd{biblistname}.
\end{keymarglist}

\paragraph{Roles, Expressed as Functions}
\label{aut:lng:key:efn}

The following keys refer to roles which are expressed as a function (<editor>, <translator>) rather than as an action (<edited by>, <translated by>).

\begin{keymarglist}
\item[editor] The term <editor>, referring to the main editor. This is the most generic editorial role.
\item[editors] The plural form of \texttt{editor}.
\item[compiler] The term <compiler>, referring to an editor whose task is to compile a work.
\item[compilers] The plural form of \texttt{compiler}.
\item[founder] The term <founder>, referring to a founding editor.
\item[founders] The plural form of \texttt{founder}.
\item[continuator] An expression like <continuator>, <continuation>, or <continued>, referring to a past editor who continued the work of the founding editor but was subsequently replaced by the current editor.
\item[continuators] The plural form of \texttt{continuator}.
\item[redactor] The term <redactor>, referring to a secondary editor.
\item[redactors] The plural form of \texttt{redactor}.
\item[reviser] The term <reviser>, referring to a secondary editor.
\item[revisers] The plural form of \texttt{reviser}.
\item[collaborator] A term like <collaborator>, <collaboration>, <cooperator>, or <cooperation>, referring to a secondary editor.
\item[collaborators] The plural form of \texttt{collaborator}.
\item[translator] The term <translator>.
\item[translators] The plural form of \texttt{translator}.
\item[commentator] The term <commentator>, referring to the author of a commentary to a work.
\item[commentators] The plural form of \texttt{commentators}.
\item[annotator] The term <annotator>, referring to the author of annotations to a work.
\item[annotators] The plural form of \texttt{annotators}.
\end{keymarglist}

\paragraph{Concatenated Editor Roles, Expressed as Functions}
\label{aut:lng:key:cef}

The following keys are similar in function to \texttt{editor}, \texttt{translator}, etc. They are used to indicate additional roles of the editor, \eg\ <editor and translator>, <editor and foreword>.

\begin{keymarglist}
\item[editortr] Used if \bibfield{editor}\slash \bibfield{translator} are identical.
\item[editorstr] The plural form of \texttt{editortr}.
\item[editorco] Used if \bibfield{editor}\slash \bibfield{commentator} are identical.
\item[editorsco] The plural form of \texttt{editorco}.
\item[editoran] Used if \bibfield{editor}\slash \bibfield{annotator} are identical.
\item[editorsan] The plural form of \texttt{editoran}.
\item[editorin] Used if \bibfield{editor}\slash \bibfield{introduction} are identical.
\item[editorsin] The plural form of \texttt{editorin}.
\item[editorfo] Used if \bibfield{editor}\slash \bibfield{foreword} are identical.
\item[editorsfo] The plural form of \texttt{editorfo}.
\item[editoraf] Used if \bibfield{editor}\slash \bibfield{aftword} are identical.
\item[editorsaf] The plural form of \texttt{editoraf}.
\end{keymarglist}
%
Keys for \bibfield{editor}\slash \bibfield{translator}\slash \prm{role} combinations:

\begin{keymarglist}
\item[editortrco] Used if \bibfield{editor}\slash \bibfield{translator}\slash \bibfield{commentator} are identical.
\item[editorstrco] The plural form of \texttt{editortrco}.
\item[editortran] Used if \bibfield{editor}\slash \bibfield{translator}\slash \bibfield{annotator} are identical.
\item[editorstran] The plural form of \texttt{editortran}.
\item[editortrin] Used if \bibfield{editor}\slash \bibfield{translator}\slash \bibfield{introduction} are identical.
\item[editorstrin] The plural form of \texttt{editortrin}.
\item[editortrfo] Used if \bibfield{editor}\slash \bibfield{translator}\slash \bibfield{foreword} are identical.
\item[editorstrfo] The plural form of \texttt{editortrfo}.
\item[editortraf] Used if \bibfield{editor}\slash \bibfield{translator}\slash \bibfield{aftword} are identical.
\item[editorstraf] The plural form of \texttt{editortraf}.
\end{keymarglist}
%
Keys for \bibfield{editor}\slash \bibfield{commentator}\slash \prm{role} combinations:

\begin{keymarglist}
\item[editorcoin] Used if \bibfield{editor}\slash \bibfield{commentator}\slash \bibfield{introduction} are identical.
\item[editorscoin] The plural form of \texttt{editorcoin}.
\item[editorcofo] Used if \bibfield{editor}\slash \bibfield{commentator}\slash \bibfield{foreword} are identical.
\item[editorscofo] The plural form of \texttt{editorcofo}.
\item[editorcoaf] Used if \bibfield{editor}\slash \bibfield{commentator}\slash \bibfield{aftword} are identical.
\item[editorscoaf] The plural form of \texttt{editorcoaf}.
\end{keymarglist}
%
Keys for \bibfield{editor}\slash \bibfield{annotator}\slash \prm{role} combinations:

\begin{keymarglist}
\item[editoranin] Used if \bibfield{editor}\slash \bibfield{annotator}\slash \bibfield{introduction} are identical.
\item[editorsanin] The plural form of \texttt{editoranin}.
\item[editoranfo] Used if \bibfield{editor}\slash \bibfield{annotator}\slash \bibfield{foreword} are identical.
\item[editorsanfo] The plural form of \texttt{editoranfo}.
\item[editoranaf] Used if \bibfield{editor}\slash \bibfield{annotator}\slash \bibfield{aftword} are identical.
\item[editorsanaf] The plural form of \texttt{editoranaf}.
\end{keymarglist}
%
Keys for \bibfield{editor}\slash \bibfield{translator}\slash \bibfield{commentator}\slash \prm{role} combinations:

\begin{keymarglist}
\item[editortrcoin] Used if \bibfield{editor}\slash \bibfield{translator}\slash \bibfield{commentator}\slash \bibfield{introduction} are identical.
\item[editorstrcoin] The plural form of \texttt{editortrcoin}.
\item[editortrcofo] Used if \bibfield{editor}\slash \bibfield{translator}\slash \bibfield{commentator}\slash \bibfield{foreword} are identical.
\item[editorstrcofo] The plural form of \texttt{editortrcofo}.
\item[editortrcoaf] Used if \bibfield{editor}\slash \bibfield{translator}\slash \bibfield{commentator}\slash \bibfield{aftword} are identical.
\item[editorstrcoaf] The plural form of \texttt{editortrcoaf}.
\end{keymarglist}
%
Keys for \bibfield{editor}\slash \bibfield{annotator}\slash \bibfield{commentator}\slash \prm{role} combinations:

\begin{keymarglist}
\item[editortranin] Used if \bibfield{editor}\slash \bibfield{annotator}\slash \bibfield{commentator}\slash \bibfield{introduction} are identical.
\item[editorstranin] The plural form of \texttt{editortranin}.
\item[editortranfo] Used if \bibfield{editor}\slash \bibfield{annotator}\slash \bibfield{commentator}\slash \bibfield{foreword} are identical.
\item[editorstranfo] The plural form of \texttt{editortranfo}.
\item[editortranaf] Used if \bibfield{editor}\slash \bibfield{annotator}\slash \bibfield{commentator}\slash \bibfield{aftword} are identical.
\item[editorstranaf] The plural form of \texttt{editortranaf}.
\end{keymarglist}

\paragraph{Concatenated Translator Roles, Expressed as Functions}
\label{aut:lng:key:ctf}

The following keys are similar in function to \texttt{translator}. They are used to indicate additional roles of the translator, \eg\ <translator and commentator>, <translator and introduction>.

\begin{keymarglist}
\item[translatorco] Used if \bibfield{translator}\slash \bibfield{commentator} are identical.
\item[translatorsco] The plural form of \texttt{translatorco}.
\item[translatoran] Used if \bibfield{translator}\slash \bibfield{annotator} are identical.
\item[translatorsan] The plural form of \texttt{translatoran}.
\item[translatorin] Used if \bibfield{translator}\slash \bibfield{introduction} are identical.
\item[translatorsin] The plural form of \texttt{translatorin}.
\item[translatorfo] Used if \bibfield{translator}\slash \bibfield{foreword} are identical.
\item[translatorsfo] The plural form of \texttt{translatorfo}.
\item[translatoraf] Used if \bibfield{translator}\slash \bibfield{aftword} are identical.
\item[translatorsaf] The plural form of \texttt{translatoraf}.
\end{keymarglist}
%
Keys for \bibfield{translator}\slash \bibfield{commentator}\slash \prm{role} combinations:

\begin{keymarglist}
\item[translatorcoin] Used if \bibfield{translator}\slash \bibfield{commentator}\slash \bibfield{introduction} are identical.
\item[translatorscoin] The plural form of \texttt{translatorcoin}.
\item[translatorcofo] Used if \bibfield{translator}\slash \bibfield{commentator}\slash \bibfield{foreword} are identical.
\item[translatorscofo] The plural form of \texttt{translatorcofo}.
\item[translatorcoaf] Used if \bibfield{translator}\slash \bibfield{commentator}\slash \bibfield{aftword} are identical.
\item[translatorscoaf] The plural form of \texttt{translatorcoaf}.
\end{keymarglist}
%
Keys for \bibfield{translator}\slash \bibfield{annotator}\slash \prm{role} combinations:

\begin{keymarglist}
\item[translatoranin] Used if \bibfield{translator}\slash \bibfield{annotator}\slash \bibfield{introduction} are identical.
\item[translatorsanin] The plural form of \texttt{translatoranin}.
\item[translatoranfo] Used if \bibfield{translator}\slash \bibfield{annotator}\slash \bibfield{foreword} are identical.
\item[translatorsanfo] The plural form of \texttt{translatoranfo}.
\item[translatoranaf] Used if \bibfield{translator}\slash \bibfield{annotator}\slash \bibfield{aftword} are identical.
\item[translatorsanaf] The plural form of \texttt{translatoranaf}.
\end{keymarglist}

\paragraph{Roles, Expressed as Actions}
\label{aut:lng:key:eac}

The following keys refer to roles which are expressed as an action (<edited by>, <translated by>) rather than as a function (<editor>, <translator>).

\begin{keymarglist}
\item[byauthor] The expression <[created] by \prm{name}>.
\item[byeditor] The expression <edited by \prm{name}>.
\item[bycompiler] The expression <compiled by \prm{name}>.
\item[byfounder] The expression <founded by \prm{name}>.
\item[bycontinuator] The expression <continued by \prm{name}>.
\item[byredactor] The expression <redacted by \prm{name}>.
\item[byreviser] The expression <revised by \prm{name}>.
\item[byreviewer] The expression <reviewed by \prm{name}>.
\item[bycollaborator] An expression like <in collaboration with \prm{name}> or <in cooperation with \prm{name}>.
\item[bytranslator] The expression <translated by \prm{name}> or <translated from \prm{language} by \prm{name}>.
\item[bycommentator] The expression <commented by \prm{name}>.
\item[byannotator] The expression <annotated by \prm{name}>.
\end{keymarglist}

\paragraph{Concatenated Editor Roles, Expressed as Actions}
\label{aut:lng:key:cea}

The following keys are similar in function to \texttt{byeditor}, \texttt{bytranslator}, etc. They are used to indicate additional roles of the editor, \eg\ <edited and translated by>, <edited and furnished with an introduction by>, <edited, with a foreword, by>.

\begin{keymarglist}
\item[byeditortr] Used if \bibfield{editor}\slash \bibfield{translator} are identical.
\item[byeditorco] Used if \bibfield{editor}\slash \bibfield{commentator} are identical.
\item[byeditoran] Used if \bibfield{editor}\slash \bibfield{annotator} are identical.
\item[byeditorin] Used if \bibfield{editor}\slash \bibfield{introduction} are identical.
\item[byeditorfo] Used if \bibfield{editor}\slash \bibfield{foreword} are identical.
\item[byeditoraf] Used if \bibfield{editor}\slash \bibfield{aftword} are identical.
\end{keymarglist}
%
Keys for \bibfield{editor}\slash \bibfield{translator}\slash \prm{role} combinations:

\begin{keymarglist}
\item[byeditortrco] Used if \bibfield{editor}\slash \bibfield{translator}\slash \bibfield{commentator} are identical.
\item[byeditortran] Used if \bibfield{editor}\slash \bibfield{translator}\slash \bibfield{annotator} are identical.
\item[byeditortrin] Used if \bibfield{editor}\slash \bibfield{translator}\slash \bibfield{introduction} are identical.
\item[byeditortrfo] Used if \bibfield{editor}\slash \bibfield{translator}\slash \bibfield{foreword} are identical.
\item[byeditortraf] Used if \bibfield{editor}\slash \bibfield{translator}\slash \bibfield{aftword} are identical.
\end{keymarglist}
%
Keys for \bibfield{editor}\slash \bibfield{commentator}\slash \prm{role} combinations:

\begin{keymarglist}
\item[byeditorcoin] Used if \bibfield{editor}\slash \bibfield{commentator}\slash \bibfield{introduction} are identical.
\item[byeditorcofo] Used if \bibfield{editor}\slash \bibfield{commentator}\slash \bibfield{foreword} are identical.
\item[byeditorcoaf] Used if \bibfield{editor}\slash \bibfield{commentator}\slash \bibfield{aftword} are identical.
\end{keymarglist}
%
Keys for \bibfield{editor}\slash \bibfield{annotator}\slash \prm{role} combinations:

\begin{keymarglist}
\item[byeditoranin] Used if \bibfield{editor}\slash \bibfield{annotator}\slash \bibfield{introduction} are identical.
\item[byeditoranfo] Used if \bibfield{editor}\slash \bibfield{annotator}\slash \bibfield{foreword} are identical.
\item[byeditoranaf] Used if \bibfield{editor}\slash \bibfield{annotator}\slash \bibfield{aftword} are identical.
\end{keymarglist}
%
Keys for \bibfield{editor}\slash \bibfield{translator}\slash \bibfield{commentator}\slash \prm{role} combinations:

\begin{keymarglist}
\item[byeditortrcoin] Used if \bibfield{editor}\slash \bibfield{translator}\slash \bibfield{commentator}\slash \bibfield{introduction} are identical.
\item[byeditortrcofo] Used if \bibfield{editor}\slash \bibfield{translator}\slash \bibfield{commentator}\slash \bibfield{foreword} are identical.
\item[byeditortrcoaf] Used if \bibfield{editor}\slash \bibfield{translator}\slash \bibfield{commentator}\slash \bibfield{aftword} are identical.
\end{keymarglist}
%
Keys for \bibfield{editor}\slash \bibfield{translator}\slash \bibfield{annotator}\slash \prm{role} combinations:

\begin{keymarglist}
\item[byeditortranin] Used if \bibfield{editor}\slash \bibfield{annotator}\slash \bibfield{commentator}\slash \bibfield{introduction} are identical.
\item[byeditortranfo] Used if \bibfield{editor}\slash \bibfield{annotator}\slash \bibfield{commentator}\slash \bibfield{foreword} are identical.
\item[byeditortranaf] Used if \bibfield{editor}\slash \bibfield{annotator}\slash \bibfield{commentator}\slash \bibfield{aftword} are identical.
\end{keymarglist}

\paragraph{Concatenated Translator Roles, Expressed as Actions}
\label{aut:lng:key:cta}

The following keys are similar in function to \texttt{bytranslator}. They are used to indicate additional roles of the translator, \eg\ <translated and commented by>, <translated and furnished with an introduction by>, <translated, with a foreword, by>.

\begin{keymarglist}
\item[bytranslatorco] Used if \bibfield{translator}\slash \bibfield{commentator} are identical.
\item[bytranslatoran] Used if \bibfield{translator}\slash \bibfield{annotator} are identical.
\item[bytranslatorin] Used if \bibfield{translator}\slash \bibfield{introduction} are identical.
\item[bytranslatorfo] Used if \bibfield{translator}\slash \bibfield{foreword} are identical.
\item[bytranslatoraf] Used if \bibfield{translator}\slash \bibfield{aftword} are identical.
\end{keymarglist}
%
Keys for \bibfield{translator}\slash \bibfield{commentator}\slash \prm{role} combinations:

\begin{keymarglist}
\item[bytranslatorcoin] Used if \bibfield{translator}\slash \bibfield{commentator}\slash \bibfield{introduction} are identical.
\item[bytranslatorcofo] Used if \bibfield{translator}\slash \bibfield{commentator}\slash \bibfield{foreword} are identical.
\item[bytranslatorcoaf] Used if \bibfield{translator}\slash \bibfield{commentator}\slash \bibfield{aftword} are identical.
\end{keymarglist}
%
Keys for \bibfield{translator}\slash \bibfield{annotator}\slash \prm{role} combinations:

\begin{keymarglist}
\item[bytranslatoranin] Used if \bibfield{translator}\slash \bibfield{annotator}\slash \bibfield{introduction} are identical.
\item[bytranslatoranfo] Used if \bibfield{translator}\slash \bibfield{annotator}\slash \bibfield{foreword} are identical.
\item[bytranslatoranaf] Used if \bibfield{translator}\slash \bibfield{annotator}\slash \bibfield{aftword} are identical.
\end{keymarglist}

\paragraph{Roles, Expressed as Objects}
\label{aut:lng:key:rob}

Roles which are related to supplementary material may also be expressed as objects (<with a commentary by>) rather than as functions (<commentator>) or as actions (<commented by>).

\begin{keymarglist}
\item[withcommentator] The expression <with a commentary by \prm{name}>.
\item[withannotator] The expression <with annotations by \prm{name}>.
\item[withintroduction] The expression <with an introduction by \prm{name}>.
\item[withforeword] The expression <with a foreword by \prm{name}>.
\item[withafterword] The expression <with an afterword by \prm{name}>.
\end{keymarglist}

\paragraph{Supplementary Material}
\label{aut:lng:key:mat}

\begin{keymarglist}
\item[commentary] The term <commentary>.
\item[annotations] The term <annotations>.
\item[introduction] The term <introduction>.
\item[foreword] The term <foreword>.
\item[afterword] The term <afterword>.
\end{keymarglist}

\paragraph{Publication Details}
\label{aut:lng:key:pdt}

\begin{keymarglist}
\item[volume] The term <volume>, referring to a book.
\item[volumes] The plural form of \texttt{volume}.
\item[involumes] The term <in>, as used in expressions like <in \prm{number of volumes} volumes>.
\item[jourvol] The term <volume>, referring to a journal.
\item[jourser] The term <series>, referring to a journal.
\item[book] The term <book>, referring to a document division.
\item[part] The term <part>, referring to a part of a book or a periodical.
\item[issue] The term <issue>, referring to a periodical.
\item[newseries] The expression <new series>, referring to a journal.
\item[oldseries] The expression <old series>, referring to a journal.
\item[edition] The term <edition>.
\item[in] The term <in>, referring to the title of a work published as part of another one, \eg\ <\prm{title of article} in \prm{title of journal}>.
\item[inseries] The term <in>, as used in expressions like <volume \prm{number} in \prm{name of series}>.
\item[ofseries] The term <of>, as used in expressions like <volume \prm{number} of \prm{name of series}>.
\item[number] The term <number>, referring to an issue of a journal.
\item[chapter] The term <chapter>, referring to a chapter in a book.
\item[version] The term <version>, referring to a revision number.
\item[reprint] The term <reprint>.
\item[reprintof] The expression <reprint of \prm{title}>.
\item[reprintas] The expression <reprinted as \prm{title}>.
\item[reprintfrom] The expression <reprinted from \prm{title}>.
\item[translationof] The expression <translation of \prm{title}>.
\item[translationas] The expression <translated as \prm{title}>.
\item[translationfrom] The expression <translated from [the] \prm{language}>.
\item[reviewof] The expression <review of \prm{title}>.
\item[origpubas] The expression <originally published as \prm{title}>.
\item[origpubin] The expression <originally published in \prm{year}>.
\item[astitle] The term <as>, as used in expressions like <published by \prm{publisher} as \prm{title}>.
\item[bypublisher] The term <by>, as used in expressions like <published by \prm{publisher}>.
\end{keymarglist}

\paragraph{Publication State}
\label{aut:lng:key:pst}

\begin{keymarglist}
\item[inpreparation] The expression <in preparation> (the manuscript is being prepared for publication).
\item[submitted] The expression <submitted> (the manuscript has been submitted to a journal or conference).
\item[forthcoming] The expression <forthcoming> (the manuscript has been accepted by a press or journal).
\item[inpress] The expression <in press> (the manuscript is fully copyedited and out of the author's hands; it is in the final stages of the production process).
\item[prepublished] The expression <pre-published> (the manuscript is published in a preliminary form or location, such as online version in advance of print publication).
\end{keymarglist}

\paragraph{Pagination}
\label{aut:lng:key:pag}

\begin{keymarglist}
\item[page] The term <page>.
\item[pages] The plural form of \texttt{page}.
\item[column] The term <column>, referring to a column on a page.
\item[columns] The plural form of \texttt{column}.
\item[section] The term <section>, referring to a document division (usually abbreviated as \S).
\item[sections] The plural form of \texttt{section} (usually abbreviated as \S\S).
\item[paragraph] The term <paragraph> (\ie a block of text, not to be confused with \texttt{section}).
\item[paragraphs] The plural form of \texttt{paragraph}.
\item[verse] The term <verse> as used when referring to a work which is cited by verse numbers.
\item[verses] The plural form of \texttt{verse}.
\item[line] The term <line> as used when referring to a work which is cited by line numbers.
\item[lines] The plural form of \texttt{line}.
\end{keymarglist}

\paragraph{Types}
\label{aut:lng:key:typ}

The following keys are typically used in the \bibfield{type} field of \bibtype{thesis}, \bibtype{report}, \bibtype{misc}, and other entries:

\begin{keymarglist}
\item[mathesis] An expression equivalent to the term <Master's thesis>.
\item[phdthesis] The term <PhD thesis>, <PhD dissertation>, <doctoral thesis>, etc.
\item[candthesis] An expression equivalent to the term <Candidate thesis>. Used for <Candidate> degrees that have no clear equivalent to the Master's or doctoral level.
\item[techreport] The term <technical report>.
\item[resreport] The term <research report>.
\item[software] The term <computer software>.
\item[datacd] The term <data \textsc{cd}> or <\textsc{cd-rom}>.
\item[audiocd] The term <audio \textsc{cd}>.
\end{keymarglist}

\paragraph{Miscellaneous}
\label{aut:lng:key:msc}

\begin{keymarglist}
\item[nodate] The term to use in place of a date when there is no date for an entry \eg\ <n.d.>
\item[and] The term <and>, as used in a list of authors or editors, for example.
\item[andothers] The expression <and others> or <et alii>, used to mark the truncation of a name list.
\item[andmore] Like \texttt{andothers} but used to mark the truncation of a literal list.
\end{keymarglist}

\paragraph{Labels}
\label{aut:lng:key:lab}

The following strings are intended for use as labels, \eg\ <Address: \prm{url}> or <Abstract: \prm{abstract}>.

\begin{keymarglist}
\item[url] The term <address> in the sense of an internet address.
\item[urlfrom] An expression like <available from \prm{url}> or <available at \prm{url}>.
\item[urlseen] An expression like <accessed on \prm{date}>, <retrieved on \prm{date}>, <visited on \prm{date}>, referring to the access date of an online resource.
\item[file] The term <file>.
\item[library] The term <library>.
\item[abstract] The term <abstract>.
\item[annotation] The term <annotations>.
\end{keymarglist}

\paragraph{Citations}
\label{aut:lng:key:cit}

Traditional scholarly expressions used in citations:

\begin{keymarglist}
\item[idem] The term equivalent to the Latin <idem> (<the same [person]>).
\item[idemsf] The feminine singular form of \texttt{idem}.
\item[idemsm] The masculine singular form of \texttt{idem}.
\item[idemsn] The neuter singular form of \texttt{idem}.
\item[idempf] The feminine plural form of \texttt{idem}.
\item[idempm] The masculine plural form of \texttt{idem}.
\item[idempn] The neuter plural form of \texttt{idem}.
\item[idempp] The plural form of \texttt{idem} suitable for a mixed gender list of names.
\item[ibidem] The term equivalent to the Latin <ibidem> (<in the same place>).
\item[opcit] The term equivalent to the Latin term <opere citato> (<[in] the work [already] cited>).
\item[loccit] The term equivalent to the Latin term <loco citato> (<[at] the place [already] cited>).
\item[confer] The term equivalent to the Latin <confer> (<compare>).
\item[sequens] The term equivalent to the Latin <sequens> (<[and] the following [page]>), as used to indicate a range of two pages when only the starting page is provided (\eg\ <25\,sq.> or <25\,f.> instead of <25--26>).
\item[sequentes] The term equivalent to the Latin <sequentes> (<[and] the following [pages]>), as used to indicate an open"=ended range of pages when only the starting page is provided (\eg\ <25\,sqq.> or <25\,ff.>).
\item[passim] The term equivalent to the Latin <passim> (<throughout>, <here and there>, <scatteredly>).
\end{keymarglist}
%
Other expressions frequently used in citations:

\begin{keymarglist}
\item[see] The term <see>.
\item[seealso] The expression <see also>.
\item[seenote] An expression like <see note \prm{footnote}> or <as in \prm{footnote}>, used to refer to a previous footnote in a citation.
\item[backrefpage] An expression like <see page \prm{page}> or <cited on page \prm{page}>, used to introduce back references in the bibliography.
\item[backrefpages] The plural form of \texttt{backrefpage}, \eg\ <see pages \prm{pages}> or <cited on pages \prm{pages}>.
\item[quotedin] An expression like <quoted in \prm{citation}>, used when quoting a passage which was already a quotation in the cited work.
\item[citedas] An expression like <henceforth cited as \prm{shorthand}>, used to introduce a shorthand in a citation.
\item[thiscite] The expression used in some verbose citation styles to differentiate between the page range of the cited item (typically an article in a journal, collection, or conference proceedings) and the page number the citation refers to. For example: \enquote{Author, Title, in: Book, pp. 45--61, \texttt{thiscite} p. 52.}
\end{keymarglist}

\paragraph{Month Names}
\label{aut:lng:key:mon}

\begin{keymarglist}
\item[january] The name <January>.
\item[february] The name <February>.
\item[march] The name <March>.
\item[april] The name <April>.
\item[may] The name <May>.
\item[june] The name <June>.
\item[july] The name <July>.
\item[august] The name <August>.
\item[september] The name <September>.
\item[october] The name <October>.
\item[november] The name <November>.
\item[december] The name <December>.
\end{keymarglist}

\paragraph{Language Names}
\label{aut:lng:key:lng}

\begin{keymarglist}
\item[langamerican] The language <American> or <American English>.
\item[langbrazilian] The language <Brazilian> or <Brazilian Portuguese>.
\item[langcatalan] The language <Catalan>.
\item[langcroatian] The language <Croatian>.
\item[langczech] The language <Czech>.
\item[langdanish] The language <Danish>.
\item[langdutch] The language <Dutch>.
\item[langenglish] The language <English>.
\item[langestonian] The language <Estonian>.
\item[langfinnish] The language <Finnish>.
\item[langfrench] The language <French>.
\item[langgerman] The language <German>.
\item[langgreek] The language <Greek>.
\item[langitalian] The language <Italian>.
\item[langjapanese] The language <Japanese>.
\item[langlatin] The language <Latin>.
\item[langnorwegian] The language <Norwegian>.
\item[langpolish] The language <Polish>.
\item[langportuguese] The language <Portuguese>.
\item[langrussian] The language <Russian>.
\item[langslovak] The language <Slovak>.
\item[langslovene] The language <Slovene>.
\item[langspanish] The language <Spanish>.
\item[langswedish] The language <Swedish>.
\end{keymarglist}
%
The following strings are intended for use in phrases like <translated from [the] English by \prm{translator}>:

\begin{keymarglist}
\item[fromamerican] The expression <from [the] American> or <from [the] American English>.
\item[frombrazilian] The expression <from [the] Brazilian> or <from [the] Brazilian Portuguese>.
\item[fromcatalan] The expression <from [the] Catalan>.
\item[fromcroatian] The expression <from [the] Croatian>.
\item[fromczech] The expression <from [the] Czech>.
\item[fromdanish] The expression <from [the] Danish>.
\item[fromdutch] The expression <from [the] Dutch>.
\item[fromenglish] The expression <from [the] English>.
\item[fromestonian] The expression <from [the] Estonian>.
\item[fromfinnish] The expression <from [the] Finnish>.
\item[fromfrench] The expression <from [the] French>.
\item[fromgerman] The expression <from [the] German>.
\item[fromgreek] The expression <from [the] Greek>.
\item[fromitalian] The expression <from [the] Italian>.
\item[fromjapanese] The expression <from [the] Japanese>.
\item[fromlatin] The expression <from [the] Latin>.
\item[fromnorwegian] The expression <from [the] Norwegian>.
\item[frompolish] The expression <from [the] Polish>.
\item[fromportuguese] The expression <from [the] Portuguese>.
\item[fromrussian] The expression <from [the] Russian>.
\item[fromslovak] The expression <from [the] Slovak>.
\item[fromslovene] The expression <from [the] Slovene>.
\item[fromspanish] The expression <from [the] Spanish>.
\item[fromswedish] The expression <from [the] Swedish>.
\end{keymarglist}

\paragraph{Country Names}
\label{aut:lng:key:cnt}

Country names are localised by using the string \texttt{country} plus the \acr{ISO}-3166 country code as the key. The short version of the translation should be the \acr{ISO}-3166 country code. Note that only a small number of country names is defined by default, mainly to illustrate this scheme. These keys are used in the \bibfield{location} list of \bibtype{patent} entries but they may be useful for other purposes as well.

\begin{keymarglist}
\item[countryde] The name <Germany>, abbreviated as \texttt{DE}.
\item[countryeu] The name <European Union>, abbreviated as \texttt{EU}.
\item[countryep] Similar to \opt{countryeu} but abbreviated as \texttt{EP}. This is intended for \bibfield{patent} entries.
\item[countryfr] The name <France>, abbreviated as \texttt{FR}.
\item[countryuk] The name <United Kingdom>, abbreviated (according to \acr{ISO}-3166) as \texttt{GB}.
\item[countryus] The name <United States of America>, abbreviated as \texttt{US}.
\end{keymarglist}

\paragraph{Patents and Patent Requests}
\label{aut:lng:key:pat}

Strings related to patents are localised by using the term \texttt{patent} plus the \acr{ISO}-3166 country code as the key. Note that only a small number of patent keys is defined by default, mainly to illustrate this scheme. These keys are used in the \bibfield{type} field of \bibtype{patent} entries.

\begin{keymarglist}
\item[patent] The generic term <patent>.
\item[patentde] The expression <German patent>.
\item[patenteu] The expression <European patent>.
\item[patentfr] The expression <French patent>.
\item[patentuk] The expression <British patent>.
\item[patentus] The expression <U.S. patent>.
\end{keymarglist}
%
Patent requests are handled in a similar way, using the string \texttt{patreq} as the base name of the key:

\begin{keymarglist}
\item[patreq] The generic term <patent request>.
\item[patreqde] The expression <German patent request>.
\item[patreqeu] The expression <European patent request>.
\item[patreqfr] The expression <French patent request>.
\item[patrequk] The expression <British patent request>.
\item[patrequs] The expression <U.S. patent request>.
\end{keymarglist}

\paragraph{Dates and Times}
\label{aut:lng:key:dt}

Abbreviation strings for standard eras. Both secular and Christian variants
are supported.

\begin{keymarglist}
\item[commonera] The era <CE>
\item[beforecommonera] The era <BCE>
\item[annodomini] The era <AD>
\item[beforechrist] The era <BC>
\end{keymarglist}

Abbreviation strings for <circa> dates:

\begin{keymarglist}
\item[circa] The string <circa>
\end{keymarglist}

Abbreviation strings for seasons parsed from \acr{EDTF} dates:

\begin{keymarglist}
\item[spring] The string <spring>
\item[summer] The string <summer>
\item[autumn] The string <autumn>
\item[winter] The string <winter>
\end{keymarglist}

Abbreviation strings for AM/PM:

\begin{keymarglist}
\item[am] The string <AM>
\item[pm] The string <PM>
\end{keymarglist}

\subsection{Formatting Commands}
\label{aut:fmt}

This section corresponds to \secref{use:fmt} in the user part of this manual. Bibliography and citation styles should incorporate the commands and facilities discussed in this section in order to provide a certain degree of high"=level configurability. Users should not be forced to write new styles if all they want to do is modify the spacing in the bibliography or the punctuation used in citations.

\subsubsection{User-definable Commands and Hooks}
\label{aut:fmt:fmt}

This section corresponds to \secref{use:fmt:fmt} in the user part of the manual. The commands and hooks discussed here are meant to be redefined by users, but bibliography and citation styles may provide a default definition which is different from the package default. These commands are defined in \path{biblatex.def}. Note that all commands starting with \cmd{mk\dots} take one mandatory argument.

\begin{ltxsyntax}

\csitem{bibnamedelima}
This delimiter controls the spacing between the elements which make up a name part. It is inserted automatically by the backend after the first name element if the element is less than three characters long and before the last element. The default definition is \cmd{addhighpenspace}, \ie a space penalized by the value of the \cnt{highnamepenalty} counter (\secref{use:fmt:len}). Please refer to \secref{use:cav:nam} for further details.

\csitem{bibnamedelimb}
This delimiter controls the spacing between the elements which make up a name part. It is inserted automatically by the backend between all name elements where \cmd{bibnamedelima} does not apply. The default definition is \cmd{addlowpenspace}, \ie a space penalized by the value of the \cnt{lownamepenalty} counter (\secref{use:fmt:len}). Please refer to \secref{use:cav:nam} for further details.

\csitem{bibnamedelimc}
This delimiter controls the spacing between name parts. The default name formats use it between the name prefix and the last name if \kvopt{useprefix}{true}. The default definition is \cmd{addhighpenspace}, \ie a space penalized by the value of the \cnt{highnamepenalty} counter (\secref{use:fmt:len}). Please refer to \secref{use:cav:nam} for further details.

\csitem{bibnamedelimd}
This delimiter controls the spacing between name parts. The default name formats use it between all name parts where \cmd{bibnamedelimc} does not apply. The default definition is \cmd{addlowpenspace}, \ie a space penalized by the value of the \cnt{lownamepenalty} counter (\secref{use:fmt:len}). Please refer to \secref{use:cav:nam} for further details.

\csitem{bibnamedelimi}
This delimiter replaces \cmd{bibnamedelima/b} after initials. Note that this only applies to initials given as such in the \file{bib} file, not to the initials automatically generated by \biblatex which use their own set of delimiters.

\csitem{bibinitperiod}
The punctuation inserted automatically by the backend after all initials unless \cmd{bibinithyphendelim} applies. The default definition is a period (\cmd{adddot}). Please refer to \secref{use:cav:nam} for further details.

\csitem{bibinitdelim}
The spacing inserted automatically by the backend between multiple initials unless \cmd{bibinithyphendelim} applies. The default definition is an unbreakable interword space. Please refer to \secref{use:cav:nam} for further details.

\csitem{bibinithyphendelim}
The punctuation inserted automatically by the backend between the initials of hyphenated name parts, replacing \cmd{bibinitperiod} and \cmd{bibinitdelim}. The default definition is a period followed by an unbreakable hyphen. Please refer to \secref{use:cav:nam} for further details.

\csitem{bibindexnamedelima}
Replaces \cmd{bibnamedelima} in the index.

\csitem{bibindexnamedelimb}
Replaces \cmd{bibnamedelimb} in the index.

\csitem{bibindexnamedelimc}
Replaces \cmd{bibnamedelimc} in the index.

\csitem{bibindexnamedelimd}
Replaces \cmd{bibnamedelimd} in the index.

\csitem{bibindexnamedelimi}
Replaces \cmd{bibnamedelimi} in the index.

\csitem{bibindexinitperiod}
Replaces \cmd{bibinitperiod} in the index.

\csitem{bibindexinitdelim}
Replaces \cmd{bibinitdelim} in the index.

\csitem{bibindexinithyphendelim}
Replaces \cmd{bibinithyphendelim} in the index.

\csitem{revsdnamepunct}
The punctuation to be printed between the first and last name parts when a name is reversed. The default is a comma. This command should be incorporated in formatting directives for name lists.  Please refer to \secref{use:cav:nam} for further details.

\csitem{bibnamedash}
The dash to be used as a replacement for recurrent authors or editors in the bibliography. The default is an <em> or an <en> dash, depending on the indentation of the list of references.

\csitem{labelnamepunct}
The separator to be printed after the name used for alphabetizing in the bibliography (\bibfield{author} or \bibfield{editor}, if the \bibfield{author} field is undefined). Use this separator instead of \cmd{newunitpunct} at this location. The default is \cmd{newunitpunct}, \ie it is not handled differently from regular unit punctuation but permits convenient reconfiguration.

\csitem{subtitlepunct}
The separator to be printed between the fields \bibfield{title} and \bibfield{subtitle}, \bibfield{booktitle} and \bibfield{booksubtitle}, as well as \bibfield{maintitle} and \bibfield{mainsubtitle}. Use this separator instead of \cmd{newunitpunct} at this location. The default is \cmd{newunitpunct}, \ie it is not handled differently from regular unit punctuation but permits convenient reconfiguration.

\csitem{intitlepunct}
The separator to be printed between the word «in» and the following title in entry types such as \bibtype{article}, \bibtype{inbook}, \bibtype{incollection}, etc. Use this separator instead of \cmd{newunitpunct} at this location. The default definition is a colon plus an interword space.

\csitem{bibpagespunct}
The separator to be printed before the \bibfield{pages} field. Use this separator instead of \cmd{newunitpunct} at this location. The default is a comma plus an interword space.

\csitem{bibpagerefpunct}
The separator to be printed before the \bibfield{pageref} field. Use this separator instead of \cmd{newunitpunct} at this location. The default is an interword space.

\csitem{multinamedelim}
The delimiter to be printed between multiple items in a name list like \bibfield{author} or \bibfield{editor} if there are more than two names in the list. If there are only two names in the list, use the \cmd{finalnamedelim} instead. This command should be incorporated in all formatting directives for name lists.

\csitem{finalnamedelim}
Use this command instead of \cmd{multinamedelim} before the final name in a name list.

\csitem{revsdnamedelim}
The extra delimiter to be printed after the first name in a name list consisting of two names (in addition to \cmd{finalnamedelim}) if the first name is reversed. This command should be incorporated in all formatting directives for name lists.

\csitem{andothersdelim}
The delimiter to be printed before the localisation string <\texttt{andothers}> if a name list like \bibfield{author} or \bibfield{editor} is truncated. This command should be incorporated in all formatting directives for name lists.

\csitem{multilistdelim}
The delimiter to be printed between multiple items in a literal list like \bibfield{publisher} or \bibfield{location} if there are more than two names in the list. If there are only two items in the list, use the \cmd{finallistdelim} instead. This command should be incorporated in all formatting directives for literal lists.

\csitem{finallistdelim}
Use this command instead of \cmd{multilistdelim} before the final item in a literal list.

\csitem{andmoredelim}
The delimiter to be printed before the localisation string <\texttt{andmore}> if a literal list like \bibfield{publisher} or \bibfield{location} is truncated. This command should be incorporated in all formatting directives for literal lists.

\csitem{multicitedelim}
The delimiter printed between citations if multiple entry keys are passed to a single citation command. This command should be incorporated in the definition of all citation commands, for example in the \prm{sepcode} argument passed to \cmd{DeclareCiteCommand}. See \secref{aut:cbx:cbx} for details.

\csitem{supercitedelim}
Similar to \cmd{multinamedelim}, but intended for the \cmd{supercite} command only.

\csitem{compcitedelim}
Similar to \cmd{multicitedelim}, but intended for citation styles that <compress> multiple citations, \ie print the author only once if subsequent citations share the same author etc.

\csitem{textcitedelim}
Similar to \cmd{multicitedelim}, but intended for \cmd{textcite} and related commands (\secref{use:cit:cbx}).

\csitem{nametitledelim}
The delimiter to be printed between the author\slash editor and the title. This command should be incorporated in the definition of all citation commands of author-title and some verbose citation styles.

\csitem{nameyeardelim}
The delimiter to be printed between the author\slash editor and the year. This command should be incorporated in the definition of all citation commands of author-year citation styles.

\csitem{namelabeldelim}
The delimiter printed between the name\slash title and the label. This command should be incorporated in the definition of all citation commands of alphabetic and numeric citation styles.

\csitem{nonameyeardelim}
The delimiter printed between the substitute for the labelname when it does not exist (usually the label or title in standard styles) and the year in author-year citation styles. This is only used when there is no labelname since when the labelname exists, \cmd{nameyeardelim} is used.

\csitem{volcitedelim}
The delimiter to be printed between the volume portion and the page/text portion of \cmd{volcite} and related commands (\secref{use:cit:spc}).

\csitem{prenotedelim}
The delimiter to be printed after the \prm{prenote} argument of a citation command.

\csitem{postnotedelim}
The delimiter to be printed after the \prm{postnote} argument of a citation command.

\csitem{extpostnotedelim}
The delimiter printed between the citation and the parenthetical \prm{postnote} argument of a citation command when the postnote occurs outside of the citation parentheses. In the standard styles, this occurs when the citation uses the shorthand field of the entry.

\cmditem{mkbibnamefamily}{text}
Formatting hook for the family name, to be used in all formatting directives for name lists.

\cmditem{mkbibnamegiven}{text}
Similar to \cmd{mkbibnamefamily}, but intended for the given name.

\cmditem{mkbibnameprefix}{text}
Similar to \cmd{mkbibnamefamily}, but intended for the name prefix.

\cmditem{mkbibnamesuffix}{text}
Similar to \cmd{mkbibnamefamily}, but intended for the name suffix.

\csitem{relatedpunct}
The separator between the relatedtype bibliography localisation string and the data from the first related entry.

\csitem{relateddelim}
The separator between the data of multiple related entries. The default definition is a linebreak.

\csitem{relateddelim$<$relatedtype$>$}
The separator between the data of multiple related entries inside related entries of type <relatedtype>. There is no default, if such a type-specific delimiter does not exist, \cmd{relateddelim} is used.

\end{ltxsyntax}

\subsubsection{Language-specific Commands}
\label{aut:fmt:lng}

This section corresponds to \secref{use:fmt:lng} in the user part of the manual. The commands discussed here are usually handled by the localisation modules, but may also be redefined by users on a per"=language basis. Note that all commands starting with \cmd{mk\dots} take one or more mandatory arguments.

\begin{ltxsyntax}

\csitem{bibrangedash}

The language specific range dash. Defaults to \cmd{textendash}.

\csitem{bibrangessep}

The language specific separator to be used between multiple ranges. Defaults to a comma followed by a space.

\csitem{bibdatesep}

The language specific separator used between date components in terse date formats. Defaults to \cmd{hyphen}.

\csitem{bibdaterangesep}

The language specific separator to be used for date ranges. Defaults to \cmd{textendash} for all date formats apart from \opt{ymd} which defaults to a \cmd{slash}. The date format option \opt{edtf} is hard-coded to \cmd{slash} since this is a standards compliant format.

\csitem{mkbibdatelong}

Takes the names of three field as arguments which correspond to three date components (in the order year\slash month\slash day) and uses their values to print the date in the language specific long date format.

\csitem{mkbibdateshort}

Similar to \cmd{mkbibdatelong} but using the language specific short date format.

\csitem{mkbibtimezone}

Modifies a timezone string passed in as the only argument. By default this changes <Z> to the value of \cmd{bibtimezone}.

\csitem{bibdateuncertain}

The language specific marker to be used after uncertain dates when the global option \opt{dateuncertain} is enabled. Defaults to a space followed by a question mark.


\csitem{bibdateeraprefix}

The language specific marker which is printed as a prefix to beginning BCE/BC dates in a date range when the option \opt{dateera} is set to <astronomical>. Defaults to \cmd{textminus}, if defined and \cmd{textendash} otherwise.

\csitem{bibdateeraendprefix}

The language specific marker which is printed as a prefix to end BCE/BC dates in a date range when the option \opt{dateera} is set to <astronomical>. Defaults to a thin space followed by \cmd{bibdateeraprefix} when \cmd{bibdaterangesep} is set to a dash and to \cmd{bibdateeraprefix} otherwise.  This is a separate macro so that you may add extra space before a negative date marker which, for example follows a dash date range marker as this can look a little odd.

\csitem{bibtimesep}

The language specific marker which separates time components. Default to a colon.

\csitem{bibutctimezone}

The language specific string printed for the UTC timezone. Defaults to <Z>.

\csitem{bibtimezonesep}

The language specific marker which separates an optional time zone component from a time. Empty by default.

\csitem{bibdatetimesep}

The language specific separator printed between date and time components when printing time components along with date components (see the \opt{$<$datetype$>$dateusetime} option in \secref{use:opt:pre:gen}). Defaults to a space for non-EDTF output formats, and 'T' for EDTF output format.

\csitem{finalandcomma}

Prints the comma to be inserted before the final <and> in an enumeration, if applicable in the respective language.

\csitem{finalandsemicolon}

Prints the semicolon to be inserted before the final <and> in an enumeration, if applicable in the respective language.

\cmditem{mkbibordinal}{integer}

Takes an integer argument and prints it as an ordinal number.

\cmditem{mkbibmascord}{integer}

Similar to \cmd{mkbibordinal}, but prints a masculine ordinal, if applicable in the respective language.

\cmditem{mkbibfemord}{integer}

Similar to \cmd{mkbibordinal}, but prints a feminine ordinal, if applicable in the respective language.

\cmditem{mkbibneutord}{integer}

Similar to \cmd{mkbibordinal}, but prints a neuter ordinal, if applicable in the respective language.

\cmditem{mkbibordedition}{integer}

Similar to \cmd{mkbibordinal}, but intended for use with the term <edition>.

\cmditem{mkbibordseries}{integer}

Similar to \cmd{mkbibordinal}, but intended for use with the term <series>.

\end{ltxsyntax}

\subsubsection{用户可定义的尺寸和计数器 User-definable Lengths and Counters}
\label{aut:fmt:len}

This section corresponds to \secref{use:fmt:len} in the user part of the manual. The length registers and counters discussed here are meant to be altered by users. Bibliography and citation styles should incorporate them where applicable and may also provide a default setting which is different from the package default.

\begin{ltxsyntax}

\lenitem{bibhang}

The hanging indentation of the bibliography, if applicable. This length is initialized to \cmd{parindent} at load-time. If \cmd{parindent} is zero length for some reason, \cmd{bibhang} will default to \texttt{1em}.

\lenitem{biblabelsep}

The horizontal space between entries and their corresponding labels. Bibliography styles which use \env{list} environments and print a label should set \len{labelsep} to \len{biblabelsep} in the definition of the respective environment.

\lenitem{bibitemsep}

The vertical space between the individual entries in the bibliography. Bibliography styles using \env{list} environments should set \len{itemsep} to \len{bibitemsep} in the definition of the respective environment.

\lenitem{bibparsep}

The vertical space between paragraphs within an entry in the bibliography. Bibliography styles using \env{list} environments should set \len{parsep} to \len{bibparsep} in the definition of the respective environment.

\cntitem{abbrvpenalty}

The penalty used by \cmd{addabbrvspace}, \cmd{addabthinspace}, and \cmd{adddotspace}, see \secref{aut:pct:spc} for details.

\cntitem{lownamepenalty}

The penalty used by \cmd{addlowpenspace} and \cmd{addlpthinspace}, see \secref{aut:pct:spc} for details.

\cntitem{highnamepenalty}

The penalty used by \cmd{addhighpenspace} and \cmd{addhpthinspace}, see \secref{aut:pct:spc} for details.

\cntitem{biburlnumpenalty}

If this counter is set to a value greater than zero, \biblatex will permit linebreaks after numbers in all strings formatted with the \cmd{url} command from the \sty{url} package. This will affect \acr{url}s and \acr{doi}s in the bibliography. The breakpoints will be penalized by the value of this counter. If \acr{url}s and/or \acr{doi}s in the bibliography run into the margin, try setting this counter to a value greater than zero but less than 10000 (you normally want to use a high value like 9000). Setting the counter to zero disables this feature. This is the default setting.

\cntitem{biburlucpenalty}

Similar to \cnt{biburlnumpenalty}, except that it will add a breakpoint after all uppercase letters.

\cntitem{biburllcpenalty}

Similar to \cnt{biburlnumpenalty}, except that it will add a breakpoint after all lowercase letters.

\end{ltxsyntax}

\subsubsection{辅助命令和钩子 Auxiliary Commands and Hooks}
\label{aut:fmt:ich}

The auxiliary commands and facilities in this section serve a special purpose. Some of them are used by \biblatex to communicate with bibliography and citation styles in some way or other.

\begin{ltxsyntax}

\cmditem{mkbibemph}{text}

A generic command which prints its argument as emphasized text. This is a simple wrapper around the standard \cmd{emph} command. Apart from that, it uses \cmd{setpunctfont} from \secref{aut:pct:new} to adapt the font of the next punctuation mark following the text set in italics. If the \opt{punctfont} package option is disabled, this command behaves like \cmd{emph}.

\cmditem{mkbibitalic}{text}

Similar in concept to \cmd{mkbibemph} but prints italicized text. This is a simple wrapper around the standard \cmd{textit} command which incorporates \cmd{setpunctfont}. If the \opt{punctfont} package option is disabled, this command behaves like \cmd{textit}.

\cmditem{mkbibbold}{text}

Similar in concept to \cmd{mkbibemph} but prints bold text. This is a simple wrapper around the standard \cmd{textbf} command which incorporates \cmd{setpunctfont}. If the \opt{punctfont} package option is disabled, this command behaves like \cmd{textbf}.

\cmditem{mkbibquote}{text}

A generic command which wraps its argument in quotation marks. If the \sty{csquotes} package is loaded, this command uses the language sensitive quotation marks provided by that package. \cmd{mkbibquote} also supports <American-style> punctuation, see \cmd{DeclareQuotePunctuation} in \secref{aut:pct:cfg} for details.

\cmditem{mkbibparens}{text}

A generic command which wraps its argument in parentheses. This command is nestable. When nested, it will alternate between parentheses and brackets, depending on the nesting level.

\cmditem{mkbibbrackets}{text}

A generic command which wraps its argument in square brackets. This command is nestable. When nested, it will alternate between brackets and parentheses, depending on the nesting level.

\cmditem{bibopenparen}<text>|{\ltxsyntaxlabelfont\cmd{bibcloseparen}}|

Alternative syntax for \cmd{mkbibparens}. This will also work across groups. Note that \cmd{bibopenparen} and \cmd{bibcloseparen} must always be balanced.

\cmditem{bibopenbracket}<text>|{\ltxsyntaxlabelfont\cmd{bibclosebracket}}|

Alternative syntax for \cmd{mkbibbrackets}. This will also work across groups. Note that \cmd{bibopenbracket} and \cmd{bibclosebracket} must always be balanced.

\cmditem{mkbibfootnote}{text}

A generic command which prints its argument as a footnote. This is a wrapper around the standard \latex \cmd{footnote} command which removes spurious whitespace preceding the footnote mark and prevents nested footnotes. By default, \cmd{mkbibfootnote} requests capitalization at the beginning of the note and automatically adds a period at the end. You may change this behavior by redefining the \cmd{bibfootnotewrapper} macro introduced below.

\cmditem{mkbibfootnotetext}{text}

Similar to \cmd{mkbibfootnote} but uses the \cmd{footnotetext} command.

\cmditem{mkbibendnote}{text}

Similar in concept to \cmd{mkbibfootnote} except that it prints its argument as an endnote. \cmd{mkbibendnote} removes spurious whitespace preceding the endnote mark and prevents nested notes. It supports the \cmd{endnote} command provided by the \sty{endnotes} package as well as the \cmd{pagenote} command provided by the \sty{pagenote} package and the \sty{memoir} class. If both commands are available, \cmd{endnote} takes precedence. If no endnote support is available, \cmd{mkbibendnote} issues an error and falls back to \cmd{footnote}. By default, \cmd{mkbibendnote} requests capitalization at the beginning of the note and automatically adds a period at the end. You may change this behavior by redefining the \cmd{bibendnotewrapper} macro introduced below.

\cmditem{mkbibendnotetext}{text}

Similar to \cmd{mkbibendnote} but uses the \cmd{endnotetext} command. Please note that as of this writing, neither the \sty{pagenote} package nor the \sty{memoir} class provide a corresponding \cmd{pagenotetext} command. In this case, \cmd{mkbibendnote} will issue an error and fall back to \cmd{footnotetext}.

\cmditem{bibfootnotewrapper}{text}

An inner wrapper which encloses the \prm{text} argument of \cmd{mkbibfootnote} and \cmd{mkbibfootnotetext}. For example, \cmd{mkbibfootnote} eventually boils down to this:

\begin{ltxexample}
\footnote{<<\bibfootnotewrapper{>>text<<}>>}
\end{ltxexample}
%
The wrapper ensures capitalization at the beginning of the note and adds a period at the end. The default definition is:

\begin{ltxexample}
\newcommand{\bibfootnotewrapper}[1]{<<\bibsentence>> #1<<\addperiod>>}
\end{ltxexample}
%
If you don't want capitalization at the beginning or a period at the end of the note, do not modify \cmd{mkbibfootnote} but redefine \cmd{bibfootnotewrapper} instead.

\cmditem{bibendnotewrapper}{text}

Similar in concept to \cmd{bibfootnotewrapper} but related to the \cmd{mkbibendnote} and \cmd{mkbibendnotetext} commands.

\cmditem{mkbibsuperscript}{text}

A generic command which prints its argument as superscripted text. This is a simple wrapper around the standard \latex \cmd{textsuperscript} command which removes spurious whitespace and allows hyphenation of the preceding word.

\cmditem{mkbibmonth}{integer}

This command takes an integer argument and prints it as a month name. Even though the output of this command is language specific, its definition is not, hence it is normally not redefined in localisation modules.

\cmditem{mkbibseason}{string}

This command takes a season localisation string and prints the version of the string corresponding to the setting of the \opt{dateabbrev} package option. Even though the output of this command is language specific, its definition is not, hence it is normally not redefined in localisation modules.

\cmditem{mkyearzeros}{integer}

This command strips leading zeros from a year or enforces them, depending on the \opt{datezeros} package option (\secref{use:opt:pre:gen}). It is intended for use in the definition of date formatting macros.

\cmditem{mkmonthzeros}{integer}

This command strips leading zeros from a month or enforces them, depending on the \opt{datezeros} package option (\secref{use:opt:pre:gen}). It is intended for use in the definition of date formatting macros.

\cmditem{mkdayzeros}{integer}

This command strips leading zeros from a day or enforces them, depending on the \opt{datezeros} package option (\secref{use:opt:pre:gen}). It is intended for use in the definition of date formatting macros.

\cmditem{mktimezeros}{integer}

This command strips leading zeros from a number or preserves them, depending on the \opt{timezeros} package option (\secref{use:opt:pre:gen}). It is intended for use in the definition of time formatting macros.

\cmditem{forcezerosy}{integer}

This command adds zeros to a year (or any number supposed to be 4-digits). It is intended for date formatting and ordinals.

\cmditem{forcezerosmdt}{integer}

This command adds zeros to a month, day or time part (or any number supposed to be 2-digits). It is intended for date/time formatting and ordinals.

\cmditem{stripzeros}{integer}

This command strips leading zeros from a number. It is intended for date formatting and ordinals.

\optitem{$<$labelfield$>$width}

For every field marked as a <Label field> in the data model, a formatting directive is created as per \texttt{shorthandwidth} above. Since \bibfield{shorthand} is so marked in the default data model, this functionality is a superset of that described for \texttt{shorthandwidth}.

\optitem{labelnumberwidth}

Similar to \texttt{shorthandwidth}, but referring to the \bibfield{labelnumber} field and the length register \cmd{labelnumberwidth}. Numeric styles should adjust this directive such that it corresponds to the format used in the bibliography.

\optitem{labelalphawidth}

Similar to \texttt{shorthandwidth}, but referring to the \bibfield{labelalpha} field and the length register \cmd{labelalphawidth}. Alphabetic styles should adjust this directive such that it corresponds to the format used in the bibliography.

\optitem{bibhyperref}

A special formatting directive for use with \cmd{printfield} and \cmd{printtext}. This directive wraps its argument in a \cmd{bibhyperref} command, see \secref{aut:aux:msc} for details.

\optitem{bibhyperlink}

A special formatting directive for use with \cmd{printfield} and \cmd{printtext}. It wraps its argument in a \cmd{bibhyperlink} command, see \secref{aut:aux:msc} for details. The \prm{name} argument passed to \cmd{bibhyperlink} is the value of the \bibfield{entrykey} field.

\optitem{bibhypertarget}

A special formatting directive for use with \cmd{printfield} and \cmd{printtext}. It wraps its argument in a \cmd{bibhypertarget} command, see \secref{aut:aux:msc} for details. The \prm{name} argument passed to \cmd{bibhypertarget} is the value of the \bibfield{entrykey} field.

\optitem{volcitepages}

A special formatting directive which controls the format of the page\slash text portion in the argument of citation commands like \cmd{volcite}.

\optitem{volcitevolume}

A special formatting directive which controls the format of the volume portion in the argument of citation commands like \cmd{volcite}.

\optitem{date}

A special formatting directive which controls the format of \cmd{printdate} (\secref{aut:bib:dat}). Note that the date format (long/short etc.) is controlled by the package option \opt{date} from \secref{use:opt:pre:gen}. This formatting directive only controls additional formatting such as fonts etc.

\optitem{labeldate}

As \texttt{date} but controls the format of \cmd{printlabeldate}.

\optitem{$<$datetype$>$date}

As \texttt{date} but controls the format of \cmd{print$<$datetype$>$date}.

\optitem{time}

A special formatting directive which controls the format of \cmd{printtime} (\secref{aut:bib:dat}). Note that the time format (24h/12h etc.) is controlled by the package option \opt{time} from \secref{use:opt:pre:gen}. This formatting directive only controls additional formatting such as fonts etc.

\optitem{labeltime}

As \texttt{time} but controls the format of \cmd{printlabeltime}.

\optitem{$<$datetype$>$time}

As \texttt{time} but controls the format of \cmd{print$<$datetype$>$time}.

\end{ltxsyntax}

\subsubsection{辅助长度计数器和其它功能 Auxiliary Lengths, Counters, and Other Features}
\label{aut:fmt:ilc}

The length registers and counters discussed here are used by \biblatex to pass information to bibliography and citation styles. Think of them as read"=only registers. Note that all counters are \latex counters. Use |\value{counter}| to read out the current value.

\begin{ltxsyntax}

\lenitem{$<$labelfield$>$width}

For every field marked as a <label> field in the data model, a length register is created as per \texttt{shorthandwidth} above. Since \bibfield{shorthand} is so marked in the default data model, this functionality is a superset of that described for \texttt{shorthandwidth}.

\lenitem{labelnumberwidth}

This length register indicates the width of the widest \bibfield{labelnumber}. Numeric bibliography styles
should incorporate this length in the definition of the bibliography environment.

\lenitem{labelalphawidth}

This length register indicates the width of the widest \bibfield{labelalpha}. Alphabetic bibliography styles should incorporate this length in the definition of the bibliography environment.

\cntitem{maxextraalpha}

This counter holds the highest number found in any \bibfield{extraalpha} field.

\cntitem{maxextrayear}

This counter holds the highest number found in any \bibfield{extrayear} field.

\cntitem{refsection}

This counter indicates the current \env{refsection} environment. When queried in a bibliography heading, the counter returns the value of the \opt{refsection} option passed to \cmd{printbibliography}.

\cntitem{refsegment}

This counter indicates the current \env{refsegment} environment. When queried in a bibliography heading, this counter returns the value of the \opt{refsegment} option passed to \cmd{printbibliography}.

\cntitem{maxnames}

This counter holds the setting of the \opt{maxnames} package option.

\cntitem{minnames}

This counter holds the setting of the \opt{minnames} package option.

\cntitem{maxitems}

This counter holds the setting of the \opt{maxitems} package option.

\cntitem{minitems}

This counter holds the setting of the \opt{minitems} package option.

\cntitem{instcount}

This counter is incremented by \biblatex for every citation as well as for every entry in the bibliography and bibliography lists. The value of this counter uniquely identifies a single instance of a reference in the document.

\cntitem{citetotal}

This counter, which is only available in the \prm{loopcode} of a citation command defined with \cmd{DeclareCiteCommand}, holds the total number of valid entry keys passed to the citation command.

\cntitem{citecount}

This counter, which is only available in the \prm{loopcode} of a citation command defined with \cmd{DeclareCiteCommand}, holds the number of the entry key currently being processed by the \prm{loopcode}.

\cntitem{multicitetotal}

This counter is similar to \cnt{citetotal} but only available in multicite commands. It holds the total number of citations passed to the multicite command. Note that each of these citations may consist of more than one entry key. This information is provided by the \cnt{citetotal} counter.

\cntitem{multicitecount}

This counter is similar to \cnt{citecount} but only available in multicite commands. It holds the number of the citation currently being processed. Note that this citation may consist of more than one entry key. This information is provided by the \cnt{citetotal} and \cnt{citecount} counters.

\cntitem{listtotal}

This counter holds the total number of items in the current list. It is intended for use in list formatting directives and does not hold a meaningful value when used anywhere else. As an exception, it may also be used in the second optional argument to \cmd{printnames} and \cmd{printlist}, see \secref{aut:bib:dat} for details. For every list, there is also a counter by the same name which holds the total number of items in the corresponding list. For example, the \cnt{author} counter holds the total number of items in the \bibfield{author} list. This applies to both name lists and literal lists. These counters are similar to \cnt{listtotal} except that they may also be used independently of list formatting directives. For example, a bibliography style might check the \cnt{editor} counter to decide Whether or not to print the term «editor» or rather its plural form «editors» after the list of editors.

\cntitem{listcount}

This counter holds the number of the list item currently being processed. It is intended for use in list formatting directives and does not hold a meaningful value when used anywhere else.

\cntitem{liststart}

This counter holds the \prm{start} argument passed to \cmd{printnames} or \cmd{printlist}. It is intended for use in list formatting directives and does not hold a meaningful value when used anywhere else.

\cntitem{liststop}

This counter holds the \prm{stop} argument passed to \cmd{printnames} or \cmd{printlist}. It is intended for use in list formatting directives and does not hold a meaningful value when used anywhere else.

\csitem{currentlang}

The name of the currently active language for \biblatex. Can be used anywhere and
defaults to the main document language. This is automatically switched
inside entries which define \bibfield{langid}, given suitable settings of the
\opt{autolang} and \opt{language} options. Note that this does not track
all document language changes, only the current \biblatex\ setting.

\csitem{currentfield}

The name of the field currently being processed by \cmd{printfield}. This information is only available locally in field formatting directives.

\csitem{currentlist}

The name of the literal list currently being processed by \cmd{printlist}. This information is only available locally in list formatting directives.

\csitem{currentname}

The name of the name list currently being processed by \cmd{printnames}. This information is only available locally in name formatting directives.

\end{ltxsyntax}

\subsubsection{多用途钩子 General Purpose Hooks}
\label{aut:fmt:hok}

\begin{ltxsyntax}

\cmditem{AtBeginBibliography}{code}

Appends the \prm{code} to an internal hook executed at the beginning of the bibliography. The \prm{code} is executed at the beginning of the list of references, immediately after the \prm{begin code} of \cmd{defbibenvironment}. This command may only be used in the preamble.

\cmditem{AtBeginShorthands}{code}

Appends the \prm{code} to an internal hook executed at the beginning of the list of shorthands. The \prm{code} is executed at the beginning of the list of shorthands, immediately after the \prm{begin code} of \cmd{defbibenvironment}. This command may only be used in the preamble.

This is just an alias for:

\begin{ltxexample}
\AtBeginBiblist{shorthand}{code}
\end{ltxexample}

\cmditem{AtBeginBiblist}{biblistname}{code}

Appends the \prm{code} to an internal hook executed at the beginning of the bibliography list \prm{biblistname}. The \prm{code} is executed at the beginning of the bibliography list, immediately after the \prm{begin code} of \cmd{defbibenvironment}. This command may only be used in the preamble.

\cmditem{AtEveryBibitem}{code}

Appends the \prm{code} to an internal hook executed at the beginning of every item in the bibliography. The \prm{code} is executed immediately after the \prm{item code} of \cmd{defbibenvironment}. The bibliographic data of the respective entry is available at this point. This command may only be used in the preamble.

\cmditem{AtEveryLositem}{code}

Appends the \prm{code} to an internal hook executed at the beginning of every item in the list of shorthands. The \prm{code} is executed immediately after the \prm{item code} of \cmd{defbibenvironment}. The bibliographic data of the respective entry is available at this point. This command may only be used in the preamble.

This is just an alias for:

\begin{ltxexample}
\AtEveryBiblistitem{shorthand}{code}
\end{ltxexample}

\cmditem{AtEveryBiblistitem}{biblistname}{code}

Appends the \prm{code} to an internal hook executed at the beginning of every item in the bibliography list named \prm{biblistname}. The \prm{code} is executed immediately after the \prm{item code} of \cmd{defbibenvironment}. The bibliographic data of the respective entry is available at this point. This command may only be used in the preamble.

\cmditem{AtNextBibliography}{code}

Similar to \cmd{AtBeginBibliography} but only affecting the next \cmd{printbibliography}. The internal hook is cleared after being executed once. This command may be used in the document body.

\cmditem{AtEveryCite}{code}

Appends the \prm{code} to an internal hook executed at the beginning of every citation command. The \prm{code} is executed immediately before the \prm{precode} of the command (see \secref{aut:cbx:cbx}). No bibliographic data is available at this point. This command may only be used in the preamble.

\cmditem{AtEveryCitekey}{code}

Appends the \prm{code} to an internal hook executed once for every entry key passed to a citation command. The \prm{code} is executed immediately before the \prm{loopcode} of the command (see \secref{aut:cbx:cbx}). The bibliographic data of the respective entry is available at this point. This command may only be used in the preamble.

\cmditem{AtEveryMultiCite}{code}

Appends the \prm{code} to an internal hook executed at the beginning of every multicite command. The \prm{code} is executed immediately before the \bibfield{multiprenote} field (\secref{aut:cbx:fld}) is printed. No bibliographic data is available at this point. This command may only be used in the preamble.

\cmditem{AtNextCite}{code}

Similar to \cmd{AtEveryCite} but only affecting the next citation command. The internal hook is cleared after being executed once. This command may be used in the document body.

\cmditem{AtEachCitekey}{code}

Similar to \cmd{AtEveryCitekey} but only affecting the current citation command. This command may be used in the document body. The \prm{code} is appended to the internal hook locally when located in a citation, as determined by \cmd{ifcitation}.

\cmditem{AtNextCitekey}{code}

Similar to \cmd{AtEveryCitekey} but only affecting the next entry key. The internal hook is cleared after being executed once. This command may be used in the document body.

\cmditem{AtNextMultiCite}{code}

Similar to \cmd{AtEveryMultiCite} but only affecting the next multicite command. The internal hook is cleared after being executed once. This command may be used in the document body.

\cmditem{AtDataInput}[entrytype]{code}

Appends the \prm{code} to an internal hook executed once for every entry as the bibliographic data is imported from the \file{bbl} file. The \prm{entrytype} is the entry type the \prm{code} applies to. If it applies to all entry types, omit the optional argument. The \prm{code} is executed immediately after the entry has been imported. This command may only be used in the preamble. Note that \prm{code} may be executed multiple times for an entry. This occurs when the same entry is cited in different \env{refsection} environments or the \opt{sorting} option settings incorporate more than one sorting scheme. The \cnt{refsection} counter holds the number of the respective reference section while the data is imported.

\cmditem{UseBibitemHook}

Executes the internal hook corresponding to \cmd{AtEveryBibitem}.

\cmditem{UseEveryCiteHook}

Executes the internal hook corresponding to \cmd{AtEveryCite}.

\cmditem{UseEveryCitekeyHook}

Executes the internal hook corresponding to \cmd{AtEveryCitekey}.

\cmditem{UseEveryMultiCiteHook}

Executes the internal hook corresponding to \cmd{AtMultiEveryCite}.

\cmditem{UseNextCiteHook}

Executes and clears the internal hook corresponding to \cmd{AtNextCite}.

\cmditem{UseNextCitekeyHook}

Executes and clears the internal hook corresponding to \cmd{AtNextCitekey}.

\cmditem{UseNextMultiCiteHook}

Executes and clears the internal hook corresponding to \cmd{AtNextMultiCite}.

\cmditem{DeferNextCitekeyHook}

Locally un-defines the internal hook specified by \cmd{AtNextCitekey}. This essentially defers the hook to the next entry key in the citation list, when executed in the \prm{precode} argument of \cmd{DeclareCiteCommand} (\secref{aut:cbx:cbx}).

\end{ltxsyntax}

\subsection{提示与警告 Hints and Caveats}
\label{aut:cav}

This section provides some additional hints concerning the author interface of this package. It also addresses common problems and potential misconceptions.

\subsubsection{条目集 Entry Sets}
\label{aut:cav:set}

Entry sets have already been introduced in \secref{use:use:set}. This section discusses how to process entry sets in a bibliography style. From the perspective of the driver, there is no difference between static and dynamic entry sets. Both types are handled in the same way. You will normally use the \cmd{entryset} command from \secref{aut:bib:dat} to loop over all set members (in the order in which they are listed in the \bibfield{entryset} field of the \bibtype{set} entry, or in the order in which they were passed to \cmd{defbibentryset}, respectively) and append \cmd{finentry} at the end. That's it. The formatting is handled by the drivers for the entry types of the individual set members:

\begin{ltxexample}
\DeclareBibliographyDriver{set}{%
  <<\entryset>>{}{}%
  \finentry}
\end{ltxexample}
%
You may have noticed that the \texttt{numeric} styles which ship with this package support subdivided entry sets, \ie the members of the set are marked with a letter or some other marker such that citations may either refer to the entire set or to a specific set member. The markers are generated as follows by the bibliography style:

\begin{ltxexample}
\DeclareBibliographyDriver{set}{%
  \entryset
    {<<\printfield{entrysetcount}>>%
     <<\setunit*{\addnbspace}>>}
    {}%
  \finentry}
\end{ltxexample}
%
The \bibfield{entrysetcount} field holds an integer indicating the position of a set member in the entry set. The conversion of this number to a letter or some other marker is handled by the formatting directive of the \bibfield{entrysetcount} field. All the driver needs to do is print the field and add some white space (or start a new line). Printing the markers in citations works in a similar way. Where a numeric style normally says |\printfield{labelnumber}|, you simply append the \bibfield{entrysetcount} field:

\begin{ltxexample}
\printfield{labelnumber}<<\printfield{entrysetcount}>>
\end{ltxexample}
%
Since this field is only defined when processing citations referring to a set member, there is no need to add any additional tests.

\subsubsection{电子出版信息 Electronic Publishing Information}
\label{aut:cav:epr}

The standard styles feature dedicated support for arXiv references. Support for other resources is easily added. The standard styles handle the \bibfield{eprint} field as follows:

\begin{ltxexample}
\iffieldundef{eprinttype}
  {\printfield{eprint}}
  {\printfield[<<eprint:\strfield{eprinttype}>>]{eprint}}
\end{ltxexample}
%
If an \bibfield{eprinttype} field is available, the above code tries to use the field format \texttt{eprint:\prm{eprinttype}}. If this format is undefined, \cmd{printfield} automatically falls back to the field format \texttt{eprint}. There are two predefined field formats, the type"=specific format \texttt{eprint:arxiv} and the fallback format \texttt{eprint}:

\begin{ltxexample}
\DeclareFieldFormat{<<eprint>>}{...}
\DeclareFieldFormat{<<eprint:arxiv>>}{...}
\end{ltxexample}
%
In other words, adding support for additional resources is as easy as defining a field format named \texttt{eprint:\prm{resource}} where \prm{resource} is an identifier to be used in the \bibfield{eprinttype} field.

\subsubsection{额外的摘要和注释 External Abstracts and Annotations}
\label{aut:cav:prf}

External abstracts and annotations have been discussed in \secref{use:use:prf}. This section provides some more background for style authors. The standard styles use the following macros (from \path{biblatex.def}) to handle abstracts and annotations:

\begin{ltxexample}
\newbibmacro*{annotation}{%
  \iffieldundef{annotation}
    {\printfile[annotation]{<<\bibannotationprefix\thefield{entrykey}.tex>>}}%
    {\printfield{annotation}}}
\newcommand*{<<\bibannotationprefix>>}{bibannotation-}

\newbibmacro*{abstract}{%
  \iffieldundef{abstract}
    {\printfile[abstract]{<<\bibabstractprefix\thefield{entrykey}.tex>>}}%
    {\printfield{abstract}}}
\newcommand*{<<\bibabstractprefix>>}{bibabstract-}
\end{ltxexample}
%
If the \bibfield{abstract}\slash \bibfield{annotation} field is undefined, the above code tries to load the abstracts\slash annotations from an external file. The \cmd{printfile} commands also incorporate file name prefixes which may be redefined by users. Note that you must enable \cmd{printfile} explicitly by setting the \opt{loadfiles} package option from \secref{use:opt:pre:gen}. This feature is disabled by default for performance reasons.

\subsubsection[消除姓名歧义 Name Disambiguation]{消除姓名歧义 Name Disambiguation\BiberOnlyMark}
\label{aut:cav:amb}
在\secref{use:opt:pre:int}节引入的\opt{uniquename}和\opt{uniquelist}选项支持多种操作模式。本节用举例方式介绍不同模式的差别。\opt{uniquename}选项消除\bibfield{labelname}列表中各姓名间的歧义,\opt{uniquelist}消除因\opt{maxnames}\slash \opt{minnames}截短导致的\bibfield{labelname}列表歧义。两个选项可以单独使用也可以联合使用:
%The \opt{uniquename} and \opt{uniquelist} options introduced in \secref{use:opt:pre:int} support various modes of operation. This section explains the differences between these modes by way of example. The \opt{uniquename} option disambiguates individual names in the \bibfield{labelname} list. The \opt{uniquelist} option disambiguates the \bibfield{labelname} list if it has become ambiguous after \opt{maxnames}\slash \opt{minnames} truncation. You can use either option stand-alone or combine both.



Name disambiguation works by taking a <base> which is composed of one or more nameparts and then determining what needs to be added, if anything, to this <base> to make the name unique in the current refsection. Name disambiguation is controlled by the uniquename template declared with the following command:

\begin{ltxsyntax}

\cmditem{DeclareUniquenameTemplate}{specification}

The \prm{specification} is a list of \cmd{namepart} commands which define the nameparts to use in determining the uniquename information

\cmditem{namepart}[options]{namepart}

\prm{namepart} is one of the datamodel nameparts defined with the \cmd{DeclareDatamodelConstant} command (see \secref{aut:bbx:drv}). The \opt{options} are:

\begin{optionlist*}

\boolitem[false]{use}

Only use the \prm{namepart} in constructing the uniquename information if there is a corresponding option \opt{use<namepart>} and that option is true.

\boolitem[false]{base}

The \prm{namepart} is part of the <base> which is the main piece of namepart(s) information which is being disambiguated by uniqueness information.

\end{optionlist*}

\end{ltxsyntax}
%
The default uniquename template is:

\begin{ltxexample}
\DeclareUniquenameTemplate{
  \namepart[use=true, base=true]{prefix}
  \namepart[base=true]{family}
  \namepart{given}
}
\end{ltxexample}
%
This means that the <base> to be disambiguated consists of the <family> namepart, along with any prefix, if the \opt{useprefix} option is true. The disambiguation is performed by adding aspects of any non <base> nameparts in the specification, here just the <given> namepart.

\paragraph{Individual Names (\opt{uniquename})}

Let's start off with some \opt{uniquename} examples. Consider the following data:
下面从一些\opt{uniquename}例子开始,考虑如下数据:

\begin{lstlisting}{}
John Doe   2008
Edward Doe 2008
John Smith 2008
Jane Smith 2008
\end{lstlisting}
%
Let's assume we're using an author-year style and set \kvopt{uniquename}{false}. In this case, we would get the following citations:

假设我们使用作者年制且设置\kvopt{uniquename}{false},这种情况下,我们得到如下引用标注:
\begin{lstlisting}{}
Doe 2008a
Doe 2008b
Smith 2008a
Smith 2008b
\end{lstlisting}
%
Since the family names are ambiguous and all works have been published in the same year, an extra letter is appended to the year to disambiguate the citations. Many style guides, however, mandate that the extra letter be used to disambiguate works by the same authors only, not works by different authors with the same family name. In order to disambiguate the author's family name, you are expected to add additional parts of the name, either as initials or in full. This requirement is addressed by the \opt{uniquename} option. Here are the same citations with \kvopt{uniquename}{init}:

因为姓有歧义,且所有的年都相同,所以年后附加的字符用来区分并消除歧义。然而,很多样式指南强制要求附加字符只能用于相同作者的区分,而不能用于作者相同的姓的区分。为了消除作者姓的歧义,需要增加姓名的其它完整部分或者缩写来区分。这一需要由\opt{uniquename}选项处理,下面是使用了\kvopt{uniquename}{init}的引用标注:
\begin{lstlisting}{}
J. Doe 2008
E. Doe 2008
Smith 2008a
Smith 2008b
\end{lstlisting}
%
\kvopt{uniquename}{init} restricts name disambiguation to initials. Since <J. Smith> would still be ambiguous, no additional name parts are added for the <Smiths>. With \kvopt{uniquename}{full}, names are printed in full where required:

\kvopt{uniquename}{init}限制了用缩写来区分姓名。但因为<J. Smith>仍然有歧义,所以没有增加。而使用\kvopt{uniquename}{full},标注如下:
\begin{lstlisting}{}
J. Doe 2008
E. Doe 2008
John Smith 2008
Jane Smith 2008
\end{lstlisting}
%
In order to illustrate the difference between \kvopt{uniquename}{init\slash full} and \texttt{allinit\slash allfull}, we need to introduce the notion of a <visible> name. In the following, <visible> names are all names at a position before the \opt{maxnames}\slash \opt{minnames}\slash \opt{uniquelist} truncation point. For example, given this data:

为了说明\kvopt{uniquename}{init\slash full}和\texttt{allinit\slash allfull}的差别,我们下面介绍 <visible> 姓名的概念。<visible> 姓名时位于\opt{maxnames}\slash \opt{minnames}\slash \opt{uniquelist}截短点前的姓名,比如,给出数据:
\begin{lstlisting}{}
William Jones/Edward Doe/Jane Smith
John Doe
John Smith
\end{lstlisting}
%
and \kvopt{maxnames}{1}, \kvopt{minnames}{1}, \kvopt{uniquename}{init/full}, we would get the following names in citations:

\kvopt{maxnames}{1}, \kvopt{minnames}{1}, \kvopt{uniquename}{init/full},我们得到如下的引用标注:
\begin{lstlisting}{}
Jones et al.
Doe
Smith
\end{lstlisting}
%
When disambiguating names, \kvopt{uniquename}{init/full} only consider the visible names. Since all visible last names are distinct in this example, no further name parts are added. Let's compare that to the output of \kvopt{uniquename}{allinit}:

在消除歧义的时候,\kvopt{uniquename}{init/full}仅考虑可见的姓名。因为本例中所有的可见姓名的姓都是不同的,所有没有姓名的其他部分附加进来。比较一下使用\kvopt{uniquename}{allinit}的输出:
\begin{lstlisting}{}
Jones et al.
J. Doe
Smith
\end{lstlisting}
%
\texttt{allinit} considers all names in all \bibfield{labelname} lists, including those which are hidden and replaced by <et al.> as the list is truncated. In this example, <John Doe> is disambiguated from <Edward Doe>. Since the ambiguity of the two <Smiths> can't be resolved by adding initials, no initials are added in this case. Now let's compare that to the output of \kvopt{uniquename}{allfull} which also disambiguates <John Smith> from <Jane Smith>:

\texttt{allinit}认为所有在\bibfield{labelname}列表中的姓名,包括列表截短后已经隐藏并且由<et al.>代替的姓名。在本例中,<John Doe> 与 <Edward Doe>存在歧义。因为两个 <Smiths> 无法通过添加缩写的方式区分,所以没有添加。现在来比较一下\kvopt{uniquename}{allfull}的输出:
\begin{lstlisting}{}
Jones et al.
J. Doe
John Smith
\end{lstlisting}
%
The options \kvopt{uniquename}{mininit/minfull} are similar to \texttt{init\slash full} in that they only consider visible names, but they perform minimal disambiguation. That is, they will disambiguate individual names only if they occur in identical lists of last names. Consider the following data:

\kvopt{uniquename}{mininit/minfull}选项类似于\texttt{init\slash full}仅考虑可见姓名,但仅执行最小的歧义消除。即,仅对姓列表的歧义进行处理,考虑如下数据:
\begin{lstlisting}{}
John Doe/William Jones
Edward Doe/William Jones
John Smith/William Edwards
Edward Smith/Allan Johnson
\end{lstlisting}
%
With \kvopt{uniquename}{init/full}, we would get:

使用\kvopt{uniquename}{init/full},得到:
\begin{lstlisting}{}
J. Doe and Jones
E. Doe and Jones
J. Smith and Edwards
E. Smith and Johnson
\end{lstlisting}
%
With \kvopt{uniquename}{mininit/minfull}:
使用\kvopt{uniquename}{mininit/minfull},得到:

\begin{lstlisting}{}
J. Doe and Jones
E. Doe and Jones
Smith and Edwards
Smith and Johnson
\end{lstlisting}
%
The <Smiths> are not disambiguated because the visible name lists are not ambiguous and the \opt{mininit/minfull} options serve to disambiguate names occurring in identical last name lists only. Another way of looking at this is that they globally disambiguate family name lists. When it comes to ambiguous lists, note that a truncated list is considered to be distinct from an untruncated one even if the visible names are identical. For example, consider the following data:

<Smiths> 并无歧义,因为姓名列表时没有歧义。\opt{mininit/minfull}选项仅对姓的列表相同情况进行处理。全局的看姓的列表,注意当未截短的列表的可见名相同的时候,截短的列表时也可能是不同的,比如下面的数据:
\begin{lstlisting}{}
John Doe/William Jones
Edward Doe
\end{lstlisting}
%
With \kvopt{maxnames}{1}, \kvopt{uniquename}{init/full}, we would get:
使用\kvopt{maxnames}{1}, \kvopt{uniquename}{init/full}:

\begin{lstlisting}{}
J. Doe et al.
E. Doe
\end{lstlisting}
%
With \kvopt{uniquename}{mininit/minfull}:
使用\kvopt{uniquename}{mininit/minfull}:

\begin{lstlisting}{}
Doe et al.
Doe
\end{lstlisting}
%
Because the lists differ in the <et al.>, the names are not disambiguated.
因为列表有 <et al.> 的不同,姓名列表就不歧义。

\paragraph{Lists of Names (\opt{uniquelist})}
\paragraph{姓名列表(独立姓名外的处理) (\opt{uniquelist})}

Ambiguity is also an issue with name lists. If the \bibfield{labelname} list is truncated by the \opt{maxnames}\slash \opt{minnames} options, it may become ambiguous. This type of ambiguity is addressed by the \opt{uniquelist} option. Consider the following data:

姓名列表也可能存在歧义问题。如果\bibfield{labelname}列表由\opt{maxnames}\slash \opt{minnames}选项截短就可能产生歧义。这类问题由\opt{uniquelist}选处理,考虑如下数据:
\begin{lstlisting}{}
Doe/Jones/Smith   2005
Smith/Johnson/Doe 2005
Smith/Doe/Edwards 2005
Smith/Doe/Jones   2005
\end{lstlisting}
%
Many author-year styles truncate long author/editor lists in citations. For example, with \kvopt{maxnames}{1} we would get:

很多作者年制样式需要在标注中截短,比如使用\kvopt{maxnames}{1}选项,得到:
\begin{lstlisting}{}
Doe et al. 2005
Smith et al. 2005a
Smith et al. 2005b
Smith et al. 2005c
\end{lstlisting}
%
Since the authors are ambiguous after truncation, the extra letter is added to the year to ensure unique citations. Here again, many style guides mandate that the extra letter be used to disambiguate works by the same authors only. In order to disambiguate author lists, you are usually required to add more names, exceeding the \opt{maxnames}\slash \opt{minnames} truncation point. The \opt{uniquelist} feature addresses this requirement. With \kvopt{uniquelist}{true}, we would get:
因为截短后作者存在歧义,所以添加额外字符确保引用标注的唯一性。同样的,一些样式强制要求额外字符只能用于所有作者都相同的情况。为了区分作者列表,必须增加更多的姓名,这样就会超出\opt{maxnames}\slash \opt{minnames}选项设定的截短点。\opt{uniquelist}选项即描述这一需求,当\kvopt{uniquelist}{true},有:

\begin{lstlisting}{}
Doe et al. 2005
Smith, Johnson et al. 2005
Smith, Doe and Edwards 2005
Smith, Doe and Jones 2005
\end{lstlisting}
%
The \opt{uniquelist} option overrides \opt{maxnames}\slash \opt{minnames} on a per-entry basis. Essentially, what happens is that the <et al.> part of the citation is expanded to the point of no ambiguity~-- but no further than that. \opt{uniquelist} may also be combined with \opt{uniquename}. Consider the following data:
\opt{uniquelist}选项以条目为限重设\opt{maxnames}\slash \opt{minnames}。大体上,标注的 <et al.> 部分扩展到无歧义的点--而且也基本到此为止。\opt{uniquelist}也可以与\opt{uniquename}联合使用,考虑如下数据:

\begin{lstlisting}{}
John Doe/Allan Johnson/William Jones  2009
John Doe/Edward Johnson/William Jones 2009
John Doe/Jane Smith/William Jones     2009
John Doe/John Smith/William Jones     2009
John Doe/John Edwards/William Jones   2009
John Doe/John Edwards/Jack Johnson    2009
\end{lstlisting}
%
With \kvopt{maxnames}{1}:
使用\kvopt{maxnames}{1},得到:

\begin{lstlisting}{}
Doe et al. 2009a
Doe et al. 2009b
Doe et al. 2009c
Doe et al. 2009d
Doe et al. 2009e
Doe et al. 2009f
\end{lstlisting}
%
With \kvopt{maxnames}{1}, \kvopt{uniquename}{full}, \kvopt{uniquelist}{true}:
使用\kvopt{maxnames}{1}, \kvopt{uniquename}{full}, \kvopt{uniquelist}{true}则有:

\begin{lstlisting}{}
Doe, A. Johnson et al. 2009
Doe, E. Johnson et al. 2009
Doe, Jane Smith et al. 2009
Doe, John Smith et al. 2009
Doe, Edwards and Jones 2009
Doe, Edwards and Johnson 2009
\end{lstlisting}
%
With \kvopt{uniquelist}{minyear}, list disambiguation only happens if the visible list is identical to another visible list with the same \bibfield{labelyear}. This is useful for author-year styles which only require that the citation as a whole be unique, but do not guarantee unambiguous authorship information in citations. This mode is conceptually related to \kvopt{uniquename}{mininit/minfull}. Consider this example:

使用\kvopt{uniquelist}{minyear},消除列表歧义仅在可见列表和\bibfield{labelyear}相同的时候。这对于仅仅需要整个标注整体具有唯一性的作者年制样式是很有用的,但是不保证作者姓名的非歧义性。这一模式概念上域\kvopt{uniquename}{mininit/minfull}选项相关。考虑如下例子:
\begin{lstlisting}{}
Smith/Jones   2000
Smith/Johnson 2001
\end{lstlisting}
%
With \kvopt{maxnames}{1} and \kvopt{uniquelist}{true}, we would get:
使用\kvopt{maxnames}{1}和\kvopt{uniquelist}{true},得到:

\begin{lstlisting}{}
Smith and Jones 2000
Smith and Johnson 2001
\end{lstlisting}
%
With \kvopt{uniquelist}{minyear}:
使用\kvopt{uniquelist}{minyear},则得到:

\begin{lstlisting}{}
Smith et al. 2000
Smith et al. 2001
\end{lstlisting}
%
With \kvopt{uniquelist}{minyear}, it is not clear that the authors are different for the two works but the citations as a whole are still unambiguous since the year is different. In contrast to that, \kvopt{uniquelist}{true} disambiguates the authorship even if this information is not required to uniquely locate the works in the bibliography. Let's consider another example:

使用\kvopt{uniquelist}{minyear},两个文献的作者是否相同并不清楚,但标注的整体是非歧义的,因为年份的不同。与此相反,\kvopt{uniquelist}{true}需要消除作者列表的歧义即便这一信息对于参考文献表的唯一引用是不必要的,看看如下例子:
\begin{lstlisting}{}
Vogel/Beast/Garble/Rook  2000
Vogel/Beast/Tremble/Bite 2000
Vogel/Beast/Acid/Squeeze 2001
\end{lstlisting}
%
With \kvopt{maxnames}{3}, \kvopt{minnames}{1}, \kvopt{uniquelist}{true}, we would get:
使用\kvopt{maxnames}{3}, \kvopt{minnames}{1}, \kvopt{uniquelist}{true},得到

\begin{lstlisting}{}
Vogel, Beast, Garble et al. 2000
Vogel, Beast, Tremble et al. 2000
Vogel, Beast, Acid et al. 2001
\end{lstlisting}
%
With \kvopt{uniquelist}{minyear}:
使用\kvopt{uniquelist}{minyear}选项,则有:

\begin{lstlisting}{}
Vogel, Beast, Garble et al. 2000
Vogel, Beast, Tremble et al. 2000
Vogel et al. 2001
\end{lstlisting}
%
In the last citation, \kvopt{uniquelist}{minyear} does not override \opt{maxnames}\slash \opt{minnames} as the citation does not need disambiguating from the other two because the year is different.

在最后一个引用中,\kvopt{uniquelist}{minyear}不重写\opt{maxnames}\slash \opt{minnames},因为年份的不同,所以不需要消除与其它两个间的歧义。
\subsubsection{浮动体和\acr{TOC}/\acr{LOT}/\acr{LOF}中的追踪器 Trackers in Floats and \acr{TOC}/\acr{LOT}/\acr{LOF}}
\label{aut:cav:flt}

If a citation is given in a float (typically in the caption of a figure or table), scholarly back references like <ibidem> or back references based on the page tracker get ambiguous because floats are objects which are (physically and logically) placed outside the flow of text, hence the logic of such references applies poorly to them. To avoid any such ambiguities, the citation and page trackers are temporarily disabled in all floats. In addition to that, these trackers plus the back reference tracker (\opt{backref}) are temporarily disabled in the table of contents, the list of figures, and the list of tables.

\subsubsection{混合编程接口 Mixing Programming Interfaces}
\label{aut:cav:mif}

The \biblatex package provides two main programming interfaces for style authors. The \cmd{DeclareBibliographyDriver} command, which defines a handler for an entry type, is typically used in \file{bbx} files. \cmd{DeclareCiteCommand}, which defines a new citation command, is typically used in \file{cbx} files. However, in some cases it is convenient to mix these two interfaces. For example, the \cmd{fullcite} command prints a verbose citation similar to the full bibliography entry. It is essentially defined as follows:

\biblatex 宏包给样式作者提供了2个主要的编程接口即: \file{bbx}文件中使用的\cmd{DeclareBibliographyDriver} 命令用来定义各类参考文献条目的驱动(格式处理器,handler),\file{cbx}文件中使用的\cmd{DeclareCiteCommand}命令用来定义新的引用命令。然而有时候,混合使用这两个接口会很方便。比如\cmd{fullcite}命令就可以打印类似于完整参考文献条目的长串引用,该命令定义大体如下:
\begin{ltxexample}
\DeclareCiteCommand{\fullcite}
  {...}
  {<<\usedriver>>{...}{<<\thefield{entrytype}>>}}
  {...}
  {...}
\end{ltxexample}
%
As you can see, the core code which prints the citations simply executes the bibliography driver defined with \cmd{DeclareBibliographyDriver} for the type of the current entry. When writing a citation style for a verbose citation scheme, it is often convenient to use the following structure:
如上所见,打印引用的核心代码简单地为当前的条目类型执行了\cmd{DeclareBibliographyDriver}定义的参考文献驱动命令。当为长引用样式(a verbose citation scheme)编写引用样式文件的时候,使用下面的结构是非常方便的:

\begin{ltxexample}
\ProvidesFile{example.cbx}[2007/06/09 v1.0 biblatex citation style]

\DeclareCiteCommand{\cite}
  {...}
  {<<\usedriver>>{...}{<<cite:\thefield{entrytype}>>}}
  {...}
  {...}

\DeclareBibliographyDriver{<<cite:article>>}{...}
\DeclareBibliographyDriver{<<cite:book>>}{...}
\DeclareBibliographyDriver{<<cite:inbook>>}{...}
...
\end{ltxexample}
%
Another case in which mixing interfaces is helpful are styles using cross"=references within the bibliography. For example, when printing an \bibtype{incollection} entry, the data inherited from the \bibtype{collection} parent entry would be replaced by a short pointer to the respective parent entry:

混合接口的另一个有用情况是在参考文献表中使用交叉引用(cross"=references)时。比如当打印\bibtype{incollection}
类型的条目,数据继承自\bibtype{collection}父条目,将会由一个简短的指向对应父条目的指针来代替。
\begin{enumerate}
\renewcommand*\labelenumi{[\theenumi]}
\setlength{\leftskip}{0.5em}
\item Audrey Author: \emph{Title of article}. In: [\textln{2}], pp.~134--165.
\item Edward Editor, ed.: \emph{Title of collection}. Publisher: Location, 1995.
\end{enumerate}

One way to implement such cross"=references within the bibliography is to think of them as citations which use the value of the \bibfield{xref} or \bibfield{crossref} field as the entry key. Here is an example:
实现这种参考文献的交叉引用的一种方法是认为他们是引用关系,使用\bibfield{xref} 或 \bibfield{crossref}域的值作为条目关键字(bibtex条目键,entry key),例子如下:

\begin{ltxexample}
\ProvidesFile{example.bbx}[2007/06/09 v1.0 biblatex bibliography style]

\DeclareCiteCommand{<<\bbx@xref>>}
  {}
  {...}% code for cross-references
  {}
  {}

\DeclareBibliographyDriver{incollection}{%
  ...
  \iffieldundef{xref}
    {...}% code if no cross-reference
    {<<\bbx@xref>>{<<\thefield{xref}>>}}%
  ...
}
\end{ltxexample}
%
When defining \cmd{bbx@xref}, the \prm{precode}, \prm{postcode}, and \prm{sepcode} arguments of \cmd{DeclareCiteCommand} are left empty in the above example because they will not be used anyway. The cross"=reference is printed by the \prm{loopcode} of \cmd{bbx@xref}. For further details on the \bibfield{xref} field, refer to \secref{bib:fld:spc} and to the hints in \secref{bib:cav:ref}. Also see the \cmd{iffieldxref}, \cmd{iflistxref}, and \cmd{ifnamexref} tests in \secref{aut:aux:tst}. The above could also be implemented using the \cmd{entrydata} command from \secref{aut:bib:dat}:
当定义\cmd{bbx@xref}命令时,\cmd{DeclareCiteCommand}命令的\prm{precode}, \prm{postcode}, 和 \prm{sepcode}参数留空,是因为上面例子中没有用到。交叉引用由\cmd{bbx@xref}命令的\prm{loopcode}参数打印。更多的关于\bibfield{xref}域的细节见\secref{bib:fld:spc}节以及\secref{bib:cav:ref}节中的注意事项。在\secref{aut:aux:tst}节我们也看到了\cmd{iffieldxref}, \cmd{iflistxref}, 和\cmd{ifnamexref}测试命令。这些都也可以用\secref{aut:bib:dat}节的\cmd{entrydata}命令来实现。

\begin{ltxexample}
\ProvidesFile{example.bbx}[2007/06/09 v1.0 biblatex bibliography style]

\DeclareBibliographyDriver{incollection}{%
  ...
  \iffieldundef{xref}
    {...}% code if no cross-reference
    {<<\entrydata>>{<<\thefield{xref}>>}{%
      % code for cross-references
      ...
    }}%
  ...
}
\end{ltxexample}

\subsubsection{使用标点追踪 Using the Punctuation Tracker}
\label{aut:cav:pct}

\paragraph{标点基础 The Basics}

There is one fundamental principle style authors should keep in mind when designing a bibliography driver: block and unit punctuation is handled asynchronously. This is best explained by way of example. Consider the following code snippet:

样式作者设计参考文献驱动时需要记住一点:块和单元的标点是异步处理的。用例子最容易解释这一点,看下面一段代码:
\begin{ltxexample}
\printfield{title}%
\newunit
\printfield{edition}%
\newunit
\printfield{note}%
\end{ltxexample}
%
If there is no \bibfield{edition} field, this piece of code will not print:
如果没有\bibfield{edition}域,那么这段代码的打印结果不会是:

\begin{lstlisting}[style=highlight]{}
Title. . Note
\end{lstlisting}
%
but rather:
而会是

\begin{lstlisting}[style=highlight]{}
Title. Note
\end{lstlisting}
%
因为单元的标点追踪器是异步方式工作的。\cmd{newunit}命令将不会立即打印标点。它仅是记录一个单元的边界并且将\cmd{newunitpunct}命令放入标点缓存中。该缓存会有\emph{接下来的}\cmd{printfield}、\cmd{printlist}或类似命令进行处理,且仅当这些命令各自处理的域或列表已定义的时候才会处理。像\cmd{printfield}这样的命令在插入任何块和单元的标点之前将首先考虑3个因素:
because the unit punctuation tracker works asynchronously. \cmd{newunit} will not print the unit punctuation immediately. It merely records a unit boundary and puts \cmd{newunitpunct} on the punctuation buffer. This buffer will be handled by \emph{subsequent} \cmd{printfield}, \cmd{printlist}, or similar commands but only if the respective field or list is defined. Commands like \cmd{printfield} will consider three factors prior to inserting any block or unit punctuation:

\begin{itemize}
\item Has a new unit/block been requested at all?\par
= Is there any preceding \cmd{newunit} or \cmd{newblock} command?
是否有新的单元/块的输出请求?\par
=前面是否有\cmd{newunit}或者\cmd{newblock}命令?
\item Did the preceding commands print anything?\par
= Is there any preceding \cmd{printfield} or similar command?\par
= Did this command actually print anything?\par
前面的命令是否有打印输出?\par
=前面是否有\cmd{printfield}或者相似命令?\par
=该命令是否实际打印了任何东西?\par
\item Are we about to print anything now?\par
= Is the field/list to be processed now defined?
现在是否要打印一些东西?\par
要进行打印处理的域或列表是否已定义?
\end{itemize}
%
Block and unit punctuation will only be inserted if \emph{all} of these conditions apply. Let's reconsider the above example:
块和单元的标点只会在上述\emph{所有}条件满足的时候才会输出。让我们再次考虑上面的例子:
\begin{ltxexample}
\printfield{title}%
\newunit
\printfield{edition}%
\newunit
\printfield{note}%
\end{ltxexample}
%
如果\bibfield{edition}域没有定义会发生什么呢?第一个\cmd{printfield}命令打印了标题并设置一个内部的<new~text>标志。第一个\cmd{newunit}命令设置一个内部的<new~unit>标志。这使没有任何标点输出。第二个\cmd{printfield}命令不进行任何处理因为\bibfield{edition}域未定义。接下来的\cmd{newunit}命令再次设置<new unit>标志,仍然没有标点输出。第三个\cmd{printfield}命令检测\bibfield{note}域是否已定义,如果是,它会寻找<new~text>和<new~unit>标志。如果两个标志都存在,那么它会在打印note前插入标点缓存。然后它会清除<new~unit>标志然后再次设置<new~text>标志。
Here's what happens if the \bibfield{edition} field is undefined. The first \cmd{printfield} command prints the title and sets an internal <new~text> flag. The first \cmd{newunit} sets an internal <new~unit> flag. No punctuation has been printed at this point. The second \cmd{printfield} does nothing because the \bibfield{edition} field is undefined. The next \cmd{newunit} command sets the internal flag <new unit> again. Still no punctuation has been printed. The third \cmd{printfield} checks if the \bibfield{note} field is defined. If so, it looks at the <new~text> and <new~unit> flags. If both are set, it inserts the punctuation buffer before printing the note. It then clears the <new~unit> flag and sets the <new~text> flag again.

所有这些听起来似乎很复杂,但实际上,这意味着可以用顺序的方式写一个具有很多部件的参考文献驱动。这种方法的优势在不使用标点追踪而实现上述代码功能时会体现的很明显。如果不用标点追踪,那么会因为大量对所有可能存在域的判断产生一个复杂的\cmd{iffieldundef}判断命令集合。
This may all sound more complicated than it is. In practice, it means that it is possible to write large parts of a bibliography driver in a sequential way. The advantage of this approach becomes obvious when trying to write the above code without using the punctuation tracker. Such an attempt will lead to a rather convoluted set of \cmd{iffieldundef} tests required to check for all possible field combinations (note that the code below handles three fields; a typical driver may need to cater for some two dozen fields):

\begin{ltxexample}
\iffieldundef{title}%
  {\iffieldundef{edition}
     {\printfield{note}}
     {\printfield{edition}%
      \iffieldundef{note}%
	{}
	{. \printfield{note}}}}
  {\printfield{title}%
   \iffieldundef{edition}
     {}
     {. \printfield{edition}}%
   \iffieldundef{note}
     {}
     {. \printfield{note}}}%
\end{ltxexample}

\paragraph{常见错误 Common Mistakes}

把单元的标点处理认为是同步处理的是一个相当常见的误解。这会导致当驱动中包含原样文本\footnote{这里literal text 理解为原样文本,如实文本,逐字文本}时出现一些典型错误。考虑下面导致标点错位的错误代码段:
It is a fairly common misconception to think of the unit punctuation as something that is handled synchronously. This typically causes problems if the driver includes any literal text. Consider this erroneous code snippet which will generate misplaced unit punctuation:

\begin{ltxexample}
\printfield{title}%
\newunit
<<(>>\printfield{series} \printfield{number}<<)>>%
\end{ltxexample}
%
This code will yield the following result:
这段代码将产生下面的结果:

\begin{lstlisting}[style=highlight]{}
Title <<(.>> Series Number<<)>>
\end{lstlisting}
%
这里发生了什么呢?第一个\cmd{printfield}命令打印了标题,然后\cmd{newunit}命令标记了一个新的单元边界但不打印任何内容。单元的标点由\emph{下一个}\cmd{printfield}命令打印。这是前面提过的异步机制。然而因为左括号在下一个
\cmd{printfield}命令插入标点前立即打印,所以导致了错误的句点。当插入\emph{任何}原样文本比如括号(还包括由
\cmd{bibopenparen}和\cmd{mkbibparens}命令打印的括号)时,总需要将这些文本用\cmd{printtext}命令包起来。要让标点追踪正常运转,需要让驱动知道所有插入的原样文本。这是\cmd{printtext}命令的作用所在。\cmd{printtext}命令联系标点追踪器确保标点缓存在原样文本打印前插入。它也设定内部 <new~text> 标志。注意本例中还有第三处原样文本即|\printfield{series}|后面的空格。在改正的例子中,我们将使用标点追踪器来处理该空格。
Here's what happens. The first \cmd{printfield} prints the title. Then \cmd{newunit} marks a unit boundary but does not print anything. The unit punctuation is printed by the \emph{next} \cmd{printfield} command. That's the asynchronous part mentioned before. However, the opening parenthesis is printed immediately before the next \cmd{printfield} inserts the unit punctuation, leading to a misplaced period. When inserting \emph{any} literal text such as parentheses (including those printed by commands such as \cmd{bibopenparen} and \cmd{mkbibparens}), always wrap the text in a \cmd{printtext} command. For the punctuation tracker to work as expected, it needs to know about all literal text inserted by a driver. This is what \cmd{printtext} is all about. \cmd{printtext} interfaces with the punctuation tracker and ensures that the punctuation buffer is inserted before the literal text gets printed. It also sets the internal <new~text> flag. Note there is in fact a third piece of literal text in this example: the space after |\printfield{series}|. In the corrected example, we will use the punctuation tracker to handle that space.

\begin{ltxexample}
\printfield{title}%
\newunit
<<\printtext{(}>>%
\printfield{series}%
<<\setunit*{\addspace}>>%
\printfield{number}%
<<\printtext{)}>>%
\end{ltxexample}
%
尽管上面的代码能够如常工作,但处理括号、引号和其它包围某个域的标点是,推荐的方式是定义一个域格式:
While the above code will work as expected, the recommended way to handle parentheses, quotes, and other things which enclose more than one field, is to define a field format:

\begin{ltxexample}
\DeclareFieldFormat{<<parens>>}{<<\mkbibparens{#1}>>}
\end{ltxexample}
%
域格式可以同时用于\cmd{printfield}和\cmd{printtext}命令,因此我们可以利用它对若干个域用一堆括号进行包裹。
Field formats may be used with both \cmd{printfield} and \cmd{printtext}, hence we can use them to enclose several fields in a single pair of parentheses:

\begin{ltxexample}
<<\printtext[parens]{>>%
  \printfield{series}%
  \setunit*{\addspace}%
  \printfield{number}%
<<}>>%
\end{ltxexample}
%
这里我们还需要处理没有series信息时的情况,因此进一步改进代码如下:
We still need to handle cases in which there is no series information at all, so let's improve the code some more:

\begin{ltxexample}
<<\iffieldundef{series}>>
  {}
  {\printtext[parens]{%
     \printfield{series}%
     \setunit*{\addspace}%
     \printfield{number}}}%
\end{ltxexample}
%
最后的一点提示: 本地化字符串对于标点追踪器来说不是原样文本。因为\cmd{bibstring}和相似命令能联系标点追踪器,因此就不需要用\cmd{printtext}包裹起来。
One final hint: localisation strings are not literal text as far as the punctuation tracker is concerned. Since \cmd{bibstring} and similar commands interface with the punctuation tracker, there is no need to wrap them in a \cmd{printtext} command.

\paragraph{高级应用 Advanced Usage}
标点追踪器也可用来处理更复杂的情况。比如,考虑需要对\bibfield{location}、 \bibfield{publisher}和 \bibfield{year}根据数据是否提供以如下的格式打印:
The punctuation tracker may also be used to handle more complex scenarios. For example, suppose that we want the fields \bibfield{location}, \bibfield{publisher}, and \bibfield{year} to be rendered in one of the following formats, depending on the available data:

\begin{ltxexample}
...text<<. Location: Publisher, Year.>> Text...
...text<<. Location: Publisher.>> Text...
...text<<. Location: Year.>> Text...
...text<<. Publisher, Year.>> Text...
...text<<. Location.>> Text...
...text<<. Publisher.>> Text...
...text<<. Year.>> Text...
\end{ltxexample}
%
这个问题可以用一个相当复杂的\cmd{iflistundef}和\cmd{iffieldundef}判断集进行处理,通过这些判断可以确定所有可能的域的组合:
This problem can be solved with a rather convoluted set of \cmd{iflistundef} and \cmd{iffieldundef} tests which check for all possible field combinations:

\begin{ltxexample}
\iflistundef{location}
  {\iflistundef{publisher}
     {\printfield{year}}
     {\printlist{publisher}%
      \iffieldundef{year}
        {}
        {, \printfield{year}}}}
  {\printlist{location}%
   \iflistundef{publisher}%
     {\iffieldundef{year}
        {}
        {: \printfield{year}}}
     {: \printlist{publisher}%
      \iffieldundef{year}
        {}
        {, \printfield{year}}}}%
\end{ltxexample}
%
可以应用\cmd{ifthenelse}命令和\secref{aut:aux:ife}讨论的布尔运算可使上面的代码更具可读性。但本质上是一样的。然而,也可以按顺序写成如下方式:
The above could be written in a somewhat more readable way by employing \cmd{ifthenelse} and the boolean operators discussed in \secref{aut:aux:ife}. The approach would still be essentially the same. However, it may also be written sequentially:

\begin{ltxexample}
\newunit
\printlist{location}%
\setunit*{\addcolon\space}%
\printlist{publisher}%
\setunit*{\addcomma\space}%
\printfield{year}%
\newunit
\end{ltxexample}
%
在实际使用中,你会经常使用标点追踪器执行一些显式或隐式的组合判断,比如,考虑如下格式(注意当没有publisher时location后面的标点)
In practice, you will often use a combination of explicit tests and the implicit tests performed by the punctuation tracker. For example, consider the following format (note the punctuation after the location if there is no publisher):

\begin{ltxexample}
...text. Location: Publisher, Year. Text...
...text. Location: Publisher. Text...
...text<<. Location, Year.>> Text...
...text. Publisher, Year. Text...
...text. Location. Text...
...text. Publisher. Text...
...text. Year. Text...
\end{ltxexample}
%
这可以用如下代码进行处理:
This can be handled by the following code:

\begin{ltxexample}
\newunit
\printlist{location}%
\iflistundef{publisher}
  {\setunit*{\addcomma\space}}
  {\setunit*{\addcolon\space}}%
\printlist{publisher}%
\setunit*{\addcomma\space}%
\printfield{year}%
\newunit
\end{ltxexample}
%
因为当没有publisher时location后面的标点的特殊性,我们需要用一个\cmd{iflistundef}判断来确保正确性。剩下其它的则有标点追踪器处理。
Since the punctuation after the location is special if there is no publisher, we need one \cmd{iflistundef} test to catch this case. Everything else is handled by the punctuation tracker.

\subsubsection{定制本地化模型 Custom Localization Modules}
\label{aut:cav:lng}

Style guides may include provisions as to how strings like <edition> should be abbreviated or they may mandate certain fixed expressions. For example, the \acr{mla} style guide requires authors to use the term <Works~Cited> rather than <Bibliography> or <References> in the heading of the bibliography. Localization commands such as \cmd{DefineBibliographyStrings} from \secref{use:lng} may indeed be used in \file{cbx} and \file{bbx} files to handle such cases. However, overloading style files with translations is rather inconvenient. This is where \cmd{DeclareLanguageMapping} from \secref{aut:lng:cmd} comes into play. This command maps an \file{lbx} file with alternative translations to a \sty{babel}/\sty{polyglossia} language. For example, you could create a file named \path{french-humanities.lbx} which provides French translations adapted for use in the humanities and map it to the \sty{babel}/\sty{polyglossia} language \texttt{french} in the preamble or in the configuration file:

\begin{ltxexample}
\DeclareLanguageMapping{french}{french-humanities}
\end{ltxexample}
%
If the document language is set to \texttt{french}, \path{french-humanities.lbx} will replace \path{french.lbx}. Coming back to the \acr{mla} example mentioned above, an \acr{mla} style may come with an \path{american-mla.lbx} file to provide strings which comply with the \acr{mla} style guide. It would declare the following mapping in the \file{cbx} and/or \file{bbx} file:

\begin{ltxexample}
\DeclareLanguageMapping{american}{american-mla}
\end{ltxexample}
%
Since the alternative \file{lbx} file can inherit strings from the standard \path{american.lbx} module, \path{american-mla.lbx} may be as short as this:

\begin{ltxexample}
\ProvidesFile{american-mla.lbx}[2008/10/01 v1.0 biblatex localization]
<<\InheritBibliographyExtras>>{<<american>>}
\DeclareBibliographyStrings{%
  <<inherit>>          = {<<american>>},
  bibliography     = {{Works Cited}{Works Cited}},
  references       = {{Works Cited}{Works Cited}},
}
\endinput
\end{ltxexample}
%
Alternative \file{lbx} files must ensure that the localisation module is complete. They should do so by inheriting data from the corresponding standard module. If the language \texttt{american} is mapped to \path{american-mla.lbx}, \biblatex will not load \path{american.lbx} unless this module is requested explicitly. In the above example, inheriting <strings> and <extras> will cause \biblatex to load \path{american.lbx} before applying the modifications in \path{american-mla.lbx}.

Note that \cmd{DeclareLanguageMapping} is not intended to handle language variants (\eg American English vs. British English) or \sty{babel}/\sty{polyglossia} language aliases (\eg \texttt{USenglish} vs. \texttt{american}). For example, \sty{babel}/\sty{polyglossia} offers the \texttt{USenglish} option which is similar to \texttt{american}. Therefore, \biblatex ships with an \path{USenglish.lbx} file which simply inherits all data from \path{american.lbx} (which in turn gets the <strings> from \path{english.lbx}). In other words, the mapping of language variants and \sty{babel}/\sty{polyglossia} language aliases happens on the file level, the point being that \biblatex's language support can be extended simply by adding additional \file{lbx} files. There is no need for centralized mapping. If you need support for, say, Portuguese (babel/polyglossia: \file{portuges}), you create a file named \path{portuges.lbx}. If \sty{babel}/\sty{polyglossia} offered an alias named \texttt{brasil}, you would create \path{brasil.lbx} and inherit the data from \path{portuges.lbx}. In contrast to that, the point of \cmd{DeclareLanguageMapping} is handling \emph{stylistic} variants like <humanities vs. natural sciences> or <\acr{mla} vs. \acr{apa}> etc. which will typically be built on top of existing \file{lbx} files.

\subsubsection{编组 Grouping}
\label{aut:cav:grp}

In a citation or bibliography style, you may need to set flags or store certain values for later use. In this case, it is crucial to understand the basic grouping structure imposed by this package. As a rule of thumb, you are working in a large group whenever author commands such as those discussed in \secref{aut:aux} are available because the author interface of this package is only enabled locally. If any bibliographic data is available, there is at least one additional group. Here are some general rules:

\begin{itemize}

\item The entire list of references printed by \cmd{printbibliography} and similar commands is processed in a group. Each entry in the list is processed in an additional group which encloses the \prm{item code} of \cmd{defbibenvironment} as well as all driver code.

\item The entire bibliography list printed by \cmd{printbiblist} is processed in a group. Each entry in the list is processed in an additional group which encloses the \prm{item code} of \cmd{defbibenvironment} as well as all driver code.

\item All citation commands defined with \cmd{DeclareCiteCommand} are processed in a group holding the complete citation code consisting of the \prm{precode}, \prm{sepcode}, \prm{loopcode}, and \prm{postcode} arguments. The \prm{loopcode} is enclosed in an additional group every time it is executed. If any \prm{wrapper} code has been specified, the entire unit consisting of the wrapper code and the citation code is wrapped in an additional group.

\item In addition to the grouping imposed by all backend commands defined with \cmd{DeclareCiteCommand}, all <autocite> and <multicite> definitions imply an additional group.

\item \cmd{printfile}, \cmd{printtext}, \cmd{printfield}, \cmd{printlist}, and \cmd{printnames} form groups. This implies that all formatting directives will be processed within a group of their own.

\item All \file{lbx} files are loaded and processed in a group. If an \file{lbx} file contains any code which is not part of \cmd{DeclareBibliographyExtras}, the definitions must be global.

\end{itemize}

Note that using \cmd{aftergroup} in citation and bibliography styles is unreliable because the precise number of groups employed in a certain context may change in future versions of this package. If the above list states that something is processed in a group, this means that there is \emph{at least one} group. There may also be several nested ones.

\subsubsection{命名空间 Namespaces}
\label{aut:cav:nam}

In order to minimize the risk of name clashes, \latex packages typically prefix the names of internal macros with a short string specific to the package. For example, if the \sty{foobar} package requires a macro for internal use, it would typically be called \cmd{FB@macro} or \cmd{foo@macro} rather than \cmd{macro} or \cmd{@macro}. Here is a list of the prefixes used or recommended by \biblatex:

\begin{marglist}

\item[\texttt{blx}] All macros with names like \cmd{blx@name} are strictly reserved for internal use. This also applies to counter names, length registers, boolean switches, and so on. These macros may be altered in backwards"=incompatible ways, they may be renamed or even removed at any time without further notice. Such changes will not even be mentioned in the revision history or the release notes. In short: never use any macros with the string \texttt{blx} in their name in any styles.

\item[\texttt{abx}] Macros prefixed with \texttt{abx} are also internal macros but they are fairly stable. It is always preferable to use the facilities provided by the official author interface, but there may be cases in which using an \texttt{abx} macro is convenient.

\item[\texttt{bbx}] This is the recommended prefix for internal macros defined in bibliography styles.

\item[\texttt{cbx}] This is the recommended prefix for internal macros defined in citation styles.

\item[\texttt{lbx}] This is the recommended base prefix for internal macros defined in localisation modules. The localisation module should add a second prefix to specify the language. For example, an internal macro defined by the Spanish localisation module would be named \cmd{lbx@es@macro}.

\end{marglist}




